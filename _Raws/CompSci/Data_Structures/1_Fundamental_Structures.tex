\documentclass[12pt letter]{report}
\input{./template/preamble}
\input{./template/macros}
\input{./template/letterfonts}

\title{\Huge{Fundamental Data Structures}}
\author{\huge{Madiba Hudson-Quansah}}
\date{}
\usepackage{parskip}

\setcounter{tocdepth}{4}
\setcounter{secnumdepth}{4}

\begin{document}
\maketitle
\newpage
\pdfbookmark[section]{\contentsname}{too}
\tableofcontents
\pagebreak

\chapter{Introduction to Data Structures}

\section{Data Structures}


\dfn{Data Type}{
  Describes the set of values that a variable can hold and the types of operations that can be performed on it.
}

\dfn{Data Structure}{
  A way of storing and organizing data in computer memory / The logical mathematical model of a particular organization of
  data.
}

In choosing a data structure, we consider the following in the following order:
\begin{enumerate}
  \item The relationships of the data in the real world
  \item The type and amount of data to be stored
  \item Cost (time complexity) of operations
  \item Memory occupations
  \item Ease of implementation
\end{enumerate}

\subsection{Types of Data Structures}

Data structures can be broadly classified into two types:
\begin{itemize}
  \item Primitive Data Structures
  \item Non-Primitive / Composite Data Structures
\end{itemize}

With data structures under the primitive category being:
\begin{itemize}
  \item Variables
        \begin{itemize}
          \item Integer
          \item Float
          \item Character
          \item Boolean
          \item Enumerated
          \item Reference / Pointer
        \end{itemize}
\end{itemize}

And data structures under the non-primitive category being:
\begin{itemize}
  \item Arrays
  \item Structure / Record
  \item Union
  \item Class
  \item Abstract Data Type
        \begin{itemize}
          \item Linear
                \begin{itemize}
                  \item List
                  \item Stack
                  \item Queue
                \end{itemize}
          \item Non-Linear / Associative
                \begin{itemize}
                  \item Tree
                  \item Graph
                  \item Hash Table
                \end{itemize}
        \end{itemize}
\end{itemize}

Data structures can also more simply be classified in two ways:
\begin{itemize}
  \item Linear / Sequential Data Structures
        \begin{itemize}
          \item Linked List
          \item Array
          \item Stack
          \item Queue
          \item Set
        \end{itemize}
  \item Non-Linear / Associative  Data Structures
        \begin{itemize}
          \item Graph
          \item Tree
          \item Hash Table
        \end{itemize}
\end{itemize}

In this way data structures are defined by an implementation of a particular abstract data type.


\subsubsection{Primitive Data Structures}

These are the basic data structures that are directly operated on by machine / CPU instructions. They are the atomic
data type, i.e. they cannot be divided further

\subsubsection{Non-Primitive Data Structures}

These are data structures composed of or derived from primitive data structures.

\subsubsection{Linear Data Structures}

These data structures are characterized by the fact their elements can be accessed in a sequential / linear manner. This
means every element, excluding the first and last elements which only have a successor and predecessor respectively, has
both a predecessor and successor, allowing bi-directional traversal.

\subsubsection{Non-Linear Data Structures}

These data structures are characterized by the fact that their elements are accessed based on some relationship /
association between elements, therefore non-linearly.

\section{Non-Primitive Data Structures}

\subsection{Arrays}

\dfn{Array}{
  An indexed collection of a fixed number of homogeneous data elements. Elements are stored in contiguous memory, i.e.
  sequentially and can be accessed by their index.
}

\section{Generics}

\dfn{Type Parameter}{
  A placeholder for a type that is used in the definition of a generic type.
}

\dfn{Generic Type}{
  A type that is defined with one or more type parameters, i.e., \lstinline{<T>}. These type parameters are used to define the type of the data that the generic type can store.
}




\end{document}
