\documentclass[12pt letter]{report}
\input{./template/preamble}
\input{./template/macros}
\input{./template/letterfonts}

\title{\Huge{Introduction}}
\author{\huge{Madiba Hudson-Quansah}}
\date{}
\usepackage{parskip}

\setcounter{tocdepth}{4}
\setcounter{secnumdepth}{4}

\begin{document}
\maketitle
\newpage
\pdfbookmark[section]{\contentsname}{too}
\tableofcontents
\pagebreak

\chapter{Introduction}

\dfn{Security}{
  A state of being secure and free from danger or harm, i.e. the actions taken to make someone or something secure.
  Protection from adversaries (those who would do harm), intentionally or otherwise, is the goal of security.
}

\dfn{Asset}{
  Anything that has value to an organization and therefore requires protection.
}

Some examples of assets include:
\begin{itemize}
  \item Sensitive data (e.g., personal information, financial records, intellectual property)
  \item Proprietary information (e.g., trade secrets, business strategies)
  \item Systems and networks (e.g., servers, databases, communication infrastructure)
\end{itemize}

\dfn{Adversary}{
  An individual, group, or entity that poses a threat to the security of an asset.
}

Some examples of adversaries include:
\begin{itemize}
  \item Cybercriminals (e.g., hackers, malware authors)
  \item Hacktivists (e.g., politically motivated groups)
  \item Nation-state actors (e.g., government-sponsored hackers)
  \item Insider threats (e.g., disgruntled employees, contractors)
\end{itemize}

Approaches for security
\begin{itemize}
  \item Prevention
  \item Detection
  \item Response
  \item Recovery and remediation
  \item Policy and Mechanism
\end{itemize}




\end{document}
