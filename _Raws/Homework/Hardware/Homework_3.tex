\documentclass[12pt letter]{report}
\input{../template/preamble}
\input{../template/macros}
\input{../template/letterfonts}

\usepackage{parskip}
\usepackage{graphicx}
\graphicspath{ {assets/} }
\title{\Huge{Homework 3}}
\author{\huge{Madiba Hudson-Quansah}}
\date{}


\begin{document}
\maketitle
\newpage

\qs{}{
  Implement the following C code in MIPS assembly. Show the contents of the stack after the function call to the
  function "compare" is made. Assume that the stack pointer is originally at address 0x7FFFFFFc.
}

\sol{
  \begin{figure}[H]
    \begin{center}
      \includegraphics[width=0.8\textwidth]{q1.png}
    \end{center}
  \end{figure}
}

\qs{}{
  Implement the following C code in the table in MIPS assembly. Suppose that fib\_iter was called with n = 4, show the
  contents of the stack after the function call to the function "fib\_iter" is made. Assume that the stack pointer is originally at address 0x7ffffffc.
}

\sol{
  \begin{figure}[H]
    \begin{center}
      \includegraphics[width=0.8\textwidth]{q2.png}
    \end{center}
  \end{figure}
}

\qs{}{
  \begin{enumerate}
    \item  Translate function f into MIPS assembly code, using the MIPS calling convention. If you need to use register
          \$t0 through \$t7, use the lower-numbered registers first.
    \item Right before your function f of Problem 3 returns, what do you know about contents of registers \$ra, and \$sp? Keep in mind that we know what the entire function f looks like, but for function func we only know its declaration.
  \end{enumerate}
}

\sol{
  \begin{figure}[H]
    \begin{center}
      \includegraphics[width=0.8\textwidth]{q3.png}
    \end{center}
  \end{figure}
}



\end{document}
