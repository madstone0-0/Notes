\documentclass[12pt letter]{report}
%%%%%%%%%%%%%%%%%%%%%%%%%%%%%%%%%
% PACKAGE IMPORTS
%%%%%%%%%%%%%%%%%%%%%%%%%%%%%%%%%


\usepackage[tmargin=2cm,rmargin=1in,lmargin=1in,margin=0.85in,bmargin=2cm,footskip=.2in]{geometry}
\usepackage{amsmath,amsfonts,amsthm,amssymb,mathtools}
\usepackage[varbb]{newpxmath}
\usepackage{xfrac}
\usepackage[makeroom]{cancel}
\usepackage{mathtools}
\usepackage{bookmark}
\usepackage{enumitem}
\usepackage{hyperref,theoremref}
\hypersetup{
  pdftitle={Assignment},
  colorlinks=true, linkcolor=doc!90,
  bookmarksnumbered=true,
  bookmarksopen=true
}
\usepackage[most,many,breakable]{tcolorbox}
\usepackage{xcolor}
\usepackage{varwidth}
\usepackage{varwidth}
\usepackage{etoolbox}
%\usepackage{authblk}
\usepackage{nameref}
\usepackage{multicol,array}
\usepackage{tikz-cd}
\usepackage[ruled,vlined,linesnumbered]{algorithm2e}
\usepackage{comment} % enables the use of multi-line comments (\ifx \fi) 
\usepackage{import}
\usepackage{xifthen}
\usepackage{pdfpages}
\usepackage{transparent}
\usepackage{xcolor,colortbl,array,amssymb}
\usepackage{venndiagram}
\usepackage{listings}
\usepackage{fontspec}

\setmainfont{LibertinusSerif}[
  Extension = .otf,
  Path = /usr/share/fonts/libertinus/,
  UprightFont = *-Regular,
  ItalicFont = *-Italic,
  BoldFont = *-Bold,
  BoldItalicFont = *-BoldItalic,
]

\setmonofont{CaskaydiaCoveNerdFontMono}[
  Extension = .ttf,
  Path = /usr/share/fonts/TTF/,
  UprightFont = *-Regular,
  ItalicFont = *-Italic,
  BoldFont = *-Bold,
  BoldItalicFont = *-BoldItalic,
]

\definecolor{listing-background}{HTML}{F7F7F7}
\definecolor{listing-rule}{HTML}{B3B2B3}
\definecolor{listing-numbers}{HTML}{B3B2B3}
\definecolor{listing-text-color}{HTML}{000000}
\definecolor{listing-keyword}{HTML}{435489}
\definecolor{listing-keyword-2}{HTML}{1284CA} % additional keywords
\definecolor{listing-keyword-3}{HTML}{9137CB} % additional keywords
\definecolor{listing-identifier}{HTML}{435489}
\definecolor{listing-string}{HTML}{00999A}
\definecolor{listing-comment}{HTML}{8E8E8E}

\lstdefinestyle{eisvogel_listing_style}{
language         = java,
xleftmargin      = 0.6em,
framexleftmargin = 0.4em,
backgroundcolor  = \color{listing-background},
basicstyle       = \color{listing-text-color}\linespread{1.0}%
\lst@ifdisplaystyle%
\fi\ttfamily{},
breaklines       = true,
frame            = single,
framesep         = 0.19em,
rulecolor        = \color{listing-rule},
frameround       = ffff,
tabsize          = 4,
numberstyle      = \color{listing-numbers},
aboveskip        = 1.0em,
belowskip        = 0.1em,
abovecaptionskip = 0em,
belowcaptionskip = 1.0em,
keywordstyle     = {\color{listing-keyword}\bfseries},
keywordstyle     = {[2]\color{listing-keyword-2}\bfseries},
keywordstyle     = {[3]\color{listing-keyword-3}\bfseries\itshape},
sensitive        = true,
identifierstyle  = \color{listing-identifier},
commentstyle     = \color{listing-comment},
stringstyle      = \color{listing-string},
showstringspaces = false,
escapeinside     = {/*@}{@*/}, % Allow LaTeX inside these special comments
literate         =
  {á}{{\'a}}1 {é}{{\'e}}1 {í}{{\'i}}1 {ó}{{\'o}}1 {ú}{{\'u}}1
{Á}{{\'A}}1 {É}{{\'E}}1 {Í}{{\'I}}1 {Ó}{{\'O}}1 {Ú}{{\'U}}1
{à}{{\`a}}1 {è}{{\`e}}1 {ì}{{\`i}}1 {ò}{{\`o}}1 {ù}{{\`u}}1
{À}{{\`A}}1 {È}{{\`E}}1 {Ì}{{\`I}}1 {Ò}{{\`O}}1 {Ù}{{\`U}}1
{ä}{{\"a}}1 {ë}{{\"e}}1 {ï}{{\"i}}1 {ö}{{\"o}}1 {ü}{{\"u}}1
{Ä}{{\"A}}1 {Ë}{{\"E}}1 {Ï}{{\"I}}1 {Ö}{{\"O}}1 {Ü}{{\"U}}1
{â}{{\^a}}1 {ê}{{\^e}}1 {î}{{\^i}}1 {ô}{{\^o}}1 {û}{{\^u}}1
{Â}{{\^A}}1 {Ê}{{\^E}}1 {Î}{{\^I}}1 {Ô}{{\^O}}1 {Û}{{\^U}}1
{œ}{{\oe}}1 {Œ}{{\OE}}1 {æ}{{\ae}}1 {Æ}{{\AE}}1 {ß}{{\ss}}1
{ç}{{\c c}}1 {Ç}{{\c C}}1 {ø}{{\o}}1 {å}{{\r a}}1 {Å}{{\r A}}1
{€}{{\EUR}}1 {£}{{\pounds}}1 {«}{{\guillemotleft}}1
{»}{{\guillemotright}}1 {ñ}{{\~n}}1 {Ñ}{{\~N}}1 {¿}{{?`}}1
{…}{{\ldots}}1 {≥}{{>=}}1 {≤}{{<=}}1 {„}{{\glqq}}1 {“}{{\grqq}}1
{”}{{''}}1
}
\lstset{style=eisvogel_listing_style,
  numbers=left}

%
% Java (Java SE 12, 2019-06-22)
%
\lstdefinelanguage{Java}{
  morekeywords={
      % normal keywords (without data types)
      abstract,assert,break,case,catch,class,continue,default,
      do,else,enum,exports,extends,final,finally,for,if,implements,
      import,instanceof,interface,module,native,new,package,private,
      protected,public,requires,return,static,strictfp,super,switch,
      synchronized,this,throw,throws,transient,try,volatile,while,
      % var is an identifier
      var
    },
  morekeywords={[2] % data types
      % primitive data types
      boolean,byte,char,double,float,int,long,short,
      % String
      String,
      % primitive wrapper types
      Boolean,Byte,Character,Double,Float,Integer,Long,Short
      % number types
      Number,AtomicInteger,AtomicLong,BigDecimal,BigInteger,DoubleAccumulator,DoubleAdder,LongAccumulator,LongAdder,Short,
      % other
      Object,Void,void
    },
  morekeywords={[3] % literals
      % reserved words for literal values
      null,true,false,
    },
  sensitive,
  morecomment  = [l]//,
  morecomment  = [s]{/*}{*/},
  morecomment  = [s]{/**}{*/},
  morestring   = [b]",
  morestring   = [b]',
}

\lstdefinelanguage{XML}{
  morestring      = [b]",
  moredelim       = [s][\bfseries\color{listing-keyword}]{<}{\ },
  moredelim       = [s][\bfseries\color{listing-keyword}]{</}{>},
  moredelim       = [l][\bfseries\color{listing-keyword}]{/>},
  moredelim       = [l][\bfseries\color{listing-keyword}]{>},
  morecomment     = [s]{<?}{?>},
  morecomment     = [s]{<!--}{-->},
  commentstyle    = \color{listing-comment},
  stringstyle     = \color{listing-string},
  identifierstyle = \color{listing-identifier}
}


\newcommand\mycommfont[1]{\footnotesize\ttfamily\textcolor{blue}{#1}}
\SetCommentSty{mycommfont}
\newcommand{\incfig}[1]{%
  \def\svgwidth{\columnwidth}
  \import{./figures/}{#1.pdf_tex}
}

\usepackage{tikzsymbols}
\renewcommand\qedsymbol{$\Laughey$}


%\usepackage{import}
%\usepackage{xifthen}
%\usepackage{pdfpages}
%\usepackage{transparent}


%%%%%%%%%%%%%%%%%%%%%%%%%%%%%%
% SELF MADE COLORS
%%%%%%%%%%%%%%%%%%%%%%%%%%%%%%



\definecolor{myg}{RGB}{56, 140, 70}
\definecolor{myb}{RGB}{45, 111, 177}
\definecolor{myr}{RGB}{199, 68, 64}
\definecolor{mytheorembg}{HTML}{F2F2F9}
\definecolor{mytheoremfr}{HTML}{00007B}
\definecolor{mylenmabg}{HTML}{FFFAF8}
\definecolor{mylenmafr}{HTML}{983b0f}
\definecolor{mypropbg}{HTML}{f2fbfc}
\definecolor{mypropfr}{HTML}{191971}
\definecolor{myexamplebg}{HTML}{F2FBF8}
\definecolor{myexamplefr}{HTML}{88D6D1}
\definecolor{myexampleti}{HTML}{2A7F7F}
\definecolor{mydefinitbg}{HTML}{E5E5FF}
\definecolor{mydefinitfr}{HTML}{3F3FA3}
\definecolor{notesgreen}{RGB}{0,162,0}
\definecolor{myp}{RGB}{197, 92, 212}
\definecolor{mygr}{HTML}{2C3338}
\definecolor{myred}{RGB}{127,0,0}
\definecolor{myyellow}{RGB}{169,121,69}
\definecolor{myexercisebg}{HTML}{F2FBF8}
\definecolor{myexercisefg}{HTML}{88D6D1}


%%%%%%%%%%%%%%%%%%%%%%%%%%%%
% TCOLORBOX SETUPS
%%%%%%%%%%%%%%%%%%%%%%%%%%%%

\setlength{\parindent}{1cm}
%================================
% THEOREM BOX
%================================

\tcbuselibrary{theorems,skins,hooks}
\newtcbtheorem[number within=section]{Theorem}{Theorem}
{%
  enhanced,
  breakable,
  colback = mytheorembg,
  frame hidden,
  boxrule = 0sp,
  borderline west = {2pt}{0pt}{mytheoremfr},
  sharp corners,
  detach title,
  before upper = \tcbtitle\par\smallskip,
  coltitle = mytheoremfr,
  fonttitle = \bfseries\sffamily,
  description font = \mdseries,
  separator sign none,
  segmentation style={solid, mytheoremfr},
}
{th}

\tcbuselibrary{theorems,skins,hooks}
\newtcbtheorem[number within=chapter]{theorem}{Theorem}
{%
  enhanced,
  breakable,
  colback = mytheorembg,
  frame hidden,
  boxrule = 0sp,
  borderline west = {2pt}{0pt}{mytheoremfr},
  sharp corners,
  detach title,
  before upper = \tcbtitle\par\smallskip,
  coltitle = mytheoremfr,
  fonttitle = \bfseries\sffamily,
  description font = \mdseries,
  separator sign none,
  segmentation style={solid, mytheoremfr},
}
{th}


\tcbuselibrary{theorems,skins,hooks}
\newtcolorbox{Theoremcon}
{%
  enhanced
  ,breakable
  ,colback = mytheorembg
  ,frame hidden
  ,boxrule = 0sp
  ,borderline west = {2pt}{0pt}{mytheoremfr}
  ,sharp corners
  ,description font = \mdseries
  ,separator sign none
}

%================================
% Corollery
%================================
\tcbuselibrary{theorems,skins,hooks}
\newtcbtheorem[number within=section]{Corollary}{Corollary}
{%
  enhanced
  ,breakable
  ,colback = myp!10
  ,frame hidden
  ,boxrule = 0sp
  ,borderline west = {2pt}{0pt}{myp!85!black}
  ,sharp corners
  ,detach title
  ,before upper = \tcbtitle\par\smallskip
  ,coltitle = myp!85!black
  ,fonttitle = \bfseries\sffamily
  ,description font = \mdseries
  ,separator sign none
  ,segmentation style={solid, myp!85!black}
}
{th}
\tcbuselibrary{theorems,skins,hooks}
\newtcbtheorem[number within=chapter]{corollary}{Corollary}
{%
  enhanced
  ,breakable
  ,colback = myp!10
  ,frame hidden
  ,boxrule = 0sp
  ,borderline west = {2pt}{0pt}{myp!85!black}
  ,sharp corners
  ,detach title
  ,before upper = \tcbtitle\par\smallskip
  ,coltitle = myp!85!black
  ,fonttitle = \bfseries\sffamily
  ,description font = \mdseries
  ,separator sign none
  ,segmentation style={solid, myp!85!black}
}
{th}


%================================
% LENMA
%================================

\tcbuselibrary{theorems,skins,hooks}
\newtcbtheorem[number within=section]{Lenma}{Lenma}
{%
  enhanced,
  breakable,
  colback = mylenmabg,
  frame hidden,
  boxrule = 0sp,
  borderline west = {2pt}{0pt}{mylenmafr},
  sharp corners,
  detach title,
  before upper = \tcbtitle\par\smallskip,
  coltitle = mylenmafr,
  fonttitle = \bfseries\sffamily,
  description font = \mdseries,
  separator sign none,
  segmentation style={solid, mylenmafr},
}
{th}

\tcbuselibrary{theorems,skins,hooks}
\newtcbtheorem[number within=chapter]{lenma}{Lenma}
{%
  enhanced,
  breakable,
  colback = mylenmabg,
  frame hidden,
  boxrule = 0sp,
  borderline west = {2pt}{0pt}{mylenmafr},
  sharp corners,
  detach title,
  before upper = \tcbtitle\par\smallskip,
  coltitle = mylenmafr,
  fonttitle = \bfseries\sffamily,
  description font = \mdseries,
  separator sign none,
  segmentation style={solid, mylenmafr},
}
{th}


%================================
% PROPOSITION
%================================

\tcbuselibrary{theorems,skins,hooks}
\newtcbtheorem[number within=section]{Prop}{Proposition}
{%
  enhanced,
  breakable,
  colback = mypropbg,
  frame hidden,
  boxrule = 0sp,
  borderline west = {2pt}{0pt}{mypropfr},
  sharp corners,
  detach title,
  before upper = \tcbtitle\par\smallskip,
  coltitle = mypropfr,
  fonttitle = \bfseries\sffamily,
  description font = \mdseries,
  separator sign none,
  segmentation style={solid, mypropfr},
}
{th}

\tcbuselibrary{theorems,skins,hooks}
\newtcbtheorem[number within=chapter]{prop}{Proposition}
{%
  enhanced,
  breakable,
  colback = mypropbg,
  frame hidden,
  boxrule = 0sp,
  borderline west = {2pt}{0pt}{mypropfr},
  sharp corners,
  detach title,
  before upper = \tcbtitle\par\smallskip,
  coltitle = mypropfr,
  fonttitle = \bfseries\sffamily,
  description font = \mdseries,
  separator sign none,
  segmentation style={solid, mypropfr},
}
{th}


%================================
% CLAIM
%================================

\tcbuselibrary{theorems,skins,hooks}
\newtcbtheorem[number within=section]{claim}{Claim}
{%
  enhanced
  ,breakable
  ,colback = myg!10
  ,frame hidden
  ,boxrule = 0sp
  ,borderline west = {2pt}{0pt}{myg}
  ,sharp corners
  ,detach title
  ,before upper = \tcbtitle\par\smallskip
  ,coltitle = myg!85!black
  ,fonttitle = \bfseries\sffamily
  ,description font = \mdseries
  ,separator sign none
  ,segmentation style={solid, myg!85!black}
}
{th}



%================================
% Exercise
%================================

\tcbuselibrary{theorems,skins,hooks}
\newtcbtheorem[number within=section]{Exercise}{Exercise}
{%
  enhanced,
  breakable,
  colback = myexercisebg,
  frame hidden,
  boxrule = 0sp,
  borderline west = {2pt}{0pt}{myexercisefg},
  sharp corners,
  detach title,
  before upper = \tcbtitle\par\smallskip,
  coltitle = myexercisefg,
  fonttitle = \bfseries\sffamily,
  description font = \mdseries,
  separator sign none,
  segmentation style={solid, myexercisefg},
}
{th}

\tcbuselibrary{theorems,skins,hooks}
\newtcbtheorem[number within=chapter]{exercise}{Exercise}
{%
  enhanced,
  breakable,
  colback = myexercisebg,
  frame hidden,
  boxrule = 0sp,
  borderline west = {2pt}{0pt}{myexercisefg},
  sharp corners,
  detach title,
  before upper = \tcbtitle\par\smallskip,
  coltitle = myexercisefg,
  fonttitle = \bfseries\sffamily,
  description font = \mdseries,
  separator sign none,
  segmentation style={solid, myexercisefg},
}
{th}

%================================
% EXAMPLE BOX
%================================

\newtcbtheorem[number within=section]{Example}{Example}
{%
  colback = myexamplebg
  ,breakable
  ,colframe = myexamplefr
  ,coltitle = myexampleti
  ,boxrule = 1pt
  ,sharp corners
  ,detach title
  ,before upper=\tcbtitle\par\smallskip
  ,fonttitle = \bfseries
  ,description font = \mdseries
  ,separator sign none
  ,description delimiters parenthesis
}
{ex}

\newtcbtheorem[number within=chapter]{example}{Example}
{%
  colback = myexamplebg
  ,breakable
  ,colframe = myexamplefr
  ,coltitle = myexampleti
  ,boxrule = 1pt
  ,sharp corners
  ,detach title
  ,before upper=\tcbtitle\par\smallskip
  ,fonttitle = \bfseries
  ,description font = \mdseries
  ,separator sign none
  ,description delimiters parenthesis
}
{ex}

%================================
% DEFINITION BOX
%================================

\newtcbtheorem[number within=section]{Definition}{Definition}{enhanced,
  before skip=2mm,after skip=2mm, colback=red!5,colframe=red!80!black,boxrule=0.5mm,
  attach boxed title to top left={xshift=1cm,yshift*=1mm-\tcboxedtitleheight}, varwidth boxed title*=-3cm,
  boxed title style={frame code={
          \path[fill=tcbcolback]
          ([yshift=-1mm,xshift=-1mm]frame.north west)
          arc[start angle=0,end angle=180,radius=1mm]
          ([yshift=-1mm,xshift=1mm]frame.north east)
          arc[start angle=180,end angle=0,radius=1mm];
          \path[left color=tcbcolback!60!black,right color=tcbcolback!60!black,
            middle color=tcbcolback!80!black]
          ([xshift=-2mm]frame.north west) -- ([xshift=2mm]frame.north east)
          [rounded corners=1mm]-- ([xshift=1mm,yshift=-1mm]frame.north east)
          -- (frame.south east) -- (frame.south west)
          -- ([xshift=-1mm,yshift=-1mm]frame.north west)
          [sharp corners]-- cycle;
        },interior engine=empty,
    },
  fonttitle=\bfseries,
  title={#2},#1}{def}
\newtcbtheorem[number within=chapter]{definition}{Definition}{enhanced,
  before skip=2mm,after skip=2mm, colback=red!5,colframe=red!80!black,boxrule=0.5mm,
  attach boxed title to top left={xshift=1cm,yshift*=1mm-\tcboxedtitleheight}, varwidth boxed title*=-3cm,
  boxed title style={frame code={
          \path[fill=tcbcolback]
          ([yshift=-1mm,xshift=-1mm]frame.north west)
          arc[start angle=0,end angle=180,radius=1mm]
          ([yshift=-1mm,xshift=1mm]frame.north east)
          arc[start angle=180,end angle=0,radius=1mm];
          \path[left color=tcbcolback!60!black,right color=tcbcolback!60!black,
            middle color=tcbcolback!80!black]
          ([xshift=-2mm]frame.north west) -- ([xshift=2mm]frame.north east)
          [rounded corners=1mm]-- ([xshift=1mm,yshift=-1mm]frame.north east)
          -- (frame.south east) -- (frame.south west)
          -- ([xshift=-1mm,yshift=-1mm]frame.north west)
          [sharp corners]-- cycle;
        },interior engine=empty,
    },
  fonttitle=\bfseries,
  title={#2},#1}{def}



%================================
% Solution BOX
%================================

\makeatletter
\newtcbtheorem{question}{Question}{enhanced,
  breakable,
  colback=white,
  colframe=myb!80!black,
  attach boxed title to top left={yshift*=-\tcboxedtitleheight},
  fonttitle=\bfseries,
  title={#2},
  boxed title size=title,
  boxed title style={%
      sharp corners,
      rounded corners=northwest,
      colback=tcbcolframe,
      boxrule=0pt,
    },
  underlay boxed title={%
      \path[fill=tcbcolframe] (title.south west)--(title.south east)
      to[out=0, in=180] ([xshift=5mm]title.east)--
      (title.center-|frame.east)
      [rounded corners=\kvtcb@arc] |-
      (frame.north) -| cycle;
    },
  #1
}{def}
\makeatother

%================================
% SOLUTION BOX
%================================

\makeatletter
\newtcolorbox{solution}{enhanced,
  breakable,
  colback=white,
  colframe=myg!80!black,
  attach boxed title to top left={yshift*=-\tcboxedtitleheight},
  title=Solution,
  boxed title size=title,
  boxed title style={%
      sharp corners,
      rounded corners=northwest,
      colback=tcbcolframe,
      boxrule=0pt,
    },
  underlay boxed title={%
      \path[fill=tcbcolframe] (title.south west)--(title.south east)
      to[out=0, in=180] ([xshift=5mm]title.east)--
      (title.center-|frame.east)
      [rounded corners=\kvtcb@arc] |-
      (frame.north) -| cycle;
    },
}
\makeatother

%================================
% Question BOX
%================================

\makeatletter
\newtcbtheorem{qstion}{Question}{enhanced,
  breakable,
  colback=white,
  colframe=mygr,
  attach boxed title to top left={yshift*=-\tcboxedtitleheight},
  fonttitle=\bfseries,
  title={#2},
  boxed title size=title,
  boxed title style={%
      sharp corners,
      rounded corners=northwest,
      colback=tcbcolframe,
      boxrule=0pt,
    },
  underlay boxed title={%
      \path[fill=tcbcolframe] (title.south west)--(title.south east)
      to[out=0, in=180] ([xshift=5mm]title.east)--
      (title.center-|frame.east)
      [rounded corners=\kvtcb@arc] |-
      (frame.north) -| cycle;
    },
  #1
}{def}
\makeatother

\newtcbtheorem[number within=chapter]{wconc}{Wrong Concept}{
  breakable,
  enhanced,
  colback=white,
  colframe=myr,
  arc=0pt,
  outer arc=0pt,
  fonttitle=\bfseries\sffamily\large,
  colbacktitle=myr,
  attach boxed title to top left={},
  boxed title style={
      enhanced,
      skin=enhancedfirst jigsaw,
      arc=3pt,
      bottom=0pt,
      interior style={fill=myr}
    },
  #1
}{def}



%================================
% NOTE BOX
%================================

\usetikzlibrary{arrows,calc,shadows.blur}
\tcbuselibrary{skins}
\newtcolorbox{note}[1][]{%
  enhanced jigsaw,
  colback=gray!20!white,%
  colframe=gray!80!black,
  size=small,
  boxrule=1pt,
  title=\textbf{Note:-},
  halign title=flush center,
  coltitle=black,
  breakable,
  drop shadow=black!50!white,
  attach boxed title to top left={xshift=1cm,yshift=-\tcboxedtitleheight/2,yshifttext=-\tcboxedtitleheight/2},
  minipage boxed title=1.5cm,
  boxed title style={%
      colback=white,
      size=fbox,
      boxrule=1pt,
      boxsep=2pt,
      underlay={%
          \coordinate (dotA) at ($(interior.west) + (-0.5pt,0)$);
          \coordinate (dotB) at ($(interior.east) + (0.5pt,0)$);
          \begin{scope}
            \clip (interior.north west) rectangle ([xshift=3ex]interior.east);
            \filldraw [white, blur shadow={shadow opacity=60, shadow yshift=-.75ex}, rounded corners=2pt] (interior.north west) rectangle (interior.south east);
          \end{scope}
          \begin{scope}[gray!80!black]
            \fill (dotA) circle (2pt);
            \fill (dotB) circle (2pt);
          \end{scope}
        },
    },
  #1,
}

%%%%%%%%%%%%%%%%%%%%%%%%%%%%%%
% SELF MADE COMMANDS
%%%%%%%%%%%%%%%%%%%%%%%%%%%%%%


\newcommand{\thm}[2]{\begin{Theorem}{#1}{}#2\end{Theorem}}
\newcommand{\cor}[2]{\begin{Corollary}{#1}{}#2\end{Corollary}}
\newcommand{\mlenma}[2]{\begin{Lenma}{#1}{}#2\end{Lenma}}
\newcommand{\mprop}[2]{\begin{Prop}{#1}{}#2\end{Prop}}
\newcommand{\clm}[3]{\begin{claim}{#1}{#2}#3\end{claim}}
\newcommand{\wc}[2]{\begin{wconc}{#1}{}\setlength{\parindent}{1cm}#2\end{wconc}}
\newcommand{\thmcon}[1]{\begin{Theoremcon}{#1}\end{Theoremcon}}
\newcommand{\ex}[2]{\begin{Example}{#1}{}#2\end{Example}}
\newcommand{\dfn}[2]{\begin{Definition}[colbacktitle=red!75!black]{#1}{}#2\end{Definition}}
\newcommand{\dfnc}[2]{\begin{definition}[colbacktitle=red!75!black]{#1}{}#2\end{definition}}
\newcommand{\qs}[2]{\begin{question}{#1}{}#2\end{question}}
\newcommand{\pf}[2]{\begin{myproof}[#1]#2\end{myproof}}
\newcommand{\nt}[1]{\begin{note}#1\end{note}}

\newcommand*\circled[1]{\tikz[baseline=(char.base)]{
    \node[shape=circle,draw,inner sep=1pt] (char) {#1};}}
\newcommand\getcurrentref[1]{%
  \ifnumequal{\value{#1}}{0}
  {??}
  {\the\value{#1}}%
}
\newcommand{\getCurrentSectionNumber}{\getcurrentref{section}}
\newenvironment{myproof}[1][\proofname]{%
  \proof[\bfseries #1: ]%
}{\endproof}

\newcommand{\mclm}[2]{\begin{myclaim}[#1]#2\end{myclaim}}
\newenvironment{myclaim}[1][\claimname]{\proof[\bfseries #1: ]}{}

\newcounter{mylabelcounter}

\makeatletter
\newcommand{\setword}[2]{%
  \phantomsection
  #1\def\@currentlabel{\unexpanded{#1}}\label{#2}%
}
\makeatother




\tikzset{
  symbol/.style={
      draw=none,
      every to/.append style={
          edge node={node [sloped, allow upside down, auto=false]{$#1$}}}
    }
}


% deliminators
\DeclarePairedDelimiter{\abs}{\lvert}{\rvert}
\DeclarePairedDelimiter{\norm}{\lVert}{\rVert}

\DeclarePairedDelimiter{\ceil}{\lceil}{\rceil}
\DeclarePairedDelimiter{\floor}{\lfloor}{\rfloor}
\DeclarePairedDelimiter{\round}{\lfloor}{\rceil}

\newsavebox\diffdbox
\newcommand{\slantedromand}{{\mathpalette\makesl{d}}}
\newcommand{\makesl}[2]{%
  \begingroup
  \sbox{\diffdbox}{$\mathsurround=0pt#1\mathrm{#2}$}%
  \pdfsave
  \pdfsetmatrix{1 0 0.2 1}%
  \rlap{\usebox{\diffdbox}}%
  \pdfrestore
  \hskip\wd\diffdbox
  \endgroup
}
\newcommand{\dd}[1][]{\ensuremath{\mathop{}\!\ifstrempty{#1}{%
      \slantedromand\@ifnextchar^{\hspace{0.2ex}}{\hspace{0.1ex}}}%
    {\slantedromand\hspace{0.2ex}^{#1}}}}
\ProvideDocumentCommand\dv{o m g}{%
  \ensuremath{%
    \IfValueTF{#3}{%
      \IfNoValueTF{#1}{%
        \frac{\dd #2}{\dd #3}%
      }{%
        \frac{\dd^{#1} #2}{\dd #3^{#1}}%
      }%
    }{%
      \IfNoValueTF{#1}{%
        \frac{\dd}{\dd #2}%
      }{%
        \frac{\dd^{#1}}{\dd #2^{#1}}%
      }%
    }%
  }%
}
\providecommand*{\pdv}[3][]{\frac{\partial^{#1}#2}{\partial#3^{#1}}}
%  - others
\DeclareMathOperator{\Lap}{\mathcal{L}}
\DeclareMathOperator{\Var}{Var} % varience
\DeclareMathOperator{\Cov}{Cov} % covarience
\DeclareMathOperator{\E}{E} % expected

% Since the amsthm package isn't loaded

% I prefer the slanted \leq
\let\oldleq\leq % save them in case they're every wanted
\let\oldgeq\geq
\renewcommand{\leq}{\leqslant}
\renewcommand{\geq}{\geqslant}

% % redefine matrix env to allow for alignment, use r as default
% \renewcommand*\env@matrix[1][r]{\hskip -\arraycolsep
%     \let\@ifnextchar\new@ifnextchar
%     \array{*\c@MaxMatrixCols #1}}


%\usepackage{framed}
%\usepackage{titletoc}
%\usepackage{etoolbox}
%\usepackage{lmodern}


%\patchcmd{\tableofcontents}{\contentsname}{\sffamily\contentsname}{}{}

%\renewenvironment{leftbar}
%{\def\FrameCommand{\hspace{6em}%
%		{\color{myyellow}\vrule width 2pt depth 6pt}\hspace{1em}}%
%	\MakeFramed{\parshape 1 0cm \dimexpr\textwidth-6em\relax\FrameRestore}\vskip2pt%
%}
%{\endMakeFramed}

%\titlecontents{chapter}
%[0em]{\vspace*{2\baselineskip}}
%{\parbox{4.5em}{%
%		\hfill\Huge\sffamily\bfseries\color{myred}\thecontentspage}%
%	\vspace*{-2.3\baselineskip}\leftbar\textsc{\small\chaptername~\thecontentslabel}\\\sffamily}
%{}{\endleftbar}
%\titlecontents{section}
%[8.4em]
%{\sffamily\contentslabel{3em}}{}{}
%{\hspace{0.5em}\nobreak\itshape\color{myred}\contentspage}
%\titlecontents{subsection}
%[8.4em]
%{\sffamily\contentslabel{3em}}{}{}  
%{\hspace{0.5em}\nobreak\itshape\color{myred}\contentspage}



%%%%%%%%%%%%%%%%%%%%%%%%%%%%%%%%%%%%%%%%%%%
% TABLE OF CONTENTS
%%%%%%%%%%%%%%%%%%%%%%%%%%%%%%%%%%%%%%%%%%%

\usepackage{tikz}
\definecolor{doc}{RGB}{0,60,110}
\usepackage{titletoc}
\contentsmargin{0cm}
\titlecontents{chapter}[3.7pc]
{\addvspace{30pt}%
  \begin{tikzpicture}[remember picture, overlay]%
    \draw[fill=doc!60,draw=doc!60] (-7,-.1) rectangle (-0.9,.5);%
    \pgftext[left,x=-3.5cm,y=0.2cm]{\color{white}\Large\sc\bfseries Chapter\ \thecontentslabel};%
  \end{tikzpicture}\color{doc!60}\large\sc\bfseries}%
{}
{}
{\;\titlerule\;\large\sc\bfseries Page \thecontentspage
  \begin{tikzpicture}[remember picture, overlay]
    \draw[fill=doc!60,draw=doc!60] (2pt,0) rectangle (4,0.1pt);
  \end{tikzpicture}}%
\titlecontents{section}[3.7pc]
{\addvspace{2pt}}
{\contentslabel[\thecontentslabel]{2pc}}
{}
{\hfill\small \thecontentspage}
[]
\titlecontents*{subsection}[3.7pc]
{\addvspace{-1pt}\small}
{}
{}
{\ --- \small\thecontentspage}
[ \textbullet\ ][]

\makeatletter
\renewcommand{\tableofcontents}{%
  \chapter*{%
    \vspace*{-20\p@}%
    \begin{tikzpicture}[remember picture, overlay]%
      \pgftext[right,x=15cm,y=0.2cm]{\color{doc!60}\Huge\sc\bfseries \contentsname};%
      \draw[fill=doc!60,draw=doc!60] (13,-.75) rectangle (20,1);%
      \clip (13,-.75) rectangle (20,1);
      \pgftext[right,x=15cm,y=0.2cm]{\color{white}\Huge\sc\bfseries \contentsname};%
    \end{tikzpicture}}%
  \@starttoc{toc}}
\makeatother

%From M275 "Topology" at SJSU
\newcommand{\id}{\mathrm{id}}
\newcommand{\taking}[1]{\xrightarrow{#1}}
\newcommand{\inv}{^{-1}}

%From M170 "Introduction to Graph Theory" at SJSU
\DeclareMathOperator{\diam}{diam}
\DeclareMathOperator{\ord}{ord}
\newcommand{\defeq}{\overset{\mathrm{def}}{=}}

%From the USAMO .tex files
\newcommand{\ts}{\textsuperscript}
\newcommand{\dg}{^\circ}
\newcommand{\ii}{\item}

% % From Math 55 and Math 145 at Harvard
% \newenvironment{subproof}[1][Proof]{%
% \begin{proof}[#1] \renewcommand{\qedsymbol}{$\blacksquare$}}%
% {\end{proof}}

\newcommand{\liff}{\leftrightarrow}
\newcommand{\lthen}{\rightarrow}
\newcommand{\opname}{\operatorname}
\newcommand{\surjto}{\twoheadrightarrow}
\newcommand{\injto}{\hookrightarrow}
\newcommand{\On}{\mathrm{On}} % ordinals
\DeclareMathOperator{\img}{im} % Image
\DeclareMathOperator{\Img}{Im} % Image
\DeclareMathOperator{\coker}{coker} % Cokernel
\DeclareMathOperator{\Coker}{Coker} % Cokernel
\DeclareMathOperator{\Ker}{Ker} % Kernel
\DeclareMathOperator{\rank}{rank}
\DeclareMathOperator{\Spec}{Spec} % spectrum
\DeclareMathOperator{\Tr}{Tr} % trace
\DeclareMathOperator{\pr}{pr} % projection
\DeclareMathOperator{\ext}{ext} % extension
\DeclareMathOperator{\pred}{pred} % predecessor
\DeclareMathOperator{\dom}{dom} % domain
\DeclareMathOperator{\ran}{ran} % range
\DeclareMathOperator{\Hom}{Hom} % homomorphism
\DeclareMathOperator{\Mor}{Mor} % morphisms
\DeclareMathOperator{\End}{End} % endomorphism

\newcommand{\eps}{\epsilon}
\newcommand{\veps}{\varepsilon}
\newcommand{\ol}{\overline}
\newcommand{\ul}{\underline}
\newcommand{\wt}{\widetilde}
\newcommand{\wh}{\widehat}
\newcommand{\vocab}[1]{\textbf{\color{blue} #1}}
\providecommand{\half}{\frac{1}{2}}
\newcommand{\dang}{\measuredangle} %% Directed angle
\newcommand{\ray}[1]{\overrightarrow{#1}}
\newcommand{\seg}[1]{\overline{#1}}
\newcommand{\arc}[1]{\wideparen{#1}}
\DeclareMathOperator{\cis}{cis}
\DeclareMathOperator*{\lcm}{lcm}
\DeclareMathOperator*{\argmin}{arg min}
\DeclareMathOperator*{\argmax}{arg max}
\newcommand{\cycsum}{\sum_{\mathrm{cyc}}}
\newcommand{\symsum}{\sum_{\mathrm{sym}}}
\newcommand{\cycprod}{\prod_{\mathrm{cyc}}}
\newcommand{\symprod}{\prod_{\mathrm{sym}}}
\newcommand{\Qed}{\begin{flushright}\qed\end{flushright}}
\newcommand{\parinn}{\setlength{\parindent}{1cm}}
\newcommand{\parinf}{\setlength{\parindent}{0cm}}
% \newcommand{\norm}{\|\cdot\|}
\newcommand{\inorm}{\norm_{\infty}}
\newcommand{\opensets}{\{V_{\alpha}\}_{\alpha\in I}}
\newcommand{\oset}{V_{\alpha}}
\newcommand{\opset}[1]{V_{\alpha_{#1}}}
\newcommand{\lub}{\text{lub}}
\newcommand{\del}[2]{\frac{\partial #1}{\partial #2}}
\newcommand{\Del}[3]{\frac{\partial^{#1} #2}{\partial^{#1} #3}}
\newcommand{\deld}[2]{\dfrac{\partial #1}{\partial #2}}
\newcommand{\Deld}[3]{\dfrac{\partial^{#1} #2}{\partial^{#1} #3}}
\newcommand{\lm}{\lambda}
\newcommand{\uin}{\mathbin{\rotatebox[origin=c]{90}{$\in$}}}
\newcommand{\usubset}{\mathbin{\rotatebox[origin=c]{90}{$\subset$}}}
\newcommand{\lt}{\left}
\newcommand{\rt}{\right}
\newcommand{\bs}[1]{\boldsymbol{#1}}
\newcommand{\exs}{\exists}
\newcommand{\st}{\strut}
\newcommand{\dps}[1]{\displaystyle{#1}}

\newcommand{\sol}{\setlength{\parindent}{0cm}\textbf{\textit{Solution:}}\setlength{\parindent}{1cm} }
\newcommand{\solve}[1]{\setlength{\parindent}{0cm}\textbf{\textit{Solution: }}\setlength{\parindent}{1cm}#1 \Qed}

\preto\tabular{\setcounter{magicrownumbers}{0}}
\newcounter{magicrownumbers}
\newcommand\rownumber{\stepcounter{magicrownumbers}\arabic{magicrownumbers}}
\def\rownumber{}

\newenvironment{deduction}
{\begin{tabular}{@{}>{$}c<{$}@{\enspace}>{$}l<{$}@{}}\arrayrulecolor{blue!50}}
		{\end{tabular}}
\newcommand{\premise}[1]{&#1\\}
\newcommand{\conclusion}[1]{\cline{2-2}\therefore&#1}


% Things Lie
\newcommand{\kb}{\mathfrak b}
\newcommand{\kg}{\mathfrak g}
\newcommand{\kh}{\mathfrak h}
\newcommand{\kn}{\mathfrak n}
\newcommand{\ku}{\mathfrak u}
\newcommand{\kz}{\mathfrak z}
\DeclareMathOperator{\Ext}{Ext} % Ext functor
\DeclareMathOperator{\Tor}{Tor} % Tor functor
\newcommand{\gl}{\opname{\mathfrak{gl}}} % frak gl group
\renewcommand{\sl}{\opname{\mathfrak{sl}}} % frak sl group chktex 6

% More script letters etc.
\newcommand{\SA}{\mathcal A}
\newcommand{\SB}{\mathcal B}
\newcommand{\SC}{\mathcal C}
\newcommand{\SF}{\mathcal F}
\newcommand{\SG}{\mathcal G}
\newcommand{\SH}{\mathcal H}
\newcommand{\OO}{\mathcal O}

\newcommand{\SCA}{\mathscr A}
\newcommand{\SCB}{\mathscr B}
\newcommand{\SCC}{\mathscr C}
\newcommand{\SCD}{\mathscr D}
\newcommand{\SCE}{\mathscr E}
\newcommand{\SCF}{\mathscr F}
\newcommand{\SCG}{\mathscr G}
\newcommand{\SCH}{\mathscr H}

% Mathfrak primes
\newcommand{\km}{\mathfrak m}
\newcommand{\kp}{\mathfrak p}
\newcommand{\kq}{\mathfrak q}

% number sets
\newcommand{\RR}[1][]{\ensuremath{\ifstrempty{#1}{\mathbb{R}}{\mathbb{R}^{#1}}}}
\newcommand{\NN}[1][]{\ensuremath{\ifstrempty{#1}{\mathbb{N}}{\mathbb{N}^{#1}}}}
\newcommand{\ZZ}[1][]{\ensuremath{\ifstrempty{#1}{\mathbb{Z}}{\mathbb{Z}^{#1}}}}
\newcommand{\QQ}[1][]{\ensuremath{\ifstrempty{#1}{\mathbb{Q}}{\mathbb{Q}^{#1}}}}
\newcommand{\CC}[1][]{\ensuremath{\ifstrempty{#1}{\mathbb{C}}{\mathbb{C}^{#1}}}}
\newcommand{\PP}[1][]{\ensuremath{\ifstrempty{#1}{\mathbb{P}}{\mathbb{P}^{#1}}}}
\newcommand{\HH}[1][]{\ensuremath{\ifstrempty{#1}{\mathbb{H}}{\mathbb{H}^{#1}}}}
\newcommand{\FF}[1][]{\ensuremath{\ifstrempty{#1}{\mathbb{F}}{\mathbb{F}^{#1}}}}
% expected value
\newcommand{\EE}{\ensuremath{\mathbb{E}}}
\newcommand{\charin}{\text{ char }}
\DeclareMathOperator{\sign}{sign}
\DeclareMathOperator{\Aut}{Aut}
\DeclareMathOperator{\Inn}{Inn}
\DeclareMathOperator{\Syl}{Syl}
\DeclareMathOperator{\Gal}{Gal}
\DeclareMathOperator{\GL}{GL} % General linear group
\DeclareMathOperator{\SL}{SL} % Special linear group

%---------------------------------------
% BlackBoard Math Fonts :-
%---------------------------------------

%Captital Letters
\newcommand{\bbA}{\mathbb{A}}	\newcommand{\bbB}{\mathbb{B}}
\newcommand{\bbC}{\mathbb{C}}	\newcommand{\bbD}{\mathbb{D}}
\newcommand{\bbE}{\mathbb{E}}	\newcommand{\bbF}{\mathbb{F}}
\newcommand{\bbG}{\mathbb{G}}	\newcommand{\bbH}{\mathbb{H}}
\newcommand{\bbI}{\mathbb{I}}	\newcommand{\bbJ}{\mathbb{J}}
\newcommand{\bbK}{\mathbb{K}}	\newcommand{\bbL}{\mathbb{L}}
\newcommand{\bbM}{\mathbb{M}}	\newcommand{\bbN}{\mathbb{N}}
\newcommand{\bbO}{\mathbb{O}}	\newcommand{\bbP}{\mathbb{P}}
\newcommand{\bbQ}{\mathbb{Q}}	\newcommand{\bbR}{\mathbb{R}}
\newcommand{\bbS}{\mathbb{S}}	\newcommand{\bbT}{\mathbb{T}}
\newcommand{\bbU}{\mathbb{U}}	\newcommand{\bbV}{\mathbb{V}}
\newcommand{\bbW}{\mathbb{W}}	\newcommand{\bbX}{\mathbb{X}}
\newcommand{\bbY}{\mathbb{Y}}	\newcommand{\bbZ}{\mathbb{Z}}

%---------------------------------------
% MathCal Fonts :-
%---------------------------------------

%Captital Letters
\newcommand{\mcA}{\mathcal{A}}	\newcommand{\mcB}{\mathcal{B}}
\newcommand{\mcC}{\mathcal{C}}	\newcommand{\mcD}{\mathcal{D}}
\newcommand{\mcE}{\mathcal{E}}	\newcommand{\mcF}{\mathcal{F}}
\newcommand{\mcG}{\mathcal{G}}	\newcommand{\mcH}{\mathcal{H}}
\newcommand{\mcI}{\mathcal{I}}	\newcommand{\mcJ}{\mathcal{J}}
\newcommand{\mcK}{\mathcal{K}}	\newcommand{\mcL}{\mathcal{L}}
\newcommand{\mcM}{\mathcal{M}}	\newcommand{\mcN}{\mathcal{N}}
\newcommand{\mcO}{\mathcal{O}}	\newcommand{\mcP}{\mathcal{P}}
\newcommand{\mcQ}{\mathcal{Q}}	\newcommand{\mcR}{\mathcal{R}}
\newcommand{\mcS}{\mathcal{S}}	\newcommand{\mcT}{\mathcal{T}}
\newcommand{\mcU}{\mathcal{U}}	\newcommand{\mcV}{\mathcal{V}}
\newcommand{\mcW}{\mathcal{W}}	\newcommand{\mcX}{\mathcal{X}}
\newcommand{\mcY}{\mathcal{Y}}	\newcommand{\mcZ}{\mathcal{Z}}


%---------------------------------------
% Bold Math Fonts :-
%---------------------------------------

%Captital Letters
\newcommand{\bmA}{\boldsymbol{A}}	\newcommand{\bmB}{\boldsymbol{B}}
\newcommand{\bmC}{\boldsymbol{C}}	\newcommand{\bmD}{\boldsymbol{D}}
\newcommand{\bmE}{\boldsymbol{E}}	\newcommand{\bmF}{\boldsymbol{F}}
\newcommand{\bmG}{\boldsymbol{G}}	\newcommand{\bmH}{\boldsymbol{H}}
\newcommand{\bmI}{\boldsymbol{I}}	\newcommand{\bmJ}{\boldsymbol{J}}
\newcommand{\bmK}{\boldsymbol{K}}	\newcommand{\bmL}{\boldsymbol{L}}
\newcommand{\bmM}{\boldsymbol{M}}	\newcommand{\bmN}{\boldsymbol{N}}
\newcommand{\bmO}{\boldsymbol{O}}	\newcommand{\bmP}{\boldsymbol{P}}
\newcommand{\bmQ}{\boldsymbol{Q}}	\newcommand{\bmR}{\boldsymbol{R}}
\newcommand{\bmS}{\boldsymbol{S}}	\newcommand{\bmT}{\boldsymbol{T}}
\newcommand{\bmU}{\boldsymbol{U}}	\newcommand{\bmV}{\boldsymbol{V}}
\newcommand{\bmW}{\boldsymbol{W}}	\newcommand{\bmX}{\boldsymbol{X}}
\newcommand{\bmY}{\boldsymbol{Y}}	\newcommand{\bmZ}{\boldsymbol{Z}}
%Small Letters
\newcommand{\bma}{\boldsymbol{a}}	\newcommand{\bmb}{\boldsymbol{b}}
\newcommand{\bmc}{\boldsymbol{c}}	\newcommand{\bmd}{\boldsymbol{d}}
\newcommand{\bme}{\boldsymbol{e}}	\newcommand{\bmf}{\boldsymbol{f}}
\newcommand{\bmg}{\boldsymbol{g}}	\newcommand{\bmh}{\boldsymbol{h}}
\newcommand{\bmi}{\boldsymbol{i}}	\newcommand{\bmj}{\boldsymbol{j}}
\newcommand{\bmk}{\boldsymbol{k}}	\newcommand{\bml}{\boldsymbol{l}}
\newcommand{\bmm}{\boldsymbol{m}}	\newcommand{\bmn}{\boldsymbol{n}}
\newcommand{\bmo}{\boldsymbol{o}}	\newcommand{\bmp}{\boldsymbol{p}}
\newcommand{\bmq}{\boldsymbol{q}}	\newcommand{\bmr}{\boldsymbol{r}}
\newcommand{\bms}{\boldsymbol{s}}	\newcommand{\bmt}{\boldsymbol{t}}
\newcommand{\bmu}{\boldsymbol{u}}	\newcommand{\bmv}{\boldsymbol{v}}
\newcommand{\bmw}{\boldsymbol{w}}	\newcommand{\bmx}{\boldsymbol{x}}
\newcommand{\bmy}{\boldsymbol{y}}	\newcommand{\bmz}{\boldsymbol{z}}

%---------------------------------------
% Scr Math Fonts :-
%---------------------------------------

\newcommand{\sA}{{\mathscr{A}}}   \newcommand{\sB}{{\mathscr{B}}}
\newcommand{\sC}{{\mathscr{C}}}   \newcommand{\sD}{{\mathscr{D}}}
\newcommand{\sE}{{\mathscr{E}}}   \newcommand{\sF}{{\mathscr{F}}}
\newcommand{\sG}{{\mathscr{G}}}   \newcommand{\sH}{{\mathscr{H}}}
\newcommand{\sI}{{\mathscr{I}}}   \newcommand{\sJ}{{\mathscr{J}}}
\newcommand{\sK}{{\mathscr{K}}}   \newcommand{\sL}{{\mathscr{L}}}
\newcommand{\sM}{{\mathscr{M}}}   \newcommand{\sN}{{\mathscr{N}}}
\newcommand{\sO}{{\mathscr{O}}}   \newcommand{\sP}{{\mathscr{P}}}
\newcommand{\sQ}{{\mathscr{Q}}}   \newcommand{\sR}{{\mathscr{R}}}
\newcommand{\sS}{{\mathscr{S}}}   \newcommand{\sT}{{\mathscr{T}}}
\newcommand{\sU}{{\mathscr{U}}}   \newcommand{\sV}{{\mathscr{V}}}
\newcommand{\sW}{{\mathscr{W}}}   \newcommand{\sX}{{\mathscr{X}}}
\newcommand{\sY}{{\mathscr{Y}}}   \newcommand{\sZ}{{\mathscr{Z}}}


%---------------------------------------
% Math Fraktur Font
%---------------------------------------

%Captital Letters
\newcommand{\mfA}{\mathfrak{A}}	\newcommand{\mfB}{\mathfrak{B}}
\newcommand{\mfC}{\mathfrak{C}}	\newcommand{\mfD}{\mathfrak{D}}
\newcommand{\mfE}{\mathfrak{E}}	\newcommand{\mfF}{\mathfrak{F}}
\newcommand{\mfG}{\mathfrak{G}}	\newcommand{\mfH}{\mathfrak{H}}
\newcommand{\mfI}{\mathfrak{I}}	\newcommand{\mfJ}{\mathfrak{J}}
\newcommand{\mfK}{\mathfrak{K}}	\newcommand{\mfL}{\mathfrak{L}}
\newcommand{\mfM}{\mathfrak{M}}	\newcommand{\mfN}{\mathfrak{N}}
\newcommand{\mfO}{\mathfrak{O}}	\newcommand{\mfP}{\mathfrak{P}}
\newcommand{\mfQ}{\mathfrak{Q}}	\newcommand{\mfR}{\mathfrak{R}}
\newcommand{\mfS}{\mathfrak{S}}	\newcommand{\mfT}{\mathfrak{T}}
\newcommand{\mfU}{\mathfrak{U}}	\newcommand{\mfV}{\mathfrak{V}}
\newcommand{\mfW}{\mathfrak{W}}	\newcommand{\mfX}{\mathfrak{X}}
\newcommand{\mfY}{\mathfrak{Y}}	\newcommand{\mfZ}{\mathfrak{Z}}
%Small Letters
\newcommand{\mfa}{\mathfrak{a}}	\newcommand{\mfb}{\mathfrak{b}}
\newcommand{\mfc}{\mathfrak{c}}	\newcommand{\mfd}{\mathfrak{d}}
\newcommand{\mfe}{\mathfrak{e}}	\newcommand{\mff}{\mathfrak{f}}
\newcommand{\mfg}{\mathfrak{g}}	\newcommand{\mfh}{\mathfrak{h}}
\newcommand{\mfi}{\mathfrak{i}}	\newcommand{\mfj}{\mathfrak{j}}
\newcommand{\mfk}{\mathfrak{k}}	\newcommand{\mfl}{\mathfrak{l}}
\newcommand{\mfm}{\mathfrak{m}}	\newcommand{\mfn}{\mathfrak{n}}
\newcommand{\mfo}{\mathfrak{o}}	\newcommand{\mfp}{\mathfrak{p}}
\newcommand{\mfq}{\mathfrak{q}}	\newcommand{\mfr}{\mathfrak{r}}
\newcommand{\mfs}{\mathfrak{s}}	\newcommand{\mft}{\mathfrak{t}}
\newcommand{\mfu}{\mathfrak{u}}	\newcommand{\mfv}{\mathfrak{v}}
\newcommand{\mfw}{\mathfrak{w}}	\newcommand{\mfx}{\mathfrak{x}}
\newcommand{\mfy}{\mathfrak{y}}	\newcommand{\mfz}{\mathfrak{z}}


\author{\huge{Madiba Hudson-Quansah}}
\date{}
\usepackage{parskip}

\begin{document}

\chapter{Optimization}
\qs{}{
	Find the derivatives of the given functions.
	\begin{enumerate}
		\item \[
			      y = \sin^{-1}(\sqrt{2})
		      \]
		\item \[
			      y = \tan^{-1}(\pi x)
		      \]
		\item \[
			      y = e^{x} \tan^{-1}(x)
		      \]
		\item \[
			      f(x) = \ln(\sin^{-1}(x))
		      \]
		\item \[
			      y = \tan^{-1}(\ln(x)) + \pi
		      \]
		\item \[
			      h(x) = \frac{\sin^{-1}(x)}{1+x}
		      \]
		\item \[
			      y = \sin(x) + x^2 \tan^{-1}(x)
		      \]
		\item \[
			      f(t) = 4 \cos^{-1}(t) - 10 \tan^{-1}(t)
		      \]
		\item \[
			      h(t) = \frac{t}{1 - e^{t}}
		      \]
		\item \[
			      f(x) = \frac{1+5x}{\ln(x)}
		      \]
	\end{enumerate}
}

\sol{
	\begin{enumerate}
		\item
		      \begin{align*}
			      y = \sin^{-1}(\sqrt{2x})                              \\
			      y = \sin^{-1}(2x^{\frac{1}{2}})                       \\
			       & y^{\prime} = \frac{2x^{-\frac{1}{2}}}{\sqrt{1-2x}}
		      \end{align*}
		\item
		      \begin{align*}
			      y = \tan^{-1}(\pi x)                    \\
			       & y^{\prime} = \frac{\pi}{1+\pi^2 x^2}
		      \end{align*}
		\item
		      \begin{align*}
			      y = e^{x} \tan^{-1}(x)                                         \\
			      y^{\prime} = e^{x} (\tan^{-1}(x)) + (\frac{1}{1+x^2}) e^{x}    \\
			       & y^{\prime} = e^{x} (\tan^{-1}(x)) + (\frac{1}{1+x^2}) e^{x}
		      \end{align*}
		\item
		      \begin{align*}
			      f(x) = \ln(\sin^{-1}(x))                              \\
			       & f(x) = \frac{\frac{1}{\sqrt{1+x^2}}}{\sin^{-1}(x)}
		      \end{align*}
		\item
		      \begin{align*}
			      y = \tan^{-1}(\ln(x)) + \pi                      \\
			       & y^{\prime} = \frac{\frac{1}{x}}{1+(\ln(x))^2}
		      \end{align*}
		\item
		      \begin{align*}
			      h(x) = \frac{\sin^{-1}(x)}{1+x}                                                      \\
			       & h^{\prime}(x) = \frac{(1+x)(\frac{1}{\sqrt{1-x^2}}) - (1)(\sin^{-1}(x))}{(1+x)^2}
		      \end{align*}
		\item
		      \begin{align*}
			      y = \sin(x) + x^2 \tan^{-1}(x)                                \\
			       & y^{\prime} = \cos(x) + 2x \tan^{-1}(x) + \frac{x^2}{1+x^2}
		      \end{align*}
		\item
		      \begin{align*}
			      f(t) = 4 \cos^{-1}(t) - 10 \tan^{-1}(t)                          \\
			      f^{\prime}(t) = 4(-\frac{1}{\sqrt{1-t^2}}) - 10(\frac{1}{1+t^2}) \\
			       & f^{\prime}(t) = -\frac{4}{\sqrt{1-t^2}} - \frac{10}{1+t^2}
		      \end{align*}
		\item
		      \begin{align*}
			      h(t) = \frac{t}{1 - e^{t}}                                     \\
			      h^{\prime}(t) = \frac{(1)(1-e^{t}) - (t)(-e^{t})}{(1-e^{t})^2} \\
			       & h^{\prime} = \frac{1-e^{t}-te^{t}}{(1-e^{t})^2}
		      \end{align*}
		\item
		      \begin{align*}
			      f(x) = \frac{1+5x}{\ln(x)}                                           \\
			      f^{\prime}(x) = \frac{(\ln(x))(5) - (\frac{1}{x})(1+5x)}{(\ln(x))^2} \\
			       & f^{\prime}(x) = \frac{5\ln(x) - \frac{1+5x}{x}}{(\ln(x))^2}
		      \end{align*}
	\end{enumerate}
}

\qs{}{
	In 2009, the population of Hungary was approximated by $P = 9.906(0.997)^{t}$ , where
	$P$ is in millions and $t$ is in years since 2009. Assume the trend continues.
	\begin{itemize}
		\item What does this model predict for the population of Hungary in the year 2020?
		\item How fast (in people/year) does this model predict Hungary’s population will
		      be changing in 2020?
	\end{itemize}
}

\sol{
	\begin{itemize}
		\item
		      \begin{align*}
			      P = 9.906(0.997)^{t}              \\
			      t = 2020 - 2009                   \\
			      t = 11                            \\
			      P = 9.906(0.997)^{11}             \\
			       & P = 9.584 \;\; \text{million }
		      \end{align*}
		\item
		      \begin{align*}
			      P^{\prime} = (0.997^{t}  \times  (1)  \times  \ln(0.997))  \times 9.906 \\
			      P^{\prime} = 9.906\ln(0.997)  \times 0.997^{11}                         \\
			       & P^{\prime} = -0.0288 \;\; \text{million }
		      \end{align*}
	\end{itemize}
}

\qs{}{
	Find the equation of the tangent line to the graph of $y = \ln(x) \log_2(x)$ at $x = 2$
}

\sol{
	\begin{align*}
		y = \ln(x) \log_2(x)                                          \\
		y^{\prime} = \frac{1}{x} \log_2(x) + \ln(x) \frac{1}{x\ln(2)} \\[10pt]
		y = \ln(2)\log_2(2)                                           \\[10pt]
		y^{\prime} = \frac{\log_2(x)}{x} + \frac{\ln(x)}{x\ln(2)}     \\
		y^{\prime} = \frac{1}{2} + \frac{1}{2}                        \\
		y^{\prime} = 1                                                \\[10pt]
		\ln(2)\log_2(2) = 2+c                                         \\
		\ln(2)\log_2(2) -2 = c                                        \\
		 & y = x + (\ln(2)\log_2(2) -2)
	\end{align*}
}

\qs{}{
	Find the tangent line to $f(x) = 7^{x} + 4e^{x}$ at $x = 0$
}
\sol {
	\begin{align*}
		f(x) = 7^{x} + 4e^{x}                 \\
		f^{\prime}(x) = 7^{x} \ln(7) + 4e^{x} \\
		f^{\prime}(0) = 7^{0} \ln(7) + 4e^{0} \\
		f^{\prime}(0) = 1 \ln(7) + 4          \\
		f^{\prime}(0) = \ln(7) + 4            \\[10pt]
		y = 5                                 \\[10pt]
		5 = 0(\ln(7) + 4) + c                 \\
		5 = c                                 \\
		 & y = (\ln(7) + 4)x + 5
	\end{align*}
}

\qs{}{
	The cost of producing a quantity, $q$, of a product is given by $C(q) = 1000 + 30e^{0.05q}$ dollars. Find the cost and the marginal cost when $q = 50$. Interpret these answers in economic terms.
}

\sol{
	\begin{align*}
		C(q) = 1000 + 30e^{0.05q}            \\
		C(50) = 1000 + 30e^{0.05(50)}        \\
		C(50) = 1365.47                      \\[10pt]
		C^{\prime}(q) = \frac{3}{2}e^{0.05q} \\
		 & C^{\prime}(50) = \$ 18.27         \\[10pt]
	\end{align*}
	$\therefore$ When the business produces 50 units of a good it's cost incurred is \$1365.47 and the rate at which the cost increases is \$18.27

}

\qs{}{
	At a time $t$ hours after it was administered, the concentration of a drug in the body is $f(t) = 27e^{-0.14t}$ mg/ml.
	\begin{itemize}
		\item What is the concentration 4 hours after it was administered?
		\item At what rate is the concentration changing at that time?
	\end{itemize}
}

\sol {
	\begin{itemize}
		\item
		      \begin{align*}
			      f(t) = 27e^{-0.14t}                \\
			      f(4) = 27e^{-0.14(4)}              \\
			       & f(4) = 15.42 \;\; \frac{mg}{ml}
		      \end{align*}
		\item
		      \begin{align*}
			      f^{\prime}(t) = -3.78e^{-0.14t}            \\
			       & f^{\prime}(4) = -2.16 \;\; \frac{mg}{h}
		      \end{align*}
	\end{itemize}
}

\qs{}{
	For the cost function $C = 1000 + 300\ln(q)$ (in dollars), find the cost and marginal cost at a production level of 500. Interpret your answers in economic terms.
}

\sol{
	\begin{align*}
		C = 1000 + 300\ln(q)       \\
		C = 1000 + 300\ln(500)     \\
		C = 2864.38                \\[10pt]
		C^{\prime} = \frac{300}{q} \\
		 & C^{\prime} = \$0.6
	\end{align*}
	$\therefore$ At a production level of 500 the business is incurring a cost of \$2864.38 and this cost is increasing at a rate of \$0.6
}

\qs{}{
Carbon-14 is a radioactive isotope used to date objects. If $A_0$ represents the initial amount of carbon-14 in the object, then the quantity remaining at time, $t$, in years, is $A(t) = A_0e^{-0.000121t}$ . A tree, originally 185 micrograms of carbon-14, is now 500 years old. At what rate is the carbon-14 decaying now?
}

\sol {
	\begin{align*}
		A(t) = A_0e^{-0.000121t}                           \\
		A^{\prime}(t) = -0.000121A_0e^{-0.000121t}         \\
		A^{\prime}(500) = -0.000121(185)e^{-0.000121(500)} \\
		 & A^{\prime}(500) = -0.0211 \;\; \frac{\mu g}{y}
	\end{align*}
}

\qs{}{
	In 2009, the population of Mexico was 111 million and growing 1.13\% annually, while the population of the US was 307 million and growing 0.975\% annually. If we measure growth rates in people/year, which population was growing faster in 2009?
}

\sol{
	\begin{align*}
		M = 111 + 1.13x    \\
		U = 307 + 0.975x   \\[10pt]
		M^{\prime} = 1.13  \\
		U^{\prime} = 0.975 \\[10pt]
	\end{align*}
	$\therefore$ Mexico's population is growing faster than the US's population
}

\pagebreak
\qs{}{
	With $t$ in years since January 1, 2010, the population $P$ of Slim Chance is predicted by $P = 35000(0.98)^{t}$ . At what rate will the population be changing on January 1, 2023?
}
\sol {
	\begin{align*}
		t = 2023 - 2010                                        \\
		t = 13                                                 \\[10pt]
		P = 35000(0.98)^{13}                                   \\
		P^{\prime} = 35000  \times 0.98^{13}  \times \ln(0.98) \\
		 & P^{\prime} = 543.772 \;\; \frac{people}{year}
	\end{align*}
}

\qs{}{
	Two nonnegative numbers are such that the first plus the square of the second is 10.
	Find the numbers if their sum is as large as possible.
}
\sol{
	\begin{align*}
		a \geq 0                                     \\
		b \geq 0                                     \\[10pt]
		a + b^2 = 10                                 \\
		a = 10 - b^2
		b = \sqrt{10 - a}                            \\[10pt]
		S = a + b                                    \\
		S = 10 - b^2 + \sqrt{10 - a}                 \\
		S = 10 - b^2 + \sqrt{10 - (10 - b^2)}        \\
		S = 10 - b^2 + \sqrt{10-10+b^2}              \\
		S = 10 -b^2 + \sqrt{b^2}                     \\
		S = 10 -b^2 + b                              \\
		S^{\prime} = -2b + 1                         \\
		0 = -2b + 1                                  \\
		2b = 1                                       \\
		b = \frac{1}{2}                              \\
		S^{\prime\prime} = -2
		\therefore b = \frac{1}{2} \;\; \text{ Max } \\[10pt]
		a = 10 - (\frac{1}{2})^2                     \\
		a = \frac{39}{4}                             \\
	\end{align*}
	\[
		\therefore a = \frac{39}{4} \;\; b = \frac{1}{2}
	\]
}

\chapter{Differential Equations}

\qs{}{
	A company earns revenue (income) and also makes payroll payments. Assume that revenue is
	earned continuously, that payroll payments are made continuously, and that the only factors of
	affecting net worth are revenue and payroll. The company’s revenue is earned at a continuous
	annual rate of 5\% times the net worth. At the same time, the company’s payroll obligations are
	paid out at a constant rate of 200 million dollars a year. Use this information to write a
	differential equation to model the net worth of the company, $W$, in millions of dollars, as a
	function of time, $t$, in years.
}
\sol{
	Let $R$ be the company's revenue

	Let $W$ be the company's net worth

	Let $\frac{dR}{dt}$ be the change in revenue

	Let $\frac{dP}{dt}$ be the change in payroll paid
	\begin{align*}
		\frac{dW}{dt} = 0.05W - 200   \\
		dW = (0.05W - 200)dt          \\
		 & \frac{dW}{0.05W -100} = dt
	\end{align*}
}


\qs{}{
	A patient having major surgery is given the antibiotic Vancomycin intravenously at a rate of
	85 $mg$ per hour. The rate at which the drug is excreted from the body is proportional to the
	quantity present, with proportionality constant 0.1 if time is in hours. Write a differential
	equation for the quantity, $Q$ in mg of Vancomycin in the body after $t$ hours.
}
\sol{
	Let $\frac{dQ}{dt}$ be the rate at which the antibiotic is administered

	Let $\frac{dE}{dt}$ be the rate at which the antibiotic is excreted

	\begin{align*}
		\frac{dQ}{dt} = 85 - \frac{dE}{dt} \\
		\frac{dE}{dt} \propto Q            \\
		\frac{dE}{dt} = KQ                 \\
		\frac{dE}{dt} = 0.1Q               \\
		\frac{dQ}{dt} = 85 - 0.1Q          \\
		 & \frac{dQ}{85 - 0.1Q} = dt       \\
	\end{align*}
}


\qs{}{
	Money in a bank account earns interest at a continuous annual rate of 5\% times the current
	balance. Write a differential equation for the balance, $B$, in the account as a function of time, $t$,
	in years.
}
\sol{
	Let $\frac{dB}{dt}$ be the change in balance
	\begin{align*}
		\frac{dB}{dt} = 0.05B    \\
		 & \frac{dB}{B} = 0.05dt
	\end{align*}
}


\qs{}{
	A bank account that initially contains \$25,000 earns interest at a continuous rate of 4\% per
	year. Withdrawals are made out of the account at a constant rate of \$2,000 per year. Write a
	differential equation for the balance, $B$, in the account as a function of the number of years, $t$.
}
\sol{
	Let $B_0$ be the principal

	Let $\frac{dB}{dt}$ be the change in balance

	Let $\frac{dW}{dt}$ be the rate of withdrawal

	\begin{align*}
		B_0 = 25000                     \\
		\frac{dW}{dt} = 2000            \\
		\frac{dB}{dt} = 0.04B - 2000    \\
		dB = (0.04B - 2000)dt           \\
		 & \frac{dB}{0.04B - 2000} = dt
	\end{align*}
}

\pagebreak
\qs{}{
	Morphine is administered to a patient intravenously at a rate of 2.5 $mg$ per hour. About 34.7\%
	of the morphine is metabolized and leaves the body each hour. Write a differential equation for
	the amount of morphine, $M$, in milligrams, in the body as a function of time, $t$, in hours.
}
\sol{
	Let $\frac{dM}{dt}$ be the rate at which the morphine is administered

	Let $\frac{dMet}{dt}$ be the rate at which the morphine is metabolized

	\begin{align*}
		\frac{dM}{dt} = 2.5           \\
		\frac{dMet}{dt} = 0.347M      \\
		\frac{dM}{dt} = 2.5 - 0.347M  \\
		 & \frac{dM}{2.5-0.347M} = dt \\
	\end{align*}
}


\qs{}{
	A cup of coffee contains about 100 $mg$ of caffeine. Caffeine is metabolized and leaves the body
	at a continuous rate of about 17\% every hour. Write a differential equation for the amount, $A$,
	of caffeine in the body as a function of the number of hours, $t$ since the coffee was consumed.
}
\sol{
	Let $A_0$ be the initial amount of caffeine ingested

	Let $\frac{dMet}{dt}$ be the rate the caffeine is metabolized

	Let $\frac{dA}{dt}$ be the rate at which the amount of caffeine in the body is changing

	\begin{align*}
		A_0 = 100                 \\
		\frac{dMet}{dt} = 0.17A   \\
		\frac{dA}{dt} = -0.17A    \\
		\frac{dA}{dt} = -0.17A    \\
		 & \frac{dA}{A} = -0.17dt
	\end{align*}
}


\qs{}{
	A person deposits money into an account at a continuous rate of \$6000 a year, and the account
	earns interest at a continuous rate of 7\% per year. Write a differential equation for the balance
	in the account, $B$, in dollars, as a function of years, $t$.
}
\sol{
	Let $\frac{dB}{dt}$ be the change in balance

	\begin{align*}
		\frac{dB}{dt} = 6000 + 0.07B  \\
		 & \frac{dB}{6000+0.07B} = dt
	\end{align*}
}

\chapter{L'hopital's Rule}
\qs{}{
	\[
		\lim_{x \to 0} (\frac{1}{x} - \frac{1}{\sin(x)})
	\]
}

\sol{
	\begin{align*}
		\lim_{x \to 0} (\frac{1}{x} - \frac{1}{\sin(x)})                                                                        \\
		\frac{1}{x} - \frac{1}{\sin(x)}                                                                                         \\
		\frac{\sin(x) - x}{x\sin(x)}                                                                                            \\
		\lim_{x \to 0} (\frac{\sin(x) -x}{x\sin(x)})                                                                            \\
		\text{\large{Using L'hopital's Rule} }                                                                                  \\
		\lim_{x \to 0} \frac{\frac{dy}{dx}(\sin(x) -x)}{\frac{dy}{dx}(x\sin(x))} \implies \frac{0}{0}                           \\
		\lim_{x \to 0}  \frac{\frac{dy}{dx}(\cos(x)-1)}{\frac{dy}{dx}(x\cos(x) + \sin(x))} \implies \frac{0}{0}                 \\
		\lim_{x \to 0} \frac{-\sin(x)}{-x\sin(x) +\cos(x) + \cos(x)}                                            = \frac{0}{0+2} \\
		\frac{0}{2}                                                                                                             \\
		 & = 0                                                                                                                  \\
	\end{align*}
	\[
		\therefore \lim_{x \to 0} (\frac{1}{x} - \frac{1}{\sin(x)}) = 0
	\]
}

\qs{}{
	\[
		\lim_{x \to \infty} \frac{e^{x}}{x^2}
	\]
}
\sol{
	\begin{align*}
		\lim_{x \to \infty} \frac{e^{x}}{x^2}                                                                \\
		\lim_{x \to \infty} \frac{e^{x}}{x^2} \implies \frac{e^{\infty}}{x^{\infty}} = \frac{\infty}{\infty} \\
		\lim_{x \to \infty} \frac{e^{x}}{2x} \implies \frac{e^{\infty}}{2(\infty)} = \frac{\infty}{\infty}   \\
		\lim_{x \to \infty} \frac{e^{x}}{2} \implies \frac{e^{\infty}}{2} = \frac{\infty}{2}                 \\
	\end{align*}
	\[
		\therefore \lim_{x \to \infty} \frac{e^{x}}{x^2} \text{ does not exist }
	\]
}

\end{document}

