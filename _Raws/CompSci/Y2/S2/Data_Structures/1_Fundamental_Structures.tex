\documentclass[12pt letter]{report}
\input{./template/preamble}
\input{./template/macros}
\input{./template/letterfonts}

\title{\Huge{Fundamental Data Structures}}
\author{\huge{Madiba Hudson-Quansah}}
\date{}
\usepackage{parskip}

\setcounter{tocdepth}{4}
\setcounter{secnumdepth}{4}

\begin{document}
\maketitle
\newpage
\pdfbookmark[section]{\contentsname}{too}
\tableofcontents
\pagebreak

\chapter{Introduction to Data Structures}

\section{Data Structures}


\dfn{Data Type}{
  Describes the set of values that a variable can hold and the types of operations that can be performed on it.
}

\dfn{Data Structure}{
  A way of storing and organizing data in computer memory / The logical mathematical model of a particular organization of
  data.
}

In choosing a data structure, we consider the following in the following order:
\begin{enumerate}
  \item The relationships of the data in the real world
  \item The type and amount of data to be stored
  \item Cost (time complexity) of operations
  \item Memory occupations
  \item Ease of implementation
\end{enumerate}

\subsection{Types of Data Structures}

Data structures can be broadly classified into two types:
\begin{itemize}
  \item Primitive Data Structures
  \item Non-Primitive / Composite Data Structures
\end{itemize}

With data structures under the primitive category being:
\begin{itemize}
  \item Variables
        \begin{itemize}
          \item Integer
          \item Float
          \item Character
          \item Boolean
          \item Enumerated
          \item Reference / Pointer
        \end{itemize}
\end{itemize}

And data structures under the non-primitive category being:
\begin{itemize}
  \item Arrays
  \item Structure / Record
  \item Union
  \item Class
  \item Abstract Data Type
        \begin{itemize}
          \item Linear
                \begin{itemize}
                  \item List
                  \item Stack
                  \item Queue
                \end{itemize}
          \item Non-Linear / Associative
                \begin{itemize}
                  \item Tree
                  \item Graph
                  \item Hash Table
                \end{itemize}
        \end{itemize}
\end{itemize}

Data structures can also more simply be classified in two ways:
\begin{itemize}
  \item Linear / Sequential Data Structures
        \begin{itemize}
          \item Linked List
          \item Array
          \item Stack
          \item Queue
          \item Set
        \end{itemize}
  \item Non-Linear / Associative  Data Structures
        \begin{itemize}
          \item Graph
          \item Tree
          \item Hash Table
        \end{itemize}
\end{itemize}

In this way data structures are defined by an implementation of a particular abstract data type.


\subsubsection{Primitive Data Structures}

These are the basic data structures that are directly operated on by machine / CPU instructions. They are the atomic
data type, i.e. they cannot be divided further

\subsubsection{Non-Primitive Data Structures}

These are data structures composed of or derived from primitive data structures.

\subsubsection{Linear Data Structures}

These data structures are characterized by the fact their elements can be accessed in a sequential / linear manner. This
means every element, excluding the first and last elements which only have a successor and predecessor respectively, has
both a predecessor and successor, allowing bi-directional traversal.

\subsubsection{Non-Linear Data Structures}

These data structures are characterized by the fact that their elements are accessed based on some relationship /
association between elements, therefore non-linearly.

\section{Non-Primitive Data Structures}

\subsection{Arrays}

\dfn{Array}{
  An indexed collection of a fixed number of homogeneous data elements. Elements are stored in contiguous memory, i.e.
  sequentially and can be accessed by their index.
}

\section{Generics}

\dfn{Type Parameter}{
  A placeholder for a type that is used in the definition of a generic type.
}

\dfn{Generic Type}{
  A type that is defined with one or more type parameters, i.e., \lstinline{<T>}. These type parameters are used to define the type of the data that the generic type can store.
}

\dfn{Generics Class and Method}{
  A class ,interface or method that is defined with one or more type parameters.
}

Generics enable classes and functions to be type safe yet reusable for multiple data types. Generic types also resolve
type-casting problems, for example when a class is defined to store integers, it cannot store any other data type. However,
with generics, the class can be defined to store any data type.

\dfn{Type Safety}{
  Prevents the program from compiling if the data type of the variable is not compatible with the data type of the
  object or if an unsupported operation is performed on the object. / Guarantees that the data type of the variable is
  compatible with the data type of the object.
}

\pagebreak
\subsection{Wildcards}

\dfn{Wildcard Character}{
  The wildcard character, \lstinline{<?>} , is used to denote that a generic type can be of any type, with the usual restrictions of
  parametrized types still applying.
}

A parametrized wildcard type on its own is called an unbounded wildcard type and essentially means the same as the
upper bounded wildcard type \lstinline{<? extends Object>}. When unbounded wildcard types in methods act similar to
an equivalent generic method, for example:

\begin{lstlisting}[language=Java]
public void print(ArrayList<?> list) {
  for (Object obj : list) System.out.println(obj);
}

public <T> void print(ArrayList<T> list) {
  for (T t: list) System.out.println(t);
}
\end{lstlisting}

These two methods perform exactly the same and work with any parametrized type.

\begin{lstlisting}[language=Java]
public void printZip(ArrayList<?> first, ArrayList<?> second) {
    for (int i = 0; i < first.size(); i++) {
        System.out.println(first.get(i) + " " + second.get(i));
    }
}

public <T> void printZip(ArrayList<T> first, ArrayList<T> second) {
    for (int i = 0; i < first.size(); i++) {
        System.out.println(first.get(i) + " " + second.get(i));
    }
}
\end{lstlisting}

This listing shows the main difference in the use of normal parametrized types and wildcard types. The wildcard
implementation can accept two difference types for each of the ArrayLists but the same type \lstinline{T} is used for
both lists in the generic method implementation.

\subsubsection{Upper Bounded Wildcards}

\dfn{Upper Bounded Wildcard }{
  A wildcard parametrized type that only accepts types of the bounded type or any subclass of the bounded type, denoted
  by:
  \begin{center}
    \lstinline{<? extends T>}
  \end{center}
}

\begin{lstlisting}[language=Java]
public double sum(ArrayList<? extends Number> numbers) {
    double sum = 0;
    for (Number number : numbers) sum += number.doubleValue();
    return sum;
}
\end{lstlisting}

In this listing the method \lstinline{sum} only accepts an ArrayList of any subclass of the \lstinline{Number} class, i.e.
\lstinline{Integer}, \lstinline{Double}, \lstinline{Float}, etc.

\subsubsection{Lower Bounded Wildcards}

\dfn{Lower Bounded Wildcard}{
  A wildcard parametrized type that only accepts types of the bounded type or any superclass of the bounded type,
  denoted by:
  \begin{center}
    \lstinline{<? super T>}
  \end{center}
}

\begin{lstlisting}[language=Java]
public int sum(ArrayList<? super Integer> numbers) {
    int sum = 0;
    for (int i = 0; i < 10; i++) {
        sum += numbers.get(i);
    }
    return sum;
}
\end{lstlisting}

In this listing the method \lstinline{sum} on accepts an ArrayList of type \lstinline{Integer}, or superclasses,
\lstinline{Number} and \lstinline{Object}.

\chapter{Abstract Data Types}

\dfn{Abstract Data Type}{
  Defined by the operations that can be performed on it, its representation and the values that can be stored in it.
}

\ex{}{
  An Integer ADT is defined below:
  \begin{itemize}
    \item Operations: Addition, Subtraction, Multiplication, Division
    \item Representation: Binary, Decimal, Hexadecimal
    \item Values: -$\infty$ to $\infty$
  \end{itemize}
}

\begin{algorithm}[H]
  \caption{Is-Empty $ \left( A, n \right) $}
  \begin{algorithmic}[1]
    \For{i:= 1 to $n$}
    \If{ $A_i \neq  \text{NULL}$}
    \State \Return false
    \EndIf
    \State \Return true
    \EndFor
  \end{algorithmic}
\end{algorithm}


\chapter{Lists}

\section{Introduction}

A list is a linearly ordered data structure that stores elements sequentially. The locations of elements in this data
structure can be described by their index, which is an integer value that starts at 0. By this definition the first
element in an list is at the 0th index and the last element is at the $n-1$ index, where $n$ is the number of elements
in the list.

\section{Abstract Data Type}

\subsection{Operations}
The List ADT is defined by the following operations:

\begin{description}
  \item[Size] - Returns the number of elements in the list
  \item[IsEmpty] - Returns true if the list is empty, false otherwise
  \item[Get] - Returns the element at the specified index
  \item[Set] - Replaces the element at the specified index with the specified element
  \item[Add] - Adds the specified element to the end of the list or at a specified index
  \item[Remove] - Removes the element at the specified index, moving all subsequent elements one index to the left to
        fill the gap left by the removed element.
\end{description}

\subsection{Representation}

A list can be represented in two ways:
\begin{itemize}
  \item Array Based
  \item Linked List Based
\end{itemize}

With linked lists discussed in chapter \ref{linked-lists}.

\section{Array List}

In representing a list using an array a common notion is to use an array $A$, where $A \left[ i \right] $, stores a
reference to the element with index $i$ in the list. This representation assumes that the array is fixed in size, i.e.
the array has a fixed capacity. This means that the array has a maximum number of elements that it can store, and if
the number of elements in the list exceeds this capacity, a new array is created with a larger capacity and the elements
are copied over to the new array. This process is known as resizing the array, and would be a $O \left( n \right) $ time
and space complexity operation as all the elements in the list would have to be copied over to the new array.

\noindent This means that the elements in the list are stored in contiguous
memory making the \textit{Set} and \textit{Get} operations operate in $O \left( 1 \right) $ time complexity due to the
random access nature of arrays.

\noindent This also means that the \textit{Add} and \textit{Remove} operations operate in $O \left( n \right) $ time
complexity worst case. In the worst case for the \textit{Add} operation a new element is added at index 0. This requires
that all the elements currently in the array would have to be shifted one index to the right to make space for the new
element, i.e. taking $n$ operations. Although in the best case, adding to the end of the list, the \textit{Add}
operation operates in $O \left( 1 \right) $ as the new element is added to the end of the list and no shifting is
required. Similarly the \textit{Remove} operation operates in $O \left( n \right) $ time complexity in the worst case
where the first element is removed, requiring all the elements to be shifted one index to the left to fill the gap left by
the removed element, and also $O \left( 1 \right) $ time complexity in the best case where the last element is removed.

\noindent The algorithms for the operations of the array list are shown below:
\begin{algorithm}[H]
  \caption{Get $ \left( A, n, i \right) $}
  \begin{algorithmic}[1]
    \If{ $i \leq 0$ or $i > n$} \Comment{Check if index is out of bounds}
    \State \Return "Index out of bounds"
    \EndIf
    \State \Return $A_i$
  \end{algorithmic}
\end{algorithm}

\begin{algorithm}[H]
  \caption{Set $ \left( A, n, nE, i \right) $}
  \begin{algorithmic}[1]
    \If{ $i \leq 0$ or $i > n$} \Comment{Check if index is out of bounds}
    \State \Return "Index out of bounds"
    \EndIf

    \State result $ = A_i$
    \State $A_i = nE$
    \State \Return result
  \end{algorithmic}
\end{algorithm}

\begin{algorithm}[H]
  \caption{Add $ \left( A, n, e, i \right) $}
  \begin{algorithmic}[1]
    \For{k $= n$ down to $i$} \Comment{Shift elements to the right of index $i$ to the right}

    $A_{k+1} = A_{k}$ \Comment{Shift elements to the right}
    \EndFor
    \State $A_{i} = e$ \Comment{Add element $e$ at index $i$}
    \State $n = n + 1$
  \end{algorithmic}
\end{algorithm}

\begin{algorithm}[H]
  \caption{Remove $ \left( A, n, i\right) $}
  \begin{algorithmic}[1]
    \If{ $i \leq 0$ or $i > n$} \Comment{Check if index is out of bounds}
    \State \Return "Index out of bounds"
    \EndIf
    \State result $ = A_i$
    \For{ $k = i$ to $n$}

    $A_k = A_{k+1}$
    \EndFor

    $A_{n} = \text{null}$ \\
    $n= n - 1$
    \State \Return result
  \end{algorithmic}
\end{algorithm}



\chapter{Linked Lists} \label{linked-lists}

\chapter{Stacks}

\chapter{Queues}

\chapter{Trees}

\chapter{Graphs}

\section{Introduction}

\dfn{Graph}{
  A non-linear data stricture that consists of a finite set of vertices and a set of edges that connect the vertices. A
  graph $G$ is an ordered pair of a set of vertices $V$ and a set of edges or arcs $E$.

  \noindent Each edge can be an ordered / directed or unordered / undirected pair of vertices. An ordered pair is
  indicated as $ \left( u, v \right) $, and an unordered pair is indicated as $ \{u, v\}$, where $u$ and $v$ are
  vertices.
}

Graph theory is further discussed in the Discrete Math topic in the same computer science topic.

\subsection{Applications of Graphs}

A graph can be used to model any collection with pairwise relationships.
\begin{itemize}
  \item Social Networks
  \item Links between web pages
  \item Road Connections
\end{itemize}

\subsection{Terminology}

\dfn{Path}{
  A sequence of vertices, where each adjacent pair is connected by an edge. In a directed graph the direction must be
  aligned.

  \noindent A simple path is a path in which no vertices and thus edges are repeated. The term walk is usually used to
  denote paths in general while path is used to indicate simple paths. Thus a walk in which no vertices are repeated is
  called a path and one where vertices are repeated is called a \textbf{Trail}
}

\dfn{Strength}{
  A directed graph is \textit{strongly connected} if there is a path from any vertex to any other vertex.

  \noindent If a graph is undirected we simply call it a \textit{connected} graph.

  \noindent If a directed graph is not strongly connected but can be turned into a connected graph by treating all edges
  as undirected it is termed a \textit{weakly connected} graph
}


\section{Abstract Data Type}
\begin{description}
  \item[Number of Vertices] Returns the number of vertices in a graph
  \item[Vertices] Returns an iteration of all the vertices in a graph
  \item[Number of Edges] Returns the number of edges in a graph
  \item[Edges] Returns an iteration of all the edges in a graph
  \item[Get Edge] Returns the edge from a vertex $u$ to a vertex $v$.
  \item[End Vertices] Returns an iteration containing the two endpoint vertices of an edge $e$.
  \item[Opposite] For an edge $e$ incident to a vertex $v$, returns the other vertex of the edge
  \item[Out Degree] Returns the number of outgoing edges from a vertex $v$
  \item[In Degree] Returns the number of incoming edges from a vertex $v$
  \item[Out Going Edges] Returns an iteration containing all the outgoing edges from a vertex $v$
  \item[Incoming Edges] Returns an iteration containing all the incoming edges from a vertex $v$
  \item[Insert Vertex] Inserts a vertex $v$ into the Graph
  \item[Insert Edge] Inserts an edge $e$ between two vertices $u$ and $v$
\end{description}

\section{Representation}


\end{document}
