\documentclass[12pt letter]{report}
\input{./template/preamble}
\input{./template/macros}
\input{./template/letterfonts}

\title{\Huge{Introduction}}
\author{\huge{Madiba Hudson-Quansah}}
\date{}
\usepackage{parskip}

\setcounter{tocdepth}{4}
\setcounter{secnumdepth}{4}

\begin{document}
\maketitle
\newpage
\pdfbookmark[section]{\contentsname}{too}
\tableofcontents
\pagebreak

\chapter{Networks}

\section{Introduction}
\dfn{Local Area Network (LAN)}{
  A network that is confined to a relatively small area. It is generally limited to a geographic area such as a writing lab, school, or building.
}

\dfn{IP (Internet Protocol) Address}{
  A unique string of numbers separated by periods that identifies each computer using the Internet Protocol to communicate over a network.
}

\dfn{Wide Area Network (WAN)}{
  A network that covers a broad area (i.e., any telecommunications network that links across metropolitan, regional, or national boundaries).
}

\dfn{Intranet}{
  A private network that is contained within an enterprise. It may consist of many interlinked local area networks and also use leased lines in the wide area network.
}

\dfn{Router}{
  A device that forwards data packets between computer networks. Routers perform the traffic directing functions on the Internet.
}

\dfn{Switch}{
  A device that connects devices together on a computer network by using packet switching to receive, process, and forward data to the destination device.
}

\dfn{Internet}{
  A global network of networks that consists of millions of private, public, academic, business, and government networks of local to global scope that are linked by a broad array of electronic, wireless, and optical networking technologies.
}

\dfn{Client}{
  A computer that accesses a server for resources.
}

\dfn{Server}{
  A computer that provides data to clients.
}

\subsection{Switch vs Router}

A switch is used to connect various devices within a network. It is used to create a network. A router is used to connect various networks. It is used to route data packets between different networks.

\section{OSI (Open Systems Interconnection) Model}

\dfn{OSI Model}{
  A conceptual framework that standardizes the functions of a telecommunication or computing system into seven
  abstraction layers.
}

This model is composed of seven layers, each specifying particular protocols and functions that are performed at each
layer, with each subsequent layer building on the one below it. The layers are from top to bottom:
\begin{itemize}
  \item Application Layer / Layer 7
  \item Presentation Layer / Layer 6
  \item Session Layer / Layer 5
  \item Transport Layer / Layer 4
  \item Network Layer / Layer 3
  \item Data Link Layer / Layer 2
  \item Physical Layer / Layer 1
\end{itemize}

\subsection{Application Layer}

\dfn{Application Layer}{
  Protocols designed for process-to-process communication and provide services directly to the application. This layer
  interacts with software applications that implement a communicating component. Examples of protocols include HTTP, SMTP, and FTP.
}

\subsection{Presentation Layer}

\dfn{Presentation Layer}{
  The presentation layer ensures that the information that the application layer of one system sends out is readable by
  the application layer of another system. It translates data between the application layer and the network format. This
  is where encoding and encryption is done.
}

\end{document}
