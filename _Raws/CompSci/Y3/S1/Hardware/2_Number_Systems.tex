\documentclass[12pt letter]{report}
\input{./template/preamble}
\input{./template/macros}
\input{./template/letterfonts}

\title{\Huge{Number Systems}}
\author{\huge{Madiba Hudson-Quansah}}
\date{}
\usepackage{parskip}

\setcounter{tocdepth}{4}
\setcounter{secnumdepth}{4}

\begin{document}
\maketitle
\newpage
\pdfbookmark[section]{\contentsname}{too}
\tableofcontents
\pagebreak

\chapter{Introduction}

\section{Number System}
\dfn{Number System}{
  A set of numbers together with one or more operations that are defined on those numbers.
}

The most commonly used number system before the advent of computers was the decimal number system derived from the
Hindu-Arabic number system. The decimal number system is a base-10 number system that uses 10 digits: 0, 1, 2, 3, 4, 5,
6, 7, 8, 9. With the advent of computers, other number systems have been developed to suit the needs of the computer
such as:
\begin{itemize}
  \item Binary number system - base-2, 1s and 0s
  \item Hexadecimal number system - base-16, 0-9 and A-F
  \item Octal number system - base-8, 0-7
\end{itemize}

These number systems are also refereed to as weighted / positional number systems as the value of a digit depends on its
position in the number. I.e. a number $D$

\begin{align*}
  D & =  d_{n-1} w_{n-1} + d_{n-2} w_{n-2} + \ldots + d_1 w_1 + d_0 w_0 \\
    & =  \displaystyle\sum_{i=0}^{n-1} d_i w_i
\end{align*}

Consists of $n$ digits with each at a position $i$. Every digit is a associated with a fixed weight $w_i$ based on its
position. The weight $w$ is derived from the base of the number system.

\subsection{Radix / Base}

\dfn{Radix / Base}{
  The number of digits in a number system including zero. This tells us the number of unique digits in the number
  system and the value of the base. The weight of a number system is derived from the radix, where:
  \[
    w_i = \text{radix}^i
  \]
  Therefore the left most digit, usually called the \textbf{Most Significant Digit}, has a weight of $n-1$ and therefore
  the highest weight and the right most digit, the \textbf{Least Significant Digit}, has a weight of 0.
}

For example in the decimal number system, the radix/base is 10, $\{0,1,2,3,4,5,6,7,8,9\} $, and the weights are $w_i = 10^i$.

Therefore in the decimal number system the number 9203 is represented as:
\begin{align*}
  D & = 9 \times 10^{3} + 2 \times 10^{2} + 0 \times 10^{1} + 3 \times 10^{0} \\
    & = 9000 + 200 + 0 + 3                                                    \\
    & = 9203
\end{align*}


When a number is fractional, the digits to the left of the radix point have positive position indices and the digits to
the right of the radix point have negative position indices. Where a number $D$ has $n$ integral digits and $m$
fractional digits, the value of the number is given by:
\begin{align*}
  D & = d_{n-1} w_{n-1} + d_{n-2} w_{n-2} + \ldots + d_1 w_1 + d_0 w_0\, \cdot \, d_{-1} w_{-1} + d_{-2} w_{-2} + \ldots
  d_{-m} w_{-m}                                                                                                          \\
    & = \displaystyle\sum_{i=-m}^{n-1} d_i w_i                                                                           \\
\end{align*}

Therefore a fractional number in the decimal number system such as $57.6528$ is represented as:
\begin{align*}
  D & =  5 \times 10^{1} + 7 \times 10^{0} + 6 \times 10^{-1} + 5 \times 10^{-2} + 2 \times 10^{-3} + 8 \times 10^{-4} \\
    & =  50 + 7 + 0.6 + 0.05 + 0.002 + 0.0008                                                                          \\
    & =  57.6528
\end{align*}

A number $D$ with base $r$ can be denoted as $D_r$ or $ \left( D \right)_r $, i.e. $57.6528_{10}$ or $(57.6528)_{10}$.

\chapter{Common Number Systems}

\section{Binary Number System}

\dfn{Binary Number System}{
  A base-2 number system that uses two digits: 0 and 1. The weights of the binary number system are derived from the
  radix, where:
  \[
    w_i = 2^i
  \]
  A \textbf{bi}nary digi\textbf{t} is called a \textbf{bit}. The left most bit is the \textbf{Most Significant Bit (MSB)} and
  the right most bit is the \textbf{Least Significant Bit (LSB)}.
}

A decimal equivalent of a binary number can be found using the same method as the decimal number system. i.e.:
\[
  D = \displaystyle\sum_{i=0}^{n-1} d_i w_i
\]
Where $w_i = 2^{i}$. For example the binary number $1011_2$ is represented as:
\begin{align*}
  D & = 1 \times 2^{3} + 0 \times 2^{2} + 1 \times 2^{1} + 1 \times 2^{0} \\
    & = 8 + 0 + 2 + 1                                                     \\
    & =  11_{10}
\end{align*}

And for a fractional binary number such as $110.11_2$:
\begin{align*}
  D & = 1 \times 2^{2} + 1 \times 2^{1} + 0 \times 2^{0} + 1\times 2^{-1} + 1 \times 2^{-2} \\
    & = 4 + 2 + 0 + 0.5 + 0.25                                                              \\
    & = 6.75_{10}                                                                           \\
\end{align*}

\section{Octal Number System}

\dfn{Octal Number System}{
  A base-8 number system that uses eight digits: 0, 1, 2, 3, 4, 5, 6, 7. The weights of the octal number system are
  derived from the radix, where:
  \[
    w_i = 8^i
  \]
}

Each octal digit can represent 3 binary digits. Therefore an octal number can be converted to a binary number by converting
each octal digit to a 3-bit binary number. For example the octal number $57_8$ is represented as:
\begin{align*}
  57 & = 101\, 111 \\
     & = 101111_2
\end{align*}

And to decimal:
\begin{align*}
  D & = 5 \times 8^{1} + 7 \times 8^{0} \\
    & = 40 + 7                          \\
    & = 47                              \\
\end{align*}

\section{Hexadecimal Number System}

\dfn{Hexadecimal Number System}{
  A base-16 number system that uses sixteen digits: 0, 1, 2, 3, 4, 5, 6, 7, 8, 9, A, B, C, D, E, F. The weights of the
  hexadecimal number system are derived from the radix, where:
  \[
    w_i = 16^i
  \]
}

All the letters are used to represent the numbers 10 to 15.
Each hexadecimal digit can represent 4 binary digits. Therefore a hexadecimal number can be converted to a binary number
by converting each hexadecimal digit to a 4-bit binary number. For example the hexadecimal number $A3F_{16}$ is
represented as:
\begin{align*}
  A3F & = 1010\,0011\,1110 \\
      & = 101000111110     \\
\end{align*}

And in decimal:
\begin{align*}
  D & = 10 \times 16^{2} + 3 \times 16^{1} + 15 \times 16^{0} \\
    & = 10 \times 256 + 3 \times 16 + 15                      \\
    & = 2623_{10}                                             \\
\end{align*}

And $9E1_{16}$ in decimal:
\begin{align*}
  D & = 9 \times 16^{2} + 14 \times 16^{1} + 1 \times 16^{0} \\
    & = 2529_{10}                                            \\
\end{align*}

\section{Common Notions}

\begin{itemize}
  \item The number of possible digits in a number system is the base of the number system.
  \item The smallest digit in a number system is 0 and the largest digit is $r-1$ where $r$ is the base of the number system.
  \item Adding 1 to the largest digit in a number system always results in a carry, i.e. a result of $ \left( 10 \right) $
  \item For any number system with base $r$ adding 1 to the largest $n$ digit number results in a carry to the $n-1$
        digit, i.e. $ \left( r^{n} \right) $
  \item  The largest $n$-digit number in a number system with base $r$ is $r^{n} - 1$
  \item The largest value that can be expressed in $m$ fractional digits is $1 - r^{-m}$
  \item The largest value that can be expressed in $n$ integral digits and $m$ fractional digits is $r^{n} - r^{-m}$
  \item The total number of unique integers that can be expressed in a number system with base $r$ is $r^{n}$ and
        the total number of both integers and fractions is $r^{n} + r^{-m}$
\end{itemize}

\chapter{Arithmetic Operations}

\section{Binary Arithmetic}

\subsection{Addition}

\dfn{Addition}{
  A binary operation, as in requiring two operands, that sums two numbers. The addition operation comprises of two
  operands the augend and the addend.
}

\begin{align*}
  \begin{split}
    0\,0\,1\,1 \\
    0\,1\,0\,1 \\
    \hline
    0\,1\,1\,0
  \end{split}
\end{align*}

\chapter{Conversions}

\section{Decimal}

\subsection{Decimal to Binary}
\begin{itemize}
  \item Divide the decimal number by 2 and note the remainder.
  \item Repeat the process until the quotient is 0.
  \item The sequence of remainders in reverse order is the binary equivalent of the decimal number.
\end{itemize}

\ex{}{
  \qs{}{
    Convert the decimal number 57 to binary.
  }

  \sol{
    \begin{align*}
      57 \div 2 & = 28 \text{ remainder } 1 \\
      28 \div 2 & = 14 \text{ remainder } 0 \\
      14 \div 2 & = 7 \text{ remainder } 0  \\
      7 \div 2  & = 3 \text{ remainder } 1  \\
      3 \div 2  & = 1 \text{ remainder } 1  \\
      1 \div 2  & = 0 \text{ remainder } 1  \\
    \end{align*}
    $\therefore$ 57 in binary is $111001$
  }
}

\subsection{Decimal to Octal}

\begin{itemize}
  \item Divide the decimal number by 8 and note the remainder.
  \item Repeat the process until the quotient is 0.
  \item The sequence of remainders in reverse order is the octal equivalent of the decimal number.
\end{itemize}

\ex{}{
  \qs{}{
    Convert the decimal number 57 to octal.
  }

  \sol{
    \begin{align*}
      57 \div 8 & = 7 \text{ remainder } 1 \\
      7 \div 8  & = 0 \text{ remainder } 7 \\
    \end{align*}
    $\therefore$ 57 in octal is $71$
  }
}

\subsection{Decimal to Hexadecimal}

\begin{itemize}
  \item Divide the decimal number by 16 and note the remainder.
  \item Repeat the process until the quotient is 0.
  \item The sequence of remainders in reverse order is the octal equivalent of the decimal number.
\end{itemize}

\ex{}{
  \qs{}{
    Convert the decimal number 57 to hexadecimal.
  }

  \sol{
    \begin{align*}
      57 \div 16 & = 3 \text{ remainder } 9 \\
      57 \div 16 & = 0 \text{ remainder } 3 \\
    \end{align*}
    $\therefore$ 57 in hexadecimal is $39$
  }
}

\subsection{Decimal Fraction}

\begin{itemize}
  \item Multiply the decimal fraction by the base of the number system.
  \item  Repeat the process until the fractional part is 0 or the desired precision is reached.
  \item The sequence of remainders in forward order is the binary equivalent of the decimal number.
\end{itemize}

\end{document}
