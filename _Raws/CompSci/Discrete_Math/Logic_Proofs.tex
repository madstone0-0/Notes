\documentclass[12pt letter]{report}
%%%%%%%%%%%%%%%%%%%%%%%%%%%%%%%%%
% PACKAGE IMPORTS
%%%%%%%%%%%%%%%%%%%%%%%%%%%%%%%%%


\usepackage[tmargin=2cm,rmargin=1in,lmargin=1in,margin=0.85in,bmargin=2cm,footskip=.2in]{geometry}
\usepackage{amsmath,amsfonts,amsthm,amssymb,mathtools}
\usepackage[varbb]{newpxmath}
\usepackage{xfrac}
\usepackage[makeroom]{cancel}
\usepackage{mathtools}
\usepackage{bookmark}
\usepackage{enumitem}
\usepackage{hyperref,theoremref}
\hypersetup{
  pdftitle={Assignment},
  colorlinks=true, linkcolor=doc!90,
  bookmarksnumbered=true,
  bookmarksopen=true
}
\usepackage[most,many,breakable]{tcolorbox}
\usepackage{xcolor}
\usepackage{varwidth}
\usepackage{varwidth}
\usepackage{etoolbox}
%\usepackage{authblk}
\usepackage{nameref}
\usepackage{multicol,array}
\usepackage{tikz-cd}
\usepackage[ruled,vlined,linesnumbered]{algorithm2e}
\usepackage{comment} % enables the use of multi-line comments (\ifx \fi) 
\usepackage{import}
\usepackage{xifthen}
\usepackage{pdfpages}
\usepackage{transparent}
\usepackage{xcolor,colortbl,array,amssymb}
\usepackage{venndiagram}
\usepackage{listings}
\usepackage{fontspec}

\setmainfont{LibertinusSerif}[
  Extension = .otf,
  Path = /usr/share/fonts/libertinus/,
  UprightFont = *-Regular,
  ItalicFont = *-Italic,
  BoldFont = *-Bold,
  BoldItalicFont = *-BoldItalic,
]

\setmonofont{CaskaydiaCoveNerdFontMono}[
  Extension = .ttf,
  Path = /usr/share/fonts/TTF/,
  UprightFont = *-Regular,
  ItalicFont = *-Italic,
  BoldFont = *-Bold,
  BoldItalicFont = *-BoldItalic,
]

\definecolor{listing-background}{HTML}{F7F7F7}
\definecolor{listing-rule}{HTML}{B3B2B3}
\definecolor{listing-numbers}{HTML}{B3B2B3}
\definecolor{listing-text-color}{HTML}{000000}
\definecolor{listing-keyword}{HTML}{435489}
\definecolor{listing-keyword-2}{HTML}{1284CA} % additional keywords
\definecolor{listing-keyword-3}{HTML}{9137CB} % additional keywords
\definecolor{listing-identifier}{HTML}{435489}
\definecolor{listing-string}{HTML}{00999A}
\definecolor{listing-comment}{HTML}{8E8E8E}

\lstdefinestyle{eisvogel_listing_style}{
language         = java,
xleftmargin      = 0.6em,
framexleftmargin = 0.4em,
backgroundcolor  = \color{listing-background},
basicstyle       = \color{listing-text-color}\linespread{1.0}%
\lst@ifdisplaystyle%
\fi\ttfamily{},
breaklines       = true,
frame            = single,
framesep         = 0.19em,
rulecolor        = \color{listing-rule},
frameround       = ffff,
tabsize          = 4,
numberstyle      = \color{listing-numbers},
aboveskip        = 1.0em,
belowskip        = 0.1em,
abovecaptionskip = 0em,
belowcaptionskip = 1.0em,
keywordstyle     = {\color{listing-keyword}\bfseries},
keywordstyle     = {[2]\color{listing-keyword-2}\bfseries},
keywordstyle     = {[3]\color{listing-keyword-3}\bfseries\itshape},
sensitive        = true,
identifierstyle  = \color{listing-identifier},
commentstyle     = \color{listing-comment},
stringstyle      = \color{listing-string},
showstringspaces = false,
escapeinside     = {/*@}{@*/}, % Allow LaTeX inside these special comments
literate         =
  {á}{{\'a}}1 {é}{{\'e}}1 {í}{{\'i}}1 {ó}{{\'o}}1 {ú}{{\'u}}1
{Á}{{\'A}}1 {É}{{\'E}}1 {Í}{{\'I}}1 {Ó}{{\'O}}1 {Ú}{{\'U}}1
{à}{{\`a}}1 {è}{{\`e}}1 {ì}{{\`i}}1 {ò}{{\`o}}1 {ù}{{\`u}}1
{À}{{\`A}}1 {È}{{\`E}}1 {Ì}{{\`I}}1 {Ò}{{\`O}}1 {Ù}{{\`U}}1
{ä}{{\"a}}1 {ë}{{\"e}}1 {ï}{{\"i}}1 {ö}{{\"o}}1 {ü}{{\"u}}1
{Ä}{{\"A}}1 {Ë}{{\"E}}1 {Ï}{{\"I}}1 {Ö}{{\"O}}1 {Ü}{{\"U}}1
{â}{{\^a}}1 {ê}{{\^e}}1 {î}{{\^i}}1 {ô}{{\^o}}1 {û}{{\^u}}1
{Â}{{\^A}}1 {Ê}{{\^E}}1 {Î}{{\^I}}1 {Ô}{{\^O}}1 {Û}{{\^U}}1
{œ}{{\oe}}1 {Œ}{{\OE}}1 {æ}{{\ae}}1 {Æ}{{\AE}}1 {ß}{{\ss}}1
{ç}{{\c c}}1 {Ç}{{\c C}}1 {ø}{{\o}}1 {å}{{\r a}}1 {Å}{{\r A}}1
{€}{{\EUR}}1 {£}{{\pounds}}1 {«}{{\guillemotleft}}1
{»}{{\guillemotright}}1 {ñ}{{\~n}}1 {Ñ}{{\~N}}1 {¿}{{?`}}1
{…}{{\ldots}}1 {≥}{{>=}}1 {≤}{{<=}}1 {„}{{\glqq}}1 {“}{{\grqq}}1
{”}{{''}}1
}
\lstset{style=eisvogel_listing_style,
  numbers=left}

%
% Java (Java SE 12, 2019-06-22)
%
\lstdefinelanguage{Java}{
  morekeywords={
      % normal keywords (without data types)
      abstract,assert,break,case,catch,class,continue,default,
      do,else,enum,exports,extends,final,finally,for,if,implements,
      import,instanceof,interface,module,native,new,package,private,
      protected,public,requires,return,static,strictfp,super,switch,
      synchronized,this,throw,throws,transient,try,volatile,while,
      % var is an identifier
      var
    },
  morekeywords={[2] % data types
      % primitive data types
      boolean,byte,char,double,float,int,long,short,
      % String
      String,
      % primitive wrapper types
      Boolean,Byte,Character,Double,Float,Integer,Long,Short
      % number types
      Number,AtomicInteger,AtomicLong,BigDecimal,BigInteger,DoubleAccumulator,DoubleAdder,LongAccumulator,LongAdder,Short,
      % other
      Object,Void,void
    },
  morekeywords={[3] % literals
      % reserved words for literal values
      null,true,false,
    },
  sensitive,
  morecomment  = [l]//,
  morecomment  = [s]{/*}{*/},
  morecomment  = [s]{/**}{*/},
  morestring   = [b]",
  morestring   = [b]',
}

\lstdefinelanguage{XML}{
  morestring      = [b]",
  moredelim       = [s][\bfseries\color{listing-keyword}]{<}{\ },
  moredelim       = [s][\bfseries\color{listing-keyword}]{</}{>},
  moredelim       = [l][\bfseries\color{listing-keyword}]{/>},
  moredelim       = [l][\bfseries\color{listing-keyword}]{>},
  morecomment     = [s]{<?}{?>},
  morecomment     = [s]{<!--}{-->},
  commentstyle    = \color{listing-comment},
  stringstyle     = \color{listing-string},
  identifierstyle = \color{listing-identifier}
}


\newcommand\mycommfont[1]{\footnotesize\ttfamily\textcolor{blue}{#1}}
\SetCommentSty{mycommfont}
\newcommand{\incfig}[1]{%
  \def\svgwidth{\columnwidth}
  \import{./figures/}{#1.pdf_tex}
}

\usepackage{tikzsymbols}
\renewcommand\qedsymbol{$\Laughey$}


%\usepackage{import}
%\usepackage{xifthen}
%\usepackage{pdfpages}
%\usepackage{transparent}


%%%%%%%%%%%%%%%%%%%%%%%%%%%%%%
% SELF MADE COLORS
%%%%%%%%%%%%%%%%%%%%%%%%%%%%%%



\definecolor{myg}{RGB}{56, 140, 70}
\definecolor{myb}{RGB}{45, 111, 177}
\definecolor{myr}{RGB}{199, 68, 64}
\definecolor{mytheorembg}{HTML}{F2F2F9}
\definecolor{mytheoremfr}{HTML}{00007B}
\definecolor{mylenmabg}{HTML}{FFFAF8}
\definecolor{mylenmafr}{HTML}{983b0f}
\definecolor{mypropbg}{HTML}{f2fbfc}
\definecolor{mypropfr}{HTML}{191971}
\definecolor{myexamplebg}{HTML}{F2FBF8}
\definecolor{myexamplefr}{HTML}{88D6D1}
\definecolor{myexampleti}{HTML}{2A7F7F}
\definecolor{mydefinitbg}{HTML}{E5E5FF}
\definecolor{mydefinitfr}{HTML}{3F3FA3}
\definecolor{notesgreen}{RGB}{0,162,0}
\definecolor{myp}{RGB}{197, 92, 212}
\definecolor{mygr}{HTML}{2C3338}
\definecolor{myred}{RGB}{127,0,0}
\definecolor{myyellow}{RGB}{169,121,69}
\definecolor{myexercisebg}{HTML}{F2FBF8}
\definecolor{myexercisefg}{HTML}{88D6D1}


%%%%%%%%%%%%%%%%%%%%%%%%%%%%
% TCOLORBOX SETUPS
%%%%%%%%%%%%%%%%%%%%%%%%%%%%

\setlength{\parindent}{1cm}
%================================
% THEOREM BOX
%================================

\tcbuselibrary{theorems,skins,hooks}
\newtcbtheorem[number within=section]{Theorem}{Theorem}
{%
  enhanced,
  breakable,
  colback = mytheorembg,
  frame hidden,
  boxrule = 0sp,
  borderline west = {2pt}{0pt}{mytheoremfr},
  sharp corners,
  detach title,
  before upper = \tcbtitle\par\smallskip,
  coltitle = mytheoremfr,
  fonttitle = \bfseries\sffamily,
  description font = \mdseries,
  separator sign none,
  segmentation style={solid, mytheoremfr},
}
{th}

\tcbuselibrary{theorems,skins,hooks}
\newtcbtheorem[number within=chapter]{theorem}{Theorem}
{%
  enhanced,
  breakable,
  colback = mytheorembg,
  frame hidden,
  boxrule = 0sp,
  borderline west = {2pt}{0pt}{mytheoremfr},
  sharp corners,
  detach title,
  before upper = \tcbtitle\par\smallskip,
  coltitle = mytheoremfr,
  fonttitle = \bfseries\sffamily,
  description font = \mdseries,
  separator sign none,
  segmentation style={solid, mytheoremfr},
}
{th}


\tcbuselibrary{theorems,skins,hooks}
\newtcolorbox{Theoremcon}
{%
  enhanced
  ,breakable
  ,colback = mytheorembg
  ,frame hidden
  ,boxrule = 0sp
  ,borderline west = {2pt}{0pt}{mytheoremfr}
  ,sharp corners
  ,description font = \mdseries
  ,separator sign none
}

%================================
% Corollery
%================================
\tcbuselibrary{theorems,skins,hooks}
\newtcbtheorem[number within=section]{Corollary}{Corollary}
{%
  enhanced
  ,breakable
  ,colback = myp!10
  ,frame hidden
  ,boxrule = 0sp
  ,borderline west = {2pt}{0pt}{myp!85!black}
  ,sharp corners
  ,detach title
  ,before upper = \tcbtitle\par\smallskip
  ,coltitle = myp!85!black
  ,fonttitle = \bfseries\sffamily
  ,description font = \mdseries
  ,separator sign none
  ,segmentation style={solid, myp!85!black}
}
{th}
\tcbuselibrary{theorems,skins,hooks}
\newtcbtheorem[number within=chapter]{corollary}{Corollary}
{%
  enhanced
  ,breakable
  ,colback = myp!10
  ,frame hidden
  ,boxrule = 0sp
  ,borderline west = {2pt}{0pt}{myp!85!black}
  ,sharp corners
  ,detach title
  ,before upper = \tcbtitle\par\smallskip
  ,coltitle = myp!85!black
  ,fonttitle = \bfseries\sffamily
  ,description font = \mdseries
  ,separator sign none
  ,segmentation style={solid, myp!85!black}
}
{th}


%================================
% LENMA
%================================

\tcbuselibrary{theorems,skins,hooks}
\newtcbtheorem[number within=section]{Lenma}{Lenma}
{%
  enhanced,
  breakable,
  colback = mylenmabg,
  frame hidden,
  boxrule = 0sp,
  borderline west = {2pt}{0pt}{mylenmafr},
  sharp corners,
  detach title,
  before upper = \tcbtitle\par\smallskip,
  coltitle = mylenmafr,
  fonttitle = \bfseries\sffamily,
  description font = \mdseries,
  separator sign none,
  segmentation style={solid, mylenmafr},
}
{th}

\tcbuselibrary{theorems,skins,hooks}
\newtcbtheorem[number within=chapter]{lenma}{Lenma}
{%
  enhanced,
  breakable,
  colback = mylenmabg,
  frame hidden,
  boxrule = 0sp,
  borderline west = {2pt}{0pt}{mylenmafr},
  sharp corners,
  detach title,
  before upper = \tcbtitle\par\smallskip,
  coltitle = mylenmafr,
  fonttitle = \bfseries\sffamily,
  description font = \mdseries,
  separator sign none,
  segmentation style={solid, mylenmafr},
}
{th}


%================================
% PROPOSITION
%================================

\tcbuselibrary{theorems,skins,hooks}
\newtcbtheorem[number within=section]{Prop}{Proposition}
{%
  enhanced,
  breakable,
  colback = mypropbg,
  frame hidden,
  boxrule = 0sp,
  borderline west = {2pt}{0pt}{mypropfr},
  sharp corners,
  detach title,
  before upper = \tcbtitle\par\smallskip,
  coltitle = mypropfr,
  fonttitle = \bfseries\sffamily,
  description font = \mdseries,
  separator sign none,
  segmentation style={solid, mypropfr},
}
{th}

\tcbuselibrary{theorems,skins,hooks}
\newtcbtheorem[number within=chapter]{prop}{Proposition}
{%
  enhanced,
  breakable,
  colback = mypropbg,
  frame hidden,
  boxrule = 0sp,
  borderline west = {2pt}{0pt}{mypropfr},
  sharp corners,
  detach title,
  before upper = \tcbtitle\par\smallskip,
  coltitle = mypropfr,
  fonttitle = \bfseries\sffamily,
  description font = \mdseries,
  separator sign none,
  segmentation style={solid, mypropfr},
}
{th}


%================================
% CLAIM
%================================

\tcbuselibrary{theorems,skins,hooks}
\newtcbtheorem[number within=section]{claim}{Claim}
{%
  enhanced
  ,breakable
  ,colback = myg!10
  ,frame hidden
  ,boxrule = 0sp
  ,borderline west = {2pt}{0pt}{myg}
  ,sharp corners
  ,detach title
  ,before upper = \tcbtitle\par\smallskip
  ,coltitle = myg!85!black
  ,fonttitle = \bfseries\sffamily
  ,description font = \mdseries
  ,separator sign none
  ,segmentation style={solid, myg!85!black}
}
{th}



%================================
% Exercise
%================================

\tcbuselibrary{theorems,skins,hooks}
\newtcbtheorem[number within=section]{Exercise}{Exercise}
{%
  enhanced,
  breakable,
  colback = myexercisebg,
  frame hidden,
  boxrule = 0sp,
  borderline west = {2pt}{0pt}{myexercisefg},
  sharp corners,
  detach title,
  before upper = \tcbtitle\par\smallskip,
  coltitle = myexercisefg,
  fonttitle = \bfseries\sffamily,
  description font = \mdseries,
  separator sign none,
  segmentation style={solid, myexercisefg},
}
{th}

\tcbuselibrary{theorems,skins,hooks}
\newtcbtheorem[number within=chapter]{exercise}{Exercise}
{%
  enhanced,
  breakable,
  colback = myexercisebg,
  frame hidden,
  boxrule = 0sp,
  borderline west = {2pt}{0pt}{myexercisefg},
  sharp corners,
  detach title,
  before upper = \tcbtitle\par\smallskip,
  coltitle = myexercisefg,
  fonttitle = \bfseries\sffamily,
  description font = \mdseries,
  separator sign none,
  segmentation style={solid, myexercisefg},
}
{th}

%================================
% EXAMPLE BOX
%================================

\newtcbtheorem[number within=section]{Example}{Example}
{%
  colback = myexamplebg
  ,breakable
  ,colframe = myexamplefr
  ,coltitle = myexampleti
  ,boxrule = 1pt
  ,sharp corners
  ,detach title
  ,before upper=\tcbtitle\par\smallskip
  ,fonttitle = \bfseries
  ,description font = \mdseries
  ,separator sign none
  ,description delimiters parenthesis
}
{ex}

\newtcbtheorem[number within=chapter]{example}{Example}
{%
  colback = myexamplebg
  ,breakable
  ,colframe = myexamplefr
  ,coltitle = myexampleti
  ,boxrule = 1pt
  ,sharp corners
  ,detach title
  ,before upper=\tcbtitle\par\smallskip
  ,fonttitle = \bfseries
  ,description font = \mdseries
  ,separator sign none
  ,description delimiters parenthesis
}
{ex}

%================================
% DEFINITION BOX
%================================

\newtcbtheorem[number within=section]{Definition}{Definition}{enhanced,
  before skip=2mm,after skip=2mm, colback=red!5,colframe=red!80!black,boxrule=0.5mm,
  attach boxed title to top left={xshift=1cm,yshift*=1mm-\tcboxedtitleheight}, varwidth boxed title*=-3cm,
  boxed title style={frame code={
          \path[fill=tcbcolback]
          ([yshift=-1mm,xshift=-1mm]frame.north west)
          arc[start angle=0,end angle=180,radius=1mm]
          ([yshift=-1mm,xshift=1mm]frame.north east)
          arc[start angle=180,end angle=0,radius=1mm];
          \path[left color=tcbcolback!60!black,right color=tcbcolback!60!black,
            middle color=tcbcolback!80!black]
          ([xshift=-2mm]frame.north west) -- ([xshift=2mm]frame.north east)
          [rounded corners=1mm]-- ([xshift=1mm,yshift=-1mm]frame.north east)
          -- (frame.south east) -- (frame.south west)
          -- ([xshift=-1mm,yshift=-1mm]frame.north west)
          [sharp corners]-- cycle;
        },interior engine=empty,
    },
  fonttitle=\bfseries,
  title={#2},#1}{def}
\newtcbtheorem[number within=chapter]{definition}{Definition}{enhanced,
  before skip=2mm,after skip=2mm, colback=red!5,colframe=red!80!black,boxrule=0.5mm,
  attach boxed title to top left={xshift=1cm,yshift*=1mm-\tcboxedtitleheight}, varwidth boxed title*=-3cm,
  boxed title style={frame code={
          \path[fill=tcbcolback]
          ([yshift=-1mm,xshift=-1mm]frame.north west)
          arc[start angle=0,end angle=180,radius=1mm]
          ([yshift=-1mm,xshift=1mm]frame.north east)
          arc[start angle=180,end angle=0,radius=1mm];
          \path[left color=tcbcolback!60!black,right color=tcbcolback!60!black,
            middle color=tcbcolback!80!black]
          ([xshift=-2mm]frame.north west) -- ([xshift=2mm]frame.north east)
          [rounded corners=1mm]-- ([xshift=1mm,yshift=-1mm]frame.north east)
          -- (frame.south east) -- (frame.south west)
          -- ([xshift=-1mm,yshift=-1mm]frame.north west)
          [sharp corners]-- cycle;
        },interior engine=empty,
    },
  fonttitle=\bfseries,
  title={#2},#1}{def}



%================================
% Solution BOX
%================================

\makeatletter
\newtcbtheorem{question}{Question}{enhanced,
  breakable,
  colback=white,
  colframe=myb!80!black,
  attach boxed title to top left={yshift*=-\tcboxedtitleheight},
  fonttitle=\bfseries,
  title={#2},
  boxed title size=title,
  boxed title style={%
      sharp corners,
      rounded corners=northwest,
      colback=tcbcolframe,
      boxrule=0pt,
    },
  underlay boxed title={%
      \path[fill=tcbcolframe] (title.south west)--(title.south east)
      to[out=0, in=180] ([xshift=5mm]title.east)--
      (title.center-|frame.east)
      [rounded corners=\kvtcb@arc] |-
      (frame.north) -| cycle;
    },
  #1
}{def}
\makeatother

%================================
% SOLUTION BOX
%================================

\makeatletter
\newtcolorbox{solution}{enhanced,
  breakable,
  colback=white,
  colframe=myg!80!black,
  attach boxed title to top left={yshift*=-\tcboxedtitleheight},
  title=Solution,
  boxed title size=title,
  boxed title style={%
      sharp corners,
      rounded corners=northwest,
      colback=tcbcolframe,
      boxrule=0pt,
    },
  underlay boxed title={%
      \path[fill=tcbcolframe] (title.south west)--(title.south east)
      to[out=0, in=180] ([xshift=5mm]title.east)--
      (title.center-|frame.east)
      [rounded corners=\kvtcb@arc] |-
      (frame.north) -| cycle;
    },
}
\makeatother

%================================
% Question BOX
%================================

\makeatletter
\newtcbtheorem{qstion}{Question}{enhanced,
  breakable,
  colback=white,
  colframe=mygr,
  attach boxed title to top left={yshift*=-\tcboxedtitleheight},
  fonttitle=\bfseries,
  title={#2},
  boxed title size=title,
  boxed title style={%
      sharp corners,
      rounded corners=northwest,
      colback=tcbcolframe,
      boxrule=0pt,
    },
  underlay boxed title={%
      \path[fill=tcbcolframe] (title.south west)--(title.south east)
      to[out=0, in=180] ([xshift=5mm]title.east)--
      (title.center-|frame.east)
      [rounded corners=\kvtcb@arc] |-
      (frame.north) -| cycle;
    },
  #1
}{def}
\makeatother

\newtcbtheorem[number within=chapter]{wconc}{Wrong Concept}{
  breakable,
  enhanced,
  colback=white,
  colframe=myr,
  arc=0pt,
  outer arc=0pt,
  fonttitle=\bfseries\sffamily\large,
  colbacktitle=myr,
  attach boxed title to top left={},
  boxed title style={
      enhanced,
      skin=enhancedfirst jigsaw,
      arc=3pt,
      bottom=0pt,
      interior style={fill=myr}
    },
  #1
}{def}



%================================
% NOTE BOX
%================================

\usetikzlibrary{arrows,calc,shadows.blur}
\tcbuselibrary{skins}
\newtcolorbox{note}[1][]{%
  enhanced jigsaw,
  colback=gray!20!white,%
  colframe=gray!80!black,
  size=small,
  boxrule=1pt,
  title=\textbf{Note:-},
  halign title=flush center,
  coltitle=black,
  breakable,
  drop shadow=black!50!white,
  attach boxed title to top left={xshift=1cm,yshift=-\tcboxedtitleheight/2,yshifttext=-\tcboxedtitleheight/2},
  minipage boxed title=1.5cm,
  boxed title style={%
      colback=white,
      size=fbox,
      boxrule=1pt,
      boxsep=2pt,
      underlay={%
          \coordinate (dotA) at ($(interior.west) + (-0.5pt,0)$);
          \coordinate (dotB) at ($(interior.east) + (0.5pt,0)$);
          \begin{scope}
            \clip (interior.north west) rectangle ([xshift=3ex]interior.east);
            \filldraw [white, blur shadow={shadow opacity=60, shadow yshift=-.75ex}, rounded corners=2pt] (interior.north west) rectangle (interior.south east);
          \end{scope}
          \begin{scope}[gray!80!black]
            \fill (dotA) circle (2pt);
            \fill (dotB) circle (2pt);
          \end{scope}
        },
    },
  #1,
}

%%%%%%%%%%%%%%%%%%%%%%%%%%%%%%
% SELF MADE COMMANDS
%%%%%%%%%%%%%%%%%%%%%%%%%%%%%%


\newcommand{\thm}[2]{\begin{Theorem}{#1}{}#2\end{Theorem}}
\newcommand{\cor}[2]{\begin{Corollary}{#1}{}#2\end{Corollary}}
\newcommand{\mlenma}[2]{\begin{Lenma}{#1}{}#2\end{Lenma}}
\newcommand{\mprop}[2]{\begin{Prop}{#1}{}#2\end{Prop}}
\newcommand{\clm}[3]{\begin{claim}{#1}{#2}#3\end{claim}}
\newcommand{\wc}[2]{\begin{wconc}{#1}{}\setlength{\parindent}{1cm}#2\end{wconc}}
\newcommand{\thmcon}[1]{\begin{Theoremcon}{#1}\end{Theoremcon}}
\newcommand{\ex}[2]{\begin{Example}{#1}{}#2\end{Example}}
\newcommand{\dfn}[2]{\begin{Definition}[colbacktitle=red!75!black]{#1}{}#2\end{Definition}}
\newcommand{\dfnc}[2]{\begin{definition}[colbacktitle=red!75!black]{#1}{}#2\end{definition}}
\newcommand{\qs}[2]{\begin{question}{#1}{}#2\end{question}}
\newcommand{\pf}[2]{\begin{myproof}[#1]#2\end{myproof}}
\newcommand{\nt}[1]{\begin{note}#1\end{note}}

\newcommand*\circled[1]{\tikz[baseline=(char.base)]{
    \node[shape=circle,draw,inner sep=1pt] (char) {#1};}}
\newcommand\getcurrentref[1]{%
  \ifnumequal{\value{#1}}{0}
  {??}
  {\the\value{#1}}%
}
\newcommand{\getCurrentSectionNumber}{\getcurrentref{section}}
\newenvironment{myproof}[1][\proofname]{%
  \proof[\bfseries #1: ]%
}{\endproof}

\newcommand{\mclm}[2]{\begin{myclaim}[#1]#2\end{myclaim}}
\newenvironment{myclaim}[1][\claimname]{\proof[\bfseries #1: ]}{}

\newcounter{mylabelcounter}

\makeatletter
\newcommand{\setword}[2]{%
  \phantomsection
  #1\def\@currentlabel{\unexpanded{#1}}\label{#2}%
}
\makeatother




\tikzset{
  symbol/.style={
      draw=none,
      every to/.append style={
          edge node={node [sloped, allow upside down, auto=false]{$#1$}}}
    }
}


% deliminators
\DeclarePairedDelimiter{\abs}{\lvert}{\rvert}
\DeclarePairedDelimiter{\norm}{\lVert}{\rVert}

\DeclarePairedDelimiter{\ceil}{\lceil}{\rceil}
\DeclarePairedDelimiter{\floor}{\lfloor}{\rfloor}
\DeclarePairedDelimiter{\round}{\lfloor}{\rceil}

\newsavebox\diffdbox
\newcommand{\slantedromand}{{\mathpalette\makesl{d}}}
\newcommand{\makesl}[2]{%
  \begingroup
  \sbox{\diffdbox}{$\mathsurround=0pt#1\mathrm{#2}$}%
  \pdfsave
  \pdfsetmatrix{1 0 0.2 1}%
  \rlap{\usebox{\diffdbox}}%
  \pdfrestore
  \hskip\wd\diffdbox
  \endgroup
}
\newcommand{\dd}[1][]{\ensuremath{\mathop{}\!\ifstrempty{#1}{%
      \slantedromand\@ifnextchar^{\hspace{0.2ex}}{\hspace{0.1ex}}}%
    {\slantedromand\hspace{0.2ex}^{#1}}}}
\ProvideDocumentCommand\dv{o m g}{%
  \ensuremath{%
    \IfValueTF{#3}{%
      \IfNoValueTF{#1}{%
        \frac{\dd #2}{\dd #3}%
      }{%
        \frac{\dd^{#1} #2}{\dd #3^{#1}}%
      }%
    }{%
      \IfNoValueTF{#1}{%
        \frac{\dd}{\dd #2}%
      }{%
        \frac{\dd^{#1}}{\dd #2^{#1}}%
      }%
    }%
  }%
}
\providecommand*{\pdv}[3][]{\frac{\partial^{#1}#2}{\partial#3^{#1}}}
%  - others
\DeclareMathOperator{\Lap}{\mathcal{L}}
\DeclareMathOperator{\Var}{Var} % varience
\DeclareMathOperator{\Cov}{Cov} % covarience
\DeclareMathOperator{\E}{E} % expected

% Since the amsthm package isn't loaded

% I prefer the slanted \leq
\let\oldleq\leq % save them in case they're every wanted
\let\oldgeq\geq
\renewcommand{\leq}{\leqslant}
\renewcommand{\geq}{\geqslant}

% % redefine matrix env to allow for alignment, use r as default
% \renewcommand*\env@matrix[1][r]{\hskip -\arraycolsep
%     \let\@ifnextchar\new@ifnextchar
%     \array{*\c@MaxMatrixCols #1}}


%\usepackage{framed}
%\usepackage{titletoc}
%\usepackage{etoolbox}
%\usepackage{lmodern}


%\patchcmd{\tableofcontents}{\contentsname}{\sffamily\contentsname}{}{}

%\renewenvironment{leftbar}
%{\def\FrameCommand{\hspace{6em}%
%		{\color{myyellow}\vrule width 2pt depth 6pt}\hspace{1em}}%
%	\MakeFramed{\parshape 1 0cm \dimexpr\textwidth-6em\relax\FrameRestore}\vskip2pt%
%}
%{\endMakeFramed}

%\titlecontents{chapter}
%[0em]{\vspace*{2\baselineskip}}
%{\parbox{4.5em}{%
%		\hfill\Huge\sffamily\bfseries\color{myred}\thecontentspage}%
%	\vspace*{-2.3\baselineskip}\leftbar\textsc{\small\chaptername~\thecontentslabel}\\\sffamily}
%{}{\endleftbar}
%\titlecontents{section}
%[8.4em]
%{\sffamily\contentslabel{3em}}{}{}
%{\hspace{0.5em}\nobreak\itshape\color{myred}\contentspage}
%\titlecontents{subsection}
%[8.4em]
%{\sffamily\contentslabel{3em}}{}{}  
%{\hspace{0.5em}\nobreak\itshape\color{myred}\contentspage}



%%%%%%%%%%%%%%%%%%%%%%%%%%%%%%%%%%%%%%%%%%%
% TABLE OF CONTENTS
%%%%%%%%%%%%%%%%%%%%%%%%%%%%%%%%%%%%%%%%%%%

\usepackage{tikz}
\definecolor{doc}{RGB}{0,60,110}
\usepackage{titletoc}
\contentsmargin{0cm}
\titlecontents{chapter}[3.7pc]
{\addvspace{30pt}%
  \begin{tikzpicture}[remember picture, overlay]%
    \draw[fill=doc!60,draw=doc!60] (-7,-.1) rectangle (-0.9,.5);%
    \pgftext[left,x=-3.5cm,y=0.2cm]{\color{white}\Large\sc\bfseries Chapter\ \thecontentslabel};%
  \end{tikzpicture}\color{doc!60}\large\sc\bfseries}%
{}
{}
{\;\titlerule\;\large\sc\bfseries Page \thecontentspage
  \begin{tikzpicture}[remember picture, overlay]
    \draw[fill=doc!60,draw=doc!60] (2pt,0) rectangle (4,0.1pt);
  \end{tikzpicture}}%
\titlecontents{section}[3.7pc]
{\addvspace{2pt}}
{\contentslabel[\thecontentslabel]{2pc}}
{}
{\hfill\small \thecontentspage}
[]
\titlecontents*{subsection}[3.7pc]
{\addvspace{-1pt}\small}
{}
{}
{\ --- \small\thecontentspage}
[ \textbullet\ ][]

\makeatletter
\renewcommand{\tableofcontents}{%
  \chapter*{%
    \vspace*{-20\p@}%
    \begin{tikzpicture}[remember picture, overlay]%
      \pgftext[right,x=15cm,y=0.2cm]{\color{doc!60}\Huge\sc\bfseries \contentsname};%
      \draw[fill=doc!60,draw=doc!60] (13,-.75) rectangle (20,1);%
      \clip (13,-.75) rectangle (20,1);
      \pgftext[right,x=15cm,y=0.2cm]{\color{white}\Huge\sc\bfseries \contentsname};%
    \end{tikzpicture}}%
  \@starttoc{toc}}
\makeatother

%From M275 "Topology" at SJSU
\newcommand{\id}{\mathrm{id}}
\newcommand{\taking}[1]{\xrightarrow{#1}}
\newcommand{\inv}{^{-1}}

%From M170 "Introduction to Graph Theory" at SJSU
\DeclareMathOperator{\diam}{diam}
\DeclareMathOperator{\ord}{ord}
\newcommand{\defeq}{\overset{\mathrm{def}}{=}}

%From the USAMO .tex files
\newcommand{\ts}{\textsuperscript}
\newcommand{\dg}{^\circ}
\newcommand{\ii}{\item}

% % From Math 55 and Math 145 at Harvard
% \newenvironment{subproof}[1][Proof]{%
% \begin{proof}[#1] \renewcommand{\qedsymbol}{$\blacksquare$}}%
% {\end{proof}}

\newcommand{\liff}{\leftrightarrow}
\newcommand{\lthen}{\rightarrow}
\newcommand{\opname}{\operatorname}
\newcommand{\surjto}{\twoheadrightarrow}
\newcommand{\injto}{\hookrightarrow}
\newcommand{\On}{\mathrm{On}} % ordinals
\DeclareMathOperator{\img}{im} % Image
\DeclareMathOperator{\Img}{Im} % Image
\DeclareMathOperator{\coker}{coker} % Cokernel
\DeclareMathOperator{\Coker}{Coker} % Cokernel
\DeclareMathOperator{\Ker}{Ker} % Kernel
\DeclareMathOperator{\rank}{rank}
\DeclareMathOperator{\Spec}{Spec} % spectrum
\DeclareMathOperator{\Tr}{Tr} % trace
\DeclareMathOperator{\pr}{pr} % projection
\DeclareMathOperator{\ext}{ext} % extension
\DeclareMathOperator{\pred}{pred} % predecessor
\DeclareMathOperator{\dom}{dom} % domain
\DeclareMathOperator{\ran}{ran} % range
\DeclareMathOperator{\Hom}{Hom} % homomorphism
\DeclareMathOperator{\Mor}{Mor} % morphisms
\DeclareMathOperator{\End}{End} % endomorphism

\newcommand{\eps}{\epsilon}
\newcommand{\veps}{\varepsilon}
\newcommand{\ol}{\overline}
\newcommand{\ul}{\underline}
\newcommand{\wt}{\widetilde}
\newcommand{\wh}{\widehat}
\newcommand{\vocab}[1]{\textbf{\color{blue} #1}}
\providecommand{\half}{\frac{1}{2}}
\newcommand{\dang}{\measuredangle} %% Directed angle
\newcommand{\ray}[1]{\overrightarrow{#1}}
\newcommand{\seg}[1]{\overline{#1}}
\newcommand{\arc}[1]{\wideparen{#1}}
\DeclareMathOperator{\cis}{cis}
\DeclareMathOperator*{\lcm}{lcm}
\DeclareMathOperator*{\argmin}{arg min}
\DeclareMathOperator*{\argmax}{arg max}
\newcommand{\cycsum}{\sum_{\mathrm{cyc}}}
\newcommand{\symsum}{\sum_{\mathrm{sym}}}
\newcommand{\cycprod}{\prod_{\mathrm{cyc}}}
\newcommand{\symprod}{\prod_{\mathrm{sym}}}
\newcommand{\Qed}{\begin{flushright}\qed\end{flushright}}
\newcommand{\parinn}{\setlength{\parindent}{1cm}}
\newcommand{\parinf}{\setlength{\parindent}{0cm}}
% \newcommand{\norm}{\|\cdot\|}
\newcommand{\inorm}{\norm_{\infty}}
\newcommand{\opensets}{\{V_{\alpha}\}_{\alpha\in I}}
\newcommand{\oset}{V_{\alpha}}
\newcommand{\opset}[1]{V_{\alpha_{#1}}}
\newcommand{\lub}{\text{lub}}
\newcommand{\del}[2]{\frac{\partial #1}{\partial #2}}
\newcommand{\Del}[3]{\frac{\partial^{#1} #2}{\partial^{#1} #3}}
\newcommand{\deld}[2]{\dfrac{\partial #1}{\partial #2}}
\newcommand{\Deld}[3]{\dfrac{\partial^{#1} #2}{\partial^{#1} #3}}
\newcommand{\lm}{\lambda}
\newcommand{\uin}{\mathbin{\rotatebox[origin=c]{90}{$\in$}}}
\newcommand{\usubset}{\mathbin{\rotatebox[origin=c]{90}{$\subset$}}}
\newcommand{\lt}{\left}
\newcommand{\rt}{\right}
\newcommand{\bs}[1]{\boldsymbol{#1}}
\newcommand{\exs}{\exists}
\newcommand{\st}{\strut}
\newcommand{\dps}[1]{\displaystyle{#1}}

\newcommand{\sol}{\setlength{\parindent}{0cm}\textbf{\textit{Solution:}}\setlength{\parindent}{1cm} }
\newcommand{\solve}[1]{\setlength{\parindent}{0cm}\textbf{\textit{Solution: }}\setlength{\parindent}{1cm}#1 \Qed}

\preto\tabular{\setcounter{magicrownumbers}{0}}
\newcounter{magicrownumbers}
\newcommand\rownumber{\stepcounter{magicrownumbers}\arabic{magicrownumbers}}
\def\rownumber{}

\newenvironment{deduction}
{\begin{tabular}{@{}>{$}c<{$}@{\enspace}>{$}l<{$}@{}}\arrayrulecolor{blue!50}}
		{\end{tabular}}
\newcommand{\premise}[1]{&#1\\}
\newcommand{\conclusion}[1]{\cline{2-2}\therefore&#1}


% Things Lie
\newcommand{\kb}{\mathfrak b}
\newcommand{\kg}{\mathfrak g}
\newcommand{\kh}{\mathfrak h}
\newcommand{\kn}{\mathfrak n}
\newcommand{\ku}{\mathfrak u}
\newcommand{\kz}{\mathfrak z}
\DeclareMathOperator{\Ext}{Ext} % Ext functor
\DeclareMathOperator{\Tor}{Tor} % Tor functor
\newcommand{\gl}{\opname{\mathfrak{gl}}} % frak gl group
\renewcommand{\sl}{\opname{\mathfrak{sl}}} % frak sl group chktex 6

% More script letters etc.
\newcommand{\SA}{\mathcal A}
\newcommand{\SB}{\mathcal B}
\newcommand{\SC}{\mathcal C}
\newcommand{\SF}{\mathcal F}
\newcommand{\SG}{\mathcal G}
\newcommand{\SH}{\mathcal H}
\newcommand{\OO}{\mathcal O}

\newcommand{\SCA}{\mathscr A}
\newcommand{\SCB}{\mathscr B}
\newcommand{\SCC}{\mathscr C}
\newcommand{\SCD}{\mathscr D}
\newcommand{\SCE}{\mathscr E}
\newcommand{\SCF}{\mathscr F}
\newcommand{\SCG}{\mathscr G}
\newcommand{\SCH}{\mathscr H}

% Mathfrak primes
\newcommand{\km}{\mathfrak m}
\newcommand{\kp}{\mathfrak p}
\newcommand{\kq}{\mathfrak q}

% number sets
\newcommand{\RR}[1][]{\ensuremath{\ifstrempty{#1}{\mathbb{R}}{\mathbb{R}^{#1}}}}
\newcommand{\NN}[1][]{\ensuremath{\ifstrempty{#1}{\mathbb{N}}{\mathbb{N}^{#1}}}}
\newcommand{\ZZ}[1][]{\ensuremath{\ifstrempty{#1}{\mathbb{Z}}{\mathbb{Z}^{#1}}}}
\newcommand{\QQ}[1][]{\ensuremath{\ifstrempty{#1}{\mathbb{Q}}{\mathbb{Q}^{#1}}}}
\newcommand{\CC}[1][]{\ensuremath{\ifstrempty{#1}{\mathbb{C}}{\mathbb{C}^{#1}}}}
\newcommand{\PP}[1][]{\ensuremath{\ifstrempty{#1}{\mathbb{P}}{\mathbb{P}^{#1}}}}
\newcommand{\HH}[1][]{\ensuremath{\ifstrempty{#1}{\mathbb{H}}{\mathbb{H}^{#1}}}}
\newcommand{\FF}[1][]{\ensuremath{\ifstrempty{#1}{\mathbb{F}}{\mathbb{F}^{#1}}}}
% expected value
\newcommand{\EE}{\ensuremath{\mathbb{E}}}
\newcommand{\charin}{\text{ char }}
\DeclareMathOperator{\sign}{sign}
\DeclareMathOperator{\Aut}{Aut}
\DeclareMathOperator{\Inn}{Inn}
\DeclareMathOperator{\Syl}{Syl}
\DeclareMathOperator{\Gal}{Gal}
\DeclareMathOperator{\GL}{GL} % General linear group
\DeclareMathOperator{\SL}{SL} % Special linear group

%---------------------------------------
% BlackBoard Math Fonts :-
%---------------------------------------

%Captital Letters
\newcommand{\bbA}{\mathbb{A}}	\newcommand{\bbB}{\mathbb{B}}
\newcommand{\bbC}{\mathbb{C}}	\newcommand{\bbD}{\mathbb{D}}
\newcommand{\bbE}{\mathbb{E}}	\newcommand{\bbF}{\mathbb{F}}
\newcommand{\bbG}{\mathbb{G}}	\newcommand{\bbH}{\mathbb{H}}
\newcommand{\bbI}{\mathbb{I}}	\newcommand{\bbJ}{\mathbb{J}}
\newcommand{\bbK}{\mathbb{K}}	\newcommand{\bbL}{\mathbb{L}}
\newcommand{\bbM}{\mathbb{M}}	\newcommand{\bbN}{\mathbb{N}}
\newcommand{\bbO}{\mathbb{O}}	\newcommand{\bbP}{\mathbb{P}}
\newcommand{\bbQ}{\mathbb{Q}}	\newcommand{\bbR}{\mathbb{R}}
\newcommand{\bbS}{\mathbb{S}}	\newcommand{\bbT}{\mathbb{T}}
\newcommand{\bbU}{\mathbb{U}}	\newcommand{\bbV}{\mathbb{V}}
\newcommand{\bbW}{\mathbb{W}}	\newcommand{\bbX}{\mathbb{X}}
\newcommand{\bbY}{\mathbb{Y}}	\newcommand{\bbZ}{\mathbb{Z}}

%---------------------------------------
% MathCal Fonts :-
%---------------------------------------

%Captital Letters
\newcommand{\mcA}{\mathcal{A}}	\newcommand{\mcB}{\mathcal{B}}
\newcommand{\mcC}{\mathcal{C}}	\newcommand{\mcD}{\mathcal{D}}
\newcommand{\mcE}{\mathcal{E}}	\newcommand{\mcF}{\mathcal{F}}
\newcommand{\mcG}{\mathcal{G}}	\newcommand{\mcH}{\mathcal{H}}
\newcommand{\mcI}{\mathcal{I}}	\newcommand{\mcJ}{\mathcal{J}}
\newcommand{\mcK}{\mathcal{K}}	\newcommand{\mcL}{\mathcal{L}}
\newcommand{\mcM}{\mathcal{M}}	\newcommand{\mcN}{\mathcal{N}}
\newcommand{\mcO}{\mathcal{O}}	\newcommand{\mcP}{\mathcal{P}}
\newcommand{\mcQ}{\mathcal{Q}}	\newcommand{\mcR}{\mathcal{R}}
\newcommand{\mcS}{\mathcal{S}}	\newcommand{\mcT}{\mathcal{T}}
\newcommand{\mcU}{\mathcal{U}}	\newcommand{\mcV}{\mathcal{V}}
\newcommand{\mcW}{\mathcal{W}}	\newcommand{\mcX}{\mathcal{X}}
\newcommand{\mcY}{\mathcal{Y}}	\newcommand{\mcZ}{\mathcal{Z}}


%---------------------------------------
% Bold Math Fonts :-
%---------------------------------------

%Captital Letters
\newcommand{\bmA}{\boldsymbol{A}}	\newcommand{\bmB}{\boldsymbol{B}}
\newcommand{\bmC}{\boldsymbol{C}}	\newcommand{\bmD}{\boldsymbol{D}}
\newcommand{\bmE}{\boldsymbol{E}}	\newcommand{\bmF}{\boldsymbol{F}}
\newcommand{\bmG}{\boldsymbol{G}}	\newcommand{\bmH}{\boldsymbol{H}}
\newcommand{\bmI}{\boldsymbol{I}}	\newcommand{\bmJ}{\boldsymbol{J}}
\newcommand{\bmK}{\boldsymbol{K}}	\newcommand{\bmL}{\boldsymbol{L}}
\newcommand{\bmM}{\boldsymbol{M}}	\newcommand{\bmN}{\boldsymbol{N}}
\newcommand{\bmO}{\boldsymbol{O}}	\newcommand{\bmP}{\boldsymbol{P}}
\newcommand{\bmQ}{\boldsymbol{Q}}	\newcommand{\bmR}{\boldsymbol{R}}
\newcommand{\bmS}{\boldsymbol{S}}	\newcommand{\bmT}{\boldsymbol{T}}
\newcommand{\bmU}{\boldsymbol{U}}	\newcommand{\bmV}{\boldsymbol{V}}
\newcommand{\bmW}{\boldsymbol{W}}	\newcommand{\bmX}{\boldsymbol{X}}
\newcommand{\bmY}{\boldsymbol{Y}}	\newcommand{\bmZ}{\boldsymbol{Z}}
%Small Letters
\newcommand{\bma}{\boldsymbol{a}}	\newcommand{\bmb}{\boldsymbol{b}}
\newcommand{\bmc}{\boldsymbol{c}}	\newcommand{\bmd}{\boldsymbol{d}}
\newcommand{\bme}{\boldsymbol{e}}	\newcommand{\bmf}{\boldsymbol{f}}
\newcommand{\bmg}{\boldsymbol{g}}	\newcommand{\bmh}{\boldsymbol{h}}
\newcommand{\bmi}{\boldsymbol{i}}	\newcommand{\bmj}{\boldsymbol{j}}
\newcommand{\bmk}{\boldsymbol{k}}	\newcommand{\bml}{\boldsymbol{l}}
\newcommand{\bmm}{\boldsymbol{m}}	\newcommand{\bmn}{\boldsymbol{n}}
\newcommand{\bmo}{\boldsymbol{o}}	\newcommand{\bmp}{\boldsymbol{p}}
\newcommand{\bmq}{\boldsymbol{q}}	\newcommand{\bmr}{\boldsymbol{r}}
\newcommand{\bms}{\boldsymbol{s}}	\newcommand{\bmt}{\boldsymbol{t}}
\newcommand{\bmu}{\boldsymbol{u}}	\newcommand{\bmv}{\boldsymbol{v}}
\newcommand{\bmw}{\boldsymbol{w}}	\newcommand{\bmx}{\boldsymbol{x}}
\newcommand{\bmy}{\boldsymbol{y}}	\newcommand{\bmz}{\boldsymbol{z}}

%---------------------------------------
% Scr Math Fonts :-
%---------------------------------------

\newcommand{\sA}{{\mathscr{A}}}   \newcommand{\sB}{{\mathscr{B}}}
\newcommand{\sC}{{\mathscr{C}}}   \newcommand{\sD}{{\mathscr{D}}}
\newcommand{\sE}{{\mathscr{E}}}   \newcommand{\sF}{{\mathscr{F}}}
\newcommand{\sG}{{\mathscr{G}}}   \newcommand{\sH}{{\mathscr{H}}}
\newcommand{\sI}{{\mathscr{I}}}   \newcommand{\sJ}{{\mathscr{J}}}
\newcommand{\sK}{{\mathscr{K}}}   \newcommand{\sL}{{\mathscr{L}}}
\newcommand{\sM}{{\mathscr{M}}}   \newcommand{\sN}{{\mathscr{N}}}
\newcommand{\sO}{{\mathscr{O}}}   \newcommand{\sP}{{\mathscr{P}}}
\newcommand{\sQ}{{\mathscr{Q}}}   \newcommand{\sR}{{\mathscr{R}}}
\newcommand{\sS}{{\mathscr{S}}}   \newcommand{\sT}{{\mathscr{T}}}
\newcommand{\sU}{{\mathscr{U}}}   \newcommand{\sV}{{\mathscr{V}}}
\newcommand{\sW}{{\mathscr{W}}}   \newcommand{\sX}{{\mathscr{X}}}
\newcommand{\sY}{{\mathscr{Y}}}   \newcommand{\sZ}{{\mathscr{Z}}}


%---------------------------------------
% Math Fraktur Font
%---------------------------------------

%Captital Letters
\newcommand{\mfA}{\mathfrak{A}}	\newcommand{\mfB}{\mathfrak{B}}
\newcommand{\mfC}{\mathfrak{C}}	\newcommand{\mfD}{\mathfrak{D}}
\newcommand{\mfE}{\mathfrak{E}}	\newcommand{\mfF}{\mathfrak{F}}
\newcommand{\mfG}{\mathfrak{G}}	\newcommand{\mfH}{\mathfrak{H}}
\newcommand{\mfI}{\mathfrak{I}}	\newcommand{\mfJ}{\mathfrak{J}}
\newcommand{\mfK}{\mathfrak{K}}	\newcommand{\mfL}{\mathfrak{L}}
\newcommand{\mfM}{\mathfrak{M}}	\newcommand{\mfN}{\mathfrak{N}}
\newcommand{\mfO}{\mathfrak{O}}	\newcommand{\mfP}{\mathfrak{P}}
\newcommand{\mfQ}{\mathfrak{Q}}	\newcommand{\mfR}{\mathfrak{R}}
\newcommand{\mfS}{\mathfrak{S}}	\newcommand{\mfT}{\mathfrak{T}}
\newcommand{\mfU}{\mathfrak{U}}	\newcommand{\mfV}{\mathfrak{V}}
\newcommand{\mfW}{\mathfrak{W}}	\newcommand{\mfX}{\mathfrak{X}}
\newcommand{\mfY}{\mathfrak{Y}}	\newcommand{\mfZ}{\mathfrak{Z}}
%Small Letters
\newcommand{\mfa}{\mathfrak{a}}	\newcommand{\mfb}{\mathfrak{b}}
\newcommand{\mfc}{\mathfrak{c}}	\newcommand{\mfd}{\mathfrak{d}}
\newcommand{\mfe}{\mathfrak{e}}	\newcommand{\mff}{\mathfrak{f}}
\newcommand{\mfg}{\mathfrak{g}}	\newcommand{\mfh}{\mathfrak{h}}
\newcommand{\mfi}{\mathfrak{i}}	\newcommand{\mfj}{\mathfrak{j}}
\newcommand{\mfk}{\mathfrak{k}}	\newcommand{\mfl}{\mathfrak{l}}
\newcommand{\mfm}{\mathfrak{m}}	\newcommand{\mfn}{\mathfrak{n}}
\newcommand{\mfo}{\mathfrak{o}}	\newcommand{\mfp}{\mathfrak{p}}
\newcommand{\mfq}{\mathfrak{q}}	\newcommand{\mfr}{\mathfrak{r}}
\newcommand{\mfs}{\mathfrak{s}}	\newcommand{\mft}{\mathfrak{t}}
\newcommand{\mfu}{\mathfrak{u}}	\newcommand{\mfv}{\mathfrak{v}}
\newcommand{\mfw}{\mathfrak{w}}	\newcommand{\mfx}{\mathfrak{x}}
\newcommand{\mfy}{\mathfrak{y}}	\newcommand{\mfz}{\mathfrak{z}}


\title{\Huge{Logic and Proofs}}
\author{\huge{Madiba Hudson-Quansah}}
\date{}
\usepackage{parskip}
\usepackage{proof}
\setcounter{tocdepth}{4}
\setcounter{secnumdepth}{4}

\usepackage{xcolor,colortbl,array,amssymb}
\usepackage{ebproof}


\begin{document}
\maketitle
\newpage
\pdfbookmark[section]{\contentsname}{too}
\tableofcontents
\pagebreak

\chapter{Propositional Logic}

\dfn{}{
	\begin{itemize}
		\item Proof - A correct mathematical argument.
		\item Theorem - A proven mathematical statement.

	\end{itemize}
}

\section{Proposition}

\dfn{A Proposition}{
	A declarative sentence that is either true or false, but not both. e.g.
	\begin{itemize}
		\item $1 + 1 = 2$ True
		\item $2 + 2 = 3$ False
	\end{itemize}
}

Propositional variables / Statement variables are used to represent propositions, by convention one of these variables
$p, q, r, s \ldots$ The truth value of a position can be denoted by $T$ if it is a \textbf{true proposition} and $F$ if it is
a \textbf{false proposition}.


Therefore, Let $p$ be a proposition. The \textit{negation of} $p$, denoted by $\neg p$ / $\overline{p}$, is the
statement

"It is not the case that $p$"

The proposition $\neg p$ is read "not p", therefore the truth value of the negation of $p$ is the inverse of the truth
value of $p$

\ex{}{
	\qs{}{
		Find the negation of the proposition

		"Michael's PC runs Linux"

		and express this in simple English.
	}
	\sol{
		"Michael's PC does not run Linux"
	}
}

\pagebreak
\ex{}{
	\qs{}{
		Find the negation of the proposition

		"Vandana's smartphone has at least 32GB of memory"

		and express this in simple English.
	}
	\sol{
		"Vandana's has less than 32GB of memory"
	}
}

\dfn{Truth Table} {
	Displays the relationships between the truth values of propositions.
}


\begin{table}[h!]
	\centering
	\begin{tabular}{|c|c|}
		\hline
		$p$ & $\neg p$ \\[0.5ex]
		\hline
		\hline
		$T$ & $F$      \\
		$F$ & $T$      \\
		\hline
	\end{tabular}
	\caption{The truth table for the negation of a proposition}
	\label{tab:1}
\end{table}

The negation of a proposition can also be considered the result of the operation of the \textit{negation operator} on
the proposition.

\section{Logical Operators / Connectives}

\subsection{Conjunction}
\dfn{Conjunction}{
	Let $p$ and $q$ be propositions. The \textit{conjunction} of $p$ and $q$, denoted by $p \wedge q$, is the proposition
	"$p \text{ and } q$". $p \wedge q$ is $T$ when both $q$ and $p$ are $T$ and is $F$ otherwise
}

\begin{table}[h!]
	\caption{The truth table of $p \wedge q$}\label{tab:2}
	\begin{center}
		\begin{tabular}{|c c|c|}
			\hline
			$p$ & $q$ & $p \wedge q$ \\[0.5ex]
			\hline
			\hline
			$T$ & $T$ & $T$          \\
			$T$ & $F$ & $F$          \\
			$F$ & $T$ & $F$          \\
			$F$ & $F$ & $F$          \\
			\hline
		\end{tabular}
	\end{center}
\end{table}


\subsection{Disjunction}
\dfn{Disjunction}{
	Let $p$ and $q$ be propositions. The \textit{disjunction} of $p$ and $q$, dented by $p \vee q $, is the proposition, "$p
		\text{ or } q$". The \textit{disjunction} $p \vee q$ is $F$ when both $p$ and $q$ are $F$ and $T$ otherwise.
}

\begin{table}[h!]
	\caption{The truth table of $p \vee q$}\label{tab:3}
	\begin{center}
		\begin{tabular}{|c c|c|}
			\hline
			$p$ & $q$ & $p \vee q$ \\[0.5ex]
			\hline
			\hline
			$T$ & $T$ & $T$        \\
			$T$ & $F$ & $T$        \\
			$F$ & $T$ & $T$        \\
			$F$ & $F$ & $F$        \\
			\hline
		\end{tabular}
	\end{center}
\end{table}

The use of the \textbf{connective} \textit{or} in a disjunction corresponds to one of the two ways the word \textit{or}
is used in English. \textbf{Inclusive or} and \textbf{Exclusive or}, e.g.

"Students who have taken calculus or computer science can take this class"


"Students who have taken calculus or computer science, but not both can take this class"

Respectively. Therefore taking the disjunction $p \vee q$ an \textbf{Exclusive or} disjunction will $F$ when $q = T
	\text{ and } p = T$ or $q = F \text{ and } p = F$, and $T$ only when $q = T \text{ and } p = F$ or $q = F \text{ and
	} p = T$


\subsubsection{Exclusive or}
\dfn{Exclusive Or}{
	Let $p$ and $q$ be propositions. The \textit{exclusive or} of $p$ and $q$, denoted by $p \oplus q$, is the proposition
	that is $T$ when exactly one of $p$ and $q$ is $T$ and is $F$ otherwise.
}

\begin{table}[h!]
	\caption{The truth table of $p \oplus q$}\label{tab:4}
	\begin{center}
		\begin{tabular}{|c c|c|}
			\hline
			$p$ & $q$ & $p \oplus q$ \\[0.5ex]
			\hline
			\hline
			$T$ & $T$ & $F$          \\
			$T$ & $F$ & $T$          \\
			$F$ & $T$ & $T$          \\
			$F$ & $F$ & $F$          \\
			\hline
		\end{tabular}
	\end{center}
\end{table}

\subsection{Conditional Statement / Implication}

\dfn{Conditional Statement / Implication}{
	Let $p$ and $q$ be propositions. The \textit{conditional statement} $p \to q$ is the proposition "if $p$, then $q$". $p
		\to q$ is $F$ when $p$ is $T$ and $q$ is $F$, and $T$ otherwise. In this connective, $p$ is called the
	\textit{hypothesis / antecedent / premise} and $q$ is called the \textit{conclusion / consequence}.
}

\begin{table}[h!]
	\caption{The truth table of $p \to  q$}\label{tab:5}
	\begin{center}
		\begin{tabular}{|c c|c|}
			\hline
			$p$ & $q$ & $p \to q$ \\[0.5ex]
			\hline
			\hline
			$T$ & $T$ & $T$       \\
			$T$ & $F$ & $F$       \\
			$F$ & $T$ & $T$       \\
			$F$ & $F$ & $T$       \\
			\hline
		\end{tabular}
	\end{center}
\end{table}

$p \to q$ is called a conditional statement because, it asserts that $q$ is $T$ on the condition that $p$ holds.
$p \to q$ is $T$ when $p$ is $F$ no matter the value of $q$.

Conditional statements can be expressed in various ways, some are listed below.

"if $p$, then $q$"

"if $p$, $q$"

"$p$ is sufficient for $q$"

"$q$ if $p$"

"$q$ when $p$"

"a necessary condition for $p$ is $q$"

"$q$ unless $\neg p$"


"$p$ implies $q$"

"$p$ only if $q$"

"a sufficient condition for $q$ is $p$"

"$q$ whenever $p$"

"$q$ is necessary for $p$"

"$q$ follows from $p$"

For the more confusing statements "$p$ only if $q$" and "$q$ unless $\neg p$", the explanation follows.

"$p$ only if $q$" corresponds to "if $p$, then $q$", because "$p$ only if $q$" says that $p$ cannot be $T$ when $q$ is not
$T$, i.e. the statement is $F$ when is $p$ is $T$ but $q$ is $F$. If $p$ is $F$ $q$ maybe either $F$ or $T$ because the
statement says nothing about the value of $q$.

"$q$ unless $\neg p$" expresses the same conditional statement as "if $p$, then $q$", because "$q$ unless $\neg p$"
means that if "if $\neg p$" is $F$ then $q$ must be $T$, That is the statement "$q$ unless $\neg p$" is $F$ when $p$ is
$T$ but $q$ is $F$, but $T$ otherwise.

Or

\begin{align*}
	q \text{ unless } \neg p  \\
	q \text{ if } \neg \neg p \\
	q \text{ if } p           \\
	\therefore \text{if } p \text{ then } q
\end{align*}

\ex{}{
	\qs{}{
		Let $p$ be the statement "Maria learns discrete mathematics" and $q$ be the statement "Maria will find a good
		job". Express the statement $p \to q$ as  a statement in English.
	}

	\sol{

		"Maria will find a good job, if she learns discrete mathematics"

		"For Maria to get a good job, it is sufficient for her to learn discrete mathematics"

		"Maria will find a good job unless she does not learn discrete mathematics"
	}
}

We can form new conditional statements from a given conditional statement, lets say $p \to q$. These are

\begin{itemize}
	\item \textbf{Converse} - $q \to  p$
	\item \textbf{Contrapositive} - $\neg q \to \neg p$
	\item \textbf{Inverse} - $\neg p \to \neg q$
\end{itemize}

\dfn{Equivalence}{
	When two compound propositions always have the same truth value.
}

\subsubsection{Contrapositive}

\dfn{Contrapositive}{
	The \textit{contrapositive} of the conditional statement $p \to q$ is the conditional statement $\neg q \to \neg p$
}

In this statement, the hypothesis ,$p$, and conclusion , $q$, are reversed and negated. This results in an identical
truth table, as the contrapositive is only $F$ when $\neg p$ is $F$ and $\neg q$ is $T$.

\nt{
	A \textbf{Conditional Statement} ( $p \to q$ ) is equivalent to it's \textbf{Contrapositive} ( $\neg q \to  \neg p$ )

}

\begin{table}[h!]
	\caption{The truth table of $\neg q \to  \neg p$}\label{tab:6}
	\begin{center}
		\begin{tabular}{|c c|c c|c|}
			\hline
			$p$ & $q$ & $ \neg p $ & $\neg q$ & $\neg q \to  \neg p$ \\[0.5ex]
			\hline
			\hline
			$T$ & $T$ & $F$        & $F$      & $T$                  \\
			$T$ & $F$ & $F$        & $T$      & $F$                  \\
			$F$ & $T$ & $T$        & $F$      & $T$                  \\
			$F$ & $F$ & $T$        & $T$      & $T$                  \\
			\hline
		\end{tabular}
	\end{center}
\end{table}


\subsubsection{Inverse}

\dfn{Inverse}{
	The  inverse of the conditional statement $p \to q$ is the conditional statement $\neg p \to \neg q$
}

In this statement, the hypothesis ,$p$, and conclusion ,$q$, are negated. This results in a truth table differing from
the original conditional statement but equivalent to the statement's \textbf{converse}.

\nt{
	A conditional statement's ( $p \to q$ ) \textbf{Inverse} ( $\neg p \to \neg q$ ) is equivalent to its \textbf{Converse} ( $q \to p$ )
}

\begin{table}[h!]
	\caption{The truth table of $\neg p \to  \neg q$}\label{tab:7}
	\begin{center}
		\begin{tabular}{|c c|c c|c|}
			\hline
			$p$ & $q$ & $ \neg p $ & $\neg q$ & $\neg p \to  \neg q$ \\[0.5ex]
			\hline
			\hline
			$T$ & $T$ & $F$        & $F$      & $T$                  \\
			$T$ & $F$ & $F$        & $T$      & $T$                  \\
			$F$ & $T$ & $T$        & $F$      & $F$                  \\
			$F$ & $F$ & $T$        & $T$      & $T$                  \\
			\hline
		\end{tabular}
	\end{center}
\end{table}

\subsubsection{Converse}

\dfn{Converse} {
	The converse of the conditional statement $p \to q$ is the conditional statement $q \to p$
}

In this statement, the hypothesis , $p$ and conclusion, , $q$, are reversed. This results in a truth table equivalent to
the conditional statement's inverse.

\nt {
	A conditional statement's ( $p \to q$ ) \textbf{Converse} ( $q \to p$ ) is equivalent to its \textbf{Inverse} (
	$\neg p \to \neg q$ )
}


\begin{table}[h!]
	\caption{The truth table of $q \to p$}\label{tab:8}
	\begin{center}
		\begin{tabular}{|c c|c|}
			\hline
			$p$ & $q$ & $q \to p$ \\[0.5ex]
			\hline
			\hline
			$T$ & $T$ & $T$       \\
			$T$ & $F$ & $T$       \\
			$F$ & $T$ & $F$       \\
			$F$ & $F$ & $T$       \\
			\hline
		\end{tabular}
	\end{center}
\end{table}


\pagebreak
\ex{}{
	\qs{}{
		What are the contrapositive, converse, and inverse of the conditional statement

		"The home team wins whenever it is raining?"
	}

	\sol{
		\begin{align*}
			p \to q                                           \\
			q \text{ whenever } p                             \\
			q = \text{Home team wins }                        \\
			p = \text{It is raining }                         \\
			 & = \text{If it is raining, the home team wins }
		\end{align*}
		\begin{itemize}
			\item Contrapositive - If the home team loses, then it's not raining.
			\item Inverse - If its not raining then the home team loses.
			\item Converse - If the home team wins, then it is raining.
		\end{itemize}
	}
}

\ex{}{
	\qs{}{
		Find the converse, inverse and contrapositive of "Raining is a sufficient condition for my not going to town"
	}

	\sol{

		If it is raining, then I will not go to town

		$p \to q$

		$p$ = It is raining

		$q$ = I will not go to town

		Converse

		$q \to p$

		If I am not going to town, then it is raining

		Contrapositive

		$\neg q \to \neg p $

		If I am going to town, then it is not raining

		Inverse

		$\neg p \to \neg q$

		If it is not raining, then I am going to town

	}
}


\subsection{Biconditionals / Bi-implications}
Another way to combine proposition that expresses they have the same truth value.

\dfn{Biconditionals / Bi-implications}{
	Let $p$ and $q$ be propositions. The \textit{biconditional statement} $p \leftrightarrow q$ is the proposition "$p$
	if and only if $q$". The biconditional statement $p \leftrightarrow q$ is $T$ when $p$ and $q$ have the same truth
	values, and is $F$ otherwise.
}

$p \leftrightarrow q$ breaks down to $ \left( p \to q \right)  \wedge  \left( q \to p \right)  $, and can be expressed as below

"$p$ is necessary and sufficient for $q$"

"if $p$ then $q$, and conversely"

"$p \text{ iff } q$"

\nt{
	"iff" - If and only If
}

\begin{table}[h!]
	\caption{The truth table of $p \leftrightarrow q$}\label{tab:9}
	\begin{center}
		\begin{tabular}{|c c|c|}
			\hline
			$p$ & $q$ & $p \leftrightarrow q$ \\ [0.5ex]
			\hline
			\hline
			$T$ & $T$ & $T$                   \\
			$T$ & $F$ & $F$                   \\
			$F$ & $T$ & $F$                   \\
			$F$ & $F$ & $T$                   \\
			\hline
		\end{tabular}
	\end{center}
\end{table}

\ex{}{
	Let $p$ be the statement "You can take the flight" and let $q$ be the statement "You buy a ticket".

	Then $p \leftrightarrow q$ is the statement:

	"You can take the flight if and only if you buy a ticket"
}

\pagebreak
\section{Compound Propositions}

\qs{}{
	Construct the truth table of the compound proposition
	\[
		\left( p \vee \neg q  \right) \to \left( p \wedge q \right)
	\]
}

\sol{
	\begin{table}[h!]
		\begin{center}
			\begin{tabular}{|c c|c|c|c|c|}
				\hline
				$p$ & $q$ & $\neg q$ & $p \vee \neg q $ & $p \wedge q$ & $\left( p \vee \neg q  \right) \to \left( p \wedge q \right)$ \\ [0.5ex]
				\hline
				\hline
				$T$ & $T$ & $F$      & $T$              & $T$          & $T$                                                           \\
				$T$ & $F$ & $T$      & $T$              & $F$          & $F$                                                           \\
				$F$ & $T$ & $F$      & $F$              & $F$          & $T$                                                           \\
				$F$ & $F$ & $T$      & $T$              & $F$          & $F$                                                           \\
				\hline
			\end{tabular}
		\end{center}
	\end{table}
}

\qs{}{
	Construct

	\[
		p \vee q \to \neg r
	\]
}

\sol {
	\begin{table}[h!]
		\begin{center}
			\begin{tabular}{|c|c|c|c|c|c|}
				\hline
				$p$ & $q$ & $r$ & $\neg r$ & $p \vee q$ & $p \vee q \to \neg r$ \\ [0.5ex]
				\hline
				\hline
				$T$ & $T$ & $T$ & $F$      & $T$        & $F$                   \\
				$T$ & $T$ & $F$ & $T$      & $T$        & $T$                   \\
				$T$ & $F$ & $F$ & $T$      & $T$        & $T$                   \\
				$T$ & $F$ & $T$ & $F$      & $T$        & $F$                   \\
				$F$ & $F$ & $F$ & $T$      & $F$        & $T$                   \\
				$F$ & $T$ & $T$ & $F$      & $T$        & $F$                   \\
				$F$ & $F$ & $T$ & $F$      & $F$        & $T$                   \\
				$F$ & $T$ & $F$ & $T$      & $T$        & $T$                   \\
				\hline
			\end{tabular}
		\end{center}
	\end{table}
}

\subsection{Precedence of Logical Operators}

\begin{table}[h!]
	\caption{Precedence Table}\label{tab:prec}
	\begin{center}
		\begin{tabular}{|c|c|}
			\hline
			Operator          & Precedence \\ [0.5ex]
			\hline
			\hline
			$\neg$            & 1          \\
			$\wedge$          & 2          \\
			$\vee$            & 3          \\
			$\to$             & 4          \\
			$\leftrightarrow$ & 5          \\
			\hline
		\end{tabular}
	\end{center}
\end{table}

Precedence shown from 1 to 5, with 1 having the highest precedence and 5 having the lowest precedence. Operators with
higher precedence are evaluated before operators with lower precedence. Precedence can be overridden by using parentheses.

\subsection{Logic and Bit Operations}


\begin{table}[h!]
	\caption{Truth value to bit table}\label{tab:bit}
	\begin{center}
		\begin{tabular}{|c|c|}
			\hline
			Truth value & Bit \\ [0.5ex]
			\hline
			\hline
			$T$         & 1   \\
			$F$         & 0   \\
			\hline
		\end{tabular}
	\end{center}
\end{table}

A bit can be used to represent a truth value due to its \textit{binary} nature. A variable representing a truth value
can be called a \textit{boolean variable}. Computer bit operations correspond to logical operations, with the operations
\textit{OR}, \textit{AND}, and \textit{XOR} corresponding to the connectives, $\vee$, $\wedge$, and $\oplus$ respectively.

\begin{table}[h!]
	\caption{Bit operations table}\label{tab:ops}
	\begin{center}
		\begin{tabular}{|c|c|c|c|c|}
			\hline
			$x$ & $y$ & $x \wedge y$ & $x \vee y$ & $x \oplus y$ \\ [0.5ex]
			\hline
			\hline
			$1$ & $1$ & $1$          & $1$        & $0$          \\
			$1$ & $0$ & $0$          & $1$        & $1$          \\
			$0$ & $1$ & $0$          & $1$        & $1$          \\
			$0$ & $0$ & $0$          & $0$        & $0$          \\
			\hline
		\end{tabular}
	\end{center}
\end{table}

\dfn{Bit String}{
	A sequence of zero or more bits. The \textit{length} of this string is the number of bits in the string.

	\ex{}{
		$101010011$ is a bit string with a length of $9$.
	}
}


Extending bit operations to bit strings we can define \textit{bitwise AND}, \textit{bitwise OR}, and \textit{bitwise XOR} of two strings of the same length. The new bit string created can be called the \textit{AND}, \textit{OR}, and
\textit{XOR} of the two strings respectively.

\ex{}{
	\qs{}{
		Find the bitwise AND, OR, and XOR of the bit strings 01 1011 0110 and 11 0001 1101.
	}

	\sol{

		\begin{center}
			\textit{AND}
			\begin{align*}
				\begin{split}
					01 1011 0110 \\
					11 0001 1101 \\
					\hline
					01 0001 0100 \\
				\end{split}
			\end{align*}

			\textit{OR}
			\begin{align*}
				\begin{split}
					01 1011 0110 \\
					11 0001 1101 \\
					\hline
					11 1011 1111 \\
				\end{split}
			\end{align*}

			\textit{XOR}
			\begin{align*}
				\begin{split}
					01 1011 0110 \\
					11 0001 1101 \\
					\hline
					10 1010 1011 \\
				\end{split}
			\end{align*}
		\end{center}
	}
}




\chapter{Applications of Propositional Logic}

Statements in natural language are often imprecise and ambiguous, to make these statements more precise they can be
represented with propositional logic.

\subsection{Translating English Sentences}

\ex{}{
	\qs{}{
		How can this English sentence be translated into a logical expression?

		"You can access the Internet from campus only if you are a computer science major or you are not a freshman".
	}

	\sol{

		Let $p$ be "You can access the internet from campus"

		Let $q$ be "You are a computer science major"

		Let $r$ be "You are a freshman"

		Therefore the sentence can be translated as
		\[
			p \to \left( q \vee \neg r \right)
		\]
	}
}

\ex{}{
	\qs{}{
		How can this English sentence be translated into a logical expression?

		“You cannot ride the roller coaster if you are under 4 feet tall unless you are older than 16 years old.”
	}

	\sol {

		Let $p$ be "You can ride the roller coaster"

		Let $q$ be "You are under 4 feet tall"

		Let $r$ "You are at least 16 years old"

		Therefore the sentence can be translated as

		\[
			\left( q \vee \neg r  \right)  \to \neg p
		\]
	}
}

\subsection{System Specifications}

In system specifications, we can use propositional logic to express requirements given in natural language in precise
and unambiguous specifications that can serve as the basis for system development.

\ex{}{
	\qs{}{
		Express the specification "The automated reply cannot be sent when the file system is full" using logical
		connectives
	}

	\sol{

		Let $p$ be "The automated reply can be sent"

		Let $q$ be "The file system is full"

		Therefore the specification can be expressed as
		\[
			q \to \neg p
		\]
	}
}

System specifications should be consistent, and therefore should not contain contradictory requirements.

\pagebreak
% \ex{}{
\qs{}{
	Determine whether these system specifications are consistent:

	“The diagnostic message is stored in the buffer or it is retransmitted.”

	“The diagnostic message is not stored in the buffer.”

	“If the diagnostic message is stored in the buffer, then it is retransmitted.”
}

\sol{

	Let	$p$ be "The diagnostic message is stored in the buffer"

	Let $q$ be "The diagnostic message is retransmitted"

	Therefore the specifications can be expressed as:
	\begin{align*}
		p \vee q \\
		\neg p   \\
		p \to q
	\end{align*}

	To have all these specifications be consistent, it must be possible for them to all be $T$ at the same time.
	In this case making $p = T$ and $q = T$  will result in $\neg p$ being $F$, which would cause the only the first
	and second specifications to pass. But making $p = F$ and $q = T$ would result in all the specifications being
	$T$, as proven by the truth table below.

	\begin{table}[h!]
		\begin{center}
			\begin{tabular}{|c c|c|c|c|c|}
				\hline
				$p$ & $q$ & $p \vee q$ & $\neg p$ & $p \to q$ & $\left( p \vee  q \right) \wedge \neg p \wedge \left( p \to q \right)$ \\ [0.5ex]
				\hline
				\hline
				$T$ & $T$ & $T$        & $F$      & $T$       & $F$                                                                    \\
				$T$ & $F$ & $T$        & $F$      & $F$       & $F$                                                                    \\
				$F$ & $T$ & $T$        & $T$      & $T$       & $T$                                                                    \\
				$F$ & $F$ & $F$        & $T$      & $T$       & $F$                                                                    \\
				\hline
			\end{tabular}
		\end{center}
	\end{table}
}
% }

\qs{}{
	Do the system specifications in the previous example remain consistent if the specification “The diagnostic message is not retransmitted” is added?
}

\sol{
	With the addition of the specification represented logically by $\neg q$ it would make it impossible for all the
	specifications to evaluate to $T$, therefore the specifications would be inconsistent.
}

\subsection{Boolean Searches}

Boolean searches are used to search for information in large collections, such as web page indexes and databases. These
searches usually use the connectives \textit{AND}, \textit{OR}, and \textit{NOT / AND NOT}.

\pagebreak
\subsection{Logic Puzzles}
Puzzles that can be solved using logical reasoning.

\qs{}{
	On an island with two types of people, knights and knaves, knights always tell the truth and knaves always lie.
	You encounter two people \textit{A} and \textit{B}. What are \textit{A} and \textit{B} if \textit{A} says
	"\textit{B} is a knight" and \textit{B} says "The two of us are opposite types"?
}

\sol{

	Let $r$ be "\textit{A} is a knight"

	Let $q$ be "\textit{B} is a knight"

	Let $\neg r$ be "\textit{A} is a knave"

	Let $\neg q$ be "B is a knave"

	First we consider the case where $r = T$. In this case \textit{A} is a knight making everything he says true meaning
	\textit{B} is a knight too. But in \textit{B} being a knight everything he says is also true but that cannot be as in
	this scenario both \textit{A} and \textit{B} are knights, making this scenario invalid.

	Next we consider the case where $\neg r = T$. In this case \textit{A} is a knave meaning he is lying about \textit{B}
	being a knight. Following this we can conclude that \textit{B} is also a knave and everything they also say is a lie,
	which is valid.

	Next we consider the case where $q = T$. In this case \textit{B} is a knight making everything he says true and thus
	\textit{A} is a knave. But as \textit{A} is a knave everything they say is false which calls \textit{B}'s knighthood
	into question, rendering this scenario invalid.

	Finally we consider the case where $\neg q = T$. This is case \textit{B} is a knave making everything they say false.
	This allows us to conclude that \textit{A} and \textit{B} are not of opposite types, and since \textit{A} is believed to
	be a knave further supports \textit{B} being a knave, making this scenario valid.

	Concluding we see in the scenario's we've gone through there are only two valid paths and both point to both \textit{A}
	and \textit{B} being knaves.


}

\qs{}{
	A father tells his two children, a boy and a girl, to play in their backyard without getting dirty.
	However, while playing, both children get mud on their foreheads. When the children stop
	playing, the father says “At least one of you has a muddy forehead,” and then asks the children
	to answer “Yes” or “No” to the question: “Do you know whether you have a muddy forehead?”
	The father asks this question twice. What will the children answer each time this question is
	asked, assuming that a child can see whether his or her sibling has a muddy forehead, but cannot
	see his or her own forehead? Assume that both children are honest and that the children answer
	each question simultaneously.
}

\sol{

	Let $q$ be "boy has muddy forehead"

	Let $r$ be "girl has muddy forehead"

	The first time the question is asked both children answer "No" because as they can see each other's foreheads they
	can tell if the other has a muddy forehead but cannot tell if they do, the boy and girl know $r = T$ but not the
	vale of $q$ and $q = T$ but not the value of $r$.

	The second time the question is asked taking account of the answer of either sibling for the first question and the
	condition that at least one of them has a muddy forehead the children can infer that since they can see mud on their
	sibling's forehead but they answered no but for statement the father made which can be put as $q \vee r$ to be true
	at least one of the propositions have to be true, resoling in them both answering "Yes".
}

\subsection{Logic Circuits}

A logic circuit receives input signals $p_1, p_2\ldots p_n$, each a bit, either $1$ or $0$, and produces output signals
$s_1, s_2\ldots s_n$, each also a bit.

Complicated digital circuits can be constructed from three basic circuits, called \textbf{gates}. These are:

\begin{itemize}
	\item Inverter / NOT gate - Takes an input bit $p$ and produces $\neg p$ as output.
	\item OR gate - Takes two input signals $p$ and $q$ and produces the output signal $p \vee q$.
	\item AND gate - Takes two input signals $p$ and $q$ and produces the output signal $p \wedge q$.
\end{itemize}

Given a circuit built from these gates we can determine the output by tracing through the circuit.

\section{Exercises}

\large{Translate the given statements into propositional logic}
\qs{}{
	You cannot edit a protected Wikipedia entry unless you
	are an administrator. Express your answer in terms of e:
	“You can edit a protected Wikipedia entry” and a: “You
	are an administrator.”
}

\sol{

	$e \to a$
}

\qs{}{
	You can see the movie only if you are over 18 years old
	or you have the permission of a parent. Express your answer in terms of m: “You can see the movie,” e: “You are
	over 18 years old,” and p: “You have the permission of a
	parent.”
}

\sol {
	$m \to  \left( e \vee p \right) $
}


\qs{}{
	You can graduate only if you have completed the requirements of your major and you do not owe money to the
	university and you do not have an overdue library book.
	Express your answer in terms of g: “You can graduate,”
	m: “You owe money to the university,” r: “You have completed the requirements of your major,” and b: “You have
	an overdue library book.”
}

\sol{
	$g \to  \left( r \wedge  \neg m \wedge \neg b \right) $
}

\qs{}{
	To use the wireless network in the airport you must pay
	the daily fee unless you are a subscriber to the service.
	Express your answer in terms of w: “You can use the wire-
	less network in the airport,” d: “You pay the daily fee,”
	and s: “You are a subscriber to the service.”
}

\sol{
	$w \to  \left( d \vee s \right) $
}

\qs{}{
	You can upgrade your operating system only if you have
	a 32-bit processor running at 1 GHz or faster, at least
	1 GB RAM, and 16 GB free hard disk space, or a 64-
	bit processor running at 2 GHz or faster, at least 2 GB
	RAM, and at least 32 GB free hard disk space. Express
	you answer in terms of: “You can upgrade your operating system,” b32: “You have a 32-bit processor,” b64:
	“You have a 64-bit processor,” g1: “Your processor runs
	at 1 GHz or faster,” g2: “Your processor runs at 2 GHz or
	faster,” r1: “Your processor has at least 1 GB RAM,” r2:
	“Your processor has at least 2 GB RAM,” h16: “You have
	at least 16 GB free hard disk space,” and h32: “You have
	at least 32 GB free hard disk space.”
}

\sol{
	$u \to  \left( b_{32} \wedge  g_1 \wedge r_1 \wedge h_{16}  \right) \vee \left( b_{64} \wedge g_2 \wedge r_2
		\wedge h_{32}\right) $
}

\qs{}{
	Construct a truth table for the compound proposition below
	\[
		\left( p \to q \right) \vee 		\left( \neg p \to r \right)
	\]

}

\sol{
	\begin{table}[h!]
		\begin{center}
			\begin{tabular}{|c c c  c|c|c|c|}
				\hline
				$p$ & $q$ & $r$ & $\neg p$ & $ p \to q $ & $\neg p \to r$ & $ \left( p \to q \right) \vee 		\left( \neg p \to r \right) $ \\ [0.5ex]
				\hline
				\hline
				$T$ & $T$ & $T$ & $F$      & $T$         & $T$            & $T$                                                           \\
				$T$ & $T$ & $F$ & $F$      & $T$         & $T$            & $T$                                                           \\
				$T$ & $F$ & $T$ & $F$      & $F$         & $T$            & $T$                                                           \\
				$F$ & $T$ & $T$ & $T$      & $T$         & $T$            & $T$                                                           \\
				$F$ & $F$ & $F$ & $T$      & $T$         & $F$            & $T$                                                           \\
				$F$ & $F$ & $T$ & $T$      & $T$         & $T$            & $T$                                                           \\
				$F$ & $T$ & $T$ & $T$      & $T$         & $T$            & $T$                                                           \\
				$T$ & $F$ & $F$ & $F$      & $F$         & $T$            & $T$                                                           \\
				\hline
			\end{tabular}
		\end{center}
	\end{table}
}

\chapter{Logical Equivalences}

\dfn{Tautology}{
	A proposition that is always $T$.
	e.g. $p \vee  \neg  p$
}

\dfn{Contradiction}{
	A proposition that is always $F$
	e.g. $p \wedge  \neg p$
}

\dfn{Contingency}{
	A proposition that is neither a tautology or a contradiction.
	e.g.. $p$
}

\begin{table}[h!]
	\caption{Demonstration}\label{tab:}
	\begin{center}
		\begin{tabular}{|c c|c|c|}
			\hline
			$p$ & $\neg p$ & $p \vee \neg p$ & $p \wedge \neg p$ \\ [0.5ex]
			\hline
			\hline
			$T$ & $F$      & $T$             & $F$               \\
			$F$ & $T$      & $T$             & $F$               \\
			\hline
		\end{tabular}
	\end{center}
\end{table}

Two compound propositions $p$ and $q$ are logically equivalent if $p \leftrightarrow q$, where $q$ and $p$ are compound propositions.

\section{Equivalence Laws}

\subsection{Logical Equivalences Involving Conditional Statements}

\begin{align*}
	p \to q \equiv \neg p \vee q                                                                  \\
	p \to q \equiv \neg q \to \neg p                                                              \\
	p \vee q \equiv \neg p \to q                                                                  \\
	p \wedge q \equiv \neg \left( p \to \neg q  \right)                                           \\
	\neg  \left( p \to q \right) \equiv p \wedge \neg q                                           \\
	\left( p \to q  \right)  \wedge \left( p \to r \right) \equiv p \to \left( q \wedge r \right) \\
	\left( p \to r \right)  \wedge \left( q \to r \right)  \equiv \left( p \vee q \right) \to r   \\
	\left( p \to q  \right) \vee \left( p \to r \right) \equiv p \to \left( q \vee r \right)      \\
	\left( p \to r \right) \vee \left( q \to r \right) \equiv \left( p \wedge q \right) \to r     \\
\end{align*}

\subsection{Logical Equivalences Involving Biconditional Statements}

\begin{align*}
	p \leftrightarrow q \equiv \left( p\to q \right) \wedge \left( q \to   p \right)              \\
	p \leftrightarrow q \equiv \neg p \leftrightarrow \neg q                                      \\
	p \leftrightarrow q \equiv \left( p \wedge q \right) \vee \left( \neg p \wedge \neg q \right) \\
	\neg \left( p \leftrightarrow q \right) \equiv p \leftrightarrow \neg q                       \\
\end{align*}

\subsection{Identity Laws}

\begin{align*}
	p \wedge T \equiv p \\
	p \vee F  \equiv p
\end{align*}

\subsection{Domination Laws}

\begin{align*}
	p \vee T \equiv T    \\
	p \wedge  F \equiv F \\
\end{align*}

\subsection{Double negation Law}

\[
	\neg \left( \neg p \right) \equiv p
\]

\subsection{Idempotent Laws}

\begin{align*}
	p \vee p  \equiv p  \\
	p \wedge p \equiv p \\
\end{align*}

\subsection{Commutative Laws}

\begin{align*}
	p \vee q \equiv q \vee p \\
	p \wedge q \equiv q \wedge p
\end{align*}

\subsection{Associative Laws}

\begin{align*}
	\left( p \vee q \right)  \vee  r \equiv p \vee \left( q \vee r \right) \\
	\left( p \wedge q \right) \wedge  r \equiv p \wedge \left( q \wedge r \right)
\end{align*}

\subsection{Distributive Laws}

\begin{align*}
	p \vee \left( q \wedge r \right) \equiv \left( p \vee q \right)  \wedge \left( p \vee r \right) \\
	p \wedge \left( q \vee r  \right)  \equiv \left( p \wedge q  \right) \vee \left( p \wedge r \right)
\end{align*}

\subsection{De Morgan's Laws}

\begin{align*}
	\neg \left( p \wedge q \right) \equiv \neg p \vee \neg  q \\
	\neg \left( p \vee q  \right) \equiv \neg p \wedge \neg q
\end{align*}

\subsection{Absorption Laws}

\begin{align*}
	p \vee \left( p \wedge q  \right)  \equiv p \\
	p \wedge \left( p \vee q  \right) \equiv p
\end{align*}

\subsection{Negation Laws}

\begin{align*}
	p \vee \neg p \equiv T \\
	p \wedge \neg p \equiv F
\end{align*}

\section{Exercises}

\ex{}{
	\qs{}{
		Show that $\neg \left( p \vee \left( \neg p \wedge q \right)  \right) $ is logically equivalent to $\neg p \wedge
			\neg q$
	}

	\sol{
		\begin{align*}
			\neg \left( p \vee \left( \neg p \wedge q \right)  \right) & =  \neg p \wedge \neg \left( \neg p \wedge q
			\right) \tag*{By the Second De Morgan's Law}                                                                                     \\
			                                                           & = \neg p \wedge \left( \neg \left(\neg p\right) \vee \neg q \right)
			\tag*{By First De Morgan's Law}                                                                                                  \\
			                                                           & =\neg p \wedge \left( p \vee \neg q \right)
			\tag*{By Double Negation Law}                                                                                                    \\
			                                                           & = \left( \neg p \wedge p   \right) \vee \left(
			\neg p \wedge \neg q \right) \tag*{By Second
			Distributive Law}                                                                                                                \\
			                                                           & = F \vee \left( \neg p \wedge \neg q  \right)
			\tag*{By Second Negation Law}                                                                                                    \\
			                                                           & = \left( \neg p \wedge \neg q  \right) \vee F
			\tag*{By First Commutative Law}                                                                                                  \\
			                                                           & =   \neg p \wedge \neg q   \tag*{By
				Second Identity Law}
		\end{align*}
	}

}

\qs{}{
	Show that
	\[
		(p \wedge q) \to (p \vee q)
	\]
	is a tautology
}

\sol{
	\begin{align*}
		(p \wedge q) \to (p \vee q) & \equiv \neg (p \wedge q ) \vee (p \vee q) \tag*{Conditional Disjunction Equivalence}                       \\
		                            & \equiv (\neg p \vee \neg q) \vee (p \vee q) \tag*{By the First De Morgan's law}                            \\
		                            & \equiv \left( \neg p \vee p \right)  \vee \left( \neg q \vee q \right) \tag*{By the First Associative law} \\
		                            & \equiv \left( p \vee \neg p \right) \vee \left( q \vee \neg q \right) \tag*{By the First Commutative law}  \\
		                            & \equiv \left( p \vee \neg p \right)  \vee T \tag*{By the First Negation Law}                               \\
		                            & \equiv T \vee T \tag*{By the First Negation Law}                                                           \\
		                            & \equiv T \tag*{By the First Domination Law}
	\end{align*}
}

\pagebreak
\qs{}{
	Use logical equivalences to prove that each of these conditional statements is a tautology
	\begin{align}
		\left( p \wedge q \right) \to p              \\
		\neg p \to \left( p \to q\right)             \\
		[\neg p \vee \left( p \vee q \right) ] \to q \\
		[p \wedge \left( p \to q \right) ] \to q
	\end{align}
}

\sol{
	\begin{enumerate}
		\item \begin{align*}
			      \left( p \wedge q \right)  \to p & \equiv \neg \left( p \wedge q \right) \vee p \tag*{Conditional Disjunction Equivalence} \\
			                                       & \equiv \left( \neg p \vee \neg q \right) \vee p \tag*{By the First De Morgan's Law}     \\
			                                       & \equiv \left( \neg p \vee p \right) \vee \neg q \tag*{By the First Associative Law}     \\
			                                       & \equiv \left( p \vee \neg p \right) \vee \neg q \tag*{By the First Commutative Law}     \\
			                                       & \equiv T \vee \neg q \tag*{By the First Negation Law}                                   \\
			                                       & \equiv \neg q \vee  T \tag*{By the First Commutative Law}                               \\
			                                       & \equiv T \tag*{By the First Domination Law}
		      \end{align*}
		\item \begin{align*}
			      \neg p \to  \left( p \to q  \right) & \equiv \neg \left( \neg p \right) \vee \left( p \to q \right)
			      \tag*{Conditional Disjunction Equivalence}                                                                           \\
			                                          & \equiv p \vee \left( p \to q \right) \tag*{By the Double Negation law}         \\
			                                          & \equiv p \vee ( \neg p \vee q)\tag*{Conditional Disjunction Equivalence}       \\
			                                          & \equiv q \vee (\neg p \vee p) \tag*{By the First Associative Law}              \\
			                                          & \equiv q \vee \left( p \vee \neg p \right) \tag*{By the First Commutative Law} \\
			                                          & \equiv q \vee T \tag*{By the First Negation Law}                               \\
			                                          & \equiv T \tag*{By the First Domination Law}                                    \\
		      \end{align*}
		\item \begin{align*}
			      [\neg p \vee \left( p \vee q \right) ] \to q & \equiv \neg \left[ \neg  p \vee \left( p \vee q \right)
			      \right] \vee  q \tag*{Conditional Disjunction Equivalence}                                                                      \\
			                                                   & \equiv \left[ \neg \left( \neg p \right) \wedge \neg
				      \left( p \vee q \right)   \right] \vee  q \tag*{By the Second De
			      Morgan's Law}                                                                                                                   \\
			                                                   & \equiv \left[ p \wedge \neg \left( p \vee q \right)
			      \right]  \vee q \tag*{By the Double Negation Law}                                                                               \\
			                                                   & \equiv \left[ p \wedge \left( \neg p \wedge \neg q
			      \right)   \right] \vee q \tag*{By the Second De Morgan's Law}                                                                   \\
			                                                   & \equiv \left[ \neg q \wedge \left( \neg p \wedge p
				      \right)   \right] \vee q \tag*{By the Second Associative
			      Law}                                                                                                                            \\
			                                                   & \equiv \left[ \neg q \wedge \left( p \wedge \neg p
				      \right)   \right] \vee q \tag*{By the Second Commutative
			      Law}                                                                                                                            \\
			                                                   & \equiv \left[ \neg q \wedge F  \right] \vee  q \tag*{By the Second Negation Law} \\
		      \end{align*}
		\item
		      \begin{align*}
			      [p \wedge \left( p \to q \right) ] \to q & \equiv [p \wedge \left( \neg p \vee q \right) ] \to q
			      \tag*{Conditional Disjunction Equivalence}                                                                                               \\
			                                               & \equiv \neg [p \wedge \left( \neg p \vee q \right) ] \vee q
			      \tag*{Conditional Disjunction Equivalence}                                                                                               \\
			                                               & \equiv [\neg p \vee \neg \left( \neg p \vee q \right) ]
			      \vee q \tag*{By the First De Morgan's Law}                                                                                               \\
			                                               & \equiv [\neg p \vee \left( \neg \left( \neg p \right)
				      \wedge \neg q \right) ] \vee q \tag*{By the Second De
			      Morgan's Law}                                                                                                                            \\
			                                               & \equiv [\neg p \vee \left( p \wedge \neg q \right) ] \vee q \tag*{By the Double Negation Law} \\
			                                               & \equiv [ \left( \neg p \vee p \right) \wedge \left( \neg p
			      \vee \neg q \right)  ] \vee  q \tag*{By the First Distributive Law}                                                                      \\
			                                               & \equiv \left( p \vee \neg p \right) \wedge \left( \neg p
			      \vee \neg q \right) \vee q \tag*{By the First Commutative
			      Law}                                                                                                                                     \\
			                                               & \equiv T \wedge \left( \neg p \vee \neg q \right) \vee q
			      \tag*{By the First Negation Law}                                                                                                         \\
			                                               & \equiv T \wedge (q \vee \neg q) \vee p \tag*{By the First
			      Associative Law and the First Commutative Law}                                                                                           \\
			                                               & \equiv T \wedge  T \vee  p \tag*{By the First Negation Law}                                   \\
			                                               & \equiv T \vee p \tag*{By the First Identity Law}                                              \\
			                                               & \equiv T \tag*{By the First Domination Law}
		      \end{align*}
	\end{enumerate}
}

\qs{}{
	Prove the first Idempotent law
	\[
		p \vee p \equiv p
	\]
}

\sol{
	\begin{align*}
		p & \equiv p \vee F \tag*{By Second Identity Law}                                                       \\
		  & \equiv  p \vee \left( p \wedge \neg p \right) \tag*{By Second Negation Law}                         \\
		  & \equiv \left( p \vee p \right) \wedge \left( p \vee \neg p \right) \tag*{By First Distributive Law} \\
		  & \equiv \left( p \vee p \right) \wedge T \tag*{By First Negation Law}                                \\
		  & \equiv p \vee p \tag*{By First Identity Law}
	\end{align*}
}

\qs{}{
	Prove the first Absorption Law
	\[
		p \vee \left( p \wedge q \right) \equiv p
	\]
}

\sol{
	\begin{align*}
		p & \equiv p \vee p \tag*{By First Idempotent Law}                                                            \\
		  & \equiv p \vee \left( p \wedge T \right) \tag*{By First Identity Law}                                      \\
		  & \equiv p \vee \left( p \wedge \left( q \vee T \right)  \right)  \tag*{By First Domination Law}            \\
		  & \equiv p \vee \left( \left( p \wedge q  \right) \vee \left( p \wedge T \right)    \right) \tag*{By Second
		Distributive Law}                                                                                             \\
		  & \equiv p \vee \left( \left( p \wedge q \right) \vee p  \right) \tag*{By First Identity Law}               \\
		  & \equiv p \vee p \vee  \left( p \wedge q  \right)  \tag*{By First Commutative Law}                         \\
		  & \equiv p \vee \left( p \wedge q \right)  \tag*{By First Idempotent Law}
	\end{align*}
}

\chapter{Predicates and Quantifiers}

Predicate logic uses the following features:

\begin{itemize}
	\item Variables
	\item Predicates
	\item Quantifiers
\end{itemize}



\section{Propositional Function}


\dfn{Propositional Function}{
	Generalization of propositions / A proposition containing variables
}


\noindent A propositional function is made up of three components:
\begin{itemize}
	\item Variables
	\item Subject
	\item Predicate
\end{itemize}

\dfn{Subject}{
	The main variable  in the function
}


\dfn{Predicate}{
	Describes the property or properties of a subject
}


\pagebreak

\ex{}{
	\qs{}{
		\[
			x \text{ is friends with } y
		\]
	}

	\sol{
		In this example the predicate is "is friends with", and the variables are $x$ and $y$ where $x$ is the subject. This
		function can be denoted by:
		\[
			Q(x, y)
		\]

		Where $Q$ is the predicate "is friends with".
	}
}




Propositional function become propositions when their variables are each replaced by a value from the domain ( $\mathbb{U}$ ) or bound by
a quantifier.

The statement $P(x)$ is said to be the value of the propositional function $P$ at $x$

\ex{}{
	\qs{}{
		Let $x + y = z$ be denoted by $R(x,y,z)$ and $\mathbb{U}$ (for all three variables) be integers. Find the truth value of

		\[
			R(2,-1,5)
		\]
	}

	\sol{
		\begin{align*}
			R(2,-1,5)     \\
			x = 2         \\
			y = -1        \\
			z = 5         \\
			2 + -1 \neq 5 \\
			\therefore R(2,-1,5) = F
		\end{align*}
	}

}

\ex{}{
	\qs{}{
		If $P(x)$ denotes $x \ge 0$, find this truth value

		\[
			P(3) \to \neg P(-1)
		\]
	}
}

\sol{
	\begin{align*}
		P(3) = 3 > 0            \\
		P(3) = T                \\ \\
		P(-1) = -1 > 0          \\
		P(-1) = F               \\ \\
		\therefore T \to \neg F \\
		 & = T \to T            \\
		 & = T                  \\
	\end{align*}
}

\ex{}{
	\qs{}{
		Let $A\left(c, n\right)$ denote the statement “Computer $c$ is connected to network $n$,” where $c$ is a variable
		representing a computer and $n$ is a variable representing a network. Suppose that the computer
		MATH1 is connected to network CAMPUS2, but not to network CAMPUS1. What are the
		values of $A\left(\text{MATH1}, \text{CAMPUS1}\right)$ and $A\left(\text{MATH1}, \text{CAMPUS2}\right)$?
	}
	\sol{
		\begin{align*}
			A\left(\text{MATH1}, \text{CAMPUS1}\right) = \text{Computer MATH1 is connected to network CAMPUS1} \\
			A\left(\text{MATH1}, \text{CAMPUS1}\right) & = F                                                   \\ \\
			A\left(\text{MATH1}, \text{CAMPUS2}\right) = \text{Computer MATH1 is connected to network CAMPUS2} \\
			A\left(\text{MATH1}, \text{CAMPUS2}\right) & = T
		\end{align*}
	}

}

\subsection{Preconditions and Postconditions}

Predicates are also used to establish the correctness of computer programs, i.e. Verifying that programs always produce
the desired output when given a valid input.

\dfn{Precondition}{
	Statements that describe valid input for a computer program
}

\dfn{Postcondition}{
	Conditions the output of a computer program should satisfy when run.
}

\ex{}{
	\qs{}{
		Consider this program designed to swap the values of two variables $x$ and $y$.
		\begin{algorithm}[H]
			temp $:= x$\;
			$x := y$\;
			$y :=$ temp\;
		\end{algorithm}
		Find predicates that we can use as the precondition and the postcondition to verify the correctness
		of this program. Then explain how to use them to verify that for all valid input the program does
		what is intended.
	}

	\sol{

		Let $P \left( x, y \right) $ be the statement $x = a \text{ and } y = b$, where $a$ and $b$ are the values of
		$x$ and $y$ before the program is run.

		Let $Q \left( x,y \right) $ be the statement $ x = b \text{ and } y = a$

		To verify this program we need to show that $P \left( x, y \right) \to Q \left( x, y \right) $ is a tautology.


		We can prove this by following the program's execution. First we assume $P \left( x, y \right) = T$. In the
		first step of the program $\text{temp} := x$ assigns the value of $x$ to temp, so at this point the state of the
		program is $x := a, \text{temp}:= a, y:= b$. In the second step of the program $x := y$ assigns the value of
		$y$ to $x$, so the state is now $ x := b, y := b, \text{temp} := a$. And the final step of the program $y :=
			\text{temp}$ assigns the value of temp to $y$, so the final state is $x := b, y := a, \text{temp} := a$, making
		$Q \left( x, y \right) = T $
	}

}


\section{Quantifiers}

\dfn{Quantifiers}{
	Used to express the meaning of English words such as "all", "some", "none", "there exists", "for one and only one", etc.
}

\dfn{Domain of Discourse / Universe of Discourse / Domain}{
	The range of all values a variable can take. Denoted as $\mathbb{U}$. Without a specified domain we cannot determine the
	truth value of a Propositional Function.
}


\subsection{Universal Quantifier}
\dfn{Universal Quantifier}{
	"For all": $\forall$

	Meaning for every variable in the domain ($\mathbb{U}$), the propositional function is true.

	\ex{}{
		\[
			\forall x P(x)
		\]

		Asserts $P(x)$ is true for every x in $\mathbb{U}$

		So if $P(x): x \text{ has stripes}$, and $x$ is a zebra then $\forall x P(x)$ means "Every zebra has stripes" or
		"All zebras have stripes"
	}

	The Universal Quantifier is equivalent to the conjunction of the propositional function for all values in its
	specified domain. i.e. Where $P \left( x \right) $ is the statement $x > 0$ and $\mathbb{U}$ is all positive real numbers
	\[
		\forall x P \left( x \right) \equiv P \left( x_{1} \right) \wedge P \left( x_2 \right) \wedge P \left( x_3 \right)
		\wedge \dots \wedge  P \left( x_{n} \right)
	\]
}

\ex{}{
	\begin{enumerate}
		\item If $P(x)$ denotes $x > 0$ and $\mathbb{U}$ is the integers then, $\forall x P(x)$ is false.
		\item If $P(x)$ denotes $x > 0$ and $\mathbb{U}$ is the positive integers, then $\forall x P(x)$ is true
	\end{enumerate}
}

\ex{}{
	\qs{}{
		What is the truth value if $\forall x P \left( x \right) $, where $P \left( x \right) $ is the statement $x^2 < 10$
		and the domain consists of the positive integers not exceeding 4.
	}

	\sol{
		\begin{align*}
			\forall x P \left( x \right) \equiv P \left( 1 \right) \wedge P \left( 2 \right)  \wedge P \left( 3 \right)
			\wedge P \left( 4 \right)                      \\
			\\
			P \left( 4 \right) = (4)^2 < 10                \\
			           & = 16 < 10                         \\
			\therefore P \left( 4 \right) = F              \\
			\therefore & \forall x P \left( x \right)  = F
		\end{align*}
	}
}

\subsection{Existential Quantifier}
\dfn{Existential Quantifier}{
	"There exists": $\exists$

	Meaning for some variables in $\mathbb{U}$, the propositional function is true.

	\ex{}{
		\[
			\exists xP(x)
		\]
		Asserts $P(x)$ is true for some $x$ in $\mathbb{U}$
	}

	The Existential Quantifier is equivalent to the disjunction of the propositional function for all values in its
	specified domain. I.e. Where $P \left( x \right) $ is the statement $x > 5$ and $\mathbb{U}$ is all positive real numbers
	\[
		\exists x P \left( x \right) \equiv P \left( x_1 \right) \vee P \left( x_2 \right)  \vee P \left( x_3 \right) \vee
		\ldots P \left( x_{n} \right)
	\]

}

\ex{}{
	\begin{enumerate}
		\item If $P(x)$ denotes $x > 0$ and $\mathbb{U}$ is the integers, then $\exists x P\left(x \right) $ is true. It is also
		      true if $\mathbb{U}$ is the positive integers
		\item If $P(x)$ denotes $x < 0$ is the positive integers, then $\exists x P \left( x \right)  $ is false
	\end{enumerate}
}

\subsection{Uniqueness Quantifier}
\dfn{Uniqueness Quantifier}{
	"There exists one and only one": $\exists !$
	\ex{}{
		\[
			\exists ! P \left( x \right)
		\]
		Asserts $P \left( x \right) $ is only true for one $x$ in $\mathbb{U}$
	}
}

\subsection{Quantifiers with Restricted Domains}

The notation for a restricted domain is in the form:

\begin{center}
	QUANTIFIER VARIABLE "CONDITION A VARIABLE MUST SATISFY"
\end{center}

\ex{}{
	\begin{enumerate}
		\item $\forall x < 0 \left( x^2 > 0 \right) $
		\item $\forall y \neq 0 \left( y^{3} \neq 0 \right) $
		\item $\exists z > 0 \left( z^2 = 2 \right) $
	\end{enumerate}

	\noindent Means:

	\begin{enumerate}
		\item The square of all negative integers is positive, or $\forall x \left( x < 0 \to x^2 > 0 \right) $
		\item The cube of every nonzero number is nonzero, or $\forall y \left( y \neq 0 \to y^3 \neq 0 \right) $
		\item There is a positive square root of 2, or $\exists z \left( z > 0 \to z^2 = 2 \right) $
	\end{enumerate}

}

\nt{
	The restriction of a universal quantifier is the same as the universal quantification of a conditional statement,
	i.e. $\forall x < 0 \left( x^2 > 0 \right) $ is just s way of expressing $\forall x \left( x < 0 \to x^2 > 0 \right) $
}

\subsection{Binding Variables}

\dfn{Binding}{
	The process of assigning a value to a variable, through quantification or value assignment.
}

\dfn{Bound Variable}{
	A variable that has been assigned a value through quantification or value assignment.
}

\dfn{Free Variable}{
	A variable that has not been assigned a value through quantification or value assignment.
}

\dfn{Scope of a Quantifier}{
	The part of a statement that is affected by a quantifier.
}

\ex{}{
	In the statement $\exists x \left( x + y = 1 \right) $, $x$ is bound by the existential quantifier $\exists x$ and
	$y$ is free.\\

	In the statement $\exists x \left( P \left( x \right) \wedge Q \left( x \right)   \right) \vee  \forall x R \left( x \right)  $
	all variables are bound. The scope of the first quantifier $\exists x$ is the expression
	$P \left( x \right) \wedge Q \left( x \right)  $ and the scope of the second quantifier $\forall x$ is the expression $R \left( x \right)  $.

}

\subsection{Logical Equivalences Involving Quantifiers}

\dfn{Logically Equivalent}{
	When two statements have the same truth value no matter which predicates are substituted into these statements and
	which the domain $\mathbb{U}$ is used for the variables in these propositional functions. The notation $S \equiv T$ is used
	to indicate that two statements $S$ and $T$ that have predicates are logically equivalent.
}

\pagebreak
\subsection{Negating Quantifications}

Consider $\forall x F \left( x \right) $, where $F \left( x \right) $ is " $x$ has taken a course in Java " and $\mathbb{U}$ is
students in your class.

The negation of this follows

\begin{align*}
	\neg  \forall x F \left( x \right) & \equiv \exists x \neg F \left( x \right) \tag*{By De Morgan's Law}  \\
	\neg \exists x F \left( x \right)  & \equiv \forall x \neg F \left( x \right)  \tag*{By De Morgan's Law}
\end{align*}


\subsection{Translating from English to Logical Expressions}

\ex{}{
	\qs{}{
		Express the statement "Every student in this class has studied calculus" using predicates and quantifiers
	}

	\sol{

		\noindent Let $\mathbb{U}$ be "students in this class"\\
		Let $C \left( x \right) $ be "$x$ has studied calculus"\\

		\[
			\forall x C \left( x \right)
		\]

		Or\\ \\
		\noindent Let $\mathbb{U}$ be "all people"\\
		Let $S \left( x \right) $ be "$x$ is a student in this class"
		Let $C \left( x \right) $ be "$x$ has studied calculus"

		\[
			\forall x \left( S \left( x \right) \to  C \left( x \right)   \right)
		\]


	}
}

\pagebreak
\section{Nested Quantifiers}

\dfn{Nested Quantifier}{
	A situation where one quantifier is within the scope of another.
}

\[
	\forall x \exists y \left( x + y = 0 \right)
\]

$ \left( x + y = 0 \right)  $ can be made the propositional function $P \left( x, y \right) $ giving us

\[
	\forall x \exists y P \left( x, y \right)
\]
We can then make $\exists y P \left( x, y \right) $ another propositional function $Q \left( x, y \right) $ giving us

\[
	\forall x Q \left( x, y \right)
\]

This statement can be written as For all $x$ there exists some $y$ where $x + y = 0$

\noindent $\forall x \forall y P \left( x, y \right) $ where $\mathbb{U}_{x} = \left\{ 0, 2, 3 \right\} $ and $\mathbb{U}_{y} =
	\left\{ 1,3 \right\} $ can be expressed as
\[
	P \left( 0,1 \right)  \wedge P \left( 0,3 \right) \wedge P \left( 2,1 \right)  \wedge P \left( 2,3 \right)  \wedge P
	\left( 3,1 \right)  \wedge P \left( 3,3 \right)
\]

\noindent $\exists y \forall x P \left( x,y \right) $ using the same domains can be expressed as
\begin{align*}
	\exists y \forall x P \left( x,y \right) \equiv \forall x P \left( x,1 \right)  \vee \forall x P \left( x,3
	\right)                                                                                         \\
	\forall x \left[ P \left( x, 1 \right) \wedge  P \left( x, 1 \right) \wedge  P \left( x,1 \right)    \right]  \vee
	\left[ P \left( x,3 \right) \wedge  P \left( x,3 \right) \wedge P \left( x,3 \right)    \right] \\
	\left[ P \left( 0,1 \right) \wedge  P \left( 2,1 \right) \wedge P \left( 3,1 \right)    \right]  \vee
	\left[ P \left( 0,3 \right) \wedge  P \left( 2,3 \right) \wedge  P \left( 3,3 \right)    \right]
\end{align*}


Or
\begin{align*}
	\exists y \forall x P \left( x,y \right) \equiv \forall x P \left( x,1 \right)  \vee \forall x P \left( x,3 \right)
	\\
	\exists y \left( P \left( 0, y \right) \wedge P \left( 2, y \right) \wedge P \left( 3, y \right)    \right) \\
	\left[ P \left( 0,1 \right) \wedge P \left( 2, 1 \right) \wedge P \left( 3, 1 \right)  \right] \vee \left[ P \left(
		0,3\right) \wedge P \left( 2,3 \right) \wedge P \left( 3,3 \right)     \right]
\end{align*}

\subsection{Order of Quantifiers}

The order of quantifiers is important when all quantifiers are not of the same type, i.e. $\forall x \exists y P \left( x, y \right) $ is not the same as $\exists y \forall x P \left( x, y \right) $

\subsection{Negating Nested Quantifiers}

\begin{align*}
	\neg  \forall x \exists y P \left( x, y \right) \equiv \exists x \forall y \neg P \left( x, y \right) \tag*{
		By De Morgan's Law
	}
\end{align*}

\subsection{Translating Mathematical Statements into Statements Involving Nested Quantifiers}

\ex{}{
	\qs{}{
		Translate the statement "The sum of two positive integers is always positive" into a logical
		expression.
	}

	\sol{
		\begin{align*}
			\mathbb{U}_{x,y} \text{: All integers} \\
			\\
			\forall x \forall y \left( \left( \left( x > 0 \right) \wedge \left( y > 0 \right)  \right) \to  \left( x +
				y > 0\right)  \right)
		\end{align*}
	}
}

\ex{}{
	\qs{}{
		Use quantifiers to express the definition of the limit of a real-valued
		function $f \left( x \right) $ of a real variable $x$ at a point $a$ in its domain.
	}

	\sol{

	}
}

\subsection{Translating from Nested Quantifiers into English}

\ex{}{
	\qs{}{
		Translate this statement
		\[
			\forall x \left( C \left( x \right) \vee \exists y \left( C \left( y \right) \wedge  F \left( x,y \right)   \right)   \right)
		\]
		\noindent Into English where $C \left( x \right) $ is "$x$ has a computer", $F \left( x,y \right) $ is "$x$ and
		$y$ are friends," and $\mathbb{U}$ for both $x$ and $y$ consists of all students in your school.
	}

	\sol{

		First we write the statement verbosely:\\

		For all students in my school, $x$, $x$ has a computer or There exists some student in my school, $y$, where $x$
		and $y$ are friends.\\

		\noindent This can be simplified to:\\

		Every student in my school has a computer or has a friend who who has a computer.
	}
}

\ex{}{
	\qs{}{
		Translate this statement
		\[
			\exists x \forall y \forall z \left( \left( F \left( x, y \right) \wedge F \left( x, z \right) \wedge \left( y
			\neq z \right)    \right) \to \neg F \left( y,z \right)   \right)
		\]
		into English, where $F \left( x,y \right) $ $x$ and $y$ are friends and $\mathbb{U}$ for $x$, $y$ and $z$
		consists of all students in your school.
	}

	\sol{

		There exists some student $x$, For all students $y$ and $z$, If $x$ and $y$ are friends and $x$ and $z$ are
		friends and $y \neq z$, then $y$ and $z$ are not friends.\\

		There is a student in my school who has no friends who are friends with each other.

	}
}

\subsection{Translating English sentences into Logical Expressions}


\ex{}{
	\qs{}{
		Express the statement "If a person is female and is a parent, then this person is someone’s
		mother" as a logical expression involving predicates, quantifiers with a domain consisting of all
		people, and logical connectives.
	}

	\sol{
		\begin{align*}
			\mathbb{U}_{x,y} \text{: All people}              \\
			\\
			F \left( x \right) \text{: $x$ is female}         \\
			P \left( x \right) \text{: $x$ is a parent}       \\
			M \left( x,y \right) \text{: $x$ is $y$'s mother} \\
			\\
			\forall x \left( \left( P \left( x \right) \wedge F \left( x \right)   \right) \to  \exists y M \left( x,y \right)   \right)
		\end{align*}
	}
}

\ex{}{
	\qs{}{
		Express the statement "Everyone has exactly one best friend" as a logical expression involving
		predicates, quantifiers with a domain consisting of all people, and logical connectives.
	}

	\sol{
		\begin{align*}
			\mathbb{U}_{x,y} \text{: All people}                   \\
			\\
			B \left( x,y \right) \text{: $x$ is $y$'s best friend} \\ \\
			\forall x \exists y \left( B \left( x,y \right) \wedge \forall z \left( \left( y \neq z \right) \to \neg B
			\left( x, z \right)  \right)    \right)                \\
			\text{Or }
			\forall x \exists! y B \left( x, y \right)
		\end{align*}
	}
}

\ex{}{
	\qs{}{
		Use quantifiers to express the statement "There is a woman who has taken a flight on every
		airline in the world."
	}

	\sol{

		\begin{align*}
			\mathbb{U}_{w} \text{: All women}                     \\
			\mathbb{U}_{a} \text{: All airlines}                  \\
			\mathbb{U}_{f} \text{: All flights}                   \\
			\\
			F \left( w, f \right) \text{: $w$ has taken $f$ }     \\
			Q \left( f, a \right): \text{ $f$ is a flight on $a$} \\
			\\
			\exists w \forall a \exists f \left( F \left( w, f \right) \wedge  Q \left( f, a \right)  \right)
		\end{align*}
	}
}


\section{Exercises}

\ex{}{
	\qs{}{
		Translate these statements into English, where $C \left( x \right) $ is " $x$ is a comedian " and $F \left( x
			\right) $ is " $x$ is funny " and the domain consists of all people.
		\begin{enumerate}
			\item $\forall x \left( C \left( x \right) \to F \left( x \right)   \right) $
			\item $\forall x \left( C \left( x \right) \wedge F \left( x \right)   \right) $
			\item $\exists x \left( C \left( x \right) \to F \left( x \right)   \right) $
			\item $\exists x \left( C \left( x \right) \wedge F \left( x \right)   \right) $
		\end{enumerate}

	}

	\sol{
		\begin{enumerate}
			\item Everyone that is a comedian is funny.
			\item All people are comedians and they are funny.
			\item There exits some people, If they are comedians, then they are funny.
			\item There exists some people that are comedians and are funny.
		\end{enumerate}
	}
}

\ex{}{
	\qs{}{
		Determine the truth value of each of these statements if the domain of each variable consists of all real numbers.
		\begin{enumerate}
			\item $\exists x \left( x^2 = 2 \right) $
			\item $\exists x \left( x^2 = -1 \right) $
			\item $\forall x \left( x^2 + 2 \geq 1 \right) $
			\item $\forall  x \left( x^2 = x \right) $
		\end{enumerate}
	}

	\sol{
		\begin{enumerate}
			\item $T$
			\item $F$
			\item $T$
			\item $F$
		\end{enumerate}
	}
}

\ex{}{
	\qs{}{

		Let $R \left( x \right) $ be $x$ is a rabbit

		Let $H \left( x \right) $ be $x$ hops

		Let $\mathbb{U}$ consists of all animals

		\begin{enumerate}
			\item $\forall x \left( R \left( x \right) \to H \left( x \right)   \right) $
			\item $\forall x \left( R \left( x \right) \wedge H \left( x \right)   \right) $
			\item $\exists x \left( R \left( x \right) \to H \left( x \right)   \right) $
			\item $\exists x \left( R \left( x \right) \wedge H \left( x \right)   \right) $
		\end{enumerate}
	}

	\sol{
		\begin{enumerate}
			\item For all animals, if they are a rabbit then they hop.
			\item All animals are rabbits and they hop.
			\item There exists some animals, if they are a rabbit then they hop.
			\item There exists some animals that are rabbits and hop.
		\end{enumerate}
	}
}

\pagebreak
\qs{}{
	Let $C(x)$ be the statement " $x$ has a cat," let $D(x)$ be the
	statement "$x$ has a dog," and let $F(x)$ be the statement "$x$
	has a ferret." Express each of these statements in terms of
	$C(x)$, $D(x)$, $F(x)$, quantifiers, and logical connectives.
	Let the domain consist of all students in your class.

	\begin{enumerate}
		\item A student in your class has a cat, a dog, and a ferret.
		\item All students in your class have a cat, a dog, or a ferret.
		\item Some student in your class has a cat and a ferret, but not a dog.
		\item No student in your class has a cat, a dog, and a ferret.
		\item For each of the three animals, cats, dogs, and ferrets, there is a student in your class who has this animal as a pet.
	\end{enumerate}
}

\sol{
	\begin{enumerate}
		\item $\exists x \left( C \left( x \right) \wedge D \left( x \right) \wedge F \left( x \right)    \right) $
		\item $\forall x \left( C \left( x \right) \wedge D \left( x \right) \wedge F \left( x \right)  \right) $
		\item $\exists x \left( C \left( x \right) \wedge F \left( x \right) \wedge \neg D \left( x \right)    \right) $
		\item $\neg \exists x \left( C \left( x \right) \wedge D \left( x \right) \wedge  F \left( x \right)    \right) $
		\item $ \left(    \exists x C \left( x \right) \right) \wedge \left( \exists x D \left( x \right) \right) \wedge  \left( \exists x F \left( x \right) \right)    $
	\end{enumerate}

}

\qs{}{
	Determine the truth value of each of these statements if
	the domain consists of all integers.
	\begin{enumerate}
		\item $\forall n \left( n + 1 > n  \right) $
		\item $\exists n \left( 2n = 3n \right) $
		\item $\exists n \left( n = -n\right) $
		\item $\forall n \left( 3n \leq 4n \right) $
	\end{enumerate}
}

\sol{
	\begin{enumerate}
		\item $T$. Because any $1 +$ any integer is greater than that integer.
		\item $T$. $n = 0$, $2 \left( 0 \right) = 3 \left( 0 \right)  $
		\item $T$. $n = 0$, $0 = -0$, $0 = 0$
		\item $T$. $n = 0$, $3 \left( 0 \right) \leq 4 \left( 0 \right)  $
	\end{enumerate}
}

\pagebreak
\qs{}{
	Suppose that the domain of the propositional function
	P(x) consists of the integers 1, 2, 3, 4, and 5. Express
	these statements without using quantifiers, instead using
	only negations, disjunctions, and conjunctions.
	\begin{enumerate}
		\item $\exists x P \left( x \right) $
		\item $\forall x P \left( x \right) $
		\item $\neg \exists x P \left( x \right) $
		\item $\neg  \forall x P \left( x \right) $
		\item $\forall x \left( \left( x \neq 3 \right) \to P \left( x \right)   \right) \vee \exists x \neg P \left( x \right)  $
	\end{enumerate}
}

\sol{
	\begin{enumerate}
		\item $P \left( 1 \right) \vee P \left( 2 \right) \vee P \left( 2 \right) \vee P \left( 3 \right) \vee P \left( 4\right) \vee P \left( 5 \right)      $
		\item $P \left( 1 \right) \wedge P \left( 2 \right) \wedge P \left( 3 \right) \wedge  P \left( 4 \right) \wedge P \left( 5 \right) $
		\item $\forall x \neg P \left( x \right) \therefore $ $\neg P \left( 1 \right) \wedge \neg P \left( 2 \right) \wedge  \neg P \left( 3 \right) \wedge  \neg P \left( 4 \right) \wedge \neg P \left( 5 \right)      $
		\item $\exists x \neg  P \left( x \right) \therefore $ $\neg P \left( 1 \right) \vee \neg P \left( 2 \right) \vee \neg P \left( 3 \right) \wedge \neg P \left( 4 \right) \wedge \neg  P \left( 5 \right)       $
		\item $ \left( P \left( 1 \right) \wedge  P \left( 2 \right) \wedge P \left( 4 \right) \wedge P \left( 5 \right)     \right)   \vee  \left( \neg P \left( 1 \right) \vee \neg  P \left( 2 \right) \vee \neg  P \left( 3 \right) \vee \neg  P \left( 4 \right) \vee  \neg P \left( 5 \right)       \right) $
	\end{enumerate}
}

\qs{}{
	Translate in two ways each of the statements into logical expressions using predicates, quantifiers, and logical
	connectives. First let the domain consist of the students in your class, and second let the domain consist of all people.
	\begin{enumerate}
		\item Everyone in your class has a cellular phone.
		\item Somebody in your class has seen a foreign movie.
		\item There is a person in your class who cannot swim.
		\item All students in your class can solve quadratic equations.
	\end{enumerate}
}

\sol{
	\begin{enumerate}
		\item
		      \begin{align*}
			      \mathbb{U} = \text{ students in your class}                   \\
			      C \left( x \right)  \text{ be "$x$ has a cellular phone"}     \\
			      \forall x C \left( x \right)                                  \\ \\
			      \mathbb{U} = \text{ all people}                               \\
			      S \left( x \right) \text{be "$x$ is a student in your class"} \\
			      C \left( x \right) \text{ be "$x$ has a cellular phone"}      \\
			      \forall x \left( S \left( x \right) \to C \left( x \right)   \right)
		      \end{align*}
		\item
		      \begin{align*}
			      \mathbb{U} = \text{ students in your class}                     \\
			      F \left( x \right) \text{ be "$x$ has seen a foreign movie"}    \\
			      \exists x F \left( x \right)                                    \\ \\
			      \mathbb{U} = \text{ all people}                                 \\
			      S \left( x \right)  \text{ be "$x$ is a student in your class"} \\
			      F \left( x \right) \text{ be "$x$ has seen a foreign movie"}    \\
			      \exists x \left( S \left( x \right) \to F \left( x \right)   \right)
		      \end{align*}
		\item
		      \begin{align*}
			      \mathbb{U} = \text{ students in your class}                     \\
			      S \left( x \right)  \text{ be "$x$ can swim"}                   \\
			      \exists x \neg S \left( x \right)                               \\
			      \\
			      \mathbb{U} = \text{ all people}                                 \\
			      S \left( x \right)  \text{ be "$x$ is a student in your class"} \\
			      W \left( x \right) \text{ be "$x$ can swim"}                    \\
			      \exists x \left( S \left( x \right) \to \neg W \left( x \right)  \right)
		      \end{align*}
		\item
		      \begin{align*}
			      \mathbb{U} = \text{ students in your class}                          \\
			      Q \left( x \right) \text{ be "$x$ can solve quadratic equations" }   \\
			      \forall x Q \left( x \right)                                         \\
			      \\
			      \mathbb{U} = \text{ all people}                                      \\
			      S \left( x \right)  \text{ be "$x$ is a student in your class"}      \\
			      Q \left( x \right) \text{ be "$x$ can solve quadratic equations" }   \\
			      \forall x \left( S \left( x \right) \to Q \left( x \right)   \right) \\
		      \end{align*}
	\end{enumerate}
}

\pagebreak
\qs{}{
	Express each of these statements using quantifiers. Then form the negation of the statement so
	that no negation is to the left of a quantifier. Next, express the negation in simple English. (Do
	not simply use the phrase “It is not the case that.”)
	\begin{enumerate}
		\item There is a horse that can add.
		\item Every koala can climb.
		\item No monkey can speak French.
		\item There exists a pig that can swim and catch fish.
	\end{enumerate}
}

\sol{
	\begin{enumerate}
		\item
		      \begin{align*}
			      \mathbb{U} = \text{ all horses}             \\
			      A \left( x \right) \text{ be "$x$ can add"} \\
			      \exists x A \left( x \right)                \\ \\
			      \neg \exists x A \left( x \right) \equiv \forall x \neg A \left( x \right) \tag*{By De Morgan's Law of
			      Quantifiers}                                \\
			       & \text{All horses cannot add}
		      \end{align*}
		\item
		      \begin{align*}
			      \mathbb{U} = \text{ all koalas}                  \\
			      C \left( x \right) \text{ be "$x$ can climb"}    \\
			      \forall x C \left( x \right)                     \\
			      \neg \forall x C \left( x \right) \equiv \exists x \neg C \left( x \right) \tag*{By De Morgan's Law for
			      Quantifiers}                                     \\
			       & \text{There exists a koala that cannot climb}
		      \end{align*}
		\item
		      \begin{align*}
			      \mathbb{U} = \text{ all monkeys}                     \\
			      F \left( x \right) \text{ be "$x$ can speak French"} \\
			      \neg \exists x F \left( x \right)                    \\
			      \neg \left( \neg \exists x F \left( x \right) \right)  \equiv \exists  x F \left( x \right)  \tag*{By De Morgan's Law
			      for Quantifiers}                                     \\
			       & \text{All monkeys cannot speak French}
		      \end{align*}
		\item
		      \begin{align*}
			      \mathbb{U} = \text{ all pigs}                                                                          \\
			      S \left( x \right) \text{ be "$x$ can swim"}                                                           \\
			      C \left( x \right) \text{ be "$x$ can catch fish"}                                                     \\
			      \exists x \left( S \left( x \right) \wedge C \left( x \right)   \right)                                \\
			      \neg \exists x \left( S \left( x \right) \wedge C \left( x \right)   \right) & \equiv \forall x \neg
			      \left( S \left( x \right) \wedge  C \left( x \right)   \right)
			      \tag*{By De Morgan's Law for Quantifiers}                                                              \\
			                                                                                   & \equiv \forall x \left(
			      \neg S \left( x \right)
			      \vee \neg C \left( x \right)  \right)
			      \tag*{By the First De Morgan's Law}                                                                    \\
			                                                                                   & \text{All pigs cannot
				      swim or catch fish}
		      \end{align*}
	\end{enumerate}
}

\qs{}{
	Assume that the universe for $x$ is all people and the universe for $y$ is the set of all movies. Write
	the English statement using the following predicates and any needed quantifiers: \\
	$S \left( x, y \right) $: $x$ saw $y$\\
	$L \left( x, y \right) $: $x$ liked $y$\\
	$A \left( y \right) $: $y$ won an award\\
	$C \left( y \right) $: $y$ is a comedy\\

	\begin{enumerate}
		\item No comedy won an award.
		\item Lois saw Casablanca but did not like it.
		\item Some people have seen every comedy.
		\item No one liked every movie he has seen.
		\item Ben has never seen a movie that won an award.
	\end{enumerate}
}

\sol{
	\begin{enumerate}
		\item $\neg \exists y \left( C \left( y \right) \wedge  A \left( y \right)   \right) $
		\item $ S \left( \text{Lois}, \text{Casablanca} \right) \wedge \neg L \left( \text{Lois}, \text{Casablanca} \right) $
		\item $\exists x \forall y \left( C \left( y \right) \to S \left( x, y \right)    \right) $
		\item $\forall  x \exists  y \left( S \left( x,y \right) \to \neg L \left( x,y \right)   \right) $
		\item $\forall y \left( A \left( y \right) \to \neg S \left( \text{Ben}, y \right)   \right) $
	\end{enumerate}
}


\chapter{Rules of Inference}

\dfn{Argument}{
	A sequence of statements that end with a conclusion.
}

\dfn{Valid Argument}{
	An argument is valid if the conclusion follows logically from the preceding statements. An argument form is valid no
	matter which specific propositions are substituted into its propositional variables, and the conclusion is true if
	all the premises are true.

}

\section{Valid Arguments in Propositional Logic}

Given the statements.\\

"If you have  a current password, then you can log onto the network."\\

"You have a current password" \\

Therefore, \\

"You can log onto the network"\\

We can express this argument as\\

\begin{math}
	\begin{deduction}
		\premise{p \to q}
		\premise{p}
		\conclusion{q}
	\end{deduction}
\end{math}

Where $p$ is "You have a current password" and $q$ is "You can log onto the network"



\section{Rules of Inference}

\subsection{Modus Ponens}

\begin{deduction}
	\premise{p}
	\premise{p \to q}
	\conclusion{q}
\end{deduction}

\subsection{Modus Tollens}

\begin{math}
	\begin{deduction}
		\premise{ \neg q }
		\premise{ p \to q }
		\conclusion{ \neg p }
	\end{deduction}
\end{math}


\subsection{Hypothetical Syllogism}

\begin{math}
	\begin{deduction}
		\premise{ p \to q }
		\premise{ q \to r }
		\conclusion{ p \to  r}
	\end{deduction}
\end{math}


\subsection{Disjunctive Syllogism}
\begin{math}
	\begin{deduction}
		\premise{ p \vee q }
		\premise{ \neg  p }
		\conclusion{ q }
	\end{deduction}
\end{math}

\subsection{Addition}
\begin{math}
	\begin{deduction}
		\premise{ p }
		\conclusion{ p \vee  q }
	\end{deduction}
\end{math}

\subsection{Simplification}
\begin{math}
	\begin{deduction}
		\premise{ p \wedge q }
		\conclusion{ p }
	\end{deduction}
\end{math}

\subsection{Conjunction}
\begin{math}
	\begin{deduction}
		\premise{ p }
		\premise{ q }
		\conclusion{ p \wedge q }
	\end{deduction}
\end{math}

\subsection{Resolution}
\begin{math}
	\begin{deduction}
		\premise{ p \vee q }
		\premise{ \neg p \vee r }
		\conclusion{ q \vee r }
	\end{deduction}
\end{math}

\section{Using Rules of Inference to Build Arguments}

When there are many premises several rules of inference may be needed to derive a conclusion. \\

For example, given the premises "It is not sunny this afternoon and it is colder than yesterday", "We will go swimming only if it is sunny",
"If we do not go swimming, then we will take a canoe trip", and "If we take a canoe trip, then we will be home by
sunset" showing that these premises lead to the conclusion "We will be home by sunset", will require several
applications of the rules of inference.\\

We solve this by first translating these premises into an argument form:

\noindent Let $p$ be "It is sunny this afternoon"\\

\noindent Let $q$ be "It is colder than yesterday"\\

\noindent Let $r$ be "We will go swimming"\\

\noindent Let $s$ be "We will take a canoe trip"\\

\noindent Let $t$ be "We will be home by sunset"\\

\begin{math}
	\begin{deduction}
		\premise{ \neg p \wedge q }
		\premise{ r \to p }
		\premise{ \neg r \to s }
		\premise{ s \to t }
		\conclusion{ t }
	\end{deduction}
\end{math}

We then construct the argument using the rules of inference.\\

\gdef\rownumber{\stepcounter{magicrownumbers}\arabic{magicrownumbers}}
\begin{table}[h!]
	\begin{center}
		\begin{tabular}{ | @{\makebox[3em][r]{\rownumber\space}} | c | c | }
			\hline
			\multicolumn{1}{ | @{\makebox[3em][r]{~}} |c| }{Steps} & \multicolumn{1}{|c|}{Reasons} \\
			\hline
			\hline
			$\neg p \wedge q$                                      & Premise 1                     \\
			$r \to p$                                              & Premise 2                     \\
			$\neg r \to s$                                         & Premise 3                     \\
			$s \to t$                                              & Premise 4                     \\
			$\neg p$                                               & By Simplification from 1      \\
			$\neg r$                                               & By Modus Tollens from 2 and 5 \\
			$s$                                                    & By Modus Ponens from 3 and 6  \\
			$t$                                                    & By Modus Ponens from 4 and 7  \\
			\hline
		\end{tabular}
	\end{center}
\end{table}

\section{Rules of Inference for Quantified Statements}

\subsection{Universal Instantiation}

This rule states that $P \left( c \right) $ is $T$, given the premise that $\forall x P \left( x \right) $ is $T$ and
$c$ is an arbitrary element in the domain.

\begin{align*}
	\begin{deduction}
		\premise{\forall x P \left( x \right) }
		\conclusion{P \left( c \right) }
	\end{deduction}
\end{align*}

\subsection{Universal Generalization}

This rule states that $\forall x P \left( x \right) $ is $T$, given the premise that $P \left( c \right) $ is $T$  for
all elements $c$ in the domain.


\begin{align*}
	\begin{deduction}
		\premise{P \left( c \right) \text{ for an arbitrary } c }
		\conclusion{ \forall x P \left( x \right)  }
	\end{deduction}
\end{align*}

\subsection{Existential Instantiation}

This rule allows us to conclude that there exists an element $c$ for which $P \left( c \right) $ is $T$ if $\exists x P
	\left( x \right) $ is $T$. This $c$ is not an arbitrary element, but one for which $P \left( c \right) $ is $T$.

\begin{align*}
	\begin{deduction}
		\premise{\exists x P \left( x \right) }
		\conclusion{P \left( c \right) \text{ for some element } c }
	\end{deduction}
\end{align*}

\subsection{Existential Generalization}

This rule is used to conclude that $\exists x P \left( x \right) $ is $T$  when a particular element $c$ is such that
$P \left( c\right) = T $

\begin{align*}
	\begin{deduction}
		\premise{P \left( c \right) \text{ for some element } c }
		\conclusion{\exists x P \left( x \right) }
	\end{deduction}
\end{align*}

\begin{align*}
	\begin{deduction}
		\premise{}
	\end{deduction}
\end{align*}

\subsection{Universal Modus Ponens}

\begin{math}
	\begin{deduction}
		\premise{ \forall x \left( P \left( x \right) \to Q \left( x \right)   \right)  }
		\premise{ P \left( a \right) \text{, where $a$ is a particular element in the domain}  }
		\conclusion{ Q \left( a \right)  }
	\end{deduction}
\end{math}

\subsection{Universal Modus Tollens}

\begin{math}
	\begin{deduction}
		\premise{ \forall x \left( P \left( x \right) \to  Q \left( x \right)   \right)  }
		\premise{ \neg Q \left( a \right) \text{, where $a$ is a particular element in the domain}  }
		\conclusion{ \neg P \left( a \right)  }
	\end{deduction}
\end{math}

\section{Exercises}

\qs{}{
	Show that the premises "If you send me an e-mail message, then I will finish writing the
	program," "If you do not send me an e-mail message, then I will go to sleep early," and "If I go
	to sleep early, then I will wake up feeling refreshed" lead to the conclusion "If I do not finish
	writing the program, then I will wake up feeling refreshed."
}

\sol{
	\begin{align*}
		p \text{: You send me an email message}      \\
		q \text{: I will finish writing the program} \\
		r \text{: I will go to sleep early}          \\
		s \text{: I will wake up feeling refreshed}  \\
		\\
	\end{align*}
}

\qs{}{
	For each of these arguments, explain which rules of inference are used for each step.
	\begin{enumerate}
		\item Each of the five room-mates, Melissa, Aaron, Ralph, Veneesha and Keeshawn, has taken a course in discrete
		      mathematics. Every student who has taken a course in discrete mathematics can take a course in algorithms.
		      Therefore, all five room-mates can take a course in algorithms next year.
		\item All moves produced by John Sayles are wonderful. John Sayles produced a movie about coal miners.
		      Therefore, there is a wonderful movie about coal miners.
		\item There is someone in this class who has been to France. Everyone who goes to France visits the
		      Louvre. Therefore, someone in this class has visited the Louvre.
	\end{enumerate}
}

\sol{

	\begin{enumerate}
		\item
		      \begin{align*}
			      \mathbb{U}_{x} \text{: All students}                                        \\
			      \\
			      D \left( x \right)  \text{: $x$ has taken a course in discrete mathematics} \\
			      A \left( x \right) \text{: x can take a course in algorithms}               \\
		      \end{align*}
		      \begin{math}
			      \begin{deduction}
				      \premise{ D \left( \text{Melissa} \right) \wedge   D \left( \text{Aaron} \right) \wedge  D \left(
					      \text{Ralph} \right) \wedge
					      D \left( \text{Veneesha} \right) \wedge  D \left( \text{Keeshawn} \right) }
				      \premise{ \forall x \left( D \left( x \right) \to A \left( x \right)   \right)  }
				      \conclusion{  A \left( \text{Melissa} \right) \wedge   A \left( \text{Aaron} \right) \wedge  A \left(
					      \text{Ralph} \right) \wedge
					      A \left( \text{Veneesha} \right) \wedge  A \left( \text{Keeshawn} \right)  }
			      \end{deduction}
		      \end{math}
		      \gdef\rownumber{\stepcounter{magicrownumbers}\arabic{magicrownumbers}}
		      \begin{table}[h!]
			      \begin{center}
				      \begin{tabular}{ | @{\makebox[3em][r]{\rownumber\space}} | c | c | }
					      \hline
					      \multicolumn{1}{ | @{\makebox[3em][r]{~}} |c| }{Steps}                       & \multicolumn{1}{|c|}{Reasons} \\
					      \hline
					      \hline
					      $ D \left( \text{Melissa} \right) \wedge   D \left( \text{Aaron} \right) \wedge  D \left(
						      \text{Ralph} \right) \wedge D \left( \text{Veneesha} \right) \wedge  D \left( \text{Keeshawn}
					      \right) $                                                                    & Premise 1                     \\
					      $ \forall x \left( D \left( x \right) \to A \left( x \right)   \right)  $    & Premise 2                     \\
					      $D \left( \text{Melissa} \right) $                                           & By Simplification
					      on 1.                                                                                                        \\
					      $D \left( \text{Aaron} \right) $                                             & By Simplification
					      on 1.                                                                                                        \\
					      $D \left( \text{Ralph} \right) $                                             & By Simplification
					      on 1.                                                                                                        \\
					      $D \left( \text{Veneesha} \right) $                                          & By Simplification
					      on 1.                                                                                                        \\
					      $D \left( \text{Keeshawn} \right) $                                          & By Simplification
					      on 1.                                                                                                        \\
					      $D \left(\text{ Melissa }\right) \to A \left(\text{ Melissa }\right)  $      & Universal Instantiation on 2  \\
					      $D \left(\text{ Aaron }\right) \to A \left(\text{ Aaron }\right)  $          & Universal Instantiation on 2  \\
					      $D \left(\text{ Ralph }\right) \to A \left(\text{ Ralph }\right)  $          & Universal Instantiation on 2  \\
					      $D \left(\text{ Veneesha }\right) \to A \left(\text{ Veneesha }\right)  $    & Universal Instantiation on 2  \\
					      $D \left(\text{ Keeshawn }\right) \to A \left(\text{ Keeshawn }\right)  $    & Universal Instantiation on 2  \\

					      $A \left( \text{Melissa} \right) $                                           & By Modus Ponens on 3 and 8    \\
					      $A \left( \text{Aaron} \right) $                                             & By Modus Ponens on 4 and 9    \\
					      $A \left( \text{Ralph} \right) $                                             & By Modus Ponens on 5 and 10   \\
					      $A \left( \text{Veneesha} \right) $                                          & By Modus Ponens on 6 and 11   \\
					      $A \left( \text{Keeshawn} \right) $                                          & By Modus Ponens on 7 and 12   \\
					      $  A \left( \text{Melissa} \right) \wedge   A \left( \text{Aaron} \right) \wedge  A \left(
						      \text{Ralph} \right) \wedge
					      A \left( \text{Veneesha} \right) \wedge  A \left( \text{Keeshawn} \right)  $ & By Conjunction
					      on 13, 14, 15, 16, and 17                                                                                    \\
					      \hline
				      \end{tabular}
			      \end{center}
		      \end{table}
		\item
		      \begin{align*}
			      \mathbb{U}_{m} \text{: All movies}                                       \\
			      \\
			      P \left(m \right) \text{: $m$ is produced by John Sayles }               \\
			      W \left( m \right) \text{: $m$ is wonderful}                             \\
			      C \left( m  \right)  \text{: $m$ is about coal miners}                   \\
			      \\
			      \forall m \left( P \left( m \right) \to W \left( m \right)    \right)    \\
			      \exists m \left( P \left( m \right) \wedge  C \left( m \right)   \right) \\
			      \exists m \left( W \left( m \right) \wedge C \left( m \right)    \right) \\
			      \\
			      \begin{prooftree}
				      \hypo{&\forall m \left( P \left( m \right) \to W \left( m \right)   \right) }
				      \infer[no rule]1{&\exists m \left( P \left( m \right) \wedge  C \left( m \right)   \right) }
				      \infer1{& \therefore \exists m \left( W \left( m \right) \wedge  C \left( m \right)   \right) }
			      \end{prooftree}
		      \end{align*}
		      \gdef\rownumber{\stepcounter{magicrownumbers}\arabic{magicrownumbers}}
		      \begin{table}[h!]
			      \begin{center}
				      \begin{tabular}{ | @{\makebox[3em][r]{\rownumber\space}} | c | c | }
					      \hline
					      \multicolumn{1}{ | @{\makebox[3em][r]{~}} |c| }{Steps}                      & \multicolumn{1}{|c|}{Reasons}  \\
					      \hline
					      \hline
					      $\forall m \left( P \left( m \right) \to W \left( m \right)    \right) $    & Premise 1                      \\
					      $\exists m \left( P \left( m \right) \wedge C \left( m \right)   \right) $  & Premise 2                      \\
					      $P \left( c \right) \wedge C \left( c \right)  $                            & Existential Instantiation of 2 \\
					      $P \left( c \right) \to W \left( c \right)  $                               & Universal Instantiation of 1   \\
					      $P \left( c \right) $                                                       & By Simplification of 3         \\
					      $C \left( c \right) $                                                       & By Simplification of 3         \\
					      $W \left(c \right) $                                                        & By Modus Ponens on 5 and 4     \\
					      $W \left( c \right) \wedge C \left( c \right)  $                            & By Conjunction on 7 and 6      \\
					      $\exists m \left( W \left( m  \right) \wedge C \left( m \right)   \right) $ & Existential
					      Generalization on 8                                                                                          \\

					      \hline
				      \end{tabular}
			      \end{center}
		      \end{table}
		\item
		      \begin{align*}
			      \mathbb{U}_{x} \text{: All people}                                       \\
			      \\
			      C \left( x \right) \text{: $x$ is someone in this class}                 \\
			      F \left( x \right) \text{: $x$ has visited France}                       \\
			      L \left( x \right) \text{: $x$ has visited the Louvre}                   \\
			      \\
			      \exists x \left( C \left( x \right) \wedge  F \left( x \right)   \right) \\
			      \forall x \left( F \left( x \right) \to  L \left( x \right)   \right)    \\
			      \exists x \left( C \left( x \right) \wedge L \left( x \right)   \right)  \\
			      \\
			      \begin{deduction}
				      \premise{\exists x \left( C \left( x \right) \wedge F \left( x \right)  \right) }
				      \premise{\forall x \left( F \left( x \right) \to L \left( x \right)   \right) }
				      \conclusion{\exists x \left( C \left( x \right) \wedge  L \left( x \right)   \right) }
			      \end{deduction}
		      \end{align*}
		      \gdef\rownumber{\stepcounter{magicrownumbers}\arabic{magicrownumbers}}
		      \begin{table}[h!]
			      \begin{center}
				      \begin{tabular}{ | @{\makebox[3em][r]{\rownumber\space}} | c | c | }
					      \hline
					      \multicolumn{1}{ | @{\makebox[3em][r]{~}} |c| }{Steps}                     & \multicolumn{1}{|c|}{Reasons}  \\
					      \hline
					      \hline
					      $\exists x \left( C \left( x \right) \wedge F \left( x \right)   \right) $ & Premise 1                      \\
					      $\forall x \left( F \left( x \right) \to L \left( x \right)   \right) $    & Premise 2                      \\
					      $C \left( c \right) \wedge F \left( c \right)  $                           & Existential Instantiation of 1 \\
					      $F \left( c \right) \to L \left( c \right)  $                              & Universal Instantiation of 2   \\
					      $C \left( c \right)$                                                       & By Simplification on 3         \\
					      $F \left( c \right) $                                                      & By Simplification on 3         \\
					      $L \left( c \right) $                                                      & By Modus Ponens on 6 and 4     \\
					      $C \left( c \right) \wedge L \left( c \right)  $                           & By Conjunction on 5 and 7      \\
					      $\exists x \left( C \left( x \right) \wedge L \left( x \right)   \right) $ & Existential
					      Generalization on 8                                                                                         \\

					      \hline
				      \end{tabular}
			      \end{center}
		      \end{table}
	\end{enumerate}
}

\chapter{Proofs}


\dfn{Proof}{
	A valid argument that establishes the truth of a statement.
}

\dfn{Theorem}{
	A statement that can be shown to be true, using:
	\begin{itemize}
		\item
	\end{itemize}

	\noindent Less important theorems are sometimes called \textbf{propositions}.
}

\dfn{Axiom / Postulate}{
	A statement that is assumed to be true.
}

\dfn{Lemma}{
	A less important theorem that is helpful in proving a more important theorem.
}

\dfn{Corollary}{
	A result that follows directly from a theorem.
}

\dfn{Conjecture}{
	A statement that is being proposed as a theorem to be proved. Once a conjecture is proved, it becomes a theorem.
}

\noindent Theorems are usually stated in the form:
\[
	\forall x \left( P \left( x \right) \to  Q \left( x \right)   \right)
\]

\noindent Where $P \left( x \right) $ is some condition and $Q \left( x \right) $ is the conclusion.

\section{Set Definitions}

\subsection{Natural Numbers}

All numbers starting from 1. Denoted by $\mathbb{N}$

\[
	\mathbb{N} = \left\{ 1,2,3,4\ldots \right\}
\]

\subsection{Whole Numbers}

All natural numbers and 0. Denoted by $\mathbb{W}$

\[
	\mathbb{W} = \left\{ 0,1,2,3\dots \right\}
\]

\subsection{Rational Numbers}

Numbers that can be written in the form $\frac{x}{y}$ where $y \neq 0$. Denoted by $\mathbb{Q}$

\subsection{Real Numbers}

All rational and irrational numbers. (All numbers on the number line). Denoted by $\mathbb{R}$

\subsection{Irrational Numbers}

Numbers that cannot be written in the form $\frac{x}{y}$ where $y \neq 0$. Denoted by $\mathbb{R} - \mathbb{Q}$


\subsection{Integers}

All whole numbers and negative of all natural numbers. Denoted by $\mathbb{Z}$

\[
	\mathbb{Z} = \left\{ -\infty, -3, -2, -1, 0, 1, 2, 3, \infty \right\}
\]

\section{Types of Proofs}

\subsection{Direct Proof}

Given the statement:
\[
	p \to q
\]
We assume $p$ as true and by forming a sequence of known facts, definitions and other theorems, we show that $q$ is true.

\ex{}{
	\qs{}{
		Give a direct proof of the theorem "If $n$ is an odd integer, then $n^2$ is an odd integer."
	}

	\sol{
		\begin{align*}
			\text{Assume $n$ is odd }                               \\
			\text{Then } \exists k \, \in \, \mathbb{Z}, n = 2k + 1 \\
			n^2 = 2t + 1 \text{ for some integer $t$}               \\
			\\
			n^2  = \left( 2k + 1 \right)^2                          \\
			n^2  = 4k^2 + 4k + 1                                    \\
			n^2  = 2 \left( 2k^2 + 2k \right)  + 1                  \\
			\text{Let $r = 2k^2 + 2k$}                              \\
			n^2 & = 2r + 1                                          \\
			\\
			\text{Because $k$ is an integer, $r$ is an integer}     \\
		\end{align*}

	}
}


\subsection{Indirect Proof}

Any proof that is not a direct proof. This includes:

\begin{itemize}
	\item Contraposition: i.e. $p \to q \equiv \neg q \to \neg p$
	\item Contradiction: Assume $p$ and $\neg q$ and show that this leads to a contradiction.
	\item Vacuous Proof: Show that $p$ is false, and thereby show that $p \to q$ is true.
\end{itemize}


\dfn{Even / Odd integers \& Rational Numbers}{
	An integer $n$ is even if there exists an integer $k$a such that $n = 3k$ and $n$ is odd if there exists an integer
	such that $n = 2k + 1$. \\

	\noindent An integer $n$ is even if $\exists $ integer $k$ such that $n = 2k$. \\
	\noindent An integer $n$ is odd id $\exists $ integer $k$ such that $n = 2k + 1$. \\

	A real number $x$ is rational if $\exists $ integers $p$, $q$ with $q \neq 0$, such that $x = \frac{p}{q}$.



}

\subsection{Proofs of Equivalence}

To prove a theorem is a biconditional statement, that is $p \leftrightarrow q$, we must prove both $p \to q$ and $q \to
	p$ are both true.

\ex{}{
	\qs{}{
		Prove the theorem "If $n$ is an integer, then $n$ is odd if and only if $n^2$ is odd".
	}

	\begin{myproof}
		To prove $p \leftrightarrow q$ where $p$ is "$n$ is odd" and $q$ is "$n^2$ is odd", we must prove $p \to q$ and
		$q \to p$ separately.\\
		\\

		First I will use a direct proof to prove $p \to q$. \\
		Assume that $p$ is true.\\
		Then $\exists k \in \mathbb{Z} \, n = 2k + 1$. \\
		For $n^2$ to be odd
		\[
			\exists t \in \mathbb{Z} \, n^2 = 2t + 1
		\]

		\begin{align*}
			n   & = 2k + 1                          \\
			n^2 & = \left( 2k + 1 \right)^2         \\
			    & = 4k^2 + 4k +1                    \\
			    & = 2 \left( 2k^2 + 2k \right)  + 1 \\
			\text{Let }t = 2k^2 + 2k                \\
			    & = 2t + 1
		\end{align*}
		Where $t$ is an integer because it is made up of the sum of integers 2 and $k$.\\
		Hence if $n$ is odd then $n^2$ is odd. \\
		\\

		Next I will use a proof by contraposition to prove $q \to p$.\\
		The contraposition of $q \to p$ is $\neg p \to \neg q$, .i.e If $n$  is even then, $n^2$ is even.\\
		Assume that $\neg p$ is true.\\
		Then $\exists f \in \mathbb{Z} \, n = 2f$. \\
		For $n^2$ to be even
		\[
			\exists j \in \mathbb{Z} \, n^2 = 2j
		\]

		\begin{align*}
			n   & = 2f                    \\
			n^2 & = 4f^2                  \\
			    & = 2 \left( 2f^2 \right) \\
			\text{Let }j = 2f^2           \\
			    & = 2j
		\end{align*}
		Where $j$ is an integer because it is made up of the product of integers 2 and $f^2$\\
		$\therefore$ if $n$ is even then, $n^2$ is even. \\
		Hence If $n^2$ is odd, then $n$ is odd. \\
		\\
		Hence if $n$ is an integer, then $n$ is odd if and only if $n^2$ is odd
	\end{myproof}
}



\section{Exercises}

\nt{
	Using a direct proof is difficult if the consequence is a function of the premise. In such cases, an indirect proof
	is used.
}

\qs{}{
	Prove that the sum of two rational numbers is rational.
}

\begin{myproof}
	We will show with a direct proof that, if $a$ and $b$ are rational numbers, then $a + b$ is a rational number. \\
	Assume that $a$ and $b$ are rational numbers. \\
	Then there exist integers $p$, $q$, $r$, and $s$ with $q \neq 0$ and $s \neq 0$ such that $a = \frac{p}{q}$ and $b =
		\frac{r}{s}$.\\
	For $a + b$ to be rational
	\begin{align*}
		a + b & = \frac{p}{q} + \frac{r}{s}  \\
		      & = \frac{ps + qr}{qs}         \\
		\text{Let $t =qr + ps$ and $u = qs$} \\
		      & = \frac{t}{u}
	\end{align*}
	Where $u \neq 0$ and $t$ and $u$ are integers. We can conclude $t$ and $u$ are integers because $q$,$r$,$p$,$s$ are
	integers. And $u\neq 0$ because it is the product of two integers $q$ and $s$ which are both nonzero\\
	$\therefore \, a + b$ is rational
	Hence if $a$ and $b$ are rational numbers, then $a + b$ is a rational number.
\end{myproof}

\qs{}{
	Prove that if $n$ is an integer and $3n + 2$ is odd then $n$ is odd.
}

\begin{myproof}
	We will show with a contrapositive proof that, if $n$ is an integer and $3n + 2$ is odd then $n$ is odd. \\
	Therefore let $p$ be the statement "$3n + 2$ is odd" and $q$ be the statement "$n$ is odd". i.e. $p \to q$ \\
	Therefore the contrapositive will be
	\[
		\neg q \to \neg p
	\]
	" If $n$ is even then, $3n + 2$ is even "\\
	Then $\exists k \in \mathbb{Z},\, n = 2k$
	\begin{align*}
		3n + 2 & = 3 \left( 2k \right)  + 2 \\
		       & = 6k + 2                   \\
		       & = 2 \left( 3k + 1 \right)  \\
		\text{Let } t = 3k + 1              \\
		       & = 2t                       \\
	\end{align*}
	Since $3$, $k$ and $1$ are integers $t$ is an integer.\\
	$\therefore \, 3k + 1$ is even \\
	Hence if $n$ is even then $3n + 2$ is even. \\
\end{myproof}


\qs{}{
	Give a proof by contradiction of the theorem "If $3n + 2$ is odd, then $n$ is odd"
}

\begin{myproof}
	Let $p$ be "$3n + 2$ is odd" and $q$ be "$n$ is odd". To construct a proof by contradiction, we need to show that
	$p \wedge \neg q \equiv F$.i.e. $3n + 2$ is odd and $n$ is even.\\
	First we must assume both $\neg  q$ and $ p$ are $T$\\
	Because $n$ is even $\exists k \in \mathbb{Z} \, n = 2k$
	\begin{align*}
		3n + 2 & = 3 \left( 2k \right)  + 2 \\
		       & = 6k + 2                   \\
		       & = 2 \left( 3k + 1 \right)  \\
		\text{Let $t = 3k + 1$}             \\
		       & = 2t                       \\
	\end{align*}
	Since $3$, $k$ and $1$ are integers $t$ is an integer.\\
	Since $3n + 2$ can be written in the form $2t$ where $t$ is an integer, $3n + 2$ is even.\\
	We now have $3n + 2$ to be both even and odd ($p \wedge \neg p$). This is a contradiction.\\
	$\therefore$ If $3n + 2$ is odd, then $n$ is odd.
\end{myproof}

\qs{}{
	Use a direct proof to show that the sum of two even integers is even.
}

\begin{myproof}
	If $a$ and $b$ are even then $a + b$ is even\\
	We assume that $a$ and $b$ are even\\
	Then $\exists k \in \mathbb{Z} \, a = 2k$ and $\exists t \in \mathbb{Z} \, b = 2t$\\

	\noindent For $a + b$ to be even\\
	\[
		a + b = 2z
	\]
	Where $z$ is an integer
	\begin{align*}
		a + b & = 2k + 2t                \\
		      & = 2 \left( k + t \right) \\
		\text{Let } z =  k + t           \\
		      & = 2z
	\end{align*}
	Since $z$ is the sum of two integers $k$ and $t$, $z$ is an integer\\
	Hence if $a$ and $b$ are even then $a + b$ is even.
\end{myproof}

\qs{}{
	Prove that $\sqrt{2}$ is irrational by giving a proof by contradiction
}

\begin{myproof}
	Let $p$ be "$\sqrt{2}$ is irrational". \\
	We assume $\neg p$ is true, i.e. "$\sqrt{2}$ is rational". \\
	Then there exists integers $q$ and $r$, where $q \neq 0$ and $r$ and $q$ have no common factors, such that $\sqrt{2}
		= \frac{r}{q}$. \\
	\begin{align*}
		\sqrt{2} = \frac{r}{q} \\
		2 = \frac{r^2}{q^2}    \\
		2q^2 = r^2             \\
	\end{align*}
	By the definition of an even integer it follows that $r^2$ is even.\\
	If $r^2$ is even then $r$ must also be even.\\
	Then $\exists c \in \mathbb{Z} \, r = 2c$. \\
	Thus: \\
	\begin{align*}
		2q^2 = \left( 2c \right)^2 \\
		2q^2 = 4c^2                \\
		q^2 = 2c^2                 \\
	\end{align*}
	Giving us $q^2$ to be even by the definition of an even integer, therefore $q$ must also be even. \\
	Leaving us with
	\[
		\sqrt{2} = \frac{r}{q}
	\]
	Where $r$ and $q$ are both even and divisible by 2. \\
	The statement $\sqrt{2} = \frac{r}{q}$, where $r$ and $q$ have no common factors implies that $2$ does not divide
	both $r$ and $q$. This leaves us with the contradiction 2 divides $r$ and $p$ and 2 does not divide $r$ and $p$,
	meaning $\neg p$ must be false. Therefore the statement $p$ must be true.



\end{myproof}

\qs{}{
	Prove that if $n$ is an integer and $n^2$ is odd, then $n$ is odd.
}

\begin{myproof}
	I will use a proof by contraposition.\\
	The statement is in the form $p \to q$ where $p$ is " $n^2$ is odd " and $q$ is $n$ is odd.\\
	\[
		\neg q \to \neg p
	\]
	If $n$ is even, then $n^2$ is even.\\
	Assume $n$ is even.\\
	Then $\exists k \in \mathbb{Z} \, n = 2k$
	For $n^2$ to be even:
	\[
		\exists t \in \mathbb{Z} \, n^2 = 2t
	\]

	\begin{align*}
		n   & = 2k                    \\
		n^2 & = 4k^2                  \\
		    & = 2 \left( 4k^2 \right) \\
		\text{Let } t = 4k^2          \\
		    & = 2t
	\end{align*}
	Since $t$ is made up of integers 4 and $k^2$ $t$ is an integer.\\
	Hence If $n$ is an integer and $n^2$ odd, then $n$ is odd.
\end{myproof}

\qs{}{
	Prove that if $m + n$ and $n + p$ are even integers, where $m$, $n$ ,and $p$ are integers, then $m + p$ is even.
}

\begin{myproof}
	I will use a direct proof to prove that if $m + n$ and $n + p$ are even, then $m + p$ is even. \\
	Assume $m + n$ and $n + p$ are even.\\
	Then $\exists k \in \mathbb{Z}\, m + n = 2k$.\\
	Then $\exists z \in \mathbb{Z} \, n + p = 2z$.

	For $m + p$ to be even
	\[
		\exists v \in \mathbb{Z} \, m + p = 2v
	\]

	\begin{align*}
		m + n & = 2k                                             \\
		m     & = 2k - n                                         \\
		n + p & = 2z                                             \\
		p     & = 2z -n                                          \\
		\\
		m + p & = \left( 2k - n \right)  + \left( 2z - n \right) \\
		      & = 2k + 2z -2                                     \\
		      & = 2 \left( k + z -1  \right)                     \\
		\text{Let } v = k + z -1                                 \\
		      & = 2v
	\end{align*}
	Since $v$ is made up of integers $k$, $z$ and $-1$ $v$ is an integer.\\
	Hence if $m + n$ and $n + p$ are even, then $m + p$ is even
\end{myproof}

\qs{}{
	Show that if $n$ is an integer and $n^3 + 5$ is odd, then $n$ is even using
	\begin{enumerate}
		\item A proof by contraposition
		\item A proof by contradiction
	\end{enumerate}
}

\begin{myproof}

	\begin{enumerate}
		\item
		      The contraposition of the statement if $n^3 + 5$ is odd, then $n$ is even, is If $n$ is odd, then $n^3 + 5$
		      is even.\\
		      Assume $n$ is odd.\\
		      Then $\exists k \in\mathbb{Z}\, n = 2k + 1$.\\
		      For $n^3 + 5$ to be even then
		      \[
			      \exists z \in \mathbb{Z} \, n^3 + 5 = 2z
		      \]
		      \begin{align*}
			      n^3 + 5 & = \left( 2k + 1 \right)^3 + 5            \\
			              & = 8k^3 + 12k^2 + 6k + 6                  \\
			              & = 2 \left( 4k^3 + 6k^2 + 3k + 3  \right) \\
			      \text{Let } z =  4k^3 + 6k^2 + 3k + 3              \\
			              & = 2z                                     \\
		      \end{align*}
		      Since $z$ is made up of integers $z$ is an integer.\\
		      $\therefore$ If $n$ is odd, then $n^3 + 5$ is even\\
		      Hence if $n^3 + 5$ is odd, then $n$ is even.

		\item
		      The contradiction of the conditional statement will be in the form $p \wedge  \neg q$, i.e. $n^3 + 5$ is odd
		      and $n$ is odd.\\
		      Assume that both $p$ and $\neg q$ are $T$.\\
		      Then $\exists k \in \mathbb{Z} \, n = 2k + 1$. \\
		      For $n^3 + 5$ to be odd
		      \[
			      \exists v \in \mathbb{Z}\, n^3 + 5 = 2v + 1
		      \]
		      \begin{align*}
			      n^3 + 5 & = \left( 2k + 1 \right)^3 + 5          \\
			              & = 8k^3 + 12k^2 + 6k + 6                \\
			              & = 2 \left( 4k^3 + 6^2 + 3k + 3 \right) \\
			      \text{Let } v =  4k^3 + 6^2 + 3k + 3             \\
			              & = 2v                                   \\
		      \end{align*}
		      Since $v$ is made up of the sum of products of integers $v$ is an integer.\\
		      Since $n^3 + 5$ can be expressed in the form $\exists v \in \mathbb{Z} \, 2v$, this implies that $n^3 + 5$
		      is even. This contradicts with my assumption that $n^3 + 5$ is odd, $\therefore$ $p \wedge \neg q$ is
		      $F$.\\
		      Hence if $n^3 + 5$ is odd, then $n$ is even.
	\end{enumerate}
\end{myproof}

\qs{}{
	Prove that if $n$ is a positive integer, then $n$ is odd if and only if $5n + 6$ is odd.
}

\begin{myproof}
	To prove this biconditional statement we have to express it in the form $\left( p \to q \right) \wedge \left( q \to
		p\right)  $\\
	\\
	First I will use a direct proof to prove $p \to q$, i.e. If $n$ is odd, then $5n + 6$ is odd. \\
	Assume $n$ is odd. \\
	Then $\exists k \in \mathbb{Z} \, n = 2k+ 1$. \\
	For $5n + 1$ to be odd
	\[
		\exists z \in \mathbb{Z} \, 5n + 6 = 2z + 1
	\]

	\begin{align*}
		5n + 6 & = 5 \left( 2k + 1 \right)  + 6 \\
		       & = 10k + 10 + 1                 \\
		       & = 2 \left( 5k + 5 \right)  + 1 \\
		\text{Let } z = 5k + 5                  \\
		       & = 2z + 1
	\end{align*}
	Since $z$ is made up of integers $5$ and $k$, $z$ is an integer.\\
	Hence if $n$ is odd, then  $5n + 6$ is odd.\\
	\\

	\noindent Next I will use a proof by contraposition to prove $q \to p$, i.e. If $5n + 6$ is odd, then $n$ is odd.\\
	The contraposition of this statement is $\neg p \to \neg q $, i.e If $n$ is even, then $5n + 6$ is even.\\
	Assume $n$ to be even.\\
	Then $\exists k \in \mathbb{Z}\, n = 2k$.\\
	For $5n + 6$ to be even
	\[
		\exists v \in \mathbb{Z} \, 5n + 6 = 2v
	\]

	\begin{align*}
		5n + 6 & = 5 \left( 2k \right)  + 6 \\
		       & = 10k + 6                  \\
		       & = 2 \left( 5k + 3 \right)  \\
		\text{Let }v = 5k + 3               \\
		       & = 2v                       \\
	\end{align*}
	Since $v$ is made up of integers $5$, $k$ ,and $3$ $v$ is an integer.\\
	$\therefore$ If $n$ is even, then $5n + 6$ is even.\\
	Hence, If $5n + 6$ is odd, then $n$ is odd.
	\\
	Hence, if $n$ is a positive integer, then $n$ is odd if and only if $5n + 6$ is odd.


\end{myproof}

\qs{}{
	Use a proof by contraposition to show that if $x + y \geq 2$, where $x$ and $y$ are real numbers then $x \geq 1$ or $y
		\geq 1$
}

\begin{myproof}
	The contraposition of the statement If $x + y \geq 2$, then $x \geq 1$ or $y \geq 1$, in the form $p \to q$, is
	$\neg q \to \neg p$, I.e. If $x < 1$ and $y < 1$, then $x + y < 2$.\\
	Assume $x < 1$ and $y < 1$ is $T$ \\
	\begin{align*}
		\left( x < 1 \right) + \left( y < 1 \right) & = x + y < 2
	\end{align*}
	$\therefore$ If $x < 1$ and $y < 1$, then $x + y < 2$\\
	Hence if $x + y \geq 2$, then $x \geq 1$ or $y \geq 1$
\end{myproof}

\qs{}{
	Prove that if $m$ and $n$ are integers and $mn$ is even, then $m$ is even or $n$ is even.
}

\begin{myproof}
	I will prove this using contraposition. The contraposition of the statement If $mn$ is even, then $m$ is even or $n$
	is even, in the form $p \to q$, is $\neg q \to \neg p$, I.e. If $m$ is odd and $n$ is odd, then $mn$ is odd. \\
	Assume $m$ is odd and $n$ is odd is $T$.\\
	\\
	Then $\exists k \in \mathbb{Z} \left( m = 2k + 1 \right) $ and $\exists t \in \mathbb{Z} \left( n = 2t + 1 \right) $\\
	For $mn$ to be odd\\
	\[
		\exists v \in \mathbb{Z} \left( mn = 2v +1 \right)
	\]
	\begin{align*}
		mn            & = \left( 2k + 1 \right) \times  \left( 2t + 1  \right) \\
		              & = 4kt + 2k + 2t + 1                                    \\
		              & = 2 \left( 2kt + k + t \right) +1                      \\
		\text{Let } v & = 2kt + k + 1                                          \\
		              & = 2v + 1                                               \\
	\end{align*}
	Since $v$ is the sum of products of integers $2$, $v$, and $1$, $v$ is an integer\\
	$\therefore$ If $m$ is odd and $n$ is odd, then $mn$ is odd.\\
	Hence if $mn$ is even, then $m$ is even or $n$ is even.

\end{myproof}


\end{document}
