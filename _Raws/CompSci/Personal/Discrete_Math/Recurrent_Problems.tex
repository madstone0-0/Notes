\documentclass[12pt letter]{report}
\input{./template/preamble}
\input{./template/macros}
\input{./template/letterfonts}

\title{\Huge{Recurrent Problems}}
\author{\huge{Madiba Hudson-Quansah}}
\date{September 2023}
\usepackage{parskip}

\begin{document}
\maketitle
\newpage
\pdfbookmark[section]{\contentsname}{too}
\tableofcontents
\pagebreak

\chapter{Introduction}

Problems that have been investigated repeatedly by mathematicians and whose solutions use the idea of recurrence, i.e.
the solution to each problem depends on the solutions to smaller instances of the same problem.

\section{Tower of Hanoi}

Invented by French mathematician Edouard Lucas in 1883. The problem consists of a tower of eight disks initially stacked
in decreasing size on one of three pegs.
The objective is to transfer the entire tower to one of the other pegs, moving only one disk at a time and never moving
a larger one onto a smaller.

\textbf{LOOK AT SMALL CASES} - The first step in solving this is to generalize the question, i.e Let $n$ be the number of disks initially stacked.

\textbf{NAME AND CONQUER} - The next step is to introduce appropriate notation, i.e. Let $T_n$ be the minimum number of moves that will transfer $n$
disks from one peg to another. Therefore $T_0 = 0$, $T_1= 1$ and $T_2 = 3$.

The next step is to use this generalization to scale to different disk numbers. Through trial and error we can see that
the best strategy for transferring $n$ disks is to first transfer the $n - 1$ smallest disk to a different peg, requiring
$T_{n-1}$ moves. Then move the largest disk to the last peg, also requiring one move. Then finally transfer the $n-1$
smallest back onto the third peg where the largest disk already is, requiring another $T_{n-1}$ moves.

Thus we can transfer $n$ disks (where $n > 0$) in at most $2T_{n-1} + 1$ moves, i.e. :

\[
	T_n \leq 2T_{n-1} + 1 \quad n > 0
\]

$\leq$ instead of $=$ due to our construction only proving that $2T_{n-1} + 1$ moves work, not that all $2T_{n-1} + 1$
moves are necessary. The however is the optimal solution, due to us having to move the largest disk at least once and
having to have the $n-1$ smallest disk be alone on one peg every time we do move the largest disk, requiring $T_{n-1}$
moves every time.

The result factoring in the base condition of $T_0 = 0$ is:

\begin{gather*}
	T_0 = 0\\
	T_n = 2T_{n-1} + 1 \quad n > 0
\end{gather*}

These sets of in equalities is called a recurrence / recurrence relation / recursion relation. It gives a boundary value
and an equation for the general value in terms of the earlier ones.

Computing from recurrence for a large value of $n$ takes too long so we need a \textit{closed form} for $T_n$ that lets
us compute large $n$s quickly



\end{document}
