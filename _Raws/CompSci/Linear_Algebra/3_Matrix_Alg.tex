\documentclass[12pt letter]{report}
%%%%%%%%%%%%%%%%%%%%%%%%%%%%%%%%%
% PACKAGE IMPORTS
%%%%%%%%%%%%%%%%%%%%%%%%%%%%%%%%%


\usepackage[tmargin=2cm,rmargin=1in,lmargin=1in,margin=0.85in,bmargin=2cm,footskip=.2in]{geometry}
\usepackage{amsmath,amsfonts,amsthm,amssymb,mathtools}
\usepackage[varbb]{newpxmath}
\usepackage{xfrac}
\usepackage[makeroom]{cancel}
\usepackage{mathtools}
\usepackage{bookmark}
\usepackage{enumitem}
\usepackage{hyperref,theoremref}
\hypersetup{
  pdftitle={Assignment},
  colorlinks=true, linkcolor=doc!90,
  bookmarksnumbered=true,
  bookmarksopen=true
}
\usepackage[most,many,breakable]{tcolorbox}
\usepackage{xcolor}
\usepackage{varwidth}
\usepackage{varwidth}
\usepackage{etoolbox}
%\usepackage{authblk}
\usepackage{nameref}
\usepackage{multicol,array}
\usepackage{tikz-cd}
\usepackage[ruled,vlined,linesnumbered]{algorithm2e}
\usepackage{comment} % enables the use of multi-line comments (\ifx \fi) 
\usepackage{import}
\usepackage{xifthen}
\usepackage{pdfpages}
\usepackage{transparent}
\usepackage{xcolor,colortbl,array,amssymb}
\usepackage{venndiagram}
\usepackage{listings}
\usepackage{fontspec}

\setmainfont{LibertinusSerif}[
  Extension = .otf,
  Path = /usr/share/fonts/libertinus/,
  UprightFont = *-Regular,
  ItalicFont = *-Italic,
  BoldFont = *-Bold,
  BoldItalicFont = *-BoldItalic,
]

\setmonofont{CaskaydiaCoveNerdFontMono}[
  Extension = .ttf,
  Path = /usr/share/fonts/TTF/,
  UprightFont = *-Regular,
  ItalicFont = *-Italic,
  BoldFont = *-Bold,
  BoldItalicFont = *-BoldItalic,
]

\definecolor{listing-background}{HTML}{F7F7F7}
\definecolor{listing-rule}{HTML}{B3B2B3}
\definecolor{listing-numbers}{HTML}{B3B2B3}
\definecolor{listing-text-color}{HTML}{000000}
\definecolor{listing-keyword}{HTML}{435489}
\definecolor{listing-keyword-2}{HTML}{1284CA} % additional keywords
\definecolor{listing-keyword-3}{HTML}{9137CB} % additional keywords
\definecolor{listing-identifier}{HTML}{435489}
\definecolor{listing-string}{HTML}{00999A}
\definecolor{listing-comment}{HTML}{8E8E8E}

\lstdefinestyle{eisvogel_listing_style}{
language         = java,
xleftmargin      = 0.6em,
framexleftmargin = 0.4em,
backgroundcolor  = \color{listing-background},
basicstyle       = \color{listing-text-color}\linespread{1.0}%
\lst@ifdisplaystyle%
\fi\ttfamily{},
breaklines       = true,
frame            = single,
framesep         = 0.19em,
rulecolor        = \color{listing-rule},
frameround       = ffff,
tabsize          = 4,
numberstyle      = \color{listing-numbers},
aboveskip        = 1.0em,
belowskip        = 0.1em,
abovecaptionskip = 0em,
belowcaptionskip = 1.0em,
keywordstyle     = {\color{listing-keyword}\bfseries},
keywordstyle     = {[2]\color{listing-keyword-2}\bfseries},
keywordstyle     = {[3]\color{listing-keyword-3}\bfseries\itshape},
sensitive        = true,
identifierstyle  = \color{listing-identifier},
commentstyle     = \color{listing-comment},
stringstyle      = \color{listing-string},
showstringspaces = false,
escapeinside     = {/*@}{@*/}, % Allow LaTeX inside these special comments
literate         =
  {á}{{\'a}}1 {é}{{\'e}}1 {í}{{\'i}}1 {ó}{{\'o}}1 {ú}{{\'u}}1
{Á}{{\'A}}1 {É}{{\'E}}1 {Í}{{\'I}}1 {Ó}{{\'O}}1 {Ú}{{\'U}}1
{à}{{\`a}}1 {è}{{\`e}}1 {ì}{{\`i}}1 {ò}{{\`o}}1 {ù}{{\`u}}1
{À}{{\`A}}1 {È}{{\`E}}1 {Ì}{{\`I}}1 {Ò}{{\`O}}1 {Ù}{{\`U}}1
{ä}{{\"a}}1 {ë}{{\"e}}1 {ï}{{\"i}}1 {ö}{{\"o}}1 {ü}{{\"u}}1
{Ä}{{\"A}}1 {Ë}{{\"E}}1 {Ï}{{\"I}}1 {Ö}{{\"O}}1 {Ü}{{\"U}}1
{â}{{\^a}}1 {ê}{{\^e}}1 {î}{{\^i}}1 {ô}{{\^o}}1 {û}{{\^u}}1
{Â}{{\^A}}1 {Ê}{{\^E}}1 {Î}{{\^I}}1 {Ô}{{\^O}}1 {Û}{{\^U}}1
{œ}{{\oe}}1 {Œ}{{\OE}}1 {æ}{{\ae}}1 {Æ}{{\AE}}1 {ß}{{\ss}}1
{ç}{{\c c}}1 {Ç}{{\c C}}1 {ø}{{\o}}1 {å}{{\r a}}1 {Å}{{\r A}}1
{€}{{\EUR}}1 {£}{{\pounds}}1 {«}{{\guillemotleft}}1
{»}{{\guillemotright}}1 {ñ}{{\~n}}1 {Ñ}{{\~N}}1 {¿}{{?`}}1
{…}{{\ldots}}1 {≥}{{>=}}1 {≤}{{<=}}1 {„}{{\glqq}}1 {“}{{\grqq}}1
{”}{{''}}1
}
\lstset{style=eisvogel_listing_style,
  numbers=left}

%
% Java (Java SE 12, 2019-06-22)
%
\lstdefinelanguage{Java}{
  morekeywords={
      % normal keywords (without data types)
      abstract,assert,break,case,catch,class,continue,default,
      do,else,enum,exports,extends,final,finally,for,if,implements,
      import,instanceof,interface,module,native,new,package,private,
      protected,public,requires,return,static,strictfp,super,switch,
      synchronized,this,throw,throws,transient,try,volatile,while,
      % var is an identifier
      var
    },
  morekeywords={[2] % data types
      % primitive data types
      boolean,byte,char,double,float,int,long,short,
      % String
      String,
      % primitive wrapper types
      Boolean,Byte,Character,Double,Float,Integer,Long,Short
      % number types
      Number,AtomicInteger,AtomicLong,BigDecimal,BigInteger,DoubleAccumulator,DoubleAdder,LongAccumulator,LongAdder,Short,
      % other
      Object,Void,void
    },
  morekeywords={[3] % literals
      % reserved words for literal values
      null,true,false,
    },
  sensitive,
  morecomment  = [l]//,
  morecomment  = [s]{/*}{*/},
  morecomment  = [s]{/**}{*/},
  morestring   = [b]",
  morestring   = [b]',
}

\lstdefinelanguage{XML}{
  morestring      = [b]",
  moredelim       = [s][\bfseries\color{listing-keyword}]{<}{\ },
  moredelim       = [s][\bfseries\color{listing-keyword}]{</}{>},
  moredelim       = [l][\bfseries\color{listing-keyword}]{/>},
  moredelim       = [l][\bfseries\color{listing-keyword}]{>},
  morecomment     = [s]{<?}{?>},
  morecomment     = [s]{<!--}{-->},
  commentstyle    = \color{listing-comment},
  stringstyle     = \color{listing-string},
  identifierstyle = \color{listing-identifier}
}


\newcommand\mycommfont[1]{\footnotesize\ttfamily\textcolor{blue}{#1}}
\SetCommentSty{mycommfont}
\newcommand{\incfig}[1]{%
  \def\svgwidth{\columnwidth}
  \import{./figures/}{#1.pdf_tex}
}

\usepackage{tikzsymbols}
\renewcommand\qedsymbol{$\Laughey$}


%\usepackage{import}
%\usepackage{xifthen}
%\usepackage{pdfpages}
%\usepackage{transparent}


%%%%%%%%%%%%%%%%%%%%%%%%%%%%%%
% SELF MADE COLORS
%%%%%%%%%%%%%%%%%%%%%%%%%%%%%%



\definecolor{myg}{RGB}{56, 140, 70}
\definecolor{myb}{RGB}{45, 111, 177}
\definecolor{myr}{RGB}{199, 68, 64}
\definecolor{mytheorembg}{HTML}{F2F2F9}
\definecolor{mytheoremfr}{HTML}{00007B}
\definecolor{mylenmabg}{HTML}{FFFAF8}
\definecolor{mylenmafr}{HTML}{983b0f}
\definecolor{mypropbg}{HTML}{f2fbfc}
\definecolor{mypropfr}{HTML}{191971}
\definecolor{myexamplebg}{HTML}{F2FBF8}
\definecolor{myexamplefr}{HTML}{88D6D1}
\definecolor{myexampleti}{HTML}{2A7F7F}
\definecolor{mydefinitbg}{HTML}{E5E5FF}
\definecolor{mydefinitfr}{HTML}{3F3FA3}
\definecolor{notesgreen}{RGB}{0,162,0}
\definecolor{myp}{RGB}{197, 92, 212}
\definecolor{mygr}{HTML}{2C3338}
\definecolor{myred}{RGB}{127,0,0}
\definecolor{myyellow}{RGB}{169,121,69}
\definecolor{myexercisebg}{HTML}{F2FBF8}
\definecolor{myexercisefg}{HTML}{88D6D1}


%%%%%%%%%%%%%%%%%%%%%%%%%%%%
% TCOLORBOX SETUPS
%%%%%%%%%%%%%%%%%%%%%%%%%%%%

\setlength{\parindent}{1cm}
%================================
% THEOREM BOX
%================================

\tcbuselibrary{theorems,skins,hooks}
\newtcbtheorem[number within=section]{Theorem}{Theorem}
{%
  enhanced,
  breakable,
  colback = mytheorembg,
  frame hidden,
  boxrule = 0sp,
  borderline west = {2pt}{0pt}{mytheoremfr},
  sharp corners,
  detach title,
  before upper = \tcbtitle\par\smallskip,
  coltitle = mytheoremfr,
  fonttitle = \bfseries\sffamily,
  description font = \mdseries,
  separator sign none,
  segmentation style={solid, mytheoremfr},
}
{th}

\tcbuselibrary{theorems,skins,hooks}
\newtcbtheorem[number within=chapter]{theorem}{Theorem}
{%
  enhanced,
  breakable,
  colback = mytheorembg,
  frame hidden,
  boxrule = 0sp,
  borderline west = {2pt}{0pt}{mytheoremfr},
  sharp corners,
  detach title,
  before upper = \tcbtitle\par\smallskip,
  coltitle = mytheoremfr,
  fonttitle = \bfseries\sffamily,
  description font = \mdseries,
  separator sign none,
  segmentation style={solid, mytheoremfr},
}
{th}


\tcbuselibrary{theorems,skins,hooks}
\newtcolorbox{Theoremcon}
{%
  enhanced
  ,breakable
  ,colback = mytheorembg
  ,frame hidden
  ,boxrule = 0sp
  ,borderline west = {2pt}{0pt}{mytheoremfr}
  ,sharp corners
  ,description font = \mdseries
  ,separator sign none
}

%================================
% Corollery
%================================
\tcbuselibrary{theorems,skins,hooks}
\newtcbtheorem[number within=section]{Corollary}{Corollary}
{%
  enhanced
  ,breakable
  ,colback = myp!10
  ,frame hidden
  ,boxrule = 0sp
  ,borderline west = {2pt}{0pt}{myp!85!black}
  ,sharp corners
  ,detach title
  ,before upper = \tcbtitle\par\smallskip
  ,coltitle = myp!85!black
  ,fonttitle = \bfseries\sffamily
  ,description font = \mdseries
  ,separator sign none
  ,segmentation style={solid, myp!85!black}
}
{th}
\tcbuselibrary{theorems,skins,hooks}
\newtcbtheorem[number within=chapter]{corollary}{Corollary}
{%
  enhanced
  ,breakable
  ,colback = myp!10
  ,frame hidden
  ,boxrule = 0sp
  ,borderline west = {2pt}{0pt}{myp!85!black}
  ,sharp corners
  ,detach title
  ,before upper = \tcbtitle\par\smallskip
  ,coltitle = myp!85!black
  ,fonttitle = \bfseries\sffamily
  ,description font = \mdseries
  ,separator sign none
  ,segmentation style={solid, myp!85!black}
}
{th}


%================================
% LENMA
%================================

\tcbuselibrary{theorems,skins,hooks}
\newtcbtheorem[number within=section]{Lenma}{Lenma}
{%
  enhanced,
  breakable,
  colback = mylenmabg,
  frame hidden,
  boxrule = 0sp,
  borderline west = {2pt}{0pt}{mylenmafr},
  sharp corners,
  detach title,
  before upper = \tcbtitle\par\smallskip,
  coltitle = mylenmafr,
  fonttitle = \bfseries\sffamily,
  description font = \mdseries,
  separator sign none,
  segmentation style={solid, mylenmafr},
}
{th}

\tcbuselibrary{theorems,skins,hooks}
\newtcbtheorem[number within=chapter]{lenma}{Lenma}
{%
  enhanced,
  breakable,
  colback = mylenmabg,
  frame hidden,
  boxrule = 0sp,
  borderline west = {2pt}{0pt}{mylenmafr},
  sharp corners,
  detach title,
  before upper = \tcbtitle\par\smallskip,
  coltitle = mylenmafr,
  fonttitle = \bfseries\sffamily,
  description font = \mdseries,
  separator sign none,
  segmentation style={solid, mylenmafr},
}
{th}


%================================
% PROPOSITION
%================================

\tcbuselibrary{theorems,skins,hooks}
\newtcbtheorem[number within=section]{Prop}{Proposition}
{%
  enhanced,
  breakable,
  colback = mypropbg,
  frame hidden,
  boxrule = 0sp,
  borderline west = {2pt}{0pt}{mypropfr},
  sharp corners,
  detach title,
  before upper = \tcbtitle\par\smallskip,
  coltitle = mypropfr,
  fonttitle = \bfseries\sffamily,
  description font = \mdseries,
  separator sign none,
  segmentation style={solid, mypropfr},
}
{th}

\tcbuselibrary{theorems,skins,hooks}
\newtcbtheorem[number within=chapter]{prop}{Proposition}
{%
  enhanced,
  breakable,
  colback = mypropbg,
  frame hidden,
  boxrule = 0sp,
  borderline west = {2pt}{0pt}{mypropfr},
  sharp corners,
  detach title,
  before upper = \tcbtitle\par\smallskip,
  coltitle = mypropfr,
  fonttitle = \bfseries\sffamily,
  description font = \mdseries,
  separator sign none,
  segmentation style={solid, mypropfr},
}
{th}


%================================
% CLAIM
%================================

\tcbuselibrary{theorems,skins,hooks}
\newtcbtheorem[number within=section]{claim}{Claim}
{%
  enhanced
  ,breakable
  ,colback = myg!10
  ,frame hidden
  ,boxrule = 0sp
  ,borderline west = {2pt}{0pt}{myg}
  ,sharp corners
  ,detach title
  ,before upper = \tcbtitle\par\smallskip
  ,coltitle = myg!85!black
  ,fonttitle = \bfseries\sffamily
  ,description font = \mdseries
  ,separator sign none
  ,segmentation style={solid, myg!85!black}
}
{th}



%================================
% Exercise
%================================

\tcbuselibrary{theorems,skins,hooks}
\newtcbtheorem[number within=section]{Exercise}{Exercise}
{%
  enhanced,
  breakable,
  colback = myexercisebg,
  frame hidden,
  boxrule = 0sp,
  borderline west = {2pt}{0pt}{myexercisefg},
  sharp corners,
  detach title,
  before upper = \tcbtitle\par\smallskip,
  coltitle = myexercisefg,
  fonttitle = \bfseries\sffamily,
  description font = \mdseries,
  separator sign none,
  segmentation style={solid, myexercisefg},
}
{th}

\tcbuselibrary{theorems,skins,hooks}
\newtcbtheorem[number within=chapter]{exercise}{Exercise}
{%
  enhanced,
  breakable,
  colback = myexercisebg,
  frame hidden,
  boxrule = 0sp,
  borderline west = {2pt}{0pt}{myexercisefg},
  sharp corners,
  detach title,
  before upper = \tcbtitle\par\smallskip,
  coltitle = myexercisefg,
  fonttitle = \bfseries\sffamily,
  description font = \mdseries,
  separator sign none,
  segmentation style={solid, myexercisefg},
}
{th}

%================================
% EXAMPLE BOX
%================================

\newtcbtheorem[number within=section]{Example}{Example}
{%
  colback = myexamplebg
  ,breakable
  ,colframe = myexamplefr
  ,coltitle = myexampleti
  ,boxrule = 1pt
  ,sharp corners
  ,detach title
  ,before upper=\tcbtitle\par\smallskip
  ,fonttitle = \bfseries
  ,description font = \mdseries
  ,separator sign none
  ,description delimiters parenthesis
}
{ex}

\newtcbtheorem[number within=chapter]{example}{Example}
{%
  colback = myexamplebg
  ,breakable
  ,colframe = myexamplefr
  ,coltitle = myexampleti
  ,boxrule = 1pt
  ,sharp corners
  ,detach title
  ,before upper=\tcbtitle\par\smallskip
  ,fonttitle = \bfseries
  ,description font = \mdseries
  ,separator sign none
  ,description delimiters parenthesis
}
{ex}

%================================
% DEFINITION BOX
%================================

\newtcbtheorem[number within=section]{Definition}{Definition}{enhanced,
  before skip=2mm,after skip=2mm, colback=red!5,colframe=red!80!black,boxrule=0.5mm,
  attach boxed title to top left={xshift=1cm,yshift*=1mm-\tcboxedtitleheight}, varwidth boxed title*=-3cm,
  boxed title style={frame code={
          \path[fill=tcbcolback]
          ([yshift=-1mm,xshift=-1mm]frame.north west)
          arc[start angle=0,end angle=180,radius=1mm]
          ([yshift=-1mm,xshift=1mm]frame.north east)
          arc[start angle=180,end angle=0,radius=1mm];
          \path[left color=tcbcolback!60!black,right color=tcbcolback!60!black,
            middle color=tcbcolback!80!black]
          ([xshift=-2mm]frame.north west) -- ([xshift=2mm]frame.north east)
          [rounded corners=1mm]-- ([xshift=1mm,yshift=-1mm]frame.north east)
          -- (frame.south east) -- (frame.south west)
          -- ([xshift=-1mm,yshift=-1mm]frame.north west)
          [sharp corners]-- cycle;
        },interior engine=empty,
    },
  fonttitle=\bfseries,
  title={#2},#1}{def}
\newtcbtheorem[number within=chapter]{definition}{Definition}{enhanced,
  before skip=2mm,after skip=2mm, colback=red!5,colframe=red!80!black,boxrule=0.5mm,
  attach boxed title to top left={xshift=1cm,yshift*=1mm-\tcboxedtitleheight}, varwidth boxed title*=-3cm,
  boxed title style={frame code={
          \path[fill=tcbcolback]
          ([yshift=-1mm,xshift=-1mm]frame.north west)
          arc[start angle=0,end angle=180,radius=1mm]
          ([yshift=-1mm,xshift=1mm]frame.north east)
          arc[start angle=180,end angle=0,radius=1mm];
          \path[left color=tcbcolback!60!black,right color=tcbcolback!60!black,
            middle color=tcbcolback!80!black]
          ([xshift=-2mm]frame.north west) -- ([xshift=2mm]frame.north east)
          [rounded corners=1mm]-- ([xshift=1mm,yshift=-1mm]frame.north east)
          -- (frame.south east) -- (frame.south west)
          -- ([xshift=-1mm,yshift=-1mm]frame.north west)
          [sharp corners]-- cycle;
        },interior engine=empty,
    },
  fonttitle=\bfseries,
  title={#2},#1}{def}



%================================
% Solution BOX
%================================

\makeatletter
\newtcbtheorem{question}{Question}{enhanced,
  breakable,
  colback=white,
  colframe=myb!80!black,
  attach boxed title to top left={yshift*=-\tcboxedtitleheight},
  fonttitle=\bfseries,
  title={#2},
  boxed title size=title,
  boxed title style={%
      sharp corners,
      rounded corners=northwest,
      colback=tcbcolframe,
      boxrule=0pt,
    },
  underlay boxed title={%
      \path[fill=tcbcolframe] (title.south west)--(title.south east)
      to[out=0, in=180] ([xshift=5mm]title.east)--
      (title.center-|frame.east)
      [rounded corners=\kvtcb@arc] |-
      (frame.north) -| cycle;
    },
  #1
}{def}
\makeatother

%================================
% SOLUTION BOX
%================================

\makeatletter
\newtcolorbox{solution}{enhanced,
  breakable,
  colback=white,
  colframe=myg!80!black,
  attach boxed title to top left={yshift*=-\tcboxedtitleheight},
  title=Solution,
  boxed title size=title,
  boxed title style={%
      sharp corners,
      rounded corners=northwest,
      colback=tcbcolframe,
      boxrule=0pt,
    },
  underlay boxed title={%
      \path[fill=tcbcolframe] (title.south west)--(title.south east)
      to[out=0, in=180] ([xshift=5mm]title.east)--
      (title.center-|frame.east)
      [rounded corners=\kvtcb@arc] |-
      (frame.north) -| cycle;
    },
}
\makeatother

%================================
% Question BOX
%================================

\makeatletter
\newtcbtheorem{qstion}{Question}{enhanced,
  breakable,
  colback=white,
  colframe=mygr,
  attach boxed title to top left={yshift*=-\tcboxedtitleheight},
  fonttitle=\bfseries,
  title={#2},
  boxed title size=title,
  boxed title style={%
      sharp corners,
      rounded corners=northwest,
      colback=tcbcolframe,
      boxrule=0pt,
    },
  underlay boxed title={%
      \path[fill=tcbcolframe] (title.south west)--(title.south east)
      to[out=0, in=180] ([xshift=5mm]title.east)--
      (title.center-|frame.east)
      [rounded corners=\kvtcb@arc] |-
      (frame.north) -| cycle;
    },
  #1
}{def}
\makeatother

\newtcbtheorem[number within=chapter]{wconc}{Wrong Concept}{
  breakable,
  enhanced,
  colback=white,
  colframe=myr,
  arc=0pt,
  outer arc=0pt,
  fonttitle=\bfseries\sffamily\large,
  colbacktitle=myr,
  attach boxed title to top left={},
  boxed title style={
      enhanced,
      skin=enhancedfirst jigsaw,
      arc=3pt,
      bottom=0pt,
      interior style={fill=myr}
    },
  #1
}{def}



%================================
% NOTE BOX
%================================

\usetikzlibrary{arrows,calc,shadows.blur}
\tcbuselibrary{skins}
\newtcolorbox{note}[1][]{%
  enhanced jigsaw,
  colback=gray!20!white,%
  colframe=gray!80!black,
  size=small,
  boxrule=1pt,
  title=\textbf{Note:-},
  halign title=flush center,
  coltitle=black,
  breakable,
  drop shadow=black!50!white,
  attach boxed title to top left={xshift=1cm,yshift=-\tcboxedtitleheight/2,yshifttext=-\tcboxedtitleheight/2},
  minipage boxed title=1.5cm,
  boxed title style={%
      colback=white,
      size=fbox,
      boxrule=1pt,
      boxsep=2pt,
      underlay={%
          \coordinate (dotA) at ($(interior.west) + (-0.5pt,0)$);
          \coordinate (dotB) at ($(interior.east) + (0.5pt,0)$);
          \begin{scope}
            \clip (interior.north west) rectangle ([xshift=3ex]interior.east);
            \filldraw [white, blur shadow={shadow opacity=60, shadow yshift=-.75ex}, rounded corners=2pt] (interior.north west) rectangle (interior.south east);
          \end{scope}
          \begin{scope}[gray!80!black]
            \fill (dotA) circle (2pt);
            \fill (dotB) circle (2pt);
          \end{scope}
        },
    },
  #1,
}

%%%%%%%%%%%%%%%%%%%%%%%%%%%%%%
% SELF MADE COMMANDS
%%%%%%%%%%%%%%%%%%%%%%%%%%%%%%


\newcommand{\thm}[2]{\begin{Theorem}{#1}{}#2\end{Theorem}}
\newcommand{\cor}[2]{\begin{Corollary}{#1}{}#2\end{Corollary}}
\newcommand{\mlenma}[2]{\begin{Lenma}{#1}{}#2\end{Lenma}}
\newcommand{\mprop}[2]{\begin{Prop}{#1}{}#2\end{Prop}}
\newcommand{\clm}[3]{\begin{claim}{#1}{#2}#3\end{claim}}
\newcommand{\wc}[2]{\begin{wconc}{#1}{}\setlength{\parindent}{1cm}#2\end{wconc}}
\newcommand{\thmcon}[1]{\begin{Theoremcon}{#1}\end{Theoremcon}}
\newcommand{\ex}[2]{\begin{Example}{#1}{}#2\end{Example}}
\newcommand{\dfn}[2]{\begin{Definition}[colbacktitle=red!75!black]{#1}{}#2\end{Definition}}
\newcommand{\dfnc}[2]{\begin{definition}[colbacktitle=red!75!black]{#1}{}#2\end{definition}}
\newcommand{\qs}[2]{\begin{question}{#1}{}#2\end{question}}
\newcommand{\pf}[2]{\begin{myproof}[#1]#2\end{myproof}}
\newcommand{\nt}[1]{\begin{note}#1\end{note}}

\newcommand*\circled[1]{\tikz[baseline=(char.base)]{
    \node[shape=circle,draw,inner sep=1pt] (char) {#1};}}
\newcommand\getcurrentref[1]{%
  \ifnumequal{\value{#1}}{0}
  {??}
  {\the\value{#1}}%
}
\newcommand{\getCurrentSectionNumber}{\getcurrentref{section}}
\newenvironment{myproof}[1][\proofname]{%
  \proof[\bfseries #1: ]%
}{\endproof}

\newcommand{\mclm}[2]{\begin{myclaim}[#1]#2\end{myclaim}}
\newenvironment{myclaim}[1][\claimname]{\proof[\bfseries #1: ]}{}

\newcounter{mylabelcounter}

\makeatletter
\newcommand{\setword}[2]{%
  \phantomsection
  #1\def\@currentlabel{\unexpanded{#1}}\label{#2}%
}
\makeatother




\tikzset{
  symbol/.style={
      draw=none,
      every to/.append style={
          edge node={node [sloped, allow upside down, auto=false]{$#1$}}}
    }
}


% deliminators
\DeclarePairedDelimiter{\abs}{\lvert}{\rvert}
\DeclarePairedDelimiter{\norm}{\lVert}{\rVert}

\DeclarePairedDelimiter{\ceil}{\lceil}{\rceil}
\DeclarePairedDelimiter{\floor}{\lfloor}{\rfloor}
\DeclarePairedDelimiter{\round}{\lfloor}{\rceil}

\newsavebox\diffdbox
\newcommand{\slantedromand}{{\mathpalette\makesl{d}}}
\newcommand{\makesl}[2]{%
  \begingroup
  \sbox{\diffdbox}{$\mathsurround=0pt#1\mathrm{#2}$}%
  \pdfsave
  \pdfsetmatrix{1 0 0.2 1}%
  \rlap{\usebox{\diffdbox}}%
  \pdfrestore
  \hskip\wd\diffdbox
  \endgroup
}
\newcommand{\dd}[1][]{\ensuremath{\mathop{}\!\ifstrempty{#1}{%
      \slantedromand\@ifnextchar^{\hspace{0.2ex}}{\hspace{0.1ex}}}%
    {\slantedromand\hspace{0.2ex}^{#1}}}}
\ProvideDocumentCommand\dv{o m g}{%
  \ensuremath{%
    \IfValueTF{#3}{%
      \IfNoValueTF{#1}{%
        \frac{\dd #2}{\dd #3}%
      }{%
        \frac{\dd^{#1} #2}{\dd #3^{#1}}%
      }%
    }{%
      \IfNoValueTF{#1}{%
        \frac{\dd}{\dd #2}%
      }{%
        \frac{\dd^{#1}}{\dd #2^{#1}}%
      }%
    }%
  }%
}
\providecommand*{\pdv}[3][]{\frac{\partial^{#1}#2}{\partial#3^{#1}}}
%  - others
\DeclareMathOperator{\Lap}{\mathcal{L}}
\DeclareMathOperator{\Var}{Var} % varience
\DeclareMathOperator{\Cov}{Cov} % covarience
\DeclareMathOperator{\E}{E} % expected

% Since the amsthm package isn't loaded

% I prefer the slanted \leq
\let\oldleq\leq % save them in case they're every wanted
\let\oldgeq\geq
\renewcommand{\leq}{\leqslant}
\renewcommand{\geq}{\geqslant}

% % redefine matrix env to allow for alignment, use r as default
% \renewcommand*\env@matrix[1][r]{\hskip -\arraycolsep
%     \let\@ifnextchar\new@ifnextchar
%     \array{*\c@MaxMatrixCols #1}}


%\usepackage{framed}
%\usepackage{titletoc}
%\usepackage{etoolbox}
%\usepackage{lmodern}


%\patchcmd{\tableofcontents}{\contentsname}{\sffamily\contentsname}{}{}

%\renewenvironment{leftbar}
%{\def\FrameCommand{\hspace{6em}%
%		{\color{myyellow}\vrule width 2pt depth 6pt}\hspace{1em}}%
%	\MakeFramed{\parshape 1 0cm \dimexpr\textwidth-6em\relax\FrameRestore}\vskip2pt%
%}
%{\endMakeFramed}

%\titlecontents{chapter}
%[0em]{\vspace*{2\baselineskip}}
%{\parbox{4.5em}{%
%		\hfill\Huge\sffamily\bfseries\color{myred}\thecontentspage}%
%	\vspace*{-2.3\baselineskip}\leftbar\textsc{\small\chaptername~\thecontentslabel}\\\sffamily}
%{}{\endleftbar}
%\titlecontents{section}
%[8.4em]
%{\sffamily\contentslabel{3em}}{}{}
%{\hspace{0.5em}\nobreak\itshape\color{myred}\contentspage}
%\titlecontents{subsection}
%[8.4em]
%{\sffamily\contentslabel{3em}}{}{}  
%{\hspace{0.5em}\nobreak\itshape\color{myred}\contentspage}



%%%%%%%%%%%%%%%%%%%%%%%%%%%%%%%%%%%%%%%%%%%
% TABLE OF CONTENTS
%%%%%%%%%%%%%%%%%%%%%%%%%%%%%%%%%%%%%%%%%%%

\usepackage{tikz}
\definecolor{doc}{RGB}{0,60,110}
\usepackage{titletoc}
\contentsmargin{0cm}
\titlecontents{chapter}[3.7pc]
{\addvspace{30pt}%
  \begin{tikzpicture}[remember picture, overlay]%
    \draw[fill=doc!60,draw=doc!60] (-7,-.1) rectangle (-0.9,.5);%
    \pgftext[left,x=-3.5cm,y=0.2cm]{\color{white}\Large\sc\bfseries Chapter\ \thecontentslabel};%
  \end{tikzpicture}\color{doc!60}\large\sc\bfseries}%
{}
{}
{\;\titlerule\;\large\sc\bfseries Page \thecontentspage
  \begin{tikzpicture}[remember picture, overlay]
    \draw[fill=doc!60,draw=doc!60] (2pt,0) rectangle (4,0.1pt);
  \end{tikzpicture}}%
\titlecontents{section}[3.7pc]
{\addvspace{2pt}}
{\contentslabel[\thecontentslabel]{2pc}}
{}
{\hfill\small \thecontentspage}
[]
\titlecontents*{subsection}[3.7pc]
{\addvspace{-1pt}\small}
{}
{}
{\ --- \small\thecontentspage}
[ \textbullet\ ][]

\makeatletter
\renewcommand{\tableofcontents}{%
  \chapter*{%
    \vspace*{-20\p@}%
    \begin{tikzpicture}[remember picture, overlay]%
      \pgftext[right,x=15cm,y=0.2cm]{\color{doc!60}\Huge\sc\bfseries \contentsname};%
      \draw[fill=doc!60,draw=doc!60] (13,-.75) rectangle (20,1);%
      \clip (13,-.75) rectangle (20,1);
      \pgftext[right,x=15cm,y=0.2cm]{\color{white}\Huge\sc\bfseries \contentsname};%
    \end{tikzpicture}}%
  \@starttoc{toc}}
\makeatother

%From M275 "Topology" at SJSU
\newcommand{\id}{\mathrm{id}}
\newcommand{\taking}[1]{\xrightarrow{#1}}
\newcommand{\inv}{^{-1}}

%From M170 "Introduction to Graph Theory" at SJSU
\DeclareMathOperator{\diam}{diam}
\DeclareMathOperator{\ord}{ord}
\newcommand{\defeq}{\overset{\mathrm{def}}{=}}

%From the USAMO .tex files
\newcommand{\ts}{\textsuperscript}
\newcommand{\dg}{^\circ}
\newcommand{\ii}{\item}

% % From Math 55 and Math 145 at Harvard
% \newenvironment{subproof}[1][Proof]{%
% \begin{proof}[#1] \renewcommand{\qedsymbol}{$\blacksquare$}}%
% {\end{proof}}

\newcommand{\liff}{\leftrightarrow}
\newcommand{\lthen}{\rightarrow}
\newcommand{\opname}{\operatorname}
\newcommand{\surjto}{\twoheadrightarrow}
\newcommand{\injto}{\hookrightarrow}
\newcommand{\On}{\mathrm{On}} % ordinals
\DeclareMathOperator{\img}{im} % Image
\DeclareMathOperator{\Img}{Im} % Image
\DeclareMathOperator{\coker}{coker} % Cokernel
\DeclareMathOperator{\Coker}{Coker} % Cokernel
\DeclareMathOperator{\Ker}{Ker} % Kernel
\DeclareMathOperator{\rank}{rank}
\DeclareMathOperator{\Spec}{Spec} % spectrum
\DeclareMathOperator{\Tr}{Tr} % trace
\DeclareMathOperator{\pr}{pr} % projection
\DeclareMathOperator{\ext}{ext} % extension
\DeclareMathOperator{\pred}{pred} % predecessor
\DeclareMathOperator{\dom}{dom} % domain
\DeclareMathOperator{\ran}{ran} % range
\DeclareMathOperator{\Hom}{Hom} % homomorphism
\DeclareMathOperator{\Mor}{Mor} % morphisms
\DeclareMathOperator{\End}{End} % endomorphism

\newcommand{\eps}{\epsilon}
\newcommand{\veps}{\varepsilon}
\newcommand{\ol}{\overline}
\newcommand{\ul}{\underline}
\newcommand{\wt}{\widetilde}
\newcommand{\wh}{\widehat}
\newcommand{\vocab}[1]{\textbf{\color{blue} #1}}
\providecommand{\half}{\frac{1}{2}}
\newcommand{\dang}{\measuredangle} %% Directed angle
\newcommand{\ray}[1]{\overrightarrow{#1}}
\newcommand{\seg}[1]{\overline{#1}}
\newcommand{\arc}[1]{\wideparen{#1}}
\DeclareMathOperator{\cis}{cis}
\DeclareMathOperator*{\lcm}{lcm}
\DeclareMathOperator*{\argmin}{arg min}
\DeclareMathOperator*{\argmax}{arg max}
\newcommand{\cycsum}{\sum_{\mathrm{cyc}}}
\newcommand{\symsum}{\sum_{\mathrm{sym}}}
\newcommand{\cycprod}{\prod_{\mathrm{cyc}}}
\newcommand{\symprod}{\prod_{\mathrm{sym}}}
\newcommand{\Qed}{\begin{flushright}\qed\end{flushright}}
\newcommand{\parinn}{\setlength{\parindent}{1cm}}
\newcommand{\parinf}{\setlength{\parindent}{0cm}}
% \newcommand{\norm}{\|\cdot\|}
\newcommand{\inorm}{\norm_{\infty}}
\newcommand{\opensets}{\{V_{\alpha}\}_{\alpha\in I}}
\newcommand{\oset}{V_{\alpha}}
\newcommand{\opset}[1]{V_{\alpha_{#1}}}
\newcommand{\lub}{\text{lub}}
\newcommand{\del}[2]{\frac{\partial #1}{\partial #2}}
\newcommand{\Del}[3]{\frac{\partial^{#1} #2}{\partial^{#1} #3}}
\newcommand{\deld}[2]{\dfrac{\partial #1}{\partial #2}}
\newcommand{\Deld}[3]{\dfrac{\partial^{#1} #2}{\partial^{#1} #3}}
\newcommand{\lm}{\lambda}
\newcommand{\uin}{\mathbin{\rotatebox[origin=c]{90}{$\in$}}}
\newcommand{\usubset}{\mathbin{\rotatebox[origin=c]{90}{$\subset$}}}
\newcommand{\lt}{\left}
\newcommand{\rt}{\right}
\newcommand{\bs}[1]{\boldsymbol{#1}}
\newcommand{\exs}{\exists}
\newcommand{\st}{\strut}
\newcommand{\dps}[1]{\displaystyle{#1}}

\newcommand{\sol}{\setlength{\parindent}{0cm}\textbf{\textit{Solution:}}\setlength{\parindent}{1cm} }
\newcommand{\solve}[1]{\setlength{\parindent}{0cm}\textbf{\textit{Solution: }}\setlength{\parindent}{1cm}#1 \Qed}

\preto\tabular{\setcounter{magicrownumbers}{0}}
\newcounter{magicrownumbers}
\newcommand\rownumber{\stepcounter{magicrownumbers}\arabic{magicrownumbers}}
\def\rownumber{}

\newenvironment{deduction}
{\begin{tabular}{@{}>{$}c<{$}@{\enspace}>{$}l<{$}@{}}\arrayrulecolor{blue!50}}
		{\end{tabular}}
\newcommand{\premise}[1]{&#1\\}
\newcommand{\conclusion}[1]{\cline{2-2}\therefore&#1}


% Things Lie
\newcommand{\kb}{\mathfrak b}
\newcommand{\kg}{\mathfrak g}
\newcommand{\kh}{\mathfrak h}
\newcommand{\kn}{\mathfrak n}
\newcommand{\ku}{\mathfrak u}
\newcommand{\kz}{\mathfrak z}
\DeclareMathOperator{\Ext}{Ext} % Ext functor
\DeclareMathOperator{\Tor}{Tor} % Tor functor
\newcommand{\gl}{\opname{\mathfrak{gl}}} % frak gl group
\renewcommand{\sl}{\opname{\mathfrak{sl}}} % frak sl group chktex 6

% More script letters etc.
\newcommand{\SA}{\mathcal A}
\newcommand{\SB}{\mathcal B}
\newcommand{\SC}{\mathcal C}
\newcommand{\SF}{\mathcal F}
\newcommand{\SG}{\mathcal G}
\newcommand{\SH}{\mathcal H}
\newcommand{\OO}{\mathcal O}

\newcommand{\SCA}{\mathscr A}
\newcommand{\SCB}{\mathscr B}
\newcommand{\SCC}{\mathscr C}
\newcommand{\SCD}{\mathscr D}
\newcommand{\SCE}{\mathscr E}
\newcommand{\SCF}{\mathscr F}
\newcommand{\SCG}{\mathscr G}
\newcommand{\SCH}{\mathscr H}

% Mathfrak primes
\newcommand{\km}{\mathfrak m}
\newcommand{\kp}{\mathfrak p}
\newcommand{\kq}{\mathfrak q}

% number sets
\newcommand{\RR}[1][]{\ensuremath{\ifstrempty{#1}{\mathbb{R}}{\mathbb{R}^{#1}}}}
\newcommand{\NN}[1][]{\ensuremath{\ifstrempty{#1}{\mathbb{N}}{\mathbb{N}^{#1}}}}
\newcommand{\ZZ}[1][]{\ensuremath{\ifstrempty{#1}{\mathbb{Z}}{\mathbb{Z}^{#1}}}}
\newcommand{\QQ}[1][]{\ensuremath{\ifstrempty{#1}{\mathbb{Q}}{\mathbb{Q}^{#1}}}}
\newcommand{\CC}[1][]{\ensuremath{\ifstrempty{#1}{\mathbb{C}}{\mathbb{C}^{#1}}}}
\newcommand{\PP}[1][]{\ensuremath{\ifstrempty{#1}{\mathbb{P}}{\mathbb{P}^{#1}}}}
\newcommand{\HH}[1][]{\ensuremath{\ifstrempty{#1}{\mathbb{H}}{\mathbb{H}^{#1}}}}
\newcommand{\FF}[1][]{\ensuremath{\ifstrempty{#1}{\mathbb{F}}{\mathbb{F}^{#1}}}}
% expected value
\newcommand{\EE}{\ensuremath{\mathbb{E}}}
\newcommand{\charin}{\text{ char }}
\DeclareMathOperator{\sign}{sign}
\DeclareMathOperator{\Aut}{Aut}
\DeclareMathOperator{\Inn}{Inn}
\DeclareMathOperator{\Syl}{Syl}
\DeclareMathOperator{\Gal}{Gal}
\DeclareMathOperator{\GL}{GL} % General linear group
\DeclareMathOperator{\SL}{SL} % Special linear group

%---------------------------------------
% BlackBoard Math Fonts :-
%---------------------------------------

%Captital Letters
\newcommand{\bbA}{\mathbb{A}}	\newcommand{\bbB}{\mathbb{B}}
\newcommand{\bbC}{\mathbb{C}}	\newcommand{\bbD}{\mathbb{D}}
\newcommand{\bbE}{\mathbb{E}}	\newcommand{\bbF}{\mathbb{F}}
\newcommand{\bbG}{\mathbb{G}}	\newcommand{\bbH}{\mathbb{H}}
\newcommand{\bbI}{\mathbb{I}}	\newcommand{\bbJ}{\mathbb{J}}
\newcommand{\bbK}{\mathbb{K}}	\newcommand{\bbL}{\mathbb{L}}
\newcommand{\bbM}{\mathbb{M}}	\newcommand{\bbN}{\mathbb{N}}
\newcommand{\bbO}{\mathbb{O}}	\newcommand{\bbP}{\mathbb{P}}
\newcommand{\bbQ}{\mathbb{Q}}	\newcommand{\bbR}{\mathbb{R}}
\newcommand{\bbS}{\mathbb{S}}	\newcommand{\bbT}{\mathbb{T}}
\newcommand{\bbU}{\mathbb{U}}	\newcommand{\bbV}{\mathbb{V}}
\newcommand{\bbW}{\mathbb{W}}	\newcommand{\bbX}{\mathbb{X}}
\newcommand{\bbY}{\mathbb{Y}}	\newcommand{\bbZ}{\mathbb{Z}}

%---------------------------------------
% MathCal Fonts :-
%---------------------------------------

%Captital Letters
\newcommand{\mcA}{\mathcal{A}}	\newcommand{\mcB}{\mathcal{B}}
\newcommand{\mcC}{\mathcal{C}}	\newcommand{\mcD}{\mathcal{D}}
\newcommand{\mcE}{\mathcal{E}}	\newcommand{\mcF}{\mathcal{F}}
\newcommand{\mcG}{\mathcal{G}}	\newcommand{\mcH}{\mathcal{H}}
\newcommand{\mcI}{\mathcal{I}}	\newcommand{\mcJ}{\mathcal{J}}
\newcommand{\mcK}{\mathcal{K}}	\newcommand{\mcL}{\mathcal{L}}
\newcommand{\mcM}{\mathcal{M}}	\newcommand{\mcN}{\mathcal{N}}
\newcommand{\mcO}{\mathcal{O}}	\newcommand{\mcP}{\mathcal{P}}
\newcommand{\mcQ}{\mathcal{Q}}	\newcommand{\mcR}{\mathcal{R}}
\newcommand{\mcS}{\mathcal{S}}	\newcommand{\mcT}{\mathcal{T}}
\newcommand{\mcU}{\mathcal{U}}	\newcommand{\mcV}{\mathcal{V}}
\newcommand{\mcW}{\mathcal{W}}	\newcommand{\mcX}{\mathcal{X}}
\newcommand{\mcY}{\mathcal{Y}}	\newcommand{\mcZ}{\mathcal{Z}}


%---------------------------------------
% Bold Math Fonts :-
%---------------------------------------

%Captital Letters
\newcommand{\bmA}{\boldsymbol{A}}	\newcommand{\bmB}{\boldsymbol{B}}
\newcommand{\bmC}{\boldsymbol{C}}	\newcommand{\bmD}{\boldsymbol{D}}
\newcommand{\bmE}{\boldsymbol{E}}	\newcommand{\bmF}{\boldsymbol{F}}
\newcommand{\bmG}{\boldsymbol{G}}	\newcommand{\bmH}{\boldsymbol{H}}
\newcommand{\bmI}{\boldsymbol{I}}	\newcommand{\bmJ}{\boldsymbol{J}}
\newcommand{\bmK}{\boldsymbol{K}}	\newcommand{\bmL}{\boldsymbol{L}}
\newcommand{\bmM}{\boldsymbol{M}}	\newcommand{\bmN}{\boldsymbol{N}}
\newcommand{\bmO}{\boldsymbol{O}}	\newcommand{\bmP}{\boldsymbol{P}}
\newcommand{\bmQ}{\boldsymbol{Q}}	\newcommand{\bmR}{\boldsymbol{R}}
\newcommand{\bmS}{\boldsymbol{S}}	\newcommand{\bmT}{\boldsymbol{T}}
\newcommand{\bmU}{\boldsymbol{U}}	\newcommand{\bmV}{\boldsymbol{V}}
\newcommand{\bmW}{\boldsymbol{W}}	\newcommand{\bmX}{\boldsymbol{X}}
\newcommand{\bmY}{\boldsymbol{Y}}	\newcommand{\bmZ}{\boldsymbol{Z}}
%Small Letters
\newcommand{\bma}{\boldsymbol{a}}	\newcommand{\bmb}{\boldsymbol{b}}
\newcommand{\bmc}{\boldsymbol{c}}	\newcommand{\bmd}{\boldsymbol{d}}
\newcommand{\bme}{\boldsymbol{e}}	\newcommand{\bmf}{\boldsymbol{f}}
\newcommand{\bmg}{\boldsymbol{g}}	\newcommand{\bmh}{\boldsymbol{h}}
\newcommand{\bmi}{\boldsymbol{i}}	\newcommand{\bmj}{\boldsymbol{j}}
\newcommand{\bmk}{\boldsymbol{k}}	\newcommand{\bml}{\boldsymbol{l}}
\newcommand{\bmm}{\boldsymbol{m}}	\newcommand{\bmn}{\boldsymbol{n}}
\newcommand{\bmo}{\boldsymbol{o}}	\newcommand{\bmp}{\boldsymbol{p}}
\newcommand{\bmq}{\boldsymbol{q}}	\newcommand{\bmr}{\boldsymbol{r}}
\newcommand{\bms}{\boldsymbol{s}}	\newcommand{\bmt}{\boldsymbol{t}}
\newcommand{\bmu}{\boldsymbol{u}}	\newcommand{\bmv}{\boldsymbol{v}}
\newcommand{\bmw}{\boldsymbol{w}}	\newcommand{\bmx}{\boldsymbol{x}}
\newcommand{\bmy}{\boldsymbol{y}}	\newcommand{\bmz}{\boldsymbol{z}}

%---------------------------------------
% Scr Math Fonts :-
%---------------------------------------

\newcommand{\sA}{{\mathscr{A}}}   \newcommand{\sB}{{\mathscr{B}}}
\newcommand{\sC}{{\mathscr{C}}}   \newcommand{\sD}{{\mathscr{D}}}
\newcommand{\sE}{{\mathscr{E}}}   \newcommand{\sF}{{\mathscr{F}}}
\newcommand{\sG}{{\mathscr{G}}}   \newcommand{\sH}{{\mathscr{H}}}
\newcommand{\sI}{{\mathscr{I}}}   \newcommand{\sJ}{{\mathscr{J}}}
\newcommand{\sK}{{\mathscr{K}}}   \newcommand{\sL}{{\mathscr{L}}}
\newcommand{\sM}{{\mathscr{M}}}   \newcommand{\sN}{{\mathscr{N}}}
\newcommand{\sO}{{\mathscr{O}}}   \newcommand{\sP}{{\mathscr{P}}}
\newcommand{\sQ}{{\mathscr{Q}}}   \newcommand{\sR}{{\mathscr{R}}}
\newcommand{\sS}{{\mathscr{S}}}   \newcommand{\sT}{{\mathscr{T}}}
\newcommand{\sU}{{\mathscr{U}}}   \newcommand{\sV}{{\mathscr{V}}}
\newcommand{\sW}{{\mathscr{W}}}   \newcommand{\sX}{{\mathscr{X}}}
\newcommand{\sY}{{\mathscr{Y}}}   \newcommand{\sZ}{{\mathscr{Z}}}


%---------------------------------------
% Math Fraktur Font
%---------------------------------------

%Captital Letters
\newcommand{\mfA}{\mathfrak{A}}	\newcommand{\mfB}{\mathfrak{B}}
\newcommand{\mfC}{\mathfrak{C}}	\newcommand{\mfD}{\mathfrak{D}}
\newcommand{\mfE}{\mathfrak{E}}	\newcommand{\mfF}{\mathfrak{F}}
\newcommand{\mfG}{\mathfrak{G}}	\newcommand{\mfH}{\mathfrak{H}}
\newcommand{\mfI}{\mathfrak{I}}	\newcommand{\mfJ}{\mathfrak{J}}
\newcommand{\mfK}{\mathfrak{K}}	\newcommand{\mfL}{\mathfrak{L}}
\newcommand{\mfM}{\mathfrak{M}}	\newcommand{\mfN}{\mathfrak{N}}
\newcommand{\mfO}{\mathfrak{O}}	\newcommand{\mfP}{\mathfrak{P}}
\newcommand{\mfQ}{\mathfrak{Q}}	\newcommand{\mfR}{\mathfrak{R}}
\newcommand{\mfS}{\mathfrak{S}}	\newcommand{\mfT}{\mathfrak{T}}
\newcommand{\mfU}{\mathfrak{U}}	\newcommand{\mfV}{\mathfrak{V}}
\newcommand{\mfW}{\mathfrak{W}}	\newcommand{\mfX}{\mathfrak{X}}
\newcommand{\mfY}{\mathfrak{Y}}	\newcommand{\mfZ}{\mathfrak{Z}}
%Small Letters
\newcommand{\mfa}{\mathfrak{a}}	\newcommand{\mfb}{\mathfrak{b}}
\newcommand{\mfc}{\mathfrak{c}}	\newcommand{\mfd}{\mathfrak{d}}
\newcommand{\mfe}{\mathfrak{e}}	\newcommand{\mff}{\mathfrak{f}}
\newcommand{\mfg}{\mathfrak{g}}	\newcommand{\mfh}{\mathfrak{h}}
\newcommand{\mfi}{\mathfrak{i}}	\newcommand{\mfj}{\mathfrak{j}}
\newcommand{\mfk}{\mathfrak{k}}	\newcommand{\mfl}{\mathfrak{l}}
\newcommand{\mfm}{\mathfrak{m}}	\newcommand{\mfn}{\mathfrak{n}}
\newcommand{\mfo}{\mathfrak{o}}	\newcommand{\mfp}{\mathfrak{p}}
\newcommand{\mfq}{\mathfrak{q}}	\newcommand{\mfr}{\mathfrak{r}}
\newcommand{\mfs}{\mathfrak{s}}	\newcommand{\mft}{\mathfrak{t}}
\newcommand{\mfu}{\mathfrak{u}}	\newcommand{\mfv}{\mathfrak{v}}
\newcommand{\mfw}{\mathfrak{w}}	\newcommand{\mfx}{\mathfrak{x}}
\newcommand{\mfy}{\mathfrak{y}}	\newcommand{\mfz}{\mathfrak{z}}


\title{\Huge{Matrix Algebra}}
\author{\huge{Madiba Hudson-Quansah}}
\date{}
\usepackage{parskip}

\setcounter{tocdepth}{4}
\setcounter{secnumdepth}{4}

\begin{document}
\maketitle
\newpage
\pdfbookmark[section]{\contentsname}{too}
\tableofcontents
\pagebreak

\chapter{Matrix Operations}


If $A$ is a $n \times m$ matrix then the scalar entry in the $i$th row and the $j$th column of $A$ is denoted by
$a_{ij}$, and is called the $ \left( i, j \right) $-entry. Each column of $A$ is a list of $m$ real numbers in the
$\mathbb{R}^{m}$ vector space. Therefore the columns of $A$ can be represented as vectors in $\mathbb{R}^{m}$:
\[
  A = \begin{bmatrix} \mbold{a}_1 & \mbold{a}_2 & \ldots \mbold{a}_n \end{bmatrix}
\]

\dfn{Diagonals}{
  The diagonal entries of a matrix $A$ of dimension $n \times m$, are the entries $a_{ij}$, where $i = j$. This is
  called the \textbf{main diagonal} of the matrix $A$. A \textbf{diagonal matrix} is a square matrix $n\times n$ whose
  non-diagonal entries are all zero.
}

\section{Sums and Scalar Multiples}

\dfn{Equality of Matrices}{
  Two matrices $A$ and $B$, are equal if:
  \begin{itemize}
    \item The are of the same size i.e, $m\times x$
    \item The corresponding entries are equal i.e, $A_{ij} = B_{ij}$
  \end{itemize}
}

\thm{Axioms of Matrix Addition}{
  Let $A, B$ and $C$ be matrices of the same size, and let $r$ and $s$ be scalars. Then the following axioms hold:
  \begin{description}
    \item[Communtativity] $A + B = B + A$
    \item[Associativity] $ \left( A + B \right) + C = A + \left( B + C\right)  $
    \item[Additive Identity] $A + 0 = A$
    \item[Distruibutivity 1] $r\left( A + B \right) = rA + rB$
    \item[Distruibutivity 2] $ \left( r + s \right) A = rA + sA $
    \item[Compatibility with Scalar Multiplication] $r\left( sA \right) = \left( rs \right)A$
  \end{description}
}

\section{Matrix Multiplication}

When a matrix $B$ multiples a vector $\mbold{x}$, it transforms $\mbold{x}$ into the vector $B\mbold{x}$. If this vector
is then multiples by another matrix $A$, the result is the vector $A \left( B \mbold{x}\right) $. Thus $A \left( B
  \mbold{x}\right) $ is produced by a composition of mappings / linear transformations. This can be also expressed as:
\[
  A \left( B \mbold{x}\right) = \left( AB \right) \mbold{x}
\]

Because, if $A$ is $m\times n$, $B$ is $n\times p$ and $\mbold{x}$ is in $\mathbb{R}^{p}$, can denote the columns of
$B$, by $\mbold{b}_1, \ldots, \mbold{b}_p$ and the entries of $\mbold{x}$ by, $x_1, \ldots, x_p$. Then
\[
  B \mbold{x} = x_1 \mbold{b}_1 + \ldots + x_p \mbold{b}_p
\]
By the linearity of matrix multiplication, we have:
\begin{align*}
  A \left( B \mbold{x} \right) & = A \left( x_1 \mbold{b}_1 \right) + \ldots + A \left( x_p \mbold{b}_p \right) \\
                               & = x_1 \left( A\mbold{b}_1 \right) + \ldots + x_p \left( A \mbold{b}_p \right)  \\
\end{align*}
The vector $A \left( B\mbold{x} \right) $ is then a linear combination of the vectors $A\mbold{b}_1, \ldots,
  A\mbold{b}_p$, using the entries of $\mbold{x}$ as weights. This can be expressed in matrix notation as:
\[
  A \left( B \mbold{x} \right)  = \begin{bmatrix} A \mbold{b}_1 & A \mbold{b}_2 & \ldots & A \mbold{b}_p \end{bmatrix}
  \mbold{x}
\]
\thm{}{
  If $A$ is an $m \times n$ matrix, and if $B$ is an $n \times p$ matrix with columns $\mbold{b}_1, \ldots, \mbold{b}_p$,
  then the product $AB$ is the $m \times p$ matrix whose columns are $A\mbold{b}_1, \ldots, A\mbold{b}_p$.That is:
  \[
    A \left( B \mbold{x} \right)  = \begin{bmatrix} A \mbold{b}_1 & A \mbold{b}_2 & \ldots & A \mbold{b}_p \end{bmatrix}
    \mbold{x}
  \]
}

\ex{}{
  \qs{}{
    Compute $AB$ where $A = \begin{bmatrix} 2 & 3 \\ 1 & -5 \end{bmatrix} $, and $B = \begin{bmatrix} 4  & 3 & 6 \\ 1 &
                -2 & 3\end{bmatrix} $
  }

  \sol{
    \begin{align*}
      A \mbold{b}_1 & = \begin{bmatrix} 2 & 3 \\ 1 & -5 \end{bmatrix} \begin{bmatrix} 4 \\ 1 \end{bmatrix}  \\
                    & = \begin{bmatrix}
                          8 + 3 \\
                          4 + -5
                        \end{bmatrix}                                                                      \\
                    & = \begin{bmatrix}
                          11 \\
                          -1
                        \end{bmatrix}                                                                      \\
      \\
      A \mbold{b}_2 & = \begin{bmatrix} 2 & 3 \\ 1 & -5 \end{bmatrix} \begin{bmatrix} 3 \\ -2 \end{bmatrix} \\
                    & = \begin{bmatrix}
                          6 - 6 \\
                          3 + 10
                        \end{bmatrix}                                                                      \\
                    & = \begin{bmatrix}
                          0 \\
                          13
                        \end{bmatrix}                                                                      \\
      \\
      A \mbold{b}_3 & = \begin{bmatrix} 2 & 3 \\ 1 & -5 \end{bmatrix} \begin{bmatrix} 6 \\ 3 \end{bmatrix}  \\
                    & = \begin{bmatrix}
                          21 \\
                          -9
                        \end{bmatrix}                                                                      \\
      \\
      AB            & =  \begin{bmatrix}
                           11 & 0  & 21 \\
                           -1 & 13 & -9
                         \end{bmatrix}                                                                     \\
    \end{align*}
  }
}

\thm{Row-Column Rule}{
If the product $AB$ is defined, them the entry in row $i$ and column $j$ of $AB$ is the sum of the products of
corresponding entries of the row $i$ of $A$ and column $j$ of $B$. If $ \left( A B \right)_{ij} $ denotes the $ \left(
  i, j \right) $-entry in $A B$, and if $A$ is an $m \times n$, then
\[
  \left( A B \right)_{ij} = a_{i1}b_{1j} + a_{i2}b_{2j} + \ldots + a_{in}b_{nj}
\]
}

\ex{}{
  Use the row–column rule to compute two of the entries in $A B$ for the
  matrices:
  \[
    A = \begin{bmatrix}
      2 & 3  \\
      1 & -5
    \end{bmatrix}, \,
    B = \begin{bmatrix}
      4 & 3  & 6 \\
      1 & -2 & 3
    \end{bmatrix}
  \]
  An inspection of the numbers involved will make it clear how
  the two methods for calculating $A B$ produce the same matrix. \\


  The dimensions of the resultant matrix is $2 \times 3$, therefore the entries of $A B$ are:
  \begin{align*}
    A B & = \begin{bmatrix}
              2 \left( 4 \right)+ 3 \left( 1 \right)  & 2 \left( 3 \right)  + 3 \left( -2 \right) & 2 \left( 6 \right) + 3
              \left( 3 \right)                                                                                             \\
              1 \left( 4 \right)  -5 \left( 1 \right) & 1 \left( 3 \right) - 5 \left( -2 \right)  & 1 \left( 6 \right) - 5
              \left( 3 \right)
            \end{bmatrix} \\
        & = \begin{bmatrix}
              11 & 0  & 21 \\
              -1 & 13 & 9  \\
            \end{bmatrix}                                                                                               \\
  \end{align*}
}

\ex{}{
  \qs{}{
    Find the entries in the second row of $AB$ where,
    \[
      A = \begin{bmatrix}
        2  & -5 & 0  \\
        -1 & 3  & -4 \\
        6  & -8 & -7 \\
        -3 & 0  & 9
      \end{bmatrix}, \,
      B = \begin{bmatrix}
        4 & -6 \\
        7 & 1  \\
        3 & 2
      \end{bmatrix}
    \]
  }

  \sol{
    \begin{align*}
      \begin{bmatrix} -1 & 3 & -4 \end{bmatrix} \begin{bmatrix} 4 & -6 \\ 7 & 1 \\ 3 & 2 \end{bmatrix} \\
      \begin{bmatrix}
        -4 + 21 - 12 & 6 + 3 - 8
      \end{bmatrix}                                                                         \\
      \begin{bmatrix} 5 & 1 \end{bmatrix}
    \end{align*}
  }
}

\thm{Axioms of Matrix Multiplication}{
  Let $A$ be an $m \times n$ matrix and let $B$ and $C$ have sizes for which  the indicated sums and products are
  defined:
  \begin{description}
    \item[Associativity]  $A \left( B C \right) = \left( A B \right)  C$
    \item[Left Distruibutivity] $A \left( B + C \right) = A B + A C $
    \item[Right Distruibutivity] $ \left( B + C \right) A = B A + C A $
    \item[Scalar Associativity] $ r \left( A B \right) = \left( r A \right) B = A \left( r B \right), \, \forall r, \, r
            \in \mathbb{F}$
    \item[Mutliplicative Identitiy] $I_{m}A = A = A I_{n} $
  \end{description}
}

\ex{}{
  \qs{}{
    Let $A = \begin{bmatrix} 5 & 1 \\ 3 & -2 \end{bmatrix} $ and $B = \begin{bmatrix} 2 & 0 \\ 4 & 3 \end{bmatrix} $. Show
    that these matrices do not commute, I.e, verify $A B \neq  B A$
  }

  \sol{
    \begin{align*}
      A B            & = \begin{bmatrix} 5 & 1 \\ 3 & -2 \end{bmatrix} \begin{bmatrix} 2 & 0 \\ 4 & 3 \end{bmatrix} \\
                     & = \begin{bmatrix}
                           14 & 3  \\
                           -2 & -6
                         \end{bmatrix}                                                                             \\
      \\
      B A            & = \begin{bmatrix} 2 & 0 \\ 4 & 3 \end{bmatrix} \begin{bmatrix} 5 & 1 \\ 3 & -2 \end{bmatrix} \\
                     & = \begin{bmatrix}
                           10 & 2  \\
                           29 & -2
                         \end{bmatrix}                                                                             \\
      \\
      \therefore A B & \neq BA
    \end{align*}
  }
}

\subsection{Powers of a Matrix}

\dfn{Powers of a Matrix}{
  If $A$ is an $n \times n$ matrix and if $k$ is a positive integer, then $A^k$ denotes the product of $k$ copies of
  $A$:
  \[
    A^{k} = A_1 \ldots A_k
  \]
  Where $A_1 = A_2 \wedge A_2 = A_3 \wedge \ldots \wedge A_{k-1} = A_k$
}

If $A$ is non-zero and if $\mbold{x}$ is in $\mathbb{R}^{n}$, then $A^{k} \mbold{x}$ is the result of left-multiplying
$\mbold{x}$ by $A$ repeatedly $k$ times.

If $k = 0$, then $A^{0} \mbold{x}$ is $\mbold{x}$. Thus $A^{0}$ is interpreted as the Identity matrix.

\section{The Transpose of a Matrix}

\dfn{The Transpose of a Matrix}{
Given a matrix $A$, its \textit{transpose}, denoted by $A^T$, is defined by transforming  the rows of $A$ into columns.
For example:
\[
  \begin{bmatrix} 1 & 2 & 3 \\ 4 & 5 & 6  \end{bmatrix}^{T} = \begin{bmatrix} 1 & 4 \\ 2 & 5 \\ 3 & 6 \end{bmatrix}
\]
Therefore formally, the transpose of a matrix $A_{m,n}$ is defined as:
\[
  A^T_{m,n}  = A_{n,m}
\]
}

Therefore, let $A$ and $B$ denote matrices whose sizes are appropriate for the following sums and products:
\begin{enumerate}
  \item $ \left( A^{T} \right)^{T} = A$
  \item $ \left( A + B \right)^{T} = A^{T} + B^{T} $
  \item $\forall r \in \mathbb{F}, \, \left( r A \right)^{T} = r A^{T}  $
  \item $ \left( A B \right)^{T} = B ^{T} A^{T} $ \label{thm:asctrans}
\end{enumerate}

Usually $ \left( A B \right)^{T} $ is not equal $A ^{T} B^{T}$, even when $A$ and $B$ have dimensions such that $A ^{T}
  B^{T}$ is defined. The generalization of axiom \ref{thm:asctrans} to products more than two factors is as follows:

\thm{}{
  The transpose of a product of matrices equals the product of their transpose in the reverse order.
}

\chapter{The Inverse Of A Matrix}

\section{Invertibility}

\dfn{Invertibility}{
  Let $A = \begin{bmatrix} a & b \\ c & d \end{bmatrix} $. If $ad - bc \neq  0$, then $A$ is invertible and
  \[
    A^{-1} = \frac{1}{ad - bc} \begin{bmatrix} d & -b \\ -c & a \end{bmatrix}
  \]
  If $ad - bc = 0$, then $A$ is not invertible. Where $ad-bc$ is known as the \textit{determinant} \label{thg:det} and denoted by
  \[
    \det A  = ad - bc
  \]
}

\thm{}{
  If $A$ is an invertible $n \times n$ matrix, then for each $\mbold{b}$ in $\mathbb{R}^{n}$, the equation $A \mbold{x}
    = \mbold{b}$ has the unique solution:
  \[
    \mbold{x} = A^{-1} \mbold{b}
  \]
}

\thm{}{
  \begin{enumerate}
    \item If $A$ is an invertible matrix, then $A^{-1}$ is invertible and
          \[
            \left( A^{-1} \right)^{-1} = A
          \]
    \item If $A$ and $B$ are $n \times n$ invertible matrices, then so $AB$, and the inverse of $A B$ is the product of
          the inverses of $A$ and $B$ in the reverse order:
          \[
            \left( A B \right) ^{-1} = B^{-1} A^{-1}
          \]
    \item If $A$ is an invertible matrix, then so is $A^{T}$ and the inverse of $A^T$ is the transpose of $A^{-1}$:
          \[
            \left( A^T \right)^{-1} = \left( A^{-1} \right) ^T
          \]

  \end{enumerate}
}

\section{Elementary Matrices}

\dfn{Elementary Matrix}{
  A matrix obtained by performing a single elementary row operation on an identity matrix.
}

\ex{}{
  \qs{}{
    Let
    \[
      E_1 = \begin{bmatrix} 1 & 0 &0 \\0 & 1 & 0 \\ -4 & 0 & 1  \end{bmatrix}, \, E_2 = \begin{bmatrix} 0 & 1 & 0 \\ 1 & 0
                  & 0     \\ 0 & 0 & 1\end{bmatrix}, \, E_3 = \begin{bmatrix} 1 & 0 & 0 \\ 0 & 1 & 0 \\ 0 & 0 & 5
      \end{bmatrix}, \, A \begin{bmatrix} a & b & c \\ d & e & f \\ g &h&i \end{bmatrix}
    \]
  }
  Compute $E_1A$, $E_2A$, $E_3A$, and describe how these products can be obtained by elementary row operations on $A$.

  \sol{
    \begin{align*}
      E_1 A & =\begin{bmatrix} 1 & 0 &0 \\0 & 1 & 0 \\ -4 & 0 & 1  \end{bmatrix} \begin{bmatrix} a & b & c \\ d & e & f \\
                g & h & i\end{bmatrix} \\
            & = \begin{bmatrix}
                  a      & b       & c       \\
                  d      & e       & f       \\
                  -4a+ g & -4b + h & -4c + 1
                \end{bmatrix}                                                                              \\
      \\
      E_2 A & = \begin{bmatrix} 0 & 1 & 0 \\ 1 & 0
                  & 0     \\ 0 & 0 & 1\end{bmatrix} \begin{bmatrix} a & b & c \\ d & e & f \\ g &h&i \end{bmatrix}      \\
            & = \begin{bmatrix}
                  d & e & f \\
                  a & b & c \\
                  g & h & i
                \end{bmatrix}                                                                                          \\
      \\
      E_3 A & = \begin{bmatrix} 1 & 0 & 0 \\ 0 & 1 & 0 \\ 0 & 0 & 5
                \end{bmatrix} \begin{bmatrix} a & b & c \\ d & e & f \\ g &h&i \end{bmatrix}                            \\
            & = \begin{bmatrix}
                  a  & b  & c  \\
                  d  & e  & f  \\
                  5g & 5h & 5i
                \end{bmatrix}                                                                                          \\
    \end{align*}
    \begin{itemize}
      \item     $E_1A$ could be obtained by the elementary row operation $-4R_1 + R_3 \to R_3$
      \item $E_2A$ could be obtained by the elementary row operation $R_1 \leftrightarrow R_2$
      \item $E_3A$ could be obtained by the elementary row operation $5R_3 \to R_3$
    \end{itemize}
  }
}


\cor{}{
  If an elementary row operation is performed on an $m \times n$ matrix $A$, the resulting matrix can be expressed as
  $E A$, where $E$ is the $m \times m$ matrix created by performing the same row operation on $I_m$
}

Since row operations are reversible, all elementary matrices are invertible. Therefore there exists an elementary matrix
$F$ such that
\[
  F E = I
\]
And since $E$ and $F$ correspond to reverse operations $EF = I$, also.

\ex{}{
  \qs{}{
    Find the inverse of $E_1 = \begin{bmatrix}
        1  & 0 & 0 \\
        0  & 1 & 0 \\
        -4 & 0 & 1
      \end{bmatrix} $
  }

  \sol{
    To transform this matrix into $I_3$ we must get rid of the $-4$ entry in the third row. This can be done by the row
    operation $4R_1 + R_3 \to R_3$, which corresponds to the elementary matrix:
    \[
      E^{-1}_1 = \begin{bmatrix}
        1 & 0 & 0 \\
        0 & 1 & 0 \\
        4 & 0 & 1
      \end{bmatrix}
    \]
    Checking our answer:
    \begin{align*}
      E_1 E^{-1}_1 & = \begin{bmatrix}
                         1  & 0 & 0 \\
                         0  & 1 & 0 \\
                         -4 & 0 & 1
                       \end{bmatrix} \begin{bmatrix}
                                       1 & 0 & 0 \\
                                       0 & 1 & 0 \\
                                       4 & 0 & 1
                                     \end{bmatrix} \\
                   & = \begin{bmatrix}
                         1 & 0 & 0 \\
                         0 & 1 & 0 \\
                         0 & 0 & 1
                       \end{bmatrix}               \\
    \end{align*}
    This is indeed the identity matrix $I_m$

  }
}
\thm{}{
  An $n\times n$ matrix $A$ is invertible if and only if $A$ is row equivalent to $I_n$, and in this case, any sequence
  of elementary row operations that reduces $A$ to $I_n$ also transforms $I_n$ into $A^{-1}$
}

\subsection{Finding $A^{-1}$}

To find the inverse of a matrix $A$, we can augment $A$ with the $n \times n$ identity matrix $I_n$ and then row reduce.
If $A$ is row equivalent to $I_{n}$ then $\begin{bmatrix} A & I \end{bmatrix} $ is row equivalent to $\begin{bmatrix} I
     & A^{-1}\end{bmatrix} $. Otherwise, $A$ does not have an inverse.

\ex{}{
  \qs{}{
    Find the inverse of the matrix $A = \begin{bmatrix}
        0 & 1  & 2 \\
        1 & 0  & 3 \\
        4 & -3 & 8
      \end{bmatrix} $
  }

  \sol{
    \begin{align*}
      \begin{bmatrix} A &I \end{bmatrix}  = \begin{bmatrix}
                                              0 & 1  & 2 & 1 & 0 & 0 \\
                                              1 & 0  & 3 & 0 & 1 & 0 \\
                                              4 & -3 & 8 & 0 & 0 & 1
                                            \end{bmatrix} \\
      R_1 \leftrightarrow R_3                                      \\
      \begin{bmatrix}
        4 & -3 & 8 & 0 & 0 & 1 \\
        1 & 0  & 3 & 0 & 1 & 0 \\
        0 & 1  & 2 & 1 & 0 & 0 \\
      \end{bmatrix}
      \\
      \frac{1}{4}R_1 - R_2  \rightarrow R_2                        \\
      \begin{bmatrix}
        4 & -3           & 8  & 0 & 0  & 1           \\
        0 & \frac{-3}{4} & -1 & 0 & -1 & \frac{1}{4} \\
        0 & 1            & 2  & 1 & 0  & 0           \\
      \end{bmatrix}
      \\
      \frac{-4}{3}R_2 - R_3  \rightarrow R_3                       \\
      \begin{bmatrix}
        4 & -3           & 8            & 0  & 0           & 1            \\
        0 & \frac{-3}{4} & -1           & 0  & -1          & \frac{1}{4}  \\
        0 & 0            & \frac{-2}{3} & -1 & \frac{4}{3} & \frac{-1}{3} \\
      \end{bmatrix}
      \\
      4R_2 - R_1  \rightarrow R_1                                  \\
      \begin{bmatrix}
        -4 & 0            & -12          & 0  & -4          & 0            \\
        0  & \frac{-3}{4} & -1           & 0  & -1          & \frac{1}{4}  \\
        0  & 0            & \frac{-2}{3} & -1 & \frac{4}{3} & \frac{-1}{3} \\
      \end{bmatrix}
      \\
      \frac{3}{2}R_3 - R_2  \rightarrow R_2                        \\
      \begin{bmatrix}
        -4 & 0           & -12          & 0            & -4          & 0            \\
        0  & \frac{3}{4} & 0            & \frac{-3}{2} & 3           & \frac{-3}{4} \\
        0  & 0           & \frac{-2}{3} & -1           & \frac{4}{3} & \frac{-1}{3} \\
      \end{bmatrix}
      \\
      18R_3 - R_1  \rightarrow R_1                                 \\
      \begin{bmatrix}
        4 & 0           & 0            & -18          & 28          & \frac{-6}{1} \\
        0 & \frac{3}{4} & 0            & \frac{-3}{2} & 3           & \frac{-3}{4} \\
        0 & 0           & \frac{-2}{3} & -1           & \frac{4}{3} & \frac{-1}{3} \\
      \end{bmatrix}
      \\
      - R_1  \rightarrow R_1                                       \\
      \begin{bmatrix}
        -4 & 0           & 0            & 18           & -28         & \frac{6}{1}  \\
        0  & \frac{3}{4} & 0            & \frac{-3}{2} & 3           & \frac{-3}{4} \\
        0  & 0           & \frac{-2}{3} & -1           & \frac{4}{3} & \frac{-1}{3} \\
      \end{bmatrix}
      \\
      \frac{-1}{4}R_1 \to R_1                                      \\
      \frac{4}{3}R_2 \to R_2                                       \\
      \frac{-3}{2}R_3 \to R_3                                      \\
      \begin{bmatrix}
        1 & 0 & 0 & \frac{-9}{2} & 7  & \frac{-3}{2} \\
        0 & 1 & 0 & -2           & 4  & -1           \\
        0 & 0 & 1 & \frac{3}{2}  & -2 & \frac{1}{2}  \\
      \end{bmatrix}
      \\
    \end{align*}
    Since $A \sim I$, $A$ is invertible and
    \[
      A^{-1}= \begin{bmatrix}
        \frac{-9}{2} & 7  & \frac{-3}{2} \\
        -2           & 4  & -1           \\
        \frac{3}{2}  & -2 & \frac{1}{2}  \\
      \end{bmatrix}
    \]
    Checking our answer:
    \begin{align*}
      A A^{-1} & = \begin{bmatrix}
                     0 & 1  & 2 \\
                     1 & 0  & 3 \\
                     4 & -3 & 8
                   \end{bmatrix} \begin{bmatrix}
                                   \frac{-9}{2} & 7  & \frac{-3}{2} \\
                                   -2           & 4  & -1           \\
                                   \frac{3}{2}  & -2 & \frac{1}{2}  \\
                                 \end{bmatrix} \\
               & = \begin{bmatrix}
                     1 & 0 & 0 \\
                     0 & 1 & 0 \\
                     0 & 0 & 1
                   \end{bmatrix}                                 \\
    \end{align*}
  }

}

\chapter{Determinants}

\section{Introduction}

To extend the concept of the determinant to  $n \times n$ matrices we must use this recursive definition:

\dfn{The Determinant of a $n\times n$ matrix}{
  For $n \geq 2$, the determinant of an $n \times n$ matrix $A = \left[ a_{ij} \right] $ is the sum of terms of the form
  $\pm a_{1j}$ det$A_{1j}$, with plus and minus signs alternating, where the entries  of $a_{11}, a_{12}, \ldots,
    a_{1n}$ are form the first row of $A$, i.e.:
  \begin{align*}
    \det A & = a_{11} \text{det} A_{11} - a_{12} \text{det} A_{12} + \ldots + (-1)^{1+n} a_{1n} \text{det} A_{1n}
    \\
           & = \displaystyle\sum_{j=1}^{n} \left( -1 \right)^{1 + j} a_{1j} \det A_{1j}                           \\
  \end{align*}
}

Where $A_{1j}$ refers to the matrix obtained by crossing out the first row and the $j$th column of $A$, which if $A$ is
a $3\times 3$ matrix would result in a $2\times 2$ one allowing us to find the determinant of $A_{1j}$ using \ref{thg:det}

\ex{}{
  \qs{}{
    Compute the determinant of
    \[
      A = \begin{bmatrix}
        1 & 5  & 0  \\
        2 & 4  & -1 \\
        0 & -2 & 0
      \end{bmatrix}
    \]
  }

  \sol{
    \begin{align*}
      \text{det } A & = \displaystyle\sum_{j=1}^{n} \left( -1 \right) ^{1+j} a_{1j} \det A_{1j} \\
                    & = a_{11} \det A_{11} - a_{12} \det A_{12} + a_{13} \det A_{13}            \\
                    & =  1 \begin{vmatrix} 4 & 1 \\ 2 & 0 \end{vmatrix} - 5 \begin{vmatrix} 2
                                                                               & -1 \\  0 &
                                                                              0\end{vmatrix} +
      0
      \begin{vmatrix}
        2 & 4 \\ 0 & -2\end{vmatrix}                                                              \\
                    & = 1 \left( 0 - 2 \right) - 5 \left( 0 \right) + 0 \left( -4 \right)       \\
                    & = -2                                                                      \\
    \end{align*}
  }
}

The definition of $\det A$ can also be written in the form of a \textit{cofactor expansion}, Given $A = \left[ a_{ij}
    \right] $, the $ \left( i, j \right)\text{-cofactor} $ of $A$ is the number $C_{ij}$ defined by:
\[
  C_{ij} = \left( -1 \right)^{i+j} \det A_{ij}
\]
Allowing us to express $\det A$ as:
\begin{align*}
  \det A & = \displaystyle\sum_{j=1}^{n} a_{1j} C_{1j} \\
         & = a_{11} C_{11} + \ldots + a_{1n} C_{1n}    \\
\end{align*}
This is termed as the \textit{cofactor expansion of the determinant along the first row} of $A$.

\thm{Cofactor Expansion}{
  The determinant of any $n\times n$ matrix $A$ can be computed by a cofactor expansion across any row or down any column.
  The expansion across the $i$th row is:
  \begin{align*}
    \det A & = \displaystyle\sum_{j=1}^{n} a_{ij} C_{ij}              \\
           & = a_{i1} C_{i1} + a_{i2} C_{i2} + \ldots + a_{in} C_{in} \\
  \end{align*}
  And the expansion down the $j$th column is:
  \begin{align*}
    \det A & = \displaystyle\sum_{i=1}^{n} a_{ij} C_{ij}              \\
           & = a_{1j} C_{1j} + a_{2j} C_{2j} + \ldots + a_{nj} C_{nj} \\
  \end{align*}
}

\ex{}{
  \qs{}{
    Use a cofactor expansion across the third row to compute the determinant of $A$, where
    \[
      A = \begin{bmatrix} 1 & 5 & 0 \\ 2 & 4 & -1 \\ 0 & -2 & 0 \end{bmatrix}
    \]
  }

  \sol{
    \begin{align*}
      \det A & = \displaystyle\sum_{j=1}^{n} a_{3j} C_{3j}                                                              \\
             & = 0  \begin{vmatrix} 5 & 0 \\ 4 & -1 \end{vmatrix} + 2 \begin{vmatrix} 1 & 0 \\ 2 & -1 \end{vmatrix} + 0
      \begin{vmatrix} 1 & 5 \\ 2 & 4 \end{vmatrix}                                                                      \\
             & = 0 + 2 \left( -1 \right) + 0                                                                            \\
             & = -2                                                                                                     \\
    \end{align*}
  }
}

In the case where we are computing the determinant of a matrix with great dimension, we take the cofactor across the row
or column with the most zeros.

\ex{}{
  \qs{}{
    Compute $\det A$, where
    \[
      A = \begin{bmatrix}
        3 & - 7 & 8  & 9  & -6 \\
        0 & 2   & -5 & 7  & 3  \\
        0 & 0   & 1  & 5  & 0  \\
        0 & 0   & 2  & 4  & -1 \\
        0 & 0   & 0  & -2 & 0
      \end{bmatrix}
    \]
  }

  \sol{
    We take the cofactor expansion down the first column of $A$.

    \begin{align*}
      \det A & = \displaystyle\sum_{i=1}^{n} a_{i1} C_{i3}                                                                \\
             & = a_{11} C_{11} + a_{21} C_{21} + a_{31} C_{31} + a_{41} C_{41} + a_{51} C_{51}                            \\
             & = 3 \begin{vmatrix} 2 & -5 & 7 & 3 \\ 0 & 1 & 5 & 0 \\ 0 & 2 & 4 & -1 \\ 0 & 0 & -2 & 0 \end{vmatrix} + 0
      C_{21} + 0 C_{31} + 0 C_{41} + 0 C_{51}                                                                             \\
             & \text{We disregard the zero terms}                                                                         \\
             & = 3 \begin{vmatrix} 2 & -5 & 7 & 3 \\ 0 & 1 & 5 & 0 \\ 0 & 2 & 4 & -1 \\ 0 & 0 & -2 & 0 \end{vmatrix}      \\
             & \text{Next we perform a cofactor expansion down the 1st column of our determinant}                         \\
             & = 3 \left(
      \displaystyle\sum_{i=1}^{n} a_{i1} C_{i1}
      \right)                                                                                                             \\
             & = 3 \left(
      a_{11} C_{11} + a_{21} C_{21} + a_{31} C_{31} + a_{41} C_{41}
      \right)                                                                                                             \\
             & = 3 \left( 2 \begin{vmatrix} 1 & 5 & 0 \\ 2 & 4 & -1 \\ 0 & -2 & 0 \end{vmatrix} - 0 C_{21} + 0 C_{31} - 0
      C_{41}\right)                                                                                                       \\
             & = 3 \times 2 \begin{vmatrix} 1 & 5 & 0 \\ 2 & 4 & -1 \\ 0 & -2 & 0 \end{vmatrix}                           \\
             & = 3 \times 2 \left(
      \displaystyle\sum_{j=1}^{n} a_{3j} C_{3j}
      \right)                                                                                                             \\
             & = 3 \times 2 \left( a_{31} C_{31} + a_{32} C_{32} + a_{33} C_{33} \right)                                  \\
             & = 3 \times 2 \left( 0 C_{31} + 2 \begin{vmatrix} 1 & 0 \\ 2 & -1 \end{vmatrix} + 0 C_{33}  \right)         \\
             & = 3 \times 2 \times 2 \left( -1 \right)                                                                    \\
             & = -12                                                                                                      \\
    \end{align*}
  }
}

\thm{}{
  If $A$ is a triangular matrix, then $\det A$ is the product of the entries on the main diagonal of $A$.
}

\subsection{Exercises}

\qs{}{
  Compute
  \[
    \begin{vmatrix}
      5  & -7 & 2 & 2  \\
      0  & 3  & 0 & -4 \\
      -5 & -8 & 0 & 3  \\
      0  & 5  & 0 & -6
    \end{vmatrix}
  \]
}

\sol{
  \begin{align*}
    \det A & = \displaystyle\sum_{j=1}^{n} a_{4j} C_{4j}                                                                                                           \\
           & = a_{41} C_{41} + a_{42} C_{42} + a_{43} C_{43} + a_{44} C_{44}                                                                                       \\
           & = 0 C_{41} - 5 \begin{vmatrix} 5 & 2 & 2\\ 0 & 0 & -4 \\ -5 & 0 & 3  \end{vmatrix} + 0 C_{43} + 6
    \begin{vmatrix} 5 & -7 & 2 \\ 0& 3 & 0 \\ -5 & -8 & 0 \end{vmatrix}                                                                                            \\
           & = 5 \left( \displaystyle\sum_{j=1}^{n} a_{2j} C_{2j} \right) + 6 \left( \displaystyle\sum_{j=1}^{n} a_{2j} C_{2j} \right)                             \\
           & = 5 \left( 0 C_{21} - 0 C_{22} -4 \begin{vmatrix} 5 & 2 \\ -5 & 0 \end{vmatrix}  \right) + 6 \left( \displaystyle\sum_{j=1}^{n} a_{2j} C_{2j} \right) \\
           & = 5 \left( 0 + 40 \right)  + 6 \left( 0 C_{21} - 3 \begin{vmatrix} 5 & 2 \\ -5 & 0 \end{vmatrix} + 0 C_{23}  \right)                                  \\
           & = 200 + 6 \left( -3 \times 10 \right)                                                                                                                 \\
           & = 200 - 180                                                                                                                                           \\
           & = 20                                                                                                                                                  \\
  \end{align*}
}

\section{Properties of Determinants}

\thm{Row Operations}{
  Let $A$ be a square matrix, Then:
  \begin{enumerate}
    \item If a multiple of one row $A$ is added to another row to produce a matrix $B$, then $\det B = \det A$
    \item If two rows of $A$ are interchanged to produce $B$, then $\det B = - \det A$
    \item If one row of $A$ is multiple by $k$ to produce $B$, then $\det B = k\cdot \det A$
  \end{enumerate}
}

\ex{}{
  \qs{}{
    Compute $\det A$, where $A = \begin{bmatrix}
        1  & - 4 & 2  \\
        -2 & 8   & -9 \\
        -1 & 7   & 0  \\
      \end{bmatrix} $
  }

  \sol{
    We can reduce the matrix $A$ to reduced row echelon form then use the fact that the determinant of a triangular matrix
    is the product of main diagonal entries.

    \begin{align*}
      \det A & = \begin{vmatrix}
                   1  & - 4 & 2  \\
                   -2 & 8   & -9 \\
                   -1 & 7   & 0  \\
                 \end{vmatrix}      \\
             & = \begin{vmatrix}
                   1  & -4 & 2  \\
                   0  & 0  & -5 \\
                   -1 & 7  & 0
                 \end{vmatrix}      \\
             & = \begin{vmatrix}
                   1 & -4 & 2  \\
                   0 & 0  & -5 \\
                   0 & 3  & 2
                 \end{vmatrix}      \\
             & = \begin{vmatrix}
                   1 & -4 & 2  \\
                   0 & 3  & 2  \\
                   0 & 0  & -5
                 \end{vmatrix}      \\
             & = 1 \times 3 \times -5 \\
             & = -15                  \\
    \end{align*}
  }
}




\chapter{Exercises}

\qs{}{
  Compute the product $A B$ using:
  \begin{itemize}
    \item The definition where $A b_1, A b_2$ are computed separately.
    \item The row-column rule.
  \end{itemize}
  \[
    A = \begin{bmatrix} -1 & 2 \\ 5 & 4 \\ 2 & -3 \end{bmatrix}, \, B \begin{bmatrix} 3 & -2 \\ -2 & 1 \end{bmatrix}
  \]
}

\sol{
  \begin{enumerate}
    \item
          \begin{align*}
            A b_1 & = \begin{bmatrix} -1 & 2 \\ 5 & 4 \\ 2 & -3 \end{bmatrix} \begin{bmatrix} 3 \\ -2 \end{bmatrix} \\
                  & = \begin{bmatrix} -3 - 4 \\ 15 - 8 \\ 6 + 6 \end{bmatrix}                                       \\
                  & = \begin{bmatrix} -7 \\ 7 \\ 12 \end{bmatrix}                                                   \\
            A b_2 & = \begin{bmatrix} -1 & 2 \\ 5 & 4 \\ 2 & -3 \end{bmatrix} \begin{bmatrix} -2 \\ 1 \end{bmatrix} \\
                  & = \begin{bmatrix} 4 \\ -6 \\ -7 \end{bmatrix}                                                   \\
            \\
            A B   & = \begin{bmatrix} -7 & 4 \\ 7 & -6 \\ 12 & -7 \end{bmatrix}
          \end{align*}
    \item
          \begin{align*}
            A B & = \begin{bmatrix}
                      -1 \times 3 + 2 \times -2 & -1 \times -2 + 2 \times 1 \\
                      5 \times 3 + 4 \times -2  & 5 \times -2 + 4 \times 1  \\
                      2 \times 3 + -3 \times -2 & -2 \times 2 + -3 \times 1
                    \end{bmatrix} \\
                & = \begin{bmatrix}
                      -7 & 4   \\
                      7  & - 6 \\
                      12 & -7
                    \end{bmatrix}                                        \\
          \end{align*}
  \end{enumerate}
}
\qs{}{
  Suppose the last column of $A B$ is entirely zero but $B$ itself has no column of zeros. What can you say
  about the columns of $A$?
}

\sol{
  If the last column of $A B$ is entirely zero, then the last column of $A$ must be a linear combination of the
  columns of $B$. Therefore the columns of $A$ are linearly dependent.
}

\qs{}{
  Find the inverses of the following matrices:
  \begin{enumerate}
    \item \label{itm:inv1}
          \[
            \begin{bmatrix} 8 & 6 \\ 5 & 4  \end{bmatrix}
          \]
    \item
          \[
            \begin{bmatrix} 3 & -4 \\ 7 & -8 \end{bmatrix}
          \]
  \end{enumerate}
}

\sol{
  \begin{enumerate}
    \item
          \begin{align*}
            \det \left( A \right) & = 32 - 30                                                    \\
                                  & = 2                                                          \\
            \\
            A^{-1}                & = \frac{1}{2} \begin{bmatrix} 4 & -6 \\ -5 & 8 \end{bmatrix} \\
                                  & = \begin{bmatrix}
                                        2    & -3 \\
                                        -5/2 & 4
                                      \end{bmatrix}                                             \\
          \end{align*}
    \item
          \begin{align*}
            \det \left( A \right) & = -24 + 28                                                   \\
                                  & = 4                                                          \\
            A^{-1}                & = \frac{1}{4} \begin{bmatrix} -8 & 4 \\ -7 & 3 \end{bmatrix} \\
                                  & = \begin{bmatrix}
                                        -4           & 1           \\
                                        -\frac{7}{4} & \frac{3}{4}
                                      \end{bmatrix}                                 \\
          \end{align*}
  \end{enumerate}
}

\qs{}{
  Use the inverse found in 6 \ref{itm:inv1} to solve the system:
  \begin{align*}
    8x_1 + 6x_2 & = 2  \\
    5x_1 + 4x_2 & = -1
  \end{align*}
}

\sol{
  \begin{align*}
    \mbold{b} = \begin{bmatrix} 2 \\  -1 \end{bmatrix}                \\
    \mbold{x} = A^{-1} \mbold{b}                                      \\
    \\
    \mbold{x} & = \begin{bmatrix}
                    2    & -3 \\
                    -5/2 & 4
                  \end{bmatrix} \begin{bmatrix} 2 \\ -1 \end{bmatrix} \\
              & = \begin{bmatrix}
                    7 \\
                    -9
                  \end{bmatrix}                                      \\
  \end{align*}
}

\qs{}{
  Find the inverse of the following matrix if it exists:
  \[
    \begin{bmatrix}
      1  & -2 & 1  \\
      4  & -7 & 3  \\
      -2 & 6  & -4
    \end{bmatrix}
  \]
}

\sol{
  \begin{align*}
    \begin{bmatrix}
      1  & -2 & 1  \\
      4  & -7 & 3  \\
      -2 & 6  & -4
    \end{bmatrix}
    \\
    4R_1 - R_2  \rightarrow R_2                    \\
    \begin{bmatrix}
      1  & -2 & 1  \\
      0  & -1 & 1  \\
      -2 & 6  & -4 \\
    \end{bmatrix}
    \\
    -2R_1 - R_3  \rightarrow R_3                   \\
    \begin{bmatrix}
      1 & -2 & 1 \\
      0 & -1 & 1 \\
      0 & -2 & 2 \\
    \end{bmatrix}
    \\
    2R_2 - R_3  \rightarrow R_3                    \\
    \begin{bmatrix}
      1 & -2 & 1 \\
      0 & -1 & 1 \\
      0 & 0  & 0 \\
    \end{bmatrix}
    \\
    \det \left( A \right) & = 1 \times -1 \times 0 \\
                          & = 0                    \\
    \therefore \text{ the matrix does not have an inverse}
  \end{align*}
}

\qs{}{
  Suppose the system below is consistent for all possible values of $f$ and $g$. What can you say about the coefficients
  $c$ and $d$? Justify your answer.
  \begin{align*}
    x_1 + 3x_2  & = f \\
    cx_1 + dx_2 & = g \\
  \end{align*}
}

\sol{
  \begin{align*}
    \begin{bmatrix}
      1 & 3 & f \\
      c & d & g
    \end{bmatrix}     \\
    cR_1 - R_2 \to R_2 \\
    \begin{bmatrix}
      1 & 3      & f      \\
      0 & 3c - d & cf - g
    \end{bmatrix}
  \end{align*}
}

\qs{}{
  Let $\mbold{u} = \begin{bmatrix} 2 \\ -1 \end{bmatrix} $ and $\mbold{v} = \begin{bmatrix} 2 \\ 1 \end{bmatrix} $. Show
  that $\begin{bmatrix} h \\ k \end{bmatrix} $ is in $\text{Span}  \{u, v\} $ for all $h$ and $k$.
}

\sol{
  \begin{align*}
    x_1\mbold{u} + x_2 \mbold{v}                 & = \begin{bmatrix} h \\ k \end{bmatrix} \\
    \begin{bmatrix} 2 & 2 \\ -1 & 1\end{bmatrix} & = \begin{bmatrix} h \\ k \end{bmatrix}
  \end{align*}
  Therefore
  \begin{align*}
    \begin{bmatrix}
      2  & 2 & h \\
      -1 & 1 & k
    \end{bmatrix}                \\
    -\frac{1}{2}R_1 - R_2 \to R_2 \\
  \end{align*}
}

\qs{}{
  A steam plant burns two types of coal: anthracite (A) and bituminous (B). For each ton of A burned,
  the plant produces 27.6 million Btu of heat, 3100 grams (g) of sulfur dioxide, and 250 g of
  particulate matter (solid-particle pollutants). For each ton of B burned, the plant produces 30.2
  million Btu, 6400 g of sulfur dioxide, and 360 g of particulate matter.
  \begin{enumerate}
    \item How much heat does the steam plant produce when it burns $x_1$ tons of A and $x_2$ tons of B?
    \item Suppose the output of the steam plant is described by a vector that lists the amounts of
          heat, sulfur dioxide, and particulate matter. Express this output as a linear combination of
          two vectors, assuming that the plant burns $x_1$ tons of A and $x_2$ tons of B.
    \item Over a certain time period, the steam plant produced 162 million Btu of heat, 23,610 g
          of sulfur dioxide, and 1623 g of particulate matter. Determine how many tons of each type
          of coal the steam plant must have burned. Include a vector equation as part of your solution.
  \end{enumerate}
}

\sol{
  \begin{align*}
    27.6x_1 + 30.2x_2 & = \text{Heat}               \\
    3100x_1 + 6400x_2 & = \text{Sulfur Dioxide}     \\
    250x_1 + 360x_2   & = \text{Particulate Matter}
  \end{align*}
  \begin{enumerate}
    \item
          \[
            27.6x_1 + 30.2x_2
          \]
    \item
          \begin{align*}
            27.6x_1 + 30.2x_2             & = H                                            \\
            3100x_1 + 6400x_2             & = SO_2                                         \\
            250x_1 + 360x_2               & = P                                            \\
            \\
            \mbold{u} x_1 + \mbold{v} x_2 & = \begin{bmatrix} H \\ SO_2 \\ P \end{bmatrix} \\
            \text{Where } \mbold{u} = \begin{bmatrix} 27.6 \\ 3100 \\ 250 \end{bmatrix} \text{ and } \mbold{v} = \begin{bmatrix} 30.2 \\ 6400 \\ 360 \end{bmatrix}
          \end{align*}
    \item
          \begin{align*}
            27.6x_1 + 30.2x_2 & = 162                  \\
            3100x_1 + 6400x_2 & = 23610                \\
            250x_1 + 360x_2   & = 1623                 \\
            \\
            \begin{bmatrix}
              27.6 & 30.2 & 162   \\
              3100 & 6400 & 23610 \\
              250  & 360  & 1623  \\
            \end{bmatrix}                        \\
            \frac{7750}{69}R_1 - R_2  \rightarrow R_2  \\
            \begin{bmatrix}
              \frac{138}{5} & \frac{151}{5}      & 162                \\
              0             & \frac{-207550}{69} & \frac{-124530}{23} \\
              250           & 360                & 1623               \\
            \end{bmatrix}
            \\
            \frac{625}{69}R_1 - R_3  \rightarrow R_3   \\
            \begin{bmatrix}
              \frac{138}{5} & \frac{151}{5}      & 162                \\
              0             & \frac{-207550}{69} & \frac{-124530}{23} \\
              0             & \frac{-5965}{69}   & \frac{-3579}{23}   \\
            \end{bmatrix}
            \\
            \frac{112}{3897}R_2 - R_3  \rightarrow R_3 \\
            \begin{bmatrix}
              \frac{138}{5} & \frac{151}{5}      & 162                \\
              0             & \frac{-207550}{69} & \frac{-124530}{23} \\
              0             & 0                  & \frac{0}{1}        \\
            \end{bmatrix}
            \\
            0R_1 - R_3  \rightarrow R_3                \\
            \begin{bmatrix}
              \frac{138}{5} & \frac{151}{5}      & 162                \\
              0             & \frac{-207550}{69} & \frac{-124530}{23} \\
              0             & 0                  & \frac{0}{1}        \\
            \end{bmatrix}
            \\
            0R_2 - R_3  \rightarrow R_3                \\
            \begin{bmatrix}
              \frac{138}{5} & \frac{151}{5}      & 162                \\
              0             & \frac{-207550}{69} & \frac{-124530}{23} \\
              0             & 0                  & \frac{0}{1}        \\
            \end{bmatrix}
            \\
            \frac{-98}{9761}R_2 - R_1  \rightarrow R_1 \\
            \begin{bmatrix}
              \frac{-138}{5} & 0                  & \frac{-2691}{25}   \\
              0              & \frac{-207550}{69} & \frac{-124530}{23} \\
              0              & 0                  & \frac{0}{1}        \\
            \end{bmatrix}
            \\
            0R_2 - R_1  \rightarrow R_1                \\
            \begin{bmatrix}
              \frac{138}{5} & 0                  & \frac{2691}{25}    \\
              0             & \frac{-207550}{69} & \frac{-124530}{23} \\
              0             & 0                  & \frac{0}{1}        \\
            \end{bmatrix}
            \\
            0R_2 - R_1  \rightarrow R_1                \\
            \frac{5}{138}R_1 \to R_1                   \\
            \frac{-1}{3008}R_2 \to R_2                 \\
            \begin{bmatrix}
              1 & 0 & \frac{39}{10} \\
              0 & 1 & \frac{9}{5}   \\
              0 & 0 & \frac{0}{1}   \\
            \end{bmatrix}
            \\
            \mbold{x} = \begin{bmatrix} \frac{39}{10} \\ \frac{9}{5} \end{bmatrix}
          \end{align*}

  \end{enumerate}
}

\qs{}{
  Describe and compare the solution sets of $x_1-3x_2+5x_3 = 0$ and $x_1-3x_2+5x_3 = 4$.
}

\sol{
  \begin{align*}
    \begin{bmatrix}
      1 & -3 & 5 & 0 \\
      0 & 0  & 0 & 0 \\
      0 & 0  & 0 & 0 \\
    \end{bmatrix}       \\
    x_1 -3x_2 + 5x_3 = 0 \\
    x_2 = x_2            \\
    x_3 = x_3            \\
    x_1 = 3x_2 - 5x_3    \\
    \\
    \mbold{x} = \begin{bmatrix} 3 \\ 1 \\ 0 \end{bmatrix} x_2 + \begin{bmatrix} -5 \\ 0 \\ 1 \end{bmatrix} x_3
  \end{align*}
}


\end{document}
