\documentclass[12pt letter]{report}
\input{./template/preamble}
\input{./template/macros}
\input{./template/letterfonts}
\setcounter{tocdepth}{4}
\setcounter{secnumdepth}{4}


\title{\Huge{Exploratory Data Analysis}}
\author{\huge{Madiba Hudson-Quansah}}
\date{}
\usepackage{parskip}

\begin{document}
\maketitle
\newpage
\pdfbookmark[section]{\contentsname}{too}
\tableofcontents
\pagebreak

\chapter{Introduction}

\dfn{Data}{
	Pieces of information about individuals (specific person / object) organized into variables (specific characteristic
	of the individual).
}

\dfn{Dataset} {
	A set of data identified with particular circumstances.
}

Variables can be classified into one of two types:

\begin{itemize}
	\item Categorical / Qualitative
	\item Quantitative
\end{itemize}

\dfn{Categorical / Quantitative Variable} {
	Take category or label values and places an individual into one of several mutually exclusive groups.
}

\dfn{Quantitative Variable} {
	Take numerical values and represents a measurement.
}

\chapter{Module 4: Examining Distributions}

\dfn{Distribution} {
	The distribution of a variable tells us what values the variable takes and how often it takes these values.
}

\section{Distribution of One Categorical Variable}

In order to summarize the distribution of a categorical variable we must first create a table of all the values the
category can take (categories) and how often it takes these values (counts). This is called a \textbf{frequency table}.


In order to visualize the summaries we've made we can use one of two graphs:

\begin{itemize}
	\item Pie Chart
	\item Bar Graph
\end{itemize}

\section{Distribution of One Quantitative Variable}

To visualize a summary of a quantitative variable we can use one of three graphs:
\begin{itemize}
	\item Histogram
	\item Stem plot
	\item Box plot
\end{itemize}

\subsection{Histogram}

To construct a histogram we must first divide the range of our data into equal sized intervals called \textbf{bins} /
\textbf{classes}. We then count how many observation fall into each bin and construct a frequency table with the bins
and counts. We then plot the bin on the $x$-axis and the count on the $y$-axis.

\dfn{Relative Frequency} {
	The proportion of observations in a category. Calculated by dividing the count of observations in a category by the
	total number of observations.
}

In interpreting the histogram we must consider the following features of the distribution:
\begin{itemize}
	\item Shape
	\item Centre
	\item Spread
	\item Outliers
\end{itemize}

Where the first three describe the distribution as a whole and the patterns found within and the last highlights deviations from that pattern.

\subsubsection{Shape}

The shape of a distribution is described by its \textbf{modality / peakedness} and \textbf{symmetry/skewness}.

\dfn{Modality}{
	Number of peaks (modes) in a distribution.
}

\dfn{Symmetric Distribution}{
	A distribution is symmetric if the right and left sides of the histogram are approximately mirror images of each
	other.
}

\dfn{Skewed Distribution}{
	A distribution is skewed if one of its tails is longer than the other.

	\begin{itemize}
		\item \textbf{Right Skewed} if the right tail is longer.
		\item \textbf{Left Skewed} if the left tail is longer.
	\end{itemize}
}

\dfn{Unimodal}{
	A distribution with one mode, i.e. a central peak the observations are concentrated around.
}

\dfn{Bimodal}{
	A distribution with two modes, i.e. two central peaks the observations are concentrated around.
}

\dfn{Multimodal}{
	A distribution with more than two modes, i.e. more than two central peaks the observations are concentrated around.
}

\dfn{Uniform}{
	A distribution with no mode, i.e. the observations are evenly distributed across the range of the data.
}

\subsubsection{Centre}

The value that divides the distribution so that approximately half the observations take smaller values, and half
take larger values.

\subsubsection{Spread}

The approximate range covered by the data. i.e. The smallest value to largest value.

\subsubsection{Outliers}

Observations that fall outside the overall pattern of the distribution.

\subsection{Stem Plot / Stem and Leaf Plot}

To construct a stem plot we must
\begin{itemize}
	\item Separate each observation into a \textbf{stem} and a \textbf{leaf}. The stem being everything except the right post digit and the leaf being the right most digit, e.g. 123 would have a stem of 12 and a leaf of 3.
	\item Write the stems in a vertical column in ascending order.
	\item Go through the data points and match each leaf to its stem in ascending order.
\end{itemize}

\qs{}{
	Construct a stem plot for the following Dataset

	34 34 27 37 42 41 36 32 41 33 31 74 33 49 38 61 21 41 26 80 42 29 33 36 45 49 39 34 26 25 33 35 35 28 30 29 61 32 33 45 29 62 22 44
}

\sol{
	\begin{table}[h!]
		% \begin{center}
		\begin{tabular}{c|c}
			2 & 1 2 5 6 6 7 8 9 9                   \\
			3 & 0 1 2 2 3 3 3 3 3 4 4 5 5 6 6 7 8 9 \\
			4 & 1 1 1 2 2 4 4 5 9 9                 \\
			5 &                                     \\
			6 & 1 1 2                               \\
			7 & 4                                   \\
			8 & 0                                   \\
		\end{tabular}
		% \end{center}
	\end{table}

	\begin{table}[h!]
		% \begin{center}
		\begin{tabular}{c|c}
			2 & 1 2                   \\
			2 & 5 6 6 7 8 9 9         \\
			3 & 0 1 2 2 3 3 3 3 3 4 4 \\
			3 & 5 5 6 6 7 8 9         \\
			4 & 1 1 1 2 2 4 4         \\
			4 & 5  9 9                \\
			5 &                       \\
			5 &                       \\
			6 & 1 1 2                 \\
			6 &                       \\
			7 & 4                     \\
			7 &                       \\
			8 & 0                     \\
		\end{tabular}
		% \end{center}
	\end{table}
}


When skewed right the stem plot can be visualize to identify the skewness.

\nt{
	The advantages of the stem plot are:
	\begin{itemize}
		\item It preserves the original data.
		\item It sorts the data.
		\item It is easy to construct for small datasets.
	\end{itemize}
}

\section{Numerical Measures}

To get a more accurate description of a discrete quantitative variable we can use numerical measures.

\subsection{Measures of Centre}

The numerical measure of a distribution's centre is basically telling us what is a typical value for the variable. The three main numerical measures of centre are the \textbf{mean}, \textbf{median} and \textbf{mode}.

\subsubsection{Mode}

The most common occurring value in a distribution.

\subsubsection{Mean}

The average of a set of observations i.e. the sum of all the observations divided by the number of observations. i.e.
\[
	\bar{x} = \frac{\sum{f}}{n}
\]

Where $f$ is the frequency of the observation and $n$ is the number of observations.

\subsubsection{Median}

The midpoint of a distribution. i.e. the value that divides the distribution so that approximately half the observations take smaller values, and half take larger values.

It can be found by:
\begin{itemize}
	\item First Ordering the observations in ascending order.
	\item Consider whether $n$ (number of observations) is even or odd. If:
	      \begin{itemize}
		      \item Even - The median is the average of the two middle observations, i.e.
		            \[
			            \frac{M_1 + M_2}{2}
		            \]  where
		            $M_1$ and $M_2$ are the two middle observations, and where $M_1$ and $M_2$ found at the $\frac{n}{2}$ and $\frac{n}{2} + 1$ positions respectively.
		      \item Odd - The median is the middle observation. i.e. $M$ where $M$ is found at the $\frac{n + 1}{2}$
		            position
	      \end{itemize}
\end{itemize}

\subsubsection{Comparing the Mean and Median}

The mean is being the average of all the observations is more sensitive to outliers than the median. This means that for:
\begin{itemize}
	\item A symmetric distribution with no outliers - $\overline{x}$ (the mean) $\approx$ (approximately equal) $M$ (the median).
	\item A right skewed distribution and/or datasets with high outliers - $\overline{x} > M$
	\item A left skewed distribution and/or datasets with low outliers - $\overline{x} < M$
\end{itemize}

Therefore it is best to use mean ($\overline{x}$) as a measure of centre for symmetrical distributions with no outliers, Otherwise
median ($M$) is a better measure of centre.

\subsection{Measures of Spread}

The numerical measure of a distribution's spread is basically telling us how spread out the data is. The three main numerical measures of spread are the \textbf{range}, \textbf{interquartile range} and \textbf{standard deviation}.

\subsubsection{Range}

The range of a distribution is the difference between the largest and smallest observations. i.e. $R = x_{max} - x_{min}$

\subsubsection{Interquartile Range}

The interquartile range (IQR) measures variability of a distribution by giving us the range covered by the middle 50\% of the
observations instead of the whole range covered by all the observations. To calculate the IQR we must

\begin{itemize}
	\item Arrange the observations in ascending order.
	\item Find the median.
	\item Find the median of the lower 50\% of observations also called the first/lower quartile (Q1). We can find the
	      position of the lower quartile using the formula:
	      \[
		      Q1_{th} = \frac{1}{4}(n + 1)
	      \]
	\item Find the median of the upper 50\% of observations also called the third/upper quartile (Q3). We can find the
	      position of the upper quartile using the formula:
	      \[
		      Q3_{th} = \frac{3}{4}(n + 1)
	      \]
	\item Calculate the IQR by subtracting Q1 from Q3. i.e. $IQR = Q3 - Q1$
\end{itemize}


\paragraph{Using the IQR to Identify Outliers.}

The IQR is used as the basis for a rule of thumb for identifying outliers called the \textbf{1.5 $\times$ IQR Rule}. According to this rule, observations that fall more than 1.5 $\times$ IQR above the third quartile or below the first quartile are considered outliers.


\paragraph{Box Plot.}


A box plot is a graphical display of the five number summary. It is constructed by drawing a box with the first side at
the lower quartile and the second side at the upper quartile, then drawing a line through the box at the median. Finally
we draw vertical lines at the maximum and minimum values and connect them to the box with horizontal lines.

\dfn{Five Number Summary}{
	The combination of the minimum, lower quartile, median, upper quartile and maximum of a distribution.
}



Another way of constructing box plots is the \textbf{Tukey method}. This consists of:
\begin{itemize}
	\item Calculating the IQR
	\item Finding 1.5 $\times$ IQR and adding it to the upper quartile. If the value is greater than or equal to the
	      maximum value of the distribution then the maximum value is used as the upper whisker. Otherwise the largest
	      value in the distribution that
	      is less than the upper quartile $+$ 1.5 $\times$ IQR is used as the upper whisker.
	\item Subtracting 1.5 $\times$ IQR from the lower quartile. If the value is less than or equal to the minimum value of
	      the distribution then the minimum value is used as the lower whisker. Otherwise the smallest value in the
	      distribution that is greater than
	      the lower quartile $-$ 1.5 $\times$ IQR is used as the lower whisker.
	\item If the maximum or minimum values are not used as the whiskers of the boxplot they are instead denoted by
	      asterisks (*).
\end{itemize}

\subsubsection{Standard Deviation}

The standard deviation measures how far the observations are from their mean $\overline{x}$ .i.e. the average distance
between a data point and the mean. The standard deviation of a sample is denoted by $s$ and is calculated by:
\[
	s = \sqrt{\frac{\sum{(X - \overline{x}})^2}{n - 1}}
\]

Where $X$ is the observation, $\overline{x}$ is the mean of the observations and $n$ is the number of observations.

\nt {
	The standard deviation is always positive.
}

\nt{
	Since the SD is dependent on the mean, it should be used as a measure of spread only when the mean is used as a measure of centre.
	Also due to the SD's dependence on the mean it is also sensitive to outliers.
}

\thm{Standard Deviation Rule / Empirical Rule}{
	For a symmetric bell shaped distribution, i.e. normal distribution, the following rule applies:
	\begin{itemize}
		\item Approximately 68\% of the observations fall within 1 SD of the mean .i.e $68\%  \times n$ lies between $\overline{x} \pm s$
		\item Approximately 95\% of the observations fall within 2 SD of the mean .i.e. $95\%  \times n$ lies between $\overline{x} \pm 2s$
		\item Approximately 99.7\% of the observations fall within 3 SD of the mean. .i.e. $99.7\%  \times n$ lies between $\overline{x} \pm 3s$
	\end{itemize}
}

\subsection{Choosing Numerical Measures}

Use $\overline{x}$ and $s$ as measures of centre and spread only for reasonably symmetric distributions with no
outliers. Otherwise use $M$ and IQR.


\chapter{Module 5: Examining Relationships}





\end{document}
