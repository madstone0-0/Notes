\documentclass[12pt letter]{report}
\input{./template/preamble}
\input{./template/macros}
\input{./template/letterfonts}

\title{\Huge{Design Patterns}}
\author{\huge{Madiba Hudson-Quansah}}
\date{}
\usepackage{parskip}

\setcounter{tocdepth}{4}
\setcounter{secnumdepth}{4}

\begin{document}
\maketitle
\newpage
\pdfbookmark[section]{\contentsname}{too}
\tableofcontents
\pagebreak

\chapter{Introduction}

\dfn{Design Pattern}{
  A design pattern describes a recurring problem in the construction phase and, then describes
  the core of the solution to the problem in a way that promotes re-usability and modularity in
  any context.
}
Design Patterns capture the best practices of experienced object oriented software developers, and are
general solutions to software development problems. Each pattern is comprised of four parts:

\begin{description}
  \item [ The pattern name ] - A handle to describe a design problem, its solution and consequences succinctly.
  \item [ The problem ] - Describes when to apply the pattern.
        \begin{itemize}
          \item Explains the problem and its context.
          \item It might describe specific design problems such as how to represent algorithms as objects.
          \item It might describe class or object structures that are symptomatic of an inflexible design.
          \item May include a list of conditions that must be met before it makes sense to apply the pattern.
        \end{itemize}
  \item [ The solution ] - Describes the elements that make up the design, their relationships, responsibilities and collaborations.
        \begin{itemize}
          \item Does not describe a particular concrete design or implementation.
          \item Provides an abstract description of a design problem and how a general arrangement of elements solves it.
        \end{itemize}
  \item [ The consequences ] - Describes the results and trade-offs of applying the pattern.
        \begin{itemize}
          \item Space and time trade-offs.
          \item Language and implementation issues.
          \item Impact on system flexibility, extensibility or portability.
        \end{itemize}
\end{description}

\chapter{Types of Design Patterns}
Erich Gamma, Richard Helm, Ralph Johnson and John Vlisides in their Design Patterns book define 23 design patterns  divided into three types:

\begin{description}
  \item[Creational Patterns]  - Patterns that create objects for you, rather than having you instantiate objects directly.
  \item[Structural Patterns] - Patterns that compose groups of objects into larger structures.
  \item[Behavioral Patterns] - Patterns that define how objects interact.
\end{description}

\section{Creational Patterns}
Here are the creational patterns:
\begin{description}
  \item[Factory Pattern]  - A decision making class that returns one of several possible subclasses of an abstract base class depending on the data provided to it.
  \item[Abstract Factory Pattern] - Provides and interface to create and return one of several families of related objects.
  \item[Builder Pattern] - A class that constructs a complex object step by step.
  \item[Prototype Pattern] - A class that creates a new object by copying an existing object rather than creating new instances.
  \item[Singleton Pattern] - A class of which only a single instance can exist.
\end{description}

\subsection{Factory Pattern}
\lstinputlisting[language=C++, caption=Factory Pattern]{/home/mads/projects/C++/Learning/src/Patterns/_patterns/Factory.hpp}

This \lstinline{ShapeFactory} class is a factory class that returns a subclass of the \lstinline{Shape} class depending on the data provided to it. The \lstinline{Shape} class is an abstract class that has a pure virtual function \lstinline{getArea()}.

The Factory Pattern should be used when:
\begin{itemize}
  \item A class can't anticipate the class of objects it must create.
  \item A class uses it's subclasses to specify the objects it creates.
  \item You want to abstract away the knowledge of the concrete classes from the client.
\end{itemize}

\subsection{Abstract Factory Pattern}
The Abstract Factory Pattern sits one level of abstraction higher than the Factory Pattern. The Abstract Factory Pattern is a factory object that returns
one of several factories, i.e. A factory of factories. An example of this could be for a GUI library that has a factory for elements of a
certain themes, e.g. a Windows theme or a Mac theme.
\lstinputlisting[language=C++, caption=Abstract Factory Pattern]{/home/mads/projects/C++/Learning/src/Patterns/_patterns/AbstractFactory.hpp}

This \lstinline{ArmyMaker} class is an abstract factory class that returns a subclass of the \lstinline{Army} class depending on the data provided to it. The \lstinline{Army} class is an abstract class that has the pure virtual functions
\lstinline{createInfantry()}, \lstinline{createRanged()} and \lstinline{createMounted()} that return instances of the \lstinline{Unit} class.

The Abstract Factory Pattern isolates the creation of objects from the client and because of this it is easier to change or interchange
the product class families freely. The Abstract Factory Pattern should be used when:
\begin{itemize}
  \item A system should be independent of how its products are created, composed and represented.
  \item A system should be configured with one of multiple families of products.
  \item A family of related product objects is designed to be used together and you need to enforce this constraint.
\end{itemize}

\subsection{Builder Pattern}
The Builder pattern separates the construction of a complex object from its representation so that the same construction process can create different representation.

\lstinputlisting[language=C++, caption=Builder Pattern]{/home/mads/projects/C++/Learning/src/Patterns/_patterns/Builder.hpp}

The builder pattern should be used when:
\begin{itemize}
  \item The algorithm for creating a complex object should be independent of the parts that make up the object and how they are assembled.
  \item The construction process must allow different representations for the object that is constructed.
\end{itemize}

\subsubsection{Consequences of the Builder Pattern}
\begin{itemize}
  \item The Builder Pattern lets you vary a product's internal representation.
  \item Each specific builder is independent of the others and the rest of the program.
  \item You have more control over the each final product that is constructed as it is constructed step by step.
  \item The Builder pattern is somewhat like an Abstract Factory in that they both return classes made up of a number of methods and objects. The
        difference is that the Abstract Factory returns a family of related classes, the Builder constructs a complex object step by step depending
        on the data provided to it.
\end{itemize}

\subsection{Prototype Pattern}
Specifies types of objects using a prototypical instance, and create new objects by copying this prototype. The Prototype Pattern should be used when:

\lstinputlisting[language=C++, caption=Prototype Pattern]{/home/mads/projects/C++/Learning/src/Patterns/_patterns/Prototype.hpp}

\begin{itemize}
  \item Creating a new instance is more expensive than copying an existing instance.
  \item You need classes that differ only in the type of processing they offer, for example parsing of strings representing different radixes.
\end{itemize}

\subsubsection{Consequences of the Prototype Pattern}
\begin{itemize}
  \item You can add and remove classes at runtime as needed.
  \item You can revise the internal data representation of classes at runtime based on program conditions.
  \item You can specify new objects at runtime without creating a lot of classes and subclasses.
  \item Classes that have circular references cannot be cloned easily.
  \item Language implementation issues may arise for example in Java, if the classes already exist, you may not be able to change them to implement
        the Cloneable interface. The deep clone method can be particularly difficult to implement if the all the class objects contained in a class
        cannot be declared to implement the Serializable interface.
\end{itemize}

\subsection{Singleton Pattern}

Sometimes it is appropriate to only have one instance of a class for example:
\begin{itemize}
  \item Window Managers
  \item Texture Managers for games
  \item Print Spoolers
  \item File Systems
\end{itemize}
Typically singleton objects are accessed by disparate objects throughout a system, and therefore require a global point of access. With a singleton pattern
you can:
\begin{itemize}
  \item Ensure that a class has only one instance.
  \item Provide a global point of access to the instance.
  \item Allow multiple instances in the future without affecting a singleton class's clients.
\end{itemize}
\lstinputlisting[language=C++, caption=Singleton Pattern]{/home/mads/projects/C++/Learning/src/Patterns/_patterns/Singleton.hpp}

\subsubsection{Consequences of the Singleton Pattern}
\begin{itemize}
  \item It can be difficult to subclass a singleton class, since it can only work if the base singleton class has not yet been initialized
  \item It is possible to have multiple instances of a singleton class if the singleton class is not implemented correctly.
  \item The number of instances of a singleton class can be controlled.
\end{itemize}


\end{document}
