\documentclass[12pt letter]{report}
\input{./template/preamble}
\input{./template/macros}
\input{./template/letterfonts}

\title{\Huge{Authentication}}
\author{\huge{Madiba Hudson-Quansah}}
\date{}
\usepackage{parskip}

\setcounter{tocdepth}{4}
\setcounter{secnumdepth}{4}

\begin{document}
\maketitle
\newpage
\pdfbookmark[section]{\contentsname}{too}
\tableofcontents
\pagebreak

\chapter{Introduction}

Controlling threats and vulnerabilities to a computer system involves
a policy that specifies, who can access what and how, i.e. which
subjects, can access which objects, and by which means. There are
three key tools to achieve this:
\begin{itemize}
  \item Authentication
  \item Access Control
  \item Cryptography
\end{itemize}

\dfn{Authentication}{
  The property of accurate identification.
}

\dfn{Access Control}{
  The selective restriction of access to a place or other resource.
}

\dfn{Cryptography}{
  Converting data into a form unreadable to unauthorized users.
}

Determining an entity's identity consists of two phases:
\begin{description}
  \item[Identification] - The act of asserting who an entity is,
    involving presenting a credential
  \item[Authentication] - The act of proving that the entity is who
    it claims to be, involving exchanging authentication information.
\end{description}

\section{Means of Authentication}
There are three primary means of authentication:
\begin{description}
  \item[Something the users knows / Credentials]  - A piece of
    private information
  \item[Something the user is / Biometrics] - A physical
    characteristic of the user.
  \item[Something the user has / Tokens] - A physical item possessed
    by the user.
\end{description}

\subsection{Credentials}

\dfn{Credentials}{
  Pieces of information presented by a user to prove their identity
  during initial authentication.
}

Credentials are data structures that binds an identity and additional
attributes to a token.

\subsection{Token}

\dfn{Token}{
  A digital artefact issued after successful authentication to
  maintain as session or delegate access without resubmitting credentials.
}

Tokens can either be hardware or software possessed by a subscriber.

\section{Multi-Factor Authentication (MFA)}

\dfn{Multi-Factor Authentication (MFA)}{
  An authentication process where a user presents two or more
  independent pieces of evidence (factors) to verify their identity.
}

For example, using a password (something you know) and a fingerprint
(something you are) to access a secure system.

\section{NIST (National Institute of Standards and Technology)
Assurance Levels for User Authentication}

These levels indicate the degree of confidence in the identity
proofing and authentication processes. It also provides options for
organization to choose between depending on the security risk assessment.

It defines three separate levels for Identity Assurance Level (IAL)
and Authenticator Assurance Level (AAL).

\subsection{Identity Assurance Level (IAL)}
Focuses on verifying who the user is.
\begin{description}
  \item[IAL1]  - No need to link the applicant to a specific
    real-life identity. Example anonymous discussion forums.
  \item[IAL2] - Provides evidence for the claimed identity using
    either remote of physically present identity proofing. Example
    requiring a National ID card or passport.
  \item[IAL3] - Requires physical presence (in-person) for identity
    proofing. Example applying for a security clearance.
\end{description}

\subsection{Authenticator Assurance Level (AAL)}
Focuses on verifying how securely the user authenticates.
\begin{description}
  \item[AAL1]  - Provides some assurance (low confidence) / Single
    Factor Authentication.
    Example password-based authentication.
  \item[AAL2] - Provides high confidence / Multi-Factor
    Authentication with proof of possession and control of two factors.
  \item[AAL3] - Provides very high confidence / MFA using
    hardware-based cryptographic tokens resistant to phishing and
    replay attacks. Example using a smart card with a PIN.
\end{description}

\section{Password Based Authentication}

\dfn{Discretionary Access Control (DAC)}{
  An access control model where the owner of the resource determines
  who can access it and what privileges they have, i.e. a user may
  grant permission to other users to access their files.
}

Uses a user ID and password pair to authenticate users, where the
user ID identifies the account and links the account to the user's
privileges and access permissions under Discretionary Access Control (DAC).

A system using this method maintains a password file indexed by user
ID, with a one-way has of the password stored in it rather than the
password itself. This ensures that even if the password file is
compromised the actual passwords are not immediately exposed.

\subsection{Attacks and Countermeasures}

\subsubsection{Offline Dictionary Attack}

An attack who gains access to the password file and systematically
compares password hashes against hashes of commonly used passwords.
If a match is found the attacker can gain access with the corresponding user ID.

\nparagraph{Countermeasures}
\begin{itemize}
  \item Prevent unauthorized access to the password file
  \item Intrusion detection systems to monitor for suspicious activity
  \item Re-issuance of passwords if compromise is suspected
\end{itemize}

\subsubsection{Specific Account Attack}

The attacker targets a specific account and submits password guesses
until the correct password is found.

\nparagraph{Countermeasures}
\begin{itemize}
  \item Account lockout after a certain number of failed attempts.
\end{itemize}

\section{UNIX Password Scheme}

\dfn{Salt}{
  A random value added to the password before applying the one-way
  hash function. The salt:
  \begin{itemize}
    \item Prevents duplicate passwords from having the same hash value.
    \item Increases the complexity of dictionary attacks by requiring
      attackers to compute hashes for each possible salt value.
    \item Almost impossible to find whether a person with passwords
      on two or more systems has used the same password on those systems.
  \end{itemize}
}

\subsection{Attacks}

\subsubsection{Dictionary Attacks}
An attacker pre-computes hash values for a large number of
commonly used passwords and their variations. When the password file
is compromised, the attacker compares the stored hash values against
the pre-computed hashes to find matches.

\subsubsection{Rainbow Table Attacks}
\dfn{Rainbow Table}{
  A pre-computed hash table used to reverse weakly hashed passwords.
}
An attacker uses a rainbow table to look up the hash values from
the compromised password file to find the corresponding plaintext passwords.

\subsubsection{Brute-Force Attacks}
An attack uses a tool like John the Ripper to systematically generate
and hash all possible password combinations until a match is found.

\section{Biometric Authentication}

\dfn{Biometric}{
  A measurable physical or behavioural characteristic used to
  uniquely identify individuals.
}

Authenticates an individual based on their unique physical or
behavioural characteristics. Harder to implement and more expense
than password-based or token-based identification. Once compromised,
biometric data cannot be changed like passwords or tokens.
Some common biometric modalities include:
\begin{itemize}
  \item Facial characteristics
  \item Fingerprints
  \item Hand geometry
  \item Retinal patterns
  \item Iris patterns
  \item Signature dynamics
  \item Voice patterns
\end{itemize}

\end{document}
