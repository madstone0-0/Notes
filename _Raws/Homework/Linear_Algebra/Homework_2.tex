\documentclass[12pt letter openany]{report}
\input{../template/preamble}
\input{../template/macros}
\input{../template/letterfonts}
\pagenumbering{gobble}

\usepackage{parskip}
\let\cleardoublepage=\clearpage
\title{\Huge{Homework 2}}
\author{\huge{Madiba Hudson-Quansah}}
\date{}


\begin{document}
\maketitle
\newpage

\qs{}{
  Suppose an economy has four sectors, Agriculture (A), Energy (E), Manufacturing (M), and Transportation (T). Sector
  A sells 10 \% of it's output to E and 25 \% to M and retains the rest. Sector E sells 30 \% of its output to A, 35
  \% to M and 25 \% to T and retains the rest. Sector M sells 30 \% of its output to A, 15 \% to E, and 40 \% to T and
  retains the rest. Sector T sells 20 \% of its output to A, 10 \% to E, and 30 \% to M and retains the rest.
  \begin{enumerate}
    \item Construct the exchange table for this economy.
    \item Find the set of equilibrium prices for the economy.
  \end{enumerate}
}

\sol{
  \begin{enumerate}
    \item
          \begin{table}[h!]
            \begin{center}
              \begin{tabular}{|c|c|c|c|c|}
                \hline
                Agriculture & Energy & Manufacturing & Transportation & Sold By        \\ [0.5ex]
                \hline
                \hline
                0.65        & 0.1    & 0.25          & 0              & Agriculture    \\
                0.3         & 0.1    & 0.35          & 0.25           & Energy         \\
                0.3         & 0.15   & 0.15          & 0.4            & Manufacturing  \\
                0.2         & 0.1    & 0.3           & 0.4            & Transportation \\
                \hline
              \end{tabular}
            \end{center}
          \end{table}
  \end{enumerate}
}

\qs{}{
  Ashesi University owns two farms. When it is operated for a month, Farm \#1 produces 50 kg of kale, 56 kg of onions and
  34 kg of tomatoes. A month's operation at Farm \#2 on the other hand yields 35 kg of kale, 14 kg of onions and 123 kg of
  tomatoes. The vectors $\mbold{v}_1 = \begin{bmatrix} 50 \\ 56 \\ 34 \end{bmatrix} $ and $\mbold{v}_2 = \begin{bmatrix}
      35 \\ 14 \\ 123 \end{bmatrix} $, then represent the output per month of Famr \#1 and Farm \#2 respectively.
  \begin{enumerate}
    \item What physical interpretation can be given to the vector $3 \mbold{v}_2$
    \item Suppose the university operates Farm \#1 for $x_1$ months and Farm \#2 for $x_2$ months. Write a vector equation
          whose solution gives the number of months each farm should operate in order to produce in total 830 kg of kale,
          728 kg of onions, 1358 kg of tomatoes. Explain. Do not solve the equation
  \end{enumerate}
}

\sol{
  \begin{enumerate}
    \item This represents the output of Farm \#2 if it were to be operated for 3 months.
    \item
          \begin{align*}
            x_1 \begin{bmatrix} 50 \\ 56 \\ 34 \end{bmatrix} + x_2 \begin{bmatrix}
                                                                     35 \\ 14 \\ 123 \end{bmatrix} & = \begin{bmatrix} 830
                                                                                                       \\ 728 \\ 1358\end{bmatrix} \\
            x_1 \mbold{v}_1 + x_2 \mbold{v}_2                                      & = \begin{bmatrix} 830
                                                                                         \\ 728 \\ 1358\end{bmatrix}
          \end{align*}
          This equation represents the total output of the two farms if they were to be operated for $x_1$ and $x_2$ months
          respectively. This is a linear combination of the output of the two farms, weighted by the number of months each farm is
          operated.
  \end{enumerate}
}

\qs{}{
  Intersections in England are often constructed as one-way "roundabouts" such as the one shown
  in the figure. Assume that traffic must travel in the directions shown.
  \begin{enumerate}
    \item Find the general solution of the network flow.
    \item Find the smallest possible value for $x_6$.
  \end{enumerate}
}
\begin{align*}
  A: & x_1  = 100 + x_2              \\
  B: & 50 + x_2 = x_3                \\
  C: & x_3 = x_4 + 120               \\
  D: & x_4 + 150 = x_5               \\
  E: & x_5 = x_6 + 80                \\
  F: & x_6 + 100 = x_1               \\
  \\
  x_1 - x_2 = 100                    \\
  x_2 - x_3 = -50                    \\
  x_3 - x_4 = 120                    \\
  x_4 - x_5 = -150                   \\
  x_5 - x_6 = 80                     \\
  x_6 - x_1 = -100                   \\
  \\
  \begin{bmatrix}
    1  & -1 & 0  & 0  & 0  & 0  & 100  \\
    0  & 1  & -1 & 0  & 0  & 0  & -50  \\
    0  & 0  & 1  & -1 & 0  & 0  & 120  \\
    0  & 0  & 0  & 1  & -1 & 0  & -150 \\
    0  & 0  & 0  & 0  & 1  & -1 & 80   \\
    -1 & 0  & 0  & 0  & 0  & 1  & -100 \\
  \end{bmatrix} \\
  R_4 \leftrightarrow R_6            \\
  \begin{bmatrix}
    1  & -1 & 0  & 0  & 0  & 0  & 100  \\
    0  & 1  & -1 & 0  & 0  & 0  & -50  \\
    0  & 0  & 1  & -1 & 0  & 0  & 120  \\
    -1 & 0  & 0  & 0  & 0  & 1  & -100 \\
    0  & 0  & 0  & 0  & 1  & -1 & 80   \\
    0  & 0  & 0  & 1  & -1 & 0  & -150 \\
  \end{bmatrix}
  \\
  -R_1 - R_4  \rightarrow R_4        \\
  \begin{bmatrix}
    1 & -1 & 0  & 0  & 0  & 0  & 100  \\
    0 & 1  & -1 & 0  & 0  & 0  & -50  \\
    0 & 0  & 1  & -1 & 0  & 0  & 120  \\
    0 & 1  & 0  & 0  & 0  & -1 & 0    \\
    0 & 0  & 0  & 0  & 1  & -1 & 80   \\
    0 & 0  & 0  & 1  & -1 & 0  & -150 \\
  \end{bmatrix}
  \\
  R_2 - R_4  \rightarrow R_4         \\
  \begin{bmatrix}
    1 & -1 & 0  & 0  & 0  & 0  & 100  \\
    0 & 1  & -1 & 0  & 0  & 0  & -50  \\
    0 & 0  & 1  & -1 & 0  & 0  & 120  \\
    0 & 0  & -1 & 0  & 0  & 1  & -50  \\
    0 & 0  & 0  & 0  & 1  & -1 & 80   \\
    0 & 0  & 0  & 1  & -1 & 0  & -150 \\
  \end{bmatrix}
  \\
  -R_3 - R_4  \rightarrow R_4        \\
  \begin{bmatrix}
    1 & -1 & 0  & 0  & 0  & 0  & 100  \\
    0 & 1  & -1 & 0  & 0  & 0  & -50  \\
    0 & 0  & 1  & -1 & 0  & 0  & 120  \\
    0 & 0  & 0  & 1  & 0  & -1 & -70  \\
    0 & 0  & 0  & 0  & 1  & -1 & 80   \\
    0 & 0  & 0  & 1  & -1 & 0  & -150 \\
  \end{bmatrix}
  \\
  \\
\end{align*}
\begin{align*}
  R_4 - R_6  \rightarrow R_6      \\
  \begin{bmatrix}
    1 & -1 & 0  & 0  & 0 & 0  & 100 \\
    0 & 1  & -1 & 0  & 0 & 0  & -50 \\
    0 & 0  & 1  & -1 & 0 & 0  & 120 \\
    0 & 0  & 0  & 1  & 0 & -1 & -70 \\
    0 & 0  & 0  & 0  & 1 & -1 & 80  \\
    0 & 0  & 0  & 0  & 1 & -1 & 80  \\
  \end{bmatrix}
  \\
  R_5 - R_6  \rightarrow R_6      \\
  \begin{bmatrix}
    1 & -1 & 0  & 0  & 0 & 0  & 100 \\
    0 & 1  & -1 & 0  & 0 & 0  & -50 \\
    0 & 0  & 1  & -1 & 0 & 0  & 120 \\
    0 & 0  & 0  & 1  & 0 & -1 & -70 \\
    0 & 0  & 0  & 0  & 1 & -1 & 80  \\
    0 & 0  & 0  & 0  & 0 & 0  & 0   \\
  \end{bmatrix}
  \\
  -R_2 - R_1  \rightarrow R_1     \\
  \begin{bmatrix}
    -1 & 0 & 1  & 0  & 0 & 0  & -50 \\
    0  & 1 & -1 & 0  & 0 & 0  & -50 \\
    0  & 0 & 1  & -1 & 0 & 0  & 120 \\
    0  & 0 & 0  & 1  & 0 & -1 & -70 \\
    0  & 0 & 0  & 0  & 1 & -1 & 80  \\
    0  & 0 & 0  & 0  & 0 & 0  & 0   \\
  \end{bmatrix}
  \\
  -R_3 - R_2  \rightarrow R_2     \\
  \begin{bmatrix}
    -1 & 0  & 1 & 0  & 0 & 0  & -50 \\
    0  & -1 & 0 & 1  & 0 & 0  & -70 \\
    0  & 0  & 1 & -1 & 0 & 0  & 120 \\
    0  & 0  & 0 & 1  & 0 & -1 & -70 \\
    0  & 0  & 0 & 0  & 1 & -1 & 80  \\
    0  & 0  & 0 & 0  & 0 & 0  & 0   \\
  \end{bmatrix} \\
  R_3 - R_1  \rightarrow R_1      \\
  \begin{bmatrix}
    1 & 0  & 0 & -1 & 0 & 0  & 170 \\
    0 & -1 & 0 & 1  & 0 & 0  & -70 \\
    0 & 0  & 1 & -1 & 0 & 0  & 120 \\
    0 & 0  & 0 & 1  & 0 & -1 & -70 \\
    0 & 0  & 0 & 0  & 1 & -1 & 80  \\
    0 & 0  & 0 & 0  & 0 & 0  & 0   \\
  \end{bmatrix}
  \\
\end{align*}
\begin{align*}
  -R_4 - R_3  \rightarrow R_3     \\
  \begin{bmatrix}
    1 & 0  & 0  & -1 & 0 & 0  & 170 \\
    0 & -1 & 0  & 1  & 0 & 0  & -70 \\
    0 & 0  & -1 & 0  & 0 & 1  & -50 \\
    0 & 0  & 0  & 1  & 0 & -1 & -70 \\
    0 & 0  & 0  & 0  & 1 & -1 & 80  \\
    0 & 0  & 0  & 0  & 0 & 0  & 0   \\
  \end{bmatrix}
  \\
  R_4 - R_2  \rightarrow R_2      \\
  \begin{bmatrix}
    1 & 0 & 0  & -1 & 0 & 0  & 170 \\
    0 & 1 & 0  & 0  & 0 & -1 & 0   \\
    0 & 0 & -1 & 0  & 0 & 1  & -50 \\
    0 & 0 & 0  & 1  & 0 & -1 & -70 \\
    0 & 0 & 0  & 0  & 1 & -1 & 80  \\
    0 & 0 & 0  & 0  & 0 & 0  & 0   \\
  \end{bmatrix}
  \\
  -R_4 - R_1  \rightarrow R_1     \\
  \begin{bmatrix}
    -1 & 0 & 0  & 0 & 0 & 1  & -100 \\
    0  & 1 & 0  & 0 & 0 & -1 & 0    \\
    0  & 0 & -1 & 0 & 0 & 1  & -50  \\
    0  & 0 & 0  & 1 & 0 & -1 & -70  \\
    0  & 0 & 0  & 0 & 1 & -1 & 80   \\
    0  & 0 & 0  & 0 & 0 & 0  & 0    \\
  \end{bmatrix} \\
  -R_1 \to R_1                    \\
  -R_2 \to R_2                    \\
  \begin{bmatrix}
    1 & 0 & 0 & 0 & 0 & -1 & 100 \\
    0 & 1 & 0 & 0 & 0 & -1 & 0   \\
    0 & 0 & 1 & 0 & 0 & -1 & 50  \\
    0 & 0 & 0 & 1 & 0 & -1 & -70 \\
    0 & 0 & 0 & 0 & 1 & -1 & 80  \\
    0 & 0 & 0 & 0 & 0 & 0  & 0   \\
  \end{bmatrix}
  \\
  x_1 - x_6 = 100                 \\
  x_2 - x_6 = 0                   \\
  x_3 - x_6 = 50                  \\
  x_4 - x_6 = -70                 \\
  x_5 - x_6 = 80                  \\
  x_6 = x_6                       \\
  \\
  \begin{cases}
    x_1 = x_6 + 100 \\
    x_2 = x_6       \\
    x_3 = x_6 + 50  \\
    x_4 = x_6 - 70  \\
    x_5 = x_6 + 80  \\
    x_6 \text{ is free}
  \end{cases}
\end{align*}

The smallest value of $x_6$ is 0 as $x_6$ is the same as the outflow of $x_2$  which cannot be negative.

\qs{}{
  Let $T:\mathbb{R}^2 \to \mathbb{R}^3$ be a linear transformation with $T \left( x_1, x_2 \right) = \left( 2x_1 - x_2,
    -3x_1 + x_2, 2x_1-3x_2 \right)  $
  \begin{enumerate}
    \item Find the standard matrix of $T$
    \item Find $\mbold{x}$ such that $T \left( \mbold{x} \right) = \left( 0,-1,4 \right)  $
    \item Is $T$ onto? Is it one-to-one? Explain.
  \end{enumerate}
}

\sol{
  \begin{enumerate}
    \item
          \begin{align*}
            T \left( \mbold{x} \right)                                                            & = A \mbold{x}                                                                     \\
            I_2 = \begin{bmatrix} 1 & 0 \\ 0 & 1 \end{bmatrix} \quad \mbold{e}_1                  & = \begin{bmatrix} 1 \\ 0 \end{bmatrix}
            \quad \mbold{e}_2 = \begin{bmatrix} 0 \\ 1 \end{bmatrix}                                                                                                                  \\
            \mbold{x}                                                                             & = x_1 \mbold{e}_1 + x_2 \mbold{e}_2                                               \\
            T \left( \mbold{x} \right)                                                            & = x_1 T \left( \mbold{e}_1 \right) + x_2 T \left( \mbold{e}_2 \right)             \\
            \therefore T \left( \mbold{e}_1 \right)  = \begin{bmatrix} 2 \\ -3 \\ 2 \end{bmatrix} & \quad T \left( \mbold{e}_2 \right)  = \begin{bmatrix} -1 \\ 1 \\ -3 \end{bmatrix} \\
            T \left( \mbold{x} \right)                                                            & = \begin{bmatrix} T \left( \mbold{e}_1 \right) & T \left( \mbold{e}_2 \right)
                                                                                                      \end{bmatrix} \begin{bmatrix} x_1 \\ x_2 \end{bmatrix} = A \mbold{x}     \\
            \text{Hence } A                                                                       & = \begin{bmatrix}
                                                                                                        2  & -1 \\
                                                                                                        -3 & 1  \\
                                                                                                        2  & -3
                                                                                                      \end{bmatrix}                                                                  \\
          \end{align*}
    \item
          \begin{align*}
            T \left( \mbold{x} \right)                             & = \begin{bmatrix} 0 \\ -1 \\ 4 \end{bmatrix} \\
            \begin{bmatrix}
              2  & -1 \\
              -3 & 1  \\
              2  & -3
            \end{bmatrix} \begin{bmatrix} x_1 \\ x_2 \end{bmatrix} & = \begin{bmatrix} 0 \\ -1 \\ 4 \end{bmatrix} \\
            \\
          \end{align*}
          \begin{align*}
            \begin{bmatrix}
              2  & -1 & 0  \\
              -3 & 1  & -1 \\
              2  & -3 & 4
            \end{bmatrix}                         \\
            \frac{-3}{2}R_1 - R_2  \rightarrow R_2 \\
            \begin{bmatrix}
              2 & -1          & 0 \\
              0 & \frac{1}{2} & 1 \\
              2 & -3          & 4 \\
            \end{bmatrix}
            \\
            R_1 - R_3  \rightarrow R_3             \\
            \begin{bmatrix}
              2 & -1          & 0  \\
              0 & \frac{1}{2} & 1  \\
              0 & 2           & -4 \\
            \end{bmatrix}
            \\
            4R_2 - R_3  \rightarrow R_3            \\
            \begin{bmatrix}
              2 & -1          & 0 \\
              0 & \frac{1}{2} & 1 \\
              0 & 0           & 8 \\
            \end{bmatrix}
            \\
            -2R_2 - R_1  \rightarrow R_1           \\
            \begin{bmatrix}
              -2 & 0           & -2 \\
              0  & \frac{1}{2} & 1  \\
              0  & 0           & 8  \\
            \end{bmatrix}
            \\
            -\frac{1}{2}R_1 \to R_1                \\
            2R_2 \to R_2                           \\
            \begin{bmatrix}
              1 & 0 & 1 \\
              0 & 1 & 2 \\
              0 & 0 & 8
            \end{bmatrix}                         \\
            x_1 = 1                                \\
            x_2 = 2                                \\
            \\
            \mbold{x} = \begin{bmatrix} 1 \\ 2 \end{bmatrix}
          \end{align*}
    \item
          $T$ is onto for $ \left( 0, -1,4 \right) $ as $ \left( 0, -1,4 \right) $ produces an image in the
          $\mathbb{R}^3$  vector space, and $T$ is also one-to-one as it produces only one, a unique, image.
  \end{enumerate}
}



\end{document}
