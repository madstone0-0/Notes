\documentclass[12pt letter]{report}
\input{./template/preamble}
\input{./template/macros}
\input{./template/letterfonts}

\title{\Huge{Introduction}}
\author{\huge{Madiba Hudson-Quansah}}
\date{}
\usepackage{parskip}

\setcounter{tocdepth}{4}
\setcounter{secnumdepth}{4}

\begin{document}
\maketitle
\newpage
\pdfbookmark[section]{\contentsname}{too}
\tableofcontents
\pagebreak

\chapter{The Internet}

\dfn{Network Edge}{
  Hosts, access networks physical media
}

\dfn{Network core}{
  Packet/Circuit switching, internet structure
}

\dfn{Performance}{
  Loss, Delay, Throughput
}

\dfn{Host}{
  End Systems, Clients/Servers
}

\dfn{Internet}{
  Network of networks, interconnected ISPs
}

\section{Nuts and Bolts Description of the Internet}

The internet is a computer network that interconnects billions of
computing devices  throughout the world.

End systems are connected together by a network of communication
links and packet switches. Different links can transmit data at
different rates, with the \textbf{transmission rate} of a link
measured in bits/second. The resulting units of information, known as
packets, are then sent throughout the network to the destination end
system, where they are reassembled into the original data.

\dfn{Transmission Rate}{
  The rate, in bits/second, at which a link can transport data from
  one end to the other
}

\dfn{Packet Switch}{
  Forwards packets arriving on incoming communication links to
  outgoing communication links.
}

Packet switches interconnect communication links, with the two main
types of packets switches being:
\begin{description}
  \item[Routers] - Typically used in wide-area networks/network core
    e.g. the Internet
  \item[Link Layer Switches] - Typically used in access networks e.g. LANs
\end{description}
Both types forward packets to their ultimate destinations, and
interconnected packet switches comprise a route/path through the network

\dfn{Route / Path}{
  The sequence of communication links and packets switches traversed
  by a packet from the sending end system to the receiving end system.
}

\dfn{Internet Service Provider (ISP)}{
  A network of packet switches and communication links that provides
  access to the Internet.
}

End systems access the internet through \textbf{Internet Service
Providers (ISP)}. ISPs can be classified into tiers:
\begin{description}
  \item[Lower-Tier ISPs (Tier 1)]  - Small regional ISPs that
    provide access to end systems and small businesses.
  \item[Upper-Tier ISPs (Tier 2)] -  Consists of high-speed routers
    interconnected with high-speed fiber-optic links. They provide
    access to lower-tier ISPs and large businesses, and are connected
    directly to one another.
\end{description}

Each ISP on either tier is managed independently, runs the IP
protocol, and conforms to certain naming addresses and conventions.

End systems, packet switches and other network devices run
\textbf{protocols \ref{def:protocol}} that manage the sending and receiving of
information within the internet. The two most important protocols on
the internet are:
\begin{description}
  \item[Transmission Control Protocol (TCP)] - Manages the sending
    and receiving of messages, ensuring reliable delivery of data.
  \item[Internet Protocol (IP)] - Specifies the format of packets
    that are sent and received among routers and end systems.
\end{description}
Together, these two protocols are referred to as TCP/IP.

The \textbf{Internet standards} are developed by the Internet
Engineering Task Force \textbf{IETF}, and are documented in
\textbf{Request For Comments (RFCs)}.

\section{Services Based Description of the Internet}

The internet can be described in terms of the services it provides,
i.e. as an infrastructure that provides services to applications.
These applications include email, web surfing, internet messaging,
mapping, video/music streaming, etc. These applications are said to
be \textbf{Distributed Applications}

\dfn{Distributed Applications}{
  Applications that involve multiple end systems that exchange data
  with each other.
}

Internet applications run on end systems, not on packet switches in
the network core, i.e. the network core is not concerned with the
applications that are the sources and sinks of data.

End systems connected to the internet provide a \textbf{socket
interface} to applications for the purpose of sending and receiving data.

\dfn{Socket Interface}{
  Specifies how a program running on one end system asks the internet
  infrastructure to deliver data to a specific destination program
  running on another end system. It is a set of rules that the
  sending program must follow so that the internet can deliver the
  data to the destination program.
}

\section{Protocols}

\dfn{Protocol}{\label{def:protocol}
  Defines the format, order of messages send and received among
  network entities, and actions taken on message transmission/receipt.
}

All activity in the internet that involves two or more communicating
remote entities is managed by a protocol.

\chapter{Network Edge}

\dfn{Network Edge}{
  End systems and access networks that connect end systems to the
  network core.
}

\section{Access Networks}

\dfn{Access Network}{
  The network that physically connects an end system to the
  first/edge router on a path from the end system to any other
  distant end system.
}

\subsection{Home Access}

The two most common types of broadband residential access are:
\begin{description}
  \item[Digial Subscriber Line (DSL)] - Operates over existing
    telephone lines provided by a telephone company (Telco), with
    maximum rates ranging from hundreds of
    Kbps to several Mbps. The speed is distance dependent, with
    shorter distances yielding higher speeds. When DSL is used a
    customer's telco is also their ISP.
  \item[Cable] - Makes use of the cable television (CATV)
    infrastructure to provide broadband access to homes. Cable
    modems provide data rates ranging from several Mbps to 100's of
    Mbps. Cable is a shared medium, with the bandwidth being shared
    among all users in a local area.
\end{description}

\dfn{Modem}{
  Converts digital signals from a computer to analogue signals for
  transmission over telephone lines, and vice versa.
}

Each DSL modem uses existing telephone lines to exchange data with a
digital subscriber line access multiplexer (DSLAM) located in the
telco's local central office  (CO). The home's DSL modem takes
digital data and translates it to high-frequency tones for
transmission over the telephone wires to the CO, with the analogue
signals from many houses being translated back into digital format at
the DSLAM. The residential telephone line carries both data telephone
signals encoded at different frequencies:
\begin{description}
  \item[High-Speed Downstream]  - 50kHz to 1MHz
  \item[Medium-Speed upstream] - 4kHz to 50kHz
  \item[Two-Way telephone] - 0 to 4kHz
\end{description}

\subsection{Hosts}

In transmitting data a host:
\begin{enumerate}
  \item Takes application data
  \item Breaks it down into smaller chunks known as packets of length $L$ bits
  \item Transmits these packets into the access network at a
    transmission rate of $R$ bps, this is also known as the link's
    transmission rate or bandwidth.
\end{enumerate}

The packet transmission delay is given by:
\begin{align*}
  \text{Packet Transmission Delay} &= \frac{L \text{ bits}}{R \text{bps}}
\end{align*}

\ex{}{
  Consider a packet of length $L = 1000 \text{ bits}$, and a link
  with a bandwidth of $R = 1 Mbps$, what is the packet transmission delay?
  \begin{align*}
    \text{Packet Transmission Delay} &= \frac{L \text{ bits}}{R \text{bps}} \\
    &= \frac{1000 \text{ bits}}{1,000,000 \text{ bps}} \\
    &= 0.001 \text{ seconds} = 1 \text{ ms}
  \end{align*}

  Now assume there are two packets, each of length $L = 1000
  \text{ bits}$, that are sent back-to-back on the same link with a
  bandwidth of $R = 1 Mbps$. What is the time taken to send both
  packets?
  \begin{align*}
    \text{Time to send 2 packets} &= 2 \times \frac{L \text{
    bits}}{R \text{bps}} \\
    &= 2 \times \frac{1000 \text{ bits}}{1,000,000 \text{ bps}} \\
    &= 0.002 \text{ seconds} = 2 \text{ ms}
  \end{align*}
}

\section{Physical Media}

\dfn{Bit}{
  The smallest unit of data in a computer network, represented as a
  0 or 1. Propagates between transmitter and receiver pairs over a
  physical medium.
}

\dfn{Physical Media}{
  The physical materials that carry bits/signals between end systems and
  packet switches (transmitters and receivers). This is not excluding
  wireless media, which passes
  through the air which is also physical.
}

Examples of physical media include:
\begin{itemize}
  \item Twisted Pair Copper Wire
  \item Coaxial Cable
  \item Multimode Fiber Optics
  \item Terrestrial Radio Spectrum
  \item Satellite Radio Spectrum
\end{itemize}

Physical media falls into two broad categories:
\begin{description}
  \item[Guided Media]  - Signals are guided along a solid medium e.g.
    twisted pair copper wire, coaxial cable, fiber optics.
  \item[Unguided Media]- Signals are transmitted through the
    atmosphere and in outer space e.g. terrestrial radio, satellite radio.
\end{description}

\subsection{Guided Media}

\subsubsection{Twisted Pair Copper Wire}

The least expensive and most commonly used guided transmission
medium. It consists of two insulated copper wires twisted together to
reduce electrical interference from similar pairs close by. Typically
a number of pairs are bundled together in a cable by wrapping the
pairs in a protective shield. A wire pair constitutes a single
communication link.

Unshielded Twisted Pair (UTP) is commonly used for computer networks
within a building, i.e. LANs. Data rates for LANs using twisted pair
range from 10 Mbps to 10 Gbps, depending on the thickness of the wire
and the distance between transmitter and receiver.

\nparagraph{Advantages}
\begin{itemize}
  \item Inexpensive
  \item Easy to install
\end{itemize}

\nparagraph{Disadvantages}
\begin{itemize}
  \item Susceptible to electromagnetic interference
  \item Limited bandwidth and distance
\end{itemize}

\subsubsection{Coaxial Cable}

Consists of two copper conductors, but the two conductors are
concentric, rather than parallel. The inner conductor is surrounded
by an insulating material, which is then surrounded by a cylindrical
conducting shield. The shield serves to reduce electromagnetic
interference and allows for higher bandwidth and longer distances
than twisted pair. Coaxial cable is commonly used for cable
television and broadband internet access. Can be used as a guided
\textbf{shared medium}, where multiple transmitters and receivers are
connected to the same coaxial

\nparagraph{Advantages}
\begin{itemize}
  \item Higher bandwidth and longer distance than twisted pair
  \item Less susceptible to electromagnetic interference than twisted pair
\end{itemize}

\nparagraph{Disadvantages}
\begin{itemize}
  \item More expensive than twisted pair
  \item More difficult to install than twisted pair
\end{itemize}

\subsubsection{Fiber Optics}

A thin, flexible medium that conducts light pulses with each pulse
representing a bit. A single optical fiber can support large bit
rates over long distances, with data rates of 10 Gbps or more over
distances of tens of kilometres. Fiber optics is commonly used for
high-speed data transmission in the core of the internet and for
long-distance telecommunications.

\nparagraph{Advantages}
\begin{itemize}
  \item Very high bandwidth and long distance capabilities
  \item Immune to electromagnetic interference
\end{itemize}

\nparagraph{Disadvantages}
\begin{itemize}
  \item More expensive than twisted pair and coaxial cable
  \item More difficult to install than twisted pair and coaxial cable
\end{itemize}

\subsection{Unguided Media}

\subsubsection{Terrestrial Radio Spectrum}

Radio channels carry signals in the electromagnetic spectrum through
the atmosphere. They are used for wireless communication between end
systems and packet switches, and between packet switches in the
network core. Examples of terrestrial radio channels include Wi-Fi,
cellular networks, and Bluetooth. These can be classified into three groups:
\begin{description}
  \item[Short Distance]  - Operate within on or to meters, e.g. Bluetooth
  \item[Local Area] - Operate within ten to a few hundred meters, e.g. Wi-Fi
  \item[Wide Area] - Operate over large areas, spanning tens of
    kilometres, e.g. cellular networks
\end{description}

\subsubsection{Satellite Radio Channels}

A communication satellite links two or more Earth-based microwave
transmitter/receivers known as ground stations. The satellite
receives signals on one frequency band, regenerates the signal using
a repeater and transmits the signal on another frequency. There are
two types of satellite radio channels:
\begin{description}
  \item[Geostationary Earth Orbit (GEO)] - Satellites that orbit
    approximately 36,000 km above the Earth and remain fixed in the
    same position relative to the Earth. GEO satellites are commonly
    used for television broadcasting and long-distance communication.
  \item[Low Earth Orbit (LEO)] - Satellites that orbit much closer to
    the Earth, typically at
    altitudes of 500 to 2,000 km. LEO satellites are used for
    applications such as satellite internet and remote sensing.
\end{description}

\chapter{Network Core}

\dfn{Network Core}{
  The mesh of interconnected packet switches and communication links
  that interconnects end systems.
}

\section{Packet Switching}

\dfn{Packet Switching / Forwarding}{
  A network in which messages are broken into smaller chunks known as
  packets that are sent from source to destination through packet
  switches.
}

\dfn{Message}{
  A chunk of data sent from a source to a destination end system.
}

In a network application end systems exchange messages with each
other. To send a message from a source end system to a destination
end system, the message is broken down into smaller chunks known as
packets. Between source and destination, each packet travels through
communication links and packet switches, (link-layer switches in the
case of the network edge). Packets are transmitted over each
communication link at a rate equal to the full transmission rate of
the link, i.e. if a source system or a packet is sending a packet of
$L$ bits over a link with a transmission rate of $R$ bps, the time
taken to transmit the packet onto the link is given by:
\begin{align*}
  \text{Packet Transmission Delay} &= \frac{L \text{ bits}}{R \text{bps}}
\end{align*}

\dfn{Forwarding}{
  The local action of transmitting a packet from the outbound link of
  one packet switch to the inbound link of the next packet switch.
}

\dfn{Routing}{
  The global action of sending a packet from a source node to a
  destination node through multiple packet switches.
}

The local action of transmitting a packet from the outbound link of
one packet switch to the inbound link of the next packet switch is
known as \textbf{forwarding}. The global action of sending a packet
from a source node to a destination node through multiple packet
switches is known as \textbf{routing}.

\subsection{Store-And=Forward Transmission}

Store-and-forward transmission, requires a packet switch receive the
entire packet before it can begin to transmit the first bit of the
packet onto the outbound link. Thus, the time taken to transmit a
packet from the source end system to the destination end system is given by:
\begin{align*}
  \text{Total Transmission Time} &= N \times \frac{L \text{ bits}}{R \text{bps}}
\end{align*}
where $N$ is the number of links (or packet switches) between source
and destination. A packet switch will always have an inbound and
outbound link so with $k$ packet switches there will be $k+1$ links.
Given $P$ packets
the total time taken to transmit all packets from source to
destination is given by:
\begin{align*}
  \text{Total Transmission Time} &= (N + P - 1) \times \frac{L \text{
  bits}}{R \text{bps}}
\end{align*}
Due to store-and-forward transmission, the first packet must be
transmitted completely before the second packet can be transmitted,
hence the $P-1$ term in the equation above.

\subsection{Queueing Delay and Packet Loss}

\dfn{Packet Loss}{
  The dropping of packets that arrive at a packet switch when the
  output buffer/queue is full.
}

Each packet switch has multiple links attached to it, and for each
attached link there is an \textbf{output buffer/ queue}, which stores
packets that the router is about to send into that link. The output
buffer allows busy links to temporarily store arriving packets until
the link is ready to transmit them. Thus in addition to the
store-and-forward delay, packets also experience \textbf{queueing
delay} at each packet switch. This delay depends on the level of
congestion in the network. Since the amount of buffer space is
finite, an arriving packet may meet a full buffer and be dropped,
this is known as \textbf{packet loss}.

\subsection{Forwarding Tables and Routing Protocols}

Packet forwarding is done in different ways depending on the type of
packet switch. In general, each packet switch maintains a
\textbf{forwarding table} that maps destination addresses to that
router's outbound links. When a packet arrives a router the router
examines the address and searches its forwarding table using the
destination address to find the appropriate outbound link for that packet.

The forwarding tables are populated using \textbf{routing protocols},
which are distributed algorithms that determine the route that
packets take from source to destination.

\section{Circuit Switching}

\dfn{Circuit Switching}{
  A network in which a dedicated path (circuit) is established
  between a sender and receiver for the duration of the communication
  session.
}

In circuit switched networks the resources needed along a path to
provide for communication between end systems are reserved for the
duration of the communication session between the end systems.
Traditional telephone networks are examples of circuit switched networks.

In a circuit switched network, when a user wishes to communicate with
another user, a dedicated end-to-end path (circuit) is established
between the two users. This path consists of a sequence of links and
switches, with each link along the path having a fixed amount
of bandwidth reserved for the duration of the communication session.
The bandwidth used by a user during a communication session is a
fraction of the total bandwidth of each link along the path, for
example if a link where has a bandwidth of $R$ bps and there are $N$ circuits
established on that link, then each circuit is allocated a bandwidth of
$\frac{R}{N}$ bps.

Due to the reservation of resources along the path the maximum number
of active users at any time is the total number of circuits that can
be established in the network.

Considering the case of sending a file of $640000$
bits from Host A to Host B in a circuit switched network, with $24$ slots
and a bit rat of $1.536 \text{ Mbps}$. Also assuming it takes $500
\text{ ms}$   to establish an end-to-end circuit between Host A and
Host B, the time taken to send the file is given by:
\begin{align*}
  \text{Single Transmission Rate} &= \frac{1.536}{24}  \\
  &= 64 \text{ kbps} \\
  \\
  t_t &= \frac{640}{64} \\
  &= 10 \text{ seconds } \\
  \\
  t_{\text{total}} &= t_{\text{establish}} + t_t \\
  &= 0.5 + 10 \\
  &= 10.5 \text{ seconds}
\end{align*}

\subsection{Multiplexing in Circuit-Switched Networks}

Circuit switching can be implemented with either \textbf{Frequency
Division Multiplexing (FDM)} or \textbf{Time Division Multiplexing (TDM)}.

\subsubsection{Frequency Division Multiplexing (FDM)}

The frequency spectrum of a link is divided up among the connections
established across the link, i.e. dedicating a frequency band to each
connection for the duration of the connection, the width of this
known as the \textbf{bandwidth} of the connection. For example, if a
link has a bandwidth of $R$ bps and there are $N$ connections
established across the link, then each connection is allocated a
bandwidth of $\frac{R}{N}$ bps.

\subsubsection{Time Division Multiplexing (TDM)}

Time is divided into frames of fixed duration, and each frame is
divided into a fixed number of time slots. When the network
establishes a connection across a link, the network dedicates one
time slot in every frame to this connection. These slots are
dedicated for the sole use of that connection, with one time slot
available for use to transmit the connection's data, in every frame.
For example if a link has 5 connections established across it, then
each connection is allocated one time slot in every frame, and thus
can transmit data in one time slot in every frame, i.e. a frame would
consist of 5 time slots, with each time slot allocated to one of the 5.

\section{Circuit Switching vs Packet Switching}

An argument against circuit switching is that it is wasteful because
the dedicated circuits are idle during silent periods, and thus the
resources reserved for a connection cannot be used by other ongoing connections.

Packet switching is more efficient than circuit switching because it
allows the resources of the network to be shared among all users, with
the resources being used only when there is data to send. However,
packet switching can lead to
congestion and packet loss when the demand for resources exceeds the
available resources, whereas circuit switching can lead to blocking
when there are too many connection requests and not enough resources
to establish all the requested connections.

\chapter{Delay, Loss, and Throughput in Packet-Switched Networks}

\section{Delay}

\dfn{Delay}{
  The time it takes for a packet to travel from source to
  destination. The most import of these delays are
  \begin{enumerate}
    \item Nodal Processing Delay
    \item Queueing Delay
    \item Transmission Delay
    \item Propagation Delay
  \end{enumerate}
  Together these give the \textbf{Total Nodal Delay}
}

\subsection{Types of Delay}

Most packet switched networks experience four types of delay at each
packet switch / node:
\begin{description}
  \item[Nodal Processing Delay]  - When the packet arrives at the inbound link
  \item[Queueing Delay] - When the packet is waiting in the output buffer
  \item[Transmission Delay] - When the packet is being transmitted
    onto the outbound link
  \item[Propagation Delay] - When the packet is moving through
    the physical medium
\end{description}

\subsubsection{Processing Delay}

The time required to examine the packet's header and determine where
to direct the packet. Processing delay can also include other
factors, like error checking. Processing delays in high-speed routers
are usually on the order or microseconds or les.

\subsubsection{Queueing Delay}

The time a packet spends waiting to be transmitted onto the outbound
link. The queueing delay of a specific packet depends on the number
of earlier arriving packets that are queued and waiting for
transmission onto the link. If the queue is empty and no other packet
is currency being transmitted the queueing delay is zero.

\subsubsection{Transmission Delay}

The time required to push all the packet's bits onto the link. It is
given by:
\begin{align*}
  \text{Transmission Delay} &= \frac{L \text{ bits}}{R \text{bps}}
\end{align*}

\subsubsection{Propagation Delay}

The time required for a bit to propagate from the beginning of the
link to the destination router at the other end. The propagation
speed depends on the physical medium of the link and is in the range of
\[
  2\cdot 10^{8} \text{m/s} \text{ to } 3 \cdot 10^{8} \text{m/s}
\]
This is less than or equal to the speed of light in a vacuum. The
propagation delay is the distance between two routers divided by he
propagation speed of the link i.e.:
\begin{align*}
  \text{Propagation Delay} &= \frac{d \text{ meters}}{s \text{
  meters/second}}
\end{align*}

\subsubsection{Transmission vs Propagation Delay}

Transmission delay is the amount of time needed for a router to push
out a packet's bits onto the link, whereas propagation delay is the
amount of time it takes a bit to traverse the link from one end to the other.

\subsubsection{Total Nodal Delay}

Given all these delays, the total nodal delay at a router is given by:
\begin{align*}
  d_{\text{nodal}} &=  d_{\text{proc}} + d_{\text{queue}} +
  d_{\text{trans}} + d_{\text{prop}} \\
  &= \text{Processing Delay} + \text{Queueing Delay} + \frac{L}{R} +
  \frac{d}{s} \\
\end{align*}

The contribution of these delays can vary significantly from one packet
switch to another, and from one packet to another at the same packet
switch.

\subsection{Queueing Delay and Packet Loss}

Unlike the three other delays the queueing delay can vary from packet
to packet, therefore when characterizing queueing delay it is common
to use statistical measurements, like average queueing delay,
variance, and the probability that the queueing delay exceeds some
specified value.

\subsection{End-to-End Delay}

The total delay from source to destination is the sum of the nodal
delays incurred at each node along the path from source to
destination. If there are $N - 1$ routers between source and destination
then there are $N$ links and the total end-to-end delay is given by:
\begin{align*}
  d_{\text{end-end}} &= N \left( d_{\text{proc}} + d_{\text{trans}} +
  d_{\text{prop}} \right)  \\
\end{align*}
Taking into account heterogeneous delays at the nodes and the present
of an average queueing delay at each node gives:
\begin{align*}
  d_{\text{end-end}} &= \sum_{i=1}^{N} \left( d_{\text{proc}_i} +
  d_{\text{queue}_i} + d_{\text{trans}_i} + d_{\text{prop}_i} \right) \\
\end{align*}

\section{Throughput}

\dfn{Throughput}{
  The rate (in bits/second) at which data is transferred between
  source and destination. There are two types of throughput:
  \begin{description}
    \item[Instantaneous Throughput] - The rate at which data is
      transferred at a given point in time.
    \item[Average Throughput] - The rate at which data is transferred
      over a longer period of time.
  \end{description}
}

\dfn{Bottleneck Link}{
  The link on the end-to-end path that constrains the end-to-end
  throughput to be less than the end-to-end throughput of all other
  links on the path.
}

Throughput can be measured at any point along the path from source to
destination, but the
end-to-end throughput is constrained by the link with the lowest
throughput, this link is known as the \textbf{bottleneck link}. Given
a path with 2 links with throughputs of $R_s$ and $R_c$, i.e. server
to link throughput and client to link throughput, the end-to-end
throughput is given by:
\begin{align*}
  R_{\text{end-end}} &= \min(R_s, R_c)
\end{align*}
Having determined the throughput we can approximate the time it takes
to transfer a large file of $F$ bits from server to client as:
\[
  t = \frac{F}{\min \left( R_s, R_c \right) }
\]
\ex{}{
  \qs{}{
    Given a file of $F = 32 \text{ MB}$ and the server has a
    transmission rate of $R_s = 2 \text{ Mbps}$, and you have an
    access link of $R_c = 1 Mbps$. How long will it take to transfer
    the file from the server to your computer?
  }

  \sol{
    \begin{align*}
      t &= \frac{32}{\min \left( 2, 1 \right) } \\
      &= \frac{32}{1} \\
      &= 32 \text{ seconds} \\
    \end{align*}
  }
}

This generalizes to a path with $N$ links with throughputs of $R_1,
R_2, \ldots, R_N$ as:
\begin{align*}
  R_{\text{end-end}} &= \min(R_1, R_2, \ldots, R_N)
\end{align*}
And the time to transfer a file of $F$ bits from server to client is given by:
\begin{align*}
  t &= \frac{F}{\min \left( R_1, R_2, \ldots, R_N \right) }
\end{align*}

In the case where multiple clients ( $N$ clients each of
transmission rate $R_c$) and servers ( $N$ servers each of
transmission rate $R_s$) are connected to the core of the computer
network, and there are $N$ simultaneous downloads in progress
involving all $N$ client-server pairs. There is a link at the core
traversed by all $N$ downloads with a transmission rate of $R$. If
the rate of $R$ is much much larger than $R_s$ and $R_c$ then the
throughput is still given by:
\[
  R_{\text{end-end}} = \min(R_s, R_c)
\]
But in the case where $R$ is comparable to $R_s$ and $R_c$ then the
common link divides its transmission rate equally among the $N$
downloads, giving a throughput of:
\[
  R_{\text{end-end}} = \min \left( R_s, R_c, \frac{R}{N} \right)
\]
Possible making the throughput much smaller than the case where $R$
is much larger than $R_s$ and $R_c$.

\end{document}
