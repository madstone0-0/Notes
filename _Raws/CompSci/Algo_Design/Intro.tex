\documentclass[12pt letter]{report}
\input{./template/preamble}
\input{./template/macros}
\input{./template/letterfonts}

\title{\Huge{Introduction}}
\author{\huge{Madiba Hudson-Quansah}}
\date{}
\usepackage{parskip}

\setcounter{tocdepth}{4}
\setcounter{secnumdepth}{4}

\begin{document}
\maketitle
\newpage
\pdfbookmark[section]{\contentsname}{too}
\tableofcontents
\pagebreak

\chapter{Introduction}

\section{What is an Algorithm}

\dfn{Algorithm}{
  A sequence of unambiguous instructions for solving a problem i.e. for obtaining a required output for any legitimate
  input in a finite amount of time.
}

Using the task of finding the greatest common divisor (GCD) of two numbers we can explicitly illustrate the general
notions of an algorithm. Namely:
\begin{itemize}
  \item The requirement of non-ambiguity in describing each step of an algorithm.
  \item  The specified range of inputs for which an algorithm works.
  \item  The different ways the same algorithm can be expressed / implemented.
  \item The non uniqueness of algorithms for a particular problem
  \item Different algorithms for the same problem may be based on different ideas and have different time and space complexities.
\end{itemize}


The GCD of two non-negative, non-zero integers $m$ and $n$, denoted $\text{gcd} \left( m, n \right) $ is defined as the
largest integer that divides both $m$ and $n$ evenly, i.e., with a remainder of zero. Euclid of Alexandria, a Greek
mathematician, outlined an algorithm for solving this problem in his book \textit{Elements}. Defined as follows:
\[
  \text{gcd} \left( m, n \right)  = \begin{cases}
    m                                      & \text{if } n = 0 \\
    \text{gcd} \left( n,  m \mod n \right) & \text{otherwise}
  \end{cases}
\]

Where $m \mod n$ is the remainder of the division of $m$ by $n$, and when n is zero the GCD is $m$ as it is the only and
largest number that divides $m$ evenly between  $m$ and $0$. In pseudocode:

\begin{algorithm}[H]
  \caption{gcd $ \left( m, n \right) $}
  \Comment{}\\
  \Comment{Computes the greatest common divisor of two numbers} \\
  \Comment{Input: Two non-negative integers $m$ and $n$} \\
  \Comment{Output: The greatest common divisor of $m$ and $n$}\\
  \begin{algorithmic}[1]
    \While{ $n \neq 0$}

    $r \gets m \mod n$

    $m \gets n$

    $n \gets r$
    \EndWhile
    \State \Return $r$
  \end{algorithmic}
\end{algorithm}

This algorithm eventually terminates as the value of $n$ decreases with each iteration as its value is set to $m \mod n$
every iteration, and the value of $m$ is always greater than $n$.

Another algorithm for solving this problem is based on the definition of the GCD as the greatest common divisor of $m$ and
$n$ as the largest integer that divides both number evenly. This leads us to the conclusion that such a common divisor
cannot be greater than the smallest of the two numbers, denoted $t = \min \left\{ m, n \right\} $. We can then check if
$t$ divides both numbers $m$ and $n$ evenly, if it does then $t$ is the GCD of $m$ and $n$. If it does not, we decrease
$t$ by 1 and then check if $t - 1$ divides both $m$ and $n$ evenly. This process is repeated until we find a number that
divide both $m$ and $n$ evenly or until $t = 1$ in which case the GCD is 1 as 1 divides all numbers evenly. In pseudocode:
\begin{algorithm}[H]
  \caption{Consecutive Integer Checking $ \left( m, n \right) $}
  \begin{algorithmic}[1] \\
    $t \gets \min \left\{ m, n \right\} $

    \While{ $t > 1$}

    \If{ $m  \mod t == 0$ and $n \mod t == 0$}
    \State \Return $t$
    \EndIf
    \State $t \gets t - 1$
    \EndWhile
    \State \Return $t$
  \end{algorithmic}
\end{algorithm}

The final algorithm considers the prime factorization of the two numbers $m$ and $n$. The GCD of two numbers is the
product of the common prime factors of the two numbers. As there is no unambiguous way to find the prime factorization
of a number we simply layout the steps for the algorithm for now.
\begin{enumerate}
  \item Find the prime factorization of $m$
  \item Find the prime factorization of $n$
  \item Identify the common prime factors of $m$ and $n$
  \item Multiply the common prime factors and return it as the GCD
\end{enumerate}

A simple way of generating consecutive prime numbers not exceeding a given integer $n > 1$ is the Sieve of Eratosthenes.
This algorithm starts by initializing a list of prime candidates with consecutive integers from $2 \text{ to } n$. Then
each iteration of the algorithm selects the next number in the list as a prime number and removes all multiples of that
prime number in the list. This continues until no more numbers can be selected as prime numbers i.e. no number can be
removed from the list. The remaining numbers in the list are the prime numbers not exceeding $n$. In pseudocode:

\begin{algorithm}[H]
  \caption{SieveNaïve $ \left( n \right) $}
  \Comment{}\\
  \Comment{Generates a list of prime numbers not exceeding $n >1$} \\
  \Comment{Input: A positive number to not exceed $n$} \\
  \Comment{Output: A list of prime numbers not exceeding $n$} \\
  \begin{algorithmic}[1]
    \Function{ZeroMultiples}{$L, p, n$}
    \For{ $i \gets 1$ to $n$}
    \If{ $L_i \neq 0$ and $p \neq L_i$ and $L_i \mod p  == 0$}
    \State $L_i \gets 0$
    \EndIf
    \EndFor
    \State \Return $L$
    \EndFunction\\

    \State $P[n]$ \Comment{ $P$ is an array of size $n$}

    \For{ $i \gets 2$ to $n$}
    \State $P_i \gets i $
    \EndFor

    \For{ $i \gets 1$ to $n$}
    \If{ $P_i \neq  0$}
    \State prime $ \gets P_i$
    \State $P \gets$ ZeroMultiples $ \left( P, \text{prime}, n \right) $
    \EndIf
    \EndFor \\

    \State $T \left[ n \right] $ \Comment{Temporary array to store prime numbers}
    \State $i \gets 0$
    \For{ $j \gets 1$ to $n$}
    \If{ $P_j \neq 0$}
    \State $T_i \gets P_j$
    \State $i \gets i + 1$
    \EndIf
    \EndFor
    \State \Comment{Remove trailing zeros}

    \State \Return $T$
  \end{algorithmic}
\end{algorithm}

But through observation, we can deduce that the largest number $p$ whose multiples can still remain to warrant further
iterations of the algorithm. We must first note that if $p$ is a number whose multiples are being eliminated on the
current pass, then the first multiple of $p$ we must consider is $p \times p$ because all of its smaller multiples $2p,
  \ldots, \left( p -1 \right) $ have been eliminated on earlier passes. This prevents us from eliminating the same number
more than once. As $p \times p$ should not be greater than $n$, $p$ cannot exceed $\sqrt{n}$ floored i.e. $\left\lfloor
  \sqrt{n}\right\rfloor$. We assume in the following pseudocode that there is function available for computing
$\left\lfloor \sqrt{n} \right\rfloor$.

\begin{algorithm}[H]
  \caption{Sieve of Eratosthenes $ \left( n \right) $}
  \Comment{}\\
  \Comment{Generates a list of prime numbers not exceeding $n > 1$} \\
  \Comment{Input: A positive number to not exceed $n$} \\
  \Comment{Output: A list of prime numbers not exceeding $n$} \\
  \begin{algorithmic}[1]
    \For{ $p \gets 2$ to $n$}
    \State $A \left[ p \right] \gets p $
    \EndFor\\

    \For{ $p \gets 2$ to $\left\lfloor \sqrt{n} \right\rfloor$}
    \If{ $A_p \neq  0$} \Comment{ $p$ hasn't been eliminated on previous passes}
    \State  $j \gets p \times p$
    \While{ $j \leq n$}
    \State $A_j \gets 0$
    \State $j \gets j + p$
    \EndWhile
    \EndIf
    \EndFor \\

    \Comment{Copy remaining elements of $A$ to array $L$ of the primes}
    \State $ i \gets 0$
    \For{ $p \gets 2$ to $n$}

    \If{ $A_p \neq 0$}

    \State $L_i \gets A_p$
    \State $i \gets i + 1$
    \EndIf
    \EndFor

    \State \Return $L$

  \end{algorithmic}
\end{algorithm}

\subsection{Exercises}

\qs{}{
  Design an algorithm for computing $\left\lfloor \sqrt{n} \right\rfloor$ for any positive integer $n$. Besides assignment
  and comparison, your algorithm may only use the four basic
  arithmetical operations.
}

\sol{
  \begin{algorithm}[H]
    \caption{FloorSqrt $\left( n \right)$ }
    \Comment{}\\
    \Comment{Computes $\left\lfloor \sqrt{n} \right\rfloor$ for any positive integer $n$} \\
    \Comment{Input: Integer $n$} \\
    \Comment{Output: The floored square root of $n$} \\
    \begin{algorithmic}[1]
      \Function{abs}{ $x$}
      \If{ $x < 0$}
      \State \Return $-1 \times x$
      \Else
      \State \Return x
      \EndIf
      \EndFunction \\

      \Comment{Newtons Method for calculating square root}
      \State $x \gets 10$
      \While{ $\text{ABS} \left( x \times x - n \right) > 0.001 $}

      \State $x \gets \left( x + \frac{n}{x} \right) / 2 $
      \EndWhile \\


      \State $r \gets x$ \Comment{Approximate definition of the floor function}
      \While{ $r \geq 1$}

      \State $r \gets r - 1$
      \EndWhile

      \State \Return $x - r$

    \end{algorithmic}
  \end{algorithm}

  \begin{algorithm}[H]
    \caption{FloorSqrtBinarySearch $ \left( n \right) $}
    \Comment{}\\
    \Comment{Finds the floored square root of a positive integer $n$ using binary search} \\
    \Comment{Input: A positive integer $n$ } \\
    \Comment{Output: The floored square root of the integer $n$} \\
    \begin{algorithmic}[1] \\
      \State low $ \gets 0$
      \State high $ \gets n$
      \While{high $>$ low}
      \State $\text{mid} \gets \left( \text{high } + \text{ low} \right) $ / 2

      \If{mid $\times$ mid $ \leq n$ and $ \left( \text{mid } + 1 \right) \times \left( \text{mid } + 1 \right) > n  $}
      \State \Return mid
      \EndIf
      \If{mid $\times $ mid $< n$}
      \State low $ \gets \text{ mid } + 1$
      \Else
      \State high $ \gets \text{ mid } - 1$
      \EndIf
      \EndWhile
    \end{algorithmic}
  \end{algorithm}
}


\chapter{Algorithm Design}

For a time function $T \left( n \right) $ if the assumptions of the RAM model cannot be confirmed, then the actual time
function of an algorithm is:
\[
  a \times  T \left( n \right)  \leq T \left( n \right)  \leq b \times T \left( n \right)
\]

\noindent Where $a$ is the least time taken by any primitive operation and $b$ is the most time taken by any primitive operation.

\section{Primitive Operations}

\dfn{Primitive Operation}{
  Any operation that takes a constant amount of time to execute. Usually primitive operations are implemented as machine
  instructions.
}

\noindent Examples of primitive operations include:
\begin{itemize}
  \item Evaluating an expression
  \item Variable Assignment
  \item Array indexing
  \item Comparisons
  \item Arithmetic operations
  \item Function calls
  \item Returning from a function
\end{itemize}

\section{Growth Rate of Running Time}

The growth rate of a function can also described informally as:
\[
  \lim_{n \to \infty} T \left( n \right) \approx O \left( n \right)
\]

\noindent For the formal definitions see \lstinline{2_Algorithm_Analysis.pdf} in the \lstinline{CompSci/Data_Structures} folder.

\ex{}{
  \qs{}{
    Show that $T \left( n \right) = 5n + 3 $ is $O \left( n \right) $ using the big-$O$ notation
  }

  \sol{
    Let $f \left( n \right) = 5n + 3 $ and $g \left( n \right) = n $. $O$ is defined as for a given function $ g\left( n
      \right) $:
    \[
      O \left( g \left( n \right)  \right) = \left\{ f \left( n \right) : \exists c > 0, n_0 \in \mathbb{N} \text{ such
        that } 0 \leq f \left( n \right) \leq c \cdot g \left( n \right) \, \forall n \geq n_0 \right\}
    \]
    $\therefore$
    \[
      5n + 3 \leq cn \, \forall n \geq n_0
    \]
    \begin{align*}
      5n \leq cn                                                \\
      c = 5, 6, 7, \ldots                                       \\
      \text{ Since } 5n + 3 \leq 5n = F \text{ we chose } c = 6 \\
      5 n + 3 \leq 6n                                           \\
      3 \leq 6n - 5n                                            \\
      3 \leq n                                                  \\
      \therefore c = 6 \, n_0 = 3                               \\
    \end{align*}
    $\therefore$ $T \left( n \right) = 5n + 3 $ is $O \left( n \right) $.
  }
}

\ex{}{
  \qs{}{
    Show that $T \left( n \right) = 5n^2 + 3n - 7 $ is not $O \left( n \right) $
  }

  \sol{
    Let $f \left( n \right) = T \left( n \right)  $ and $g \left( n \right) = n $. From the definition of $O \left( n
      \right) $:
    \[
      O \left( g \left( n \right)  \right) = f \left( n \right) \leq c g \left( n \right) \, \forall n \geq n_0
    \]
    \begin{align*}
      5n^2 + 3n - 7 & \leq cn \\
      5n^2 + 3n     & \leq cn \\
      5n + 3        & \leq c  \\
    \end{align*}
    As $c$ is a constant but is dependent on $n$ there is no $c$ that satisfies the inequality \\
    $\therefore$ $T \left( n \right) $ is not $O \left( n \right) $

  }
}

\end{document}
