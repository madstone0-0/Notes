\documentclass[12pt letter]{report}
\input{./template/preamble}
\input{./template/macros}
\input{./template/letterfonts}

\title{\Huge{Boolean Algebra}}
\author{\huge{Madiba Hudson-Quansah}}
\date{February 2024}
\usepackage{parskip}

\usepackage{proof}
\setcounter{tocdepth}{4}
\setcounter{secnumdepth}{4}

\begin{document}
\maketitle
\newpage
\pdfbookmark[section]{\contentsname}{too}
\tableofcontents
\pagebreak

\chapter{Boolean Functions}

\section{Introduction}

Boolean Algebra provides the operations and rules for working with the set $\{0,1\} $. The three operations that will be
discussed are the:
\begin{itemize}
	\item Boolean sum (\textbf{OR}) - $0 + 1 = 1$
	\item Boolean product (\textbf{AND}) - $0 \cdot 1 = 0$
	\item Complementation (\textbf{NOT}) - $\overline{0} = 1$
\end{itemize}

\subsection{Boolean Product (AND)}

\dfn{Boolean Product}{
	The Boolean product of two variables $x$ and $y$ is denoted by $x \cdot  y$ and is defined by the following values:
	\begin{align*}
		1 \cdot 1 = 1 \\
		1 \cdot 0 = 0 \\
		0 \cdot 1 = 0 \\
		0 \cdot 0 = 0 \\
	\end{align*}
}

\subsection{Boolean Sum (OR)}

\dfn{Boolean Sum}{
	The Boolean sum of two variables $x$ and $y$ is denoted by $x + y$ and is defined by the following values:
	\begin{align*}
		1 + 1 = 1 \\
		1 + 0 = 1 \\
		0 + 1 = 1 \\
		0 + 0 = 0 \\
	\end{align*}
}

\subsection{Complementation (NOT)}

\dfn{Complementation}{
	The complement of a variable $x$ is denoted by $\overline{x}$ and is defined by the following values:
	\begin{align*}
		\overline{1} = 0 \\
		\overline{0} = 1 \\
	\end{align*}
}

\ex{}{
	\qs{}{
		Find the value of $1 \cdot 0 + \overline{\left( 0 + 1 \right) }$
	}

	\sol{
		\begin{align*}
			1 \cdot 0 + \overline{\left( 0 + 1 \right) } & = 1 \cdot 0 + \overline{1} \\
			                                             & = 0 + \overline{1}         \\
			                                             & = 0 + 0                    \\
			                                             & = 0                        \\
		\end{align*}
	}
}

\ex{}{
	\qs{}{
		Translate $1 \cdot 0 + \overline{\left( 0 + 1 \right) } = 0$, into a logical equivalence.
	}

	\sol{
		\[
			T \wedge F \vee \neg \left( F \vee  T \right) \equiv F
		\]
	}
}

\section{Boolean Expressions and Functions}

Let $B = \{0,1\} $, then $B^{n} = \{\left( x_1,x_2,\ldots,x_n \right)  \mid x_i \in B \text{ for } 1 \leq i \leq n \} $
is the set of all possible $n$-tuples of 0's and 1's. The variable $x$ is a \textit{Boolean variable}.

\dfn{Boolean variable}{
	A variable that can take on the values 0 or 1.
}

\dfn{Boolean Function}{
A function $f:B^{n} \to B$ is called a \textit{Boolean function} of degree $n$. I.e. takes $n$ inputs and returns a single output.

\ex{}{
	The function $F \left( x, y \right) = x $ from the set of ordered pairs of Boolean variables to the set $\{0,1\} $,
	has a degree of 2.
}
}

\subsection{Complement of a Boolean function}

\dfn{Complement of a Boolean function}{
	The complement of a Boolean function $F$ is denoted by $\overline{F}$ and is defined by:

	\[
		\overline{F} \left( x_1,x_2,\ldots,x_n \right) = \overline{f \left( x_1,x_2,\ldots,x_n \right)}
	\]

}

\section{Boolean Identities}

\subsection{Law of Double Complement}

\[
	\overline{\overline{x}} = x
\]

\subsection{Idempotent Laws}

\begin{align*}
	x + x = x \\
	x \cdot x = x
\end{align*}

\subsection{Identity Laws}

\begin{align*}
	x + 0 = x     \\
	x \cdot 1 = x \\
\end{align*}

\subsection{Domination Laws}

\begin{align*}
	x + 1 = 1 \\
	x \cdot 0 = 0
\end{align*}

\subsection{Commutative Laws}

\begin{align*}
	x + y = y + x \\
	xy = yx
\end{align*}

\subsection{Associative Laws}

\begin{align*}
	x + \left( y + z \right)  = \left( x + y \right)  + z \\
	x \left( yz \right)  = \left( xy \right) z
\end{align*}

\subsection{Distributive Laws}

\begin{align*}
	x + yz = \left( x + y \right) \left( x + z \right) \\
	x \left( y + z \right) = xy + xz
\end{align*}

\subsection{De Morgan's Laws}

\begin{align*}
	\overline{\left( xy \right) } = \overline{x} + \overline{y} \\
	\overline{\left( x + y \right) } = \overline{x} \cdot \overline{y}
\end{align*}

\subsection{Absorption Laws}
\begin{align*}
	x + xy = x \\
	x \left( x + y \right)  = x
\end{align*}

\subsection{Unit Property}

\[
	x + \overline{x} = 1
\]

\subsection{Zero Property}
\[
	x \overline{x} = 0
\]

\section{Duality}

\dfn{Dual}{
	The dual of a Boolean expression is obtained by replacing the \textbf{AND} operation with \textbf{OR} and the
	\textbf{OR} operation with \textbf{AND}, and interchanging 1s and 0s.

}

\ex{}{
	\qs{}{
		Find the duals of $x \left( y + 0 \right) $ and $\overline{x} \cdot 1 + \left( \overline{y} + z \right) $
	}

	\sol{
		\begin{align*}
			x \left( y + 0 \right)                                 & = x + \left( y \cdot 1  \right)                                       \\
			\\
			\overline{x} \cdot 1 + \left( \overline{y} + z \right) & = \left( \overline{x} + 0 \right)  \cdot \left( \overline{y}z \right)
		\end{align*}
	}
}

The dual of a boolean function $F$ is the function representing the dual of the expression representing $F$, denoted by
$F^{d}$

\dfn{Duality Principle}{
	An identity between functions represented by boolean expressions remains valid when the duals of both sides of the
	expression are taken.
}

\ex{}{
	\qs{}{
		Construct an identity from the absorption law $x \left( x + y \right) = x $ by taking duals
	}

	\sol{
		\begin{align*}
			x \left( x + y \right)  = x                                                                      \\
			\text{Let } F \left( x,y \right)  = x \left( x + y \right)  \text{ and } G \left( x \right)  = x \\
			\\
			F \left( x,y \right)  = G \left( x \right)                                                       \\
			F^{d} \left( x,y \right)  = G^{d} \left( x \right)                                               \\
			F^{d} \left( x, y \right)  = x + xy                                                              \\
			G^{d} \left( x \right) = x                                                                       \\
			\\
			x + xy = x                                                                                       \\
		\end{align*}
	}
}

\section{Exercises}

\chapter{Representing Boolean Functions}

\section{Sum of Products Expansion}

\dfn{Literal}{
	A variable or its complement.
}

\dfn{Minterm}{
	A product of literals in which each variable appears exactly once. I.e. the minterm of boolean variables
	$x_1,x_2,\ldots,x_n$ is a boolean product $y_1y_2\ldots y_n$, where
	\[
		y_i = x_i \text{ or } y_i = \overline{x_i}
	\]
	I.e. $y_1y_2\ldots y_n$ is a minterm in of the variables $x_1,x_2,\ldots,x_n$
}


\subsubsection{Sum of Products / Disjunctive normal form (DNF)}

Form a product (using logical and) term for each row in the truth table where the function is 1. Then sum (using logical or) all the terms together.

\begin{table}[h!]
	\begin{center}
		\begin{tabular}{|c c c|c|c|}
			\hline
			$x$ & $y$ & $z$ & $F \left( x, y, z \right) $ & $G \left( x,y,z \right) $ \\ [0.5ex]
			\hline
			\hline
			1   & 1   & 1   & 0                           & 0                         \\
			1   & 1   & 0   & 0                           & 1                         \\
			1   & 0   & 1   & 1                           & 0                         \\
			1   & 0   & 0   & 0                           & 0                         \\
			0   & 1   & 1   & 0                           & 0                         \\
			0   & 1   & 0   & 0                           & 1                         \\
			0   & 0   & 1   & 0                           & 0                         \\
			0   & 0   & 0   & 0                           & 0                         \\
			\hline
		\end{tabular}
	\end{center}
\end{table}



\ex{}{
	\qs{}{
		Find Boolean expressions that represent the functions, using the truth table above.
		\begin{enumerate}
			\item $F \left( x,y,z \right) $
			\item $G \left( x,y,z \right) $
		\end{enumerate}


	}

	\sol{
		\begin{enumerate}
			\item First we look for the rows where $F$ is 1. There is only one row, row 3. Then we determine the minterm for
			      this row which is $x\overline{y}z$. Then we boolean sum all the found minterms to derive the
			      function's boolean expression but since there is only one minterm the result is simply
			      \[
				      F \left( x,y,z \right) = x\overline{y}z
			      \]
			\item We repeat the same process for the function $G$, and as there are two rows where $G$ is 1 we will
			      have two minterms, $xy\overline{z}$ and $\overline{x}y\overline{z}$, making the boolean expression
			      expression
			      \[
				      G \left( x, y,z \right) = xy\overline{z} + \overline{x}y\overline{z}
			      \]
		\end{enumerate}
	}

}


\ex{}{
	\qs{}{
		Find the sum-of-products of the expansion for the function $F \left( x,y,z \right) = \left( x + y \right)
			\overline{z}  $
	}

	\sol{
		\begin{align*}
			F \left( x,y,z \right) & = \left( x + y \right)\overline{z}                                                                                                  \\
			                       & = x\overline{z} + y \overline{x} \tag*{By Second Distributive Law}                                                                  \\
			                       & = x1\overline{z} + y1\overline{z} \tag*{By Second Identity Law}                                                                     \\
			                       & = x \left( y + \overline{y} \right) \overline{z} + y \left( x + \overline{x} \right) \overline{z} \tag*{By First Unit Property Law} \\
			                       & = xy\overline{z} + x\overline{y}\overline{z} + xy\overline{z} + \overline{x}y\overline{z} \tag*{By Second
			Distributive Law}                                                                                                                                            \\
			                       & =  xy\overline{z} + xy\overline{z} + x\overline{y}\overline{z} +
			\overline{x}y\overline{z} \tag*{By Second Commutative Law}                                                                                                   \\
			                       & = xy\overline{z} + x\overline{y}\overline{z} + \overline{x}y\overline{z} \tag*{By First Idempotent Law}                             \\
		\end{align*}
		$\therefore$ $F \left( x,y,z \right) = xy\overline{z} + x\overline{y}\overline{z}+ \overline{x}y\overline{z} $
	}
}



\subsection{Product of Sums Expansion / Conductive Normal Form (CNF)}

A product of sums expansion is the dual of a sum of product expansion.

\ex{}{
	$F \left( x,y,z \right) = xy\overline{z} + x\overline{y}\overline{z} + \overline{x}y\overline{z} $ can be expressed as a product of sums expansion
	\[
		F \left( x,y,z \right) =\left( x + y + \overline{z} \right)  \cdot \left( x + \overline{y} + \overline{z} \right)
		\cdot \left( \overline{x} + y + \overline{z} \right)
	\]
}

\section{Exercises}


\qs{}{
	Use truth tables to prove the domination laws for Boolean.
}

\sol{
	\begin{table}[h!]
		\begin{center}
			\begin{tabular}{|c|c|c|c|c|}
				\hline
				$x$ & 1 & 0 & $x + 1$ & $x \cdot 0$ \\ [0.5ex]
				\hline
				\hline
				1   & 1 & 0 & 1       & 0           \\
				0   & 1 & 0 & 1       & 0           \\
				\hline
			\end{tabular}
		\end{center}
	\end{table}
	Conclusion: $x + 1 = 1$ from column 2 and 4 and $x \cdot 0 = 0$ from column 3 and 5.
}

\qs{}{
	The Boolean operator $\oplus$, called \textbf{XOR} is defined by $1 \oplus 1 = 0$, $1 \oplus 0 = 1$
	\begin{enumerate}
		\item $x \oplus x$
		\item $x \oplus \overline{x}$
	\end{enumerate}
}

\sol{
	\begin{enumerate}
		\item
		      \begin{align*}
			      x \oplus x        \\
			      \\
			      \text{When } x= 1 \\
			      1 \oplus 1 = 0    \\
			      \\
			      \text{When } x= 0 \\
			      0 \oplus 0 = 0    \\
			      \\
			      x \oplus x = 0
		      \end{align*}
		\item
		      \begin{align*}
			      x \oplus \overline{x} \\
			      \\
			      \text{When } x= 1     \\
			      1 \oplus \overline{1} \\
			      1 \oplus 0 = 1        \\
			      \\
			      \text{When } x = 0    \\
			      0 \oplus \overline{0} \\
			      0 \oplus 1            \\
			      0 \oplus 1 = 1        \\
			      \\
			      x \oplus \overline{x} = 1
		      \end{align*}
	\end{enumerate}
}

\qs{}{
	Prove the absorption law $x + xy = x$ using the other boolean identities
}

\sol{
	\begin{align*}
		x + xy & = x \cdot 1 + xy \tag*{By Second Identity Law}             \\
		       & = x \left( 1 + y \right) \tag*{By Second Distributive Law} \\
		       & = x \cdot 1 \tag*{By First Domination Law}                 \\
		       & = x \tag*{By Second Identity Law}                          \\
	\end{align*}

	\begin{align*}
		x \left( x + y \right) & = \left( x + 0 \right) \left( x + y \right) \tag*{By First Identity Law} \\
		                       & = x + 0 \cdot y \tag*{By First Distributive Law }                        \\
		                       & = x + 0 \tag*{By Second Domination Law}                                  \\
		                       & = x \tag*{By First Identity Law}                                         \\
	\end{align*}
}

\qs{}{
	Find the sum of products expansion of these Boolean functions
	\begin{enumerate}
		\item $F \left( x, y \right) = x + y $
		\item $F \left( x, y \right) = xy $
		\item $F \left( x,y \right) = 1 $
		\item $F \left( x,y \right) = y $
	\end{enumerate}
}

\sol{
	\begin{enumerate}
		\item
		\item
		\item
		      \begin{align*}
			      F \left( x, y \right) & = 1                                                                       \\
			                            & = x + \overline{x} \tag*{By Unit Property}                                \\
			                            & = x \cdot 1 + \overline{x} \cdot 1 \tag*{By Second Identity Law}          \\
			                            & = x \cdot \left( y + \overline{y} \right) + \overline{x} \cdot \left( y +
			      \overline{y} \right) \tag*{By Unit Property}                                                      \\
			                            & =  xy + x\overline{y} + \overline{x}y + \overline{x}\overline{y} \tag*{By
			      Second Distributive Law}                                                                          \\
		      \end{align*}
		\item
		      \begin{align*}
			      F \left( x, y \right) & = y                                                                                          \\
			                            & = y + y \tag*{By First Idempotent Law}                                                       \\
			                            & = y\cdot 1 + y \cdot 1 \tag*{By Second Identity Law}                                         \\
			                            & = y \cdot \left( x + 1 \right) + y\cdot  \left( x + 1 \right) \tag*{By First Domination Law} \\
			                            & = xy + y + xy + y \tag*{By Second Distributive Law}                                          \\
			                            & = xy + xy + y + y \tag*{By First Commutative Law}                                            \\
			                            & = xy + y \tag*{By First Idempotent Law}                                                      \\
			                            & =  xy + y \cdot 1 \tag*{By Second Identity Law}                                              \\
		      \end{align*}
	\end{enumerate}
}


\end{document}
