\documentclass[12pt letter]{report}
%%%%%%%%%%%%%%%%%%%%%%%%%%%%%%%%%
% PACKAGE IMPORTS
%%%%%%%%%%%%%%%%%%%%%%%%%%%%%%%%%


\usepackage[tmargin=2cm,rmargin=1in,lmargin=1in,margin=0.85in,bmargin=2cm,footskip=.2in]{geometry}
\usepackage{amsmath,amsfonts,amsthm,amssymb,mathtools}
\usepackage[varbb]{newpxmath}
\usepackage{xfrac}
\usepackage[makeroom]{cancel}
\usepackage{mathtools}
\usepackage{bookmark}
\usepackage{enumitem}
\usepackage{hyperref,theoremref}
\hypersetup{
  pdftitle={Assignment},
  colorlinks=true, linkcolor=doc!90,
  bookmarksnumbered=true,
  bookmarksopen=true
}
\usepackage[most,many,breakable]{tcolorbox}
\usepackage{xcolor}
\usepackage{varwidth}
\usepackage{varwidth}
\usepackage{etoolbox}
%\usepackage{authblk}
\usepackage{nameref}
\usepackage{multicol,array}
\usepackage{tikz-cd}
\usepackage[ruled,vlined,linesnumbered]{algorithm2e}
\usepackage{comment} % enables the use of multi-line comments (\ifx \fi) 
\usepackage{import}
\usepackage{xifthen}
\usepackage{pdfpages}
\usepackage{transparent}
\usepackage{xcolor,colortbl,array,amssymb}
\usepackage{venndiagram}
\usepackage{listings}
\usepackage{fontspec}

\setmainfont{LibertinusSerif}[
  Extension = .otf,
  Path = /usr/share/fonts/libertinus/,
  UprightFont = *-Regular,
  ItalicFont = *-Italic,
  BoldFont = *-Bold,
  BoldItalicFont = *-BoldItalic,
]

\setmonofont{CaskaydiaCoveNerdFontMono}[
  Extension = .ttf,
  Path = /usr/share/fonts/TTF/,
  UprightFont = *-Regular,
  ItalicFont = *-Italic,
  BoldFont = *-Bold,
  BoldItalicFont = *-BoldItalic,
]

\definecolor{listing-background}{HTML}{F7F7F7}
\definecolor{listing-rule}{HTML}{B3B2B3}
\definecolor{listing-numbers}{HTML}{B3B2B3}
\definecolor{listing-text-color}{HTML}{000000}
\definecolor{listing-keyword}{HTML}{435489}
\definecolor{listing-keyword-2}{HTML}{1284CA} % additional keywords
\definecolor{listing-keyword-3}{HTML}{9137CB} % additional keywords
\definecolor{listing-identifier}{HTML}{435489}
\definecolor{listing-string}{HTML}{00999A}
\definecolor{listing-comment}{HTML}{8E8E8E}

\lstdefinestyle{eisvogel_listing_style}{
language         = java,
xleftmargin      = 0.6em,
framexleftmargin = 0.4em,
backgroundcolor  = \color{listing-background},
basicstyle       = \color{listing-text-color}\linespread{1.0}%
\lst@ifdisplaystyle%
\fi\ttfamily{},
breaklines       = true,
frame            = single,
framesep         = 0.19em,
rulecolor        = \color{listing-rule},
frameround       = ffff,
tabsize          = 4,
numberstyle      = \color{listing-numbers},
aboveskip        = 1.0em,
belowskip        = 0.1em,
abovecaptionskip = 0em,
belowcaptionskip = 1.0em,
keywordstyle     = {\color{listing-keyword}\bfseries},
keywordstyle     = {[2]\color{listing-keyword-2}\bfseries},
keywordstyle     = {[3]\color{listing-keyword-3}\bfseries\itshape},
sensitive        = true,
identifierstyle  = \color{listing-identifier},
commentstyle     = \color{listing-comment},
stringstyle      = \color{listing-string},
showstringspaces = false,
escapeinside     = {/*@}{@*/}, % Allow LaTeX inside these special comments
literate         =
  {á}{{\'a}}1 {é}{{\'e}}1 {í}{{\'i}}1 {ó}{{\'o}}1 {ú}{{\'u}}1
{Á}{{\'A}}1 {É}{{\'E}}1 {Í}{{\'I}}1 {Ó}{{\'O}}1 {Ú}{{\'U}}1
{à}{{\`a}}1 {è}{{\`e}}1 {ì}{{\`i}}1 {ò}{{\`o}}1 {ù}{{\`u}}1
{À}{{\`A}}1 {È}{{\`E}}1 {Ì}{{\`I}}1 {Ò}{{\`O}}1 {Ù}{{\`U}}1
{ä}{{\"a}}1 {ë}{{\"e}}1 {ï}{{\"i}}1 {ö}{{\"o}}1 {ü}{{\"u}}1
{Ä}{{\"A}}1 {Ë}{{\"E}}1 {Ï}{{\"I}}1 {Ö}{{\"O}}1 {Ü}{{\"U}}1
{â}{{\^a}}1 {ê}{{\^e}}1 {î}{{\^i}}1 {ô}{{\^o}}1 {û}{{\^u}}1
{Â}{{\^A}}1 {Ê}{{\^E}}1 {Î}{{\^I}}1 {Ô}{{\^O}}1 {Û}{{\^U}}1
{œ}{{\oe}}1 {Œ}{{\OE}}1 {æ}{{\ae}}1 {Æ}{{\AE}}1 {ß}{{\ss}}1
{ç}{{\c c}}1 {Ç}{{\c C}}1 {ø}{{\o}}1 {å}{{\r a}}1 {Å}{{\r A}}1
{€}{{\EUR}}1 {£}{{\pounds}}1 {«}{{\guillemotleft}}1
{»}{{\guillemotright}}1 {ñ}{{\~n}}1 {Ñ}{{\~N}}1 {¿}{{?`}}1
{…}{{\ldots}}1 {≥}{{>=}}1 {≤}{{<=}}1 {„}{{\glqq}}1 {“}{{\grqq}}1
{”}{{''}}1
}
\lstset{style=eisvogel_listing_style,
  numbers=left}

%
% Java (Java SE 12, 2019-06-22)
%
\lstdefinelanguage{Java}{
  morekeywords={
      % normal keywords (without data types)
      abstract,assert,break,case,catch,class,continue,default,
      do,else,enum,exports,extends,final,finally,for,if,implements,
      import,instanceof,interface,module,native,new,package,private,
      protected,public,requires,return,static,strictfp,super,switch,
      synchronized,this,throw,throws,transient,try,volatile,while,
      % var is an identifier
      var
    },
  morekeywords={[2] % data types
      % primitive data types
      boolean,byte,char,double,float,int,long,short,
      % String
      String,
      % primitive wrapper types
      Boolean,Byte,Character,Double,Float,Integer,Long,Short
      % number types
      Number,AtomicInteger,AtomicLong,BigDecimal,BigInteger,DoubleAccumulator,DoubleAdder,LongAccumulator,LongAdder,Short,
      % other
      Object,Void,void
    },
  morekeywords={[3] % literals
      % reserved words for literal values
      null,true,false,
    },
  sensitive,
  morecomment  = [l]//,
  morecomment  = [s]{/*}{*/},
  morecomment  = [s]{/**}{*/},
  morestring   = [b]",
  morestring   = [b]',
}

\lstdefinelanguage{XML}{
  morestring      = [b]",
  moredelim       = [s][\bfseries\color{listing-keyword}]{<}{\ },
  moredelim       = [s][\bfseries\color{listing-keyword}]{</}{>},
  moredelim       = [l][\bfseries\color{listing-keyword}]{/>},
  moredelim       = [l][\bfseries\color{listing-keyword}]{>},
  morecomment     = [s]{<?}{?>},
  morecomment     = [s]{<!--}{-->},
  commentstyle    = \color{listing-comment},
  stringstyle     = \color{listing-string},
  identifierstyle = \color{listing-identifier}
}


\newcommand\mycommfont[1]{\footnotesize\ttfamily\textcolor{blue}{#1}}
\SetCommentSty{mycommfont}
\newcommand{\incfig}[1]{%
  \def\svgwidth{\columnwidth}
  \import{./figures/}{#1.pdf_tex}
}

\usepackage{tikzsymbols}
\renewcommand\qedsymbol{$\Laughey$}


%\usepackage{import}
%\usepackage{xifthen}
%\usepackage{pdfpages}
%\usepackage{transparent}


%%%%%%%%%%%%%%%%%%%%%%%%%%%%%%
% SELF MADE COLORS
%%%%%%%%%%%%%%%%%%%%%%%%%%%%%%



\definecolor{myg}{RGB}{56, 140, 70}
\definecolor{myb}{RGB}{45, 111, 177}
\definecolor{myr}{RGB}{199, 68, 64}
\definecolor{mytheorembg}{HTML}{F2F2F9}
\definecolor{mytheoremfr}{HTML}{00007B}
\definecolor{mylenmabg}{HTML}{FFFAF8}
\definecolor{mylenmafr}{HTML}{983b0f}
\definecolor{mypropbg}{HTML}{f2fbfc}
\definecolor{mypropfr}{HTML}{191971}
\definecolor{myexamplebg}{HTML}{F2FBF8}
\definecolor{myexamplefr}{HTML}{88D6D1}
\definecolor{myexampleti}{HTML}{2A7F7F}
\definecolor{mydefinitbg}{HTML}{E5E5FF}
\definecolor{mydefinitfr}{HTML}{3F3FA3}
\definecolor{notesgreen}{RGB}{0,162,0}
\definecolor{myp}{RGB}{197, 92, 212}
\definecolor{mygr}{HTML}{2C3338}
\definecolor{myred}{RGB}{127,0,0}
\definecolor{myyellow}{RGB}{169,121,69}
\definecolor{myexercisebg}{HTML}{F2FBF8}
\definecolor{myexercisefg}{HTML}{88D6D1}


%%%%%%%%%%%%%%%%%%%%%%%%%%%%
% TCOLORBOX SETUPS
%%%%%%%%%%%%%%%%%%%%%%%%%%%%

\setlength{\parindent}{1cm}
%================================
% THEOREM BOX
%================================

\tcbuselibrary{theorems,skins,hooks}
\newtcbtheorem[number within=section]{Theorem}{Theorem}
{%
  enhanced,
  breakable,
  colback = mytheorembg,
  frame hidden,
  boxrule = 0sp,
  borderline west = {2pt}{0pt}{mytheoremfr},
  sharp corners,
  detach title,
  before upper = \tcbtitle\par\smallskip,
  coltitle = mytheoremfr,
  fonttitle = \bfseries\sffamily,
  description font = \mdseries,
  separator sign none,
  segmentation style={solid, mytheoremfr},
}
{th}

\tcbuselibrary{theorems,skins,hooks}
\newtcbtheorem[number within=chapter]{theorem}{Theorem}
{%
  enhanced,
  breakable,
  colback = mytheorembg,
  frame hidden,
  boxrule = 0sp,
  borderline west = {2pt}{0pt}{mytheoremfr},
  sharp corners,
  detach title,
  before upper = \tcbtitle\par\smallskip,
  coltitle = mytheoremfr,
  fonttitle = \bfseries\sffamily,
  description font = \mdseries,
  separator sign none,
  segmentation style={solid, mytheoremfr},
}
{th}


\tcbuselibrary{theorems,skins,hooks}
\newtcolorbox{Theoremcon}
{%
  enhanced
  ,breakable
  ,colback = mytheorembg
  ,frame hidden
  ,boxrule = 0sp
  ,borderline west = {2pt}{0pt}{mytheoremfr}
  ,sharp corners
  ,description font = \mdseries
  ,separator sign none
}

%================================
% Corollery
%================================
\tcbuselibrary{theorems,skins,hooks}
\newtcbtheorem[number within=section]{Corollary}{Corollary}
{%
  enhanced
  ,breakable
  ,colback = myp!10
  ,frame hidden
  ,boxrule = 0sp
  ,borderline west = {2pt}{0pt}{myp!85!black}
  ,sharp corners
  ,detach title
  ,before upper = \tcbtitle\par\smallskip
  ,coltitle = myp!85!black
  ,fonttitle = \bfseries\sffamily
  ,description font = \mdseries
  ,separator sign none
  ,segmentation style={solid, myp!85!black}
}
{th}
\tcbuselibrary{theorems,skins,hooks}
\newtcbtheorem[number within=chapter]{corollary}{Corollary}
{%
  enhanced
  ,breakable
  ,colback = myp!10
  ,frame hidden
  ,boxrule = 0sp
  ,borderline west = {2pt}{0pt}{myp!85!black}
  ,sharp corners
  ,detach title
  ,before upper = \tcbtitle\par\smallskip
  ,coltitle = myp!85!black
  ,fonttitle = \bfseries\sffamily
  ,description font = \mdseries
  ,separator sign none
  ,segmentation style={solid, myp!85!black}
}
{th}


%================================
% LENMA
%================================

\tcbuselibrary{theorems,skins,hooks}
\newtcbtheorem[number within=section]{Lenma}{Lenma}
{%
  enhanced,
  breakable,
  colback = mylenmabg,
  frame hidden,
  boxrule = 0sp,
  borderline west = {2pt}{0pt}{mylenmafr},
  sharp corners,
  detach title,
  before upper = \tcbtitle\par\smallskip,
  coltitle = mylenmafr,
  fonttitle = \bfseries\sffamily,
  description font = \mdseries,
  separator sign none,
  segmentation style={solid, mylenmafr},
}
{th}

\tcbuselibrary{theorems,skins,hooks}
\newtcbtheorem[number within=chapter]{lenma}{Lenma}
{%
  enhanced,
  breakable,
  colback = mylenmabg,
  frame hidden,
  boxrule = 0sp,
  borderline west = {2pt}{0pt}{mylenmafr},
  sharp corners,
  detach title,
  before upper = \tcbtitle\par\smallskip,
  coltitle = mylenmafr,
  fonttitle = \bfseries\sffamily,
  description font = \mdseries,
  separator sign none,
  segmentation style={solid, mylenmafr},
}
{th}


%================================
% PROPOSITION
%================================

\tcbuselibrary{theorems,skins,hooks}
\newtcbtheorem[number within=section]{Prop}{Proposition}
{%
  enhanced,
  breakable,
  colback = mypropbg,
  frame hidden,
  boxrule = 0sp,
  borderline west = {2pt}{0pt}{mypropfr},
  sharp corners,
  detach title,
  before upper = \tcbtitle\par\smallskip,
  coltitle = mypropfr,
  fonttitle = \bfseries\sffamily,
  description font = \mdseries,
  separator sign none,
  segmentation style={solid, mypropfr},
}
{th}

\tcbuselibrary{theorems,skins,hooks}
\newtcbtheorem[number within=chapter]{prop}{Proposition}
{%
  enhanced,
  breakable,
  colback = mypropbg,
  frame hidden,
  boxrule = 0sp,
  borderline west = {2pt}{0pt}{mypropfr},
  sharp corners,
  detach title,
  before upper = \tcbtitle\par\smallskip,
  coltitle = mypropfr,
  fonttitle = \bfseries\sffamily,
  description font = \mdseries,
  separator sign none,
  segmentation style={solid, mypropfr},
}
{th}


%================================
% CLAIM
%================================

\tcbuselibrary{theorems,skins,hooks}
\newtcbtheorem[number within=section]{claim}{Claim}
{%
  enhanced
  ,breakable
  ,colback = myg!10
  ,frame hidden
  ,boxrule = 0sp
  ,borderline west = {2pt}{0pt}{myg}
  ,sharp corners
  ,detach title
  ,before upper = \tcbtitle\par\smallskip
  ,coltitle = myg!85!black
  ,fonttitle = \bfseries\sffamily
  ,description font = \mdseries
  ,separator sign none
  ,segmentation style={solid, myg!85!black}
}
{th}



%================================
% Exercise
%================================

\tcbuselibrary{theorems,skins,hooks}
\newtcbtheorem[number within=section]{Exercise}{Exercise}
{%
  enhanced,
  breakable,
  colback = myexercisebg,
  frame hidden,
  boxrule = 0sp,
  borderline west = {2pt}{0pt}{myexercisefg},
  sharp corners,
  detach title,
  before upper = \tcbtitle\par\smallskip,
  coltitle = myexercisefg,
  fonttitle = \bfseries\sffamily,
  description font = \mdseries,
  separator sign none,
  segmentation style={solid, myexercisefg},
}
{th}

\tcbuselibrary{theorems,skins,hooks}
\newtcbtheorem[number within=chapter]{exercise}{Exercise}
{%
  enhanced,
  breakable,
  colback = myexercisebg,
  frame hidden,
  boxrule = 0sp,
  borderline west = {2pt}{0pt}{myexercisefg},
  sharp corners,
  detach title,
  before upper = \tcbtitle\par\smallskip,
  coltitle = myexercisefg,
  fonttitle = \bfseries\sffamily,
  description font = \mdseries,
  separator sign none,
  segmentation style={solid, myexercisefg},
}
{th}

%================================
% EXAMPLE BOX
%================================

\newtcbtheorem[number within=section]{Example}{Example}
{%
  colback = myexamplebg
  ,breakable
  ,colframe = myexamplefr
  ,coltitle = myexampleti
  ,boxrule = 1pt
  ,sharp corners
  ,detach title
  ,before upper=\tcbtitle\par\smallskip
  ,fonttitle = \bfseries
  ,description font = \mdseries
  ,separator sign none
  ,description delimiters parenthesis
}
{ex}

\newtcbtheorem[number within=chapter]{example}{Example}
{%
  colback = myexamplebg
  ,breakable
  ,colframe = myexamplefr
  ,coltitle = myexampleti
  ,boxrule = 1pt
  ,sharp corners
  ,detach title
  ,before upper=\tcbtitle\par\smallskip
  ,fonttitle = \bfseries
  ,description font = \mdseries
  ,separator sign none
  ,description delimiters parenthesis
}
{ex}

%================================
% DEFINITION BOX
%================================

\newtcbtheorem[number within=section]{Definition}{Definition}{enhanced,
  before skip=2mm,after skip=2mm, colback=red!5,colframe=red!80!black,boxrule=0.5mm,
  attach boxed title to top left={xshift=1cm,yshift*=1mm-\tcboxedtitleheight}, varwidth boxed title*=-3cm,
  boxed title style={frame code={
          \path[fill=tcbcolback]
          ([yshift=-1mm,xshift=-1mm]frame.north west)
          arc[start angle=0,end angle=180,radius=1mm]
          ([yshift=-1mm,xshift=1mm]frame.north east)
          arc[start angle=180,end angle=0,radius=1mm];
          \path[left color=tcbcolback!60!black,right color=tcbcolback!60!black,
            middle color=tcbcolback!80!black]
          ([xshift=-2mm]frame.north west) -- ([xshift=2mm]frame.north east)
          [rounded corners=1mm]-- ([xshift=1mm,yshift=-1mm]frame.north east)
          -- (frame.south east) -- (frame.south west)
          -- ([xshift=-1mm,yshift=-1mm]frame.north west)
          [sharp corners]-- cycle;
        },interior engine=empty,
    },
  fonttitle=\bfseries,
  title={#2},#1}{def}
\newtcbtheorem[number within=chapter]{definition}{Definition}{enhanced,
  before skip=2mm,after skip=2mm, colback=red!5,colframe=red!80!black,boxrule=0.5mm,
  attach boxed title to top left={xshift=1cm,yshift*=1mm-\tcboxedtitleheight}, varwidth boxed title*=-3cm,
  boxed title style={frame code={
          \path[fill=tcbcolback]
          ([yshift=-1mm,xshift=-1mm]frame.north west)
          arc[start angle=0,end angle=180,radius=1mm]
          ([yshift=-1mm,xshift=1mm]frame.north east)
          arc[start angle=180,end angle=0,radius=1mm];
          \path[left color=tcbcolback!60!black,right color=tcbcolback!60!black,
            middle color=tcbcolback!80!black]
          ([xshift=-2mm]frame.north west) -- ([xshift=2mm]frame.north east)
          [rounded corners=1mm]-- ([xshift=1mm,yshift=-1mm]frame.north east)
          -- (frame.south east) -- (frame.south west)
          -- ([xshift=-1mm,yshift=-1mm]frame.north west)
          [sharp corners]-- cycle;
        },interior engine=empty,
    },
  fonttitle=\bfseries,
  title={#2},#1}{def}



%================================
% Solution BOX
%================================

\makeatletter
\newtcbtheorem{question}{Question}{enhanced,
  breakable,
  colback=white,
  colframe=myb!80!black,
  attach boxed title to top left={yshift*=-\tcboxedtitleheight},
  fonttitle=\bfseries,
  title={#2},
  boxed title size=title,
  boxed title style={%
      sharp corners,
      rounded corners=northwest,
      colback=tcbcolframe,
      boxrule=0pt,
    },
  underlay boxed title={%
      \path[fill=tcbcolframe] (title.south west)--(title.south east)
      to[out=0, in=180] ([xshift=5mm]title.east)--
      (title.center-|frame.east)
      [rounded corners=\kvtcb@arc] |-
      (frame.north) -| cycle;
    },
  #1
}{def}
\makeatother

%================================
% SOLUTION BOX
%================================

\makeatletter
\newtcolorbox{solution}{enhanced,
  breakable,
  colback=white,
  colframe=myg!80!black,
  attach boxed title to top left={yshift*=-\tcboxedtitleheight},
  title=Solution,
  boxed title size=title,
  boxed title style={%
      sharp corners,
      rounded corners=northwest,
      colback=tcbcolframe,
      boxrule=0pt,
    },
  underlay boxed title={%
      \path[fill=tcbcolframe] (title.south west)--(title.south east)
      to[out=0, in=180] ([xshift=5mm]title.east)--
      (title.center-|frame.east)
      [rounded corners=\kvtcb@arc] |-
      (frame.north) -| cycle;
    },
}
\makeatother

%================================
% Question BOX
%================================

\makeatletter
\newtcbtheorem{qstion}{Question}{enhanced,
  breakable,
  colback=white,
  colframe=mygr,
  attach boxed title to top left={yshift*=-\tcboxedtitleheight},
  fonttitle=\bfseries,
  title={#2},
  boxed title size=title,
  boxed title style={%
      sharp corners,
      rounded corners=northwest,
      colback=tcbcolframe,
      boxrule=0pt,
    },
  underlay boxed title={%
      \path[fill=tcbcolframe] (title.south west)--(title.south east)
      to[out=0, in=180] ([xshift=5mm]title.east)--
      (title.center-|frame.east)
      [rounded corners=\kvtcb@arc] |-
      (frame.north) -| cycle;
    },
  #1
}{def}
\makeatother

\newtcbtheorem[number within=chapter]{wconc}{Wrong Concept}{
  breakable,
  enhanced,
  colback=white,
  colframe=myr,
  arc=0pt,
  outer arc=0pt,
  fonttitle=\bfseries\sffamily\large,
  colbacktitle=myr,
  attach boxed title to top left={},
  boxed title style={
      enhanced,
      skin=enhancedfirst jigsaw,
      arc=3pt,
      bottom=0pt,
      interior style={fill=myr}
    },
  #1
}{def}



%================================
% NOTE BOX
%================================

\usetikzlibrary{arrows,calc,shadows.blur}
\tcbuselibrary{skins}
\newtcolorbox{note}[1][]{%
  enhanced jigsaw,
  colback=gray!20!white,%
  colframe=gray!80!black,
  size=small,
  boxrule=1pt,
  title=\textbf{Note:-},
  halign title=flush center,
  coltitle=black,
  breakable,
  drop shadow=black!50!white,
  attach boxed title to top left={xshift=1cm,yshift=-\tcboxedtitleheight/2,yshifttext=-\tcboxedtitleheight/2},
  minipage boxed title=1.5cm,
  boxed title style={%
      colback=white,
      size=fbox,
      boxrule=1pt,
      boxsep=2pt,
      underlay={%
          \coordinate (dotA) at ($(interior.west) + (-0.5pt,0)$);
          \coordinate (dotB) at ($(interior.east) + (0.5pt,0)$);
          \begin{scope}
            \clip (interior.north west) rectangle ([xshift=3ex]interior.east);
            \filldraw [white, blur shadow={shadow opacity=60, shadow yshift=-.75ex}, rounded corners=2pt] (interior.north west) rectangle (interior.south east);
          \end{scope}
          \begin{scope}[gray!80!black]
            \fill (dotA) circle (2pt);
            \fill (dotB) circle (2pt);
          \end{scope}
        },
    },
  #1,
}

%%%%%%%%%%%%%%%%%%%%%%%%%%%%%%
% SELF MADE COMMANDS
%%%%%%%%%%%%%%%%%%%%%%%%%%%%%%


\newcommand{\thm}[2]{\begin{Theorem}{#1}{}#2\end{Theorem}}
\newcommand{\cor}[2]{\begin{Corollary}{#1}{}#2\end{Corollary}}
\newcommand{\mlenma}[2]{\begin{Lenma}{#1}{}#2\end{Lenma}}
\newcommand{\mprop}[2]{\begin{Prop}{#1}{}#2\end{Prop}}
\newcommand{\clm}[3]{\begin{claim}{#1}{#2}#3\end{claim}}
\newcommand{\wc}[2]{\begin{wconc}{#1}{}\setlength{\parindent}{1cm}#2\end{wconc}}
\newcommand{\thmcon}[1]{\begin{Theoremcon}{#1}\end{Theoremcon}}
\newcommand{\ex}[2]{\begin{Example}{#1}{}#2\end{Example}}
\newcommand{\dfn}[2]{\begin{Definition}[colbacktitle=red!75!black]{#1}{}#2\end{Definition}}
\newcommand{\dfnc}[2]{\begin{definition}[colbacktitle=red!75!black]{#1}{}#2\end{definition}}
\newcommand{\qs}[2]{\begin{question}{#1}{}#2\end{question}}
\newcommand{\pf}[2]{\begin{myproof}[#1]#2\end{myproof}}
\newcommand{\nt}[1]{\begin{note}#1\end{note}}

\newcommand*\circled[1]{\tikz[baseline=(char.base)]{
    \node[shape=circle,draw,inner sep=1pt] (char) {#1};}}
\newcommand\getcurrentref[1]{%
  \ifnumequal{\value{#1}}{0}
  {??}
  {\the\value{#1}}%
}
\newcommand{\getCurrentSectionNumber}{\getcurrentref{section}}
\newenvironment{myproof}[1][\proofname]{%
  \proof[\bfseries #1: ]%
}{\endproof}

\newcommand{\mclm}[2]{\begin{myclaim}[#1]#2\end{myclaim}}
\newenvironment{myclaim}[1][\claimname]{\proof[\bfseries #1: ]}{}

\newcounter{mylabelcounter}

\makeatletter
\newcommand{\setword}[2]{%
  \phantomsection
  #1\def\@currentlabel{\unexpanded{#1}}\label{#2}%
}
\makeatother




\tikzset{
  symbol/.style={
      draw=none,
      every to/.append style={
          edge node={node [sloped, allow upside down, auto=false]{$#1$}}}
    }
}


% deliminators
\DeclarePairedDelimiter{\abs}{\lvert}{\rvert}
\DeclarePairedDelimiter{\norm}{\lVert}{\rVert}

\DeclarePairedDelimiter{\ceil}{\lceil}{\rceil}
\DeclarePairedDelimiter{\floor}{\lfloor}{\rfloor}
\DeclarePairedDelimiter{\round}{\lfloor}{\rceil}

\newsavebox\diffdbox
\newcommand{\slantedromand}{{\mathpalette\makesl{d}}}
\newcommand{\makesl}[2]{%
  \begingroup
  \sbox{\diffdbox}{$\mathsurround=0pt#1\mathrm{#2}$}%
  \pdfsave
  \pdfsetmatrix{1 0 0.2 1}%
  \rlap{\usebox{\diffdbox}}%
  \pdfrestore
  \hskip\wd\diffdbox
  \endgroup
}
\newcommand{\dd}[1][]{\ensuremath{\mathop{}\!\ifstrempty{#1}{%
      \slantedromand\@ifnextchar^{\hspace{0.2ex}}{\hspace{0.1ex}}}%
    {\slantedromand\hspace{0.2ex}^{#1}}}}
\ProvideDocumentCommand\dv{o m g}{%
  \ensuremath{%
    \IfValueTF{#3}{%
      \IfNoValueTF{#1}{%
        \frac{\dd #2}{\dd #3}%
      }{%
        \frac{\dd^{#1} #2}{\dd #3^{#1}}%
      }%
    }{%
      \IfNoValueTF{#1}{%
        \frac{\dd}{\dd #2}%
      }{%
        \frac{\dd^{#1}}{\dd #2^{#1}}%
      }%
    }%
  }%
}
\providecommand*{\pdv}[3][]{\frac{\partial^{#1}#2}{\partial#3^{#1}}}
%  - others
\DeclareMathOperator{\Lap}{\mathcal{L}}
\DeclareMathOperator{\Var}{Var} % varience
\DeclareMathOperator{\Cov}{Cov} % covarience
\DeclareMathOperator{\E}{E} % expected

% Since the amsthm package isn't loaded

% I prefer the slanted \leq
\let\oldleq\leq % save them in case they're every wanted
\let\oldgeq\geq
\renewcommand{\leq}{\leqslant}
\renewcommand{\geq}{\geqslant}

% % redefine matrix env to allow for alignment, use r as default
% \renewcommand*\env@matrix[1][r]{\hskip -\arraycolsep
%     \let\@ifnextchar\new@ifnextchar
%     \array{*\c@MaxMatrixCols #1}}


%\usepackage{framed}
%\usepackage{titletoc}
%\usepackage{etoolbox}
%\usepackage{lmodern}


%\patchcmd{\tableofcontents}{\contentsname}{\sffamily\contentsname}{}{}

%\renewenvironment{leftbar}
%{\def\FrameCommand{\hspace{6em}%
%		{\color{myyellow}\vrule width 2pt depth 6pt}\hspace{1em}}%
%	\MakeFramed{\parshape 1 0cm \dimexpr\textwidth-6em\relax\FrameRestore}\vskip2pt%
%}
%{\endMakeFramed}

%\titlecontents{chapter}
%[0em]{\vspace*{2\baselineskip}}
%{\parbox{4.5em}{%
%		\hfill\Huge\sffamily\bfseries\color{myred}\thecontentspage}%
%	\vspace*{-2.3\baselineskip}\leftbar\textsc{\small\chaptername~\thecontentslabel}\\\sffamily}
%{}{\endleftbar}
%\titlecontents{section}
%[8.4em]
%{\sffamily\contentslabel{3em}}{}{}
%{\hspace{0.5em}\nobreak\itshape\color{myred}\contentspage}
%\titlecontents{subsection}
%[8.4em]
%{\sffamily\contentslabel{3em}}{}{}  
%{\hspace{0.5em}\nobreak\itshape\color{myred}\contentspage}



%%%%%%%%%%%%%%%%%%%%%%%%%%%%%%%%%%%%%%%%%%%
% TABLE OF CONTENTS
%%%%%%%%%%%%%%%%%%%%%%%%%%%%%%%%%%%%%%%%%%%

\usepackage{tikz}
\definecolor{doc}{RGB}{0,60,110}
\usepackage{titletoc}
\contentsmargin{0cm}
\titlecontents{chapter}[3.7pc]
{\addvspace{30pt}%
  \begin{tikzpicture}[remember picture, overlay]%
    \draw[fill=doc!60,draw=doc!60] (-7,-.1) rectangle (-0.9,.5);%
    \pgftext[left,x=-3.5cm,y=0.2cm]{\color{white}\Large\sc\bfseries Chapter\ \thecontentslabel};%
  \end{tikzpicture}\color{doc!60}\large\sc\bfseries}%
{}
{}
{\;\titlerule\;\large\sc\bfseries Page \thecontentspage
  \begin{tikzpicture}[remember picture, overlay]
    \draw[fill=doc!60,draw=doc!60] (2pt,0) rectangle (4,0.1pt);
  \end{tikzpicture}}%
\titlecontents{section}[3.7pc]
{\addvspace{2pt}}
{\contentslabel[\thecontentslabel]{2pc}}
{}
{\hfill\small \thecontentspage}
[]
\titlecontents*{subsection}[3.7pc]
{\addvspace{-1pt}\small}
{}
{}
{\ --- \small\thecontentspage}
[ \textbullet\ ][]

\makeatletter
\renewcommand{\tableofcontents}{%
  \chapter*{%
    \vspace*{-20\p@}%
    \begin{tikzpicture}[remember picture, overlay]%
      \pgftext[right,x=15cm,y=0.2cm]{\color{doc!60}\Huge\sc\bfseries \contentsname};%
      \draw[fill=doc!60,draw=doc!60] (13,-.75) rectangle (20,1);%
      \clip (13,-.75) rectangle (20,1);
      \pgftext[right,x=15cm,y=0.2cm]{\color{white}\Huge\sc\bfseries \contentsname};%
    \end{tikzpicture}}%
  \@starttoc{toc}}
\makeatother

%From M275 "Topology" at SJSU
\newcommand{\id}{\mathrm{id}}
\newcommand{\taking}[1]{\xrightarrow{#1}}
\newcommand{\inv}{^{-1}}

%From M170 "Introduction to Graph Theory" at SJSU
\DeclareMathOperator{\diam}{diam}
\DeclareMathOperator{\ord}{ord}
\newcommand{\defeq}{\overset{\mathrm{def}}{=}}

%From the USAMO .tex files
\newcommand{\ts}{\textsuperscript}
\newcommand{\dg}{^\circ}
\newcommand{\ii}{\item}

% % From Math 55 and Math 145 at Harvard
% \newenvironment{subproof}[1][Proof]{%
% \begin{proof}[#1] \renewcommand{\qedsymbol}{$\blacksquare$}}%
% {\end{proof}}

\newcommand{\liff}{\leftrightarrow}
\newcommand{\lthen}{\rightarrow}
\newcommand{\opname}{\operatorname}
\newcommand{\surjto}{\twoheadrightarrow}
\newcommand{\injto}{\hookrightarrow}
\newcommand{\On}{\mathrm{On}} % ordinals
\DeclareMathOperator{\img}{im} % Image
\DeclareMathOperator{\Img}{Im} % Image
\DeclareMathOperator{\coker}{coker} % Cokernel
\DeclareMathOperator{\Coker}{Coker} % Cokernel
\DeclareMathOperator{\Ker}{Ker} % Kernel
\DeclareMathOperator{\rank}{rank}
\DeclareMathOperator{\Spec}{Spec} % spectrum
\DeclareMathOperator{\Tr}{Tr} % trace
\DeclareMathOperator{\pr}{pr} % projection
\DeclareMathOperator{\ext}{ext} % extension
\DeclareMathOperator{\pred}{pred} % predecessor
\DeclareMathOperator{\dom}{dom} % domain
\DeclareMathOperator{\ran}{ran} % range
\DeclareMathOperator{\Hom}{Hom} % homomorphism
\DeclareMathOperator{\Mor}{Mor} % morphisms
\DeclareMathOperator{\End}{End} % endomorphism

\newcommand{\eps}{\epsilon}
\newcommand{\veps}{\varepsilon}
\newcommand{\ol}{\overline}
\newcommand{\ul}{\underline}
\newcommand{\wt}{\widetilde}
\newcommand{\wh}{\widehat}
\newcommand{\vocab}[1]{\textbf{\color{blue} #1}}
\providecommand{\half}{\frac{1}{2}}
\newcommand{\dang}{\measuredangle} %% Directed angle
\newcommand{\ray}[1]{\overrightarrow{#1}}
\newcommand{\seg}[1]{\overline{#1}}
\newcommand{\arc}[1]{\wideparen{#1}}
\DeclareMathOperator{\cis}{cis}
\DeclareMathOperator*{\lcm}{lcm}
\DeclareMathOperator*{\argmin}{arg min}
\DeclareMathOperator*{\argmax}{arg max}
\newcommand{\cycsum}{\sum_{\mathrm{cyc}}}
\newcommand{\symsum}{\sum_{\mathrm{sym}}}
\newcommand{\cycprod}{\prod_{\mathrm{cyc}}}
\newcommand{\symprod}{\prod_{\mathrm{sym}}}
\newcommand{\Qed}{\begin{flushright}\qed\end{flushright}}
\newcommand{\parinn}{\setlength{\parindent}{1cm}}
\newcommand{\parinf}{\setlength{\parindent}{0cm}}
% \newcommand{\norm}{\|\cdot\|}
\newcommand{\inorm}{\norm_{\infty}}
\newcommand{\opensets}{\{V_{\alpha}\}_{\alpha\in I}}
\newcommand{\oset}{V_{\alpha}}
\newcommand{\opset}[1]{V_{\alpha_{#1}}}
\newcommand{\lub}{\text{lub}}
\newcommand{\del}[2]{\frac{\partial #1}{\partial #2}}
\newcommand{\Del}[3]{\frac{\partial^{#1} #2}{\partial^{#1} #3}}
\newcommand{\deld}[2]{\dfrac{\partial #1}{\partial #2}}
\newcommand{\Deld}[3]{\dfrac{\partial^{#1} #2}{\partial^{#1} #3}}
\newcommand{\lm}{\lambda}
\newcommand{\uin}{\mathbin{\rotatebox[origin=c]{90}{$\in$}}}
\newcommand{\usubset}{\mathbin{\rotatebox[origin=c]{90}{$\subset$}}}
\newcommand{\lt}{\left}
\newcommand{\rt}{\right}
\newcommand{\bs}[1]{\boldsymbol{#1}}
\newcommand{\exs}{\exists}
\newcommand{\st}{\strut}
\newcommand{\dps}[1]{\displaystyle{#1}}

\newcommand{\sol}{\setlength{\parindent}{0cm}\textbf{\textit{Solution:}}\setlength{\parindent}{1cm} }
\newcommand{\solve}[1]{\setlength{\parindent}{0cm}\textbf{\textit{Solution: }}\setlength{\parindent}{1cm}#1 \Qed}

\preto\tabular{\setcounter{magicrownumbers}{0}}
\newcounter{magicrownumbers}
\newcommand\rownumber{\stepcounter{magicrownumbers}\arabic{magicrownumbers}}
\def\rownumber{}

\newenvironment{deduction}
{\begin{tabular}{@{}>{$}c<{$}@{\enspace}>{$}l<{$}@{}}\arrayrulecolor{blue!50}}
		{\end{tabular}}
\newcommand{\premise}[1]{&#1\\}
\newcommand{\conclusion}[1]{\cline{2-2}\therefore&#1}


% Things Lie
\newcommand{\kb}{\mathfrak b}
\newcommand{\kg}{\mathfrak g}
\newcommand{\kh}{\mathfrak h}
\newcommand{\kn}{\mathfrak n}
\newcommand{\ku}{\mathfrak u}
\newcommand{\kz}{\mathfrak z}
\DeclareMathOperator{\Ext}{Ext} % Ext functor
\DeclareMathOperator{\Tor}{Tor} % Tor functor
\newcommand{\gl}{\opname{\mathfrak{gl}}} % frak gl group
\renewcommand{\sl}{\opname{\mathfrak{sl}}} % frak sl group chktex 6

% More script letters etc.
\newcommand{\SA}{\mathcal A}
\newcommand{\SB}{\mathcal B}
\newcommand{\SC}{\mathcal C}
\newcommand{\SF}{\mathcal F}
\newcommand{\SG}{\mathcal G}
\newcommand{\SH}{\mathcal H}
\newcommand{\OO}{\mathcal O}

\newcommand{\SCA}{\mathscr A}
\newcommand{\SCB}{\mathscr B}
\newcommand{\SCC}{\mathscr C}
\newcommand{\SCD}{\mathscr D}
\newcommand{\SCE}{\mathscr E}
\newcommand{\SCF}{\mathscr F}
\newcommand{\SCG}{\mathscr G}
\newcommand{\SCH}{\mathscr H}

% Mathfrak primes
\newcommand{\km}{\mathfrak m}
\newcommand{\kp}{\mathfrak p}
\newcommand{\kq}{\mathfrak q}

% number sets
\newcommand{\RR}[1][]{\ensuremath{\ifstrempty{#1}{\mathbb{R}}{\mathbb{R}^{#1}}}}
\newcommand{\NN}[1][]{\ensuremath{\ifstrempty{#1}{\mathbb{N}}{\mathbb{N}^{#1}}}}
\newcommand{\ZZ}[1][]{\ensuremath{\ifstrempty{#1}{\mathbb{Z}}{\mathbb{Z}^{#1}}}}
\newcommand{\QQ}[1][]{\ensuremath{\ifstrempty{#1}{\mathbb{Q}}{\mathbb{Q}^{#1}}}}
\newcommand{\CC}[1][]{\ensuremath{\ifstrempty{#1}{\mathbb{C}}{\mathbb{C}^{#1}}}}
\newcommand{\PP}[1][]{\ensuremath{\ifstrempty{#1}{\mathbb{P}}{\mathbb{P}^{#1}}}}
\newcommand{\HH}[1][]{\ensuremath{\ifstrempty{#1}{\mathbb{H}}{\mathbb{H}^{#1}}}}
\newcommand{\FF}[1][]{\ensuremath{\ifstrempty{#1}{\mathbb{F}}{\mathbb{F}^{#1}}}}
% expected value
\newcommand{\EE}{\ensuremath{\mathbb{E}}}
\newcommand{\charin}{\text{ char }}
\DeclareMathOperator{\sign}{sign}
\DeclareMathOperator{\Aut}{Aut}
\DeclareMathOperator{\Inn}{Inn}
\DeclareMathOperator{\Syl}{Syl}
\DeclareMathOperator{\Gal}{Gal}
\DeclareMathOperator{\GL}{GL} % General linear group
\DeclareMathOperator{\SL}{SL} % Special linear group

%---------------------------------------
% BlackBoard Math Fonts :-
%---------------------------------------

%Captital Letters
\newcommand{\bbA}{\mathbb{A}}	\newcommand{\bbB}{\mathbb{B}}
\newcommand{\bbC}{\mathbb{C}}	\newcommand{\bbD}{\mathbb{D}}
\newcommand{\bbE}{\mathbb{E}}	\newcommand{\bbF}{\mathbb{F}}
\newcommand{\bbG}{\mathbb{G}}	\newcommand{\bbH}{\mathbb{H}}
\newcommand{\bbI}{\mathbb{I}}	\newcommand{\bbJ}{\mathbb{J}}
\newcommand{\bbK}{\mathbb{K}}	\newcommand{\bbL}{\mathbb{L}}
\newcommand{\bbM}{\mathbb{M}}	\newcommand{\bbN}{\mathbb{N}}
\newcommand{\bbO}{\mathbb{O}}	\newcommand{\bbP}{\mathbb{P}}
\newcommand{\bbQ}{\mathbb{Q}}	\newcommand{\bbR}{\mathbb{R}}
\newcommand{\bbS}{\mathbb{S}}	\newcommand{\bbT}{\mathbb{T}}
\newcommand{\bbU}{\mathbb{U}}	\newcommand{\bbV}{\mathbb{V}}
\newcommand{\bbW}{\mathbb{W}}	\newcommand{\bbX}{\mathbb{X}}
\newcommand{\bbY}{\mathbb{Y}}	\newcommand{\bbZ}{\mathbb{Z}}

%---------------------------------------
% MathCal Fonts :-
%---------------------------------------

%Captital Letters
\newcommand{\mcA}{\mathcal{A}}	\newcommand{\mcB}{\mathcal{B}}
\newcommand{\mcC}{\mathcal{C}}	\newcommand{\mcD}{\mathcal{D}}
\newcommand{\mcE}{\mathcal{E}}	\newcommand{\mcF}{\mathcal{F}}
\newcommand{\mcG}{\mathcal{G}}	\newcommand{\mcH}{\mathcal{H}}
\newcommand{\mcI}{\mathcal{I}}	\newcommand{\mcJ}{\mathcal{J}}
\newcommand{\mcK}{\mathcal{K}}	\newcommand{\mcL}{\mathcal{L}}
\newcommand{\mcM}{\mathcal{M}}	\newcommand{\mcN}{\mathcal{N}}
\newcommand{\mcO}{\mathcal{O}}	\newcommand{\mcP}{\mathcal{P}}
\newcommand{\mcQ}{\mathcal{Q}}	\newcommand{\mcR}{\mathcal{R}}
\newcommand{\mcS}{\mathcal{S}}	\newcommand{\mcT}{\mathcal{T}}
\newcommand{\mcU}{\mathcal{U}}	\newcommand{\mcV}{\mathcal{V}}
\newcommand{\mcW}{\mathcal{W}}	\newcommand{\mcX}{\mathcal{X}}
\newcommand{\mcY}{\mathcal{Y}}	\newcommand{\mcZ}{\mathcal{Z}}


%---------------------------------------
% Bold Math Fonts :-
%---------------------------------------

%Captital Letters
\newcommand{\bmA}{\boldsymbol{A}}	\newcommand{\bmB}{\boldsymbol{B}}
\newcommand{\bmC}{\boldsymbol{C}}	\newcommand{\bmD}{\boldsymbol{D}}
\newcommand{\bmE}{\boldsymbol{E}}	\newcommand{\bmF}{\boldsymbol{F}}
\newcommand{\bmG}{\boldsymbol{G}}	\newcommand{\bmH}{\boldsymbol{H}}
\newcommand{\bmI}{\boldsymbol{I}}	\newcommand{\bmJ}{\boldsymbol{J}}
\newcommand{\bmK}{\boldsymbol{K}}	\newcommand{\bmL}{\boldsymbol{L}}
\newcommand{\bmM}{\boldsymbol{M}}	\newcommand{\bmN}{\boldsymbol{N}}
\newcommand{\bmO}{\boldsymbol{O}}	\newcommand{\bmP}{\boldsymbol{P}}
\newcommand{\bmQ}{\boldsymbol{Q}}	\newcommand{\bmR}{\boldsymbol{R}}
\newcommand{\bmS}{\boldsymbol{S}}	\newcommand{\bmT}{\boldsymbol{T}}
\newcommand{\bmU}{\boldsymbol{U}}	\newcommand{\bmV}{\boldsymbol{V}}
\newcommand{\bmW}{\boldsymbol{W}}	\newcommand{\bmX}{\boldsymbol{X}}
\newcommand{\bmY}{\boldsymbol{Y}}	\newcommand{\bmZ}{\boldsymbol{Z}}
%Small Letters
\newcommand{\bma}{\boldsymbol{a}}	\newcommand{\bmb}{\boldsymbol{b}}
\newcommand{\bmc}{\boldsymbol{c}}	\newcommand{\bmd}{\boldsymbol{d}}
\newcommand{\bme}{\boldsymbol{e}}	\newcommand{\bmf}{\boldsymbol{f}}
\newcommand{\bmg}{\boldsymbol{g}}	\newcommand{\bmh}{\boldsymbol{h}}
\newcommand{\bmi}{\boldsymbol{i}}	\newcommand{\bmj}{\boldsymbol{j}}
\newcommand{\bmk}{\boldsymbol{k}}	\newcommand{\bml}{\boldsymbol{l}}
\newcommand{\bmm}{\boldsymbol{m}}	\newcommand{\bmn}{\boldsymbol{n}}
\newcommand{\bmo}{\boldsymbol{o}}	\newcommand{\bmp}{\boldsymbol{p}}
\newcommand{\bmq}{\boldsymbol{q}}	\newcommand{\bmr}{\boldsymbol{r}}
\newcommand{\bms}{\boldsymbol{s}}	\newcommand{\bmt}{\boldsymbol{t}}
\newcommand{\bmu}{\boldsymbol{u}}	\newcommand{\bmv}{\boldsymbol{v}}
\newcommand{\bmw}{\boldsymbol{w}}	\newcommand{\bmx}{\boldsymbol{x}}
\newcommand{\bmy}{\boldsymbol{y}}	\newcommand{\bmz}{\boldsymbol{z}}

%---------------------------------------
% Scr Math Fonts :-
%---------------------------------------

\newcommand{\sA}{{\mathscr{A}}}   \newcommand{\sB}{{\mathscr{B}}}
\newcommand{\sC}{{\mathscr{C}}}   \newcommand{\sD}{{\mathscr{D}}}
\newcommand{\sE}{{\mathscr{E}}}   \newcommand{\sF}{{\mathscr{F}}}
\newcommand{\sG}{{\mathscr{G}}}   \newcommand{\sH}{{\mathscr{H}}}
\newcommand{\sI}{{\mathscr{I}}}   \newcommand{\sJ}{{\mathscr{J}}}
\newcommand{\sK}{{\mathscr{K}}}   \newcommand{\sL}{{\mathscr{L}}}
\newcommand{\sM}{{\mathscr{M}}}   \newcommand{\sN}{{\mathscr{N}}}
\newcommand{\sO}{{\mathscr{O}}}   \newcommand{\sP}{{\mathscr{P}}}
\newcommand{\sQ}{{\mathscr{Q}}}   \newcommand{\sR}{{\mathscr{R}}}
\newcommand{\sS}{{\mathscr{S}}}   \newcommand{\sT}{{\mathscr{T}}}
\newcommand{\sU}{{\mathscr{U}}}   \newcommand{\sV}{{\mathscr{V}}}
\newcommand{\sW}{{\mathscr{W}}}   \newcommand{\sX}{{\mathscr{X}}}
\newcommand{\sY}{{\mathscr{Y}}}   \newcommand{\sZ}{{\mathscr{Z}}}


%---------------------------------------
% Math Fraktur Font
%---------------------------------------

%Captital Letters
\newcommand{\mfA}{\mathfrak{A}}	\newcommand{\mfB}{\mathfrak{B}}
\newcommand{\mfC}{\mathfrak{C}}	\newcommand{\mfD}{\mathfrak{D}}
\newcommand{\mfE}{\mathfrak{E}}	\newcommand{\mfF}{\mathfrak{F}}
\newcommand{\mfG}{\mathfrak{G}}	\newcommand{\mfH}{\mathfrak{H}}
\newcommand{\mfI}{\mathfrak{I}}	\newcommand{\mfJ}{\mathfrak{J}}
\newcommand{\mfK}{\mathfrak{K}}	\newcommand{\mfL}{\mathfrak{L}}
\newcommand{\mfM}{\mathfrak{M}}	\newcommand{\mfN}{\mathfrak{N}}
\newcommand{\mfO}{\mathfrak{O}}	\newcommand{\mfP}{\mathfrak{P}}
\newcommand{\mfQ}{\mathfrak{Q}}	\newcommand{\mfR}{\mathfrak{R}}
\newcommand{\mfS}{\mathfrak{S}}	\newcommand{\mfT}{\mathfrak{T}}
\newcommand{\mfU}{\mathfrak{U}}	\newcommand{\mfV}{\mathfrak{V}}
\newcommand{\mfW}{\mathfrak{W}}	\newcommand{\mfX}{\mathfrak{X}}
\newcommand{\mfY}{\mathfrak{Y}}	\newcommand{\mfZ}{\mathfrak{Z}}
%Small Letters
\newcommand{\mfa}{\mathfrak{a}}	\newcommand{\mfb}{\mathfrak{b}}
\newcommand{\mfc}{\mathfrak{c}}	\newcommand{\mfd}{\mathfrak{d}}
\newcommand{\mfe}{\mathfrak{e}}	\newcommand{\mff}{\mathfrak{f}}
\newcommand{\mfg}{\mathfrak{g}}	\newcommand{\mfh}{\mathfrak{h}}
\newcommand{\mfi}{\mathfrak{i}}	\newcommand{\mfj}{\mathfrak{j}}
\newcommand{\mfk}{\mathfrak{k}}	\newcommand{\mfl}{\mathfrak{l}}
\newcommand{\mfm}{\mathfrak{m}}	\newcommand{\mfn}{\mathfrak{n}}
\newcommand{\mfo}{\mathfrak{o}}	\newcommand{\mfp}{\mathfrak{p}}
\newcommand{\mfq}{\mathfrak{q}}	\newcommand{\mfr}{\mathfrak{r}}
\newcommand{\mfs}{\mathfrak{s}}	\newcommand{\mft}{\mathfrak{t}}
\newcommand{\mfu}{\mathfrak{u}}	\newcommand{\mfv}{\mathfrak{v}}
\newcommand{\mfw}{\mathfrak{w}}	\newcommand{\mfx}{\mathfrak{x}}
\newcommand{\mfy}{\mathfrak{y}}	\newcommand{\mfz}{\mathfrak{z}}


\title{\Huge{Eigenvalues and Eigenvectors}}
\author{\huge{Madiba Hudson-Quansah}}
\date{}
\usepackage{parskip}

\setcounter{tocdepth}{4}
\setcounter{secnumdepth}{4}

\begin{document}
\maketitle
\newpage
\pdfbookmark[section]{\contentsname}{too}
\tableofcontents
\pagebreak

\chapter{Introduction}

\dfn{Eigenvector}{
  An eigenvector of a of a $n\times n$ matrix $A$ is a non-zero vector $\mbold{x}$ such that $A\mbold{x} = \lambda \mbold{x}$ for
  some scalar $\lambda$. A scalar $\lambda $ is called an eigenvalue of $A$ if there is a non-trivial solution
  $\mbold{x}$ of $A\mbold{x} = \lambda \mbold{x}$; such an $\mbold{x}$ is called an \textit{eigenvector corresponding to
    $\lambda $}.
}

\ex{}{
  \qs{}{
    Let $A = \begin{bmatrix} 1 & 6 \\ 5 & 2 \end{bmatrix} $, $\mbold{u} = \begin{bmatrix} 6 \\ -5 \end{bmatrix} $, and
    $\mbold{v} = \begin{bmatrix} 3 \\ -2 \end{bmatrix} $. Are $\mbold{u}$ and $\mbold{v}$ eigenvectors of $A$?
  }

  \sol{
    \begin{align*}
      A \mbold{u} & = \begin{bmatrix} 1 & 6 \\ 5 & 2 \end{bmatrix} \begin{bmatrix} 6 \\ -5 \end{bmatrix} =
      \begin{bmatrix} -24 \\ 20 \end{bmatrix} = -4 \begin{bmatrix} 6 \\ -5 \end{bmatrix} = -4 \mbold{u}    \\
      A \mbold{v} & = \begin{bmatrix} 1 & 6 \\ 5 & 2 \end{bmatrix} \begin{bmatrix} 3 \\ -2 \end{bmatrix} =
      \begin{bmatrix} -9 \\ 11 \end{bmatrix} \neq \lambda \begin{bmatrix} 3 \\ -2 \end{bmatrix}            \\
    \end{align*}
  }
  \qs{}{
    Show that 7 is an eigenvalue of $A$ and find the corresponding eigenvector.
  }

  \sol{
    To show this we need to prove that $A \mbold{x} = \lambda \mbold{x}$ where $\lambda = 7$ has non-trivial solutions.
    \begin{align*}
      A \mbold{x}                                                                                                          & = \lambda \mbold{x} \\
      A \mbold{x}                                                                                                          & = 7 \mbold{x}       \\
      A\mbold{x} - 7\mbold{x}                                                                                              & = \mbold{0}         \\
      \left( A - 7 I \right) \mbold{x}                                                                                     & = \mbold{0}         \\
      \left( \begin{bmatrix} 1 & 6 \\ 5 & 2 \end{bmatrix} - 7\begin{bmatrix} 1 & 0 \\ 0 & 1 \end{bmatrix}
      \right)\mbold{x}                                                                                                     & = \mbold{0}         \\
      \left( \begin{bmatrix} 1 & 6 \\ 5 & 2 \end{bmatrix} - \begin{bmatrix} 7 & 0 \\ 0 & 7 \end{bmatrix} \right) \mbold{x} & =
      \mbold{0}                                                                                                                                  \\
      \begin{bmatrix} -6 & 6 \\ 5 & -5  \end{bmatrix} \mbold{x}                                                            & = \mbold{0}         \\
      \begin{bmatrix} -6 & 6 & 0 \\ 5 & -5 & 0  \end{bmatrix}                                                                                    \\
      \frac{-5}{6}R_1 - R_2  \rightarrow R_2                                                                                                     \\
      \begin{bmatrix}
        -6 & 6 & 0 \\
        0  & 0 & 0 \\
      \end{bmatrix}
      \\
      \frac{-1}{6}R_1 \to R_1                                                                                                                    \\
      \begin{bmatrix}
        1 & -1 & 0 \\
        0 & 0  & 0 \\
      \end{bmatrix}
      \\
      x_1 -x_2 = 0                                                                                                                               \\
      x_1 = x_2                                                                                                                                  \\
      x_2 =x_2                                                                                                                                   \\
      \mbold{x} = \begin{bmatrix} x_1 \\ x_2 \end{bmatrix} = x_2\begin{bmatrix} 1 \\ 1  \end{bmatrix}
    \end{align*}
    This system has non-trivial solutions as the columns are multiples of themselves and such linearly dependent.
    Therefore 7 is a eigenvalue of $A$, with the corresponding eigenvectors in the form $x_2 \begin{bmatrix} 1 \\ 1
      \end{bmatrix} $ where $x_2 \neq 0$
  }
}

This brings us to the next conclusion:

A scalar $\lambda $ is an eigenvalue of a matrix $A$ if and only if
\begin{equation}
  \left( A - \lambda I \right) \mbold{x} = \mbold{0}
  \label{eq:1}
\end{equation}
Has a non-trivial solution, where the corresponding eigenvectors is in the form of the parametric vector equation of the
solution set of this non-homogeneous system.

The set of all solutions of \ref{eq:1} is just the null space of the matrix $A - \lambda I$. This solution set is a
subspace of $\mathbb{R}^{n}$ and is called the \textit{eigenspace} of $A$ corresponding to $\lambda$

\dfn{Eigenspace}{
  The eigenspace of a matrix $A$ corresponding to an eigenvalue $\lambda$ is the set of all eigenvectors of $A$ corresponding
  to $\lambda$, together with the zero vector.
}

\ex{}{
  \qs{}{
    Let $A = \begin{bmatrix}
        4 & -1 & 6 \\
        2 & 1  & 6 \\
        2 & -1 & 8
      \end{bmatrix} $. An eigenvalue of $A$ is 2. Find a basis of for the eigenspace of $A$ corresponding to $\lambda = 2$.
  }

  \sol{
    \begin{align*}
      \left( A - 2I \right)                                                                             \\
      \left( \begin{bmatrix}
                 4 & -1 & 6 \\
                 2 & 1  & 6 \\
                 2 & -1 & 8
               \end{bmatrix} - \begin{bmatrix} 2 & 0 & 0 \\ 0 & 2 & 0 \\ 0 & 0 & 2 \end{bmatrix}  \right) \\
      \begin{bmatrix}
        2 & -1 & 6 \\
        2 & -1 & 6 \\
        2 & -1 & 6 \\
      \end{bmatrix}                                                                                    \\
      R_1 - R_2  \rightarrow R_2                                                                        \\
      \begin{bmatrix}
        2 & -1 & 6 \\
        0 & 0  & 0 \\
        2 & -1 & 6 \\
      \end{bmatrix}
      \\
      R_1 - R_3  \rightarrow R_3                                                                        \\
      \begin{bmatrix}
        2 & -1 & 6 \\
        0 & 0  & 0 \\
        0 & 0  & 0 \\
      \end{bmatrix}
      \\
      \frac{1}{2}R_1 \to R_1                                                                            \\
      \begin{bmatrix}
        1 & \frac{-1}{2} & 3 \\
        0 & 0            & 0 \\
        0 & 0            & 0 \\
      \end{bmatrix}
      \\
      x_1 = \frac{1}{2}x_2 - 3x_3
      x_2 = x_2                                                                                         \\
      x_3 = x_3
      \\
      \mbold{x} = \begin{bmatrix} \frac{1}{2}x_2 - 3x_3 \\ x_2 \\ x_3 \end{bmatrix} = x_2\begin{bmatrix} \frac{1}{2} \\
                                                                                           1           \\ 0\end{bmatrix} + x_3 \begin{bmatrix} -3 \\ 0 \\ 1 \end{bmatrix}
    \end{align*}
  }
}


\subsection{Exercises}

\qs{}{
  Is $\lambda = 2$ an eigenvalue of $\begin{bmatrix} 3 &2 \\ 3 & 8 \end{bmatrix} $? Why or why not?
}

\sol{
  \begin{align*}
    \left( \begin{bmatrix} 3 &2 \\ 3 & 8 \end{bmatrix} - \begin{bmatrix} 2 & 0 \\ 0 & 2 \end{bmatrix}    \right) \\
    \begin{bmatrix} 1 & 2 \\ 3 & 6 \end{bmatrix}                                                                 \\
    \begin{bmatrix} 1 & 2 & 0 \\ 3 & 6 & 0 \end{bmatrix}                                                         \\
    3R_1 - R_2  \rightarrow R_2                                                                                  \\
    \begin{bmatrix}
      1 & 2 & 0 \\
      0 & 0 & 0 \\
    \end{bmatrix}
    \\
    x_1 = -2x_2                                                                                                  \\
    x_2 = x_2                                                                                                    \\
    \mbold{x} = \begin{bmatrix} x_1 \\ x_2 \end{bmatrix} = \begin{bmatrix} -2x_2 \\ x_2 \end{bmatrix} = x_2
    \begin{bmatrix} -2 \\ 1 \end{bmatrix}
  \end{align*}
  This columns of the matrix are linearly dependent therefore $2$ is an eigenvalue of the matrix. And the eigenspace is
  the set of all vectors in the form $x_2 \begin{bmatrix} -2 \\ 1 \end{bmatrix} $ where $x_2 \neq 0$, i.e:
  \[
    \left\{ x_2 \begin{bmatrix} -2 \\ 1 \end{bmatrix}: x_2 \in \mathbb{R} \wedge x_2 \neq 0  \right\}
  \]
}

\qs{}{
  Is $\begin{bmatrix} 4 \\ -3 \\ 1 \end{bmatrix} $ and eigenvector of in $\begin{bmatrix} 3 & 7 & 9 \\ -4 & -5 & 1 \\ 2
                  & 4 & 4\end{bmatrix} $? If so find the corresponding eigenvalue.
}

\sol{
  \begin{align*}
    \begin{bmatrix} 3 & 7 & 9 \\ -4 & -5 & 1 \\ 2 & 4 & 4\end{bmatrix} \begin{bmatrix} 4 \\ -3 \\ 1 \end{bmatrix} & =
    \begin{bmatrix} 0 \\ 0 \\ 0 \end{bmatrix}                                                                                                                   \\
    0 \begin{bmatrix} -4 \\ -5 \\ 1 \end{bmatrix}                                                                 & = \begin{bmatrix} 0 \\  0\\ 0 \end{bmatrix} \\
    \therefore \lambda                                                                                            & = 0                                         \\
  \end{align*}
  $\therefore$ $\begin{bmatrix} 4 \\ -3 \\ 1 \end{bmatrix} $ is an eigenvector of the matrix, with 0 as its eigenvalue.
}

\qs{}{
  Is $\lambda  = 4$ an eigenvalue of $\begin{bmatrix} 3 & 0 &-1\\ 2 &3 & 1 \\ -3 & 4 & 5 \end{bmatrix} $? If so, find
  one corresponding eigenvector.
}

\sol{
  \begin{align*}
    \left( A - 4 I  \right)  = \mbold{0}                                                                                \\
    \begin{bmatrix} 3 & 0 &-1\\ 2 &3 & 1 \\ -3 & 4 & 5 \end{bmatrix} - \begin{bmatrix} 4 & 0 & 0 \\ 0 & 4 & 0 \\ 0 & 0 &
                4\end{bmatrix} \\
    \begin{bmatrix}
      -1 & 0  & -1 \\
      2  & -1 & 1  \\
      -3 & 4  & 1
    \end{bmatrix}                                                                                                      \\
    \begin{bmatrix}
      -1 & 0  & -1 & 0 \\
      2  & -1 & 1  & 0 \\
      -3 & 4  & 1  & 0
    \end{bmatrix}                                                                                                    \\
    -2R_1 - R_2  \rightarrow R_2                                                                                        \\
    \begin{bmatrix}
      -1 & 0 & -1 & 0 \\
      0  & 1 & 1  & 0 \\
      -3 & 4 & 1  & 0 \\
    \end{bmatrix}
    \\
    3R_1 - R_3  \rightarrow R_3                                                                                         \\
    \begin{bmatrix}
      -1 & 0  & -1 & 0 \\
      0  & 1  & 1  & 0 \\
      0  & -4 & -4 & 0 \\
    \end{bmatrix}
    \\
    -4R_2 - R_3  \rightarrow R_3                                                                                        \\
    \begin{bmatrix}
      -1 & 0 & -1 & 0 \\
      0  & 1 & 1  & 0 \\
      0  & 0 & 0  & 0 \\
    \end{bmatrix}
    \\
    -1R_1 \to R_1                                                                                                       \\
    \begin{bmatrix}
      1 & 0 & 1 & 0 \\
      0 & 1 & 1 & 0 \\
      0 & 0 & 0 & 0 \\
    \end{bmatrix}
    \\
    x_1 + x_3 = 0                                                                                                       \\
    x_2 + x_3 = 0                                                                                                       \\
    x_3 = 0                                                                                                             \\
    \\
    x_1 = -x_3                                                                                                          \\
    x_2 = -x_3                                                                                                          \\
    x_3 = x_3                                                                                                           \\
    \\
    \mbold{x} = \begin{bmatrix} x_1 \\ x_2 \\ x_3 \end{bmatrix}  = \begin{bmatrix} x_3 \\ x_3 \\ x_3 \end{bmatrix} = x_3
    \begin{bmatrix} -1 \\ -1 \\ 1 \end{bmatrix}
  \end{align*}
  $\therefore$ Since the columns of $ \left( A - 4I \right) $ are linearly dependent, 4 is an eigenvalue of the matrix $A$

  \noindent One eigenvector is found when $x_3=1$, $\begin{bmatrix} -1 \\ -1 \\ 1 \end{bmatrix} $
}

\qs{}{
  Find a basis for the eigenspace of $A = \begin{bmatrix} 5 & 0 \\ 2 & 1 \end{bmatrix} $ with eigenvalues $\lambda =1,5$
}

\sol{
  \begin{align*}
    \left( A - 1 I \right)                                                                      \\
    \begin{bmatrix} 5 & 0 \\ 2 & 1 \end{bmatrix} - \begin{bmatrix} 1 & 0 \\ 0 & 1 \end{bmatrix} \\
    \begin{bmatrix} 4 & 0 \\ 2 & 0 \end{bmatrix}                                                \\
    \begin{bmatrix} 4 & 0 & 0 \\ 2 & 0 & 0 \end{bmatrix}                                        \\
    \frac{1}{2}R_1 - R_2  \rightarrow R_2                                                       \\
    \begin{bmatrix}
      4 & 0 & 0 \\
      0 & 0 & 0 \\
    \end{bmatrix}
    \\
    \frac{1}{4}R_1 \to R_1                                                                      \\
    \begin{bmatrix}
      1 & 0 & 0 \\
      0 & 0 & 0 \\
    \end{bmatrix}
    \\
    x_1 = 0                                                                                     \\
    x_2 = x_2                                                                                   \\
    \\
    \mbold{x} = \begin{bmatrix} x_1 \\ x_2 \end{bmatrix} = \begin{bmatrix} 0 \\ x_2 \end{bmatrix}  =
    x_2
    \begin{bmatrix} 0
      \\ 1
    \end{bmatrix}                                                                            \\
    \therefore \text{ the basis of the eigenspace of $A$ with $\lambda = 1$ is } \left\{ \begin{bmatrix} 0 \\ 1 \end{bmatrix}  \right\}
  \end{align*}

  \begin{align*}
    \left( A - 5 I \right)                                                                      \\
    \begin{bmatrix} 5 & 0 \\ 2 & 1 \end{bmatrix} - \begin{bmatrix} 5 & 0 \\ 0 & 5 \end{bmatrix} \\
    \begin{bmatrix}
      0 & 0  \\
      2 & -4
    \end{bmatrix}                                                                              \\
    \begin{bmatrix}
      0 & 0  & 0 \\
      2 & -4 & 0
    \end{bmatrix}                                                                              \\
    R_1 \leftrightarrow R_2                                                                     \\
    \begin{bmatrix}
      2 & -4 & 0 \\
      0 & 0  & 0 \\
    \end{bmatrix}
    \\
    \frac{1}{2}R_1 \to R_1                                                                      \\
    \begin{bmatrix}
      1 & -2 & 0 \\
      0 & 0  & 0 \\
    \end{bmatrix}
    \\
    x_1 -2x_2 = 0                                                                               \\
    x_2  = x_2                                                                                  \\
    \\
    x_1 = 2x_2                                                                                  \\
    x_2 = x_2                                                                                   \\
    \mbold{x} = \begin{bmatrix} x_1 \\ x_2 \end{bmatrix} = \begin{bmatrix} 2x_2 \\ x_2 \end{bmatrix}  = x_2
    \begin{bmatrix} 2 \\ 1 \end{bmatrix}                                                        \\
    \therefore \text{ the basis of the eigenspace of $A$ with $\lambda = 5$ is } \left\{ \begin{bmatrix} 2 \\ 1 \end{bmatrix}  \right\}
  \end{align*}
}

\chapter{The Characteristic Equation}

\thm{}{
  Let $A$ be an $n\times n$ matrix. Then $A$ is only invertible if and only if:
  \begin{itemize}
    \item The number 0 is not an eigenvalue of $A$
    \item The determinant of $A$ is not zero
  \end{itemize}
}

Therefore the updated properties of determinants are:
\thm{}{
  Let $A$ and $B$ be $n\times n$ matrices
  \begin{enumerate}
    \item $A$ is invertible if and only if $\det A \neq 0$
    \item $\det AB = \left( \det A \right) \left( \det B \right)  $
    \item $\det A^{T} = \det A$
    \item If $A$ is triangular, then $\det A$ is the product of the entries on the main diagonal of $A$
    \item A row replacement operation on $A$ does not change the determinant of $A$. A row interchange changes the
          sign of the determinant. A row scaling also scales the determinant by the same factor.
  \end{enumerate}
}


Useful information about the eigenvalues of a square matrix $A$ is found in a special scalar equation called the
characteristic equation of $A$.

\qs{}{
  Find the eigenvalues of $A = \begin{bmatrix} 2 & 3 \\ 3 & -6 \end{bmatrix} $
}

\sol{
  We must find all scalars $\lambda $ such that the matrix equation
  \[
    \left( A - \lambda I \right)\mbold{x}  = \mbold{0}
  \]
  Has the non-trivial solution. By the invertible matrix theorem, this is the same as finding all the scalars $\lambda $
  where the matrix $A - \lambda I$ is non-invertible, i.e. $\det \left( A - \lambda I \right) = 0 $. Therefore
  \begin{align*}
    A - \lambda I & = \begin{bmatrix} 2 & 3 \\ 3 & 6 \end{bmatrix} - \begin{bmatrix} \lambda & 0 \\ 0 & \lambda  \end{bmatrix} \\
                  & = \begin{bmatrix}
                        2 - \lambda & 3            \\
                        3           & -6 - \lambda
                      \end{bmatrix}                                                                               \\
    \\
    \det \left( A - \lambda I \right)  = 0 \therefore                                                                          \\
    \det \left( \begin{bmatrix}
                  2 - \lambda & 3            \\
                  3           & -6 - \lambda\end{bmatrix} \right) = 0                                                            \\
    \left( 2- \lambda  \right) \left( -6 - \lambda  \right) - 9   = 0                                                          \\
    \lambda^2 + 4 \lambda  - 21 = 0                                                                                            \\
    \left( \lambda - 7 \right)  \left( \lambda - 3 \right)  =  0                                                               \\
    \lambda  = 7                                                                                                               \\
    \lambda  = 3
  \end{align*}
}

\dfn{Characteristic Equation}{
  A scalar $\lambda $ is an eigenvalue of an $n \times n$ matrix $A$ if and only if $\lambda $ satisfies the
  characteristic equation
  \[
    \det \left( A - \lambda I \right)  = 0
  \]
}

\section{Characteristic Polynomial}

The characteristic polynomial of a matrix $A$ is a polynomial of degree $n$ in the variable $\lambda $, where $n$ is the
the size of the matrix $A$. The characteristic polynomial of $A$ is defined as:
\[
  \lambda^{n} - \left( \text{trace} A \right) \lambda^{n-1} + \left( \text{trace} A \right) \lambda^{n-2} + \ldots + \left( -1 \right)^{n} \det A
\]

\section{Similarity}

\dfn{Similarity}{

  If $A$ and $B$ are $n\times n$ matrices, then $A$ is similar to $B$ if there is an invertible matrix $P$ such that
  $P^{-1}AP = B $. If $A$ is similar to $B$, then $B$ is also similar to $A$, therefore $A$ and $B$ are similar.
}

\thm{}{
  If $n\times n$ matrices of $A$ and $B$ are similar, then they have the same characteristic polynomial and hence have
  the same eigenvalues, with the same multiplicities. Therefore
  \[
    \det \left( A - \lambda I \right) = \det \left( B - \lambda I \right)
  \]

}

\begin{myproof}
  If $B = P^{-1}AP$, Then
  \begin{align*}
    B - \lambda I                     & = P^{-1}AP  - \lambda I                                      \\
    B - \lambda I                     & = P^{-1}AP - \lambda P^{-1}P                                 \\
                                      & = P^{-1} \left( AP - \lambda P \right)                       \\
                                      & = P^{-1} \left( AP - \lambda PI \right)                      \\
                                      & = P^{-1} \left( A - \lambda I \right)P                       \\
    \text{The determinants of the two matrices are equal, then:}                                     \\
    \det \left( B - \lambda I \right) & = \det \left( P^{-1} \left( A - \lambda I \right) P  \right) \\
                                      & = \det P^{-1} \det \left( A - \lambda I \right) \det P       \\
                                      & = \det P^{-1} \det P \det \left( A - \lambda I \right)       \\
                                      & = 1 \det \left( A - \lambda I \right)                        \\
    \det \left( B - \lambda I \right) & = \det \left( A - \lambda I \right)                          \\
  \end{align*}
\end{myproof}


\section{Exercises}

\qs{}{
  Find the characteristic polynomial and the eigenvalues of the
  matrices
  \begin{enumerate}
    \item
          \[
            \begin{bmatrix} 2 & 7 \\ 7 & 2 \end{bmatrix}
          \]
  \end{enumerate}
}

\sol{
  \begin{enumerate}
    \item
          \begin{align*}
            \det \left( A - \lambda I \right)  = 0                     \\
            \det \begin{bmatrix}
                   2 - \lambda & 7           \\
                   7           & 2 - \lambda
                 \end{bmatrix} = 0                             \\
            \left( 2 - \lambda  \right)^2 -49 = 0                      \\
            \lambda^2 -4 \lambda + 4 - 49 = 0                          \\
            \left( \lambda - 9 \right) \left( \lambda + 5 \right)  = 0 \\
            \lambda = 9                                                \\
            \lambda = -4
          \end{align*}
  \end{enumerate}
}


\chapter{Diagonalization}

In many cases the eigenvalue-eigenvector information contained in a matrix $A$ can be displayed in the factorization $A
  = PDP^{-1}$, where $D$ is a diagonal matrix. This makes it easy compute $A^{k}$ for large values of $k$.

\ex{}{
  If $D = \begin{bmatrix} 5 & 0 \\ 0 & 3 \end{bmatrix} $, then:
  \begin{align*}
    D^2   & ^= \begin{bmatrix} 5 & 0 \\ 0 & 3 \end{bmatrix} \times \begin{bmatrix} 5 & 0 \\ 0 & 3 \end{bmatrix} \\
          & = \begin{bmatrix}
                5^2 & 0   \\
                0   & 3^2
              \end{bmatrix}                                                                                    \\
    D^3   & = \begin{bmatrix}
                5^2 & 0   \\
                0   & 3^2
              \end{bmatrix} \begin{bmatrix} 5 & 0 \\ 0 & 3 \end{bmatrix}                                        \\
          & = \begin{bmatrix}
                5^3 & 0   \\
                0   & 3^3
              \end{bmatrix}                                                                                    \\
    \text{Therefore generally}                                                                                  \\
    D^{k} & = \begin{bmatrix}
                5^{k} & 0     \\
                0     & 3^{k}
              \end{bmatrix}
  \end{align*}
}

\thm{Diagonalization Theorem}{
  An $n\times n$ matrix $A$ is diagonalizable if and only if $A$ has $n$ linearly independent eigenvectors.

  $A = PDP^{-1}$, with the diagonal matrix $D$, if an only if the columns of $P$ are $n$ linearly independent
  eigenvectors of $A$. In this case the diagonal entries of $D$ are the eigenvalues of $A$ that correspond, respectively
  to the eigenvectors in $P$
}

\ex{}{
  \qs{}{
    Diagonalize the following matrix if possible:
    \[
      A = \begin{bmatrix}
        1  & 3  & 3  \\
        -3 & -5 & -3 \\
        3  & 3  & 1
      \end{bmatrix}
    \]
  }

  \sol{
    To do this we must complete the following steps:
    \begin{enumerate}
      \item Find the eigenvalues of $A$
      \item Find three linearly independent eigenvectors of $A$
      \item Construct $P$ from the vectors found in step 2
      \item Construct $D$ from the eigenvalues found in step 1
    \end{enumerate}

    Therefore\\
    \begin{enumerate}
      \item
            \begin{align*}
              \det \left( A - \lambda I \right)  = 0 \\
            \end{align*}
            Eigenvalues: $1, -2, -2$
      \item
            $\lambda  = -2$
            \begin{align*}
              A + 2 I  =  \begin{bmatrix}
                            3  & 3  & 3  \\
                            -3 & -3 & -3 \\
                            3  & 3  & 3
                          \end{bmatrix}   \\
              -1R_1 - R_2  \rightarrow R_2 \\
              \begin{bmatrix}
                3 & 3 & 3 & 0 \\
                0 & 0 & 0 & 0 \\
                3 & 3 & 3 & 0 \\
              \end{bmatrix}
              \\
              R_1 - R_3  \rightarrow R_3   \\
              \begin{bmatrix}
                3 & 3 & 3 & 0 \\
                0 & 0 & 0 & 0 \\
                0 & 0 & 0 & 0 \\
              \end{bmatrix}
              \\
              \frac{1}{3}R_1 \to R_1       \\
              \begin{bmatrix}
                1 & 1 & 1 & 0 \\
                0 & 0 & 0 & 0 \\
                0 & 0 & 0 & 0 \\
              \end{bmatrix}
              \\
              x_1 = -x_2 - x_3             \\
              x_2 = x_2                    \\
              x_3 = x_2                    \\
              \\
              \mbold{x} = \begin{bmatrix} -x_2 - x_3 \\ x_2 \\ x_3 \end{bmatrix} = x_2 \begin{bmatrix} -1 \\ 1 \\ 0
                                                                                       \end{bmatrix} + x_3 \begin{bmatrix} -1 \\ 0 \\ 1 \end{bmatrix}
            \end{align*}
            $\lambda = 1$
            \begin{align*}
              A - I = \begin{bmatrix}
                        0  & 3  & 3  \\
                        -3 & -6 & -3 \\
                        3  & 3  & 0
                      \end{bmatrix}       \\
              R_1 \leftrightarrow R_2      \\
              \begin{bmatrix}
                -3 & -6 & -3 & 0 \\
                0  & 3  & 3  & 0 \\
                3  & 3  & 0  & 0 \\
              \end{bmatrix}
              \\
              R_2 \leftrightarrow R_3      \\
              \begin{bmatrix}
                -3 & -6 & -3 & 0 \\
                3  & 3  & 0  & 0 \\
                0  & 3  & 3  & 0 \\
              \end{bmatrix}
              \\
              -1R_1 - R_2  \rightarrow R_2 \\
              \begin{bmatrix}
                -3 & -6 & -3 & 0 \\
                0  & 3  & 3  & 0 \\
                0  & 3  & 3  & 0 \\
              \end{bmatrix}
              \\
              R_2 - R_3  \rightarrow R_3   \\
              \begin{bmatrix}
                -3 & -6 & -3 & 0 \\
                0  & 3  & 3  & 0 \\
                0  & 0  & 0  & 0 \\
              \end{bmatrix}
              \\
              -2R_2 - R_1  \rightarrow R_1 \\
              \begin{bmatrix}
                3 & 0 & -3 & 0 \\
                0 & 3 & 3  & 0 \\
                0 & 0 & 0  & 0 \\
              \end{bmatrix}
              \\
              \frac{1}{3}R_1 \to R_1       \\
              \begin{bmatrix}
                1 & 0 & -1 & 0 \\
                0 & 3 & 3  & 0 \\
                0 & 0 & 0  & 0 \\
              \end{bmatrix}
              \\
              \frac{1}{3}R_2 \to R_2       \\
              \begin{bmatrix}
                1 & 0 & -1 & 0 \\
                0 & 1 & 1  & 0 \\
                0 & 0 & 0  & 0 \\
              \end{bmatrix}
              \\
              x_1 = x_3                    \\
              x_2 = -x_3                   \\
              x_3 = x_3                    \\
              \\
              \mbold{x} = \begin{bmatrix} x_3 \\ -x_3 \\ x_3 \end{bmatrix} = x_3 \begin{bmatrix} 1 \\ -1 \\ 1 \end{bmatrix}
            \end{align*}
            Our linearly dependent eigenvectors are therefore
            \[
              \{ \begin{bmatrix} -1 \\ 1 \\ 0  \end{bmatrix}, \begin{bmatrix} -1 \\ 0 \\ 1 \end{bmatrix}, \begin{bmatrix} 1 \\ -1
                \\ 1\end{bmatrix}     \}
            \]
      \item
            Therefore our $P$:
            \[
              P = \begin{bmatrix}
                -1 & 1 & 1  \\
                1  & 0 & -1 \\
                0  & 1 & 1  \\
              \end{bmatrix}
            \]
      \item
            We start the matrix with the columns from the eigenspace from the repeated eigenvalue $-2$, therefore we must list the entries in
            $D$ in the same order.
            \[
              D = \begin{bmatrix}
                -2 & 0  & 0 \\
                0  & -2 & 0 \\
                0  & 0  & 1
              \end{bmatrix}
            \]
            In checking our answers we check if the sides of the equation below are equal
            \[
              AP = PD
            \]
    \end{enumerate}
  }
}


\thm{}{
  An $n\times n$ matrix with $n$ distinct eigenvalues is diagonalizable
}

\ex{}{
  \qs{}{
    Determine if the following matrix is diagonalizable
    \[
      A = \begin{bmatrix}
        5 & -8 & 1  \\
        0 & 0  & 7  \\
        0 & 0  & -2
      \end{bmatrix}
    \]
  }

  \sol{
    \begin{align*}
      \det \left( A - \lambda I \right) = 0                                                                                                         \\
      \begin{bmatrix}
        5 & -8 & 1  \\
        0 & 0  & 7  \\
        0 & 0  & -2
      \end{bmatrix} - \begin{bmatrix} \lambda & 0 & 0 \\ 0 & \lambda & 0 \\ 0 & 0 & \lambda  \end{bmatrix}                                          \\
      \begin{bmatrix}
        5 - \lambda & -8       & 1            \\
        0           & -\lambda & 7            \\
        0           & 0        & -2 - \lambda
      \end{bmatrix}                                                                                                         \\
      \det \left( A - \lambda I \right) & = \displaystyle\sum_{j=1}^{n} a_{ij}C_{ij}                                                                \\
                                        & = 0 + 0 + \left( -1 \right)^{3+3} \left( -2 - \lambda \begin{vmatrix} 5 -
                                                                                                    \lambda & -8 \\ 0 & - \lambda\end{vmatrix}  \right) \\
                                        & = \lambda ^3 -3
      \lambda ^2 - 10 \lambda
      \\
      \lambda                           & = 5                                                                                                       \\
      \lambda                           & = -3                                                                                                      \\
    \end{align*}
  }
}

\section{Exercises}

\qs{}{
  Let $A = PDP^{-1}$ and compute $A^{4}$
  \[
    P = \begin{bmatrix} 5 & 7 \\ 2 & 3 \end{bmatrix} , D = \begin{bmatrix} 2 & 0 \\ 0 & 1 \end{bmatrix}
  \]
}

\sol{
  \begin{align*}
    A   & = PDP^{-1}                     \\
    A^2 & = PDP^{-1} \times PDP^{-1}     \\
    P \times P^{-1} = I                  \\
    A^2 & = PD^2IP^{-1}                  \\
    A^4 & = PD^2P^{-1} \times PD^2P^{-1} \\
        & = PD^{4}P^{-1}                 \\
  \end{align*}
}

\qs{}{
  The matrix $A$ is factored in the form $PDP^{-1}$, Use the Diagonalization Theorem to find the eigenvalues and the basis
  of each eigenspace
  \[
    A = \begin{bmatrix} 2 & 2& 1 \\ 1 & 3 & 1 \\ 1 & 2& 2 \end{bmatrix} = \begin{bmatrix} 1  & 1 & 2 \\ 1 & 0 & -1 \\ 1 &
                -1 & 0\end{bmatrix} \begin{bmatrix} 5 & 0 & 0 \\ 0 & 1& 0 \\ 0 & 0 & 1 \end{bmatrix} \begin{bmatrix} \frac{1}{4} &
                \frac{1}{2} & \frac{1}{4} \\ \frac{1}{4} & \frac{1}{2} & -\frac{3}{4} \\ \frac{1}{4} & -\frac{1}{2} & \frac{1}{4}\end{bmatrix}
  \]
}

\sol{
  By the Diagonalization theory, the eigenvalues of $A$ are the diagonal entries of $D$, and the corresponding basis of
  the eigenspace of each eigenvalue is the respective column of $P$. Therefore the eigenvalues are $5, 1, 1$ and the
  basis for the eigenspace $\lambda = 5$:
  \[
    \left\{     \begin{bmatrix} 1 \\ 1\\ 1 \end{bmatrix}
    \right\}   \]
  And the basis for the eigenspace $\lambda = 1$:
  \[
    \left\{     \begin{bmatrix} 1 \\ 0\\ -1 \end{bmatrix}, \begin{bmatrix} 2 \\ -1\\ 0 \end{bmatrix}
    \right\}
  \]
}

\qs{}{
  Diagonalize the matrices below
  \begin{enumerate}
    \item
          \[
            \begin{bmatrix} 1 & 0 \\ 6 & -1 \end{bmatrix}
          \]
  \end{enumerate}
}

\sol{
  \begin{enumerate}
    \item
          \begin{align*}
            \begin{bmatrix} 1 & 0 \\ 6 & -1 \end{bmatrix}                             \\
            \text{The eigenvalues are on the diagonal as this is a triangular matrix} \\
            \lambda = 1                                                               \\
            A - I                                                                     \\
            \begin{bmatrix} 0 & 0 \\ 6 & -2 \end{bmatrix}                             \\
            R_1 \leftrightarrow R_2                                                   \\
            \begin{bmatrix}
              6 & -2 & 0 \\
              0 & 0  & 0 \\
            \end{bmatrix}
            \\
            \frac{1}{6}R_1 \to R_1                                                    \\
            \begin{bmatrix}
              1 & \frac{-1}{3} & 0 \\
              0 & 0            & 0 \\
            \end{bmatrix}
            \\
            x_1 = \frac{1}{3}x_2                                                      \\
            x_2 = x_2                                                                 \\
            \mbold{x} = \begin{bmatrix} \frac{1}{3}x_2 \\ x_2 \end{bmatrix} = x_2 \begin{bmatrix} \frac{1}{3} \\ 1 \end{bmatrix}
            \\
            \\
            \lambda = -1                                                              \\
            A + I                                                                     \\
            \begin{bmatrix}
              2 & 0 \\
              6 & 0 \\
            \end{bmatrix}                                                            \\
            3R_1 - R_2  \rightarrow R_2                                               \\
            \begin{bmatrix}
              2 & 0 & 0 \\
              0 & 0 & 0 \\
            \end{bmatrix}
            \\
            \frac{1}{2}R_1 \to R_1                                                    \\
            \begin{bmatrix}
              1 & 0 & 0 \\
              0 & 0 & 0 \\
            \end{bmatrix}
            \\
            x_1 = 0                                                                   \\
            x_2 = x_2                                                                 \\
            \mbold{x} = x_2\begin{bmatrix} 0  \\ 1 \end{bmatrix}
            \\
            P = \begin{bmatrix}
                  \frac{1}{3} & 0 \\
                  1           & 1
                \end{bmatrix}                                                       \\
            D = \begin{bmatrix}
                  1 & 0  \\
                  0 & -1
                \end{bmatrix}
          \end{align*}
  \end{enumerate}
}

\chapter{Eigenvectors and Linear Transformations}

\section{The Matrix of a Linear Transformation}

Let $V$ be an $n$-dimensional matrix, and $W$ be an $m$-dimensional vector space, and let $T$ be any linear
transformation from $V$ to $W$. To associate a matrix with $T$, choose ordered bases $\mathcal{B}$ and $\mathcal{C}$ for $V$ and
$W$, respectively.

\noindent Given any $\mbold{x}$ in $V$ the coordinates vector $ \left[ \mbold{x} \right]_{\mathcal{B}} $ is in $\mathbb{R}^{n}$
and the coordinate vector of its image, $ \left[ T \left( \mbold{x} \right)  \right]_{\mathcal{C}} $, is in
$\mathbb{R}^{m}$. The connection between the coordinate vectors $ \left[ \mbold{x} \right]_{\mathcal{B}} $ and $ \left[ T \left( \mbold{x}
    \right)  \right]_{\mathcal{C}} $ can be found. Let $\{\mbold{b}_1, \ldots, \mbold{b}_n\} $ be the basis $\mathcal{B}$
for $V$. If $\mbold{x} = r_1\mbold{b}_1 + \ldots + r_n \mbold{b}_n$, then:
\[
  \left[ \mbold{x} \right]_{\mathcal{B}}  = \begin{bmatrix} r_1 \\ \vdots \\ r_n \end{bmatrix}
\]
And therefore:
\[
  T \left( \mbold{x} \right) = T \left( r_1\mbold{b}_1 + \ldots + r_n \mbold{b}_n \right)  = r_1T \left( \mbold{b}_1
  \right) + \ldots + r_n T \left( \mbold{b}_n \right)
\]

\chapter{Discrete Dynamical Systems}
\section{Introduction}
Eigenvalues and vectors can be used to understand the long term behaviour / evolution of a dynamical system, described
by the difference equation:
\[
  \mbold{x}_{k+1} = A \mbold{x}_k
\]
This equation can be used to model population movement, Markov chains, and other systems. The matrix $A$ is called the
stage matrix of the system.

Assuming $A$ is diagonalizable, with $n$ linearly independent eigenvectors, $\mbold{v}_1, \ldots, \mbold{v}_n$, and
corresponding eigenvalues, $\lambda_1, \ldots, \lambda_n$, we assume that the eigenvalues are arranged so that $ \mid
  \lambda _1 \mid \geq  \mid \lambda _2  \mid \geq \ldots \geq \lambda_n$. Since as $\{\mbold{v}_1, \ldots,
  \mbold{v}_n\} $ is set of linearly independent vectors in $\mathbb{R}^{n}$, it is a basis for $\mathbb{R}^{n}$ as any
vector in $\mathbb{R}^{n}$ can be written as a linear combination of the vectors in $\{\mbold{v}_1, \ldots,
  \mbold{v}_n\} $. Therefore since $\mbold{x}_0$ is in $\mathbb{R}^{n}$:
\[
  \mbold{x}_0 = c_1 \mbold{v}_1 + \ldots + c_n \mbold{v}_n
\]
This is called the \textit{eigenvector decomposition} of $\mbold{x}_0$ and determines what happens to the sequence
$\{\mbold{x}_k\} $. The sequence is:
\begin{align*}
  \mbold{x}_1 = A \mbold{x}_0 & = c_1 A\mbold{v}_1 + c_2 A \mbold{v}_2 + \ldots + c_n A \mbold{v}_n                           \\
                              & = c_1 \lambda_1 \mbold{v}_1 + c_1 \lambda_2 \mbold{v}_2 + \ldots + c_n \lambda _n \mbold{v}_n \\
\end{align*}
And generally:
\[
  \mbold{x}_k = c_1 \lambda^{k}_1 \mbold{v}_1 + \ldots + c_n \lambda ^{k}_n \mbold{v}_n, \quad \left( k= 0,1,2, \ldots \right)
\]

\section{Exercises}

\qs{}{
  In a certain town, 30 percent of the married women get divorced each year  and 20 percent of the single women get married
  each year. There are currently 8000 married women and 200 single women.

  \noindent Assuming the total  population of women remains constant, how many married women and how many single women
  will there be after on year? Two years? Can you predict what would happen in the long run, say in a hundred years?
}


\sol{
  First we express the first year as an equation:
  \begin{align*}
    m_k & = 0.7 m_k + 0.2 s_k       \\
    s_k & = 0.3 m_k + 0.8 s_k       \\
    \\
    \mbold{x}_k = A \mbold{x}_{k-1} \\
    \begin{bmatrix} m_k \\ s_k \end{bmatrix} = \begin{bmatrix} 0.7 & 0.2 \\ 0.3 & 0.8 \end{bmatrix}
  \end{align*}
  $x_0 = \begin{bmatrix} 8000 \\ 2000 \end{bmatrix} $. We then need to see if $A$ is diagonalizable, and if so, find the eigenvalues and eigenvectors.
  \begin{align*}
    \det \left( A - \lambda I \right) = 0                                                                            \\
    \begin{bmatrix} 0.7 & 0.2 \\ 0.3 & 0.8 \end{bmatrix} - \begin{bmatrix} \lambda & 0 \\ 0 & \lambda  \end{bmatrix} \\
    \begin{bmatrix}
      0.7 - \lambda & 0.2           \\
      0.3           & 0.8 - \lambda
    \end{bmatrix}                                                                                    \\
    \left( 0.7 - \lambda  \right) \left( 0.8 - \lambda  \right) - \left( 0.2 \times 0.3 \right) = 0                  \\
    \lambda ^2 - 1.5 \lambda + 0.5 = 0                                                                               \\
    \left( \lambda -1 \right)  \left( 2 \lambda - 1 \right)                                                          \\
    \lambda = 1, 0.5                                                                                                 \\
    \\
    \lambda = 1                                                                                                      \\
    \begin{bmatrix} 0.7 & 0.2 \\ 0.3 & 0.8 \end{bmatrix} - \begin{bmatrix} 1 & 0 \\ 0 & 1 \end{bmatrix}              \\
    \begin{bmatrix}
      -0.3 & 0.2  \\
      0.3  & -0.2
    \end{bmatrix}                                                                                                   \\
    -1R_1 - R_2  \rightarrow R_2                                                                                     \\
    \begin{bmatrix}
      \frac{-3}{10} & \frac{1}{5} \\
      0             & 0           \\
    \end{bmatrix}
    \\
    \frac{-10}{3}R_1 \to R_1                                                                                         \\
    \begin{bmatrix}
      1 & \frac{-2}{3} \\
      0 & 0            \\
    \end{bmatrix}
    x_1 = \frac{2}{3}x_2                                                                                             \\
    x_2 = x_2                                                                                                        \\
    \mbold{x} = x_2\begin{bmatrix} \frac{2}{3} \\ 1 \end{bmatrix}                                                    \\
    \left\{ \begin{bmatrix} \frac{2}{3} \\ 1 \end{bmatrix} \right\}
  \end{align*}

  \begin{align*}
    \lambda = 0.5                                                                                            \\
    \begin{bmatrix} 0.7 & 0.2 \\ 0.3 & 0.8 \end{bmatrix} - \begin{bmatrix} 0.5 & 0 \\ 0 & 0.5  \end{bmatrix} \\
    \begin{bmatrix}
      0.2 & 0.2 \\
      0.3 & 0.3
    \end{bmatrix}                                                                                           \\
    \frac{3}{2}R_1 - R_2  \rightarrow R_2                                                                    \\
    \begin{bmatrix}
      \frac{1}{5} & \frac{1}{5} \\
      0           & 0           \\
    \end{bmatrix}
    \\
    5R_1 \to R_1                                                                                             \\
    \begin{bmatrix}
      1 & 1 \\
      0 & 0 \\
    \end{bmatrix}
    \\
    x_1 = -x_2                                                                                               \\
    x_2 = x_2                                                                                                \\
    \mbold{x} = x_2 \begin{bmatrix} -1 \\ 1 \end{bmatrix}                                                    \\
    \left\{ \begin{bmatrix} -1 \\ 1 \end{bmatrix} \right\}                                                   \\
    P = \begin{bmatrix} \frac{2}{3} & -1 \\ 1 & 1 \end{bmatrix}                                              \\
    D = \begin{bmatrix} 1 & 0 \\ 0 & 0.5 \end{bmatrix}
  \end{align*}

  Since the columns of $P$ are linearly independent and in $\mathbb{R}^2$, they form a basis for $\mathbb{R}^2$. As
  $\mbold{x}_0$ exists in $\mathbb{R}^2$, therefore:
  \[
    \mbold{x}_0 = c_1 \mbold{v}_1 + c_2 \mbold{v}_2
  \]
  Where $\mbold{v}_1 = \begin{bmatrix} \frac{2}{3} \\ 1 \end{bmatrix} $, and $\mbold{v}_2 = \begin{bmatrix} -1 \\  1
    \end{bmatrix} $ \\

  \noindent  Next we solve for $c_1$ and $c_2$ in the linear combination:
  \begin{align*}
    \mbold{x}_0 & = c_1 \lambda^{0}_1 \mbold{v}_1 + c_2 \lambda^{0}_2 \mbold{v}_2                                  \\
                & = c_1 \begin{bmatrix} \frac{2}{3} \\ 1 \end{bmatrix} + c_2 \begin{bmatrix} -1 \\ 1 \end{bmatrix} \\
  \end{align*}
  Giving us the equation
  \[
    \begin{bmatrix}
      \frac{2}{3} & -1 \\
      1           & 1
    \end{bmatrix} = \begin{bmatrix} 8000 \\ 200 \end{bmatrix}
  \]
  \begin{align*}
    \frac{3}{2}R_1 - R_2  \rightarrow R_2 \\
    \begin{bmatrix}
      \frac{2}{3} & -1           & 8000            \\
      0           & \frac{-5}{2} & \frac{11800}{1} \\
    \end{bmatrix}
    \\
    R_1 - R_2  \rightarrow R_2            \\
    \begin{bmatrix}
      \frac{2}{3} & -1          & 8000             \\
      0           & \frac{5}{2} & \frac{-11800}{1} \\
    \end{bmatrix}
    \\
    R_1 - R_2  \rightarrow R_2            \\
    \begin{bmatrix}
      \frac{2}{3} & -1           & 8000            \\
      0           & \frac{-5}{2} & \frac{11800}{1} \\
    \end{bmatrix}
    \\
    R_1 - R_2  \rightarrow R_2            \\
    \begin{bmatrix}
      \frac{2}{3} & -1          & 8000             \\
      0           & \frac{5}{2} & \frac{-11800}{1} \\
    \end{bmatrix}
    \\
    R_1 - R_2  \rightarrow R_2            \\
    \begin{bmatrix}
      \frac{2}{3} & -1           & 8000            \\
      0           & \frac{-5}{2} & \frac{11800}{1} \\
    \end{bmatrix}
    \\
    R_1 - R_2  \rightarrow R_2            \\
    \begin{bmatrix}
      \frac{2}{3} & -1          & 8000             \\
      0           & \frac{5}{2} & \frac{-11800}{1} \\
    \end{bmatrix}
    \\
    R_1 - R_2  \rightarrow R_2            \\
    \begin{bmatrix}
      \frac{2}{3} & -1           & 8000            \\
      0           & \frac{-5}{2} & \frac{11800}{1} \\
    \end{bmatrix}
    \\
    R_1 - R_2  \rightarrow R_2            \\
    \begin{bmatrix}
      \frac{2}{3} & -1          & 8000             \\
      0           & \frac{5}{2} & \frac{-11800}{1} \\
    \end{bmatrix}
    \\
    R_1 - R_2  \rightarrow R_2            \\
    \begin{bmatrix}
      \frac{2}{3} & -1           & 8000            \\
      0           & \frac{-5}{2} & \frac{11800}{1} \\
    \end{bmatrix}
    \\
    R_1 - R_2  \rightarrow R_2            \\
    \begin{bmatrix}
      \frac{2}{3} & -1          & 8000             \\
      0           & \frac{5}{2} & \frac{-11800}{1} \\
    \end{bmatrix}
    \\
    R_1 - R_2  \rightarrow R_2            \\
    \begin{bmatrix}
      \frac{2}{3} & -1           & 8000            \\
      0           & \frac{-5}{2} & \frac{11800}{1} \\
    \end{bmatrix}
    \\
    R_1 - R_2  \rightarrow R_2            \\
    \begin{bmatrix}
      \frac{2}{3} & -1          & 8000             \\
      0           & \frac{5}{2} & \frac{-11800}{1} \\
    \end{bmatrix}
    \\
    R_1 - R_2  \rightarrow R_2            \\
    \begin{bmatrix}
      \frac{2}{3} & -1           & 8000            \\
      0           & \frac{-5}{2} & \frac{11800}{1} \\
    \end{bmatrix}
    \\
    R_1 - R_2  \rightarrow R_2            \\
    \begin{bmatrix}
      \frac{2}{3} & -1          & 8000             \\
      0           & \frac{5}{2} & \frac{-11800}{1} \\
    \end{bmatrix}
    \\
    R_1 - R_2  \rightarrow R_2            \\
    \begin{bmatrix}
      \frac{2}{3} & -1           & 8000            \\
      0           & \frac{-5}{2} & \frac{11800}{1} \\
    \end{bmatrix}
    \\
    R_1 - R_2  \rightarrow R_2            \\
    \begin{bmatrix}
      \frac{2}{3} & -1          & 8000             \\
      0           & \frac{5}{2} & \frac{-11800}{1} \\
    \end{bmatrix}
  \end{align*}

  \begin{align*}
    \\
    R_1 - R_2  \rightarrow R_2                                                             \\
    \begin{bmatrix}
      \frac{2}{3} & -1           & 8000            \\
      0           & \frac{-5}{2} & \frac{11800}{1} \\
    \end{bmatrix}
    \\
    R_1 - R_2  \rightarrow R_2                                                             \\
    \begin{bmatrix}
      \frac{2}{3} & -1          & 8000             \\
      0           & \frac{5}{2} & \frac{-11800}{1} \\
    \end{bmatrix}
    \\
    R_1 - R_2  \rightarrow R_2                                                             \\
    \begin{bmatrix}
      \frac{2}{3} & -1           & 8000            \\
      0           & \frac{-5}{2} & \frac{11800}{1} \\
    \end{bmatrix}
    \\
    R_1 - R_2  \rightarrow R_2                                                             \\
    \begin{bmatrix}
      \frac{2}{3} & -1          & 8000             \\
      0           & \frac{5}{2} & \frac{-11800}{1} \\
    \end{bmatrix}
    \\
    R_1 - R_2  \rightarrow R_2                                                             \\
    \begin{bmatrix}
      \frac{2}{3} & -1           & 8000            \\
      0           & \frac{-5}{2} & \frac{11800}{1} \\
    \end{bmatrix}
    \\
    \frac{2}{5}R_2 - R_1  \rightarrow R_1                                                  \\
    \begin{bmatrix}
      \frac{-2}{3} & 0            & \frac{-3280}{1} \\
      0            & \frac{-5}{2} & \frac{11800}{1} \\
    \end{bmatrix}
    \\
    \frac{-3}{2}R_1 \to R_1                                                                \\
    \begin{bmatrix}
      1 & 0            & \frac{4920}{1}  \\
      0 & \frac{-5}{2} & \frac{11800}{1} \\
    \end{bmatrix}
    \\
    \frac{-2}{5}R_2 \to R_2                                                                \\
    \begin{bmatrix}
      1 & 0 & \frac{4920}{1}  \\
      0 & 1 & \frac{-4720}{1} \\
    \end{bmatrix}
    \\
    R_2 - R_1  \rightarrow R_1                                                             \\
    \begin{bmatrix}
      -1 & 0 & \frac{-4920}{1} \\
      0  & 1 & \frac{-4720}{1} \\
    \end{bmatrix}
    \\
    x_2   -1R_1 \to R_1                                                                    \\
    \begin{bmatrix}
      1 & 0 & \frac{4920}{1}  \\
      0 & 1 & \frac{-4720}{1} \\
    \end{bmatrix}                                                                \\
    c_1 = 4920                                                                             \\
    c_2 = -4720                                                                            \\
    \\
    x_k & = 4920 \left( 1 \right)^{k}\mbold{v}_1 + -4720 \left( 0.5 \right)^{k}\mbold{v}_2 \\
  \end{align*}



}

\qs{}{
  Construct a stage matrix model for an animal speices that has two life stages, juvenile and adule. Suppsoe
}

\sol{
  \begin{enumerate}
    \item   \begin{align*}
            j_k = 1.6 a_k           \\
            a_k = 0.3 j_k + 0.8 a_k \\
            \\
            A = \begin{bmatrix}
                  0   & 1.6 \\
                  0.3 & 0.8
                \end{bmatrix}
          \end{align*}

    \item
          \begin{align*}
            \det \left( A - \lambda I \right)  = 0                                   \\
            \begin{bmatrix}
              0   & 1.6 \\
              0.3 & 0.8
            \end{bmatrix} - \begin{bmatrix} \lambda & 0 \\ 0 & \lambda \end{bmatrix} \\
            \begin{bmatrix}
              - \lambda & 1.6           \\
              0.3       & 0.8 - \lambda
            \end{bmatrix}                                                \\
            \left( - \lambda  \right) \left( 0.8 - \lambda  \right)  - 0.48 = 0      \\
            \lambda^2 - 0.8 \lambda - 0.48 = 0                                       \\
            \lambda = 1.2, -0.4                                                      \\
            \\
            \lambda = 1.2                                                            \\
            \begin{bmatrix}
              -1.2 & 1.6  \\
              0.3  & -0.4
            \end{bmatrix}                                                           \\
            \frac{-1}{4}R_1 - R_2  \rightarrow R_2                                   \\
            \begin{bmatrix}
              \frac{-6}{5} & \frac{8}{5} \\
              0            & 0           \\
            \end{bmatrix}
            \\
            \frac{-5}{6}R_1 \to R_1                                                  \\
            \begin{bmatrix}
              1 & \frac{-4}{3} \\
              0 & 0            \\
            \end{bmatrix}
            \\
            \\
            x = x_2\begin{bmatrix} \frac{4}{3} \\ 1 \end{bmatrix}                    \\
            \left\{ \begin{bmatrix} \frac{4}{3} \\ 1 \end{bmatrix} \right\}
            \\
            \lambda = -0.4                                                           \\
            \begin{bmatrix}
              0   & 1.6 \\
              0.3 & 0.8
            \end{bmatrix} - \begin{bmatrix} -0.4 & 0 \\ 0 & -0.4 \end{bmatrix}       \\
            \begin{bmatrix}
              0.4 & 1.6 \\
              0.3 & 1.2
            \end{bmatrix}                                                           \\
            \frac{3}{4}R_1 - R_2  \rightarrow R_2                                    \\
            \begin{bmatrix}
              \frac{2}{5} & \frac{8}{5} \\
              0           & 0           \\
            \end{bmatrix}
            \\
            \frac{5}{2}R_1 \to R_1                                                   \\
            \begin{bmatrix}
              1 & 4 \\
              0 & 0 \\
            \end{bmatrix}
            \\
            \mbold{x} = x_2 \begin{bmatrix} - 4\\ 1 \end{bmatrix}                    \\
            \left\{ \begin{bmatrix} - 4\\ 1 \end{bmatrix} \right\}
          \end{align*}

          \begin{align*}
            P = \begin{bmatrix} \frac{4}{3} &-4 \\ 1 & 1 \end{bmatrix} \\
            D = \begin{bmatrix} 1.2 & 0 \\ 0 & -0.4 \end{bmatrix}
          \end{align*}
          Therefore:
          \[
            \mbold{x}_0 = c_1 \mbold{v}_1 + c_2 \mbold{v}_2
          \]
          Where $\mbold{v}_1 = \begin{bmatrix} \frac{4}{3} \\ 1 \end{bmatrix} $, and $\mbold{v}_2 = \begin{bmatrix} -4
              \\ 1\end{bmatrix} $

          \noindent And in the long run:
          \[
            \mbold{x}_k = c_1 \lambda_1^{k} \mbold{v}_1 + c_2 \lambda_2^{k}\mbold{v}_2
          \]
          Therefore:
          \[
            \mbold{x}_k = c_1 \left( 1.2 \right)^{k}\mbold{v}_1 + c_2 \left( -0.4 \right) ^{k} \mbold{v}_2
          \]
          For large values of $k$:
          \[
            \mbold{x}_k \sim c_1 \left( 1.2 \right)^{k} \mbold{v}_1
          \]
          Therefore $\mbold{x}_{k+1}$:
          \[
            \mbold{x}_{k+1} \sim 1.2 \mbold{x}_k
          \]
          Therefore the number of juveniles will increase by 20 percent each year, and the number of adults will increase by 20 percent each year.


  \end{enumerate}


}

\end{document}
