\documentclass[12pt letter]{report}
\input{./template/preamble}
\input{./template/macros}
\input{./template/letterfonts}

\title{\Huge{Optimization}}
\author{\huge{Madiba Hudson-Quansah}}
\date{March 2023}
\usepackage{parskip}

\begin{document}
\maketitle
\newpage
\pdfbookmark[section]{\contentsname}{too}
\tableofcontents
\pagebreak

\chapter{Maximum and Minimum}
\nt{
	Local - Subsection of range\\
	Global - Whole range
}

\section{Maximize}
\[
	R(x) = 45 - \frac{x^2}{3}, \; 0 \leq x \leq 1
\]
\begin{center}
	Find all local maximum \underline{values}\\
	Find the global maximum \underline{value}
\end{center}

\section{Minimize}

\begin{itemize}
	\item Local Minimum - Minimum in specified range
	\item Global Minimum - Overall minimum
\end{itemize}

\section{Local/Relative Maximum/Minimum (Optimum)}

\begin{itemize}
	\item Find the critical values of the function:
	      \subitem Stationary points, i.e. $f^{\prime}(?) = 0$
	      \subitem Undefined points, i.e. $f^{\prime}(?) = \emptyset$
	\item Assess them for potential local maximum/minimum:
	      \subitem Find the first derivative, input values from the left and right of the critical points and check the change in signs:
	      \subsubitem $+$ to $-$: Maximum
	      \subsubitem $-$ to $+$: Minimum
	      \subitem Find second derivative, input the critical values and check the sign:
	      \subsubitem $-$: Maximum
	      \subsubitem $+$: Minimum
\end{itemize}
\pagebreak

\section{Global/Absolute Maximum/Minimum (Optima)}
To find the Absolute Optima of a function whose domain is unrestricted:

\begin{align*}
	\begin{array}{c c}
		\lim_{x \to \infty} f(x) & \lim_{x \to -\infty} f(x)
	\end{array}
\end{align*}

\subsection{Conditions for finding the Absolute Optima easily}

\begin{enumerate}
	\item Closed Domain, i.e. $[x_1,x_2]$
	\item Function is continuous for the duration of the closed domai
\end{enumerate}

\label{thm:evt}\thm{Extreme Value Theorem}{
	If a real valued function $f$ is continuous on the closed interval $[a,b]$, the $f$ must attain a maximum and minimum at least once.
	\begin{gather*}
		f(c) \geq f(x) \geq F(d) \\
		\forall x \in [a,b]
	\end{gather*}
	Where $f(c)$ is the function's minimum value and $F(d)$ is the function's maximum value.
}

\ex{}{
\[
	f(x)=x^3 \;\; \text{on} \;\; [-1,10]
\]

\begin{itemize}
	\item $f(x)$ is continuous due to it being a polynomial
	\item The function's domain is closed due to the end values being included in the domain
\end{itemize}


By EVT(\ref{thm:evt}) $f(x)$ must attain absolute maximum and minimum at least once on the interval. Possibly at:
\begin{enumerate}
	\item End points of the domain
	\item Critical values of $f(x)$
\end{enumerate}

\begin{align*}
	f(x) = x^3                                  \\
	f^{\prime}(x) = 3x^2                        \\
	0 = 3x^2                                    \\
	\frac{0}{3} = x^2                           \\
	0 = x                                       \\[10pt]
	f(-1) = -1                                  \\
	f(10) = 1000                                \\[20pt]
	\therefore \text{Absolute Maximum is } 1000 \\
	\text{Absolute Minimum is } -1
\end{align*}
}
\pagebreak

\chapter{Concavity}
Let $f$ be a function that is differentiable over an open interval $I$

\begin{itemize}
	\item If $f^{\prime}$ is increasing over $I$, we say $f$ is concave up over $I$, i.e. $f^{\prime\prime} > 0$
	\item If $f^{\prime}$ is decreasing over $I$, we say $f$ is concave down over $I$, i.e $f^{\prime\prime} < 0$
\end{itemize}

\section{Inflection}

A point where a function switches concavity, i.e:

\begin{gather*}
	f^{\prime\prime}(x^{-}) = +\text{ve } \; \text{to } \; f^{\prime\prime}(x^{+}) = -\text{ve }\\
	\text{or } \\
	f^{\prime\prime}(x^{-}) = -\text{ve } \; \text{to } \; f^{\prime\prime}(x^{+}) = +\text{ve }
\end{gather*}

\section{Curvature}
\subsection{Concave Up}
The \underline{cave} is facing up

\subsection{Concave Down}
The \underline{cave} is facing down

\pagebreak

\section{Questions}
\qs{}{
	A closed box with a square base is to contain 252 cubic feet. The bottom costs \$ 5 per square foot, the top costs \$2 per square foot, and the sides costs \$ 3 per square foot. Find the dimensions that minimize the cost.
}

\sol{
	\begin{figure}[ht]
		\centering
		\incfig{box}
		\caption{Box}
		\label{fig:box}
	\end{figure}

	\begin{align*}
		V = L  \times B  \times H                                 \\
		V = L  \times L  \times  H                                \\
		V = 252                                                   \\
		\therefore 252 = L^2 H                                    \\[10pt]
		\text{Cost of the top (CT) } = 2L^2                       \\
		\text{Cost of the bottom (CB)} = 5L^2                     \\
		\text{Cost of one side (CS) } = 3HL                       \\
		\text{Total cost (TC)} = CT + CB + 4(CS)                  \\[10pt]
		TC = 2L^2 + 4(3HL) + 5L^2                                 \\
		H = \frac{252}{L^2}                                       \\
		TC = 7L^2 + 12(\frac{252}{L^2})L                          \\
		TC = 7L^2 + \frac{3024}{L}                                \\[10pt]
		TC^{\prime} = 14L - \frac{3024}{L^2}                      \\
		0 = 14L^2 - \frac{3024}{L^2}                              \\
		0 = 14L^3 - 3024                                          \\
		216 = L^3                                                 \\
		6 = L                                                     \\[10pt]
		TC^{\prime\prime} = 14 + \frac{6048}{L^3}                 \\
		TC^{\prime\prime}(6) = 42 \;\; \therefore \text{Minimum } \\[10pt]
		H = \frac{252}{(6)^2}                                     \\
		H = 7                                                     \\[10pt]
	\end{align*}
	$\therefore$ at a width of 6 ft and a height of 7 ft the total cost is minimized
}
\pagebreak
\qs{}{
	A wire 16 ft long has to be formed into a rectangle. What dimensions should  the rectangle have to maximize area?
}
\sol{
	\begin{figure}[h]
		\centering
		\incfig{square}
		\caption{Square}
		\label{fig:square}
	\end{figure}
	\begin{align*}
		A = L  \times B                   \\
		2L + 2B = 16                      \\
		2(L + B) = 16                     \\
		L + B = 8                         \\
		B = 8 - L                         \\[10pt]
		A = L(8 - L)                      \\
		A = 8L - L^2                      \\
		A^{\prime} = -2L + 8              \\
		0 = -2L + 8                       \\
		L = 4                             \\[10pt]
		A^{\prime\prime} = -2             \\
		\therefore L = 4 \text{ Maximum } \\[10pt]
		B = 8 - 4                         \\
		B = 4
	\end{align*}
	$\therefore$ at a length of 4 ft and a width of 4 ft the area is maximized
}

\end{document}
