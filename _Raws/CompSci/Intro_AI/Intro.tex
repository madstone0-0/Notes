\documentclass[12pt letter]{report}
\input{./template/preamble}
\input{./template/macros}
\input{./template/letterfonts}

\title{\Huge{Introduction to Artificial Intelligence}}
\author{\huge{Madiba Hudson-Quansah}}
\date{}
\usepackage{parskip}

\setcounter{tocdepth}{4}
\setcounter{secnumdepth}{4}

\begin{document}
\maketitle
\newpage
\pdfbookmark[section]{\contentsname}{too}
\tableofcontents
\pagebreak

\begin{enumerate}
  \item Invention that drives the industrial revolution
  \item The pillars that are driving the 4th industrial revolution
  \item The ecosystem that is supporting AI.
  \item The current AI tech and the future of AI
\end{enumerate}



\begin{enumerate}
  \item The steam engine. The steam engine showed the potential mechanization had to improve production and everyday
        life.
  \item
        \begin{itemize}
          \item IOT (Internet of Things) - Computers, Internet
          \item Big Data Analytics - Data, Computers
          \item Cloud Computing - To increase processing power and storage available to process large amounts of data
          \item AI - AI models are trained on large amounts of data and require large amounts of processing power to train.
          \item Cybersecurity - To protect the data and the systems that are processing the data.
          \item Augmented Reality / Virtual Reality
          \item 3D Printing
          \item  Nano Technology
        \end{itemize}
\end{enumerate}

\chapter{Introduction to AI}

\section{Big Data}
\dfn{Big Data}{
  \begin{itemize}
    \item Volume - The amount of data that is being generated.
    \item Velocity - The speed at which the data is being generated.
    \item Variability - The diversity if data being generated over time.
    \item Volatility - The rate at which the data is changing / How unpredictable the data is.
    \item Variety - The different types of data that is being generated.
    \item Veracity - Reliability of the data.
    \item Value - Importance of the data.
    \item Visualization - How the data is being presented / displayed
  \end{itemize}
}

\subsection{Challenges of Big Data}
\begin{itemize}
  \item Storage - ROM
  \item Memory - RAM
  \item Compute power - CPU, TPU, GPU
\end{itemize}

\section{Decision Making Process}

\begin{itemize}
  \item Descriptive Analytics  - What happened? Looking at historical data to understand what happened.
  \item Diagnostic Analytics - Why did it happen? Looking at historical data to understand why something happened.
  \item Discovery Analytics - What can we learn from the data? Looking at historical data to discover new insights.
  \item Predictive Analytics - What is likely to happen, When will it happen, Why will it happen? Using historical data to predict future events.
  \item Prescriptive Analytics - What should we do based on what is likely to happen? Using predictions  to recommend actions.
\end{itemize}

\subsection{Descriptive Analytics}
\begin{enumerate}
  \item Statistical Exploratory Data Analysis (EDA).
\end{enumerate}

\subsection{Diagnostic Analytics}
\begin{enumerate}
  \item Data Mining.
\end{enumerate}
\subsection{Discovery Analytics}
\begin{enumerate}
  \item Identifying previously unknown trends and patterns.
  \item Adapting to identified trends and patterns for profit.
\end{enumerate}
\subsection{Predictive Analytics}
\begin{enumerate}
  \item Machine Learning
  \item Deep Learning
  \item Machine Modelling
\end{enumerate}
\subsection{Prescriptive Analytics}
\begin{enumerate}
  \item Simulation
  \item Variable Analysis
\end{enumerate}

\section{What is AI?}

\dfn{Artificial Intelligence}{
  Technology that simulates human intelligence.
}

\dfn{General Programming}{
  \begin{itemize}
    \item Input
    \item Process
    \item Output
  \end{itemize}
  In general programming the machine is given a set of precise instructions to follow, by a programmer.
}

\dfn{Machine Learning Model}{
  \begin{itemize}
    \item Input
    \item Output
    \item Process
  \end{itemize}
  In machine learning the machine is given a set of data and is trained to learn from the data. The machine then uses
  the data to process new data different from the training data..
}

\subsection{Data Science vs AI vs Machine Learning}

Data science deals with the extraction of knowledge from data, while AI is the simulation of human intelligence, and
Machine Learning is the vehicle AI uses to mimic human intelligence, giving it the ability to learn and improve from new
data.
\begin{itemize}
  \item Data science is based on strict analytical evidence, while AI imparts human-like intelligence to machines.
  \item Deals with structured and instructed data, while AI deals with logic and decision trees
  \item Included various data operations while AI includes machine learning.
\end{itemize}

\section{AI Environment}

\subsection{AI Community}

\begin{itemize}
  \item AI Innovators
  \item Standardisation Organization
  \item AI Developers
  \item AI Distributors
  \item Policy Makers
  \item Service Providers
  \item Researchers
  \item AI Manufacturers
  \item Consumers
\end{itemize}

\nparagraph{AI Innovators}
\begin{itemize}
  \item Develop new AI technologies
  \item Develop new AI applications
  \item Develop new AI algorithms
  \item Develop new AI models
\end{itemize}

\nparagraph{AI Developers}

Implement software that uses AI technologies.

\nparagraph{Researchers}

Improve AI technologies.

\nparagraph{AI Manufacturers}

Companies that produce AI hardware.

\nparagraph{AI Distributors}

Companies that distribute software that uses AI technologies.

\nparagraph{Service Providers}

Companies that provide services to host AI technologies.

\nparagraph{Data centre vs HPC}
Data centre - Data storage and application deployment
HPC (High Performance Computing) - Computing power.

\begin{itemize}
  \item Simple regression
  \item Multi-linear regression
  \item Polynomial Regression
  \item Gradient Descent Algorithm
  \item Regularization of Regression Models
  \item Decision Tree Regressors
  \item Evaluating Regression Models
\end{itemize}


\end{document}
