\documentclass[12pt letter]{report}
\input{./template/preamble}
\input{./template/macros}
\input{./template/letterfonts}

\title{\Huge{Probability}}
\author{\huge{Madiba Hudson-Quansah}}
\date{}
\usepackage{parskip}

\begin{document}
\maketitle
\newpage
\pdfbookmark[section]{\contentsname}{too}
\tableofcontents
\pagebreak

\chapter{Module 8: Introduction}

\section{Introduction To Probability}

\dfn{Probability}{
	A mathematical description of randomness and uncertainty / The likelihood of an event occurring.

	The notation for Probability is $P \left( X \right) $ where $X$ is the event.

	Probability is always between $0 \leq P \left( X \right) \leq 1$ or $0\% \leq P \left( X \right) \leq 100\%$.
}

There are two ways of determining probability:

\begin{itemize}
	\item Theoretical / Classical - Determined by the nature of the experiment
	\item Empirical / Observational - Determined by the results of the experiment
\end{itemize}

\section{Relative Frequency}

\dfn{Relative Frequency}{
	Relative frequency is the number of times an event occurs divided by the total number of trials.

	\[
		P \left( X \right) = \frac{\text{Number of times event occurs}}{\text{Total number of trials}}
	\]
}

\thm{The Law of Large Numbers}{
	As the number of trials increases, the relative frequency of an event approaches the theoretical probability of the event.
}


\chapter{Module 9: Find the Probability of Events}

\section{Sample Spaces and Events}

\dfn{Random Experiment}{
	An experiment whose outcome is determined by chance.
}

\dfn{Sample Space}{
	The list of possible outcomes of a random experiment, denoted by $S$.
}

\dfn{Event}{
	A statement about the nature of the outcome after the experiment has been conducted, denoted by any capital letter
	except $S$.
}

\section{Equally Likely Outcomes}

\[
	P \left( A \right) = \frac{\text{Number of outcomes in }A}{\text{Number of outcomes in } S}
\]

Where $A$ is an event and $S$ is the sample space.








\end{document}
