\documentclass[12pt letter]{report}
\input{./template/preamble}
\input{./template/macros}
\input{./template/letterfonts}

\title{\Huge{Integration}}
\author{\huge{Madiba Hudson-Quansah}}
\date{\today}
\usepackage{parskip}

\begin{document}
\maketitle
\newpage
\pdfbookmark[section]{\contentsname}{too}
\tableofcontents
\pagebreak

\chapter{Indefinite Integration}

\section{Anti-derivatives}

A derivative $f^{\prime}(x)$ is the result of performing differentiation on a function $f(x)$.

An anti-derivative $f(x)$ is the result of performing integration on a derivative $f^{\prime}(x)$.

The result of an indeterminate integration on a derivate is a family of functions, each of which has a possibility of
being the derivative's source function.

\begin{align*}
	\int f^{\prime}(x) \enspace  \mathrm{d}x \\
	 & = f(x) + c                            \\
	\\
\end{align*}

I.e.

\begin{align*}
	\int x^{n} \enspace  \mathrm{d}x \\
	 & = \frac{x^{n+1}}{n+1}         \\
\end{align*}

\pagebreak

\chapter{Integration Techniques}

\section{Integration by Substitution}

\qs{}{
	\[
		\int x^2 \sqrt{x^3 + 5} \enspace  \mathrm{d}x \text{using } u = x^3 + 5
	\]
}

\sol {
	\begin{align*}
		\int x^2 \sqrt{u} \enspace  \mathrm{d}x                       \\
		\\
		\frac{du}{dx} = x^3 + 5                                       \\
		\frac{du}{dx} = 3x^2                                          \\
		\frac{1}{3} \enspace  \mathrm{d}u = x^2 \enspace  \mathrm{d}x \\
		\\
		\int \sqrt{u} \frac{1}{3} \enspace  \mathrm{d}u               \\
		\frac{1}{3} \int u \frac{1}{2} \enspace  \mathrm{d}u          \\
		\frac{1}{3}  \times (\frac{u^{\frac{3}{2}}}{\frac{3}{2}}) + c \\
		\frac{2u^{\frac{3}{2}}}{9} + c                                \\
		 & = \frac{2}{9}  \times  (x^3 + 5)^{\frac{3}{2}} +c          \\
	\end{align*}
}

\pagebreak
\section{Integration by Parts}
\huge{LIATE} \label{liate:1} \normalsize

\label{lg:1}Logarithm

\label{in:1}Inverse function

\label{al:1}Algebra

\label{tr:1}Trigonometry

\label{ex:1}Exponent

\qs{}{
	\[
		\int x e^{x} \enspace  \mathrm{d}x
	\]
}

\sol{
	$\int x e^{x}\enspace  \mathrm{d}x$ where $u = x$ and $\frac{dv}{dx} = e^{x}$ due to A \ref{al:1} coming before E
	\ref{ex:1} in LIATE \ref{liate:1}
	\begin{align*}
		u = x                                                                     \\
		u^{\prime} = 1                                                            \\
		v = \int \frac{dv}{dx} \enspace  \mathrm{d}x                              \\
		v = \int e^{x} \enspace  \mathrm{d}x                                      \\
		v = e^{x}                                                                 \\
		\\
		u  \times  v - \int v  \times u^{\prime} \enspace  \mathrm{d}x \therefore \\
		\\
		x  \times  e^{x} - \int e^{x}  \times  1 \enspace  \mathrm{d}x            \\
		xe^{x} - e^{x} + c                                                        \\
		 & = e^{x}(x-1) + c                                                       \\
	\end{align*}
}

\chapter{Applications of Integration}

\section{Economics}

\section{Probability}

\subsection{Probability Density Function (P.D.F)}

For a continuous random variable $X$, a Probability Density Function (P.D.F.) is a function $f(x)$ such that over a
given interval $[a,b]$ / $a \leq x \leq b$:

\begin{itemize}
	\item $f(x)$ must be continuous over the domain $[a,b]$
	\item $f(x) \geq 0$ for all $x$ in $[a,b]$
	\item $\int_{a}^{b} f(x) \enspace  \mathrm{d}x = 1$

\end{itemize}

\qs{}{
	Let
	\[
		f(x) = \begin{cases}
			\frac{3}{4}(2x-x^2), & 0 \leq x \leq 2  \\
			0,                   & \text{otherwise}
		\end{cases}
	\]
	\begin{enumerate}
		\item Show that $f(x)$ is a probability density function.
		\item Find \begin{enumerate}
			      \item $P\left( 0.3 \leq x \leq 1.5 \right) $
			      \item $P\left( x \leq 0.25 \right) $
			      \item $P\left( x \geq 1.4 \right) $
			      \item $P\left( x > 0.25 \right) $
		      \end{enumerate}
	\end{enumerate}
}

\sol{
	\begin{enumerate}
		\item \begin{description}
			      \item[Cond 1:] $f(x)$ is continuous for all real numbers.
			      \item[Cond 2:] $f(x) \geq 0$ for all real numbers / $\left( -\infty, \infty \right)$\\Check:
				      \begin{align*}
					      x=1                   \\
					      \frac{3}{4}(2x - x^2) \\
					      f(x) = 0.75           \\
					      \\
					      x = -1                \\
					      f(x) = 0              \\
					      \\
					      x = 3                 \\
					      f(x) = 0              \\
					      \\
				      \end{align*}
			      \item[Cond 3:] $\int^{b}_{a} f(x) = 1$\\
				      Check:
				      \begin{align*}
					      \int^0_{-\infty} f(x) \enspace  \mathrm{d}x + \int^2_0 f(x) \enspace  \mathrm{d}x +
					      \int^{\infty}_2 f(x) \enspace  \mathrm{d}x              \\
					      \int^{0}_{-\infty} 0 \enspace  \mathrm{d}x + \int^2_0 \frac{3}{4}(2x-x^2) \enspace
					      \mathrm{d}x + \int^{\infty}_{2} 0 \enspace  \mathrm{d}x \\
					      0 + 1 + 1                                               \\
					      \therefore \int^{\infty}_{-\infty} f(x) \enspace  \mathrm{d}x = 1
				      \end{align*}
		      \end{description}
		\item
		      \begin{align*}
			      P \left( 0.3 \leq x \leq 1.5 \right)  = \int^{1.5}_{0.3} \frac{3}{4}(2x+x^2) \enspace  \mathrm{d}x \\
			      \frac{3}{4}[x^2 - \frac{1}{3}x^3]^{1.5}_{0.3}                                                      \\
			      \frac{3}{4}( \frac{9}{8} - \frac{81}{1000} )                                                       \\
			       & P \left( 0.3 \leq x \leq 1.5  \right)  = 0.7830                                                 \\
		      \end{align*}
		\item
		      \begin{align*}
			      P\left(  x\leq 0.25 \right)  = \int^{0.25}_{-\infty} f(x) \enspace  \mathrm{d}x                    \\
			      \int^0_{-\infty} 0 \enspace  \mathrm{d}x + \int^{0.25}_0 \frac{3}{4}(2x+x^2) \enspace  \mathrm{d}x \\
			      \frac{3}{4}x^2 - \frac{1}{4}x^3|^{0.25}_0                                                          \\
			       & P\left(  x\leq 0.25 \right)  =
		      \end{align*}
	\end{enumerate}
}

\qs{}{
	The continuous random variable $X$ has a P.D.F. given by
	\[
		f(x) = \begin{cases}
			2x + k, & 3 \leq x \leq_4   \\
			0,      & \text{otherwise }
		\end{cases}
	\]

	\begin{enumerate}
		\item Show that $k = -6$
		\item Determine \begin{enumerate}
			      \item  $P\left( x > 3.5 \right) $
			      \item $P\left( 2.5 \leq x \leq 3.5 \right) $
			      \item $P\left( x > 6 \right) $
		      \end{enumerate}
		\item Find the expected value of $X$
	\end{enumerate}
}

\sol{

	\begin{enumerate}
		\item
		      \begin{align*}
			      \int^4_3 f(x) \enspace  \mathrm{d}x = 1 \\
			      \int^4_3 2x+k \enspace  \mathrm{d}x = 1 \\
			      x^2 + kx |^4_3 = 1                      \\
			      16 + 4k - 9 - 3k = 1                    \\
			       & k = -6                               \\
		      \end{align*}
		\item \begin{enumerate}
			      \item
			            \begin{align*}
				            \int ^{\infty}_{3.5} f(x) \enspace  \mathrm{d}x \\
				            \int ^{4}_{3.5} 2x - 6 \enspace  \mathrm{d}x    \\
				            x^2 - 6x |^{4}_{3.5}                            \\
				             & = 0.7560                                     \\
			            \end{align*}
			      \item
			            \begin{align*}
				            \int ^{3.5}_{2.5} f(x) \enspace  \mathrm{d}x   \\
				            \int ^{3.5}_{2.5} 2x - 6 \enspace  \mathrm{d}x \\
				            x^2 - 6x |^{3.5}_{2.5}                         \\
				             & = 0.2580                                    \\
			            \end{align*}
		      \end{enumerate}
	\end{enumerate}
}

\chapter{Improper Integrals}

\section{Infinite Limits}
For
\[
	\int^{\infty}_{a} \frac{1}{x^{p}} \enspace  \mathrm{d}x
\]

\begin{itemize}
	\item If $a > 0$ and $p > 1$, then the integral is Convergent.
	\item If $a > 0$ an $p \leq 1$, then the integral is Divergent.
\end{itemize}


\qs{}{
	\[
		\int^{\infty}_{a} f(x) \enspace  \mathrm{d}x
	\]
}

\sol{
	\begin{align*}
		\int^{\infty}_{1} \frac{1}{x} \enspace  \mathrm{d}x                     \\
		{\int^{t}_{1}}_{\lim_{t \to \infty} } \frac{1}{x} \enspace  \mathrm{d}x \\
		\ln (x) |^{t}_{1}                                                       \\
		\lim_{t \to \infty} \ln (t) - \ln (1)                                   \\
		\ln (\infty) \implies \infty                                            \\
		\infty - 0                                                              \\
		\lim_{t \to \infty} \ln (t) - \ln (1) = \infty                          \\
		 & \int^{\infty}_{1} \frac{1}{x} \enspace  \mathrm{d}x = \infty         \\
	\end{align*}
	Since the limit of the integral is $\infty$, the integral is said to be \textbf{Divergent}.

}


\pagebreak
\qs{}{
	\[
		\int^{\infty}_{1} \frac{1}{x^2} \enspace  \mathrm{d}x
	\]
}

\sol{
	\begin{align*}
		\int^{\infty}_{1} \frac{1}{x^2} \enspace  \mathrm{d}x                   \\
		\lim_{t \to \infty} \int^{t}_{1} \frac{1}{x^2} \enspace  \mathrm{d}x    \\
		-\frac{1}{x} |^{t}_{1}                                                  \\
		\lim_{t \to \infty} [-\frac{1}{t} + \frac{1}{1} ]                       \\
		0 - (-1)                                                                \\
		 & \therefore \int^{\infty}_{1} \frac{1}{x^2} \enspace  \mathrm{d}x = 1 \\
	\end{align*}
	Since the limit of the integral is infinite, the integral is said to be \textbf{Convergent}
}

\qs{}{
	\[
		\int_{-\infty}^{\infty} \frac{1}{x^2} \enspace  \mathrm{d}x
	\]
}

\sol{
	\begin{align*}
		\int_{-\infty}^{\infty} \frac{1}{x^2} \enspace  \mathrm{d}x                                                    \\
		\\
		\int_{-\infty}^{3} \frac{1}{x^2} \enspace  \mathrm{d}x + \int_{3}^{\infty} \frac{1}{x^2} \enspace  \mathrm{d}x \\
		\\
		\lim_{t \to -\infty} \int_{t}^{3} \frac{1}{x^2} \enspace  \mathrm{d}x                                          \\
		-\frac{1}{x}|_{t}^{3}                                                                                          \\
		\lim_{t \to -\infty} [-\frac{1}{3} + \frac{1}{t} ]                                                             \\
		\lim_{t \to -\infty} [-\frac{1}{3} + \frac{1}{t^2}]                                                            \\
		-\frac{1}{3} + \frac{1}{-\infty}                                                                               \\
		-\frac{1}{-\infty} \implies 0                                                                                  \\
		\int_{-\infty}^{3} \frac{1}{x^2} \enspace  \mathrm{d}x = -\frac{1}{3}                                          \\
		\\
		\lim_{t \to \infty} \int_{3}^{t} \frac{1}{x^2} \enspace  \mathrm{d}x                                           \\
		\lim_{t \to \infty} [ -\frac{1}{t} + \frac{1}{3} ]                                                             \\
		-\frac{1}{\infty} + \frac{1}{3}                                                                                \\
		-\frac{1}{\infty} \implies \infty                                                                              \\
		\int_{3}^{\infty} \frac{1}{x^2} \enspace  \mathrm{d}x = \infty + \frac{1}{3}                                   \\
		\\
		\\
		\int_{-\infty}^{\infty} \frac{1}{x^2} \enspace  \mathrm{d}x = -\frac{1}{3} + \infty + \frac{1}{3}              \\
		\therefore \int_{-\infty}^{\infty} \frac{1}{x^2} \enspace  \mathrm{d}x = \infty                                \\
	\end{align*}
	Since the overall integral is infinite, the integral is Divergent / Since one of the sub-integrals are Divergent the
	overall integral is Divergent
}
\pagebreak

\qs{}{
	\[
		\int_{0}^{3} \frac{1}{x -3} \enspace  \mathrm{d}x
	\]
}

\sol{
	\begin{align*}
		\int_{0}^{3} \frac{1}{x -3} \enspace  \mathrm{d}x                 \\
		\lim_{t \to 3}  \int_{0}^{t} \frac{1}{x -3} \enspace  \mathrm{d}x \\
		\lim_{t \to 3} \left( \ln |x-3| \right)_{0}^{t}                   \\ \lim_{t \to 3}  \left( \ln | t- 3 | - \ln
		| -3 |\right)                                                     \\
		\ln(0) - \ln(3)                                                   \\
		\ln(0) \implies -\infty                                           \\
		\int_{0}^{3} \frac{1}{x -3} \enspace  \mathrm{d}x = -\infty       \\
	\end{align*}
	Since the limit of the integral is infinite, the integral is \textbf{Divergent}
}



\end{document}
