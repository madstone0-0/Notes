\documentclass[12pt letter]{report}
\input{../template/preamble}
\input{../template/macros}
\input{../template/letterfonts}

\usepackage{parskip}
\title{\Huge{Homework 1}}
\author{\huge{Madiba Hudson-Quansah}}
\date{}


\begin{document}
\maketitle
\newpage

\qs{}{
  Find the general solution of the system whose augmented matrix is given below
  \[
    \begin{bmatrix}
      1 & 0 & -5 & 0  & -8 & 3 \\
      0 & 1 & 4  & -1 & 0  & 6 \\
      0 & 0 & 0  & 0  & 1  & 0 \\
      0 & 0 & 0  & 0  & 0  & 0 \\
    \end{bmatrix}
  \]
}

\sol{
  \begin{align*}
    \begin{bmatrix}
      1 & 0 & -5 & 0  & -8 & 3 \\
      0 & 1 & 4  & -1 & 0  & 6 \\
      0 & 0 & 0  & 0  & 1  & 0 \\
      0 & 0 & 0  & 0  & 0  & 0 \\
    \end{bmatrix}
    \\
    -\frac{8}{1}R_3 - R_1 \to R_1 \\
    \begin{bmatrix}
      -1 & 0 & 5 & 0  & 0 & -3 \\
      0  & 1 & 4 & -1 & 0 & 6  \\
      0  & 0 & 0 & 0  & 1 & 0  \\
      0  & 0 & 0 & 0  & 0 & 0  \\
    \end{bmatrix}
    \\
    -R_1 \to R_1                  \\
    \begin{bmatrix}
      1 & 0 & -5 & 0  & 0 & 3 \\
      0 & 1 & 4  & -1 & 0 & 6 \\
      0 & 0 & 0  & 0  & 1 & 0 \\
      0 & 0 & 0  & 0  & 0 & 0 \\
    \end{bmatrix}
    \\
    x_1 -5x_2 = 3                 \\
    x_2 +4x_3 -x_4 = 6            \\
    x_3 = x_3                     \\
    x_5 = 0                       \\
    x_4 = x_4                     \\
    \\
    x_1 = 3 + 5x_2                \\
    x_2 = 6 - 4x_3 + x_4          \\
    x_3 = x_3                     \\
    x_4 = x_4                     \\
    x_5 = 0                       \\
    \\
    \begin{cases}
      x_1 = 3 + 5x_2       \\
      x_2 = 6 - 4x_3 + x_4 \\
      x_3 \text{ is free}  \\
      x_4 \text{ is free}  \\
      x_5 = 0
    \end{cases}
  \end{align*}
}

\qs{}{
  Choose $h$ and $k$ such that the following system has:
  \begin{enumerate}
    \item No solution
    \item A unique solution
    \item Many solutions
  \end{enumerate}
  Give separate answers for each part.
  \begin{align*}
    x_1 - 3x = 1    \\
    2x_1 + hx_2 = k \\
  \end{align*}
}

\sol{
  \begin{enumerate}
    \item
          \begin{align*}
            \begin{bmatrix}
              1 & -3 & 1 \\
              2 & h  & k \\
            \end{bmatrix}
            \\
            2R_1 - R_2 \to R_2 \\
            \begin{bmatrix}
              1 & -3                    & 1                    \\
              0 & \left( -6 - h \right) & \left( 2 - k \right) \\
            \end{bmatrix}
            \\
            \\
            -6 -h = 0          \\
            h = -6             \\
            \\
            2 - k \neq 0       \\
            k \neq 2           \\
          \end{align*}
          $\therefore$ the system has no solution when $h = -6$ and $k \neq  2$. So for example $h = -6$ and $k = 0$
    \item
          \begin{align*}
            \begin{bmatrix}
              1 & -3                    & 1                    \\
              0 & \left( -6 - h \right) & \left( 2 - k \right) \\
            \end{bmatrix}
            \\
            \\
            -6 -h \neq 0 \\
            h \neq -6    \\
            \\
            2 - k \neq 0 \\
            k \neq 2     \\
          \end{align*}
          $\therefore$ the system has a unique solution when $h \neq -6$ and $k \neq 2$. So for example $h = 0$ and $k = 0$
    \item
          \begin{align*}
            \begin{bmatrix}
              1 & -3                    & 1                    \\
              0 & \left( -6 - h \right) & \left( 2 - k \right) \\
            \end{bmatrix}
            \\
            \\
            -6 - h = 0 \\
            h = -6     \\
            \\
            2 - k = 0  \\
            k = 2      \\
          \end{align*}
          $\therefore$ the system has many solutions when $h = -6$ and $k = 2$.
  \end{enumerate}
}

\qs{}{
  A system of linear equations with fewer equations than unknowns is sometimes called an
  underdetermined system. Give an example of an inconsistent underdetermined system of two equations
  in three unknowns.
}

\sol{
  \begin{align*}
    2x_1 + 3x_2 + 4x_3 = 1  \\
    2x_1 + 3x_2 + 4x_3 = 10 \\
  \end{align*}
}

\pagebreak
\qs{}{
  Determine if $\mathbf{b}$ is a linear combination of $\mathbf{a}_1, \mathbf{a}_2, \mathbf{a}_3$.
  \[
    \mathbf{a}_1 = \begin{bmatrix} 1 \\  0 \\ 1 \end{bmatrix}, \mathbf{a}_2 = \begin{bmatrix} -2 \\  3 \\  -2 \end{bmatrix}
    , \mathbf{a}_3 = \begin{bmatrix} -6 \\ 7 \\  5 \end{bmatrix}, \mathbf{b} = \begin{bmatrix} 11 \\ -5 \\ 9 \end{bmatrix}
  \]
}

\sol{
  To determine if $\mathbf{b}$ is a linear combination of the given vectors I must prove that there exists weights $x_1,
    x_2, x_3$ such that
  \[
    x_1 \mathbf{a}_1 + x_2 \mathbf{a}_2 + x_3 \mathbf{a}_2 = \mathbf{b}
  \]
  Therefore:
  \begin{align*}
    x_1 -2x_2 - 6x_3 = 11                   \\
    3x_2 + 7x_3 = -5                        \\
    x_1 - 2x_2 + 5x_3 = 9                   \\
    \\
    \begin{bmatrix}
      1 & -2 & -6 & 11 \\
      0 & 3  & 7  & -5 \\
      1 & -2 & 5  & 9
    \end{bmatrix}                        \\
    R_2 \leftrightarrow R_3                 \\
    \begin{bmatrix}
      1 & -2 & -6 & 11 \\
      1 & -2 & 5  & 9  \\
      0 & 3  & 7  & -5 \\
    \end{bmatrix}
    \\
    R_1 - R_2  \rightarrow R_2              \\
    \begin{bmatrix}
      1 & -2 & -6  & 11 \\
      0 & 0  & -11 & 2  \\
      0 & 3  & 7   & -5 \\
    \end{bmatrix}
    \\
    R_2 \leftrightarrow R_3                 \\
    \begin{bmatrix}
      1 & -2 & -6  & 11 \\
      0 & 3  & 7   & -5 \\
      0 & 0  & -11 & 2  \\
    \end{bmatrix}
    \\
    \frac{-2}{3}R_2 - R_1  \rightarrow R_1  \\
    \begin{bmatrix}
      -1 & 0 & \frac{4}{3} & \frac{-23}{3} \\
      0  & 3 & 7           & -5            \\
      0  & 0 & -11         & 2             \\
    \end{bmatrix}
    \\
    \frac{-7}{11}R_3 - R_2  \rightarrow R_2 \\
    \begin{bmatrix}
      -1 & 0  & \frac{4}{3} & \frac{-23}{3} \\
      0  & -3 & 0           & \frac{41}{11} \\
      0  & 0  & -11         & 2             \\
    \end{bmatrix}
    \\
    \frac{-4}{33}R_3 - R_1  \rightarrow R_1 \\
    \begin{bmatrix}
      1 & 0  & 0   & \frac{245}{33} \\
      0 & -3 & 0   & \frac{41}{11}  \\
      0 & 0  & -11 & 2              \\
    \end{bmatrix}
    \\
    -\frac{1}{3}R_2 \to R_2                 \\
    \begin{bmatrix}
      1 & 0 & 0   & \frac{245}{33} \\
      0 & 1 & 0   & -\frac{41}{33} \\
      0 & 0 & -11 & 2              \\
    \end{bmatrix}            \\
    -\frac{1}{11} R_3 \to R_3               \\
    \begin{bmatrix}
      1 & 0 & 0 & \frac{245}{33} \\
      0 & 1 & 0 & -\frac{41}{33} \\
      0 & 0 & 1 & -\frac{2}{11}  \\
    \end{bmatrix}              \\
  \end{align*}
  $\therefore$ the vector $\mathbf{b}$ is a linear combination of the vectors $\mathbf{a}_1, \mathbf{a}_2, \mathbf{a}_3$ for
  weights:
  \begin{align*}
    x_1 = \frac{245}{33} \\
    x_2 = -\frac{41}{33} \\
    x_3 = -\frac{2}{11}  \\
  \end{align*}

}

\qs{}{
  Show that the equation $A \mathbf{x} = \mathbf{b}$ does not have a solution for all possible $\mathbf{b}$ and describe
  the set of all $\mathbf{b}$ for which the equation does have a solution.
  \[
    A = \begin{bmatrix} 1 & 2 & -1 \\ -2 & 2 & 0 \\  4 & -1 & 3 \end{bmatrix}, \text{ and } \mathbf{b} = \begin{bmatrix}
      b_1 \\ b_2 \\ b_3 \end{bmatrix}
  \]
}

\sol{

  The equation $A \mathbf{x} = \mathbf{b}$ is as follows:
  \[
    \begin{bmatrix} 1 & -2 & -1 \\ -2 & 2 & 0 \\  4 & -1 & 3 \end{bmatrix} \begin{bmatrix} x_1 \\  x_2 \\ x_3 \end{bmatrix}
    = \begin{bmatrix}
      b_1 \\ b_2 \\ b_3 \end{bmatrix}
  \]
  resulting in the system:
  \begin{align*}
    x_1 - 2x_2 -x_3 = b_1 \\
    -2x_1 + 2x_2 = b_1    \\
    4x_1 -x_2 + 3x_2 = b_3
  \end{align*}
  Finding the solutions to the system:
  \begin{align*}
    \begin{bmatrix}
      1  & -2 & -1 & b_1 \\
      -2 & 2  & 0  & b_2 \\
      4  & -1 & 3  & b_3
    \end{bmatrix}                        \\
    -2R_1 - R_2 \to R_2                       \\
    \begin{bmatrix}
      1 & -2 & -1 & b_1         \\
      0 & 2  & 2  & -2b_1 - b_2 \\
      4 & -1 & 3  & b_3
    \end{bmatrix}                 \\
    4R_1 - R_3 \to R_3                        \\
    \begin{bmatrix}
      1 & -2 & -1 & b_1         \\
      0 & 2  & 2  & -2b_1 - b_2 \\
      0 & -7 & -7 & 4b_1 - b_3
    \end{bmatrix}                 \\
    -\frac{7}{2}R_2 - R_3 \to R_3             \\
    \begin{bmatrix}
      1 & -2 & -1 & b_1                         \\
      0 & 2  & 2  & -2b_1 - b_2                 \\
      0 & 0  & 0  & 3b_1 + \frac{7}{2}b_2 + b_3
    \end{bmatrix} \\
  \end{align*}
  From the echelon form I can see that this system cannot be consistent for all values of $\mathbf{b}$ because not all
  values of $\mathbf{b}$ will result in the expression $3b_1 + \frac{7}{2}b_2 + b_3$ evaluating to 0.
  \[
    3b_1 + \frac{7}{2}b_2 + b_3 = 0
  \]
  The set of $\mathbf{b}$ for which the equation does have a solution is the set of all $\mathbf{b}$ that satisfy the
  above equation.
}

\end{document}
