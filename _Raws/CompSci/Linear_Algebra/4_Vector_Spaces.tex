\documentclass[12pt letter]{report}
\input{./template/preamble}
\input{./template/macros}
\input{./template/letterfonts}

\title{\Huge{Vector Spaces}}
\author{\huge{Madiba Hudson-Quansah}}
\date{}
\usepackage{parskip}

\setcounter{tocdepth}{4}
\setcounter{secnumdepth}{4}

\begin{document}
\maketitle
\newpage
\pdfbookmark[section]{\contentsname}{too}
\tableofcontents
\pagebreak

\chapter{Vector Spaces and Subspaces}

\section{Introduction}

\dfn{Vector Space}{
  A \textit{vector space} is a non empty set $V$ of objects, called vectors, on which are defined two operations,
  addition and multiplication by scalars, e.g. real numbers, subject to the following axioms which must hold for all
  vectors $\mbold{u}, \mbold{v}$ and $\mbold{w}$ in $V$ and for all scalars $c$ and $d$.
  \begin{enumerate}
    \item The sum of $\mbold{u}$ and $\mbold{v}$, denoted by $\mbold{u} + \mbold{v}$, is in $V$.
    \item $\mbold{u} + \mbold{v} = \mbold{v} + \mbold{u}$
    \item $ \left( \mbold{u} + \mbold{v} \right) + \mbold{w} = \mbold{u} + \left( \mbold{v} + \mbold{w} \right)  $
    \item There is a zero vector $\mbold{0}$ in $V$ such that $\mbold{u} + \mbold{0} = \mbold{u}$
    \item For each $\mbold{u}$ in $V$, there is a vector $-\mbold{u}$ in $V$ such that $\mbold{u} + \left( -
            \mbold{u} \right) = \mbold{0} $
    \item The scalar multiple of $\mbold{u}$ by $c$, denoted by $c\mbold{u}$, is in $V$.
    \item $c\left( \mbold{u} + \mbold{v} \right) = c\mbold{u} + c\mbold{v}$
    \item $\left( c + d \right)\mbold{u} = c\mbold{u} + d\mbold{u}$
    \item $c\left( d\mbold{u} \right) = \left( cd \right)\mbold{u}$
    \item $1\mbold{u} = \mbold{u}$
  \end{enumerate}
}

\section{Subspaces}

\dfn{Subspace}{
  A subset $H$ of the vector space $V$, where:
  \begin{enumerate}
    \item The zero vector of $V$ is in $H$.
    \item $H$ is closed under vector addition. That is for each $\mbold{u}$ and $\mbold{v}$ in $H$, the sum of
          $\mbold{u} + \mbold{v}$ is in $H$.
    \item $H$ is closed under scalar multiplication. That is for each $\mbold{u}$ in $H$ and each scalar $c$, the
          scalar multiple $c\mbold{u}$ is in $H$.
  \end{enumerate}
}

\subsection{Subspace Spanned by a Set}

\thm{}{
  If $\mbold{v}_1, \ldots, \mbold{v}_p$ are in vector space $V$, then $\text{Span}\{\mbold{v}_1, \ldots, \mbold{v}_p\}
  $ is a subspace of $V$.
}

\chapter{Null Space, Column Space, and Linear Transformations}

\section{The Null Space of a Matrix}

\dfn{Null Space}{
  The \textit{null space} of an $m \times n$ matrix $A$, denoted by $\text{Nul } A$, is the set of all solutions of
  the homogeneous equation $A \mbold{x} = \mbold{0}$. In set notation:
  \[
    \text{Nul }A = \{\mbold{x} : \mbold{x} \text{ is in $\mathbb{R}^{n}$ and } A\mbold{x} = \mbold{0}\}
  \]
}

\ex{}{
  \qs{}{
    Let $A$ be the matrix $\begin{bmatrix}
        1  & -3 & -2 \\
        -5 & 9  & 1
      \end{bmatrix} $, and let $\mbold{u} = \begin{bmatrix} 5 \\ 3 \\ -2 \end{bmatrix} $. Determine if $\mbold{u}$ belongs
    to the null space of $A$.
  }

  \sol{
    This is basically asking us to verify if $\mbold{u}$ satisfies the equation $A \mbold{u} = \mbold{0}$

    \begin{align*}
      \begin{bmatrix}
        1  & -3 & -2 \\
        -5 & 9  & 1
      \end{bmatrix}       \begin{bmatrix} 5 \\ 3\\ -2 \end{bmatrix} & = \begin{bmatrix} 0 \\  0  \end{bmatrix} \\
      \begin{bmatrix}
        5  -9 + 4    \\
        -25 + 27  -2 \\
      \end{bmatrix}                                               & = \begin{bmatrix} 0 \\ 0 \end{bmatrix}     \\
    \end{align*}
    $\therefore$ $\mbold{u}$ is in the null space of $A$.
  }
}

\thm{}{
  The null space of an $m\times n$ matrix $A$ is a subspace of $\mathbb{R}^{n}$, equivalently, the set of all
  solutions to a system $A\mbold{x} = \mbold{0} $ of $m$ homogeneous linear equations in $n$ unknowns is a subspace
  of $\mathbb{R}^{n}$.
}

\subsection{An Explicit Description of the Null Space of a Matrix}

\end{document}
