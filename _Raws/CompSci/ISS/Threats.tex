\documentclass[12pt letter]{report}
\input{./template/preamble}
\input{./template/macros}
\input{./template/letterfonts}

\title{\Huge{Threats and Attacks}}
\author{\huge{Madiba Hudson-Quansah}}
\date{}
\usepackage{parskip}

\setcounter{tocdepth}{4}
\setcounter{secnumdepth}{4}

\begin{document}
\maketitle
\newpage
\pdfbookmark[section]{\contentsname}{too}
\tableofcontents
\pagebreak

\chapter{Threat Categories}

\dfn{Threat}{
  A threat is a potential cause of an unwanted incident, which may result in harm to a system or organization.
}

There are 12 categories of threats:
\begin{itemize}
  \item Compromises to Intellectual Property
  \item Deviations in quality of service
  \item Espionage / Trespass
  \item Forces of Nature
  \item Human error / Failure
  \item Information Extortion
  \item Sabotage / Vandalism
  \item Technical Hardware failures / Errors
  \item Technical Software failures / Errors
  \item Technological Obsolescence
\end{itemize}

\section{Compromises to Intellectual Property}

\dfn{Intellectual Property}{
  Creation, ownership, and control of original ideas, information, and creative works.
}

IP breaches compromise the \textbf{Confidentiality} of information, some examples are piracy, and copyright infringement.

\subsection{Technical Controls}

\begin{itemize}
  \item Watermarking
  \item Embedded Code
  \item Registration Keys
  \item Intentional Corruption
  \item Obfuscation
\end{itemize}

\subsection{Administrative/Legal Controls}

\begin{itemize}
  \item  End User License Agreements (EULA)
  \item Copyrights
  \item Licenses
\end{itemize}

\section{Deviations in Quality of Service}

Deviations in quality of service occur when products or services are not delivered as expected, this compromises the \textbf{Availability} of a system. This is commonly cause by:
\begin{itemize}
  \item Failure of interdependent support systems
  \item Supply Chain failures
  \item Failure of critical infrastructure (e.g power grid)
\end{itemize}

\subsection{Technical Controls}
\begin{itemize}
  \item Redundancy
  \item Failover Systems / Redundant Systems (e.g. Redundant Array of Independent Disks (RAID))
  \item Backup Internet Service Providers (ISPs)
  \item Load Balancing
  \item Cloud Geographic Distribution
\end{itemize}

\subsection{Administrative Controls}
\begin{itemize}
  \item Service Level Agreements (SLAs) - Contract between service provider and customer that specifies the level of service expected during its term.
  \item Policies and Procedures to deal with QoS issues
\end{itemize}

\section{Espionage / Trespass}

\dfn{Espionage / Spying}{
  Gaining unauthorized access to the information of an organization.
  The practice of obtaining information about an organization without the permission of the holder of the information.
}

\dfn{Trespass}{
  The act of entering someone's land or property without permission.
}

Espionage and Trespass primarily compromise the \textbf{Confidentiality}. Some examples include:
\begin{itemize}
  \item Shoulder Surfing - Observing a person's private information over their shoulder.
  \item Industrial Espionage
  \item Hacking
  \item Governmental Espionage
  \item Social Engineering
\end{itemize}

\section{Information Extortion / Cyberextortion}

An attacker steals information from a computer system and demands compensation for it's return or non-disclosure. Cyberextortion primarily compromises \textbf{Availability}, then \textbf{Integrity} and \textbf{Confidentiality} if data is exfiltrated. There are two types of ransomware attacks:
\begin{itemize}
  \item Lock screen - Prevents access to the device
  \item Encryption - Encryptions sensitive files
\end{itemize}


\subsection{Technical Controls}
\begin{itemize}
  \item Anti-Malware Software
  \item Email and Web Filtering
  \item Data Backups
\end{itemize}

\subsection{Administrative Controls}

\begin{itemize}
  \item User Training
  \item Incident Response Plan
  \item Regular Software Updates and Patch Management
\end{itemize}

\section{Sabotage / Vandalism}

\dfn{Sabotage}{
  The obstruction of a system's operations or intended functions. Sabotage is often directed with an explicit reason / goal.
}

\dfn{Vandalism}{
  The intentional destruction or defacement of property. The main intention of vandalism is destruction.
}

Involves the deliberate sabotage of a computer system or business, or acts of vandalism to destroy and asset or damage the image of an organization. This primarily compromises \textbf{Integrity} and \textbf{Availability}. Some examples include:

\begin{itemize}
  \item Website defacing - Erodes consumer confidence
  \item Hacktivism / Cyberactivism
  \item Cyberterrorism / Cyberwarfare.
\end{itemize}

\subsection{Technological Controls}

\begin{itemize}
  \item Web Application Firewalls (WAF)
  \item Distributed Denial of Service (DDoS) protection
  \item Website defacement monitoring
  \item Backups
\end{itemize}

\subsection{Administrative Controls}
\begin{itemize}
  \item Policies and Procedures
  \item Legal Enforcement
  \item Awareness Training
\end{itemize}

\subsection{Operational Controls}
\begin{itemize}
  \item Physical security of equipment.
  \item Host Redundancy
  \item Incident Response
  \item Monitoring
\end{itemize}

\section{Software Attacks}

\dfn{Malware}{
  Malicious software designed to infiltrate or damage a computer system without the owner's informed consent.
}

Involve designing and deploying malware to compromise a system. Software attacks may overwhelm the processing capabilities of online systems or allow access to protected systems by hidden means. Software attacks primarily compromise \textbf{Confidentiality}, \textbf{Integrity}, and \textbf{Availability} to varying severities. Some examples include:
\begin{description}
  \item[Traditinoal Malware] Viruses, Worms, Trojans, Spyware, Adware, Ransomware, Rootkits, Keyloggers
  \item[Access/Control Malware] Backdoors, Botnets, Remote Access Trojans (RATs)
  \item[Disruption Attacks] Denial of Service (DoS), Distributed Denial of Service (DDoS)
  \item[Interception Attacks] Spoofing, Man-In-the-Middle (MitM), Session Hijacking, Packet sniffing, pharming
\end{description}

\dfn{Viruses}{
  A type of malware that attaches itself to a legitimate program or file and spreads to other programs and files when executed.
}

\dfn{Worm}{
  A type of malware that can replicate itself and spread independently without needing to attach to a host program.
}

\dfn{Trojan}{
  A type of malware that disguises itself as a legitimate program or file to trick users into installing it, often creating a backdoor for unauthorized access.
}

\dfn{Logic Bomb}{
  A type of malware that is triggered by a specific event or condition, such as a date or the deletion of a file.
}

\dfn{Zero-Day Attack}{
  An attack that exploits a previously unknown vulnerability in a computer application, meaning there is no existing patch or fix for the vulnerability.
}

\dfn{Pharming}{
  A cyberattack that redirects a website's traffic to a fraudulent website, often through DNS cache poisoning or modifying the hosts file on a victim's computer.
}

\dfn{Phising}{
  A cyberattack that uses fraudulent emails or websites to trick individuals into revealing sensitive information, such as passwords or credit card numbers.
}

\subsection{Technical Controls}

\begin{itemize}
  \item Anti-Virus / Anti-Malware Software
  \item Firewalls
  \item Intrusion Detection Systems (IDS) / Intrusion Prevention Systems (IPS)
  \item Patch Management
  \item Secure Software Development Practices
\end{itemize}

\subsection{Administrative Controls}

\begin{itemize}
  \item Security Policies on patching
  \item User Training
  \item Software Inventory Management
\end{itemize}

\subsection{Operational Controls}

\begin{itemize}
  \item Incident Response Planning and Policies
  \item  Malware Analysis Teams
  \item Threat Intelligence Sharing
  \item Backups
\end{itemize}

\section{Technical Hardware Failures / Errors}

Occurs when a manufacturer distributes equipment containing a known or unknown flaw. This primarily compromises \textbf{Availability} and \textbf{Integrity}. Some examples include:
\begin{itemize}
  \item Pentium II Floating Point Division Bug
  \item Spectre and Meltdown CPU Vulnerabilities
  \item Hardware crashes
  \item RAM failures
\end{itemize}

\subsection{Technical Controls}

\begin{itemize}
  \item Redundancy (e.g RAID)
  \item Error Checking RAM
  \item Uninterruptible Power Supplies (UPS)
\end{itemize}

\subsection{Administrative Controls}

\begin{itemize}
  \item Vendor Risk Assessment
  \item Procurement Policies
  \item Warranty and Maintenance Agreements
\end{itemize}

\subsection{Operational Controls}

\begin{itemize}
  \item Preventative Maintenance Schedules
  \item Hardware Monitoring Tools
  \item Asset Lifecycle Management
\end{itemize}


\end{document}
