\documentclass[12pt letter]{report}
\input{./template/preamble}
\input{./template/macros}
\input{./template/letterfonts}

\title{\Huge{Systems of Linear Equations}}
\author{\huge{Madiba Hudson-Quansah}}
\date{}
\usepackage{parskip}

\setcounter{tocdepth}{4}
\setcounter{secnumdepth}{4}

\begin{document}
\maketitle
\newpage
\pdfbookmark[section]{\contentsname}{too}
\tableofcontents
\pagebreak

\chapter{Introduction}

\dfn{Linear Equation}{
  An equation in the form
  \[
    a_1x_1 + a_2x_2 + \ldots + a_nx_n = b
  \]
  where the constant $b$ and coefficients $a_1, a_2, \ldots, a_n$  are real or complex numbers.
}

\dfn{System of Linear Equations}{
  A collection of one or more linear equations involving the same set of variables. When a system of linear equations is written in the form
  \begin{align*}
    a_1 x_1 + a_2 x_2 + a_3 x_3 = b_1 \\
    a_4 x_1 + a_6 x_3 = b_2           \\
  \end{align*}
  The set of variables takes on the longest subscript in the system. In this case, the variables are $x_1, x_2, x_3$.
}

\dfn{Solution of a System of Linear Equations}{
  The \textit{solution} of a system of linear equations is a list of values, $\left( s_1, s_2, \ldots, s_n \right) $
  that makes each equation in the system a true statement when the values are substituted for the variables, i.e. $x_1,
    x_2, \ldots, x_n$ and $s_1, s_2, \ldots, s_n$, where $s_n$ is substituted for $x_n$
}

\dfn{Solution Set}{
  The set of all possible solutions of a system of linear equations.
}

\dfn{Equivalence}{
  Two linear systems are said to be \textit{equivalent} if they have the same solution set.
}

\dfn{Consistency}{
  A system of linear equations is said to be \textit{consistent} if it has at least one solution, and
  \textit{inconsistent} if it has no solution.
}

A system of linear equations can either have:
\begin{itemize}
  \item No solution - Equations do not intersect
  \item Exactly one / Unique solution - Equations intersect at a single point
  \item Infinitely many solutions - Equations are the same
\end{itemize}

\section{Matrix Notation}

A system of linear equations can be represented in matrix form two ways:
\begin{itemize}
  \item Coefficient Matrix
  \item Augmented Matrix
\end{itemize}

\subsection{Coefficient Matrix}

\dfn{Coefficient Matrix}{
  Denoted by $A$, the coefficient matrix is a matrix that contains the coefficients of the variables in the system of
  linear equations with the coefficients of each equation making up each row.
}

\ex{}{
  For the system of linear equations:
  \begin{align*}
    a_1 x_2 + a_2 x_2 + a_3 x_3 = b_1 \\
    a_4 x_1 + a_5 x_2 + a_6 x_3 = b_2 \\
    a_7 x_1 + a_8 x_2 + a_9 x_3 = b_3 \\
  \end{align*}
  The coefficient matrix is:
  \[
    \begin{bmatrix} a_1 & a_2 & a_3 \\ a_4 & a_5 & a_6 \\  a_7 & a_8 & a_9 \end{bmatrix}
  \]

}

\subsection{Augmented Matrix}
\dfn{Augmented Matrix}{
  Denoted by $[A|B]$, the augmented matrix is a matrix that contains the coefficients of the variables in the system of linear equations with the constant terms of each equation making up the last column.
}

\ex{}{
  For the system of linear equations:
  \begin{align*}
    a_1 x_2 + a_2 x_2 + a_3 x_3 = b_1 \\
    a_4 x_1 + a_5 x_2 + a_6 x_3 = b_2 \\
    a_7 x_1 + a_8 x_2 + a_9 x_3 = b_3 \\
  \end{align*}
  The augmented matrix is:
  \[
    \begin{bmatrix} a_1 & a_2 & a_3 & b_1\\ a_4 & a_5 & a_6 & b_2 \\ a_7 & a_8 & a_9  & b_3 \end{bmatrix}
  \]
}

\dfn{Size of a Matrix}{
  The size of a matrix, denoted by $m \times n$, is the number of rows and columns in the matrix respectively.
  If $n = m$ then the matrix is said to be square, if not, it is said to be rectangular.
}

\section{Solving Linear Systems}

\dfn{Pivot}{
  Diagonal non-zero elements  of a linear system
}

\dfn{Forward Elimination Process}{
  The process used to change a system into an upper triangular matrix
}
\dfn{Backward Substitution Method}{
  The process of deriving a solution from an upper triangular matrix
}

\dfn{Identity Matrix}{
  A matrix containing all zeros with pivots of 1
  \[
    \begin{bmatrix} 1 & 0 & 0 \\ 0 & 1 & 0 \\ 0 & 0 & 1 \end{bmatrix}
  \]
}



One procedure used to solve linear systems is that of \textit{simplification}. This involves replacing one linear system
with a simpler equivalent system. This is done by applying the following operations to the system:
\begin{description}
  \item[Replacement] Replace one equation by the sum of itself and a multiple of another equation.
  \item[Interchange] Interchange two equations.
  \item[Scaling] Multiply all the terms in an equation by a non-zero constant.
\end{description}

\ex{}{
  \qs{}{
    Solve the system
    \begin{align*}
      x_1    -2x_2   + x_3    = 0 \\
      2x_2    - 8x_3   = 8        \\
      5x_1           -5x_3    = 10
    \end{align*}
  }

  \sol{
    Using the augmented matrix representation, we have:
    \[
      \begin{bmatrix} 1 & -2 & 1 & 0 \\ 0 & 2 & -8 & 8 \\ 5 & 0 & -5 &10   \end{bmatrix}
    \]
    Then we times the first equation through by $-5$ and add it to the third equation to replace the third equation:
    \begin{align*}
      \begin{split}
        -5x_1 + 10x_2 - 5x_3 = 0 \\
        5x_1 - 5x_3 = 10         \\
        \hline
        10x_2 - 10x_3 = 10       \\
      \end{split}
    \end{align*}
    Giving us:
    \[
      \begin{bmatrix} 1 & -2 &  1 &  0  \\ 0 & 2 & -8 & 8 \\ 0 & 10 & -10 & 10  \end{bmatrix}
    \]
    We then eliminate $x_2$ by multiplying equation 2 by $-5$ and add it again to the third equation again replacing it:
    \begin{align*}
      \begin{split}
        -10x_2 + 40x_3 = -40 \\
        10x_2 - 10x_3 = 10   \\
        \hline
        30x_3 = -30          \\
      \end{split}
    \end{align*}
    Giving us:
    \[
      \begin{bmatrix} 1 & -2 & 1 & 0 \\ 0 & 2 & -8 & 8 \\ 0 & 0 & 30 & -30 \end{bmatrix}
    \]
    This new system has a triangular form, i.e.
    \begin{align*}
      x_1 - 2x_2 + x_3 = 0 \\
      2x_2 - 8x_3 = 8      \\
      30x_3 = -30          \\
    \end{align*}

    \noindent We then continue eliminating variables until one remains in each equation:
    \begin{align*}
      \begin{split}
        -x_3 = 1             \\
        x_1 - 2x_2 + x_3 = 0 \\
        \hline
        x_1 - 2x_2 = 1
      \end{split}
    \end{align*}
    \[
      \begin{bmatrix} 1 & -2 & 0 & 1 \\ 0 & 2 & -8 & 8 \\ 0 & 0 & 30  &-30 \end{bmatrix}
    \]


    \begin{align*}
      \begin{split}
        8x_3 = -8       \\
        2x_2 - 8x_3 = 8 \\
        \hline
        2x_2 = 0        \\
      \end{split}
    \end{align*}

    \[
      \begin{bmatrix} 1 & -2 & 0 & 1 \\ 0 & 2& 0 & 0 \\ 0 & 0 & 30 & -30  \end{bmatrix}
    \]

    \begin{align*}
      \begin{split}
        2x_2 = 0       \\
        x_1 - 2x_2 = 1 \\
        \hline
        x_1 = 1        \\
      \end{split}
    \end{align*}

    \[
      \begin{bmatrix} 1 & 0 & 0 & 1 \\ 0 & 2 & 0 & 0 \\ 0 & 0 & 30 & -30  \end{bmatrix}
    \]

    Giving us the system:
    \begin{align*}
      x_1 = 1     \\
      2x_2 = 0    \\
      30x_3 = -30 \\
    \end{align*}
    Which simplifies into:
    \begin{align*}
      x_1 = 1 \\
      x_2 = 0 \\
      x_3 = -1
    \end{align*}

  }
}

\dfn{Row Equivalence}{
  Two matrices are row equivalent if there is a sequence of elementary row operations that transforms one matrix into
  the other
}

\thm{}{
  If the augmented matrices of two linear systems are row equivalent, then the two equations have the same solution set.
}

\section{Identifying Existence and Uniqueness}

To determine the nature of a linear system we must answer two fundamental questions:
\begin{itemize}
  \item Is the system consistent? / Does a solution exist?
  \item If a solution exists, is it the only one? / Is the solution unique
\end{itemize}

\ex{}{
  \qs{}{
    Determine if the following system is consistent:
    \begin{align*}
      x_1    -2x_2   + x_3    = 0 \\
      2x_2    - 8x_3   = 8        \\
      5x_1           -5x_3    = 10
    \end{align*}
  }

  \sol{
    Having already found the solution for this system:
    \begin{align*}
      x_1 = 1 \\
      x_2 = 0 \\
      x_3 = -1
    \end{align*}
    We can determine that a solution exists, and due to the fact $x_2$ is uniquely determined by equation two, $x_3$ has
    only one possible value, and $x_1$ is also uniquely determined by equation one, we can also conclude this solution
    is unique.
  }
}

\ex{}{
  \qs{}{
    Determine if the following system is consistent:
    \begin{align*}
      x_2 - 4x_3 = 8         \\
      2x_1 - 3x_2 + 2x_3 = 1 \\
      4x_1 - 8x_2 + 12x_3 = 1
    \end{align*}
  }

  \sol{
    The augmented matrix is:
    \[
      \begin{bmatrix}  0 & 1 & -4 & 8 \\ 2 & -3 & 2 & 1 \\  4 & -8 & 12 & 1 \end{bmatrix}
    \]

    We interchange equations 1 and 2:
    \[
      \begin{bmatrix} 2 & -3 & 2 & 1 \\ 0 & 1 & -4 & 8 \\ 4 & -8 & 12 & 1 \end{bmatrix}
    \]

    \begin{align*}
      \begin{split}
        -4x_1 + 6x_2 - 4x_3 = -2 \\
        4x_1 - 8x_2 + 12x_3 = 1  \\
        \hline
        -2x_2 + 8x_3 = -1
      \end{split}
    \end{align*}

    \[
      \begin{bmatrix} 2 & -3 & 2 & 1 \\ 0 & 1 & -4 & 8 \\ 0 & -2 & 8 & -1 \end{bmatrix}
    \]

    \begin{align*}
      \begin{split}
        2x_2 - 8x_3 = 16  \\
        -2x_2 + 8x_3 = -1 \\
        \hline
        0 = 15
      \end{split}
    \end{align*}

    \[
      \begin{bmatrix} 2 & -3 & 2 & 1 \\ 0 & 1 & -4 & 8 \\ 0 & 0 & 0 & 15 \end{bmatrix}
    \]

    Now in its triangular form, we can determine the existence and uniqueness of the solutions:
    \begin{align*}
      2x_1 - 3x_2 + 2x_3 = 1 \\
      x_2 - 4x_3 = 8         \\
      0 = 15                 \\
    \end{align*}

    Since there are no coefficients for $x_1$, $x_2$, and $x_3$ in equation 3 equation 3 has no solution. This makes the
    solution set for this linear system $\{1, 8\}$. Because this set is the same as the solution set for the original
    linear system, $\{8, 1, 1\} $, the original system is inconsistent


  }
}



\section{Exercises}

\qs{}{
  Determine if the linear system represented by the augmented matrix below is consistent:
  \[
    \begin{bmatrix} 1 & 5 & 2 & -6 \\ 0 & 4 & - 7 & 2 \\ 0 & 0 & 5 & 0 \end{bmatrix}
  \]
}

\sol{
  \begin{align*}
    x_1 + 5x_2 + 2x_3 = -6                   \\
    4x_2 -7x_3 = 2                           \\
    5x_3 = 0                                 \\
    \\
    x_3 = 0                                  \\
    x_1 + 5x_2 = -6                          \\
    x_1 = -6 - 5x_2                          \\
    \\
    4x_2 = 2                                 \\
    x_2 = \frac{1}{2}                        \\
    \\
    x_1  = - 6- 5 \left( \frac{1}{2} \right) \\
    x_1 = -\frac{17}{2}
  \end{align*}

}

\qs{}{
  Solve the following systems:
  \begin{enumerate}
    \item
          \begin{align*}
            x_2 + 4x_3 = -5        \\
            x_1 + 3x_2 + 5x_3 = -2 \\
            3x_1 + 7x_2 + 7x_3 = 6
          \end{align*}
    \item
          \begin{align*}
            x_1 - 2x_4 = -3 \\
            2x_2 + 2x_3 = 0 \\
            x_3 + 3x_4 = 1  \\
            -2x_1 + 3x_2 + 2x_3 + x_4 = 5
          \end{align*}
  \end{enumerate}

}

\sol{
  \begin{enumerate}
    \item
          \[
            \begin{bmatrix} 0 & 1 & 4 & -5 \\ 1&3 & 5 & -2 \\ 3 & 7 & 7 & 6 \end{bmatrix}
          \]
          $R_1 \leftrightarrow R_3$
          \[
            \begin{bmatrix} 3 & 7 & 7 & 6 \\ 1&3 & 5 & -2 \\  0 & 1 & 4 & -5 \end{bmatrix}
          \]
          $\frac{1}{3}R_1 - R_2 \to R_2$
          \begin{align*}
            \begin{split}
              x_1 + \frac{7}{3}x_2 + \frac{7}{3}x_3 = 2 \\
              x_1 + 3x_2 + 5x_3 = -2                    \\
              \hline
              -\frac{2}{3}x_2 - \frac{8}{3}x_3 = 4
            \end{split}
          \end{align*}

          \[
            \begin{bmatrix} 3 & 7 & 7 & 6 \\ 0&-\frac{2}{3} & -\frac{8}{3} & 4 \\  0 & 1 & 4 & -5 \end{bmatrix}
          \]
          $3R_2$
          \[
            \begin{bmatrix} 3 & 7 & 7 & 6 \\ 0&-2 & -8 & 12 \\  0 & 1 & 4 & -5 \end{bmatrix}
          \]

          $-\frac{1}{2}R_2 - R_3 \to R_3$
          \begin{align*}
            \begin{split}
              x_2 + 4x_3 = -6 \\
              x_2 + 4x_3 = -5 \\
              \hline
              0 = -1          \\
            \end{split}
          \end{align*}

          \[
            \begin{bmatrix} 3 & 7 & 7 & 6 \\ 0&-2 & -8 & 12 \\  0 & 0 & 0 & -1 \end{bmatrix}
          \]

          Because the system has a contradiction in row 3, $0x_1 + 0x_2 + 0x_3 = -1$, the system has no solution and is
          therefore inconsistent.

    \item
          \[
            \begin{bmatrix} 1 & 0 & 0 & -2 & -3 \\ 0 & 2 & 2 & 0 & 0 \\ 0 & 0 & 1 & 3 & 1 \\ -2 & 3 & 2 & 1 & 5 \end{bmatrix}
          \]

          $R_1 \leftrightarrow R_4$
          \[
            \begin{bmatrix}
              -2 & 3 & 2 & 1  & 5  \\
              0  & 2 & 2 & 0  & 0  \\
              0  & 0 & 1 & 3  & 1  \\
              1  & 0 & 0 & -2 & -3 \\
            \end{bmatrix}
          \]

          $-\frac{1}{2} R_1 - R_4 \to R_4$
          \begin{align*}
            \begin{split}
              x_1 - \frac{3}{2}x_2 - x_3 - \frac{1}{2}x_4 = -\frac{5}{2} \\
              x_1 + 0 + 0 - 2x_4 = -3                                    \\
              \hline
              -\frac{3}{2}x_2 - x_3 + \frac{3}{2}x_4 = \frac{1}{2}
            \end{split}
          \end{align*}
          \[
            \begin{bmatrix}
              -2 & 3            & 2  & 1           & 5           \\
              0  & 2            & 2  & 0           & 0           \\
              0  & 0            & 1  & 3           & 1           \\
              0  & -\frac{3}{2} & -1 & \frac{3}{2} & \frac{1}{2} \\
            \end{bmatrix}
          \]

          $-\frac{3}{4}R_2 - R_4 \to R_4$
          \begin{align*}
            \begin{split}
              0  -\frac{3}{2}x_2 - \frac{3}{2}x_3 + 0 = 0             \\
              0 - \frac{3}{2}x_2 - x_3 + \frac{3}{2}x_4 = \frac{1}{2} \\
              \hline
              -\frac{1}{2}x_3 - \frac{3}{2}x_4 = -\frac{1}{2}
            \end{split}
          \end{align*}

          \[
            \begin{bmatrix}
              -2 & 3 & 2            & 1            & 5            \\
              0  & 2 & 2            & 0            & 0            \\
              0  & 0 & 1            & 3            & 1            \\
              0  & 0 & -\frac{1}{2} & -\frac{3}{2} & -\frac{1}{2} \\
            \end{bmatrix}
          \]

          $-\frac{1}{2}R_3 - R_4 \to R_4$

          \begin{align*}
            \begin{split}
              0 + 0 - \frac{1}{2}x_3 - \frac{3}{2}x_4 = -\frac{1}{2} \\
              0 + 0-\frac{1}{2}x_3 - \frac{3}{2}x_4 = -\frac{1}{2}   \\
              \hline
              0x_1 + 0x_2 + 0x_3 + 0x_4 = 0
            \end{split}
          \end{align*}

          \[
            \begin{bmatrix}
              -2 & 3 & 2 & 1 & 5 \\
              0  & 2 & 2 & 0 & 0 \\
              0  & 0 & 1 & 3 & 1 \\
              0  & 0 & 0 & 0 & 0 \\
            \end{bmatrix}
          \]

          $\frac{3}{2}R_2 - R_1 \to R_1$

          \begin{align*}
            \begin{split}
              0 + 3x_2 + 3x_3 + 0 = 0      \\
              -2x_1 + 3x_2+ 2x_3 + x_4 = 5 \\
              \hline
              2x_1 + 0 + x_3 - x_4 = -5
            \end{split}
          \end{align*}

          \[
            \begin{bmatrix}
              2 & 0 & 1 & -1 & -5 \\
              0 & 2 & 2 & 0  & 0  \\
              0 & 0 & 1 & 3  & 1  \\
              0 & 0 & 0 & 0  & 0  \\
            \end{bmatrix}
          \]
          $R_3 - R_1 \to R_1$
          \begin{align*}
            \begin{split}
              0 + 0 + x_3 + 3x_4 = 1    \\
              2x_1 + 0 + x_3 - x_4 = -5 \\
              \hline
              -2x_1 + 0 + 0 + 4x_4 = 6
            \end{split}
          \end{align*}

          \[
            \begin{bmatrix}
              -2 & 0 & 0 & 4 & 6 \\
              0  & 2 & 2 & 0 & 0 \\
              0  & 0 & 1 & 3 & 1 \\
              0  & 0 & 0 & 0 & 0 \\
            \end{bmatrix}
          \]
          $2R_3 - R_2 \to R_2$
          \begin{align*}
            \begin{split}
              0 + 0 + 2x_3 + 6x_4 = 2 \\
              0 + 2x_2 + 2x_3 + 0 = 0 \\
              \hline
              -2x_2 + 6x_4 = 2
            \end{split}
          \end{align*}

          \[
            \begin{bmatrix}
              -2 & 0  & 0 & 4 & 6 \\
              0  & -2 & 0 & 6 & 2 \\
              0  & 0  & 1 & 3 & 1 \\
              0  & 0  & 0 & 0 & 0 \\
            \end{bmatrix}
          \]
          $\frac{R_1}{-2}$
          \[
            \begin{bmatrix}
              1 & 0  & 0 & -2 & -3 \\
              0 & -2 & 0 & 6  & 2  \\
              0 & 0  & 1 & 3  & 1  \\
              0 & 0  & 0 & 0  & 0  \\
            \end{bmatrix}
          \]
          $\frac{R_2}{-2}$
          \[
            \begin{bmatrix}
              1 & 0 & 0 & -2 & -3 \\
              0 & 1 & 0 & -3 & -1 \\
              0 & 0 & 1 & 3  & 1  \\
              0 & 0 & 0 & 0  & 0  \\
            \end{bmatrix}
          \]

          \begin{align*}
            x_1 + -2x_4 = -3                   \\
            x_2 -3x_4 = -1                     \\
            x_3 + 3x_4 = 1                     \\
            0 = 0                              \\
            \\
            x_4 = \frac{1}{2}x_1 + \frac{3}{2} \\
            x_2 = -1 + 3x_4                    \\
            x_3 = 1 - 3x_4                     \\
            x_1 = -3 + 2x_4
          \end{align*}

  \end{enumerate}
}

\pagebreak
\qs{}{
  For the following matrices find the elementary row operation that transforms
  the first matrix into the second, and then find the reverse row operation that
  transforms the second matrix into the first
  \begin{enumerate}
    \item
          \[
            \begin{bmatrix}
              1 & 3   & -1 \\
              0 & -2  & 6  \\
              0 & - 5 & 9
            \end{bmatrix} ,
            \begin{bmatrix}
              1 & 3  & -4 \\
              0 & 1  & -3 \\
              0 & -5 & 9
            \end{bmatrix}
          \]
    \item
          \[
            \begin{bmatrix}
              1 & -2 & 1  & 0  \\
              0 & 5  & -2 & 8  \\
              4 & -1 & 3  & -6
            \end{bmatrix} ,
            \begin{bmatrix}
              1 & -2 & 1  & 0  \\
              0 & 5  & -2 & 8  \\
              0 & 7  & -1 & -6
            \end{bmatrix}
          \]
  \end{enumerate}
}

\sol{
  \begin{enumerate}
    \item
          Let the first matrix be $M_1$ and the second be $M_2$
          \begin{align*}
            M_1 \to M_2 & = -\frac{1}{2}R_2          \\
            M_2 \to M_1 & = \frac{R_2}{-\frac{1}{2}} \\
          \end{align*}
    \item
          Let the first matrix be $M_1$ and the second be $M_2$
          \begin{align*}
            M_1 \to M_2 & = -4 R_1 + R_3 \to R_3 \\
            M_2 \to M_1 & = R_3 - 4R_1           \\
          \end{align*}
  \end{enumerate}
}

\chapter{Row Reduction and Echelon Forms}
\dfn{Leading Entry}{
  The leftmost entry in a non-zero row
}

\dfn{Upper triangular matrix / Echelon Form}{
  A rectangular matrix is in \textit{echelon form / row echelon form}  if it has the following proprieties:
  \begin{itemize}
    \item All non-zero rows are above any rows of all zeros
    \item Each leading entry of a row is in a column to the right of the leading entry of the row above it
    \item All entries in a column below a leading entry are zeros
  \end{itemize}
}

\dfn{Reduced Row Echelon Form}{
  A matrix is in \textit{reduced row echelon form} if it meets all the conditions of a matrix in echelon form and:
  \begin{itemize}
    \item The leading entry in each non-zero row is 1
    \item Each leading 1 is the only non-zero entry in its column
  \end{itemize}
}

\dfn{Echelon Matrix}{
  A matrix that is in echelon form
}

\dfn{Reduced Echelon Matrix}{
  A matrix that is in reduced row echelon form
}


\thm{Uniqueness of a Row Reduced Echelon Form}{
  Each matrix is row equivalent to one and only one reduced row echelon form
}

Therefore if matrix $A$ is row equivalent to an echelon matrix $U$, $U$ is the \textbf{echelon form of} $A$, and if $A$ is row equivalent to a reduced echelon matrix $R$, $R$ is the \textbf{reduced echelon form of} $A$.

\section{Pivot Positions}

\dfn{Pivot Position}{
  A \textit{pivot position} in matrix $A$ is a location in $A$ that corresponds to a leading 1 in the reduced echelon
  form of $A$.
}
\dfn{Pivot Column}{
  A column of a matrix $A$ that contains a pivot position.
}

A pivot cannot be 0 and a pivot column cannot contain any other non-zero entries. Therefore when identifying pivot
positions we look for the first non-zero entry in each row, that is not in a column that already contains a pivot.

\ex{}{
  For the reduced echelon matrix:
  \[
    \begin{bmatrix} 1 & 4 & 5 & -9 & -7 \\ 0 & 2 & 4 & -6 & -6 \\ 0 & 0 & 0 & -5 & 0 \\  0 & 0 & 0 & 0 & 0 \end{bmatrix}
  \]

  The pivots positions are $\left( 1,1 \right), \left( 2,2 \right), \left( 3,4 \right)   $ and values are $\left( 1,2,-5 \right) $
}

\section{The Row Reduction Algorithm}

The row Reduction Algorithm reduces a matrix to it's echelon form in four steps with the fifth reducing the echelon
matrix to a reduced echelon matrix. These steps are:
\begin{enumerate}
  \item Start with the leftmost non-zero column, this is the first pivot column, with its pivot position at the top of
        the column
  \item Select a non-zero entry in the pivot column as a pivot, interchanging rows if necessary
  \item Use row replacement operations to create all zeros below the pivot.
  \item Cover the row containing the pivot and repeat the process for the sub matrix that remains.
  \item Beginning with the rightmost pivot and working upward to the left, create zeros above each pivot. If the pivot
        is not 1 make it 1 via a scaling operation
\end{enumerate}

\ex{}{
  Given the matrix:
  \[
    \begin{bmatrix}
      0 & 3  & -6 & 6  & 4 & 5  \\
      3 & -7 & 8  & -5 & 8 & 9  \\
      3 & -9 & 12 & -9 & 6 & 15
    \end{bmatrix}
  \]

  We interchange the first and second rows to have the first item in the leftmost column be non-zero:
  \\
  $R_1 \leftrightarrow R_2$
  \[
    \begin{bmatrix}
      3 & -7 & 8  & -5 & 8 & 9  \\
      0 & 3  & -6 & 6  & 4 & 5  \\
      3 & -9 & 12 & -9 & 6 & 15
    \end{bmatrix}
  \]
  We then select the non-zero entry in the first column, 3, as the pivot for the first column, and use it to render zeros
  in the first column below the pivot
  $R_1 - R_3 \to R_3$
  \begin{align*}
    \begin{split}
      3x_1 - 7x_2 + 8x_3 - 5x_4 + 8x_5 = 9 \\
      3x_1 - 9x_2 + 12x_3 -9x_4 +6x_5 = 15 \\
      \hline
      2x_2 - 4x_3 + 4x_4 + 2x_5 = -6       \\
    \end{split}
  \end{align*}

  \[
    \begin{bmatrix}
      3 & -7 & 8  & -5 & 8 & 9  \\
      0 & 3  & -6 & 6  & 4 & 5  \\
      0 & 2  & -4 & 4  & 2 & -6
    \end{bmatrix}
  \]
  And repeat the process for the remaining columns: \\
  \[
    \frac{2}{3}R_2 - R_3 \to R_3
  \]
  \begin{align*}
    \begin{split}
      0 + 2x_2 - 4x_3 + 4x_4 + \frac{8}{3}x_5 = \frac{10}{3} \\
      0 + 2x_2 - 4x_3 + 4x_4 + 2x_5 = -6                     \\
      \hline
      0 + 0 + 0 + 0 + \frac{2}{3}x_5 = \frac{28}{3}
    \end{split}
  \end{align*}
  \[
    \begin{bmatrix}
      3 & -7 & 8  & -5 & 8           & 9            \\
      0 & 3  & -6 & 6  & 4           & 5            \\
      0 & 0  & 0  & 0  & \frac{2}{3} & \frac{28}{3}
    \end{bmatrix}
  \]

  \[
    6R_3 - R_2 \to R_2
  \]
  \begin{align*}
    \begin{split}
      0 + 0 + 0 + 0 + 4x_5 = 56         \\
      0 + 3x_2 - 6x_3 + 6x_4 + 4x_5 = 5 \\
      \hline
      -3x_2 + 6x_3 -6x_4 = 51
    \end{split}
  \end{align*}

  \[
    \begin{bmatrix}
      3 & -7 & 8 & -5 & 8           & 9            \\
      0 & -3 & 6 & -6 & 0           & 51           \\
      0 & 0  & 0 & 0  & \frac{2}{3} & \frac{28}{3}
    \end{bmatrix}
  \]

  \[
    12R_3 - R_1 \to R_1
  \]
  \begin{align*}
    \begin{split}
      0 + 0 + 0 + 0 + 8x_5 = 612          \\
      3x_1 -7x_2 + 8x_3 - 5x_4 + 8x_5 = 9 \\
      \hline
      -3x_1 + 7x_2 - 8x_3 + 5x_4  = 603
    \end{split}
  \end{align*}
  \[
    \begin{bmatrix}
      -3 & 7  & -8 & 5  & 0           & 603          \\
      0  & -3 & 6  & -6 & 0           & 51           \\
      0  & 0  & 0  & 0  & \frac{2}{3} & \frac{28}{3}
    \end{bmatrix}
  \]

  \[
    -\frac{7}{3}R_2 - R_1 \to R_1
  \]

  \begin{align*}
    \begin{split}
      0 + 7x_2 -14x_3  + 14x_4 + 0 = -119  \\
      -3x_1 + 7x_2 - 8x_3 + 5x_4 + 0 = 603 \\
      \hline
      3x_1 + 0 -6x_3 + 9x_4 = -722
    \end{split}
  \end{align*}
  \[
    \begin{bmatrix}
      3 & 0  & -6 & 9  & 0           & -722         \\
      0 & -3 & 6  & -6 & 0           & 51           \\
      0 & 0  & 0  & 0  & \frac{2}{3} & \frac{28}{3}
    \end{bmatrix}
  \]
  \[
    \frac{3}{2}R_3 \to R_3
  \]
  \[
    \begin{bmatrix}
      3 & 0  & -6 & 9  & 0 & -722 \\
      0 & -3 & 6  & -6 & 0 & 51   \\
      0 & 0  & 0  & 0  & 1 & 14   \\
    \end{bmatrix}
  \]

  \[
    -\frac{1}{3}R_2 \to R_2
  \]
  \[
    \begin{bmatrix}
      3 & 0 & -6 & 9 & 0 & -722 \\
      0 & 1 & -2 & 2 & 0 & -17  \\
      0 & 0 & 0  & 0 & 1 & 14   \\
    \end{bmatrix}
  \]

  \[
    \frac{1}{3}R_1 \to R_1
  \]
  \[
    \begin{bmatrix}
      1 & 0 & -2 & 3 & 0 & -\frac{722}{3} \\
      0 & 1 & -2 & 2 & 0 & -17            \\
      0 & 0 & 0  & 0 & 1 & 14             \\
    \end{bmatrix}
  \]
}

\section{Solutions of Linear Systems}

\dfn{Free Variable}{
  A variable that does not exist in a row of a matrix
}

\dfn{Parametric Equations}{
  Any equation that expresses the variables in a system of linear equations in terms of a free variable.
}

\dfn{Basic Variable}{
  A variable that exists in a row of a matrix
}

In deriving a reduced echelon matrix we can determine the consistency of a system of linear equations, and thereby
describe the solution set. For example in the case below we have the reduced echelon matrix:
\begin{align*}
  \begin{bmatrix}
    1 & 0 & -5 & 1 \\
    0 & 1 & 1  & 4 \\
    0 & 0 & 0  & 0
  \end{bmatrix}
  \\
  \\
  x_1 - 5x_3 = 1 \\
  x_2 + x_3 = 4  \\
  0 = 0
\end{align*}
There are three variables $x_1, x_2, \text{ and } x_3$, as there are four columns in this augmented matrix. The
variables $x_1$ and $x_2$ are referred to as basic variables as they have pivots in their columns, and $x_3$ is a free
variable as it does not have a pivot in its column, thus this system is consistent has has infinitely many solutions.
Describing the solution set of this equation we can say:
\begin{align*}
  \begin{cases}
    x_1 = 1 + 5x_3 \\
    x_2 = 4 - x_3  \\
    x_3 \text{ is free}
  \end{cases}
\end{align*}
The statement "$x_3$ is free" means that $x_3$ can take on any value, and by extension the values of $x_1$ and $x_2$
are determined by the value $x_3$ takes on.

\ex{}{
  \qs{}{
    Find the general solution of the linear system whose augmented matrix has been reduced to:
    \[
      \begin{bmatrix}
        1 & 6 & 2 & -5 & -2 & -4 \\
        0 & 0 & 2 & -8 & -1 & 3  \\
        0 & 0 & 0 & 0  & 1  & 7
      \end{bmatrix}
    \]
  }

  \sol{
    \[
      \begin{bmatrix}
        1 & 6 & 2 & -5 & -2 & -4 \\
        0 & 0 & 2 & -8 & -1 & 3  \\
        0 & 0 & 0 & 0  & 1  & 7
      \end{bmatrix}
    \]

    \[
      -1R_3 - R_2 \to R_2
    \]
    \begin{align*}
      \begin{split}
        0 + 0 + 0 + 0 - 1 = -7   \\
        0 + 0 + 2 - 8 - 1 = -3   \\
        \hline
        0 + 0 + -2 + 8 + 0 = -10 \\
      \end{split}
    \end{align*}
    \[
      \begin{bmatrix}
        1 & 6 & 2  & -5 & -2 & -4  \\
        0 & 0 & -2 & 8  & 0  & -10 \\
        0 & 0 & 0  & 0  & 1  & 7
      \end{bmatrix}
    \]

    \[
      -2R_3 - R_1 \to R_1
    \]
    \begin{align*}
      \begin{split}
        0 + 0 + 0 + 0 -2 = -14 \\
        1 + 6 + 2 - 5 - 2 = -4 \\
        \hline
        -1 - 6 - 2 + 5 + 0 = -10
      \end{split}
    \end{align*}
    \[
      \begin{bmatrix}
        -1 & -6 & -2 & 5 & 0 & -10 \\
        0  & 0  & -2 & 8 & 0 & -10 \\
        0  & 0  & 0  & 0 & 1 & 7
      \end{bmatrix}
    \]
    \[
      R_2 - R_1 \to R_1
    \]

    \begin{align*}
      \begin{split}
        0 + 0 -2 + 8 + 0 = -10 \\
        -1 -6 -2 + 5 + 0 = -10 \\
        \hline
        1 + 6 + 0 + 3 + 0 = 0
      \end{split}
    \end{align*}
    \[
      \begin{bmatrix}
        1 & 6 & 0  & 3 & 0 & 0   \\
        0 & 0 & -2 & 8 & 0 & -10 \\
        0 & 0 & 0  & 0 & 1 & 7
      \end{bmatrix}
    \]

    \begin{align*}
      -\frac{1}{2}R_2 \to R_2 \\
      \begin{bmatrix}
        1 & 6 & 0 & 3  & 0 & 0 \\
        0 & 0 & 1 & -4 & 0 & 5 \\
        0 & 0 & 0 & 0  & 1 & 7
      \end{bmatrix}
    \end{align*}

    \begin{align*}
      x_1 + 6x_2  +3x_4 = 0 \\
      x_3 - 4x_4 -  = 5     \\
      x_5 = 7               \\
      \\
      \\
      x_1 = -6x_2 - 3x_4    \\
      x_3 =  5 + 4x_4       \\
      x_5 = 7
    \end{align*}
    $\therefore$
    \begin{align*}
      \begin{cases}
        x_1 = -6x_2 - 3x_4   \\
        x_2 \text{ is free } \\
        x_3 =  5 + 4x_4      \\
        x_4 \text{ is free}  \\
        x_5 = 7
      \end{cases}
    \end{align*}
  }
}

\subsection{Parametric Description of Solution Sets}
\dfn{Parametric Description}{
  A description of the solution set of a system of linear equations in terms of a free variable.
}

The descriptions given so far are all parametric, i.e., the free variables act as parameters that determine the values
of the basic variables.

\subsection{Uniqueness and Existence of Solutions}

\thm{Existence and Uniqueness Theorem}{
  A linear system is consistent if and only if the rightmost column of the augmented matrix is not a pivot column, i.e.,
  if and only if an echelon form of the augmented matrix has no row of the form:
  \[
    \begin{bmatrix} 0 & 0 & \cdots & 0 & b \end{bmatrix}
  \]
  where $b$ is a non-zero number.
  \\
  If a linear system is consistent then the solution set contains either:
  \begin{itemize}
    \item A unique solution, when there are no free variables.
    \item Infinitely many solutions in the presence of at least one free variable.
  \end{itemize}
}

\section{Exercises}


\qs{}{
  Find the general solutions of the systems whose augmented matrices are:
  \begin{enumerate}
    \item
          \[
            \begin{bmatrix}
              1 & 3 & 4 & 7 \\
              3 & 9 & 7 & 6
            \end{bmatrix}
          \]
    \item
          \[
            \begin{bmatrix}
              1  & -7 & 0  & 6  & 5  \\
              0  & 0  & 1  & -2 & -3 \\
              -1 & 7  & -4 & 2  & 7
            \end{bmatrix}
          \]
    \item
          \[
            \begin{bmatrix}
              1 & -3 & 0 & -1 & 0  & -2 \\
              0 & 1  & 0 & 0  & -4 & 1  \\
              0 & 0  & 0 & 1  & 9  & 4  \\
              0 & 0  & 0 & 0  & 0  & 0
            \end{bmatrix}
          \]
  \end{enumerate}
}

\sol{
  \begin{enumerate}
    \item
          \begin{align*}
            \begin{bmatrix}
              1 & 3 & 4 & 7 \\
              3 & 9 & 7 & 6
            \end{bmatrix}
            \\
            \\
            3R_1 - R_2 \to R_2           \\
            \begin{split}
              3 + 9 + 12 = 21 \\
              3 + 9 + 7 = 6   \\
              \hline
              0 + 0 + 5 = 15  \\
            \end{split}              \\
            \begin{bmatrix}
              1 & 3 & 4 & 7  \\
              0 & 0 & 5 & 15
            \end{bmatrix}               \\
            \\
            \frac{4}{5}R_2 - R_1 \to R_1 \\
            \begin{split}
              0 + 0 + 4  = 12 \\
              1 + 3 + 4  = 7  \\
              \hline
              -1 -3 + 0 = 5
            \end{split}              \\
            \begin{bmatrix}
              -1 & -3 & 0 & 5  \\
              0  & 0  & 5 & 15
            \end{bmatrix}             \\
            \\
            \frac{1}{5}R_2               \\
            \begin{bmatrix}
              -1 & -3 & 0 & 5 \\
              0  & 0  & 1 & 3
            \end{bmatrix}              \\
            \\
            -R_1                         \\
            \begin{bmatrix}
              1 & 3 & 0 & -5 \\
              0 & 0 & 1 & 3
            \end{bmatrix}               \\
            \\
            x_1 + 3x_2 = -5              \\
            x_3 = 3                      \\
            \\
            \begin{cases}
              x_1 = -5 -3x_2      \\
              x_2 \text{ is free} \\
              x_3 = 3             \\
            \end{cases}
          \end{align*}
    \item
          \begin{align*}
            \begin{bmatrix}
              1  & -7 & 0  & 6  & 5  \\
              0  & 0  & 1  & -2 & -3 \\
              -1 & 7  & -4 & 2  & 7
            \end{bmatrix}       \\
            \\
            R_2 \leftrightarrow R_3      \\
            \begin{bmatrix}
              1  & -7 & 0  & 6  & 5  \\
              -1 & 7  & -4 & 2  & 7  \\
              0  & 0  & 1  & -2 & -3 \\
            \end{bmatrix}       \\
            \\
            R_1 + R_2 \to R_2            \\
            \begin{split}
              1 - 7 + 0 + 6 = 5  \\
              -1 + 7 - 4 + 2 = 7 \\
              \hline
              0 + 0 - 4+ 8 = 12  \\
            \end{split}           \\
            \begin{bmatrix}
              1 & -7 & 0  & 6  & 5  \\
              0 & 0  & -4 & 8  & 12 \\
              0 & 0  & 1  & -2 & -3 \\
            \end{bmatrix}        \\
            \\
            \frac{1}{4}R_2 + R_3 \to R_3 \\
            \begin{split}
              0 + 0 -1 + 2 = 3   \\
              0 + 0 + 1 - 2 = -3 \\
              \hline
              0 + 0 + 0 + 0 = 0  \\
            \end{split}           \\
            \begin{bmatrix}
              1 & -7 & 0  & 6 & 5  \\
              0 & 0  & -4 & 8 & 12 \\
              0 & 0  & 0  & 0 & 0  \\
            \end{bmatrix}         \\
            \\
            -\frac{1}{4}R_2 \to R_2      \\
            \begin{bmatrix}
              1 & -7 & 0 & 6  & 5  \\
              0 & 0  & 1 & -2 & -3 \\
              0 & 0  & 0 & 0  & 0  \\
            \end{bmatrix}         \\
            \\
            x_1 - 7x_2 + 6x_4 = 5        \\
            x_3 - 2x_4 = -3              \\
            0 = 0                        \\
            \\
            x_1 = 5 - 7x_2 + 6x_4        \\
            x_3 = -3 + 2x_4              \\
            \\
            \begin{cases}
              x_1 = 5 - 7x_2 + 6x_4 \\
              x_2 \text{ is free }  \\
              x_3 = -3 + 2x_4       \\
              x_4 \text{ is free }
            \end{cases}
          \end{align*}
  \end{enumerate}
}

\qs{}{
  Find $h$ and $k$ such that the system has:
  \begin{enumerate}
    \item No solutions
    \item A unique solution
    \item Many solutions
  \end{enumerate}

  \begin{enumerate}
    \item
          \begin{align*}
            x_1 + hx_2 = 2 \\
            4x_1 + 8x_2 = k
          \end{align*}
    \item
          \begin{align*}
            x_1 + 3x_2 = 2 \\
            3x_1 + hx_2 = k
          \end{align*}
  \end{enumerate}
}

\sol{
  \begin{enumerate}
    \item
          \begin{align*}
            x_1 + hx_2 = 2 \\
            4x_1 + 8x_2 = k
          \end{align*}
          \begin{enumerate}
            \item
                  \begin{align*}
                    \begin{bmatrix}
                      1 & h & 2 \\
                      4 & 8 & k
                    \end{bmatrix}                                    \\
                    \\
                    -4R_1 + R_2 \to R_2                               \\
                    \begin{split}
                      -4 - 4h = 8 \\
                      4 + 8 = k   \\
                      0 + \left( -4h + 8 \right) = \left( -8 + k \right)
                    \end{split} \\
                    \begin{bmatrix}
                      1 & h                      & 2                    \\
                      0 & \left( -4h + 8 \right) & \left( -8 +k \right)
                    \end{bmatrix} \\
                    \\
                    -4h + 8 = 0                                       \\
                    h = 2                                             \\
                    \\
                    -8 + k \neq 0                                     \\
                    k \neq 8
                  \end{align*}
                  $\therefore$ $h = 2$ and $k \neq  8$
            \item
                  \begin{align*}
                    \begin{bmatrix}
                      1 & h & 2 \\
                      4 & 8 & k
                    \end{bmatrix}                                    \\
                    -4R_1 + R_2 \to R_2                               \\
                    \begin{split}
                      -4 - 4h = 8 \\
                      4 + 8 = k   \\
                      0 + \left( -4h + 8 \right) = \left( -8 + k \right)
                    \end{split} \\
                    \begin{bmatrix}
                      1 & h                      & 2                    \\
                      0 & \left( -4h + 8 \right) & \left( -8 +k \right)
                    \end{bmatrix} \\
                    \\
                    h \neq 2
                  \end{align*}
                  $\therefore$ $h \neq 2$
            \item
                  \begin{align*}
                    \begin{bmatrix}
                      1 & h & 2 \\
                      4 & 8 & k
                    \end{bmatrix}                                    \\
                    -4R_1 + R_2 \to R_2                               \\
                    \begin{split}
                      -4 - 4h = 8 \\
                      4 + 8 = k   \\
                      0 + \left( -4h + 8 \right) = \left( -8 + k \right)
                    \end{split} \\
                    \begin{bmatrix}
                      1 & h                      & 2                    \\
                      0 & \left( -4h + 8 \right) & \left( -8 +k \right)
                    \end{bmatrix} \\
                    -4h + 8 = 0                                       \\
                    h = 2                                             \\
                    \\
                    -8 + k = 0                                        \\
                    k = 8
                  \end{align*}
                  $\therefore$ $h = 2$ and $k = 8$
          \end{enumerate}
  \end{enumerate}
}

\chapter{Vector Equations}

Instead of matrix notation, we can represent a system of linear equations as a vector equation. For example, the system:
\begin{align*}
  2x_1 + 4x_2 = 3 \\
  3x_1 + 2x_2 = 7
\end{align*}
Can be represented as:
\[
  x_1 \begin{bmatrix} 2 \\ 3 \end{bmatrix}  + x_2 \begin{bmatrix} 4 \\ 2 \end{bmatrix}  = \begin{bmatrix} 3 \\ 7  \end{bmatrix}
\]

\subsection{Vectors in $\mathbb{R}^2$}

\dfn{Vector}{
  A matrix with a single column
}

$\mathbb{R}^{2}$ refers to the set of all vectors with two real number entries this is also called the vector space. For example:
\[
  \mathbf{w} = \begin{bmatrix} w_1 \\ w_2 \end{bmatrix}
\]

\subsubsection{Axioms of a Vector Space}

\begin{description}
  \item[Commutativity] $\mathbf{v} + \mathbf{w} = \mathbf{w} + \mathbf{v} $, $\forall \mathbf{v} \in V$
  \item[Associativity] $\left( \mathbf{u} + \mathbf{v} \right) + \mathbf{w} = \mathbf{u} + \left( \mathbf{v} +
          \mathbf{w}  \right)  $ for all $\mathbf{u}\, , \, \mathbf{v}\, , \mathbf{w} \in V$
  \item[Zero Vector] there exists a special vector, denoted by $\mathbf{0}$ such that $\forall \mathbf{v} \in V\, ,
          \mathbf{v} + \mathbf{0} = \mathbf{v} $
  \item[Additive Inverse] For every vector $\mathbf{v} \in V$ there exists a vector $\mathbf{w} \in V$ such that
        $\mathbf{v} + \mathbf{w} = \mathbf{0}$. Such additive inverse is denoted by $-\mathbf{v}$
  \item[Multiplicative Identity] $1\mathbf{v} = \mathbf{v}$, $\forall \mathbf{v} \in V$
  \item[Multiplicative Associativity] $\left( \alpha \beta \right) \mathbf{v} = \alpha \left( \beta\mathbf{v} \right)$,
        $\forall \mathbf{v} \in V$ and all scalars $\alpha$, $\beta$
  \item[Distributive Propety 1] $\alpha \left( \mathbf{u} + \mathbf{v} \right) = \alpha \mathbf{u} + \alpha +
          \mathbf{v}$, for all $\mathbf{u}\, , \, \mathbf{v} \in V$ and all scalars $\alpha$
  \item[Distributive Propoerty 2] $\left( \alpha + \beta \right)\mathbf{v} = \alpha\mathbf{v} + \beta\mathbf{v} $
        $\forall \mathbf{v} \in V$ and all scalars $\alpha$, $\beta$
\end{description}

\ex{}{
  \qs{}{
    Given the $\mathbf{u} = \begin{bmatrix} 1 \\  -2 \end{bmatrix} $ and $\mathbf{v} = \begin{bmatrix} 2 \\ -5
      \end{bmatrix}  $, find $4\mathbf{u}$, $\left( -3 \right)\mathbf{v} $ and $4\mathbf{u} + \left( -3 \right)\mathbf{v}$
  }

  \sol{
    \begin{enumerate}
      \item
            \begin{align*}
              4\mathbf{u} = \begin{bmatrix} 4 \\ -8 \end{bmatrix}
            \end{align*}
      \item
            \[
              -3\mathbf{v} = \begin{bmatrix} -6 \\ 15 \end{bmatrix}
            \]
      \item
            \begin{align*}
              4\mathbf{u} + \left( -3 \right) \mathbf{v} & = \begin{bmatrix} 4 \\ -8 \end{bmatrix}  + \begin{bmatrix} -6 \\
                                                                                                        15\end{bmatrix} \\
                                                         & = \begin{bmatrix} -2 \\ 7 \end{bmatrix}                       \\
            \end{align*}
    \end{enumerate}
  }
}

\section{Linear Combinations}

\dfn{Linear Combination}{
  Given vectors $\mathbf{v_1}, \mathbf{v_2}, \ldots, \mathbf{v_p}$ in $\mathbb{R}^{n}$, and given scalars $c_1, c_2,
    \ldots, c_p$, then the vector $\mathbf{y}$ defined by
  \[
    y = c_1\mathbf{v}_1 + c_2 \mathbf{v}_2, \ldots, c_p \mathbf{v}_p
  \]
  Is a linear combination of $\mathbf{v}_1, \ldots, \mathbf{v}_p$, with weights $c_1, \ldots, c_p$
}

\dfn{Mapping Function}{
  A function in the form
  \[
    f: \mathbb{R}^{n} \to R
  \]
  That takes in a vector as input, $\mathbb{R}^{n}$ and returns a scalar, $R$.
}

\ex{}{
  \qs{}{
    Let $\mathbf{a}_1 = \begin{bmatrix} 1 \\ -2 \\ -5 \end{bmatrix} $, $\mathbf{a}_2 = \begin{bmatrix} 2 \\ 5\\ 6
      \end{bmatrix} $, and $\mathbf{b} = \begin{bmatrix} 7 \\ 4\\ -3 \end{bmatrix} $, Determine whether $\mathbf{b}$ can
    be generated as a linear combination of $\mathbf{a}_1$ and $\mathbf{a}_2$. That is determine whether weights $x_1$
    and $x_2$ exist such that
    \[
      x_1\mathbf{a}_1 + x_2 \mathbf{a}_2 = \mathbf{b}
    \]
    If such an equation has a solution find it.
  }

  \sol{
    \begin{align*}
      x_1 \begin{bmatrix} 1 \\ -2 \\ -5 \end{bmatrix} + x_2 \begin{bmatrix} 2 \\ 5\\ 6
                                                            \end{bmatrix}          & = \begin{bmatrix} 7 \\ 4\\ -3 \end{bmatrix}             \\
      \begin{bmatrix} x_1 \\ -2x_1 \\ -5x_1 \end{bmatrix} + \begin{bmatrix} 2x_2 \\ 5x_2\\ 6x_2
                                                            \end{bmatrix} & = \begin{bmatrix} 7 \\ 4\\ -3 \end{bmatrix}                      \\
      \begin{bmatrix}
        x_1 +2x_2    \\
        -2x_1 + 5x_2 \\
        -5x_1 + 6x_2
      \end{bmatrix}                                                                           & = \begin{bmatrix} 7 \\ 4 \\ -3 \end{bmatrix} \\
      \\
      x_1 +2x_2 = 7                                                                                                                          \\
      -2x_1 + 5x_2 = 4                                                                                                                       \\
      -5x_1 + 6x_2 = -3                                                                                                                      \\
      \\
      \begin{bmatrix}
        1  & 2 & 7  \\
        -2 & 5 & 4  \\
        -5 & 6 & -3
      \end{bmatrix} \sim
      \begin{bmatrix}
        1  & 2  & 7   \\
        0  & -9 & -18 \\
        -5 & 6  & -3  \\
      \end{bmatrix} \sim
      \begin{bmatrix}
        1 & 2   & 7   \\
        0 & -9  & -18 \\
        0 & -16 & -32 \\
      \end{bmatrix} \sim
      \\
      \begin{bmatrix}
        1 & 2  & 7   \\
        0 & -9 & -18 \\
        0 & 0  & 0   \\
      \end{bmatrix} \sim
      \begin{bmatrix}
        1 & 0  & 3   \\
        0 & -9 & -18 \\
        0 & 0  & 0
      \end{bmatrix} \sim
      \begin{bmatrix}
        1 & 0 & 3 \\
        0 & 1 & 2 \\
        0 & 0 & 0
      \end{bmatrix}                                                                                                                         \\
      \\
      x_1 = 3                                                                                                                                \\
      x_2 = 2                                                                                                                                \\
      \\
      3 \mathbf{a}_1 + 2 \mathbf{a}_2 = \begin{bmatrix} 7 \\ 4 \\ -3 \end{bmatrix}
    \end{align*}
  }
}

This example leads to the following theorem:
\thm{}{
  A vector equation
  \[
    x_1\mathbf{a}_1 + x_2\mathbf{a}_2 + \ldots + x_n \mathbf{a}_n = \mathbf{b}
  \]
  has the same solution set as the linear system whose augmented matrix is
  \[
    \begin{bmatrix} \mathbf{a}_1 &  \mathbf{a}_2 & \ldots & \mathbf{a}_n & \mathbf{b} \end{bmatrix}
  \]
  In particular, $\mathbf{b}$ can be generated by a linear combination of $\mathbf{a}_1,\ldots,\mathbf{a}_n$ if an only if
  there exists a solution to the linear system corresponding to the augmented matrix above.
}


\subsection{Span}
\dfn{Span}{
  If $\mathbf{v}_1, \ldots, \mathbf{v}_p$ are in $\mathbb{R}^{n}$, then the set of all linear combinations of
  $\mathbf{v}_1, \ldots, \mathbf{v}_p$ is denoted by $\text{Span}\{\mathbf{v}_1, \ldots, \mathbf{v}_p\} $ and is called
  the \textbf{subset of} $\mathbb{R}^{n}$ \textbf{spanned / generated by} $\mathbf{v}_1, \ldots, \mathbf{v}_p$. That is,
  $\text{Span}\{\mathbf{v}_1, \ldots, \mathbf{v}_p\} $ is the collection of all vectors that can be written in the form
  \[
    c_1 \mathbf{v}_1 + c_2 \mathbf{v}_2 + \ldots + c_p \mathbf{v}_p
  \]
  with $c_1,\ldots,c_p$ scalars.
}

Therefore asking whether a vector $\mathbf{b}$ is in $\text{Span}\{\mathbf{v}_1, \ldots,\mathbf{v}_p\} $, amounts to
asking if $\mathbf{b}$ can be expressed as the vector equation:
\[
  x_1 \mathbf{v}_1 + x_2 \mathbf{v}_2 + \ldots + x_p \mathbf{v}_p = \mathbf{b}
\]
And whether the equation has a solution.

\chapter{The Matrix Equation $A\mathbf{x} = \mathbf{b}$}

\dfn{The Matrix Equation}{
  If $A$ is a $m\times n$ matrix, with columns $\mathbf{a}_1, \ldots, \mathbf{a}_n$ and if $\mathbf{x}$ is in
  $\mathbb{R}^{n}$, then the columns product of $A$ and $\mathbf{x}$, denoted by $A\mathbf{x}$, is the linear
  combination of columns of $A$ using the corresponding entries as weights, i.e.
  \[
    A\mathbf{x} = \begin{bmatrix} \mathbf{a}_1 & \mathbf{a}_2 & \ldots & \mathbf{a}_n  \end{bmatrix} \begin{bmatrix} x_1
      \\ x_2 \\ x_n\end{bmatrix}  = x_1 \mathbf{a}_1 + x_2 \mathbf{a}_2 + \ldots + x_n \mathbf{a}_n
  \]
  Note that the matrix equation is only defined when the number of columns in $A$ is equal to the number rows in
  $\mathbf{x}$.
}
\ex{}{
  \begin{align*}
    \begin{bmatrix}
      1 & 2  & -1 \\
      0 & -5 & 3
    \end{bmatrix}  \begin{bmatrix} 4 \\  3 \\  7 \end{bmatrix} & = 4 \begin{bmatrix} 1 \\
                                                                       0\end{bmatrix} + 3 \begin{bmatrix} 2 \\
                                                                                          -5\end{bmatrix}  + 7 \begin{bmatrix} -1 \\
                                                                                                               3\end{bmatrix}                                                           \\
                                                               & = \begin{bmatrix} 4 \\ 0 \end{bmatrix} + \begin{bmatrix} 6 \\ -15 \end{bmatrix} + \begin{bmatrix} -7 \\ 21 \end{bmatrix} \\
                                                               & = \begin{bmatrix} 3 \\ 6 \end{bmatrix}                                                                                   \\
  \end{align*}
}

\ex{}{
  \qs{}{
    For $\mathbf{v}_1, \mathbf{v}_2, \mathbf{v}_3$, in $\mathbb{R}_n$, write the linear combination $3 \mathbf{v}_1 - 5
      \mathbf{v}_2 + 7 \mathbf{v}_3$ as a matrix times a vector.
  }

  \sol{
    \begin{align*}
      3 \mathbf{v}_1 - 5 \mathbf{v}_2 + 7 \mathbf{v}_3 & = \begin{bmatrix} \mathbf{v}_1 & \mathbf{v}_2 & \mathbf{v}_3
                                                           \end{bmatrix} \begin{bmatrix} 3 \\ -5 \\ 7 \end{bmatrix}                                                          \\
      A \mathbf{x}                                     & = \begin{bmatrix} \mathbf{v}_1 & \mathbf{v}_2 & \mathbf{v}_3 \end{bmatrix} \begin{bmatrix} 3 \\ -5 \\ 7 \end{bmatrix} \\
    \end{align*}
  }
}

\thm{}{
  If $A$ is an $m\times n$ matrix, with columns $\mathbf{a}_1, \ldots, \mathbf{a}_n$, and if $\mathbf{b}$ is in
  $\mathbb{R}^{n}$, the matrix equation
  \[
    A \mathbf{x} = \mathbf{b}
  \]
  has the same solution set as the vector equation
  \[
    x_1 \mathbf{a}_1 + x_2 \mathbf{a}_2 + \ldots + x_n \mathbf{a}_n = \mathbf{b}
  \]
  which in turn, has the same solution set as the system of linear equations whose augmented matrix is
  \[
    \begin{bmatrix} \mathbf{a}_1 & \mathbf{a}_2 & \ldots & \mathbf{a}_n & \mathbf{b} \end{bmatrix}
  \]
}

\subsection{Existence of Solutions}

\thm{}{
  The matrix equation $A\mathbf{x} = \mathbf{b}$ has a solution if and only if $\mathbf{b}$ is a linear combination of
  the columns of $A$.
}

\ex{}{
  \qs{}{
    Let $A = \begin{bmatrix} 1 & 3 & 4 \\ -4 & 2 & -6 \\ -3 & -2 & -7 \end{bmatrix} $ and $\mathbf{b} = \begin{bmatrix}
        b_1 \\ b_2 \\ b_3 \end{bmatrix} $. Is the equation $A \mathbf{x} = \mathbf{b}$ consistent for all possible $b_1, b_2,
      b_3$
  }

  \sol{
    \begin{align*}
      \begin{bmatrix} 1 & 3 & 4 & b_1 \\ -4 & 2 & -6 &  b_2 \\ -3 & -2 & -7 & b_3    \end{bmatrix} \\
      \begin{bmatrix} 1 & 3 & 4 & b_1 \\ -4 & 2 & -6 &  b_2 \\ -3 & -2 & -7 & b_3    \end{bmatrix} \sim \begin{bmatrix}
                                                                                                          1  & 3   & 4   & b_1         \\
                                                                                                          0  & -14 & -10 & -4b_1 - b_2 \\
                                                                                                          -3 & -2  & -7  & b_3         \\
                                                                                                        \end{bmatrix} \sim
      \begin{bmatrix}
        1 & 3   & 4   & b_1         \\
        0 & -14 & -10 & -4b_1  -b_2 \\
        0 & -7  & -5  & -3b_1 - b_3 \\
      \end{bmatrix} \sim
      \\
      \begin{bmatrix}
        1 & 3   & 4   & b_1                       \\
        0 & -14 & -10 & -4b_1 - b_2               \\
        0 & 0   & 0   & b_1 -\frac{1}{2}b_2 + b_3
      \end{bmatrix}
    \end{align*}
    The equation $A \mathbf{x} = \mathbf{b}$ is not consistent for every $\mathbf{b}$ because some choices of
    $\mathbf{b}$ can make $b_1 - \frac{1}{2}b_2 + b_3$ non-zero.
  }
}

\thm{}{
  Let $A$ be an $m \times  n$ coefficient matrix. Then the following statements are logically equivalent. That is, for a particular
  $A$, either they are all true statements or they are all false.
  \begin{itemize}
    \item For each $\mathbf{b}$ in $\mathbb{R}^{m}$, the equation $A\mathbf{x} = \mathbf{b}$ has a solution.
    \item Each $\mathbf{b}$ in $\mathbb{R}^{m}$ is a linear combination of the columns of $A$.
    \item The columns $A$ span $\mathbb{R}^{m}$
    \item $A$ has a pivot position in every row.
  \end{itemize}
}

\subsection{Computation of $A \mathbf{x}$}

\dfn{Row-Vector Rule for computing $A \mathbf{x}$}{
  If the product $A \mathbf{x}$ is defined, then the $i$th entry in $A \mathbf{x}$ is the sum of the products of
  corresponding entries from row $i$ of $A$ and from the vector $\mathbf{x}$
}

\ex{}{
  \qs{}{
    Compute $A \mathbf{x}$, where $A = \begin{bmatrix} 2 & 3 & 4 \\ -1 & 5 & -3 \\ 6 & -2 & 8 \end{bmatrix} $ , and
    $\mathbf{x} = \begin{bmatrix} x_1 \\ x_2 \\ x_3 \end{bmatrix} $
  }

  \sol{
    \begin{align*}
      \begin{bmatrix}
        2  & 3  & 4  \\
        -1 & 5  & -3 \\
        6  & -2 & 8
      \end{bmatrix}
      \begin{bmatrix}
        x_1 \\
        x_2 \\
        x_3
      \end{bmatrix} & =
      \begin{bmatrix}
        2\cdot x_1  + 3 \cdot x_2 + 4 \cdot x_3  \\
        -1 \cdot x_1 + 5 \cdot x_2 - 3 \cdot x_3 \\
        6 \cdot x_1 - 2 \cdot x_2 + 8 \cdot x_3
      \end{bmatrix} \\
                      & =
      \begin{bmatrix}
        2 x_1  + 3  x_2 + 4  x_3 \\
        -x_1 + 5  x_2 - 3  x_3   \\
        6  x_1 - 2  x_2 + 8  x_3
      \end{bmatrix}
      \\
      \\
    \end{align*}
  }
}

\subsection{Properties of Matrix-Vector Product $A \mathbf{x}$}

If $A$ is an $m \times n$ matrix, $\mathbf{u}$ are $\mathbf{v}$ are vectors in $\mathbb{R}^{n}$ and $c$ is a scalar,
then:

\begin{enumerate}
  \item $A \left( \mathbf{u} + \mathbf{v} \right) = A \mathbf{u} + A \mathbf{v} $
  \item $A \left( c \mathbf{u} \right)  = c \left( A \mathbf{u} \right)  $
\end{enumerate}

\chapter{Solution Sets of Linear systems}

\section{Homogeneous Linear Systems}

\dfn{Homogeneous Linear System}{
  A system that can be expressed in the form
  \[
    A\mathbf{x} = \mathbf{0}
  \]
  Where $A$ is an $m\times n$ matrix and $\mathbf{0}$ is the zero vector $\mathbb{R}^{m}$.
}

\noindent Such a system $A\mathbf{x} =
  \mathbf{0}$ always has at least one solution, namely $\mathbf{x} = \mathbf{0}$. This zero solution is also called the
\textit{trivial solution}. The question is then is whether there exists a \textit{non-trivial solution} that satisfies
$A\mathbf{x} = \mathbf{0}$.

\dfn{Trivial solution}{
  A solution of zero
}

\dfn{Non-Trivial solution}{
  A solution that is not zero
}

\thm{}{
  The homogeneous equation $A\mathbf{x} = \mathbf{0}$ has a non-trivial solution if and only if the equation has at least one
  free variable.
}

\ex{}{
  \qs{}{
    Determine if the following homogeneous system has a non-trivial solution. Then describe the solution set.

    \begin{align*}
      3x_1 + 5x_2 - 4x_3 = 0  \\
      -3x_1 - 2x_2 + 4x_3 = 0 \\
      6x_1 + x_2 - 8x_3 = 0
    \end{align*}
  }

  \sol{
    \begin{align*}
      \begin{bmatrix}
        3  & 5  & -4 & 0 \\
        -3 & -2 & 4  & 0 \\
        6  & 1  & -8 & 0
      \end{bmatrix} \sim
      \begin{bmatrix}
        3 & 5  & -4 & 0 \\
        0 & -3 & 0  & 0 \\
        6 & 1  & -8 & 0
      \end{bmatrix} \sim
      \begin{bmatrix}
        3 & 5  & -4 & 0 \\
        0 & -3 & 0  & 0 \\
        0 & 9  & 0  & 0 \\
      \end{bmatrix}
      \\
      \begin{bmatrix}
        1 & 0 & -\frac{4}{3} & 0 \\
        0 & 1 & 0            & 0 \\
        0 & 0 & 0            & 0
      \end{bmatrix}
      \\
      x_1 -\frac{4}{3}x_3 = 0                                                                                  \\
      x_2 = 0                                                                                                  \\
      0 = 0                                                                                                    \\
      \text{This gives us $x_3$ as free. Due to the representation of a vector $x_3=x_3$}                      \\
      \\
      \text{In vector notation we describe the solution}                                                       \\
      \mathbf{x} = \begin{bmatrix} x_1 \\ x_2 \\ x_3 \end{bmatrix} = \begin{bmatrix} \frac{4}{3}x_3 \\ 0 \\ x_3
                                                                     \end{bmatrix} = x_3 \mathbf{v} \\
      \text{Where } \mathbf{v} = \begin{bmatrix} \frac{4}{3} \\ 0 \\ 1 \end{bmatrix}
    \end{align*}
  }
}

\ex{}{
  \qs{}{
    A single linear equation can be treated as a very simple system of equations. Describe all solutions of the
    homogeneous system
    \[
      10x_1 -3x_2 -2x_3 = 0
    \]
  }

  \sol{
    \begin{align*}
      \begin{bmatrix}
        10 & -3 & -2 & 0 \\
        0  & 0  & 0  & 0 \\
        0  & 0  & 0  & 0
      \end{bmatrix}
      \\
      x_1 = \frac{3}{10}x_2 + \frac{1}{5}x_3 \\
      x_2 = x_2                              \\
      x_3 = x_3                              \\
      \\
      \mathbf{x} = \begin{bmatrix} x_1 \\  x_2 \\ x_3 \end{bmatrix} = \begin{bmatrix} \frac{3}{10}x_2 + \frac{1}{5}x_3
                                                                        \\ x_2 \\ x_3\end{bmatrix} = \begin{bmatrix} 0 \\ 0 \\ 0  \end{bmatrix} + x_2 \begin{bmatrix} \frac{3}{10} \\ 1
                                                                                                                                                      \\ 0\end{bmatrix} + x_3 \begin{bmatrix} \frac{1}{5} \\ 0 \\ 1 \end{bmatrix}
    \end{align*}
  }
}

\section{Parametric Vector Form}

\dfn{Parametric Vector Form}{
  A description of the solution set of a system of linear equations in terms of a free variable. In the form:
  \[
    \mathbf{x} = s\mathbf{u} + t \mathbf{v}, \text{ where} s, t \in \mathbb{R}
  \]
}


\section{Non-Homogeneous Linear Systems}

\dfn{Non-Homogeneous Linear System}{
  A system that can be expressed in the form
  \[
    A\mathbf{x} = b
  \]
  Where $A$ is a coefficient matrix, $\mathbf{x}$ is a vector in $\mathbb{R}^{n}$, and $\mathbf{b}$ is a vector in $\mathbb{R}^{m}$.
}

When a non-homogeneous system has many solutions, the general solution can be written in parametric vector form as one
vector plus an arbitrary linear combination of other vectors that satisfy the system.

\ex{}{
  \qs{}{
    Describe all solutions of $A\mathbf{x} = b$, where
    \[
      A = \begin{bmatrix}
        3  & 5  & -4 \\
        -3 & -2 & 4  \\
        6  & 1  & -8
      \end{bmatrix} \text{ and } \mathbf{b}= \begin{bmatrix}
        7 \\ -1 \\ -4
      \end{bmatrix}
    \]
  }

  \sol{

  }
}

\thm{}{
  Suppose the equation $A\mathbf{x} = \mathbf{b}$ is consistent for some given $\mathbf{b}$, and let $\mathbf{p}$ be a
  solution. Then the solution set of $A\mathbf{x}= \mathbf{b}$ is the set of all vectors the form $\mathbf{w} =
    \mathbf{p} + \mathbf{v}_h$, where $\mathbf{v}_h$ is any solution of the homogeneous equation $A\mathbf{x} = \mathbf{0}$.
}

\chapter{Linear Independence}

\dfn{Linear Independence}{
  An indexed set of vectors $\{\mathbf{v}_1, \ldots, \mathbf{v}_p\} $  in $\mathbb{R}^n$ is said to be \textit{linearly
    independent} if the vector equation
  \[
    x_1\mathbf{v}_1 + x_2\mathbf{v}_2 + x_3\mathbf{v}_3 +  \ldots + x_p \mathbf{v}_p = \mathbf{0}
  \]
  has only the trivial solution.
}

\dfn{Linear Dependence}{
  The set $\{\mathbf{v}_1, \ldots, \mathbf{v}_p\} $ is said to be \textit{linearly dependent} if there exist weights
  $c_1, \ldots, c_p$, not all zero, such that
  \[
    c_1\mathbf{v}_1 + c_2\mathbf{v}_2 + c_3\mathbf{v}_3 +  \ldots + c_p \mathbf{v}_p = \mathbf{0}
  \]
}

\ex{}{
  \qs{}{
    Let $\mathbf{v}_1 = \begin{bmatrix} 1 \\  2\\  3 \end{bmatrix} $, $\mathbf{v}_2 = \begin{bmatrix} 4 \\  5\\  6
      \end{bmatrix} $, $\mathbf{v}_3 = \begin{bmatrix} 2 \\ 1 \\ 0 \end{bmatrix} $
    \begin{enumerate}
      \item Determine if the set $\{\mathbf{v}_1, \mathbf{v}_2, \mathbf{v}_3\} $ is linearly independent
      \item If possible find a linear dependence relationship between the vectors
    \end{enumerate}
  }

  \sol{
    \begin{enumerate}
      \item
            \begin{align*}
              \mathbf{v}_1 + \mathbf{v}_2 +  \mathbf{v}_3 = x_1\begin{bmatrix} 1 \\ 2 \\ 3 \end{bmatrix} + x_2\begin{bmatrix}
                                                                                                                4 \\ 5
                                                                                                                \\ 6
                                                                                                              \end{bmatrix}
              +
              x_3\begin{bmatrix}
                   2 \\ 1
                   \\ 0
                 \end{bmatrix}
              =
              \begin{bmatrix}
                0 \\0 \\ 0 \end{bmatrix}    \\
              \\
              x_1 + 4x_2 + 2x_3 = 0       \\
              2x_1 + 5x_2 + x_3 = 0       \\
              3x_1 + 6x_2 = 0             \\
              \\
              \begin{bmatrix}
                1 & 4 & 2 & 0 \\
                2 & 5 & 1 & 0 \\
                3 & 6 & 0 & 0
              \end{bmatrix}              \\
              2R_1 - R_2  \rightarrow R_2 \\
              \begin{bmatrix}
                1 & 4 & 2 & 0 \\
                0 & 3 & 3 & 0 \\
                3 & 6 & 0 & 0 \\
              \end{bmatrix}
              \\
              3R_1 - R_3  \rightarrow R_3 \\
              \begin{bmatrix}
                1 & 4 & 2 & 0 \\
                0 & 3 & 3 & 0 \\
                0 & 6 & 6 & 0 \\
              \end{bmatrix}
              \\
              2R_2 - R_3  \rightarrow R_3 \\
              \begin{bmatrix}
                1 & 4 & 2 & 0 \\
                0 & 3 & 3 & 0 \\
                0 & 0 & 0 & 0 \\
              \end{bmatrix}
              \\
            \end{align*}
            $\therefore$ The set of vectors $\{\mathbf{v}_1, \mathbf{v}_2, \mathbf{v}_3\} $ is linearly dependent
      \item
            \begin{align*}
              \frac{4}{3}R_2 - R_1  \rightarrow R_1 \\
              \begin{bmatrix}
                -1 & 0 & 2 & 0 \\
                0  & 3 & 3 & 0 \\
                0  & 0 & 0 & 0 \\
              \end{bmatrix}
              \\
              \\
              -x_1 + 2x_3 = 0                       \\
              3x_2 + 3x_3 = 0                       \\
              x_3 = x_3                             \\
              \\
              x_1 = 2x_3                            \\
              x_2 = -x_3                            \\
              x_3 = x_3                             \\
              \\
              \begin{cases}
                x_1 = 2x_3 \\
                x_2 = -x_3 \\
                x_3 \text{ is free}
              \end{cases}
            \end{align*}
    \end{enumerate}
  }
}

\subsection{Linear Independence of Matrix Columns}

\dfn{Linear Independence of Matrix Columns}{
  A matrix $A$ is said to have linearly independent columns if the equation $A\mathbf{x} = \mathbf{0}$ has only the
  trivial solution.}{
}

\ex{}{
  \qs{}{
    Determine if the columns of the matrix are linearly independent
    \[
      A = \begin{bmatrix}
        0 & 1 & 4  \\
        1 & 2 & -1 \\
        5 & 8 & 0
      \end{bmatrix}
    \]
  }

  \sol{
    \begin{align*}
      \begin{bmatrix}
        0 & 1 & 4  & 0 \\
        1 & 2 & -1 & 0 \\
        5 & 8 & 0  & 0
      \end{bmatrix}              \\
      R_1 \leftrightarrow R_2     \\
      \begin{bmatrix}
        1 & 2 & -1 & 0 \\
        0 & 1 & 4  & 0 \\
        5 & 8 & 0  & 0 \\
      \end{bmatrix}
      \\
      5R_1 - R_3  \rightarrow R_3 \\
      \begin{bmatrix}
        1 & 2 & -1 & 0 \\
        0 & 1 & 4  & 0 \\
        0 & 2 & -5 & 0 \\
      \end{bmatrix}
      \\
      2R_2 - R_3  \rightarrow R_3 \\
      \begin{bmatrix}
        1 & 2 & -1 & 0 \\
        0 & 1 & 4  & 0 \\
        0 & 0 & 13 & 0 \\
      \end{bmatrix}
    \end{align*}
    $\therefore$ The columns of the matrix $A$ are linearly independent as the equation $A\mathbf{x} = \mathbf{0}$ has
    a only the trivial solution, as all the column have a pivot position, and there are no rows of zeros
  }
}

\chapter{Linear Transformations}

The matrix equation $A \mathbf{x} = \mathbf{b}$, can be though of as the transformation of the vector $\mathbf{x}$ by
$A$ into the vector $\mathbf{b}$. For example, the equations:
\[
  \begin{bmatrix} 4 & -3 & 1 & 3 \\ 2 & 0 & 5 & 1 \end{bmatrix} \begin{bmatrix} 1 \\ 1 \\  1\\  1 \\   \end{bmatrix} =
  \begin{bmatrix} 5 \\  8 \end{bmatrix}  \text{ and } \begin{bmatrix} 4 & -3 & 1 & 3 \\ 2 & 0 & 5 & 1 \end{bmatrix}
  \begin{bmatrix} 1 \\  4 \\  -1 \\  3 \end{bmatrix} = \begin{bmatrix} 0 \\  0 \end{bmatrix}
\]

Infer that multiplication by matrix $A$ transforms the vectors in vector space $\mathbb{R}^{4}$ $\mathbf{x}$ and $\mathbf{u}$ into the respective
$\mathbb{R}^{2}$ vectors

\dfn{Transformation / Function / Mapping / $T$}{
  A function $T$ from $\mathbb{R}^{n}$ to $\mathbb{R}^{m}$ is a rule that assigns to each vector $\mathbf{x}$ in
  $\mathbb{R}^{n}$ a vector $T \left( \mathbf{x} \right) $ in $\mathbb{R}^{m}$. The set $\mathbb{R}^{n}$ is called the
  \textit{domain} of $T$, and $\mathbb{R}^{m}$  is called the \textit{codomain} of $T$. The notation
  \[
    T: \mathbb{R}^{n} \to \mathbb{R}^{m}
  \]
  indicates that the domain of $T$ is $\mathbb{R}^{n}$ and the codomain of $T$ is $\mathbb{R}^{m}$.

  For $\mathbf{x}$ in $\mathbb{R}^{n}$ the vector $T \left( \mathbf{x} \right) $ in $\mathbb{R}^{m}$ is called the
  \textit{image} of $\mathbf{x}$ (under the action of $T$). The set of all images $T \left( x \right) $ is called the
  \textit{range} of $T$
}

\section{Matrix Transformations}

\dfn{Matrix Transformation}{
  Given an $m \times n$ matrix $A$, the transformation $T$ from $\mathbb{R}^{n}$ to $\mathbb{R}^{m}$ defined by
  \[
    T \left( \mathbf{x} \right) = A\mathbf{x}
  \]
  is called a \textit{matrix transformation}, denoted as
  \[
    \mathbf{x} \mapsto A \mathbf{x}
  \]
}
Note that the domain of $T$ is $\mathbb{R}^{n}$ when $A$ has $n$ columns, and the codomain of $T$ is $\mathbb{R}^{m}$
when $A$ has $m$ rows.

\ex{}{
  \qs{}{
    Let $A = \begin{bmatrix} 1 & -3 \\  3 & 5 \\  -1 & 7  \end{bmatrix} $, $\mathbf{u} = \begin{bmatrix} 2 \\ -1
      \end{bmatrix} $, $\mathbf{b} = \begin{bmatrix} 3 \\  2 \\ -5 \end{bmatrix} $, $\mathbf{c} = \begin{bmatrix} 3 \\ 2\\ 5
      \end{bmatrix} $, and define a transformation $T: \mathbb{R}^{2} \to \mathbb{R}^{3}$, by $T \left( \mathbf{x}
      \right) = A \mathbf{x} $, so that
    \[
      T \left( \mathbf{x} \right)  = A \mathbf{x} = \begin{bmatrix} 1 & -3 \\  3 & 5 \\  -1 & 7  \end{bmatrix}
      \begin{bmatrix} x_1 \\ x_2 \end{bmatrix} = \begin{bmatrix} x_1 - 3x_2 \\ 3x_1 + 5x_2 \\ -x_1 + 7x_2 \end{bmatrix}
    \]
    \begin{enumerate}
      \item Find $T \left( \mathbf{u} \right) $, the image of $\mathbf{u}$ under the transformation $T$
      \item Find an $\mathbf{x}$ in $\mathbb{R}^2$ whose image under $T$ is $\mathbf{b}$
      \item Is there more than one $\mathbf{x}$, whose image under $T$ is $\mathbf{b}$?
      \item Determine if $\mathbf{c}$ is in the range of the transformation $T$
    \end{enumerate}
  }

  \sol{
    \begin{enumerate}
      \item
            \begin{align*}
              T \left( \mathbf{x} \right)  = A \mathbf{x} = \begin{bmatrix} 1 & -3 \\  3 & 5 \\  -1 & 7  \end{bmatrix}
              \begin{bmatrix} 2 \\ -1 \end{bmatrix} & = \begin{bmatrix} \left(2\right) - 3\left(-1\right) \\ 3\left(2\right) + 5\left(-1\right) \\ -\left(2\right) + 7\left(-1\right) \end{bmatrix} \\
                                                    & = \begin{bmatrix} 5 \\ 1 \\ -9 \end{bmatrix}                                                                                                  \\
            \end{align*}
      \item
            \begin{align*}
              T \left( \mathbf{x} \right)  = A \mathbf{x} =  \mathbf{b}                                           \\
              =                            \begin{bmatrix} x_1 - 3x_2 \\ 3x_1 + 5x_2 \\
                                             -x_1 + 7x_2\end{bmatrix} = \begin{bmatrix} 3 \\  2 \\ -5 \end{bmatrix} \\
              \begin{bmatrix}
                1  & -3 & 3  \\
                3  & 5  & 2  \\
                -1 & 7  & -5 \\
              \end{bmatrix}
              \\
              3R_1 - R_2  \rightarrow R_2                                                                         \\
              \begin{bmatrix}
                1  & -3  & 3  \\
                0  & -14 & 7  \\
                -1 & 7   & -5 \\
              \end{bmatrix}
              \\
              R_2 \leftrightarrow R_3                                                                             \\
              \begin{bmatrix}
                1  & -3  & 3  \\
                0  & -14 & 7  \\
                -1 & 7   & -5 \\
              \end{bmatrix}
              \\
              -2R_2 - R_3  \rightarrow R_3                                                                        \\
              \begin{bmatrix}
                1  & -3 & 3  \\
                -1 & 7  & -5 \\
                2  & 0  & 3  \\
              \end{bmatrix}
              \\
              -1R_1 - R_2  \rightarrow R_2                                                                        \\
              \begin{bmatrix}
                1 & -3 & 3 \\
                0 & -4 & 2 \\
                2 & 0  & 3 \\
              \end{bmatrix}
              \\
              2R_1 - R_3  \rightarrow R_3                                                                         \\
              \begin{bmatrix}
                1 & -3 & 3 \\
                0 & -4 & 2 \\
                0 & -6 & 3 \\
              \end{bmatrix}
              \\
              \frac{3}{2}R_2 - R_3  \rightarrow R_3                                                               \\
              \begin{bmatrix}
                1 & -3 & 3 \\
                0 & -4 & 2 \\
                0 & 0  & 0 \\
              \end{bmatrix}
              \\
              \frac{3}{4}R_2 - R_1  \rightarrow R_1                                                               \\
              \begin{bmatrix}
                -1 & 0  & \frac{-3}{2} \\
                0  & -4 & 2            \\
                0  & 0  & 0            \\
              \end{bmatrix}
              \\
              x_1 = \frac{3}{2}                                                                                   \\
              x_2 = -\frac{1}{2}                                                                                  \\
            \end{align*}
            $\therefore$ $\mathbf{x} = \begin{bmatrix}
                \frac{3}{2} \\
                -\frac{1}{2}
              \end{bmatrix} $

      \item
            \begin{align*}
              T \left( \mathbf{x} \right) = \mathbf{b} \\
              \begin{bmatrix}
                x_1 - 3x_2  \\
                3x_1 + 5x_2 \\
                -x_1 + 7x_2
              \end{bmatrix} = \begin{bmatrix}
                                3 \\  2\\  -5
                              \end{bmatrix}           \\
              \begin{bmatrix}
                1  & -3 & 3  \\
                3  & 5  & 2  \\
                -1 & 7  & -5 \\
              \end{bmatrix} \sim               \begin{bmatrix}
                                                 -1 & 0  & \frac{-3}{2} \\
                                                 0  & -4 & 2            \\
                                                 0  & 0  & 0            \\
                                               \end{bmatrix}
            \end{align*}
            As the equation $A \mbold{x} = \mbold{b}$ has a unique solution, there is exactly one $\mbold{x}$ whose
            image is $\mbold{b}$
      \item
            For $\mbold{c}$ to be in the range of the transformation $T$, a $\mbold{x}$ transformed by $T$ must map to
            $\mbold{c}$, therefore:
            \begin{align*}
              \begin{bmatrix}
                1  & -3 & 3 \\
                3  & 5  & 2 \\
                -1 & 7  & 5 \\
              \end{bmatrix}                        \\
              3R_1 - R_2  \rightarrow R_2           \\
              \begin{bmatrix}
                1  & -3  & 3 \\
                -1 & 7   & 5 \\
                0  & -14 & 7 \\
              \end{bmatrix}
              \\
              R_2 \leftrightarrow R_3               \\
              \begin{bmatrix}
                1  & -3  & 3 \\
                -1 & 7   & 5 \\
                0  & -14 & 7 \\
              \end{bmatrix}
              \\
              -2R_2 - R_3  \rightarrow R_3          \\
              \begin{bmatrix}
                1  & -3 & 3   \\
                -1 & 7  & 5   \\
                2  & 0  & -17 \\
              \end{bmatrix}
              \\
              -R_1 - R_2  \rightarrow R_2           \\
              \begin{bmatrix}
                1 & -3 & 3   \\
                0 & -4 & -8  \\
                2 & 0  & -17 \\
              \end{bmatrix}
              \\
              2R_1 - R_3  \rightarrow R_3           \\
              \begin{bmatrix}
                1 & -3 & 3  \\
                0 & -4 & -8 \\
                0 & -6 & 23 \\
              \end{bmatrix}
              \\
              \frac{3}{2}R_2 - R_3  \rightarrow R_3 \\
              \begin{bmatrix}
                1 & -3 & 3   \\
                0 & -4 & -8  \\
                0 & 0  & -35 \\
              \end{bmatrix}
            \end{align*}
            The system is inconsistent, meaning $\mbold{c}$ is not in the range of $T$
    \end{enumerate}
  }
}

\section{Linear Transformations}

\dfn{Linear Transform}{
  A transformation / mapping $T$ is linear if:
  \begin{itemize}
    \item $T \left( \mathbf{u} + \mathbf{v} \right) = T \left( \mathbf{u} \right) + T \left( \mathbf{v} \right),
            \,\forall \mathbf{u}, \mathbf{v} \text{ in the domain of } T   $
    \item $T \left( c\mathbf{u} \right) = c T \left( \mathbf{u} \right) \text{ for all scalars $c$ and all $\mathbf{u}$
              in the domain of } T  $
  \end{itemize}
}

Linear Transformations preserve the operations of vector addition and scalar multiplication.
\thm{}{
  If $T$ is a linear transformation, then
  \[
    T \left( \mathbf{0} \right) = \mathbf{0}
  \]
  and
  \[
    T \left( c\mathbf{u} + d \mathbf{v} \right)  = c T \left( \mathbf{u} \right)  + d T \left( \mathbf{v} \right)
  \]
  for all vectors $\mathbf{u}$, $\mathbf{v}$ in the domain of $T$ and all scalars $c$, $d$.
}

\dfn{Contraction}{
  Given a scalar $r$, and a transformation $T: \mathbb{R}^2 \to \mathbb{R}^2$ defined by $T \left( \mathbf{x} \right) =
    r \mathbf{x}$. $T$ is a contraction when $0 \leq r \leq 1$.
}

\dfn{Dilation}{
  Given a scalar $r$, and a transformation $T:\mathbb{R}^2 \to \mathbb{R}^2$ defined by $T \left( \mathbf{x} \right) = r
    \mathbf{x}$. $T$ is a dilation when $r > 1$
}

\ex{}{
  \qs{}{
    Let $r = 3$, and show that $T$ is a linear transformation.
  }

  \sol{
    Let $\mathbf{u}$, and  $\mathbf{v}$ be vectors in $\mathbb{R}^2$, and $c$, and $d$ be scalars.
    \begin{align*}
      T \left( c\mathbf{u} + d \mathbf{v} \right) & = 3 \left( c \mathbf{u} + d \mathbf{v} \right)                  \\
                                                  & = 3c\mathbf{u} + 3d\mathbf{v}                                   \\
                                                  & = c \left( 3\mathbf{u} \right) + d \left( 3 \mathbf{v} \right)  \\
                                                  & = c T \left( \mathbf{u} \right) + d T \left( \mathbf{v} \right) \\
    \end{align*}
    Thus $T$ is a linear transformation because its satisfies the properties of linearity.
  }
}

\section{The Matrix of a Linear Transformation}

\dfn{Matrix of a Linear Transformation}{
  Let $T$ be a linear transformation from $\mathbb{R}^n$ to $\mathbb{R}^m$. Then there exists a unique matrix $A$ such
  that
  \[
    T \left( \mathbf{x} \right) = A \mathbf{x}
  \]
  for all $\mathbf{x}$ in $\mathbb{R}^n$. The matrix $A$ is called the \textit{standard matrix} for the linear
  transformation $T$.
}

The key to finding the matrix $A$ of a linear transformation $T$ is determine the images of the standard unit vectors
found in the columns of $n\times n$ identity matrix $I_n$

\ex{}{
  \qs{}{
    The columns of $I_2 = \begin{bmatrix} 1 & 0 \\ 0 & 1 \end{bmatrix} $ are $\mathbf{e}_1 = \begin{bmatrix} 1 \\ 0
      \end{bmatrix} $, and $\mathbf{e}_2 = \begin{bmatrix} 0 \\ 1 \end{bmatrix} $. Suppose $T$ is a linear transformation from
    $\mathbb{R}^2$ into $\mathbb{R}^3$ such that
    \[
      T \left( \mathbf{e}_1 \right) = \begin{bmatrix} 5 \\ -7 \\ 2 \end{bmatrix} \text{ and } T \left( \mathbf{e}_2 \right) =
      \begin{bmatrix} -3 \\ 8 \\ 0 \end{bmatrix}
    \]
    With no additional information, find a formula for the image of an arbitrary $\mathbf{x}$ in $\mathbb{R}^2$
  }

  \sol{
    \[
      \mathbf{x} = \begin{bmatrix} x_1 \\ x_2 \end{bmatrix}  = x_1 \begin{bmatrix} 1 \\ 0 \end{bmatrix} + x_2
      \begin{bmatrix} 0 \\ 1 \end{bmatrix} = x_1 \mbold{e}_1 + x_1 \mbold{e}_2
    \]
    $\therefore$
    \begin{align*}
      T \left( \mbold{x} \right) & = x_1 T \left( \mbold{e}_1 \right)  + x_2 T \left( \mbold{e}_2 \right)                             \\
                                 & = x_1  \begin{bmatrix} 5 \\ -7 \\ 2 \end{bmatrix} + x_2 \begin{bmatrix} -3 \\ 8 \\ 0 \end{bmatrix} \\
                                 & = \begin{bmatrix}
                                       5x_1 -3 x_2  \\
                                       -7x_1 + 8x_2 \\
                                       2x_1
                                     \end{bmatrix}                                                                                   \\
      \therefore                                                                                                                      \\
      T \left( \mbold{x} \right) & = \begin{bmatrix} T \left( \mbold{e}_1 \right) & T \left( \mbold{e}_2 \right)
                                     \end{bmatrix} \begin{bmatrix} x_1 \\ x_2 \end{bmatrix}                      \\
    \end{align*}
  }
}

\thm{}{
  Let $T: \mathbb{R}^{n} \to \mathbb{R}^{m}$ be a linear transformation. Them there exists a unique matrix $A$ such that
  \[
    T \left( \mbold{x} \right) = A \mbold{x}, \, \forall \mbold{x} \in \mathbb{R}^{n}
  \]
  In fact , $A$ is the $m \times n$ matrix whose $j$th column is the vector $T \left( \mbold{e}_j \right) $,where
  $\mbold{e}_j$ is the $j$th column of the identity matrix in $\mathbb{R}^{n}$:
  \[
    A = \begin{bmatrix} T \left( \mbold{e}_1 \right) \ldots T \left( \mbold{e}_n \right)   \end{bmatrix}
  \]
}

\section{Geometric Linear Transformations of $\mathbb{R}^2$}

\chapter{Exercises}

\qs{}{
  \begin{enumerate}
    \item   Suppose a $3\times 5$ coefficient matrix for a system has three pivot columns. Is the system consistent ? Why or why
          not?

    \item Suppose the coefficient matrix of a linear system of three equations in three variables has a pivot in each column.
          Explain why the system has a unique solution.

  \end{enumerate}
}

\sol{
  \begin{enumerate}
    \item The system is consistent as if there are three pivot columns then there is a pivot in each row which means no
          row is of the form:
          \[
            \begin{bmatrix} 0 & 0 & 0 & 0 & 0 \end{bmatrix}
          \]
          Making the value of $b$ irrelevant to the consistency of the system. Thus the system is consistent.
    \item If the coefficient matrix has a pivot in each column then there is a pivot in each row, and no row is of the
          form:
          \[
            \begin{bmatrix} 0 & 0 & 0 & 0 \end{bmatrix}
          \]
          As such the system is consistent and has a unique solution.
  \end{enumerate}
}



\end{document}
