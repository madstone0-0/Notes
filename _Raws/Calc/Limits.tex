\documentclass[12pt letter]{article}
\title{Limits}
\author{Madiba Hudson-Quansah}
\usepackage{amsmath}
\usepackage{amssymb}
\setcounter{secnumdepth}{0}

\begin{document}
\part{Limits}
Reporting about the behaviour of function within the range of its dangerous values

\begin{displaymath}
	f(x) = x^2 + \frac{1}{x}
\end{displaymath}

Input variable = $x$

Output variable = $f(x)$

Name of function = $f$


"Acceptable"/Permissible input values of x - All real numbers except zero

\begin{gather*}
	(x, f(x)), (x+h, f(x+h))      \\
	f(x+h) - \frac{f(x)}{x+h-x} \\
	f(x+h) - \frac{f(x)}{h}         \\
\end{gather*}

\subsection{Proof}

\[y = -16t^2+100t+6\]

Points used:
$(0, 6) \; (1, 90) \; (3,162)$

When $t = 0$ and $y = 6$

\begin{align*}
	y = at^2 + bt+c      \\
	6 = a(0)^2 + b(0) +c \\
	 & c = 6
\end{align*}

When $t=1$ and $y=90$

\begin{align*}
	90 = a(1)^2 +b +6 \\
	90 = a + b + 6    \\
	84 = a + b        \\
	84 - b = a
\end{align*}

When $t=3$ and $y=162$

\begin{align*}
	162 = a(3)^2 + b + 6     \\
	162 = 9a + 3b + 6        \\
	162 = 9(84 - b) + 3b + 6 \\
	162 = 756 - 9b +3b +6    \\
	-594 = -6b +6            \\
	-600 = -6b               \\
	 & b = 100               \\[20pt]
	\therefore b = 100
\end{align*}

\begin{align*}
	84 - 100 = a \\
	 & a = -16
\end{align*}

Therefore $a=16$, $b=100$ and $c=6$
\pagebreak

\large{Given $f(x) = x^2$ find the Limit of $f(x)$ at $x=3$}

\end{document}
