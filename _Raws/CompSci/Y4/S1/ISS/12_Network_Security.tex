\documentclass[12pt letter]{report}
\input{./template/preamble}
\input{./template/macros}
\input{./template/letterfonts}

\title{\Huge{Network Security}}
\author{\huge{Madiba Hudson-Quansah}}
\date{}
\usepackage{parskip}

\setcounter{tocdepth}{4}
\setcounter{secnumdepth}{4}

\begin{document}
\maketitle
\newpage
\pdfbookmark[section]{\contentsname}{too}
\tableofcontents
\pagebreak

\chapter{Introduction}

\section{Networks}

\dfn{Computer Network}{
  A collection of interconnected devices that communicate to share
  data, resources and services. Examples:
  \begin{itemize}
    \item LAN (Local Area Network)
    \item WAN (Wide Area Network)
    \item WLAN (Wireless Local Area Network)
  \end{itemize}
}

A network consists of:
\begin{description}
  \item[Devices] - Computers, services, routers, and switches.
  \item[Transmission Medium] -
    \begin{itemize}
      \item Physical (Ethernet cables, fiber optics)
      \item Wireless (Wi-Fi, Bluetooth)
    \end{itemize}
  \item[Data and Protocols] - Standardized methods for data
    transmission ensuring communication
\end{description}

\subsection{Terms}

\dfn{Data Packet}{
  A subdivision of data formatted for transmission over a network.
  Packets contain a portion of the original data and necessary
  metadata for routing and reassembly.
}

\dfn{Routing}{
  Directing data packets from source to destination across networks.
}

\dfn{MAC Address}{
  Unique identifier of a Network Interface Card (NIC), that connects
  a computer and a network
}

\dfn{Port}{
  A number associated with an application program that serves or
  monitors for a network service
}

\dfn{Daemon}{
  A service program that runs in the background and receives and
  passes data to the associated program.
}

\section{Networking Models}

Networking models standardize communications between computers,
enabling different systems to communicate. There are two main models:
\begin{itemize}
  \item OSI Model (Open Systems Interconnection)
  \item TCP/IP Model (Transmission Control Protocol/Internet Protocol)
\end{itemize}

\subsection{OSI Model}
Consists of seven layers:
\begin{description}
  \item[Physical] - Hardware, cables, and signals
  \item[Data Link] - Frames, MAC addresses, switches
  \item[Network] - IP addresses, routers, routing
  \item[Transport] - TCP/UDP, ports, data transfers
  \item[Session] - Establishes, manages and terminates connections
  \item[Presentation] - Data translation, encryption, compression
  \item[Application] - User Interfaces, email, web browsers
\end{description}

\subsection{TCP/IP Model}
Consists of four layers:
\begin{description}
  \item[Link]  - Combination of physical and data link layers, hardware and LAN
  \item[Internet] - IP addressing and routing
  \item[Transport] - TCP/UDP, manages end-to-end data transmission
  \item[Application] - Combines application, presentation and session
    layers. Handles high-level protocols.
\end{description}

\section{Domain Name System (DNS)}
DNS translates human-readable domain names (e.g., www.example.com)
into IP addresses (e.g., 154.158.23.4) that computers use to identify
each other on the network.

\chapter{Network Security Concepts}

Involves policies, practices, and tools designed to protect data and
infrastructure from unauthorized access, misuse, or harm. It's goal
is to ensure the integrity, confidentiality, and availability of
network resources.

\section{Common Threats}

\begin{description}
  \item[Interception] - Eavesdropping and Wiretapping (Confidentiality loss)
  \item[Modification] - Data Corruption (Integrity loss)
    \begin{itemize}
      \item Sequencing
      \item Substitution
      \item Insertion
      \item Replay
      \item Physical replay
    \end{itemize}
  \item[Interruption] - Denial of Service (Availability loss)
\end{description}

\dfn{Security Perimeter}{
  A virtual boundary that separates a trusted internal network, i.e a
  protected zone containing a set of computing resources, and an
  untrusted external network, i.e the internet.
}

\subsection{Eavesdropping and Wiretapping}
\dfn{Wiretapping}{
  Covert and unauthorized data interception during transmission.
}
The things that make a network vulnerable to eavesdropping and
wiretapping include:
\begin{description}
  \item[Anonymity] - Attackers can hide their identity
  \item[Multiple points of access] - Data may pass through many hosts
    to get to the user
  \item[Shared resources] - Networks enable resource and workload
    sharing giving potential access to more attack vectors
  \item[Complexity] - More complex systems have more vulnerabilities,
    as users have no idea of all the processes active in the background
  \item[No obvious perimeter] - One host may be a node on two
    different networks.
\end{description}

Some countermeasures include:
\begin{itemize}
  \item Encryption
  \item Physical security
  \item Dedicated lines
  \item Controlled routing, i.e. ensuring that communication travels
    only along certain paths
\end{itemize}

\subsection{Data Corruption / Modification}
This is caused by:
\begin{description}
  \item[Modification] - Change of data en route
  \item[Insertion] - Addition of extra data
  \item[Replay] - Repeating a previous communication
\end{description}

\subsubsection{Sequencing}
\dfn{Sequencing}{
  A sequencing attack permutes the order of data packets. This can lead to
  misinterpretation of data, as the receiving system may not be able to
  correctly reassemble the original message.
}

Network protocols such as TCP/IP include features to check for and
correct transmission errors. However application programs do not
always detect or correct sequencing problems.

\subsubsection{Substitution}

\dfn{Substitution}{
  A substitution attack is the replacement of one piece of a data
  stream with another.
}

\subsubsection{Insertion}
\dfn{Insertion}{
  An insertion attack involves the insertion of new data values into
  a communication stream without replacing or removing the existing
  data. The attacker doesn't even need to break the encryption scheme
  as long as they know precisely where to insert the new data.
}

Countermeasures for both insertion and substitution include:
\begin{itemize}
  \item Encryption
  \item Integrity checks (e.g. checksums, hash functions)
\end{itemize}

\subsubsection{Replay}
\dfn{Replay}{
  A replay attack involves the reuse of legitimate data transmissions.
}

Countermeasures include:
\begin{itemize}
  \item Sequencing
  \item Timestamps
\end{itemize}

\subsection{Interruption / Denial of Service (DoS)}

\dfn{Denial of Service (DoS)}{
  An attack that aims to make a network resource unavailable to its
  intended users by overwhelming it with a flood of illegitimate requests.
}

\section{Basic Network Security Measures}
\begin{description}
  \item[Firewalls]  - Act as a barrier between a network and external
    threats, monitoring and controlling incoming and outgoing traffic.
  \item[Encryption] - Encrypts all data transmissions, making
    intercepted data unreadable without the correct decryption key.
  \item[Antivirus and Anti-malware Software] - Detects and removes malicious
    software that could compromise network security.
\end{description}

\chapter{Practical Network Security}

\section{Firewalls}

\dfn{Firewall}{
  A combination of hardware and software that filters or prevents
  specific types of information from moving between an untrusted
  network, like the internet and a trusted network, like a private
  LAN. Firewalls may be a separate computer system, a software
  service running on an existing router, or server or a separate
  network that contains several supporting devices.
}

Firewalls can be categorized based on processing modes:
\begin{itemize}
  \item Packet Filtering
  \item Application layer proxy
  \item Media access control layer
  \item Hybrid
\end{itemize}

\subsection{Packet Filtering Firewalls}

Examines the header information of data packets that come into a
network. When installed on a TCP/IP based network, it functions at
the IP layer and denies/drops or allows/forwards the packets to the
next network connection based on the rules programmed into the firewall.

Restrictions can be based on:
\begin{itemize}
  \item IP source and destination addresses
  \item Direction, i.e. incoming or outgoing
  \item Protocol type (TCP, UDP, ICMP)
  \item Port number
\end{itemize}

Packet filtering firewalls can further be classified into:
\begin{description}
  \item[Static packet filtering]
    \begin{itemize}
      \item Filters based on rules developed and installed with the firewall
      \item Rules are created and changed by the firewall administrator
      \item Common in network routers and gateways
    \end{itemize}
  \item[Dynamic packet filtering] -
  \item Firewall reacts positively or negatively to traffic patterns
  \item More secure than static filtering
  \item[Stateful Packet Inspection (SPI)] -
    \begin{itemize}
      \item Keeps track of each network connection between internal
        and external systems using a state table
    \end{itemize}
\end{description}

\section{Virtual Private Networks (VPNs)}

\dfn{Virtual Private Network (VPN)}{
  A private, secure network operated over a public and insecure
  network, implemented using cryptographic technology by means of a
  tunnelling protocol coupled with security procedures.
}

A VPN keeps the contents of the network messages hidden from
observers who have access to public traffic and is used to securely
connect remote users to a private network. VPNs often use protocols
like IPSec to encrypt traffic.

\end{document}
