\documentclass[12pt letter]{report}
\input{./template/preamble}
\input{./template/macros}
\input{./template/letterfonts}

\title{\Huge{Introduction}}
\author{\huge{Madiba Hudson-Quansah}}
\date{}
\usepackage{parskip}

\setcounter{tocdepth}{4}
\setcounter{secnumdepth}{4}

\begin{document}
\maketitle
\newpage
\pdfbookmark[section]{\contentsname}{too}
\tableofcontents
\pagebreak

\chapter{Introduction}

\section{Machine Learning}

\begin{itemize}
  \item Performing a Task
  \item With Experience
  \item Improving Performance
\end{itemize}

\section{Artificial Intelligence (AI)}


\dfn{Artificial Intelligence}{
  The science and engineering of making intelligent machines, especially intelligent computer programs.
}

\section{Deep Learning vs Machine Learning}

\subsection{Machine Learning}
\begin{itemize}
  \item Subfield of AI focused on algorithms that learn from data.
  \item Works well with structured data.
  \item Simpler models.
  \item Requires manual feature extraction and selection.
  \item Involves predictive modelling, clustering, and classification.
  \item Feature extraction and application are done separately.
\end{itemize}

\subsection{Deep Learning}
\begin{itemize}
  \item Subfield of ML using neural networks with many layers.
  \item  Works well with large amounts of unstructured data.
  \item  Complex models with multiple layers.
  \item Automatically extracts features from raw data.
  \item Involves image and speech recognition, natural language processing, and recommendation systems.
  \item Feature extraction and application are done together by the neural network.
\end{itemize}

\end{document}
