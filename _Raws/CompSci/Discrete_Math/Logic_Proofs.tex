\documentclass[12pt letter]{report}
\input{./template/preamble}
\input{./template/macros}
\input{./template/letterfonts}

\title{\Huge{Logic and Proofs}}
\author{\huge{Madiba Hudson-Quansah}}
\date{}
\usepackage{parskip}
\setcounter{tocdepth}{4}
\setcounter{secnumdepth}{4}

\begin{document}
\maketitle
\newpage
\pdfbookmark[section]{\contentsname}{too}
\tableofcontents
\pagebreak

\chapter{Propositional Logic}

\dfn{}{
	\begin{itemize}
		\item Proof - A correct mathematical argument.
		\item Theorem - A proven mathematical statement.

	\end{itemize}
}

\section{Proposition}

\dfn{A Proposition}{
	A declarative sentence that is either true or false, but not both. e.g.
	\begin{itemize}
		\item $1 + 1 = 2$ True
		\item $2 + 2 = 3$ False
	\end{itemize}
}

Propositional variables / Statement variables are used to represent propositions, by convention one of these variables
$p, q, r, s \ldots$ The truth value of a position can be denoted by $T$ if it is a \textbf{true proposition} and $F$ if it is
a \textbf{false proposition}.


Therefore, Let $p$ be a proposition. The \textit{negation of} $p$, denoted by $\neg p$ / $\overline{p}$, is the
statement

"It is not the case that $p$"

The proposition $\neg p$ is read "not p", therefore the truth value of the negation of $p$ is the inverse of the truth
value of $p$

\ex{}{
	\qs{}{
		Find the negation of the proposition

		"Michael's PC runs Linux"

		and express this in simple English.
	}
	\sol{
		"Michael's PC does not run Linux"
	}
}

\pagebreak
\ex{}{
	\qs{}{
		Find the negation of the proposition

		"Vandana's smartphone has at least 32GB of memory"

		and express this in simple English.
	}
	\sol{
		"Vandana's has less than 32GB of memory"
	}
}

\dfn{Truth Table} {
	Displays the relationships between the truth values of propositions.
}


\begin{table}[h!]
	\centering
	\begin{tabular}{|c|c|}
		\hline
		$p$ & $\neg p$ \\[0.5ex]
		\hline
		\hline
		$T$ & $F$      \\
		$F$ & $T$      \\
		\hline
	\end{tabular}
	\caption{The truth table for the negation of a proposition}
	\label{tab:1}
\end{table}

The negation of a proposition can also be considered the result of the operation of the \textit{negation operator} on
the proposition.

\section{Logical Operators / Connectives}

\subsection{Conjunction}
\dfn{Conjunction}{
	Let $p$ and $q$ be propositions. The \textit{conjunction} of $p$ and $q$, denoted by $p \wedge q$, is the proposition
	"$p \text{ and } q$". $p \wedge q$ is $T$ when both $q$ and $p$ are $T$ and is $F$ otherwise
}

\begin{table}[h!]
	\caption{The truth table of $p \wedge q$}\label{tab:2}
	\begin{center}
		\begin{tabular}{|c c|c|}
			\hline
			$p$ & $q$ & $p \wedge q$ \\[0.5ex]
			\hline
			\hline
			$T$ & $T$ & $T$          \\
			$T$ & $F$ & $F$          \\
			$F$ & $T$ & $F$          \\
			$F$ & $F$ & $F$          \\
			\hline
		\end{tabular}
	\end{center}
\end{table}


\subsection{Disjunction}
\dfn{Disjunction}{
	Let $p$ and $q$ be propositions. The \textit{disjunction} of $p$ and $q$, dented by $p \vee q $, is the proposition, "$p
		\text{ or } q$". The \textit{disjunction} $p \vee q$ is $F$ when both $p$ and $q$ are $F$ and $T$ otherwise.
}

\begin{table}[h!]
	\caption{The truth table of $p \vee q$}\label{tab:3}
	\begin{center}
		\begin{tabular}{|c c|c|}
			\hline
			$p$ & $q$ & $p \vee q$ \\[0.5ex]
			\hline
			\hline
			$T$ & $T$ & $T$        \\
			$T$ & $F$ & $T$        \\
			$F$ & $T$ & $T$        \\
			$F$ & $F$ & $F$        \\
			\hline
		\end{tabular}
	\end{center}
\end{table}

The use of the \textbf{connective} \textit{or} in a disjunction corresponds to one of the two ways the word \textit{or}
is used in English. \textbf{Inclusive or} and \textbf{Exclusive or}, e.g.

"Students who have taken calculus or computer science can take this class"


"Students who have taken calculus or computer science, but not both can take this class"

Respectively. Therefore taking the disjunction $p \vee q$ an \textbf{Exclusive or} disjunction will $F$ when $q = T
	\text{ and } p = T$ or $q = F \text{ and } p = F$, and $T$ only when $q = T \text{ and } p = F$ or $q = F \text{ and
	} p = T$


\subsubsection{Exclusive or}
\dfn{Exclusive Or}{
	Let $p$ and $q$ be propositions. The \textit{exclusive or} of $p$ and $q$, denoted by $p \oplus q$, is the proposition
	that is $T$ when exactly one of $p$ and $q$ is $T$ and is $F$ otherwise.
}

\begin{table}[h!]
	\caption{The truth table of $p \oplus q$}\label{tab:4}
	\begin{center}
		\begin{tabular}{|c c|c|}
			\hline
			$p$ & $q$ & $p \oplus q$ \\[0.5ex]
			\hline
			\hline
			$T$ & $T$ & $F$          \\
			$T$ & $F$ & $T$          \\
			$F$ & $T$ & $T$          \\
			$F$ & $F$ & $F$          \\
			\hline
		\end{tabular}
	\end{center}
\end{table}

\subsection{Conditional Statement / Implication}

\dfn{Conditional Statement / Implication}{
	Let $p$ and $q$ be propositions. The \textit{conditional statement} $p \to q$ is the proposition "if $p$, then $q$". $p
		\to q$ is $F$ when $p$ is $T$ and $q$ is $F$, and $T$ otherwise. In this connective, $p$ is called the
	\textit{hypothesis / antecedent / premise} and $q$ is called the \textit{conclusion / consequence}.
}

\begin{table}[h!]
	\caption{The truth table of $p \to  q$}\label{tab:5}
	\begin{center}
		\begin{tabular}{|c c|c|}
			\hline
			$p$ & $q$ & $p \to q$ \\[0.5ex]
			\hline
			\hline
			$T$ & $T$ & $T$       \\
			$T$ & $F$ & $F$       \\
			$F$ & $T$ & $T$       \\
			$F$ & $F$ & $T$       \\
			\hline
		\end{tabular}
	\end{center}
\end{table}

$p \to q$ is called a conditional statement because, it asserts that $q$ is $T$ on the condition that $p$ holds.
$p \to q$ is $T$ when $p$ is $F$ no matter the value of $q$.

Conditional statements can be expressed in various ways, some are listed below.

"if $p$, then $q$"

"if $p$, $q$"

"$p$ is sufficient for $q$"

"$q$ if $p$"

"$q$ when $p$"

"a necessary condition for $p$ is $q$"

"$q$ unless $\neg p$"


"$p$ imples $q$"

"$p$ only if $q$"

"a sufficient condition for $q$ is $p$"

"$q$ whenever $p$"

"$q$ is necessary for $p$"

"$q$ follows from $p$"

For the more confusing statements "$p$ only if $q$" and "$q$ unless $\neg p$", the explanation follows.

"$p$ only if $q$" corresponds to "if $p$, then $q$", because "$p$ only if $q$" says that $p$ cannot be $T$ when $q$ is not
$T$, i.e. the statement is $F$ when is $p$ is $T$ but $q$ is $F$. If $p$ is $F$ $q$ maybe either $F$ or $T$ because the
statement says nothing about the value of $q$.

"$q$ unless $\neg p$" expresses the same conditional statement as "if $p$, then $q$", because "$q$ unless $\neg p$"
means that if "if $\neg p$" is $F$ then $q$ must be $T$, That is the statement "$q$ unless $\neg p$" is $F$ when $p$ is
$T$ but $q$ is $F$, but $T$ otherwise.

Or

\begin{align*}
	q \text{ unless } \neg p  \\
	q \text{ if } \neg \neg p \\
	q \text{ if } p           \\
	\therefore \text{if } p \text{ then } q
\end{align*}

\ex{}{
	\qs{}{
		Let $p$ be the statement "Maria learns discrete mathematics" and $q$ be the statement "Maria will find a good
		job". Express the statement $p \to q$ as  a statement in English.
	}

	\sol{

		"Maria will find a good job, if she learns discrete mathematics"

		"For Maria to get a good job, it is sufficient for her to learn discrete mathematics"

		"Maria will find a good job unless she does not learn discrete mathematics"
	}
}

We can form new conditional statements from a given conditional statement, lets say $p \to q$. These are

\begin{itemize}
	\item \textbf{Converse} - $q \to  p$
	\item \textbf{Contrapositive} - $\neg q \to \neg p$
	\item \textbf{Inverse} - $\neg p \to \neg q$
\end{itemize}

\dfn{Equivalence}{
	When two compound propositions always have the same truth value.
}

\subsubsection{Contrapositive}

\dfn{Contrapositive}{
	The \textit{contrapositive} of the conditional statement $p \to q$ is the conditional statement $\neg q \to \neg p$
}

In this statement, the hypothesis ,$p$, and conclusion , $q$, are reversed and negated. This results in an identical
truth table, as the contrapositive is only $F$ when $\neg p$ is $F$ and $\neg q$ is $T$.

\nt{
	A \textbf{Conditional Statement} ( $p \to q$ ) is equivalent to it's \textbf{Contrapositive} ( $\neg q \to  \neg p$ )

}

\begin{table}[h!]
	\caption{The truth table of $\neg q \to  \neg p$}\label{tab:6}
	\begin{center}
		\begin{tabular}{|c c|c c|c|}
			\hline
			$p$ & $q$ & $ \neg p $ & $\neg q$ & $\neg q \to  \neg p$ \\[0.5ex]
			\hline
			\hline
			$T$ & $T$ & $F$        & $F$      & $T$                  \\
			$T$ & $F$ & $F$        & $T$      & $F$                  \\
			$F$ & $T$ & $T$        & $F$      & $T$                  \\
			$F$ & $F$ & $T$        & $T$      & $T$                  \\
			\hline
		\end{tabular}
	\end{center}
\end{table}


\subsubsection{Inverse}

\dfn{Inverse}{
	The  inverse of the conditional statement $p \to q$ is the conditional statement $\neg p \to \neg q$
}

In this statement, the hypothesis ,$p$, and conclusion ,$q$, are negated. This results in a truth table differing from
the original conditional statement but equivalent to the statement's \textbf{converse}.

\nt{
	A conditional statement's ( $p \to q$ ) \textbf{Inverse} ( $\neg p \to \neg q$ ) is equivalent to its \textbf{Converse} ( $q \to p$ )
}

\begin{table}[h!]
	\caption{The truth table of $\neg p \to  \neg q$}\label{tab:7}
	\begin{center}
		\begin{tabular}{|c c|c c|c|}
			\hline
			$p$ & $q$ & $ \neg p $ & $\neg q$ & $\neg p \to  \neg q$ \\[0.5ex]
			\hline
			\hline
			$T$ & $T$ & $F$        & $F$      & $T$                  \\
			$T$ & $F$ & $F$        & $T$      & $T$                  \\
			$F$ & $T$ & $T$        & $F$      & $F$                  \\
			$F$ & $F$ & $T$        & $T$      & $T$                  \\
			\hline
		\end{tabular}
	\end{center}
\end{table}

\subsubsection{Converse}

\dfn{Converse} {
	The converse of the conditional statement $p \to q$ is the conditional statement $q \to p$
}

In this statement, the hypothesis , $p$ and conclusion, , $q$, are reversed. This results in a truth table equivalent to
the conditional statement's inverse.

\nt {
	A conditional statement's ( $p \to q$ ) \textbf{Converse} ( $q \to p$ ) is equivalent to its \textbf{Inverse} (
	$\neg p \to \neg q$ )
}


\begin{table}[h!]
	\caption{The truth table of $q \to p$}\label{tab:8}
	\begin{center}
		\begin{tabular}{|c c|c|}
			\hline
			$p$ & $q$ & $q \to p$ \\[0.5ex]
			\hline
			\hline
			$T$ & $T$ & $T$       \\
			$T$ & $F$ & $T$       \\
			$F$ & $T$ & $F$       \\
			$F$ & $F$ & $T$       \\
			\hline
		\end{tabular}
	\end{center}
\end{table}


\pagebreak
\ex{}{
	\qs{}{
		What are the contrapositive, converse, and inverse of the conditional statement

		"The home team wins whenever it is raining?"
	}

	\sol{
		\begin{align*}
			p \to q                                           \\
			q \text{ whenever } p                             \\
			q = \text{Home team wins }                        \\
			p = \text{It is raining }                         \\
			 & = \text{If it is raining, the home team wins }
		\end{align*}
		\begin{itemize}
			\item Contrapositive - If the home team loses, then it's not raining.
			\item Inverse - If its not raining then the home team loses.
			\item Converse - If the home team wins, then it is raining.
		\end{itemize}
	}
}

\ex{}{
	\qs{}{
		Find the converse, inverse and contrapositive of "Raining is a sufficient condition for my not going to town"
	}

	\sol{

		If it is raining, then I will not go to town

		$p \to q$

		$p$ = It is raining

		$q$ = I will not go to town

		Converse

		$q \to p$

		If I am not going to town, then it is raining

		Contrapositive

		$\neg q \to \neg p $

		If I am going to town, then it is not raining

		Inverse

		$\neg p \to \neg q$

		If it is not raining, then I am going to town

	}
}


\subsection{Biconditionals / Bi-implications}
Another way to combine proposition that expresses they have the same truth value.

\dfn{Biconditionals / Bi-implications}{
	Let $p$ and $q$ be propositions. The \textit{biconditional statement} $p \leftrightarrow q$ is the proposition "$p$
	if and only if $q$". The biconditional statement $p \leftrightarrow q$ is $T$ when $p$ and $q$ have the same truth
	values, and is $F$ otherwise.
}

$p \leftrightarrow q$ breaks down to $ \left( p \to q \right)  \wedge  \left( q \to p \right)  $, and can be expressed as below

"$p$ is necessary and sufficient for $q$"

"if $p$ then $q$, and conversely"

"$p \text{ iff } q$"

\nt{
	"iff" - If and only If
}

\begin{table}[h!]
	\caption{The truth table of $p \leftrightarrow q$}\label{tab:9}
	\begin{center}
		\begin{tabular}{|c c|c|}
			\hline
			$p$ & $q$ & $p \leftrightarrow q$ \\ [0.5ex]
			\hline
			\hline
			$T$ & $T$ & $T$                   \\
			$T$ & $F$ & $F$                   \\
			$F$ & $T$ & $F$                   \\
			$F$ & $F$ & $T$                   \\
			\hline
		\end{tabular}
	\end{center}
\end{table}

\ex{}{
	Let $p$ be the statement "You can take the flight" and let $q$ be the statement "You buy a ticket".

	Then $p \leftrightarrow q$ is the statement:

	"You can take the flight if and only if you buy a ticket"
}

\pagebreak
\section{Compound Propositions}

\qs{}{
	Construct the truth table of the compound proposition
	\[
		\left( p \vee \neg q  \right) \to \left( p \wedge q \right)
	\]
}

\sol{
	\begin{table}[h!]
		\begin{center}
			\begin{tabular}{|c c|c|c|c|c|}
				\hline
				$p$ & $q$ & $\neg q$ & $p \vee \neg q $ & $p \wedge q$ & $\left( p \vee \neg q  \right) \to \left( p \wedge q \right)$ \\ [0.5ex]
				\hline
				\hline
				$T$ & $T$ & $F$      & $T$              & $T$          & $T$                                                           \\
				$T$ & $F$ & $T$      & $T$              & $F$          & $F$                                                           \\
				$F$ & $T$ & $F$      & $F$              & $F$          & $T$                                                           \\
				$F$ & $F$ & $T$      & $T$              & $F$          & $F$                                                           \\
				\hline
			\end{tabular}
		\end{center}
	\end{table}
}

\qs{}{
	Construct

	\[
		p \vee q \to \neg r
	\]
}

\sol {
	\begin{table}[h!]
		\begin{center}
			\begin{tabular}{|c|c|c|c|c|c|}
				\hline
				$p$ & $q$ & $r$ & $\neg r$ & $p \vee q$ & $p \vee q \to \neg r$ \\ [0.5ex]
				\hline
				\hline
				$T$ & $T$ & $T$ & $F$      & $T$        & $F$                   \\
				$T$ & $T$ & $F$ & $T$      & $T$        & $T$                   \\
				$T$ & $F$ & $F$ & $T$      & $T$        & $T$                   \\
				$T$ & $F$ & $T$ & $F$      & $T$        & $F$                   \\
				$F$ & $F$ & $F$ & $T$      & $F$        & $T$                   \\
				$F$ & $T$ & $T$ & $F$      & $T$        & $F$                   \\
				$F$ & $F$ & $T$ & $F$      & $F$        & $T$                   \\
				$F$ & $T$ & $F$ & $T$      & $T$        & $T$                   \\
				\hline
			\end{tabular}
		\end{center}
	\end{table}
}

\subsection{Precedence of Logical Operators}

\begin{table}[h!]
	\caption{Precedence Table}\label{tab:prec}
	\begin{center}
		\begin{tabular}{|c|c|}
			\hline
			Operator          & Precedence \\ [0.5ex]
			\hline
			\hline
			$\neg$            & 1          \\
			$\wedge$          & 2          \\
			$\vee$            & 3          \\
			$\to$             & 4          \\
			$\leftrightarrow$ & 5          \\
			\hline
		\end{tabular}
	\end{center}
\end{table}

Precedence shown from 1 to 5, with 1 having the highest precedence and 5 having the lowest precedence. Operators with
higher precedence are evaluated before operators with lower precedence. Precedence can be overridden by using parentheses.

\subsection{Logic and Bit Operations}


\begin{table}[h!]
	\caption{Truth value to bit table}\label{tab:bit}
	\begin{center}
		\begin{tabular}{|c|c|}
			\hline
			Truth value & Bit \\ [0.5ex]
			\hline
			\hline
			$T$         & 1   \\
			$F$         & 0   \\
			\hline
		\end{tabular}
	\end{center}
\end{table}

A bit can be used to represent a truth value due to its \textit{binary} nature. A variable representing a truth value
can be called a \textit{boolean variable}. Computer bit operations correspond to logical operations, with the operations
\textit{OR}, \textit{AND}, and \textit{XOR} corresponding to the connectives, $\vee$, $\wedge$, and $\oplus$ respectively.

\begin{table}[h!]
	\caption{Bit operations table}\label{tab:ops}
	\begin{center}
		\begin{tabular}{|c|c|c|c|c|}
			\hline
			$x$ & $y$ & $x \wedge y$ & $x \vee y$ & $x \oplus y$ \\ [0.5ex]
			\hline
			\hline
			$1$ & $1$ & $1$          & $1$        & $0$          \\
			$1$ & $0$ & $0$          & $1$        & $1$          \\
			$0$ & $1$ & $0$          & $1$        & $1$          \\
			$0$ & $0$ & $0$          & $0$        & $0$          \\
			\hline
		\end{tabular}
	\end{center}
\end{table}

\dfn{Bit String}{
	A sequence of zero or more bits. The \textit{length} of this string is the number of bits in the string.

	\ex{}{
		$101010011$ is a bit string with a length of $9$.
	}
}


Extending bit operations to bit strings we can define \textit{bitwise AND}, \textit{bitwise OR}, and \textit{bitwise XOR} of two strings of the same length. The new bit string created can be called the \textit{AND}, \textit{OR}, and
\textit{XOR} of the two strings respectively.

\ex{}{
	\qs{}{
		Find the bitwise AND, OR, and XOR of the bit strings 01 1011 0110 and 11 0001 1101.
	}

	\sol{

		\begin{center}
			\textit{AND}
			\begin{align*}
				\begin{split}
					01 1011 0110 \\
					11 0001 1101 \\
					\hline
					01 0001 0100 \\
				\end{split}
			\end{align*}

			\textit{OR}
			\begin{align*}
				\begin{split}
					01 1011 0110 \\
					11 0001 1101 \\
					\hline
					11 1011 1111 \\
				\end{split}
			\end{align*}

			\textit{XOR}
			\begin{align*}
				\begin{split}
					01 1011 0110 \\
					11 0001 1101 \\
					\hline
					10 1010 1011 \\
				\end{split}
			\end{align*}
		\end{center}
	}
}


\chapter{Applications of Propositional Logic}

Statements in natural language are often imprecise and ambigous, to make these statements more precise they can be
represented with propositional logic.

\subsection{Translating English Sentences}

\ex{}{
	\qs{}{
		How can this English sentence be translated into a logical expression?

		"You can access the Internet from campus only if you are a computer science major or you are not a freshman".
	}

	\sol{

		Let $p$ be "You can access the internet from campus"

		Let $q$ be "You are a computer science major"

		Let $r$ be "You are a freshman"

		Therefore the sentence can be translated as
		\[
			p \to \left( q \vee \neg r \right)
		\]
	}
}

\ex{}{
	\qs{}{
		How can this English sentence be translated into a logical expression?

		“You cannot ride the roller coaster if you are under 4 feet tall unless you are older than 16 years old.”
	}

	\sol {

		Let $p$ be "You can ride the roller coaster"

		Let $q$ be "You are under 4 feet tall"

		Let $r$ "You are at least 16 years old"

		Therefore the sentence can be translated as

		\[
			\left( q \vee \neg r  \right)  \to \neg p
		\]
	}
}

\subsection{System Specifications}

In system specifications, we can use propositional logic to express requirements given in natural language in precise
and unambiguous specifications that can serve as the basis for system development.

\ex{}{
	\qs{}{
		Express the specification "The automated reply cannot be sent when the file system is full" using logical
		connectives
	}

	\sol{

		Let $p$ be "The automated reply can be sent"

		Let $q$ be "The file system is full"

		Therefore the specification can be expressed as
		\[
			q \to \neg p
		\]
	}
}

System specifications should be consistent, and therefore should not contain contradictory requirements.

\pagebreak
% \ex{}{
\qs{}{
	Determine whether these system specifications are consistent:

	“The diagnostic message is stored in the buffer or it is retransmitted.”

	“The diagnostic message is not stored in the buffer.”

	“If the diagnostic message is stored in the buffer, then it is retransmitted.”
}

\sol{

	Let	$p$ be "The diagnostic message is stored in the buffer"

	Let $q$ be "The diagnostic message is retransmitted"

	Therefore the specifications can be expressed as:
	\begin{align*}
		p \vee q \\
		\neg p   \\
		p \to q
	\end{align*}

	To have all these specifications be consistent, it must be possible for them to all be $T$ at the same time.
	In this case making $p = T$ and $q = T$  will result in $\neg p$ being $F$, which would cause the only the first
	and second specifications to pass. But making $p = F$ and $q = T$ would result in all the specifications being
	$T$, as proven by the truth table below.

	\begin{table}[h!]
		\begin{center}
			\begin{tabular}{|c c|c|c|c|c|}
				\hline
				$p$ & $q$ & $p \vee q$ & $\neg p$ & $p \to q$ & $\left( p \vee  q \right) \wedge \neg p \wedge \left( p \to q \right)$ \\ [0.5ex]
				\hline
				\hline
				$T$ & $T$ & $T$        & $F$      & $T$       & $F$                                                                    \\
				$T$ & $F$ & $T$        & $F$      & $F$       & $F$                                                                    \\
				$F$ & $T$ & $T$        & $T$      & $T$       & $T$                                                                    \\
				$F$ & $F$ & $F$        & $T$      & $T$       & $F$                                                                    \\
				\hline
			\end{tabular}
		\end{center}
	\end{table}
}
% }

\qs{}{
	Do the system specifications in the previous example remain consistent if the specification “The diagnostic message is not retransmitted” is added?
}

\sol{
	With the addition of the specification represented logically by $\neg q$ it would make it impossible for all the
	specifications to evaluate to $T$, therefore the specifications would be inconsistent.
}

\subsection{Boolean Searches}

Boolean searches are used to search for information in large collections, such as web page indexes and databases. These
searches usually use the connectives \textit{AND}, \textit{OR}, and \textit{NOT / AND NOT}.

\pagebreak
\subsection{Logic Puzzles}
Puzzles that can be solved using logical reasoning.

\qs{}{
	On an island with two types of people, knights and knaves, knights always tell the truth and knaves always lie.
	You encounter two people \textit{A} and \textit{B}. What are \textit{A} and \textit{B} if \textit{A} says
	"\textit{B} is a knight" and \textit{B} says "The two of us are opposite types"?
}

\sol{

	Let $r$ be "\textit{A} is a knight"

	Let $q$ be "\textit{B} is a knight"

	Let $\neg r$ be "\textit{A} is a knave"

	Let $\neg q$ be "B is a knave"

	First we consider the case where $r = T$. In this case \textit{A} is a knight making everything he says true meaning
	\textit{B} is a knight too. But in \textit{B} being a knight everything he says is also true but that cannot be as in
	this scenario both \textit{A} and \textit{B} are knights, making this scenario invalid.

	Next we consider the case where $\neg r = T$. In this case \textit{A} is a knave meaning he is lying about \textit{B}
	being a knight. Following this we can conclude that \textit{B} is also a knave and everything they also say is a lie,
	which is valid.

	Next we consider the case where $q = T$. In this case \textit{B} is a knight making everything he says true and thus
	\textit{A} is a knave. But as \textit{A} is a knave everything they say is false which calls \textit{B}'s knighthood
	into question, rendering this scenario invalid.

	Finally we consider the case where $\neg q = T$. This is case \textit{B} is a knave making everything they say false.
	This allows us to conclude that \textit{A} and \textit{B} are not of opposite types, and since \textit{A} is believed to
	be a knave further supports \textit{B} being a knave, making this scenario valid.

	Concluding we see in the scenario's we've gone through there are only two valid paths and both point to both \textit{A}
	and \textit{B} being knaves.


}

\qs{}{
	A father tells his two children, a boy and a girl, to play in their backyard without getting dirty.
	However, while playing, both children get mud on their foreheads. When the children stop
	playing, the father says “At least one of you has a muddy forehead,” and then asks the children
	to answer “Yes” or “No” to the question: “Do you know whether you have a muddy forehead?”
	The father asks this question twice. What will the children answer each time this question is
	asked, assuming that a child can see whether his or her sibling has a muddy forehead, but cannot
	see his or her own forehead? Assume that both children are honest and that the children answer
	each question simultaneously.
}

\sol{

	Let $q$ be "boy has muddy forehead"

	Let $r$ be "girl has muddy forehead"

	The first time the question is asked both children answer "No" because as they can see each other's foreheads they
	can tell if the other has a muddy forehead but cannot tell if they do, the boy and girl know $r = T$ but not the
	vale of $q$ and $q = T$ but not the value of $r$.

	The second time the question is asked taking account of the answer of either sibling for the first question and the
	condition that at least one of them has a muddy forehead the children can infer that since they can see mud on their
	sibling's forehead but they answered no but for statement the father made which can be put as $q \vee r$ to be true
	at least one of the propositions have to be true, resoling in them both answering "Yes".
}

\subsection{Logic Circuits}

A logic circuit receives input signals $p_1, p_2\ldots p_n$, each a bit, either $1$ or $0$, and produces output signals
$s_1, s_2\ldots s_n$, each also a bit.

Complicated digital circuits can be constructed from three basic circuits, called \textbf{gates}. These are:

\begin{itemize}
	\item Inverter / NOT gate - Takes an input bit $p$ and produces $\neg p$ as output.
	\item OR gate - Takes two input signals $p$ and $q$ and produces the output signal $p \vee q$.
	\item AND gate - Takes two input signals $p$ and $q$ and produces the output signal $p \wedge q$.
\end{itemize}

Given a circuit built from these gates we can determine the output by tracing through the circuit.

\subsubsection{Exercises}

\large{Translate the given statements into propositional logic}
\qs{}{
	You cannot edit a protected Wikipedia entry unless you
	are an administrator. Express your answer in terms of e:
	“You can edit a protected Wikipedia entry” and a: “You
	are an administrator.”
}

\sol{

	$e \to a$
}

\qs{}{
	You can see the movie only if you are over 18 years old
	or you have the permission of a parent. Express your answer in terms of m: “You can see the movie,” e: “You are
	over 18 years old,” and p: “You have the permission of a
	parent.”
}

\sol {
	$m \to  \left( e \vee p \right) $
}


\qs{}{
	You can graduate only if you have completed the requirements of your major and you do not owe money to the
	university and you do not have an overdue library book.
	Express your answer in terms of g: “You can graduate,”
	m: “You owe money to the university,” r: “You have completed the requirements of your major,” and b: “You have
	an overdue library book.”
}

\sol{
	$g \to  \left( r \wedge  \neg m \wedge \neg b \right) $
}

\qs{}{
	To use the wireless network in the airport you must pay
	the daily fee unless you are a subscriber to the service.
	Express your answer in terms of w: “You can use the wire-
	less network in the airport,” d: “You pay the daily fee,”
	and s: “You are a subscriber to the service.”
}

\sol{
	$w \to  \left( d \vee s \right) $
}

\qs{}{
	You can upgrade your operating system only if you have
	a 32-bit processor running at 1 GHz or faster, at least
	1 GB RAM, and 16 GB free hard disk space, or a 64-
	bit processor running at 2 GHz or faster, at least 2 GB
	RAM, and at least 32 GB free hard disk space. Express
	you answer in terms of u: “You can upgrade your operating system,” b32: “You have a 32-bit processor,” b64:
	“You have a 64-bit processor,” g1: “Your processor runs
	at 1 GHz or faster,” g2: “Your processor runs at 2 GHz or
	faster,” r1: “Your processor has at least 1 GB RAM,” r2:
	“Your processor has at least 2 GB RAM,” h16: “You have
	at least 16 GB free hard disk space,” and h32: “You have
	at least 32 GB free hard disk space.”
}

\sol{
	$u \to  \left( b_{32} \wedge  g_1 \wedge r_1 \wedge h_{16}  \right) \vee \left( b_{64} \wedge g_2 \wedge r_2
		\wedge h_{32}\right) $
}



\end{document}
