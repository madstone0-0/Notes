\documentclass[12pt letter]{report}
\input{./template/preamble}
\input{./template/macros}
\input{./template/letterfonts}

\title{\Huge{Mathematical Induction }}
\author{\huge{Madiba Hudson-Quansah}}
\date{March 2024}
\usepackage{parskip}
\usepackage{proof}

\setcounter{tocdepth}{4}
\setcounter{secnumdepth}{4}

\begin{document}
\maketitle
\newpage
\pdfbookmark[section]{\contentsname}{too}
\tableofcontents
\pagebreak

\chapter{Mathematical Induction}

\section{Introduction}

\dfn{Mathematical Induction}{
  Mathematical induction  can be used to prove statements that assert $P \left( n \right) $ is true for all positive
  integers $n$, where $P \left( n \right) $ is a propositional function. A proof by mathematical induction has two
  parts, a \textbf{basis step}, where we show that $P \left( 1 \right) $ is true and an \textbf{inductive step}, where
  we show that for all positive integers $k$ if $P \left( k \right) $ is true then $P \left( k + 1 \right) $ is also
  true.
}

Mathematical Induction can be expressed as the rule of inference $P \left( 1 \right) \wedge \left( P \left( k \right)
  \to P \left( k + 1 \right)   \right) \to  \forall n P \left( n \right)   $, where the domain is the set of positive
integers.\\

\ex{}{
  \qs{}{
    Show that
    \[
      P \left( n \right): 1 + 2 + 3 + \ldots + n = \frac{n \left( n + 1 \right) }{2}
    \]
  }

  \sol{
    We first start with the basis step, $P \left( 1 \right) $, which is
    \begin{align*}
      P \left( 1 \right) : 1 & = \frac{1 \left( 1 + 1 \right) }{2} \\
      1                      & =  1                                \\
    \end{align*}
    $\therefore$ $P \left( 1 \right) $ is true.\\
    Next is the inductive step, where we need to show that $P \left( k \right) \to P \left( k + 1 \right)  $ \\
    We assume
    \[
      P \left( k \right) : 1 + 2 + 3 + \ldots + k = \frac{k \left( k + 1 \right) }{2}
    \]
    and prove
    \[
      P \left( k + 1 \right): 1 + 2 + 3 + \ldots + \left( k + 1 \right) = \frac{\left( k + 1 \right) \left(  k +
        2  \right)  }{2}
    \]
    To prove this we need to relate these two predicates. And because $P \left( k + 1 \right) $ can be expressed as
    $P \left( k \right) + \left( k + 1 \right)  $ we can write
    \[
      P \left( k + 1 \right): 1 + 2 + 3 + \ldots + \left( k + 1 \right) = \frac{k \left( k + 1 \right) }{2} + \left( k
      + 1\right)
    \]
    So we simplify
    \begin{align*}
      \frac{k \left( k + 1 \right) }{2} + \left( k + 1 \right) & = \frac{k \left( k + 1 \right) + 2 \left( k + 1
      \right)  }{2}                                                                                                      \\
                                                               & = \frac{\left( k + 1 \right) \left( k + 2 \right)  }{2} \\
    \end{align*}
  }

}


\ex{}{
  \qs{}{
    Show that
    \[
      P \left( n \right) : 1 + 3 + 5 + \ldots + \left( 2n - 1 \right)  = n^2 \, \, \forall n \in \mathbb{Z}
    \]
  }

  \sol{
    The basis step is $P \left( 1 \right) $
    \[
      P \left( n \right) : 1 = 1^2
    \]

    Assume
    \[
      P \left( k \right) : 1 + 3 + 5 + \ldots + \left( 2k - 1 \right)  = k^2
    \]
    and show that
    \[
      P \left( k + 1 \right) : 1 + 3 + 5 + \ldots + \left( 2 \left( k + 1 \right) - 1  \right) = \left( k + 1
      \right)^2
    \]
    $P \left( k + 1 \right) $ can be expressed as $P \left( k \right) + \left( 2k + 1 \right)   $
    \begin{align*}
      P \left( k + 1 \right) & : 1 + 3 + 5 \ldots + \left( 2k - 1 \right) + \left( 2k + 1 \right) \\
                             & = k^2 + 2k + 1                                                     \\
                             & = \left( k + 1 \right)^2                                           \\
    \end{align*}
    $\therefore$ $P \left( k + 1 \right) $ is true\\
    Conclusion: Having completed the basis and induction steps we can conclude that $1 + 3 + 5 + \ldots + \left( 2n - 1
      \right) = n^2 $
  }
}

\section{Why Mathematical Induction is Valid}

\dfn{Well Ordering Property of Positive Integers}{
  This axiom states that every non-empty subset of positive integers has a least element
}


The validity of mathematical induction as a proof technique comes from the well ordering property, as an axiom of
positive integers. So suppose we know that $P \left( 1 \right) $ is true for all positive integers $n$ and that the
proposition $P \left( k \right) \to P \left( k + 1 \right)  $ is true for all positive integers $k$. To show that $P
  \left( n \right) $ must be true for all positive integers $n$ we can use a proof by contradiction. \\

We first assume that there is at least positive integer for which $P \left( n \right) $ is false, i.e. $\exists x \in
  \mathbb{Z}^{+}\, \neg P \left( x \right) $ \\

Then the set of positive integers for which $P \left( n \right) $ is false, let this be $S$, is non-empty. Thus by the
well ordering property, $S$ has a least element, which we will denote as $m$. \\

We know that $m$ cannot be 1, because $P \left( 1 \right) $ is true, and since $m$ is a positive integer it must be
greater than one ergo $m - 1$ is also a positive integer. \\

Again due to well ordering property we know the least value in $S$ is $m$ therefore $m - 1$ is not in set $S$, as a
result $P \left( m - 1 \right) $ is true. \\

Because the statements $P \left( m - 1 \right) \to P \left( m \right)  $ is also true, it must be the case that $P
  \left( m \right) $ is true, which is a contradiction of the statement that $P \left( m \right) $ must be false. \\

Hence $P \left( n \right) $ must be true for every positive integer.

\section{Examples of Proofs by Mathematical Induction}

\subsection{Proving Summation Formulae}


\qs{}{
  Conjecture a formula for the sum of the first $n$ positive odd integers. Then prove your conjecture using mathematical induction.
}

\sol{
  The sums of the first $n$ positive odd integers for $n = 1,2,3,4,5$ are
  \begin{align*}
    1                 & = 1  \\
    1 + 3             & = 4  \\
    1 + 3 + 5         & = 9  \\
    1 + 3 + 5 + 7     & = 16 \\
    1 + 3 + 5 + 7 + 9 & = 25 \\
  \end{align*}

  \noindent From these values it is reasonable to conjecture that the sum of the first $n$ odd integers is $n^2$, i.e.
  \[
    1+3+5+\ldots+ (2n - 1) = n^2
  \]

  Let $P \left( n \right) $ be $1+3+5+\ldots+ \left( 2n - 1 \right) = n^2 $ \\

  \noindent\underline{Basis Step}:\\
  \\

  \begin{align*}
    P \left( 1 \right): 1 & = 1^2 \\
  \end{align*}

  \noindent\underline{Inductive Step}: \\
  \\
  To complete this step we need to prove that $\forall k \in \mathbb{Z}^{+}\, \, \left( P \left( k \right) \to P
    \left( k + 1 \right) \right)   $ \\

  \noindent Assume $\exists k \in Z^{+}\, P \left( k \right) = T $, then

  \[
    P \left( k \right) : 1 + 3 + 5 + \ldots + \left( 2k - 1 \right) = k^2                        \\
  \]

  \begin{align*}
    P \left( k + 1 \right): 1 + 3 + 5 + \ldots + \left( 2k +1 \right)                          & = \left( k + 1 \right) ^2                  \\
    P \left( k + 1 \right): 1 + 3 + 5 + \ldots + \left( 2k - 1 \right) + \left( 2k + 1 \right) & = \left( k + 1
    \right)^2                                                                                                                               \\
    \\
    P \left( k + 1 \right): P \left( k \right) + \left( 2k + 1 \right)                                                                      \\
    1 + 3 + 5 + \ldots + \left( 2k + 1 \right)                                                 & = k^2 + 2k + 1                             \\
                                                                                               & = \left( k + 1 \right)\left( k + 1 \right) \\
                                                                                               & = \left( k + 1 \right)^2                   \\
  \end{align*}

  \noindent$\therefore$ $P \left( k + 1 \right) $ is $T$ \\
  Hence we can conclude that $P \left( n \right) $ is true for all positive integers $n$.
}


\qs{}{
  Use mathematical induction to show that
  \begin{align*}
    1 +  2 + 2^2 + \ldots + 2^{n} = 2^{n + 1} - 1
  \end{align*}
  For all non-negative integers $n$
}

\sol{

\noindent Let $P \left( n \right) $ be $1 + 2 + 2^2 + \ldots + 2^{n} = 2^{n + 1} - 1$\\

\noindent \underline{Basis Step}: \\
\\
\begin{align*}
  P \left( 0 \right): 2^{0} & = 2^{0 + 1} - 1 \\
  1                         & = 2-1           \\
  1                         & = 1             \\
\end{align*}

\noindent \underline{Induction Step} \\
\\

\noindent To complete this step I must prove $P \left( k \right) \to P \left( k +1 \right)  $ where $k$ is any non-negative
integer. \\

\noindent Assume $P \left( k \right) $ is true for some non-negative integer $k$, then
\[
P \left( k \right): 1 + 2 + 2^2+ \ldots + 2^{k} = 2^{k + 1} - 1
    \]

    \noindent Then $P \left( k + 1 \right) $ is

    \[
      P \left( k + 1 \right): 1 + 2 + 2^2 + \ldots + 2^{k} + 2^{k + 1} = 2^{k + 2} - 1
    \]

    \noindent $P \left( k + 1 \right) $ can also be expressed as $P \left( k + 1 \right)  = P \left( k \right) + 2^{k + 1}$

    \begin{align*}
      P \left( k + 1 \right): 1 + 2 + 2^2 + \ldots + 2^{k + 1} & = 2^{k + 1} - 1 + 2^{k + 1}                   \\
                                                               & = 2^{k} \times 2^{1} + 2^{k} \times 2^{1} - 1 \\
                                                               & = 2^{k} \left( 2^{1} + 2^{1} \right)  - 1     \\
                                                               & = 2^{k} \left( 2^2 \right) - 1                \\
                                                               & = 2^{k + 2} - 1                               \\
    \end{align*}

    \noindent$\therefore$ $P \left( k + 1 \right) $ is $T$ \\
    Hence we can conclude that $P \left( n \right) $ is true for all non-negative integers $n$.


  }


\nparagraph{Sums of Geometric Progressions}
\qs{}{
  Use mathematical induction to prove this formula for the sum of a finite number of terms of a geometric progression with
  initial term $a$ and a common ratio $r$
  \[
    \displaystyle\sum_{j = 0}^{n} ar^{j} = a + ar + ar^2 + \ldots + ar^{n} = \frac{ar^{n + 1} - a}{r - 1} \, \, \text{ where } r \neq 1
  \]
  where $n$ is a non-negative integer.
}

\sol{
  Let $P \left( n \right) $ be "the sum of the first $n + 1$ terms of a geometric
  progression in this formula is correct".

  \noindent \underline{Basis Step} \\
  \\
  \begin{align*}
    P \left( 0 \right): \displaystyle\sum_{j = 0}^{0} ar^{j} & = \frac{ar^{0 + 1} - a}{r - 1}          \\
    a                                                        & = \frac{ar - a}{r - 1}                  \\
    a                                                        & = \frac{a \left( r - 1 \right) }{r - 1} \\
    a                                                        & = a                                     \\
  \end{align*}

  \noindent \underline{Induction  Step} \\
  \\

  \noindent To complete this step I must prove $P \left( k \right) - P \left( k + 1 \right)  $ where $k$ is any non-negative
  integer. \\

  \noindent Assume $P \left( k \right) $ for some non-negative integer $k$, then \\
  \[
    P \left( k \right): \displaystyle\sum_{j = 0}^{k} ar^{j} = a + ar + ar^2 + \ldots + ar^{k} = \frac{ar^{k + 1} - a}{r - 1}
  \]

  Then $P \left( k + 1 \right) $ is
  \[
    P \left( k + 1 \right): \displaystyle\sum_{j = 0}^{k + 1} ar^{j} = a + ar + ar^2 + \ldots + ar^{k + 1} =
    \frac{ar^{k + 2} - a}{r - 1}
  \]

  And can be expressed as $P \left( k + 1 \right) = P \left( k \right) + ar^{k + 1}  $, therefore:
  \begin{align*}
    P \left( k + 1 \right): \displaystyle\sum_{j = 0}^{k + 1}ar^{j} = a + ar + ar^2 + \ldots + ar^{k} + ar^{k + 1} &
    = \frac{ar^{k + 1} - a}{r - 1} + ar^{k + 1}                                                                                                                               \\
                                                                                                                   & = \frac{ar^{k + 1} - a}{r - 1} + \frac{ar^{k + 1}}{1}    \\
                                                                                                                   & =
    \frac{ar^{k
          + 1}
      - a +
      \left(
      ar^{k + 1}
      \right) \left(
      r - 1\right)  }{r
    - 1}                                                                                                                                                                      \\
                                                                                                                   & = \frac{ar^{k + 1} + ar^{k + 2} - ar^{k + 1} - a}{r - 1} \\
                                                                                                                   & = \frac{ar^{k
          + 2} - a}{r
    - 1}                                                                                                                                                                      \\
  \end{align*}

  \noindent $\therefore$ $P \left( k + 1 \right) $ is $T$ \\
  Hence we can conclude that $P \left( n \right) $ is true for all non-negative integers $n$
}


\section{Exercises}

\qs{}{
  Prove that
  \[
    3 + 3 \times 5 + 3 \times 5^2 + \ldots + 3 \times 5^{n} = \frac{3 \left( 5^{n + 1} - 1 \right) }{4}
  \]
  whenever $n$ is non-negative integer
}

\sol{

  \noindent Let $P \left( n \right) $ be $3 + 3 \times 5 + 3 \times 5^2 + \ldots + 3 \times 5^{n} = \frac{3 \left( 5^{n + 1} - 1
      \right) }{4}$

  \noindent \underline{Basis Step} \\
  \\

  \begin{align*}
    P \left( 0 \right): 3 & = \frac{3 \left( 5^{1} - 1 \right) }{4} \\
    3                     & = \frac{12}{4}                          \\
    3                     & = 3                                     \\
  \end{align*}


  \noindent \underline{Induction  Step} \\
  \\

  To complete this step I must prove $P \left( k \right) \to P \left( k + 1 \right)  $ for any non-negative integer $k$

  Assume $P \left( k \right) $ is $T$ for some non-negative integer $k$, then

  \[
    P \left( k \right): 3 + 3 \times 5 + 3 \times 5^2 + \ldots + 3 \times 5^{k} = \frac{3 \left( 5^{k + 1} - 1 \right) }{4}
  \]

  Then $P \left( k + 1 \right) $ is

  \[
    P \left( k + 1 \right): 3 + 3 \times 5 + 3 \times 5^2 + \ldots + 3 \times 5 ^{k + 1} = \frac{3 \left( 5^{k + 2} - 1 \right) }{4}
  \]

  And can be expressed as $P \left( k + 1 \right) = P \left( k \right) + 3 \times 5^{k + 1}  $, therefore:
  \begin{align*}
    P \left( k + 1 \right) : 3 + 3\times 5 + 3 \times 5^2 + \ldots + 3 \times 5^{k} + 3 \times 5^{k + 1} & = \frac{3
    \left( 5^{k + 1} - 1 \right) }{4} + \left( 3 \times 5^{k + 1} \right)                                                                                                                     \\
                                                                                                         & = \frac{3 \left( 5^{k + 1} - 1 \right) + 4 \left( 3 \times 5^{k + 1} \right)  }{4} \\
                                                                                                         & = \frac{3 \times 5^{k + 1} - 3 + 12 \times 5^{k + 1} }{4}                          \\
                                                                                                         & = \frac{3 \left( 5^{k + 1} + 4 \times 5^{k + 1} - 1 \right) }{4}                   \\
    \text{Let } x =
    5^{k + 1}                                                                                                                                                                                 \\
                                                                                                         & = \frac{3 \left( x + 4x - 1 \right) }{4}                                           \\
                                                                                                         & = \frac{3 \left( 5x - 1 \right)}{4}                                                \\
                                                                                                         & = \frac{3 \left( 5^{1} \times 5^{k + 1} \right) - 1 }{4}                           \\
                                                                                                         & = \frac{3 \left( 5^{k + 2} - 1 \right) }{4}                                        \\
  \end{align*}

  \noindent $\therefore$ $P \left( k +1 \right) $ is $T$ \\
  Hence we can conclude that $P \left( n \right) $ is true for all non-negative integers $n$

}

\qs{}{
  Prove that
  \[
    1 \times 1! + 2 \times 2! + \ldots + n \times n! = \left( n + 1 \right) ! - 1
  \]

  Whenever $n$ is a positive integer.
}

\sol{

  \noindent Let $P \left( n \right) $ be $1 \times 1! + 2 \times 2! + \ldots + n \times n! = \left( n + 1 \right) ! - 1$ \\

  \noindent \underline{Basis Step} \\
  \\

  \begin{align*}
    P \left( 1 \right): 1 \times 1! & = \left( 1 + 1 \right) ! - 1
    1                               & = 2 - 1                      \\
    1                               & = 1                          \\
  \end{align*}

  \noindent \underline{Induction  Step} \\
  \\

  \noindent To conclude this step I need to prove $P \left( k \right) \to P \left( k + 1 \right)  $ for any positive integer $k$
  \\

  \noindent Assume $P \left( k \right) $ is $T$  for some positive integer $k$, then \\

  \[
    P \left( k \right): 1\times 1! + 2 \times 2! + \ldots + k \times k! = \left( k + 1 \right)! - 1
  \]

  \noindent Then $P \left( k + 1 \right) $ is:
  \[
    P \left( k + 1 \right): 1 \times 1! + 2 \times 2! + \ldots + \left( k + 1 \right) \times \left( k + 1 \right) ! =
    \left( k + 2 \right)! - 1
  \]

  \noindent And can be expressed as $P \left( k + 1 \right): P \left( k \right) + \left( k + 1 \right) \times \left( k + 1
    \right)!    $, therefore
  \begin{align*}
    P \left( k + 1 \right): 1 \times 1! + 2 \times 2! + \ldots + k \times k! + \left( k + 1 \right) \times \left( k +
    1\right)! & = \left( \left( k + 1 \right)! - 1  \right) + \left( \left( k + 1 \right) \times \left( k + 1 \right)!   \right) \\
              & = \left( k + 1 \right)! + \left( k + 1 \right) \times \left( k + 1 \right)! - 1                                  \\
              & = \left( k + 1 \right)! \left( k + 2 \right)  -1                                                                 \\
              & = \left( k + 2 \right)! - 1                                                                                      \\
  \end{align*}


}

\qs{}{
  \begin{enumerate}
    \item
          Find a formula for
          \[
            \frac{1}{2} + \frac{1}{4} + \frac{1}{8} + \ldots + \frac{1}{2^{n}}
          \]
          by examining the values of this expression  for small values of $n$

    \item Prove the formula you conjectured in part 1.
  \end{enumerate}
}

\sol{
  \begin{enumerate}
    \item
          \begin{align*}
            S_{n} = \frac{1}{2} + \frac{1}{4} + \frac{1}{8} + \ldots + \frac{1}{2^{n}} \\
            S_1 =  \frac{1}{2}                                                         \\
            S_2 = \frac{3}{4}                                                          \\
            S_3 = \frac{7}{8}                                                          \\
            S_4 = \frac{15}{16}                                                        \\
            S_n = \frac{2^{n} - 1}{2^{n}}                                              \\
          \end{align*}
    \item
          \begin{myproof}

            Let $P \left( n \right) $ be $\frac{1}{2} + \frac{1}{4} + \frac{1}{8} + \ldots + \frac{1}{2^{n}} = \frac{2^{n} -
                1}{2^{n}}$ \\


            \noindent \underline{Basis Step} \\
            \\

            \begin{align*}
              P \left( 1 \right): \frac{1}{2} & = \frac{2^{1} - 1}{2^{1}} \\
                                              & \frac{1}{2}= \frac{1}{2}  \\
            \end{align*}

            \noindent \underline{Induction  Step} \\
            \\

            To complete this step I need to prove $P \left( k \right) \to  P \left( k + 1 \right)  $ for any positive
            integer $k$. \\

            Assume $P \left( k \right) $ is $T$  for some positive integer $k$. Then:
            \[
              P \left( k \right): \frac{1}{2} + \frac{1}{4} + \frac{1}{8} + \ldots + \frac{1}{2^{k}} = \frac{2^{k} - 1}{2^{k}}
            \]

            Then $P \left( k + 1 \right) $ is:
            \[
              P \left( k + 1 \right): \frac{1}{2} + \frac{1}{4} + \frac{1}{8} + \ldots + \frac{1}{2^{k + 1}} =
              \frac{2^{k + 1} - 1}{2^{k + 1}}
            \]

            And can be expressed as $P \left( k + 1 \right): P \left( k \right) + \frac{1}{2^{k + 1}}  $, therefore
            \begin{align*}
              P \left( k + 1 \right): \frac{1}{2} + \frac{1}{4} + \frac{1}{8} + \ldots + \frac{1}{2^{k + 1}} & =
              \frac{2^{k} - 1}{2^{k}} + \frac{1}{2^{k + 1}}                                                      \\
                                                                                                             & =
              \frac{\left( 2^{k} - 1 \right) \left( 2^{k + 1} \right) + 2^{k}  }{\left(
                2^{k}\right)\left(
              2^{k + 1}\right)  }                                                                                \\
            \end{align*}

          \end{myproof}
  \end{enumerate}
}

\qs{}{
  Show that 3 divides $n^3 + 2n$, whenever $n \in \mathbb{Z}^{+}$
}

\begin{myproof}

  This has the same meaning as:

  "$n^3 + 2n$ is divisible by 3"

  "$n^3 + 2n$ is a multiple of 3"

  Let $P \left( n \right) $ be $n^3 +2n$ is a multiple of 3 \\

  \noindent \underline{Basis Step} \\
  \\

  \[
    P \left( 1 \right) : 1^3 + 2 \times  1 \text{ is a multiple of 3}
  \]
  True because 3 is a multiple of three. \\

  \noindent \underline{Induction  Step} \\
  \\
  Assume $P \left( k \right) $ is $T$ then \\

  "$k^3 + 2k$ is multiple of 3 " -  $\exists m \in \mathbb{Z} \, , k^3 + 2k = 3m $


  \noindent And show $P \left( k + 1 \right) $ is \\

  "$\left( k + 1 \right)^3 + 2 \left( k + 1 \right)  $ is a multiple of 3" \\


  \noindent   We have \\
  \begin{align*}
    \left( k + 1 \right) ^3 + 2 \left( k + 1 \right) & = k^3 + 3k^2 + 3k + 1 + 2k + 2            \\
                                                     & = \left( k^3 + 2k \right) + 3k^2 + 3k + 3 \\
                                                     & = 3m + 3k^2 + 3k + 3                      \\
                                                     & = 3 \left( m + k^2 + k + 1 \right)        \\
    \text{Let } z = m + k^2 + k + 1                                                              \\
                                                     & = 3z                                      \\
  \end{align*}
  Since $z$ is made up of positive integers $z$ is a positive integer\\
  $\therefore$ $P \left( k + 1 \right) $\\
  Hence we can conclude that $P \left( n \right) $ is true for all positive integers.


\end{myproof}

\chapter{Exercises}

\qs{}{
  Use mathematical induction to prove that 43 divides $6^{n + 1} + 7^{2n - 1}$ for every positive integer $n$
}

\begin{myproof}
  The given statements can be written as "$6^{n + 1} + 7^{2n - 1}$ is a multiple of 43".\\
  Let $P \left( n \right) $ be $6^{n+ 1} + 7^{2n - 1}$ is a multiple of 43. \\

  \noindent \underline{Basis Step} \\
  \\
  \begin{align*}
    P \left( 1 \right) & : 6^{1 + 1} + 7^{2 \times 1 - 1} \text{ is a multiple of } 43 \\
                       & : 43 \text{ is a multiple of } 43                             \\
  \end{align*}
  $P \left( 1 \right) $ is true as 43 is a multiple of 43. \\

  \noindent \underline{Induction  Step} \\
  \\
  Assume $P \left( k \right) $ is $T$, then: \\

  "$6^{k + 1} + 7^{2k - 1}$" is a multiple of 43", means $\exists m \in \mathbb{Z}, 6^{k + 1} + 7^{2k - 1} = 43m$ \\

  \noindent I must now show that $P \left( k + 1 \right) $:
  "$6^{k + 2} + 7^{2 \left( k + 1 \right)  - 1}$" is a multiple of 43.\\

  \begin{align*}
    6^{k + 2} + 7^{2 \left( k + 1 \right) - 1 } & = 6^{k + 2} + 7^{2k + 1}                                         \\
                                                & = 6^{k + 1 + 1} + 7^{2k + 2 - 1}                                 \\
                                                & = 6^{k + 1}\times 6^{1} + 7^{2k - 1} \times 7^{2}                \\
                                                & =  6^{k + 1}\times 6^{1} + 7^{2k - 1} \times 49                  \\
                                                & = 6^{k+ 1}\times 6^1 + 7^{2k - 1}\times 6 + 7^{2k - 1} \times 43 \\
                                                & = 6 \left( 6^{k + 1} + 7^{2k - 1} \right) + 7^{2k - 1} \times 43 \\
                                                & = 6 \left( 43m \right) + 7^{2k - 1} \times 43                    \\
                                                & =  43 \left( 6m + 7^{2k - 1} \right)                             \\
    \text{Let } u = 6m + 7^{2k - 1}                                                                                \\
                                                & = 43u                                                            \\
  \end{align*}
  Since $u$ is made up of integers $u$ is an integer \\
  $\therefore \, P \left( k + 1 \right) $ is $T$. \\
  Hence I can conclude $P \left( n \right) $ is true for all positive integers
\end{myproof}

\qs{}{
  Prove by induction that $\sum_{j = 0}^{n} \left( -\frac{1}{2} \right)^j = \frac{2^{n + 1} + \left( -1 \right)^n }{3\times 2^n} $, whenever $n$ is a non-negative integer
}

\begin{myproof}
  Let $P \left( n \right) $ be:
  \[
    1 + \left( -\frac{1}{2} \right)^1 + \left( -\frac{1}{2} \right)^2 + \ldots + \left( -\frac{1}{2} \right)^n = \frac{2^{n + 1} + \left( -1 \right)^n }{3\times 2^n}
  \]

  \noindent \underline{Basis Step} \\
  \\
  \begin{align*}
    P \left( 0 \right): 1 & = \frac{2 + 1}{3 \times 1 } \\
    1                     & = 1                         \\
  \end{align*}
  $\therefore$ $P \left( 0 \right) $ is $T$ \\

  \noindent \underline{Induction  Step} \\
  \\
  To complete this step I must prove $P \left( k \right) \to P \left( k + 1 \right)  $ for every non non-negative integer
  $k$ \\

  \noindent Assume $P \left( k \right) $ is true for some post integer $k$, then:
  \[
    P \left( k \right): 1 + \left( -\frac{1}{2} \right)^1 + \left( -\frac{1}{2} \right)^2 + \ldots + \left( -\frac{1}{2}
    \right)^k = \frac{2^{k + 1} + \left( -1 \right)^k }{3\times 2^k}
  \]

  Then $P \left( k + 1 \right) $ is:
  \[
    P \left( k + 1 \right): 1 + \left( -\frac{1}{2} \right)^1 + \left( -\frac{1}{2} \right)^2 + \ldots + \left(
    -\frac{1}{2} \right)^{k + 1} = \frac{2^{k + 2} + \left( -1 \right)^{k + 1} }{3\times 2^{k+ 1}}
  \]
  And can be expressed as $P \left( k + 1 \right): P \left( k \right) + \left( -\frac{1}{2} \right)^{k + 1}  $, therefore:
  \begin{align*}
    P \left( k + 1 \right): 1 + \left( -\frac{1}{2} \right) + \left( -\frac{1}{2} \right)^2 + \ldots + \left(
    -\frac{1}{2} \right)^{k} + \left( -\frac{1}{2} \right)^{k + 1} & = \frac{2^{k + 1}
      + \left( -1 \right)^{k} }{3
    \times 2^{k}} + \left( -\frac{1}{2} \right)^{k + 1}                                                                                                                         \\
                                                                   & = \frac{2^{2k + 2} + \left( -1 \right)^{k} \times
    2^{k + 1} - 3\times 2^k \times 1^{k + 1} }{3\times 2^{2k + 1}}                                                                                                              \\
                                                                   & = \frac{2^{k} \left( 2^{k + 2} + \left( -1 \right)^k\times 2 - 3  \right) }{2^k \times 3 \times 2^{k + 1}} \\
                                                                   & = \frac{2^{k + 2} + \left( -1 \right)^k \times 2 - 3 }{3 \times 2^{k + 1}}                                 \\
  \end{align*}

\end{myproof}

\end{document}

