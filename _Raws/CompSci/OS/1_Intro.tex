\documentclass[12pt letter]{report}
\input{./template/preamble}
\input{./template/macros}
\input{./template/letterfonts}

\title{\Huge{Introduction}}
\author{\huge{Madiba Hudson-Quansah}}
\date{}
\usepackage{parskip}

\setcounter{tocdepth}{4}
\setcounter{secnumdepth}{4}

\begin{document}
\maketitle
\newpage
\pdfbookmark[section]{\contentsname}{too}
\tableofcontents
\pagebreak

\chapter{Introduction}

\dfn{Operating System}{
  Central software component that manages all hardware and software, controlling
  \begin{itemize}
    \item Files, devices, section of main memory, and CPU time
    \item Who can use the system and how the system is used
  \end{itemize}
}

\dfn{Subsystem}{
  A component of an operating system that manages a specific resource.
}

Operating systems include for essential subsystem managers, where each manager works with other manager, and performs
a unique role.

\begin{itemize}
  \item Memory Manager
  \item Processor Manager
  \item  Device Manager
  \item File Manager
  \item Optionally, Network Manager
\end{itemize}

Each manager performs the following classes of tasks:
\begin{itemize}
  \item Monitor the system's resources continuously
  \item Enforce the system's security and protection mechanisms, i.e. who gets what and how much
  \item Allocate a resource when appropriate
  \item Deallocate a resource when it is no longer needed
\end{itemize}

\dfn{User Interface}{
  Allows the user to issue commands to the operating system.
}

\section{Main Memory Manager}

In charge of main memory (RAM) and read-only memory (ROM).
The memory manager is responsible for:
\begin{itemize}
  \item Checking validity and legality of memory space request
  \item Reallocating memory to make more useable space available
  \item Deallocating memory to reclaim it
  \item Protecting space in main memory occupied by the operating system
\end{itemize}

\subsection{ROM}
The ROM holds firmware, and is non-volatile.
\dfn{Firmware}{
  Determines when and how to load each piece of the operating system after the power is turned on. For example, loading the kernel, BIOS, etc.
}

\section{Processor Manager}

In charge of allocating the CPU.
\begin{itemize}
  \item Tracks process status, including which process is currently using the CPU
  \item Reclaims the CPU when a process if finished or reached the maximum computation time.
\end{itemize}

Processes are identified by a unique process ID (PID), which is stored in a Process Control Block (PCB).

\section{Device Manager}

In charge of connecting with every available device, responsible for:
\begin{itemize}
  \item Choosing the most efficient resource allocation method, i.e. scheduling processor.
  \item Identifying each device uniquely
  \item Starting device operation when appropriate.
  \item Monitoring device operation and progress
  \item Deallocating the device when the operation is complete
\end{itemize}

\section{File Manager}

In charge of tracking every file in the system, data files, program files, compilers, application programs, etc. is responsible for:
\begin{itemize}
  \item Enforcing user/program resource access restrictions
  \item Controlling user/program modification restrictions
  \item Allocating space for a file on secondary storage
  \item Retrieving files efficiently
\end{itemize}

\section{Network Manager}

In change of sharing resources between multiple computers, responsible for:
\begin{itemize}
  \item Authorizing users to share resources
  \item Controlling access to shared resources
  \item Monitoring network performance
  \item Essentially every aspect of network connectivity
\end{itemize}

\section{User Interface}

\dfn{User Interface}{
  The section of the operating system that allows fir direct interaction with users
}

There are two primary types of user interfaces:

\begin{itemize}
  \item Graphical User Interface (GUI)
        \begin{itemize}
          \item Input from pointing devices
          \item Menu options, desktop metaphors, and windows
        \end{itemize}
  \item Command-Line Interface (CLI)
        \begin{itemize}
          \item Input from keyboard
          \item Commands with options and arguments
        \end{itemize}
\end{itemize}

\section{Cooperation Issues}

As no single manager con work in isolation, cooperation between managers is essential. Each manager performs its tasks and interacts with other managers as needed.
This requires incredible precision for the operating system to work correctly, and becomes even more complex when networking is involved.

\section{Types of Operating Systems}

Operating Systems fall into five categories:
\begin{itemize}
  \item Batch
  \item Interactive
  \item Real-time
  \item Hybrid
  \item Embedded
\end{itemize}
These categories are distinguished by two features:
\begin{itemize}
  \item Response time
  \item Method of data entry
\end{itemize}

\subsection{Batch Systems}
\begin{itemize}
  \item Jobs entered as a whole and in sequence
  \item Input relied no punched cards or tape
  \item  Efficiency measured in throughput
\end{itemize}

\subsection{Interactive Systems}
\begin{itemize}
  \item Allows multiple jobs
  \item Faster turnaround than batch systems
  \item Slower than real-time systems
  \item Introduced to provide fast turnaround when debugging programs
  \item Requires complex algorithms as jobs share processing power
\end{itemize}

\subsection{Hybrid Systems}
\begin{itemize}
  \item Combination of batch and interactive systems
  \item Light interactive load
  \item Accepts and runs batch programs in the background
\end{itemize}

\subsection{Real-time Systems}
\begin{itemize}
  \item Used when reliability is critical
  \item Used in time-critical environments (Spacecraft, Airport traffic control, Fly-by-wire aircraft, etc.)
  \item Two types of real-time systems
        \begin{description}
          \item[Hard real-time systems] - Risk total system failure if a single predict time deadline is missed
          \item[Soft real-time systems] - Suffer performance degradation as a consequence if a deadline is missed
        \end{description}
\end{itemize}

\subsection{Network Systems}

\begin{itemize}
  \item Special class of software, where users perform tasks using few, if ant, local resources, solely relying on network resources, e.g. cloud computing
  \item Requires Wireless Networking Capability, which is a standard feature in many computing devices (cell phone, tables, and other handheld web browsers)
\end{itemize}

\subsection{Embedded Systems}
\begin{itemize}
  \item Computers placed inside other products (automobiles, digital music players, elevators, pacemakers, etc.)
  \item Adds features and capabilities
  \item Performs a specific set of programs/functions
  \item Embedded systems are  non-interchangeable among systems, i.e. they are purpose-built
  \item Small kernel supporing simple and flexible function capabilities
\end{itemize}


\end{document}

