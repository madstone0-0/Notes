\documentclass[12pt letter]{report}
\input{./template/preamble}
\input{./template/macros}
\input{./template/letterfonts}

\title{\Huge{Introduction to Databases}}
\author{\huge{Madiba Hudson-Quansah}}
\date{}
\usepackage{parskip}

\setcounter{tocdepth}{4}
\setcounter{secnumdepth}{4}

\begin{document}
\maketitle
\newpage
\pdfbookmark[section]{\contentsname}{too}
\tableofcontents
\pagebreak

\chapter{Introduction}


\section{Uses of Databases}

\begin{description}
  \item[Retail]  - Department stores, supermarkets, mail order companies, etc.
  \item[Social Media] - Store information about users, their friends, and their activities.
\end{description}

\section{File Based Systems}

\dfn{File-Based Approach}{
  A collection of application programs that perform services for the end-users, where each program defines and manages
  its own data
}

\subsection{Disadvantages of File-Based Systems}

\begin{itemize}
  \item Space intensive
  \item Data redundancy / Duplication - Due to each program having and controlling its own data, if data is needed by multiple programs, it must be duplicated.
  \item Separation and isolation of data - When data is scattered across different files, it becomes difficult to
        access data that should be available.
  \item Data Dependence / Program-Data Dependence - The inherent relationship between the application program and the data it uses. This leads
        to:
        \begin{itemize}
          \item Difficulty in updating data
        \end{itemize}
  \item Incompatible file formats.
  \item Fixed Queries / Proliferation of application programs - Due to the nature of file-based systems creating new
        queries is difficult as a developer would have to write the new queries into the program.
  \item Difficulty in accessing data
  \item Inconsistent data
\end{itemize}


\section{Database Management Systems (DBMS)}
\subsection{Database}
\dfn{Database}{
  A collection of related data
}

\dfn{Database}{
  A \textit{shared} collection of \textit{logically} related data, and descriptions of this data, designed to meet the
  information needs of an organization.
}

\dfn{System Catalogue / Data Dictionary}{
  A description of the data in the database (metadata).
}

\dfn{Entity / Record}{
  A distinct real world object that is to be represented in a database
}

\dfn{Attribute}{
  A property that describes an entity
}

\dfn{Relationship}{
  An association between entities
}

\subsection{Database Management System (DBMS)}
\dfn{Database Management System (DBMS)}{
  A software system that enables users to define, create, maintain, and control access to the database. A DBMS usually
  provides the following functions:
  \begin{description}
    \item[Data Definition Language (DDL)] - Allows users to define a database by describing the data types, structures,
          and constraints on the data to be stored in the database.
    \item[Data Manipulation Language (DML)] - Allows users to insert, update, delete, and retrieve data from the
          database. The DML, then provides a \textbf{query language}, which is used for inquiry and reporting.
    \item[Controlled Access] -
          \begin{itemize}
            \item Ensures that only authorized users can access the database,
            \item Ensures the consistence of the stored data is maintained
            \item Manages concurrent access to the database
            \item Provides a recovery system to ensure consistency of data in the presence of system failures.
          \end{itemize}
  \end{description}
}

\subsubsection{Properties of a DBMS}

\begin{description}
  \item[Massive] Be able to manage large amounts of data.
  \item[Persistence]  Data should be available even after the application has been closed.
  \item[Safe]  Hardware and software failures should not cause data loss.
  \item[Multi-user / Concurrent] Multiple users should be able to access the database at the same time>
  \item[Convenient] Be able to declaratively specify queries and operations.
  \item[Efficient] Be able to perform operations and process queries quickly.
  \item[Reliable] Be able to recover from failures.
\end{description}

\subsubsection{Difference between File-Based Systems and DBMS}

\nparagraph{The Self-Describing Nature of a Database System}

The database system contains the complete definition of the database structure and constraints. This definition is

\nparagraph{Multiple Views of Data}

Each user may see a different view of the database, and the DBMS must control the access of these users to the database.

\nparagraph{The Concurrent Access to the Data}

Each user must be able to access the data concurrently, and the DBMS must include concurrency control techniques to ensure that the user's operations are correctly synchronized.


\dfn{Database System}{
  A DBMS together with the database itself.
}

\section{(Database) Application Programs}

\dfn{(Database) Application Programs}{
  Programs that interact with the DBMS to access the database, using the DBMS's DML and query language.
}

\subsection{Views}

\dfn{View / View Mechanism}{
  A subset of the database, defined using queries. Views provide an abstracted view of the database, hiding irrelevant
  information from the end-user.
}

Views not only serve as an abstraction layer but also provide:

\begin{description}
  \item[A Level of Security] - Views can be used to exclude sensitive data to unauthorized parties.
  \item[Customization] - Views can be used to better present data for end-users
  \item[Consistency] - Views can be used to preserve the state of the database even though the underlying data may have
        changed.
\end{description}

\section{Comparison of File-Based Systems and DBMS}

\subsection{Advantages of Using the DBMS Approach}

\begin{itemize}
  \item Control of data redundancy - Data is stored in a central location, and is not duplicated unnecessarily across multiple
        programs. As some data will need to be duplicated for various reasons such as performance.
  \item Data Consistency - If a data item is stored only once in the database, an update to it will reflect in all
        places it is used, ensuring consistency.
  \item Data Sharing - Data can be shared across departments, and applications without the need for duplication.
  \item Improved Data Integrity - Database integrity refers to the validity and consistency of stored data. Integrity is
        defined in terms of constraints, which are consistency rules that the database is not allowed to violate. With DBMS,
        the Database Admin can define constraints that the DBMS will enforce.
  \item Economy of Scale - The cost of developing and maintaining the database is spread over all the applications that
        use the database.
\end{itemize}

\subsection{Disadvantages of Using the DBMS Approach}

\begin{itemize}
  \item Complexity - The DBMS is a complex piece of software, and as such, it requires a high level of expertise to
        manage.
  \item Size - The complexity and breadth of functionality usually makes the DBMS large and resource-intensive.
  \item Cost of DBMS - The cost of acquiring a DBMS can be high.
  \item Additional hardware costs - The disk storage requirements of a DBMS and the database may require the purchase of
        additional storage space.
  \item Higher impact of a failure - The centralization of resources increases the vulnerability of the system to
        failure as a failure will affect all the applications that use the database which due to centralization, is a lot.
  \item Performance - The DBMS may not be as efficient as a custom-built file-based system.
\end{itemize}


\chapter{Database System Environment}

\subsection{Database Users / Roles}

\begin{description}
  \item[Database Administrators] Responsible for authorizing access to the database for coordinating and monitoring its
        use. Acquiring software and hardware resources, controlling its use, and monitoring efficiency of operations.
  \item[Data Administrators] Responsible for the management of the organization's data resources, including the
        database planning, development, maintenance of standards, polices, and procedures.
  \item [Database Designers] Responsible for defining the content structure, constraints, and functions or transactions of
        the database. Database designers can be split into two categories:
        \begin{itemize}
          \item Logical Database Designers - Focus on the logical structure of the database, i.e. the constraints on
                data to be stored in the database (business rules).
          \item Physical Database Designers - Focuses on how the logical database is to be actually implemented. This
                involves:
                \begin{itemize}
                  \item Mapping the logical structure to a set of tables and integrity constraints
                  \item Selecting storage structures and access methods
                  \item Designing security  measures required on the data.
                \end{itemize}
        \end{itemize}
  \item[Application Developers] Responsible for developing the application programs that provide required
        functionality for the end users, using the implemented database.
  \item[End Users] Clients of the database, which has been designed and implemented to serve their information needs.
        The can be classified according to the way the interact with the system.
        \begin{itemize}
          \item Naive Users - Unsophisticated users who interact with the system through application programs.
          \item Casual - Users who access the database occasionally, and may not be familiar with the system.
          \item Sophisticated - Users who interact with the system regularly and are familiar with the structure of the
                database and the facilities provided by the DBMS.
          \item Stand-alone - Users who maintain personal databases using ready-to-use packaged applications.
        \end{itemize}
\end{description}





\end{document}
