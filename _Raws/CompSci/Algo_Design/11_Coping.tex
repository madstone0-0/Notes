\documentclass[12pt letter]{report}
\input{./template/preamble}
\input{./template/macros}
\input{./template/letterfonts}

\title{\Huge{Coping with the Limitations of Algorithm Power}}
\author{\huge{Madiba Hudson-Quansah}}
\date{}
\usepackage{parskip}

\setcounter{tocdepth}{4}
\setcounter{secnumdepth}{4}

\begin{document}
\maketitle
\newpage
\pdfbookmark[section]{\contentsname}{too}
\tableofcontents
\pagebreak

\chapter{Introduction}

\chapter{Backtracking}

The main idea behind backtracking stems from the foundation of the brute-force exhaustive search. Instead of enumerating
every possible solution to a problem regardless of its validity backtracking constructs a solution incrementally, one
component at a time and evaluate its validity at each step. If the partial solution is found to be invalid, the
algorithm backtracks to replace the last component of the partially constructed solution with the next option. If the
solution is found to be valid, the algorithm continues to build on the partial solution until a complete solution is found.

\section{ $N$-Queens Problem}

\dfn{ $N$-Queens Problem}{
  The $N$-Queens problem is to place $n$ queens on an $n\times n$ chessboard such that no two queens attack each other
  by either being on the same row, column or diagonal.
}

For $n \leq 3$, the solution is trivial, so we will only consider $n > 3$. We start with an empty board of size $4 \times 4$ and place the first queen in the first possible position of its row, $(1, 1)$. Then, we place queen 2 in the first available position of the second row that does not threaten queen 1. If queen 2 cannot be placed in any available column of the second row, we backtrack to queen 1 and move it to the next valid position.

Next, we place queen 3 in the first available position of the third row that does not threaten either queen 1 or queen 2. Similarly, if queen 3 cannot be placed in a valid position, we backtrack again, moving queen 2 to the next position. Finally, we place queen 4 in the first valid position of the fourth row. If all queens are placed successfully, we have found a solution.

If at any point, it is impossible to place a queen in any valid position due to conflicts, we backtrack to the previous queen and attempt a different position. This process continues until all queens are placed on the board without conflict or until we exhaust all possibilities, at which point we conclude there is no solution.

The solution process follows the backtracking approach, where we attempt to place each queen in the first available position, and if a conflict arises, we backtrack to a previous step and try the next possible configuration.

\end{document}
