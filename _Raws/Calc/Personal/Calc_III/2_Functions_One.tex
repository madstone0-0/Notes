\documentclass[12pt letter]{report}
\input{./template/preamble}
\input{./template/macros}
\input{./template/letterfonts}

\title{\Huge{Vector Valued functions of one variable}}
\author{\huge{Madiba Hudson-Quansah}}
\date{}
\usepackage{parskip}

\setcounter{tocdepth}{4}
\setcounter{secnumdepth}{4}

\begin{document}
\maketitle
\newpage
\pdfbookmark[section]{\contentsname}{too}
\tableofcontents
\pagebreak

\chapter{Paths and Curves}

Describing curves in $\mathbb{R}^{n}$ space, formally does not involve a defining equation which is intuitive in $\mathbb{R}^2$ and $\mathbb{R}^{3}$ but rather by how it it swept out like the trace of a skywriter. Thus the characteristics of a curve are not only given by its length, but also by the tracer's velocity and acceleration.

\section{Parametrizations}

\dfn{Path}{
  Let $I$ be an interval of real numbers, typically $I = \left[ a, b \right], \left( a, b \right) \text{ or } \mathbb{R} $.

  A continuous function $ \alpha: I \to  \mathbb{R}^{n} $, is called a \textbf{path}. As $t$ varies over $I$, $ \alpha \left( t \right) $ traces out a \textbf{curve}, $C$, i.e.
  \[
    C = \left\{ \mbold{x} \in \mathbb{R}^{n} : \mbold{x} = \alpha \left( t \right)\,  \exists t \in I  \right\}
  \]
  This is also called the \textbf{image} of $\alpha$, we then say that $\alpha$ \textbf{parametrizes} the curve $C$. We often refer to the input variable $t$ as the \textbf{time} and $ \alpha \left( t \right) $ as the \textbf{position} of a moving object at time $t$.
}

A path is a vector valued function of one variable, i.e. $t$. For each $t$ in $I$, $ \alpha \left( t \right) $ is a point in $\mathbb{R}^{n}$, so can be written
$ \alpha \left( t \right) = \left( x_1 \left( t \right), x_2 \left( t \right), \ldots x_n \left( t \right)  \right)  $, where each of the $n$ coordinates is a real number that depends on $t$.

\ex{Circles in $\mathbb{R}^2$}{
  I.e. $x^2 + y^2 = a^2$, where $a$ is the radius of the circle. This can be rewritten:
  \[
    \frac{x^2}{a^2} + \frac{y^2}{a^2} = 1
  \]
  To parametrize a circle we use the identity $ \cos^2 t + \sin^2 t = 1 $, and let $\frac{x}{a} = \cos(t)$ and $\frac{y}{a} = \sin(t)$, or
  $x = a \cos(t)$ and $y = a \sin(t)$, Thus we have the parametrization:
  \[
    \alpha \left( t \right)  = \left( a \cos(t), a \sin(t) \right), \, a > 0, \, t \in \mathbb{R}
  \]
}

\ex{Graphs of $y = f \left( x \right) $}{
  Consider the curve described by $y = \sin(x)$ in $\mathbb{R}^2$, where $0 \leq x \leq \pi $, it consists of points of the form $\left( x, \sin(x) \right)$, where $x$ varies over the interval $[0, \pi]$. $x$ can be used as a parameter, thus we can write the parametrization as:
  \[
    \alpha \left( x \right)  = \left( x, \sin(x) \right)
  \]

}

\section{Velocity, Acceleration, Speed and Arc Length}

\dfn{Derivative of a path}{
  Given a path $ \alpha: I \to  \mathbb{R}^{n}$, the \textbf{derivative} of $ \alpha$ is defined by
  \begin{align*}
    \alpha^{\prime} \left( t \right) & = \lim_{h \to 0} \frac{ \alpha \left( t + h \right) - \alpha \left( t \right)  }{h}                                                                                                                                        \\
                                     & = \lim_{h  \to -0} \left(
    \frac{1}{h} \left(  \left( x_1 \left( t + h \right), x_2 \left( t + h \right)  , \ldots, x_n \left( t + h \right) \right)
      - \left( x_1 \left( t \right), x_2 \left( t \right), \ldots, x_n \left( t \right)    \right)
      \right)
    \right)                                                                                                                                                                                                                                                       \\
                                     & = \lim_{h \to 0} \left( \frac{x_1 \left( t + h \right) - x_1 \left( t \right) }{h}, \frac{x_2 \left( t + h \right) - x_2 \left( t \right) }{h}, \ldots, \frac{x_n \left( t + h \right) - x_n \left( t \right) }{h} \right) \\
                                     & = \left( x_1^{\prime} \left( t \right), x_2^{\prime} \left( t \right), \ldots x_n^{\prime} \left( t \right)  \right)                                                                                                       \\
  \end{align*}
  Provided the limit exists. The derivative is also called the \textbf{velocity} ($\mbold{v} \left( t \right) $) of $ \alpha$. And thus $ \alpha ^{\prime\prime} $ is the \textbf{acceleration} ($\mbold{a} \left( t \right) $) of $ \alpha$.
}

For some time $t_0$, and for any time $t$, let $s \left( t \right) $ be the distance travelled by the path from $ \alpha \left( t_0 \right) $ to $ \alpha \left( t \right) $, $s \left( t \right) $ is called the \textbf{arclength} function and the magnitude of it's derivative is the speed.


\dfn{Speed}{
  If $ \alpha: I \to  \mathbb{R}^{n}$ is a differentiable path, then its speed, denoted bu $v \left( t \right) $, is defined:
  \[
    v \left( t \right)  = \|\mbold{v} \left( t \right) \|
  \]
  Where $v \left( t \right) $ is a scalar quantity.
}

\dfn{Arclength}{
  The arclength from $t = a$ to $t = b$ is defined to be
  \[
    s \left( t \right) = \displaystyle\int_{a}^{b} v \left( t \right) \, dt = \displaystyle\int_{a}^{b} \|\mbold{v} \left( t \right) \| \, dt
  \]
}



\end{document}
