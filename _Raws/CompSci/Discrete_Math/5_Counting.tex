\documentclass[12pt letter]{report}
\input{./template/preamble}
\input{./template/macros}
\input{./template/letterfonts}

\title{\Huge{Counting}}
\author{\huge{Madiba Hudson-Quansah}}
\date{}
\usepackage{parskip}

\setcounter{tocdepth}{4}
\setcounter{secnumdepth}{4}

\begin{document}
\maketitle
\newpage
\pdfbookmark[section]{\contentsname}{too}
\tableofcontents
\pagebreak

\chapter{Basics of Counting}

\section{Basic Counting Principles}

\subsection{Product Rule}
\dfn{Product Rule}{
  This rule applies when a procedure is made up of separate tasks. Suppose that a procedure can be broken down into
  two tasks. If there are $n_1$ ways to do task 1 and for each of these ways of doing task 1, there are $n_2$ ways to
  do task 2, then there are $n_1n_2$ ways to do the procedure.
}
If $A_1, A_2, \ldots, A_m$ are finite sets, then the number of elements in the Cartesian product of these sets in the
product of the number of elements of each set. \\
Therefore it follows that the product rule then becomes
\[
  \mid A_1 \times A_2 \times \ldots \times A_m  \mid  =  \mid A_1  \mid  \cdot  \mid A_2  \mid \cdot \ldots \cdot   \mid A_m  \mid
\]
\ex{}{
  \qs{}{
    A new company with just two employees, Sanchez and Patel, rents a floor of a building with 12 offices. How many ways
    are there to assign different offices to these two employees.
  }

  \sol{
    This procedure is made up of two tasks, assigning an office to Sanchez, then assigning an office to Patel. The
    first task can be done in 12 ways, and the second can be done in 11 since one office would be occupied. this comes
    to $12 \times 11$ ways
  }
}

\subsection{Sum Rule}
\dfn{Sum Rule}{
  If a task can be done in either one of $n_1$ ways or in one of $n_2$ ways, where none of the set $n_1$ ways is the
  same of any of the set of $n_2$ ways, then there are $n_1 + n_2$ ways to do the task
}
The sum rule can be phrased in terms of sets
\[
  \mid A \cup B  \mid  =  \mid A \mid +  \mid B \mid + \ldots +  \mid A_m  \mid   \text{ as long as $A$ and $B$ are disjoint sets}
\]

Or

\[
  \mid A_1 \cup A_2 \cup \ldots\cup A_{m}  \mid =  \mid A_1 \mid  +  \mid A_2  \mid \text{ when } A_{i} \cap A_{j} = \O
  \text{ for all } i, j
\]

\ex{}{
  \qs{}{
    The mathematics depart must choose either a student or a faculty member as a representative for a university committee.
    How many choices are there for this representative if there are 37 member of the mathematics faculty and 83 math
    majors and no one is both a faculty member and a student.
  }

  \sol{
    \[
      37 + 83 = 120
    \]
  }
}

\ex{}{
  \qs{}{
    How many bit strings are there of length 6 or less, not including the empty strings
  }

  \sol{
    First we add all the bit strings of lengths 6, 5, 4, 3, 2, 1. \\
    To find the number of bit string of each length we use the product rule, i.e.
    \begin{align*}
      \displaystyle\sum_{i = 1}^{6} 2^{n} & = 2^{1} + 2^{2} + 2^{3} + 2^{4} + 2^{5} + 2^{6} \\
                                          & = 126                                           \\
    \end{align*}
  }
}

\subsection{Subtraction Rule}


\section{Combining the sum and product rule}

\ex{}{
  \qs{}{
    Count all passwords of length 6,7,or 8. \\
    A character in a password can either be an upper-case letter or a digit \\
    A password must contain at least 1 digit
  }

  \sol{
    Passwords of length 6 with either upper-case letter or digit - $\left( 26 + 10 \right) ^{6}$ \\
    Minus number of passwords that are only made up letters - $\left( 26 + 10 \right) ^{6}$ \\
    Times the number of digit orders - $10 \times 6$
    \[
      \left( 26 + 10 \right)^{6} - 26^{6}
    \]
    Passwords of length 6 with either upper-case letter or digit - $\left( 26 + 10 \right)^{7}$ \\
    Minus number of passwords that are only made up letters - $26^{7}$ \\
    \[
      \left( 26 + 10 \right)^{7} - 26^{7}
    \]
    Passwords of length 6 with either upper-case letter or digit - $\left( 26 + 10 \right)^{8}$ \\
    Minus number of passwords that are only made up letters - $26^{8}$ \\
    \[
      \left( 26 + 10 \right)^{8} - 26^{8}
    \]

    \[
      \left( 26 + 10 \right)^{6} - 26^{6} +       \left( 26 + 10 \right)^{7} - 26^{7} +       \left( 26 + 10 \right)^{8} - 26^{8}
    \]
  }
}

\qs{}{
  How many bit strings of length 8 start with a 1 or end with a 00
}

\sol{
  \[
    2^{7} + 2^{6} - 2^{5}
  \]
  Number of bit strings that start with 1 + Number of bit strings that end with 00 - Number of the intersection of both
}




\section{The Pigeon-hole Principle}

\dfn{Pigeon-hole Principle}{
  If $k$ is a positive integer and $k+1$ or more objects are placed into $k$ boxes, then there is at least one box
  containing two or more of the objects
}

\cor{
  A function $f$ from a set with $k + 1$ or more elements to a set with $k$ elements is not one-to-one
}

\subsection{Generalized Pigeon-hole Principle}

\dfn{Generalized Pigeon-hole Principle}{
  If $N$ objects are placed into $k$ boxes, then there is at least one box containing at least $\lceil \frac{N}{k} \rceil$
  objects. i.e.
  \[
    k \left( \left\lceil \frac{N}{k} - 1 \right\rceil  \right)  < k \left( \left( \frac{N}{k} + 1 \right) - 1  \right) = N
  \]
}

\ex{}{
  \qs{}{
    Among 100 people how many must be born in the same month
  }

  \sol{
    \begin{align*}
      \left\lceil \frac{100}{12} \right\rceil & = 9 \\
    \end{align*}
  }
}

\section{Permutations and Combinations}

\subsection{Permutations}

\dfn{Permutation}{
  An arrangement of $r$ objects from a set of $n$ objects is called a permutation of $n$ objects taken $r$ at a time,
  where the order of the objects is important. The number of permutations of $n$ objects taken $r$ at a time is denoted
  by $P(n, r)$ and is given by
  \[
    P(n, r) = \frac{n!}{(n - r)!}
  \]
}

\ex{}{
  \qs{}{
    How many permutations of the letters ABCDEFGH contain the string ABC
  }

  \sol{
    We treat the string ABC as a single object, then we have 6 objects to permute. \\
    \begin{align*}
      P \left( 6, 6 \right) & = 6!  \\
                            & = 720 \\
    \end{align*}
  }
}

\subsection{Combinations}

\dfn{Combination}{
  An arrangement of $r$ objects from a set of $n$ objects is called a combination of $n$ objects taken $r$ at a time,
  where the order of the objects is not important. The number of combinations of $n$ objects taken $r$ at a time is
  denoted by $C(n, r)$ and is given by
  \[
    C(n, r) = \frac{n!}{r! \left( n - r \right)!}
  \]
}

\nparagraph{Binomial Theorem}
\thm{Binomial Theorem}{
  Let $x$ and $y$ be variables, and let $n$ be a non-negative integer then
  \[
    \left( x + y \right)^{n} = \displaystyle\sum_{k = 0}^{n} C \left( n, k \right) x^{n - k} y^{k}
  \]
}

\ex{}{
  \qs{}{
    How many 2-combinations of the set $\{a,b,c,d\} $ are there.

  }

  \sol{
    \begin{align*}
      C \left( 4, 2 \right) & = \frac{4!}{2! \left( 4 - 2 \right)! } \\
                            & = 6                                    \\
    \end{align*}
  }
}

\ex{}{
  \qs{}{
    Suppose that there are 9 faculty members in the mathematics department. How many ways are there to select a committee
    and 11 in the computer science department. How many ways are there to develop a discrete mathematics course at a school
    if the committee is to consist of three faculty member from the mathematics department and four from the computer
    science?
  }

  \sol{
    Using the product rule, we can split the problem into two tasks, selecting the committee from the math department
    and selecting the committee from the computer science department, giving us \\
    \begin{align*}
      C \left( 9, 3 \right) \times C \left( 11, 4 \right) & = \frac{9!}{\left( 9 - 3 \right)! \times 3! } \times \frac{11!}{\left(
      11 - 4\right)! \times  4! }                                                                                                  \\
                                                          & = 27,720                                                               \\
    \end{align*}
  }
}


\section{Exercises}

\qs{}{
  There are four major auto routes from Boston to Detroit and six from Detroit to Los Angeles. How many major auto routes
  are there from Boston to Los Angeles via Detroit.
}

\sol{

  \noindent Using the product rule we can split the trips into two steps, Boston to Detroit $\left( 4 \right) $ and
  Detroit to Los Angeles $\left( 6 \right) $, giving us
  \[
    4 \times 6 =  24
  \]
}

\qs{}{
  \begin{enumerate}
    \item How many different three-letter initials can people have
    \item How many different three letter initials with none of the letters repeated can people have
  \end{enumerate}
}

\sol{
  \begin{enumerate}
    \item $26^3 = 17576$
    \item $26 \times 25 \times 24 = 15600 $
  \end{enumerate}
}

\qs{}{
  \begin{enumerate}
    \item How many bit strings of length ten both begin and end with a 1
    \item How many bit strings of length $n$, where $n$ is a positive integer, start and end with 1s
  \end{enumerate}
}

\sol{
  \begin{enumerate}
    \item $2^8$
    \item $2^{n - 2}$ where $n \geq 2$ and when $n = 1$, there is one bit string
  \end{enumerate}
}


\qs{}{
  How many stings are there of four letters that have the letter x in them
}

\sol{
  \begin{align*}
    26^{4} - 25^{4} & = 66351 \\
  \end{align*}
}

\qs{}{
  Suppose that a password for a computer system must  have at least 8 but no more 12, characters where each character in
  the password is a lower-case English letter, an upper-case English letter, a digit, or one of the six special characters,
  *, >, <, !, + and =
  \begin{enumerate}
    \item How many different passwords are available for this computer system
    \item How many of these passwords contain at least one occurrence of at least one of the six special characters.
    \item Using your answer to part 1., determine how long it takes a hacker to try every possible password assuming that
          it takes one nanosecond for a hacker to check each possible password
  \end{enumerate}
}

\sol{
  \begin{enumerate}
    \item
          The possible number of ways to choose one character is $26 + 26 + 10 + 6 = 68$
          \begin{align*}
            8: 68^8     \\
            9: 68^9     \\
            10: 68^{10} \\
            11: 68^{10} \\
            12: 68^{12} \\
            \displaystyle\sum_{12}^{i = 8} 68^{i}
          \end{align*}
    \item
          \begin{align*}
            \displaystyle\sum_{12}^{i = 8} 68^i - \displaystyle\sum_{12}^{i = 8} 62^{i}
          \end{align*}
  \end{enumerate}
}

\qs{}{
  Show that if there are 30 students in a class, then at least two have last names that begin with the same letter
}

\sol{
  $N = 30$ objects to be placed into $k = 26$ boxes \\
  According to the generalized pigeon hole principle:
  \begin{align*}
    \left\lceil \frac{30}{26} \right\rceil & = 2 \\
  \end{align*}
  $\therefore$ at least 2
}

\qs{}{
  Show that there are at least siz people in California (population: 37 million) with the same three initials who were
  born on the same day of the year (but not necessarily in the same year). Assume that everyone has three initials
}

\sol{
  $N = 37 \text{ million}$ to be place into $k$ boxes \\
  $k = 26^3 \times 366$ \\
  According to the generalized pigeon hole principle:
  \begin{align*}
    \left\lceil \frac{37,000,000}{26^3 \times 366} \right\rceil & = 6 \\
  \end{align*}
  $\therefore$ at least 6

}

\qs{}{
  Let $S = \{1,2,3,4,5\} $
  \begin{enumerate}
    \item List all the 3-permutations of $S$
    \item List all the 3-combinations of $S$
  \end{enumerate}
}

\sol{
  \begin{enumerate}
    \item
          \begin{align*}
            P \left( 5,3 \right) & = \frac{5!}{3!} \\
                                 & = 60            \\
          \end{align*}
    \item
          \begin{align*}
            C \left( 5,3 \right) & = \frac{5!}{2!\times 3!} \\
                                 & = 10                     \\
          \end{align*}
          \begin{align*}
            \{1,2,3\} , \{1,2,4\} , \{1,2,5\} , \{1,3,5\}, \\
            \{1,3,4\} , \{3,4,5\} , \{2,4,5\} ,            \\
            \{1,4,5\} , \{2,3,5\} , \{2,3,4\}
          \end{align*}
  \end{enumerate}
}

\qs{}{
  A club has 25 members
  \begin{enumerate}
    \item How many ways are there to choose four members of the club to serve on an executive committee
    \item How many ways are there to choose a president, vice-president, secretary, and treasurer of the club where no
          person can hold more than one office
  \end{enumerate}
}

\sol{
  \begin{enumerate}
    \item
          \begin{align*}
            C \left( 25, 4 \right) & = \frac{25!}{4! \times \left( 25 - 4 \right) !} \\
                                   & = 12,650
          \end{align*}
    \item
          \begin{align*}
            25 \times 24 \times 23 \times 22 & =  303,600
          \end{align*}
  \end{enumerate}
}

\qs{}{
  Seven women and nine men are on the faculty in the mathematics department of a school
  \begin{enumerate}
    \item How many ways are there to select a committee of five members of the department if at least one woman must be
          on the committee
    \item How many ways are there to select a committee of five members of the department if at least one woman and at
          least one man must be on the committee.
  \end{enumerate}
}

\sol{
  \begin{enumerate}
    \item
          First we find the number of all the possible ways a five member committee can be selected, i.e. $C \left( 16, 5
            \right) $, then we find the number of ways a committee can be selected consisting of all men, i.e. $C \left(
            9, 5 \right)$
          \begin{align*}
            C \left( 16, 5 \right)  - C \left( 9, 5 \right) & =  4242
          \end{align*}
    \item
          First we find the number of all the possible ways a five member committee can be selected, i.e. $C \left( 16, 5
            \right) $, then we find the number of ways a committee can be selected consisting of all men and all women,
          $C \left( 9, 5 \right) $ and $C \left( 7, 5 \right) $, respectively
          \begin{align*}
            C \left( 16, 5 \right) - \left( C \left( 9, 5 \right) + C \left( 7, 5 \right)   \right) = 4221
          \end{align*}
  \end{enumerate}

}

\qs{}{
  How many subsets of a set with 10 elements
  \begin{enumerate}
    \item Have fewer than 4 elements
    \item Have more than 7 elements
  \end{enumerate}
}

\sol{
  \begin{enumerate}
    \item
          \begin{align*}
            C \left( 10, 4 \right)  + C \left( 10, 3 \right) + C \left( 10, 2 \right) + C \left( 10, 1 \right) + C \left(
            10, 0 \right) & =  386
          \end{align*}
  \end{enumerate}
}

\qs{}{
  Find the expansion of $\left( x + y \right)^{6}$
}

\sol{
  \begin{align*}
    \left( x + y \right)^{6} & = \displaystyle\sum_{i = 0}^{6} C \left( 6, i \right) \times x^{n - i}\times y^{i}                                  \\
                             & = C \left( 6, 0 \right) x^{6} + C \left(6, 1  \right) x^{5}y^{1} + C \left( 6, 2 \right) x^{4}y^{2} + C \left( 6, 3
    \right) x^{3}y^{3} + C \left( 6, 4 \right) x^{2}y^{4} + C \left( 6, 5 \right) x^{1}y^{5} + C \left( 6, 6 \right)
    y^{6}                                                                                                                                          \\
                             & = x^{6} + 6x^{5}y + 15x^{4}y^2 + 20x^3y^3 + 15x^2y^{4} + 6xy^{5} + y^{6}                                            \\
  \end{align*}
}

\chapter{Exercises}

\qs{}{
  Suppose you have 30 books (15 novels, 10 history books, and 5 math books). Assume that all 30 books
  are different.
  \begin{enumerate}
    \item  In how many ways can you put the 30 books in a row on a shelf?
    \item In how many ways can you get a bunch of four books to give to a friend?
    \item In how many ways can you get a bunch of three history books and seven novels to give to a friend?
    \item In how many ways can you put the 30 books in a row on a shelf if the novels are on the left, the math books are in the middle, and the history books are on the right?
  \end{enumerate}
}

\sol{
  \begin{enumerate}
    \item
          As each of the books are distinct the order matters, therefore using permutation:
          \begin{align*}
            P \left( 30, 30 \right) & = \frac{30!}{0!} \\
                                    & = 30!            \\
          \end{align*}
    \item
          As the order and type of book doesn't matter in this case, using combination:
          \begin{align*}
            C \left( 30, 4 \right) & = \frac{30!}{\left( 30 - 4 \right)! 4! } \\
                                   & = 27405                                  \\
          \end{align*}
    \item As we want $3$ out of the $10$ history books and $7$ out of the $15$ novels where the order doesn't
          matter using combinations and product rule:
          \begin{align*}
            C \left( 10, 3 \right)  \times  C \left( 15, 7  \right) & = \frac{10!}{\left( 10 - 3 \right)! 3! } \times
            \frac{15!}{\left( 15 - 7  \right)! 7! }                                                                   \\
                                                                    & = 120 \times 6435                               \\
                                                                    & = 772200                                        \\
          \end{align*}
    \item
          Breaking this task into 3 separate steps as this deals with the ordering of distinct items the order matters
          in each step, with the first choosing novels, the second choosing math books, and the third choosing history
          books, using permutation and the product rule:
          \begin{align*}
            P \left( 15, 15 \right) \times P \left( 10, 10 \right) \times P \left( 5, 5 \right) & = 15! \times 10!
            \times 5!                                                                                                        \\
                                                                                                & = 15! \times 10! \times 5! \\
          \end{align*}
  \end{enumerate}
}

\qs{}{
  What is the minimum number of students, each of whom comes from one of the 50 states of the USA,
  who must be enrolled in a university to guarantee that there are at least 100 who come from the same
  state?
}

\sol{
  Using the generalized Pigeon-hole principle we have our $k$ as the number of states, 50 and we must find $N$ such that
  $\left\lceil \frac{N}{50} \right\rceil $:
  \begin{align*}
    \left\lceil \frac{N}{50} \right\rceil = 100 \\
    99 < \frac{N}{50} \leq 100                  \\
    \\
    \frac{N}{50} > 99                           \\
    N > 4950
    \\
    \frac{N}{50} \leq 100
    N \leq 5000                                 \\
    \therefore 4950 < N \leq 5000               \\
  \end{align*}
  Hence the minimum $N$ is 4951
}

\qs{}{
  \begin{enumerate}
    \item Find the expansion of $\left( x + y \right)^6 $
    \item Find the coefficient of $x^6y^9$ in the expansion of $\left( 3x -2y \right)^{15} $
  \end{enumerate}
}

\sol{
  \begin{enumerate}
    \item
          \begin{align*}
            \left( x + y \right)^6 & = C \left( 6, 0 \right) x^{6} + C \left( 6, 1 \right) x^{5}y^{1} + C \left( 6, 2 \right)
            x^{4}y^{2} + C \left( 6, 3 \right)  x^{3}y^{3} + C \left( 6, 4 \right) x^{2}y^{4} + C \left( 6, 5 \right)
            x^{1}y^{5} + C \left( 6, 6 \right) x^{0}y^{6}                                                                     \\
                                   & = x^{6} + 6x^{5}y + 15x^{4}y^{2} + 20x^{3}y^{3} + 15x^{2}y^{4} + 6xy^{5} + y^{6}         \\
          \end{align*}
    \item
          Let $a$ be the coefficient of $x^{6}y^{9}$
          \begin{align*}
            ax^{6}y^{9} & = C \left( 15, 9 \right) \left( 3x \right)^{6} \left(- 2y \right) ^{9} \\
                        & = 5005 \times 729x^{6} \times -512y^{9}                                \\
                        & = -5005\times 729 \times 512 x^{6}y^{9}                                \\
            a           & = - 5005 \times 729 \times 512                                         \\
          \end{align*}
  \end{enumerate}
}

\end{document}
