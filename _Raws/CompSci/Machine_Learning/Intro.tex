\documentclass[12pt letter]{report}
\input{./template/preamble}
\input{./template/macros}
\input{./template/letterfonts}

\title{\Huge{Introduction}}
\author{\huge{Madiba Hudson-Quansah}}
\date{}
\usepackage{parskip}

\setcounter{tocdepth}{4}
\setcounter{secnumdepth}{4}

\begin{document}
\maketitle
\newpage
\pdfbookmark[section]{\contentsname}{too}
\tableofcontents
\pagebreak

\chapter{Introduction}

\section{Machine Learning}

\begin{itemize}
  \item Performing a Task
  \item With Experience
  \item Improving Performance
\end{itemize}

\section{Artificial Intelligence (AI)}


\dfn{Artificial Intelligence}{
  The science and engineering of making intelligent machines, especially intelligent computer programs.
}

\section{Deep Learning vs Machine Learning}

\subsection{Machine Learning}
\begin{itemize}
  \item Subfield of AI focused on algorithms that learn from data.
  \item Works well with structured data.
  \item Simpler models.
  \item Requires manual feature extraction and selection.
  \item Involves predictive modelling, clustering, and classification.
  \item Feature extraction and application are done separately.
\end{itemize}

\subsection{Deep Learning}
\begin{itemize}
  \item Subfield of ML using neural networks with many layers.
  \item  Works well with large amounts of unstructured data.
  \item  Complex models with multiple layers.
  \item Automatically extracts features from raw data.
  \item Involves image and speech recognition, natural language processing, and recommendation systems.
  \item Feature extraction and application are done together by the neural network.
\end{itemize}

\section{Supervised Learning}

\dfn{Supervised Learning}{
  A subfield of Machine Learning where labelled datasets are used to train algorithms that classify data or predict
  outcomes.
}

\subsection{Terminology}

\dfn{Feature / Input Feature / Independent Variable / $X$}{
  A feature is an individual measurable property or characteristic of a phenomenon being observed.
}

\dfn{Label / Dependent Variable / $Y$}{
  The output / target variable that we are trying to predict.
}

\dfn{Classification}{
  Involves predicting a categorical label.
}

\dfn{Regression}{
  Involves predicting a quantitative continuous label.
}

\subsection{Supervised Learning Pipeline}

\begin{enumerate}
  \item Determine the type of training dataset.
  \item Gather the labelled training data.
  \item Split the training dataset into training dataset, test dataset.
  \item Determine the most suitable algorithm for the model.
  \item Execute the algorithm on the training dataset.
  \item Evaluate the accuracy of the model by providing the test set.
\end{enumerate}

\dfn{Independent Identical Distribution  (IID)}{

}

\subsection{Math}
For a model:
\[
  h \left( x \right)  = \theta_0 + \theta_1 x
\]
Where $h \left( x\right) $ is he hypothesis, The $\theta$ are our parameters, and $x$ is an input feature.
\[
  h \left( x \right) = \mbold{\theta} \cdot \mbold{x}
\]
Where $x_0 = 1$, where the number of elements in $\mbold{\theta}$ and $\mbold{x}$ is $n+1$ or
\[
  h \left( x \right) = \displaystyle\sum_{i = 0}^{n} \theta_i x_i
\]
Where $x_0 = 1$, For multiple input features.

For the training set $ \left( X^i, Y^i\right) $, represents the $i$-th input and the  $i$-th label





\end{document}
