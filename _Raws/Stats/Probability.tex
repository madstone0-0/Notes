\documentclass[12pt letter]{report}
\input{./template/preamble}
\input{./template/macros}
\input{./template/letterfonts}

\title{\Huge{Probability}}
\author{\huge{Madiba Hudson-Quansah}}
\date{}
\usepackage{parskip}

\begin{document}
\maketitle
\newpage
\pdfbookmark[section]{\contentsname}{too}
\tableofcontents
\pagebreak

\chapter{Module 8: Introduction}

\section{Introduction To Probability}

\dfn{Probability}{
	A mathematical description of randomness and uncertainty / The likelihood of an event occurring.

	The notation for Probability is $\mathbb{P} \left( X \right) $ where $X$ is the event.

	Probability is always between $0 \leq \mathbb{P} \left( X \right) \leq 1$ or $0\% \leq \mathbb{P} \left( X \right) \leq 100\%$.
}

There are two ways of determining probability:

\begin{itemize}
	\item Theoretical / Classical - Determined by the nature of the experiment
	\item Empirical / Observational - Determined by the results of the experiment
\end{itemize}

\section{Relative Frequency}

\dfn{Relative Frequency}{
	Relative frequency is the number of times an event occurs divided by the total number of trials.

	\[
		\mathbb{P} \left( X \right) = \frac{\text{Number of times event occurs}}{\text{Total number of trials}}
	\]
}

\thm{The Law of Large Numbers}{
	As the number of trials increases, the relative frequency of an event approaches the theoretical probability of the event.
}


\chapter{Module 9: Find the Probability of Events}

\section{Sample Spaces and Events}

\dfn{Random Experiment}{
	An experiment whose outcome is determined by chance.
}

\dfn{Sample Space}{
	The list of possible outcomes of a random experiment, denoted by $S$.
}

\dfn{Event}{
	A statement about the nature of the outcome after the experiment has been conducted, denoted by any capital letter
	except $S$.
}

\section{Equally Likely Outcomes}

\[
	\mathbb{P} \left( A \right) = \frac{\text{Number of outcomes in }A}{\text{Number of outcomes in } S}
\]

Where $A$ is an event and $S$ is the sample space.

\section{Probability Rules}

\subsection{Rule 1: Probability is a Number Between 0 and 1}

For any event $A$, $0 \le \mathbb{P} \left( A \right) \le 1$.


\subsection{Rule 2: Addition Rule}

$\mathbb{P} \left( S \right) = 1 $, that is the sum of the probabilities of all possible outcomes is 1.

\subsection{Rule 3: Complement Rule}

$\mathbb{P} \left( A' \right) = 1 - \mathbb{P} \left( A \right)  $, that is the probability of the complement of an event is 1 minus the
probability the event occurs.


\subsection{Rule 4: Addition Rule for Mutually Exclusive Events}

\dfn{Mutually Exclusive / Disjoint events}{
	Events that cannot happen at the same time.

}

$\mathbb{P} \left( A \text{ or } B \right) = \mathbb{P} \left( \text{ event } A \text{ occurs or event } B \text{ occurs
		or both occur} \right)  $\\


\noindent If $A$ and $B$ are mutually exclusive, then $\mathbb{P} \left( A \text{ or } B \right) = \mathbb{P} \left( A
	\right) + \mathbb{P} \left( B \right)   $

\subsection{Rule 5: Multiplication Rule for Independent Events}
$\mathbb{P} \left( A \text{ and } B \right) = \mathbb{P} \left( \text{ event } A \text{ occurs and event } B \text{
		occurs } \right)  $

\dfn{Independent Events}{
	Two events $A$ and $B$ are said to be independent if the occurrence of one event does not affect the probability of the
	other event occurring.

}

\dfn{Dependent Events}{
	Two events $A$ and $B$ are said to be dependent if the occurrence of one event affects the probability of the other event
	occurring.

}

\nodindent If $A$ and $B$ are two independent events, then $\mathbb{P} \left( A \text{ and } B \right) = \mathbb{P} \left( A
	\right)  \times \mathbb{P} \left( B \right)  $

\subsection{Rule 6: General Addition Rule}

For any two events $A$ and $B$, $\mathbb{P} \left( A \text{ or } B \right) = \mathbb{P} \left( A \right) + \mathbb{P}
	\left( B \right) - \mathbb{P} \left( A \text{ and } B \right)    $


\chapter{Conditional Probability and Independence}

\dfn{Conditional Probability}{
	The probability an event occurs as a result of another event. i.e. probability of event $B$, given event event $A$
	is,

	\[
		\mathbb{P} \left( B  \mid  A \right)  = \frac{\mathbb{P} \left( A \text{ and } B \right) }{P \left( A \right) }
	\]
}

\section{Independence}

When two events are independent, the probability of one event occurring does not affect the probability of the other
event, i.e.

\begin{align*}
	\mathbb{P} \left( B  \mid  A \right) = \mathbb{P} \left( B \right)           \\
	\mathbb{P} \left( A  \mid B \right)  = \mathbb{P} \left( A \right)           \\
	\mathbb{P} \left( B  \mid A \right)  = \mathbb{P} \left( B  \mid A'  \right) \\
	\mathbb{P} \left( A \text{ and } B \right) = \mathbb{P} \left( A \right)  \times \mathbb{P} \left( B \right)
\end{align*}

\section{The General Multiplication Rule}

For any two events $A$ and $B$
\[
	\mathbb{P} \left( A \text{ and } B \right) = \mathbb{P} \left( A \right) \times \mathbb{P} \left( B  \mid A \right)
\]

\section{Probability Trees}

\dfn{Probability Tree}{
	A diagram that shows the sample space of a random experiment and the probability of each outcome.
}

\end{document}
