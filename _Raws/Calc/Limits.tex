\documentclass[12pt letter]{report}
\input{./template/preamble}
\input{./template/macros}
\input{./template/letterfonts}

\title{\Huge{Limits}}
\author{\huge{Madiba Hudson-Quansah}}
\date{}
\usepackage{parskip}

\begin{document}
\maketitle
\newpage
\pdfbookmark[section]{\contentsname}{toc}
\tableofcontents
\pagebreak

\chapter{}
\section{The Limit of a Function}
Reporting about the behaviour of function within the range of its dangerous values


\begin{displaymath}
	f(x) = x^2 + \frac{1}{x}
\end{displaymath}

Input variable = $x$

Output variable = $f(x)$

Name of function = $f$


"Acceptable"/Permissible input values of x - All real numbers except zero

\begin{gather*}
	x, f(x) \;\; x+h, f(x+h)      \\
	f(x+h) - \frac{f(x)}{x+h-x} \\
	f(x+h) - \frac{f(x)}{h}         \\
\end{gather*}

\subsection{Proof}

\[y = -16t^2+100t+6\]

Points used:
$(0, 6) \; (1, 90) \; (3,162)$

When $t = 0$ and $y = 6$

\begin{align*}
	y = at^2 + bt+c      \\
	6 = a(0)^2 + b(0) +c \\
	 & c = 6
\end{align*}

When $t=1$ and $y=90$

\begin{align*}
	90 = a(1)^2 +b +6 \\
	90 = a + b + 6    \\
	84 = a + b        \\
	84 - b = a
\end{align*}

When $t=3$ and $y=162$

\begin{align*}
	162 = a(3)^2 + b + 6     \\
	162 = 9a + 3b + 6        \\
	162 = 9(84 - b) + 3b + 6 \\
	162 = 756 - 9b +3b +6    \\
	-594 = -6b +6            \\
	-600 = -6b               \\
	 & b = 100               \\[20pt]
	\therefore b = 100
\end{align*}

\begin{align*}
	84 - 100 = a \\
	 & a = -16
\end{align*}

Therefore $a=16$, $b=100$ and $c=6$

\qs{}{Given $f(x) = x^2$ find the Limit of $f(x)$ at $x=3$}
\sol{}{
	\begin{align*}
		\text{As x} \to 3^-, \text{ } f(x) \text{->} 9 \\
		\text{As x} \to 3^+, \text{ } f(x) \text{->} 9
	\end{align*}
	Or
	\begin{align*}
		\lim_{x\to3^-} f(x) =9 \\
		\lim_{x\to3^+} f(x) = 9
	\end{align*}
	The first $9$ is known as the left limit of $f(x)$ and the other $9$ is known as the right limit of $f(x)$

	Therefore the limit of $f(x)$ at $x=3$ is:
	\begin{align*}
		\lim_{x\to3} f(x) = 9
	\end{align*}

	This is because the left limit and right limit converge.
}

\noindent\large{In the case where:}

\begin{align*}
	\lim_{x\to1^-} f(x) = 5 \\
	\lim_{x\to1^+} f(x) = 4
\end{align*}

\noindent The left and right limits do not converge so there is no limit of $f(x)$ for $x = 1$ and is written as:

\[
	\lim_{x\to1} f(x) = \text{No such unique number}
	.\]
\pagebreak

\noindent\large{In the case where one limit does not exist, i.e. Increasing without bounds:}

\begin{align*}
	\lim_{x \to 1^-} f(x) \to \infty \\
	\lim_{x \to 1^+} f(x) \to 4
\end{align*}

The limit does not exist because the left limit does not exist. This is written as:

\begin{align*}
	\lim_{x\to1} f(x) = \text{does not exist} \\
\end{align*}

\end{document}
