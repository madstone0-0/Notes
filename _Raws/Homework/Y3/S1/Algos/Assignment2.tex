\documentclass[12pt letter]{report}
\input{../template/preamble}
\input{../template/macros}
\input{../template/letterfonts}

\usepackage{caption}% http://ctan.org/pkg/caption
\usepackage{parskip}
\title{\Huge{Assignment 2}}
\author{\huge{Madiba Hudson-Quansah}}
\date{}


\begin{document}
\maketitle
\newpage

\section*{Part A}

\qs{}{
  A network topology specifies how computers, printers, and other devices are connected
  over a network. You are given a boolean matrix $A \left[ 0\ldots n - 1, 0 \ldots n - 1 \right] $, where $n > 3$, which is supposed to be
  the adjacency matrix of a graph modeling a network with one of these topologies. \\
  Your task is to determine which of these three topologies, if any, the matrix represents. Design a brute-force algorithm for this task and indicate its time efficiency class.
}

\sol{
  \begin{algorithm}[H]
    \caption{DetermineTopology $ \left( A, n \right) $}
    \Comment{}\\
    \Comment{Determine if the network topology of a given adjacency matrix $A$ is or isn't any of the three mentioned
      topologies} \\
    \Comment{Input: An boolean adjacency matrix $A$ of size $n\times n$} \\
    \Comment{Output: 1 if star, 2 if mesh, 3 if ring and -1 if none} \\
    \begin{algorithmic}[1]
      \State isStar $ \gets T$
      \State isRing $ \gets T$
      \State isMesh $ \gets T$
      \State $mI  \gets -1$
      \State allTCount $ \gets 0$
      \State oneTCount $ \gets 0$\\

      \For{ $i \gets 0$ to $n -1$}
      \State tCount $ \gets 0$
      \For{ $j \gets 0$ to $n-1$}
      \If{ $ i == j$}
      continue
      \EndIf\\

      \If{ $A_{ij} == T$}
      \State tCount $ \gets \text{tCount } + 1$
      \EndIf
      \EndFor\\

      \If{tCount $== 1$}
      \State oneTCount $ \gets \text{oneTCount } + 1$
      \EndIf \\

      \If{tCount $\neq  2$}
      \State isRing $ \gets F$
      \EndIf \\


      \If{ $\text{tCount} == n - 1$}
      \If{ $mI == -1$}
      \State $mI \gets i$
      \Else
      \State isStar $ \gets F$
      \EndIf
      \State allTCount $ \gets \text{allTCount } + 1$
      \EndIf
      \EndFor
      \algstore{determinetopo}


    \end{algorithmic}
  \end{algorithm}

  \begin{algorithm}[H]
    \caption{DetermineTopology (Continued) $ \left( A, n \right) $}
    \begin{algorithmic}[1]
      \algrestore{determinetopo}
      \If{ allTCount $\neq m$}
      \State isMesh $ \gets F$
      \EndIf \\

      \If{ $mI == -1$}
      \State isStar $ \gets F$
      \EndIf \\

      \If{oneTCount $\neq m - 1$}
      \State isStar $ \gets F$
      \EndIf \\


      \If{isStar}
      \State \Return 1
      \ElsIf{isMesh}
      \State \Return 2
      \ElsIf{isRing}
      \State \Return 3
      \Else
      \State \Return -1
      \EndIf
    \end{algorithmic}
  \end{algorithm}
}

\qs{}{
  "Word find" (or "Word search") puzzles ask the player to find each of a given set of
  words in a square table filled with single letters. A word can read horizontally (left or
  right), vertically (up or down), or along a 45 degree diagonal (in any of the four
  directions) formed by consecutively adjacent cells of the table; it may wrap around the
  table’s boundaries, but it must read in the same direction with no zigzagging. The same
  cell of the table may be used in different words, but, in a given word, the same cell may
  be used no more than once. Design a brute-force algorithm for solving this puzzle.
}

\sol{
  \begin{algorithm}[H]
    \caption{WordFind $ \left( A, n, \text{targetWords}, m \right) $}
    \Comment{}\\
    \Comment{Finds the indices and directions of a set of words in a word find puzzle} \\
    \Comment{Input: The word square $A$ with dimensions $n\times n$, A set of target words of length $m$} \\
    \Comment{Output: A list containing the word, the start index and the found direction of each of the found words in the
      word square in the format $ \left[\text{word}, \text{start}, \text{direction} \right] $} \\
    \begin{algorithmic}[1]
      \Function{checkHR}{ $\text{word}, w, A, i, j$} \Comment{Function to check if a word can be found horizontally
        right} \\
      \Comment{word - the word to find, $w$ - the length of the word, $A$ - the word square, $i$ - the row index, $j$ -
        the column index}
      \State $\text{idx} \gets j$
      \State $\text{count} \gets 0$
      \While{ $\text{count} < w$}
      \If{ $\text{word}[ \text{count} ] \neq  A[i][ \text{idx} ]$}
      \State \Return $F$
      \EndIf
      \State $\text{idx} \gets \left( \text{idx} + 1 \right) \bmod n $
      \State count $ \gets \text{count} + 1 $
      \EndWhile
      \State \Return $T$
      \EndFunction \\

      \Function{checkHL}{ $\text{word}, w, A, i, j$} \Comment{Function to check if a word can be found horizontally
        left} \\

      \Comment{word - the word to find, $w$ - the length of the word, $A$ - the word square, $i$ - the row index, $j$ -
        the column index}
      \State $\text{idx} \gets j$
      \State $\text{count} \gets 0$
      \While{ $\text{count} < w$}
      \If{ $\text{word}[ \text{count} ] \neq  A[i][ \text{idx} ]$}
      \State \Return $F$
      \EndIf
      \State $\text{idx} \gets \left( \text{idx} - 1 \right) \bmod n $
      \State count $ \gets \text{count} + 1 $
      \EndWhile
      \State \Return $T$
      \EndFunction \\
      \algstore{wordfind}

    \end{algorithmic}
  \end{algorithm}

  \begin{algorithm}[H]
    \ContinuedFloat
    \caption{WordFind (Continued) $ \left( A, n, \text{targetWords}, m \right) $}
    \begin{algorithmic}[1]
      \algrestore{wordfind}
      \Function{checkVD}{ $\text{word}, w, A, i, j$} \Comment{Function to check if a word can be found vertically down} \\
      \Comment{word - the word to find, $w$ - the length of the word, $A$ - the word square, $i$ - the row index, $j$ -
        the column index}
      \State $\text{idx} \gets i$
      \State $\text{count} \gets 0$
      \While{ $\text{count} < w$}
      \If{ $\text{word}[ \text{count} ] \neq  A[ \text{idx} ][j]$}
      \State \Return $F$
      \EndIf
      \State $\text{idx} \gets \left( \text{idx} + 1 \right) \bmod n $
      \State count $ \gets \text{count} + 1 $
      \EndWhile
      \State \Return $T$
      \EndFunction \\

      \Function{checkVU}{ $\text{word}, w, A, i, j$} \Comment{Function to check if a word can be found vertically up} \\
      \Comment{word - the word to find, $w$ - the length of the word, $A$ - the word square, $i$ - the row index, $j$ -
        the column index}
      \State $\text{idx} \gets j$
      \State $\text{count} \gets 0$
      \While{ $\text{count} < w$}
      \If{ $\text{word}[ \text{count} ] \neq  A[ \text{idx} ][j]$}
      \State \Return $F$
      \EndIf
      \State $\text{idx} \gets \left( \text{idx} - 1 \right) \bmod n $
      \State count $ \gets \text{count} + 1 $
      \EndWhile
      \State \Return $T$
      \EndFunction \\

      \Function{checkDTR}{ $\text{word}, w, A, i, j$} \Comment{Function to check if a word can be found diagonally top
        right} \\
      \Comment{word - the word to find, $w$ - the length of the word, $A$ - the word square, $i$ - the row index, $j$ -
        the column index}
      \State $\text{idxR} \gets i$
      \State $\text{idxC} \gets j$
      \State $\text{count} \gets 0$
      \While{ $\text{count} < w$}
      \If{ $\text{word}[\text{count}] \neq A[\text{idxR}][\text{idxC}]$}
      \State \Return $F$
      \EndIf
      \State $\text{idxR} \gets \left( \text{idxR} - 1 \right) \bmod n $
      \State $\text{idxC} \gets \left( \text{idxC} + 1 \right) \bmod n $
      \State count $ \gets \text{count} + 1 $
      \EndWhile
      \State \Return $T$
      \EndFunction \\

      \algstore{wordfind}
    \end{algorithmic}
  \end{algorithm}
}

\begin{algorithm}[H]
  \caption{WordFind (Continued) $ \left( A, n, \text{targetWords}, m \right) $}
  \begin{algorithmic}[1]
    \algrestore{wordfind}
    \Function{checkDTL}{ $\text{word}, w, A, i, j$} \Comment{Function to check if a word can be found diagonally top
      left} \\
    \Comment{word - the word to find, $w$ - the length of the word, $A$ - the word square, $i$ - the row index, $j$ -
      the column index}
    \State $\text{idxR} \gets i$
    \State $\text{idxC} \gets j$
    \State $\text{count} \gets 0$
    \While{ $\text{count} < w$}
    \If{ $\text{word}[\text{count}] \neq A[\text{idxR}][\text{idxC}]$}
    \State \Return $F$
    \EndIf
    \State $\text{idxR} \gets \left( \text{idxR} - 1 \right) \bmod n $ \Comment{Move up}
    \State $\text{idxC} \gets \left( \text{idxC} - 1 \right) \bmod n $ \Comment{Move left}
    \State $\text{count} \gets \text{count} + 1$
    \EndWhile
    \State \Return $T$
    \EndFunction \\

    \Function{checkDBL}{ $\text{word}, w, A, i, j$} \Comment{Function to check if a word can be found diagonally bottom
      left} \\
    \Comment{word - the word to find, $w$ - the length of the word, $A$ - the word square, $i$ - the row index, $j$ -
      the column index}
    \State $\text{idxR} \gets i$
    \State $\text{idxC} \gets j$
    \State $\text{count} \gets 0$
    \While{ $\text{count} < w$}
    \If{ $\text{word}[\text{count}] \neq A[\text{idxR}][\text{idxC}]$}
    \State \Return $F$
    \EndIf
    \State $\text{idxR} \gets \left( \text{idxR} + 1 \right) \bmod n $ \Comment{Move down}
    \State $\text{idxC} \gets \left( \text{idxC} - 1 \right) \bmod n $ \Comment{Move left}
    \State $\text{count} \gets \text{count} + 1$
    \EndWhile
    \State \Return $T$
    \EndFunction \\

    \Function{checkDBR}{ $\text{word}, w, A, i, j$} \Comment{Function to check if a word can be found diagonally bottom
      right} \\
    \Comment{word - the word to find, $w$ - the length of the word, $A$ - the word square, $i$ - the row index, $j$ -
      the column index}
    \State $\text{idxR} \gets i$
    \State $\text{idxC} \gets j$
    \State $\text{count} \gets 0$
    \While{ $\text{count} < w$}
    \If{ $\text{word}[\text{count}] \neq A[\text{idxR}][\text{idxC}]$}
    \State \Return $F$
    \EndIf
    \State $\text{idxR} \gets \left( \text{idxR} + 1 \right) \bmod n $ \Comment{Move down}
    \State $\text{idxC} \gets \left( \text{idxC} + 1 \right) \bmod n $ \Comment{Move right}
    \State $\text{count} \gets \text{count} + 1$
    \EndWhile
    \State \Return $T$
    \EndFunction \\

    \State foundIds $ \gets \left[  \right] $
    \State $\text{wIdx} \gets 0$
    \State $\text{found} \gets F$
    \algstore{wordfind}
  \end{algorithmic}
\end{algorithm}

\begin{algorithm}[H]
  \caption{WordFind (Continued) $ \left( A, n, \text{targetWords}, m \right) $}
  \begin{algorithmic}[1]
    \algrestore{wordfind}
    \For{ $\text{idx} \gets 0$ to $m-1$}
    \State word $ \gets \text{targetWords}[\text{idx}]$
    \State $w \gets$ length of word
    \State $\text{found} \gets F$
    \For{ $i \gets 0$ to $n-1$}
    \If{ $\text{found}$}  \Comment{Assuming a word only appears once we break out of the word loop once it is found}
    \State break
    \EndIf
    \For{ $j \gets 0$ to $n-1$}
    \If{\Call{checkHR}{ $\text{word}, w, A, i, j$}}
    \State $\text{foundIds}[\text{wIdx}] \gets [ \text{word}, [i,j], \text{"HR"}]$
    \State $\text{wIdx} \gets \text{wIdx} + 1$
    \State $\text{found} \gets T$
    \State break
    \ElsIf{\Call{checkHL}{ $\text{word}, w, A, i, j$}}
    \State $\text{foundIds}[\text{wIdx}] \gets [\text{word},[i,j], \text{"HL"}]$
    \State $\text{wIdx} \gets \text{wIdx} + 1$
    \State $\text{found} \gets T$
    \State break
    \ElsIf{\Call{checkVD}{ $\text{word}, w, A, i, j$}}
    \State $\text{foundIds}[\text{wIdx}] \gets [\text{word},[i,j], \text{"VD"}]$
    \State $\text{wIdx} \gets \text{wIdx} + 1$
    \State $\text{found} \gets T$
    \State break
    \ElsIf{\Call{checkVU}{ $\text{word}, w, A, i, j$}}
    \State $\text{foundIds}[\text{wIdx}] \gets [\text{word},[i,j], \text{"VU"}]$
    \State $\text{wIdx} \gets \text{wIdx} + 1$
    \State $\text{found} \gets T$
    \State break
    \ElsIf{\Call{checkDTR}{ $\text{word}, w, A, i, j$}}
    \State $\text{foundIds}[\text{wIdx}] \gets [\text{word},[i,j], \text{"TR"}]$
    \State $\text{wIdx} \gets \text{wIdx} + 1$
    \State $\text{found} \gets T$
    \State break
    \ElsIf{\Call{checkDBR}{ $\text{word}, w, A, i, j$}}
    \State $\text{foundIds}[\text{wIdx}] \gets [\text{word},[i,j], \text{"BR"}]$
    \State $\text{wIdx} \gets \text{wIdx} + 1$
    \State $\text{found} \gets T$
    \State break
    \ElsIf{\Call{checkDTL}{ $\text{word}, w, A, i, j$}}
    \State $\text{foundIds}[\text{wIdx}] \gets [\text{word},[i,j], \text{"TL"}]$
    \State $\text{wIdx} \gets \text{wIdx} + 1$
    \State $\text{found} \gets T$
    \State break
    \ElsIf{\Call{checkDBL}{ $\text{word}, w, A, i, j$}}
    \State $\text{foundIds}[\text{wIdx}] \gets [\text{word},[i,j], \text{"BL"}]$
    \State $\text{wIdx} \gets \text{wIdx} + 1$
    \State $\text{found} \gets T$
    \State break
    \EndIf
    \EndFor
    \EndFor
    \EndFor \\
    \State \Return $\text{foundIds}$
  \end{algorithmic}
\end{algorithm}

\qs{}{
  \begin{enumerate}
    \item Apply the DFS-based algorithm to solve the topological sorting problem for the
          following directed acyclic graph.
    \item Apply the source-removal algorithm to solve the topological sorting problem for the
          following directed acyclic graph.
  \end{enumerate}
}

\sol{
  \begin{enumerate}
    \item The DFS-based algorithm involves performing a DFS on all the nodes of a graph and reversing the order each
          node was visited to obtain a topological sort. This can be done by observing the stack frames made by the
          recursive algorithm and assigning numbers to each of the nodes based on the order they where pushed on to
          the stack and popped of the stack. Assuming the nodes are arranged in alphabetical order and prioritizing
          alphabetical order, the stack  order
          is:
          \begin{align*}
            a\, 1, 6 \\
            b\, 2, 4 \\
            e\, 3, 1 \\
            g\, 4, 3 \\
            f\, 5, 2 \\
            c\, 6, 5 \\
            d\, 7, 7 \\
          \end{align*}
          And the resulting topological sort is:\\
          \[
            d, a, c, b, g, f, e
          \]

    \item Using the source-removal algorithm, we remove nodes with an in-degree of 0 and add them to the beginning of
          the topological sort. But since this graph has a cycle from $e$ to $e$, after the first source removal of $f$,
          there are no other nodes with an in-degree of 0. Thus the graph has no topological sort and a partial topological sort is of just $f$.
  \end{enumerate}
}

\qs{}{
  A digraph is called strongly connected if for any pair of two distinct vertices $u$ and $v$
  there exists a directed path from $u$ to $v$ and a directed path from $v$ to $u$. In general, a
  digraph’s vertices can be partitioned into disjoint maximal subsets of vertices that are
  mutually accessible via directed paths; these subsets are called strongly connected
  components of the digraph. There are two DFS-based algorithms for identifying strongly
  connected components. Here is the simpler (but somewhat less efficient) one of the two:
  \begin{description}
    \item[Step 1] Perform a DFS traversal of the digraph given and number its vertices in the
          order they become dead ends.
    \item[Step 2]  Reverse the directions of all the edges of the digraph.
    \item[Step 3] Perform a DFS traversal of the new digraph by starting (and, if necessary,
          restarting) the traversal at the highest numbered vertex among still
          unvisited vertices.
  \end{description}
  The strongly connected components are exactly the vertices of the DFS trees obtained
  during the last traversal.
  \begin{enumerate}
    \item Apply this algorithm to the following digraph to determine its strongly
          connected components.
    \item  What is the time efficiency class of this algorithm? Give separate answers for the
          adjacency matrix representation and adjacency list representation of an input digraph.
    \item How many strongly connected components does a dag have?  Justify your answer.
  \end{enumerate}
}

\sol{
  \begin{enumerate}
    \item Applying this algorithm first I start with the DFS traversal of the graph and number the
          vertices in the order they become dead ends. The order is:
          \begin{align*}
            a\, 4 \\
            b\, 3 \\
            g\, 2 \\
            f\, 1 \\
            c\, 8 \\
            d\, 5 \\
            e\, 7 \\
            h\, 6 \\
          \end{align*}
          Next I reverse the directions of the edges and perform a DFS traversal starting at the highest numbered node
          $c$, the DFS trees obtained are:
          \begin{align*}
            c \to h \to e \\
            d             \\
            a \to  f \to g \to b
          \end{align*}
          Therefore the strongly connected components are: $\{c, h, e\} $, $\{d\}$ and $\{a, f, g, b\}$.
    \item
          \begin{description}
            \item[Adjacency Martix] - The time efficiency class of this algorithm for the adjacency matrix representation
                  is $O \left( n^2 \right) $, where $n$ is the number of vertices in the graph.

                  This is beacause the
                  first step of the algorithm involves performing a DFS traversal of the graph which takes $O \left(
                    n^2\right) $ to check all the possible edges in the graph.

                  The second step involves reversing the
                  directions of all the edges which also takes $O \left( n^2 \right) $ to check and reverse all
                  existing edges.

                  The third step involves another a DFS traversal of the new graph which also takes
                  $O \left( n^2 \right) $.

                  In total this is $O \left( n^2 \right) +  O \left( n^2 \right) + O \left(
                    n^2 \right) $ which is $O \left( n^2 \right) $.
            \item[Adjacency List] - The time efficiency class of this algorithm for the adjacency list
                  representation is $O \left( n + m \right) $, where $n$ is the number of vertices in the graph and $m$ is the number of edges in the graph.

                  This is because the first step of the algorithm does a DFS traversal of the graph which in a
                  adjacency list represented graph takes $O \left( n + m \right)$ as it only check each edge once
                  using this representation.

                  The second step involves reversing the directions of all the edges which takes $O \left( m \right) $
                  time where $m$ is the number of edges in the graph as the algorithm just needs to iterate through
                  the list of edges and reverse them.

                  Finally the third step invloves another DFS traversal of the new graph which also takes $O \left( n + m \right) $ time.

                  In total this is $O \left( n + m \right) + O \left( m \right) + O \left( n + m \right) $ which is $O \left( n + m \right) $.
          \end{description}
    \item A DAG can only have exactly $n$ strongly connected components. This is because a DAG has no cycles thus
          it is impossible to have a directed path from $v$ to $u$ and another from $u$ to $v$, i.e. a cycle. Thus
          each strongly connected component is a single node and there are $n$ nodes in the graph.
  \end{enumerate}
}



\end{document}
