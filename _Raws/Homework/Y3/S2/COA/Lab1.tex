\documentclass[12pt letter]{report}
\input{../template/preamble}
\input{../template/macros}
\input{../template/letterfonts}

\usepackage{graphicx}
\usepackage{parskip}
\usepackage{booktabs}
\usepackage{karnaugh-map}
\title{\Huge{Lab 1}}
\author{\huge{Madiba Hudson-Quansah}}
\graphicspath{ {assets/} }
\date{}



\begin{document}
\maketitle
\newpage

\qs{}{
  \begin{enumerate}
    \item Complete the truth table for R and C as a function of $Q_1$, $Q_2$, $Q_3$, and $X$
    \item Complete the state table based on the state transition diagram as above.
  \end{enumerate}
}

\sol{
  \begin{enumerate}
  \item
  \begin{table}[h!]
    \begin{center}
      \begin{tabular}{|c c c c |c|c|c|c|}
        \hline
        $Q_1$ & $Q_2$ & $Q_3$ & $X$ & Total Amount & Temp & $R$ & $C$ \\ [0.5ex]
        \hline
        \hline
        0     & 0     & 0     & 0   & \$0.00       & 00   & 0   & 0   \\
        0     & 0     & 0     & 1   & \$0.00       & 00   & 0   & 0   \\
        0     & 0     & 1     & 0   & \$0.25       & 01   & 0   & 0   \\
        0     & 0     & 1     & 1   & \$0.25       & 01   & 0   & 0   \\
        0     & 1     & 0     & 0   & \$0.25       & 01   & 0   & 0   \\
        0     & 1     & 0     & 1   & \$0.25       & 01   & 0   & 0   \\
        0     & 1     & 1     & 0   & \$0.50       & 10   & 0   & 0   \\
        0     & 1     & 1     & 1   & \$0.50       & 10   & 1   & 0   \\
        1     & 0     & 0     & 0   & \$0.25       & 01   & 0   & 0   \\
        1     & 0     & 0     & 1   & \$0.25       & 01   & 0   & 0   \\
        1     & 0     & 1     & 0   & \$0.50       & 10   & 0   & 0   \\
        1     & 0     & 1     & 1   & \$0.50       & 10   & 1   & 0   \\
        1     & 1     & 0     & 0   & \$0.50       & 10   & 0   & 0   \\
        1     & 1     & 0     & 1   & \$0.50       & 10   & 1   & 0   \\
        1     & 1     & 1     & 0   & \$0.75       & 11   & 1   & 1   \\
        1     & 1     & 1     & 1   & \$0.75       & 11   & 1   & 0   \\

        \hline
      \end{tabular}
    \end{center}
  \end{table}
  \item
  \begin{table}[H]
    \begin{center}
      \begin{tabular}{|c c c c c|c c c|}
        \hline
        Present State & $Q_1$ & $Q_2$ & $Q_3$ & $X$ & Next State & $R$ & $C$ \\ [0.5ex]
        \hline
        \hline
        S0            & 0     & 0     & 0     & 0   & S0         & 0   & 0   \\
        S0            & 0     & 0     & 1     & 0   & S1         & 0   & 0   \\
        S0            & 0     & 1     & 0     & 0   & S1         & 0   & 0   \\
        S0            & 1     & 0     & 0     & 0   & S1         & 0   & 0   \\
        S0            & 0     & 0     & 1     & 1   & S1         & 0   & 0   \\
        S0            & 0     & 1     & 0     & 1   & S1         & 0   & 0   \\
        S0            & 1     & 0     & 0     & 1   & S1         & 0   & 0   \\
        S1            & 0     & 0     & 1     & 0   & S2         & 0   & 0   \\
        S1            & 0     & 1     & 0     & 0   & S2         & 0   & 0   \\
        S1            & 1     & 0     & 0     & 0   & S2         & 0   & 0   \\
        S1            & 0     & 0     & 1     & 1   & S2         & 1   & 0   \\
        S1            & 0     & 1     & 0     & 1   & S2         & 1   & 0   \\
        S1            & 1     & 0     & 0     & 1   & S2         & 1   & 0   \\
        S2            & 0     & 0     & 1     & 0   & S3         & 1   & 0   \\
        S2            & 0     & 1     & 0     & 0   & S3         & 1   & 0   \\
        S2            & 1     & 0     & 0     & 0   & S3         & 1   & 0   \\
        S2            & 0     & 0     & 1     & 1   & S3         & 1   & 1   \\
        S2            & 0     & 1     & 0     & 1   & S3         & 1   & 1   \\
        S2            & 1     & 0     & 0     & 1   & S3         & 1   & 1   \\
        S3            & 1     & 0     & 0     & 0   & S3         & 1   & 0   \\
        S3            & 0     & 0     & 1     & 1   & S3         & 1   & 1   \\
        S3            & 0     & 1     & 0     & 1   & S3         & 1   & 1   \\
        S3            & 1     & 0     & 0     & 1   & S3         & 1   & 1   \\

        \hline
      \end{tabular}
    \end{center}
  \end{table}
  \item
  \begin{figure}[H]
  \centering
  \begin{karnaugh-map}[4][4][1][ $Q_3$ $X$ ][ $Q_1$ $Q_2$ ]
  \minterms{6, 7, 10, 11, 12, 13, 14, 15}
  \maxterms{0,1,2,3,4,5,8,9}
  \implicant{7}{14}
  \implicant{15}{10}
  \implicant{12}{14}
  \end{karnaugh-map}
  \caption{ $D_1$}
  \end{figure}
  \[
    D_1 = Q_1Q_2 + Q_2Q_3 + Q_1Q_3
  \]

  \begin{figure}[H]
  \centering
  \begin{karnaugh-map}[4][4][1][$Q_3X$][$Q_1Q_2$]
  \minterms{2, 3, 4, 5, 8, 9, 14, 15}
  \maxterms{0,1,6,7,10,11,12,13}
  \implicant{3}{2}
  \implicant{4}{5}
  \implicant{15}{14}
  \implicant{8}{9}
  \end{karnaugh-map}
  \caption{ $D_0$}
  \end{figure}

  \begin{align*}
    D_0 & = \overline{Q_1}\overline{Q_2} Q_3 + \overline{Q_1} Q_2 \overline{Q_3} + Q_1Q_2Q_3 +
    Q_1\overline{Q_2}\overline{Q_3}                                                            \\
        & = Q_1 \oplus Q_2 \oplus Q_3                                                          \\
  \end{align*}


  \end{enumerate}
}

\qs{}{
  Xilinx Vivado
}

\sol{
  \begin{figure}[H]
    \centering
    \includegraphics[width=0.75\textwidth]{pic2.png}
    \caption{Test Bench}
  \end{figure}

  \begin{figure}[H]
    \centering
    \includegraphics[width=0.75\textwidth]{pic1.png}
    \caption{Test Bench Behavioural simulation}
  \end{figure}
}

\qs{}{
  What assumptions did you make while designing the vending machine coin counter? Were these assumptions realistic and clearly justified?
}

\sol{
  \begin{itemize}
    \item Coin Slots are independent - Each slot is independent of the other and the machine does not check which
          specific slot the quarters were placed in only the total amount. This is a realistic assumption as this reflects
          how real vending machines work.
    \item Change is limited to 1 quarter - The machine only returns 1 quarter as change. This is an assumption to keep
          the design simple and easy to implement.
    \item Slots only accept valid quarters - Only valid U.S. quarters are accepted by the machine. This is a realistic
          assumption as the machine is designed to accept only quarters.
  \end{itemize}
}

\qs{}{
  How did you decide the outputs for R (release soda) and C (return change)? Could there be alternative conditions or approaches to achieve the same functionality?
}

\sol{
  The outputs for $R$ and $C$ were decided based on the total amount of money given to the machine and the soda chosen
  by the user. The machine releases a soda if the total amount is greater than or equal to the cost of the soda. An
  alternative implementation could be to have the machine release a soda if the total amount is greater than the cost of
  the soda by a certain amount. This would allow the machine to release a soda even if the user gives more money than
  required for the soda.
}

\qs{}{
  While creating the state table, how did you ensure all possible transitions were accounted for? Were there any ambiguous or conflicting states that required special handling?
}

\sol{
  I ensured all possible transitions were accounted for by creating a state transition diagram and then creating a state
  table based on the diagram. I made sure to include all possible states and transitions in the state table. There were no
  ambiguous or conflicting states that required special handling.
}

\qs{}{
  What were the key challenges you faced during the installation of Xilinx Vivado? Were there any system requirements or errors you needed to address?
}

\sol{
  I faced no challenges during the installation of Xilinx Vivado.
}

\qs{}{
  How would you explain the importance of hardware description languages (like VHDL) in modern digital design to someone unfamiliar with the field?
}

\sol{
  Hardware description languages serve as a way to describe the behaviour of digital circuits in a way that is easily
  understood by both humans and computers. They allow designers to create complex digital circuits without having to
  worry about the low-level details of the hardware. This makes it easier to design, test, and debug digital circuits
  and allows for faster development times.
}



\end{document}
