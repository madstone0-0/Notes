\documentclass[12pt letter]{report}
%%%%%%%%%%%%%%%%%%%%%%%%%%%%%%%%%
% PACKAGE IMPORTS
%%%%%%%%%%%%%%%%%%%%%%%%%%%%%%%%%


\usepackage[tmargin=2cm,rmargin=1in,lmargin=1in,margin=0.85in,bmargin=2cm,footskip=.2in]{geometry}
\usepackage{amsmath,amsfonts,amsthm,amssymb,mathtools}
\usepackage[varbb]{newpxmath}
\usepackage{xfrac}
\usepackage[makeroom]{cancel}
\usepackage{mathtools}
\usepackage{bookmark}
\usepackage{enumitem}
\usepackage{hyperref,theoremref}
\hypersetup{
  pdftitle={Assignment},
  colorlinks=true, linkcolor=doc!90,
  bookmarksnumbered=true,
  bookmarksopen=true
}
\usepackage[most,many,breakable]{tcolorbox}
\usepackage{xcolor}
\usepackage{varwidth}
\usepackage{varwidth}
\usepackage{etoolbox}
%\usepackage{authblk}
\usepackage{nameref}
\usepackage{multicol,array}
\usepackage{tikz-cd}
\usepackage[ruled,vlined,linesnumbered]{algorithm2e}
\usepackage{comment} % enables the use of multi-line comments (\ifx \fi) 
\usepackage{import}
\usepackage{xifthen}
\usepackage{pdfpages}
\usepackage{transparent}
\usepackage{xcolor,colortbl,array,amssymb}
\usepackage{venndiagram}
\usepackage{listings}
\usepackage{fontspec}

\setmainfont{LibertinusSerif}[
  Extension = .otf,
  Path = /usr/share/fonts/libertinus/,
  UprightFont = *-Regular,
  ItalicFont = *-Italic,
  BoldFont = *-Bold,
  BoldItalicFont = *-BoldItalic,
]

\setmonofont{CaskaydiaCoveNerdFontMono}[
  Extension = .ttf,
  Path = /usr/share/fonts/TTF/,
  UprightFont = *-Regular,
  ItalicFont = *-Italic,
  BoldFont = *-Bold,
  BoldItalicFont = *-BoldItalic,
]

\definecolor{listing-background}{HTML}{F7F7F7}
\definecolor{listing-rule}{HTML}{B3B2B3}
\definecolor{listing-numbers}{HTML}{B3B2B3}
\definecolor{listing-text-color}{HTML}{000000}
\definecolor{listing-keyword}{HTML}{435489}
\definecolor{listing-keyword-2}{HTML}{1284CA} % additional keywords
\definecolor{listing-keyword-3}{HTML}{9137CB} % additional keywords
\definecolor{listing-identifier}{HTML}{435489}
\definecolor{listing-string}{HTML}{00999A}
\definecolor{listing-comment}{HTML}{8E8E8E}

\lstdefinestyle{eisvogel_listing_style}{
language         = java,
xleftmargin      = 0.6em,
framexleftmargin = 0.4em,
backgroundcolor  = \color{listing-background},
basicstyle       = \color{listing-text-color}\linespread{1.0}%
\lst@ifdisplaystyle%
\fi\ttfamily{},
breaklines       = true,
frame            = single,
framesep         = 0.19em,
rulecolor        = \color{listing-rule},
frameround       = ffff,
tabsize          = 4,
numberstyle      = \color{listing-numbers},
aboveskip        = 1.0em,
belowskip        = 0.1em,
abovecaptionskip = 0em,
belowcaptionskip = 1.0em,
keywordstyle     = {\color{listing-keyword}\bfseries},
keywordstyle     = {[2]\color{listing-keyword-2}\bfseries},
keywordstyle     = {[3]\color{listing-keyword-3}\bfseries\itshape},
sensitive        = true,
identifierstyle  = \color{listing-identifier},
commentstyle     = \color{listing-comment},
stringstyle      = \color{listing-string},
showstringspaces = false,
escapeinside     = {/*@}{@*/}, % Allow LaTeX inside these special comments
literate         =
  {á}{{\'a}}1 {é}{{\'e}}1 {í}{{\'i}}1 {ó}{{\'o}}1 {ú}{{\'u}}1
{Á}{{\'A}}1 {É}{{\'E}}1 {Í}{{\'I}}1 {Ó}{{\'O}}1 {Ú}{{\'U}}1
{à}{{\`a}}1 {è}{{\`e}}1 {ì}{{\`i}}1 {ò}{{\`o}}1 {ù}{{\`u}}1
{À}{{\`A}}1 {È}{{\`E}}1 {Ì}{{\`I}}1 {Ò}{{\`O}}1 {Ù}{{\`U}}1
{ä}{{\"a}}1 {ë}{{\"e}}1 {ï}{{\"i}}1 {ö}{{\"o}}1 {ü}{{\"u}}1
{Ä}{{\"A}}1 {Ë}{{\"E}}1 {Ï}{{\"I}}1 {Ö}{{\"O}}1 {Ü}{{\"U}}1
{â}{{\^a}}1 {ê}{{\^e}}1 {î}{{\^i}}1 {ô}{{\^o}}1 {û}{{\^u}}1
{Â}{{\^A}}1 {Ê}{{\^E}}1 {Î}{{\^I}}1 {Ô}{{\^O}}1 {Û}{{\^U}}1
{œ}{{\oe}}1 {Œ}{{\OE}}1 {æ}{{\ae}}1 {Æ}{{\AE}}1 {ß}{{\ss}}1
{ç}{{\c c}}1 {Ç}{{\c C}}1 {ø}{{\o}}1 {å}{{\r a}}1 {Å}{{\r A}}1
{€}{{\EUR}}1 {£}{{\pounds}}1 {«}{{\guillemotleft}}1
{»}{{\guillemotright}}1 {ñ}{{\~n}}1 {Ñ}{{\~N}}1 {¿}{{?`}}1
{…}{{\ldots}}1 {≥}{{>=}}1 {≤}{{<=}}1 {„}{{\glqq}}1 {“}{{\grqq}}1
{”}{{''}}1
}
\lstset{style=eisvogel_listing_style,
  numbers=left}

%
% Java (Java SE 12, 2019-06-22)
%
\lstdefinelanguage{Java}{
  morekeywords={
      % normal keywords (without data types)
      abstract,assert,break,case,catch,class,continue,default,
      do,else,enum,exports,extends,final,finally,for,if,implements,
      import,instanceof,interface,module,native,new,package,private,
      protected,public,requires,return,static,strictfp,super,switch,
      synchronized,this,throw,throws,transient,try,volatile,while,
      % var is an identifier
      var
    },
  morekeywords={[2] % data types
      % primitive data types
      boolean,byte,char,double,float,int,long,short,
      % String
      String,
      % primitive wrapper types
      Boolean,Byte,Character,Double,Float,Integer,Long,Short
      % number types
      Number,AtomicInteger,AtomicLong,BigDecimal,BigInteger,DoubleAccumulator,DoubleAdder,LongAccumulator,LongAdder,Short,
      % other
      Object,Void,void
    },
  morekeywords={[3] % literals
      % reserved words for literal values
      null,true,false,
    },
  sensitive,
  morecomment  = [l]//,
  morecomment  = [s]{/*}{*/},
  morecomment  = [s]{/**}{*/},
  morestring   = [b]",
  morestring   = [b]',
}

\lstdefinelanguage{XML}{
  morestring      = [b]",
  moredelim       = [s][\bfseries\color{listing-keyword}]{<}{\ },
  moredelim       = [s][\bfseries\color{listing-keyword}]{</}{>},
  moredelim       = [l][\bfseries\color{listing-keyword}]{/>},
  moredelim       = [l][\bfseries\color{listing-keyword}]{>},
  morecomment     = [s]{<?}{?>},
  morecomment     = [s]{<!--}{-->},
  commentstyle    = \color{listing-comment},
  stringstyle     = \color{listing-string},
  identifierstyle = \color{listing-identifier}
}


\newcommand\mycommfont[1]{\footnotesize\ttfamily\textcolor{blue}{#1}}
\SetCommentSty{mycommfont}
\newcommand{\incfig}[1]{%
  \def\svgwidth{\columnwidth}
  \import{./figures/}{#1.pdf_tex}
}

\usepackage{tikzsymbols}
\renewcommand\qedsymbol{$\Laughey$}


%\usepackage{import}
%\usepackage{xifthen}
%\usepackage{pdfpages}
%\usepackage{transparent}


%%%%%%%%%%%%%%%%%%%%%%%%%%%%%%
% SELF MADE COLORS
%%%%%%%%%%%%%%%%%%%%%%%%%%%%%%



\definecolor{myg}{RGB}{56, 140, 70}
\definecolor{myb}{RGB}{45, 111, 177}
\definecolor{myr}{RGB}{199, 68, 64}
\definecolor{mytheorembg}{HTML}{F2F2F9}
\definecolor{mytheoremfr}{HTML}{00007B}
\definecolor{mylenmabg}{HTML}{FFFAF8}
\definecolor{mylenmafr}{HTML}{983b0f}
\definecolor{mypropbg}{HTML}{f2fbfc}
\definecolor{mypropfr}{HTML}{191971}
\definecolor{myexamplebg}{HTML}{F2FBF8}
\definecolor{myexamplefr}{HTML}{88D6D1}
\definecolor{myexampleti}{HTML}{2A7F7F}
\definecolor{mydefinitbg}{HTML}{E5E5FF}
\definecolor{mydefinitfr}{HTML}{3F3FA3}
\definecolor{notesgreen}{RGB}{0,162,0}
\definecolor{myp}{RGB}{197, 92, 212}
\definecolor{mygr}{HTML}{2C3338}
\definecolor{myred}{RGB}{127,0,0}
\definecolor{myyellow}{RGB}{169,121,69}
\definecolor{myexercisebg}{HTML}{F2FBF8}
\definecolor{myexercisefg}{HTML}{88D6D1}


%%%%%%%%%%%%%%%%%%%%%%%%%%%%
% TCOLORBOX SETUPS
%%%%%%%%%%%%%%%%%%%%%%%%%%%%

\setlength{\parindent}{1cm}
%================================
% THEOREM BOX
%================================

\tcbuselibrary{theorems,skins,hooks}
\newtcbtheorem[number within=section]{Theorem}{Theorem}
{%
  enhanced,
  breakable,
  colback = mytheorembg,
  frame hidden,
  boxrule = 0sp,
  borderline west = {2pt}{0pt}{mytheoremfr},
  sharp corners,
  detach title,
  before upper = \tcbtitle\par\smallskip,
  coltitle = mytheoremfr,
  fonttitle = \bfseries\sffamily,
  description font = \mdseries,
  separator sign none,
  segmentation style={solid, mytheoremfr},
}
{th}

\tcbuselibrary{theorems,skins,hooks}
\newtcbtheorem[number within=chapter]{theorem}{Theorem}
{%
  enhanced,
  breakable,
  colback = mytheorembg,
  frame hidden,
  boxrule = 0sp,
  borderline west = {2pt}{0pt}{mytheoremfr},
  sharp corners,
  detach title,
  before upper = \tcbtitle\par\smallskip,
  coltitle = mytheoremfr,
  fonttitle = \bfseries\sffamily,
  description font = \mdseries,
  separator sign none,
  segmentation style={solid, mytheoremfr},
}
{th}


\tcbuselibrary{theorems,skins,hooks}
\newtcolorbox{Theoremcon}
{%
  enhanced
  ,breakable
  ,colback = mytheorembg
  ,frame hidden
  ,boxrule = 0sp
  ,borderline west = {2pt}{0pt}{mytheoremfr}
  ,sharp corners
  ,description font = \mdseries
  ,separator sign none
}

%================================
% Corollery
%================================
\tcbuselibrary{theorems,skins,hooks}
\newtcbtheorem[number within=section]{Corollary}{Corollary}
{%
  enhanced
  ,breakable
  ,colback = myp!10
  ,frame hidden
  ,boxrule = 0sp
  ,borderline west = {2pt}{0pt}{myp!85!black}
  ,sharp corners
  ,detach title
  ,before upper = \tcbtitle\par\smallskip
  ,coltitle = myp!85!black
  ,fonttitle = \bfseries\sffamily
  ,description font = \mdseries
  ,separator sign none
  ,segmentation style={solid, myp!85!black}
}
{th}
\tcbuselibrary{theorems,skins,hooks}
\newtcbtheorem[number within=chapter]{corollary}{Corollary}
{%
  enhanced
  ,breakable
  ,colback = myp!10
  ,frame hidden
  ,boxrule = 0sp
  ,borderline west = {2pt}{0pt}{myp!85!black}
  ,sharp corners
  ,detach title
  ,before upper = \tcbtitle\par\smallskip
  ,coltitle = myp!85!black
  ,fonttitle = \bfseries\sffamily
  ,description font = \mdseries
  ,separator sign none
  ,segmentation style={solid, myp!85!black}
}
{th}


%================================
% LENMA
%================================

\tcbuselibrary{theorems,skins,hooks}
\newtcbtheorem[number within=section]{Lenma}{Lenma}
{%
  enhanced,
  breakable,
  colback = mylenmabg,
  frame hidden,
  boxrule = 0sp,
  borderline west = {2pt}{0pt}{mylenmafr},
  sharp corners,
  detach title,
  before upper = \tcbtitle\par\smallskip,
  coltitle = mylenmafr,
  fonttitle = \bfseries\sffamily,
  description font = \mdseries,
  separator sign none,
  segmentation style={solid, mylenmafr},
}
{th}

\tcbuselibrary{theorems,skins,hooks}
\newtcbtheorem[number within=chapter]{lenma}{Lenma}
{%
  enhanced,
  breakable,
  colback = mylenmabg,
  frame hidden,
  boxrule = 0sp,
  borderline west = {2pt}{0pt}{mylenmafr},
  sharp corners,
  detach title,
  before upper = \tcbtitle\par\smallskip,
  coltitle = mylenmafr,
  fonttitle = \bfseries\sffamily,
  description font = \mdseries,
  separator sign none,
  segmentation style={solid, mylenmafr},
}
{th}


%================================
% PROPOSITION
%================================

\tcbuselibrary{theorems,skins,hooks}
\newtcbtheorem[number within=section]{Prop}{Proposition}
{%
  enhanced,
  breakable,
  colback = mypropbg,
  frame hidden,
  boxrule = 0sp,
  borderline west = {2pt}{0pt}{mypropfr},
  sharp corners,
  detach title,
  before upper = \tcbtitle\par\smallskip,
  coltitle = mypropfr,
  fonttitle = \bfseries\sffamily,
  description font = \mdseries,
  separator sign none,
  segmentation style={solid, mypropfr},
}
{th}

\tcbuselibrary{theorems,skins,hooks}
\newtcbtheorem[number within=chapter]{prop}{Proposition}
{%
  enhanced,
  breakable,
  colback = mypropbg,
  frame hidden,
  boxrule = 0sp,
  borderline west = {2pt}{0pt}{mypropfr},
  sharp corners,
  detach title,
  before upper = \tcbtitle\par\smallskip,
  coltitle = mypropfr,
  fonttitle = \bfseries\sffamily,
  description font = \mdseries,
  separator sign none,
  segmentation style={solid, mypropfr},
}
{th}


%================================
% CLAIM
%================================

\tcbuselibrary{theorems,skins,hooks}
\newtcbtheorem[number within=section]{claim}{Claim}
{%
  enhanced
  ,breakable
  ,colback = myg!10
  ,frame hidden
  ,boxrule = 0sp
  ,borderline west = {2pt}{0pt}{myg}
  ,sharp corners
  ,detach title
  ,before upper = \tcbtitle\par\smallskip
  ,coltitle = myg!85!black
  ,fonttitle = \bfseries\sffamily
  ,description font = \mdseries
  ,separator sign none
  ,segmentation style={solid, myg!85!black}
}
{th}



%================================
% Exercise
%================================

\tcbuselibrary{theorems,skins,hooks}
\newtcbtheorem[number within=section]{Exercise}{Exercise}
{%
  enhanced,
  breakable,
  colback = myexercisebg,
  frame hidden,
  boxrule = 0sp,
  borderline west = {2pt}{0pt}{myexercisefg},
  sharp corners,
  detach title,
  before upper = \tcbtitle\par\smallskip,
  coltitle = myexercisefg,
  fonttitle = \bfseries\sffamily,
  description font = \mdseries,
  separator sign none,
  segmentation style={solid, myexercisefg},
}
{th}

\tcbuselibrary{theorems,skins,hooks}
\newtcbtheorem[number within=chapter]{exercise}{Exercise}
{%
  enhanced,
  breakable,
  colback = myexercisebg,
  frame hidden,
  boxrule = 0sp,
  borderline west = {2pt}{0pt}{myexercisefg},
  sharp corners,
  detach title,
  before upper = \tcbtitle\par\smallskip,
  coltitle = myexercisefg,
  fonttitle = \bfseries\sffamily,
  description font = \mdseries,
  separator sign none,
  segmentation style={solid, myexercisefg},
}
{th}

%================================
% EXAMPLE BOX
%================================

\newtcbtheorem[number within=section]{Example}{Example}
{%
  colback = myexamplebg
  ,breakable
  ,colframe = myexamplefr
  ,coltitle = myexampleti
  ,boxrule = 1pt
  ,sharp corners
  ,detach title
  ,before upper=\tcbtitle\par\smallskip
  ,fonttitle = \bfseries
  ,description font = \mdseries
  ,separator sign none
  ,description delimiters parenthesis
}
{ex}

\newtcbtheorem[number within=chapter]{example}{Example}
{%
  colback = myexamplebg
  ,breakable
  ,colframe = myexamplefr
  ,coltitle = myexampleti
  ,boxrule = 1pt
  ,sharp corners
  ,detach title
  ,before upper=\tcbtitle\par\smallskip
  ,fonttitle = \bfseries
  ,description font = \mdseries
  ,separator sign none
  ,description delimiters parenthesis
}
{ex}

%================================
% DEFINITION BOX
%================================

\newtcbtheorem[number within=section]{Definition}{Definition}{enhanced,
  before skip=2mm,after skip=2mm, colback=red!5,colframe=red!80!black,boxrule=0.5mm,
  attach boxed title to top left={xshift=1cm,yshift*=1mm-\tcboxedtitleheight}, varwidth boxed title*=-3cm,
  boxed title style={frame code={
          \path[fill=tcbcolback]
          ([yshift=-1mm,xshift=-1mm]frame.north west)
          arc[start angle=0,end angle=180,radius=1mm]
          ([yshift=-1mm,xshift=1mm]frame.north east)
          arc[start angle=180,end angle=0,radius=1mm];
          \path[left color=tcbcolback!60!black,right color=tcbcolback!60!black,
            middle color=tcbcolback!80!black]
          ([xshift=-2mm]frame.north west) -- ([xshift=2mm]frame.north east)
          [rounded corners=1mm]-- ([xshift=1mm,yshift=-1mm]frame.north east)
          -- (frame.south east) -- (frame.south west)
          -- ([xshift=-1mm,yshift=-1mm]frame.north west)
          [sharp corners]-- cycle;
        },interior engine=empty,
    },
  fonttitle=\bfseries,
  title={#2},#1}{def}
\newtcbtheorem[number within=chapter]{definition}{Definition}{enhanced,
  before skip=2mm,after skip=2mm, colback=red!5,colframe=red!80!black,boxrule=0.5mm,
  attach boxed title to top left={xshift=1cm,yshift*=1mm-\tcboxedtitleheight}, varwidth boxed title*=-3cm,
  boxed title style={frame code={
          \path[fill=tcbcolback]
          ([yshift=-1mm,xshift=-1mm]frame.north west)
          arc[start angle=0,end angle=180,radius=1mm]
          ([yshift=-1mm,xshift=1mm]frame.north east)
          arc[start angle=180,end angle=0,radius=1mm];
          \path[left color=tcbcolback!60!black,right color=tcbcolback!60!black,
            middle color=tcbcolback!80!black]
          ([xshift=-2mm]frame.north west) -- ([xshift=2mm]frame.north east)
          [rounded corners=1mm]-- ([xshift=1mm,yshift=-1mm]frame.north east)
          -- (frame.south east) -- (frame.south west)
          -- ([xshift=-1mm,yshift=-1mm]frame.north west)
          [sharp corners]-- cycle;
        },interior engine=empty,
    },
  fonttitle=\bfseries,
  title={#2},#1}{def}



%================================
% Solution BOX
%================================

\makeatletter
\newtcbtheorem{question}{Question}{enhanced,
  breakable,
  colback=white,
  colframe=myb!80!black,
  attach boxed title to top left={yshift*=-\tcboxedtitleheight},
  fonttitle=\bfseries,
  title={#2},
  boxed title size=title,
  boxed title style={%
      sharp corners,
      rounded corners=northwest,
      colback=tcbcolframe,
      boxrule=0pt,
    },
  underlay boxed title={%
      \path[fill=tcbcolframe] (title.south west)--(title.south east)
      to[out=0, in=180] ([xshift=5mm]title.east)--
      (title.center-|frame.east)
      [rounded corners=\kvtcb@arc] |-
      (frame.north) -| cycle;
    },
  #1
}{def}
\makeatother

%================================
% SOLUTION BOX
%================================

\makeatletter
\newtcolorbox{solution}{enhanced,
  breakable,
  colback=white,
  colframe=myg!80!black,
  attach boxed title to top left={yshift*=-\tcboxedtitleheight},
  title=Solution,
  boxed title size=title,
  boxed title style={%
      sharp corners,
      rounded corners=northwest,
      colback=tcbcolframe,
      boxrule=0pt,
    },
  underlay boxed title={%
      \path[fill=tcbcolframe] (title.south west)--(title.south east)
      to[out=0, in=180] ([xshift=5mm]title.east)--
      (title.center-|frame.east)
      [rounded corners=\kvtcb@arc] |-
      (frame.north) -| cycle;
    },
}
\makeatother

%================================
% Question BOX
%================================

\makeatletter
\newtcbtheorem{qstion}{Question}{enhanced,
  breakable,
  colback=white,
  colframe=mygr,
  attach boxed title to top left={yshift*=-\tcboxedtitleheight},
  fonttitle=\bfseries,
  title={#2},
  boxed title size=title,
  boxed title style={%
      sharp corners,
      rounded corners=northwest,
      colback=tcbcolframe,
      boxrule=0pt,
    },
  underlay boxed title={%
      \path[fill=tcbcolframe] (title.south west)--(title.south east)
      to[out=0, in=180] ([xshift=5mm]title.east)--
      (title.center-|frame.east)
      [rounded corners=\kvtcb@arc] |-
      (frame.north) -| cycle;
    },
  #1
}{def}
\makeatother

\newtcbtheorem[number within=chapter]{wconc}{Wrong Concept}{
  breakable,
  enhanced,
  colback=white,
  colframe=myr,
  arc=0pt,
  outer arc=0pt,
  fonttitle=\bfseries\sffamily\large,
  colbacktitle=myr,
  attach boxed title to top left={},
  boxed title style={
      enhanced,
      skin=enhancedfirst jigsaw,
      arc=3pt,
      bottom=0pt,
      interior style={fill=myr}
    },
  #1
}{def}



%================================
% NOTE BOX
%================================

\usetikzlibrary{arrows,calc,shadows.blur}
\tcbuselibrary{skins}
\newtcolorbox{note}[1][]{%
  enhanced jigsaw,
  colback=gray!20!white,%
  colframe=gray!80!black,
  size=small,
  boxrule=1pt,
  title=\textbf{Note:-},
  halign title=flush center,
  coltitle=black,
  breakable,
  drop shadow=black!50!white,
  attach boxed title to top left={xshift=1cm,yshift=-\tcboxedtitleheight/2,yshifttext=-\tcboxedtitleheight/2},
  minipage boxed title=1.5cm,
  boxed title style={%
      colback=white,
      size=fbox,
      boxrule=1pt,
      boxsep=2pt,
      underlay={%
          \coordinate (dotA) at ($(interior.west) + (-0.5pt,0)$);
          \coordinate (dotB) at ($(interior.east) + (0.5pt,0)$);
          \begin{scope}
            \clip (interior.north west) rectangle ([xshift=3ex]interior.east);
            \filldraw [white, blur shadow={shadow opacity=60, shadow yshift=-.75ex}, rounded corners=2pt] (interior.north west) rectangle (interior.south east);
          \end{scope}
          \begin{scope}[gray!80!black]
            \fill (dotA) circle (2pt);
            \fill (dotB) circle (2pt);
          \end{scope}
        },
    },
  #1,
}

%%%%%%%%%%%%%%%%%%%%%%%%%%%%%%
% SELF MADE COMMANDS
%%%%%%%%%%%%%%%%%%%%%%%%%%%%%%


\newcommand{\thm}[2]{\begin{Theorem}{#1}{}#2\end{Theorem}}
\newcommand{\cor}[2]{\begin{Corollary}{#1}{}#2\end{Corollary}}
\newcommand{\mlenma}[2]{\begin{Lenma}{#1}{}#2\end{Lenma}}
\newcommand{\mprop}[2]{\begin{Prop}{#1}{}#2\end{Prop}}
\newcommand{\clm}[3]{\begin{claim}{#1}{#2}#3\end{claim}}
\newcommand{\wc}[2]{\begin{wconc}{#1}{}\setlength{\parindent}{1cm}#2\end{wconc}}
\newcommand{\thmcon}[1]{\begin{Theoremcon}{#1}\end{Theoremcon}}
\newcommand{\ex}[2]{\begin{Example}{#1}{}#2\end{Example}}
\newcommand{\dfn}[2]{\begin{Definition}[colbacktitle=red!75!black]{#1}{}#2\end{Definition}}
\newcommand{\dfnc}[2]{\begin{definition}[colbacktitle=red!75!black]{#1}{}#2\end{definition}}
\newcommand{\qs}[2]{\begin{question}{#1}{}#2\end{question}}
\newcommand{\pf}[2]{\begin{myproof}[#1]#2\end{myproof}}
\newcommand{\nt}[1]{\begin{note}#1\end{note}}

\newcommand*\circled[1]{\tikz[baseline=(char.base)]{
    \node[shape=circle,draw,inner sep=1pt] (char) {#1};}}
\newcommand\getcurrentref[1]{%
  \ifnumequal{\value{#1}}{0}
  {??}
  {\the\value{#1}}%
}
\newcommand{\getCurrentSectionNumber}{\getcurrentref{section}}
\newenvironment{myproof}[1][\proofname]{%
  \proof[\bfseries #1: ]%
}{\endproof}

\newcommand{\mclm}[2]{\begin{myclaim}[#1]#2\end{myclaim}}
\newenvironment{myclaim}[1][\claimname]{\proof[\bfseries #1: ]}{}

\newcounter{mylabelcounter}

\makeatletter
\newcommand{\setword}[2]{%
  \phantomsection
  #1\def\@currentlabel{\unexpanded{#1}}\label{#2}%
}
\makeatother




\tikzset{
  symbol/.style={
      draw=none,
      every to/.append style={
          edge node={node [sloped, allow upside down, auto=false]{$#1$}}}
    }
}


% deliminators
\DeclarePairedDelimiter{\abs}{\lvert}{\rvert}
\DeclarePairedDelimiter{\norm}{\lVert}{\rVert}

\DeclarePairedDelimiter{\ceil}{\lceil}{\rceil}
\DeclarePairedDelimiter{\floor}{\lfloor}{\rfloor}
\DeclarePairedDelimiter{\round}{\lfloor}{\rceil}

\newsavebox\diffdbox
\newcommand{\slantedromand}{{\mathpalette\makesl{d}}}
\newcommand{\makesl}[2]{%
  \begingroup
  \sbox{\diffdbox}{$\mathsurround=0pt#1\mathrm{#2}$}%
  \pdfsave
  \pdfsetmatrix{1 0 0.2 1}%
  \rlap{\usebox{\diffdbox}}%
  \pdfrestore
  \hskip\wd\diffdbox
  \endgroup
}
\newcommand{\dd}[1][]{\ensuremath{\mathop{}\!\ifstrempty{#1}{%
      \slantedromand\@ifnextchar^{\hspace{0.2ex}}{\hspace{0.1ex}}}%
    {\slantedromand\hspace{0.2ex}^{#1}}}}
\ProvideDocumentCommand\dv{o m g}{%
  \ensuremath{%
    \IfValueTF{#3}{%
      \IfNoValueTF{#1}{%
        \frac{\dd #2}{\dd #3}%
      }{%
        \frac{\dd^{#1} #2}{\dd #3^{#1}}%
      }%
    }{%
      \IfNoValueTF{#1}{%
        \frac{\dd}{\dd #2}%
      }{%
        \frac{\dd^{#1}}{\dd #2^{#1}}%
      }%
    }%
  }%
}
\providecommand*{\pdv}[3][]{\frac{\partial^{#1}#2}{\partial#3^{#1}}}
%  - others
\DeclareMathOperator{\Lap}{\mathcal{L}}
\DeclareMathOperator{\Var}{Var} % varience
\DeclareMathOperator{\Cov}{Cov} % covarience
\DeclareMathOperator{\E}{E} % expected

% Since the amsthm package isn't loaded

% I prefer the slanted \leq
\let\oldleq\leq % save them in case they're every wanted
\let\oldgeq\geq
\renewcommand{\leq}{\leqslant}
\renewcommand{\geq}{\geqslant}

% % redefine matrix env to allow for alignment, use r as default
% \renewcommand*\env@matrix[1][r]{\hskip -\arraycolsep
%     \let\@ifnextchar\new@ifnextchar
%     \array{*\c@MaxMatrixCols #1}}


%\usepackage{framed}
%\usepackage{titletoc}
%\usepackage{etoolbox}
%\usepackage{lmodern}


%\patchcmd{\tableofcontents}{\contentsname}{\sffamily\contentsname}{}{}

%\renewenvironment{leftbar}
%{\def\FrameCommand{\hspace{6em}%
%		{\color{myyellow}\vrule width 2pt depth 6pt}\hspace{1em}}%
%	\MakeFramed{\parshape 1 0cm \dimexpr\textwidth-6em\relax\FrameRestore}\vskip2pt%
%}
%{\endMakeFramed}

%\titlecontents{chapter}
%[0em]{\vspace*{2\baselineskip}}
%{\parbox{4.5em}{%
%		\hfill\Huge\sffamily\bfseries\color{myred}\thecontentspage}%
%	\vspace*{-2.3\baselineskip}\leftbar\textsc{\small\chaptername~\thecontentslabel}\\\sffamily}
%{}{\endleftbar}
%\titlecontents{section}
%[8.4em]
%{\sffamily\contentslabel{3em}}{}{}
%{\hspace{0.5em}\nobreak\itshape\color{myred}\contentspage}
%\titlecontents{subsection}
%[8.4em]
%{\sffamily\contentslabel{3em}}{}{}  
%{\hspace{0.5em}\nobreak\itshape\color{myred}\contentspage}



%%%%%%%%%%%%%%%%%%%%%%%%%%%%%%%%%%%%%%%%%%%
% TABLE OF CONTENTS
%%%%%%%%%%%%%%%%%%%%%%%%%%%%%%%%%%%%%%%%%%%

\usepackage{tikz}
\definecolor{doc}{RGB}{0,60,110}
\usepackage{titletoc}
\contentsmargin{0cm}
\titlecontents{chapter}[3.7pc]
{\addvspace{30pt}%
  \begin{tikzpicture}[remember picture, overlay]%
    \draw[fill=doc!60,draw=doc!60] (-7,-.1) rectangle (-0.9,.5);%
    \pgftext[left,x=-3.5cm,y=0.2cm]{\color{white}\Large\sc\bfseries Chapter\ \thecontentslabel};%
  \end{tikzpicture}\color{doc!60}\large\sc\bfseries}%
{}
{}
{\;\titlerule\;\large\sc\bfseries Page \thecontentspage
  \begin{tikzpicture}[remember picture, overlay]
    \draw[fill=doc!60,draw=doc!60] (2pt,0) rectangle (4,0.1pt);
  \end{tikzpicture}}%
\titlecontents{section}[3.7pc]
{\addvspace{2pt}}
{\contentslabel[\thecontentslabel]{2pc}}
{}
{\hfill\small \thecontentspage}
[]
\titlecontents*{subsection}[3.7pc]
{\addvspace{-1pt}\small}
{}
{}
{\ --- \small\thecontentspage}
[ \textbullet\ ][]

\makeatletter
\renewcommand{\tableofcontents}{%
  \chapter*{%
    \vspace*{-20\p@}%
    \begin{tikzpicture}[remember picture, overlay]%
      \pgftext[right,x=15cm,y=0.2cm]{\color{doc!60}\Huge\sc\bfseries \contentsname};%
      \draw[fill=doc!60,draw=doc!60] (13,-.75) rectangle (20,1);%
      \clip (13,-.75) rectangle (20,1);
      \pgftext[right,x=15cm,y=0.2cm]{\color{white}\Huge\sc\bfseries \contentsname};%
    \end{tikzpicture}}%
  \@starttoc{toc}}
\makeatother

%From M275 "Topology" at SJSU
\newcommand{\id}{\mathrm{id}}
\newcommand{\taking}[1]{\xrightarrow{#1}}
\newcommand{\inv}{^{-1}}

%From M170 "Introduction to Graph Theory" at SJSU
\DeclareMathOperator{\diam}{diam}
\DeclareMathOperator{\ord}{ord}
\newcommand{\defeq}{\overset{\mathrm{def}}{=}}

%From the USAMO .tex files
\newcommand{\ts}{\textsuperscript}
\newcommand{\dg}{^\circ}
\newcommand{\ii}{\item}

% % From Math 55 and Math 145 at Harvard
% \newenvironment{subproof}[1][Proof]{%
% \begin{proof}[#1] \renewcommand{\qedsymbol}{$\blacksquare$}}%
% {\end{proof}}

\newcommand{\liff}{\leftrightarrow}
\newcommand{\lthen}{\rightarrow}
\newcommand{\opname}{\operatorname}
\newcommand{\surjto}{\twoheadrightarrow}
\newcommand{\injto}{\hookrightarrow}
\newcommand{\On}{\mathrm{On}} % ordinals
\DeclareMathOperator{\img}{im} % Image
\DeclareMathOperator{\Img}{Im} % Image
\DeclareMathOperator{\coker}{coker} % Cokernel
\DeclareMathOperator{\Coker}{Coker} % Cokernel
\DeclareMathOperator{\Ker}{Ker} % Kernel
\DeclareMathOperator{\rank}{rank}
\DeclareMathOperator{\Spec}{Spec} % spectrum
\DeclareMathOperator{\Tr}{Tr} % trace
\DeclareMathOperator{\pr}{pr} % projection
\DeclareMathOperator{\ext}{ext} % extension
\DeclareMathOperator{\pred}{pred} % predecessor
\DeclareMathOperator{\dom}{dom} % domain
\DeclareMathOperator{\ran}{ran} % range
\DeclareMathOperator{\Hom}{Hom} % homomorphism
\DeclareMathOperator{\Mor}{Mor} % morphisms
\DeclareMathOperator{\End}{End} % endomorphism

\newcommand{\eps}{\epsilon}
\newcommand{\veps}{\varepsilon}
\newcommand{\ol}{\overline}
\newcommand{\ul}{\underline}
\newcommand{\wt}{\widetilde}
\newcommand{\wh}{\widehat}
\newcommand{\vocab}[1]{\textbf{\color{blue} #1}}
\providecommand{\half}{\frac{1}{2}}
\newcommand{\dang}{\measuredangle} %% Directed angle
\newcommand{\ray}[1]{\overrightarrow{#1}}
\newcommand{\seg}[1]{\overline{#1}}
\newcommand{\arc}[1]{\wideparen{#1}}
\DeclareMathOperator{\cis}{cis}
\DeclareMathOperator*{\lcm}{lcm}
\DeclareMathOperator*{\argmin}{arg min}
\DeclareMathOperator*{\argmax}{arg max}
\newcommand{\cycsum}{\sum_{\mathrm{cyc}}}
\newcommand{\symsum}{\sum_{\mathrm{sym}}}
\newcommand{\cycprod}{\prod_{\mathrm{cyc}}}
\newcommand{\symprod}{\prod_{\mathrm{sym}}}
\newcommand{\Qed}{\begin{flushright}\qed\end{flushright}}
\newcommand{\parinn}{\setlength{\parindent}{1cm}}
\newcommand{\parinf}{\setlength{\parindent}{0cm}}
% \newcommand{\norm}{\|\cdot\|}
\newcommand{\inorm}{\norm_{\infty}}
\newcommand{\opensets}{\{V_{\alpha}\}_{\alpha\in I}}
\newcommand{\oset}{V_{\alpha}}
\newcommand{\opset}[1]{V_{\alpha_{#1}}}
\newcommand{\lub}{\text{lub}}
\newcommand{\del}[2]{\frac{\partial #1}{\partial #2}}
\newcommand{\Del}[3]{\frac{\partial^{#1} #2}{\partial^{#1} #3}}
\newcommand{\deld}[2]{\dfrac{\partial #1}{\partial #2}}
\newcommand{\Deld}[3]{\dfrac{\partial^{#1} #2}{\partial^{#1} #3}}
\newcommand{\lm}{\lambda}
\newcommand{\uin}{\mathbin{\rotatebox[origin=c]{90}{$\in$}}}
\newcommand{\usubset}{\mathbin{\rotatebox[origin=c]{90}{$\subset$}}}
\newcommand{\lt}{\left}
\newcommand{\rt}{\right}
\newcommand{\bs}[1]{\boldsymbol{#1}}
\newcommand{\exs}{\exists}
\newcommand{\st}{\strut}
\newcommand{\dps}[1]{\displaystyle{#1}}

\newcommand{\sol}{\setlength{\parindent}{0cm}\textbf{\textit{Solution:}}\setlength{\parindent}{1cm} }
\newcommand{\solve}[1]{\setlength{\parindent}{0cm}\textbf{\textit{Solution: }}\setlength{\parindent}{1cm}#1 \Qed}

\preto\tabular{\setcounter{magicrownumbers}{0}}
\newcounter{magicrownumbers}
\newcommand\rownumber{\stepcounter{magicrownumbers}\arabic{magicrownumbers}}
\def\rownumber{}

\newenvironment{deduction}
{\begin{tabular}{@{}>{$}c<{$}@{\enspace}>{$}l<{$}@{}}\arrayrulecolor{blue!50}}
		{\end{tabular}}
\newcommand{\premise}[1]{&#1\\}
\newcommand{\conclusion}[1]{\cline{2-2}\therefore&#1}


% Things Lie
\newcommand{\kb}{\mathfrak b}
\newcommand{\kg}{\mathfrak g}
\newcommand{\kh}{\mathfrak h}
\newcommand{\kn}{\mathfrak n}
\newcommand{\ku}{\mathfrak u}
\newcommand{\kz}{\mathfrak z}
\DeclareMathOperator{\Ext}{Ext} % Ext functor
\DeclareMathOperator{\Tor}{Tor} % Tor functor
\newcommand{\gl}{\opname{\mathfrak{gl}}} % frak gl group
\renewcommand{\sl}{\opname{\mathfrak{sl}}} % frak sl group chktex 6

% More script letters etc.
\newcommand{\SA}{\mathcal A}
\newcommand{\SB}{\mathcal B}
\newcommand{\SC}{\mathcal C}
\newcommand{\SF}{\mathcal F}
\newcommand{\SG}{\mathcal G}
\newcommand{\SH}{\mathcal H}
\newcommand{\OO}{\mathcal O}

\newcommand{\SCA}{\mathscr A}
\newcommand{\SCB}{\mathscr B}
\newcommand{\SCC}{\mathscr C}
\newcommand{\SCD}{\mathscr D}
\newcommand{\SCE}{\mathscr E}
\newcommand{\SCF}{\mathscr F}
\newcommand{\SCG}{\mathscr G}
\newcommand{\SCH}{\mathscr H}

% Mathfrak primes
\newcommand{\km}{\mathfrak m}
\newcommand{\kp}{\mathfrak p}
\newcommand{\kq}{\mathfrak q}

% number sets
\newcommand{\RR}[1][]{\ensuremath{\ifstrempty{#1}{\mathbb{R}}{\mathbb{R}^{#1}}}}
\newcommand{\NN}[1][]{\ensuremath{\ifstrempty{#1}{\mathbb{N}}{\mathbb{N}^{#1}}}}
\newcommand{\ZZ}[1][]{\ensuremath{\ifstrempty{#1}{\mathbb{Z}}{\mathbb{Z}^{#1}}}}
\newcommand{\QQ}[1][]{\ensuremath{\ifstrempty{#1}{\mathbb{Q}}{\mathbb{Q}^{#1}}}}
\newcommand{\CC}[1][]{\ensuremath{\ifstrempty{#1}{\mathbb{C}}{\mathbb{C}^{#1}}}}
\newcommand{\PP}[1][]{\ensuremath{\ifstrempty{#1}{\mathbb{P}}{\mathbb{P}^{#1}}}}
\newcommand{\HH}[1][]{\ensuremath{\ifstrempty{#1}{\mathbb{H}}{\mathbb{H}^{#1}}}}
\newcommand{\FF}[1][]{\ensuremath{\ifstrempty{#1}{\mathbb{F}}{\mathbb{F}^{#1}}}}
% expected value
\newcommand{\EE}{\ensuremath{\mathbb{E}}}
\newcommand{\charin}{\text{ char }}
\DeclareMathOperator{\sign}{sign}
\DeclareMathOperator{\Aut}{Aut}
\DeclareMathOperator{\Inn}{Inn}
\DeclareMathOperator{\Syl}{Syl}
\DeclareMathOperator{\Gal}{Gal}
\DeclareMathOperator{\GL}{GL} % General linear group
\DeclareMathOperator{\SL}{SL} % Special linear group

%---------------------------------------
% BlackBoard Math Fonts :-
%---------------------------------------

%Captital Letters
\newcommand{\bbA}{\mathbb{A}}	\newcommand{\bbB}{\mathbb{B}}
\newcommand{\bbC}{\mathbb{C}}	\newcommand{\bbD}{\mathbb{D}}
\newcommand{\bbE}{\mathbb{E}}	\newcommand{\bbF}{\mathbb{F}}
\newcommand{\bbG}{\mathbb{G}}	\newcommand{\bbH}{\mathbb{H}}
\newcommand{\bbI}{\mathbb{I}}	\newcommand{\bbJ}{\mathbb{J}}
\newcommand{\bbK}{\mathbb{K}}	\newcommand{\bbL}{\mathbb{L}}
\newcommand{\bbM}{\mathbb{M}}	\newcommand{\bbN}{\mathbb{N}}
\newcommand{\bbO}{\mathbb{O}}	\newcommand{\bbP}{\mathbb{P}}
\newcommand{\bbQ}{\mathbb{Q}}	\newcommand{\bbR}{\mathbb{R}}
\newcommand{\bbS}{\mathbb{S}}	\newcommand{\bbT}{\mathbb{T}}
\newcommand{\bbU}{\mathbb{U}}	\newcommand{\bbV}{\mathbb{V}}
\newcommand{\bbW}{\mathbb{W}}	\newcommand{\bbX}{\mathbb{X}}
\newcommand{\bbY}{\mathbb{Y}}	\newcommand{\bbZ}{\mathbb{Z}}

%---------------------------------------
% MathCal Fonts :-
%---------------------------------------

%Captital Letters
\newcommand{\mcA}{\mathcal{A}}	\newcommand{\mcB}{\mathcal{B}}
\newcommand{\mcC}{\mathcal{C}}	\newcommand{\mcD}{\mathcal{D}}
\newcommand{\mcE}{\mathcal{E}}	\newcommand{\mcF}{\mathcal{F}}
\newcommand{\mcG}{\mathcal{G}}	\newcommand{\mcH}{\mathcal{H}}
\newcommand{\mcI}{\mathcal{I}}	\newcommand{\mcJ}{\mathcal{J}}
\newcommand{\mcK}{\mathcal{K}}	\newcommand{\mcL}{\mathcal{L}}
\newcommand{\mcM}{\mathcal{M}}	\newcommand{\mcN}{\mathcal{N}}
\newcommand{\mcO}{\mathcal{O}}	\newcommand{\mcP}{\mathcal{P}}
\newcommand{\mcQ}{\mathcal{Q}}	\newcommand{\mcR}{\mathcal{R}}
\newcommand{\mcS}{\mathcal{S}}	\newcommand{\mcT}{\mathcal{T}}
\newcommand{\mcU}{\mathcal{U}}	\newcommand{\mcV}{\mathcal{V}}
\newcommand{\mcW}{\mathcal{W}}	\newcommand{\mcX}{\mathcal{X}}
\newcommand{\mcY}{\mathcal{Y}}	\newcommand{\mcZ}{\mathcal{Z}}


%---------------------------------------
% Bold Math Fonts :-
%---------------------------------------

%Captital Letters
\newcommand{\bmA}{\boldsymbol{A}}	\newcommand{\bmB}{\boldsymbol{B}}
\newcommand{\bmC}{\boldsymbol{C}}	\newcommand{\bmD}{\boldsymbol{D}}
\newcommand{\bmE}{\boldsymbol{E}}	\newcommand{\bmF}{\boldsymbol{F}}
\newcommand{\bmG}{\boldsymbol{G}}	\newcommand{\bmH}{\boldsymbol{H}}
\newcommand{\bmI}{\boldsymbol{I}}	\newcommand{\bmJ}{\boldsymbol{J}}
\newcommand{\bmK}{\boldsymbol{K}}	\newcommand{\bmL}{\boldsymbol{L}}
\newcommand{\bmM}{\boldsymbol{M}}	\newcommand{\bmN}{\boldsymbol{N}}
\newcommand{\bmO}{\boldsymbol{O}}	\newcommand{\bmP}{\boldsymbol{P}}
\newcommand{\bmQ}{\boldsymbol{Q}}	\newcommand{\bmR}{\boldsymbol{R}}
\newcommand{\bmS}{\boldsymbol{S}}	\newcommand{\bmT}{\boldsymbol{T}}
\newcommand{\bmU}{\boldsymbol{U}}	\newcommand{\bmV}{\boldsymbol{V}}
\newcommand{\bmW}{\boldsymbol{W}}	\newcommand{\bmX}{\boldsymbol{X}}
\newcommand{\bmY}{\boldsymbol{Y}}	\newcommand{\bmZ}{\boldsymbol{Z}}
%Small Letters
\newcommand{\bma}{\boldsymbol{a}}	\newcommand{\bmb}{\boldsymbol{b}}
\newcommand{\bmc}{\boldsymbol{c}}	\newcommand{\bmd}{\boldsymbol{d}}
\newcommand{\bme}{\boldsymbol{e}}	\newcommand{\bmf}{\boldsymbol{f}}
\newcommand{\bmg}{\boldsymbol{g}}	\newcommand{\bmh}{\boldsymbol{h}}
\newcommand{\bmi}{\boldsymbol{i}}	\newcommand{\bmj}{\boldsymbol{j}}
\newcommand{\bmk}{\boldsymbol{k}}	\newcommand{\bml}{\boldsymbol{l}}
\newcommand{\bmm}{\boldsymbol{m}}	\newcommand{\bmn}{\boldsymbol{n}}
\newcommand{\bmo}{\boldsymbol{o}}	\newcommand{\bmp}{\boldsymbol{p}}
\newcommand{\bmq}{\boldsymbol{q}}	\newcommand{\bmr}{\boldsymbol{r}}
\newcommand{\bms}{\boldsymbol{s}}	\newcommand{\bmt}{\boldsymbol{t}}
\newcommand{\bmu}{\boldsymbol{u}}	\newcommand{\bmv}{\boldsymbol{v}}
\newcommand{\bmw}{\boldsymbol{w}}	\newcommand{\bmx}{\boldsymbol{x}}
\newcommand{\bmy}{\boldsymbol{y}}	\newcommand{\bmz}{\boldsymbol{z}}

%---------------------------------------
% Scr Math Fonts :-
%---------------------------------------

\newcommand{\sA}{{\mathscr{A}}}   \newcommand{\sB}{{\mathscr{B}}}
\newcommand{\sC}{{\mathscr{C}}}   \newcommand{\sD}{{\mathscr{D}}}
\newcommand{\sE}{{\mathscr{E}}}   \newcommand{\sF}{{\mathscr{F}}}
\newcommand{\sG}{{\mathscr{G}}}   \newcommand{\sH}{{\mathscr{H}}}
\newcommand{\sI}{{\mathscr{I}}}   \newcommand{\sJ}{{\mathscr{J}}}
\newcommand{\sK}{{\mathscr{K}}}   \newcommand{\sL}{{\mathscr{L}}}
\newcommand{\sM}{{\mathscr{M}}}   \newcommand{\sN}{{\mathscr{N}}}
\newcommand{\sO}{{\mathscr{O}}}   \newcommand{\sP}{{\mathscr{P}}}
\newcommand{\sQ}{{\mathscr{Q}}}   \newcommand{\sR}{{\mathscr{R}}}
\newcommand{\sS}{{\mathscr{S}}}   \newcommand{\sT}{{\mathscr{T}}}
\newcommand{\sU}{{\mathscr{U}}}   \newcommand{\sV}{{\mathscr{V}}}
\newcommand{\sW}{{\mathscr{W}}}   \newcommand{\sX}{{\mathscr{X}}}
\newcommand{\sY}{{\mathscr{Y}}}   \newcommand{\sZ}{{\mathscr{Z}}}


%---------------------------------------
% Math Fraktur Font
%---------------------------------------

%Captital Letters
\newcommand{\mfA}{\mathfrak{A}}	\newcommand{\mfB}{\mathfrak{B}}
\newcommand{\mfC}{\mathfrak{C}}	\newcommand{\mfD}{\mathfrak{D}}
\newcommand{\mfE}{\mathfrak{E}}	\newcommand{\mfF}{\mathfrak{F}}
\newcommand{\mfG}{\mathfrak{G}}	\newcommand{\mfH}{\mathfrak{H}}
\newcommand{\mfI}{\mathfrak{I}}	\newcommand{\mfJ}{\mathfrak{J}}
\newcommand{\mfK}{\mathfrak{K}}	\newcommand{\mfL}{\mathfrak{L}}
\newcommand{\mfM}{\mathfrak{M}}	\newcommand{\mfN}{\mathfrak{N}}
\newcommand{\mfO}{\mathfrak{O}}	\newcommand{\mfP}{\mathfrak{P}}
\newcommand{\mfQ}{\mathfrak{Q}}	\newcommand{\mfR}{\mathfrak{R}}
\newcommand{\mfS}{\mathfrak{S}}	\newcommand{\mfT}{\mathfrak{T}}
\newcommand{\mfU}{\mathfrak{U}}	\newcommand{\mfV}{\mathfrak{V}}
\newcommand{\mfW}{\mathfrak{W}}	\newcommand{\mfX}{\mathfrak{X}}
\newcommand{\mfY}{\mathfrak{Y}}	\newcommand{\mfZ}{\mathfrak{Z}}
%Small Letters
\newcommand{\mfa}{\mathfrak{a}}	\newcommand{\mfb}{\mathfrak{b}}
\newcommand{\mfc}{\mathfrak{c}}	\newcommand{\mfd}{\mathfrak{d}}
\newcommand{\mfe}{\mathfrak{e}}	\newcommand{\mff}{\mathfrak{f}}
\newcommand{\mfg}{\mathfrak{g}}	\newcommand{\mfh}{\mathfrak{h}}
\newcommand{\mfi}{\mathfrak{i}}	\newcommand{\mfj}{\mathfrak{j}}
\newcommand{\mfk}{\mathfrak{k}}	\newcommand{\mfl}{\mathfrak{l}}
\newcommand{\mfm}{\mathfrak{m}}	\newcommand{\mfn}{\mathfrak{n}}
\newcommand{\mfo}{\mathfrak{o}}	\newcommand{\mfp}{\mathfrak{p}}
\newcommand{\mfq}{\mathfrak{q}}	\newcommand{\mfr}{\mathfrak{r}}
\newcommand{\mfs}{\mathfrak{s}}	\newcommand{\mft}{\mathfrak{t}}
\newcommand{\mfu}{\mathfrak{u}}	\newcommand{\mfv}{\mathfrak{v}}
\newcommand{\mfw}{\mathfrak{w}}	\newcommand{\mfx}{\mathfrak{x}}
\newcommand{\mfy}{\mathfrak{y}}	\newcommand{\mfz}{\mathfrak{z}}


\title{\Huge{Systems of Linear Equations}}
\author{\huge{Madiba Hudson-Quansah}}
\date{}
\usepackage{parskip}

\setcounter{tocdepth}{4}
\setcounter{secnumdepth}{4}

\begin{document}
\maketitle
\newpage
\pdfbookmark[section]{\contentsname}{too}
\tableofcontents
\pagebreak

\chapter{Introduction}

\dfn{Linear Equation}{
  An equation in the form
  \[
    a_1x_1 + a_2x_2 + \ldots + a_nx_n = b
  \]
  where the constant $b$ and coefficients $a_1, a_2, \ldots, a_n$  are real or complex numbers.
}

\dfn{System of Linear Equations}{
  A collection of one or more linear equations involving the same set of variables. When a system of linear equations is written in the form
  \begin{align*}
    a_1 x_1 + a_2 x_2 + a_3 x_3 = b_1 \\
    a_4 x_1 + a_6 x_3 = b_2           \\
  \end{align*}
  The set of variables takes on the longest subscript in the system. In this case, the variables are $x_1, x_2, x_3$.
}

\dfn{Solution of a System of Linear Equations}{
  The \textit{solution} of a system of linear equations is a list of values, $\left( s_1, s_2, \ldots, s_n \right) $
  that makes each equation in the system a true statement when the values are substituted for the variables, i.e. $x_1,
    x_2, \ldots, x_n$ and $s_1, s_2, \ldots, s_n$, where $s_n$ is substituted for $x_n$
}

\dfn{Solution Set}{
  The set of all possible solutions of a system of linear equations.
}

\dfn{Equivalence}{
  Two linear systems are said to be \textit{equivalent} if they have the same solution set.
}

\dfn{Consistency}{
  A system of linear equations is said to be \textit{consistent} if it has at least one solution, and
  \textit{inconsistent} if it has no solution.
}

A system of linear equations can either have:
\begin{itemize}
  \item No solution - Equations do not intersect
  \item Exactly one / Unique solution - Equations intersect at a single point
  \item Infinitely many solutions - Equations are the same
\end{itemize}

\section{Matrix Notation}

A system of linear equations can be represented in matrix form two ways:
\begin{itemize}
  \item Coefficient Matrix
  \item Augmented Matrix
\end{itemize}

\subsection{Coefficient Matrix}

\dfn{Coefficient Matrix}{
  Denoted by $A$, the coefficient matrix is a matrix that contains the coefficients of the variables in the system of
  linear equations with the coefficients of each equation making up each row.
}

\ex{}{
  For the system of linear equations:
  \begin{align*}
    a_1 x_2 + a_2 x_2 + a_3 x_3 = b_1 \\
    a_4 x_1 + a_5 x_2 + a_6 x_3 = b_2 \\
    a_7 x_1 + a_8 x_2 + a_9 x_3 = b_3 \\
  \end{align*}
  The coefficient matrix is:
  \[
    \begin{bmatrix} a_1 & a_2 & a_3 \\ a_4 & a_5 & a_6 \\  a_7 & a_8 & a_9 \end{bmatrix}
  \]

}

\subsection{Augmented Matrix}
\dfn{Augmented Matrix}{
  Denoted by $[A|B]$, the augmented matrix is a matrix that contains the coefficients of the variables in the system of linear equations with the constant terms of each equation making up the last column.
}

\ex{}{
  For the system of linear equations:
  \begin{align*}
    a_1 x_2 + a_2 x_2 + a_3 x_3 = b_1 \\
    a_4 x_1 + a_5 x_2 + a_6 x_3 = b_2 \\
    a_7 x_1 + a_8 x_2 + a_9 x_3 = b_3 \\
  \end{align*}
  The augmented matrix is:
  \[
    \begin{bmatrix} a_1 & a_2 & a_3 & b_1\\ a_4 & a_5 & a_6 & b_2 \\ a_7 & a_8 & a_9  & b_3 \end{bmatrix}
  \]
}

\dfn{Size of a Matrix}{
  The size of a matrix, denoted by $m \times n$, is the number of rows and columns in the matrix respectively.
  If $n = m$ then the matrix is said to be square, if not, it is said to be rectangular.
}

\section{Solving Linear Systems}

\dfn{Pivot}{
  Diagonal non-zero elements  of a linear system
}

\dfn{Forward Elimination Process}{
  The process used to change a system into an upper triangular matrix
}
\dfn{Backward Substitution Method}{
  The process of deriving a solution from an upper triangular matrix
}

\dfn{Identity Matrix}{
  A matrix containing all zeros with pivots of 1
  \[
    \begin{bmatrix} 1 & 0 & 0 \\ 0 & 1 & 0 \\ 0 & 0 & 1 \end{bmatrix}
  \]
}



One procedure used to solve linear systems is that of \textit{simplification}. This involves replacing one linear system
with a simpler equivalent system. This is done by applying the following operations to the system:
\begin{description}
  \item[Replacement] Replace one equation by the sum of itself and a multiple of another equation.
  \item[Interchange] Interchange two equations.
  \item[Scaling] Multiply all the terms in an equation by a non-zero constant.
\end{description}

\ex{}{
  \qs{}{
    Solve the system
    \begin{align*}
      x_1    -2x_2   + x_3    = 0 \\
      2x_2    - 8x_3   = 8        \\
      5x_1           -5x_3    = 10
    \end{align*}
  }

  \sol{
    Using the augmented matrix representation, we have:
    \[
      \begin{bmatrix} 1 & -2 & 1 & 0 \\ 0 & 2 & -8 & 8 \\ 5 & 0 & -5 &10   \end{bmatrix}
    \]
    Then we times the first equation through by $-5$ and add it to the third equation to replace the third equation:
    \begin{align*}
      \begin{split}
        -5x_1 + 10x_2 - 5x_3 = 0 \\
        5x_1 - 5x_3 = 10         \\
        \hline
        10x_2 - 10x_3 = 10       \\
      \end{split}
    \end{align*}
    Giving us:
    \[
      \begin{bmatrix} 1 & -2 &  1 &  0  \\ 0 & 2 & -8 & 8 \\ 0 & 10 & -10 & 10  \end{bmatrix}
    \]
    We then eliminate $x_2$ by multiplying equation 2 by $-5$ and add it again to the third equation again replacing it:
    \begin{align*}
      \begin{split}
        -10x_2 + 40x_3 = -40 \\
        10x_2 - 10x_3 = 10   \\
        \hline
        30x_3 = -30          \\
      \end{split}
    \end{align*}
    Giving us:
    \[
      \begin{bmatrix} 1 & -2 & 1 & 0 \\ 0 & 2 & -8 & 8 \\ 0 & 0 & 30 & -30 \end{bmatrix}
    \]
    This new system has a triangular form, i.e.
    \begin{align*}
      x_1 - 2x_2 + x_3 = 0 \\
      2x_2 - 8x_3 = 8      \\
      30x_3 = -30          \\
    \end{align*}

    \noindent We then continue eliminating variables until one remains in each equation:
    \begin{align*}
      \begin{split}
        -x_3 = 1             \\
        x_1 - 2x_2 + x_3 = 0 \\
        \hline
        x_1 - 2x_2 = 1
      \end{split}
    \end{align*}
    \[
      \begin{bmatrix} 1 & -2 & 0 & 1 \\ 0 & 2 & -8 & 8 \\ 0 & 0 & 30  &-30 \end{bmatrix}
    \]


    \begin{align*}
      \begin{split}
        8x_3 = -8       \\
        2x_2 - 8x_3 = 8 \\
        \hline
        2x_2 = 0        \\
      \end{split}
    \end{align*}

    \[
      \begin{bmatrix} 1 & -2 & 0 & 1 \\ 0 & 2& 0 & 0 \\ 0 & 0 & 30 & -30  \end{bmatrix}
    \]

    \begin{align*}
      \begin{split}
        2x_2 = 0       \\
        x_1 - 2x_2 = 1 \\
        \hline
        x_1 = 1        \\
      \end{split}
    \end{align*}

    \[
      \begin{bmatrix} 1 & 0 & 0 & 1 \\ 0 & 2 & 0 & 0 \\ 0 & 0 & 30 & -30  \end{bmatrix}
    \]

    Giving us the system:
    \begin{align*}
      x_1 = 1     \\
      2x_2 = 0    \\
      30x_3 = -30 \\
    \end{align*}
    Which simplifies into:
    \begin{align*}
      x_1 = 1 \\
      x_2 = 0 \\
      x_3 = -1
    \end{align*}

  }
}

\dfn{Row Equivalence}{
  Two matrices are row equivalent if there is a sequence of elementary row operations that transforms one matrix into
  the other
}

\thm{}{
  If the augmented matrices of two linear systems are row equivalent, then the two equations have the same solution set.
}

\section{Identifying Existence and Uniqueness}

To determine the nature of a linear system we must answer two fundamental questions:
\begin{itemize}
  \item Is the system consistent? / Does a solution exist?
  \item If a solution exists, is it the only one? / Is the solution unique
\end{itemize}

\ex{}{
  \qs{}{
    Determine if the following system is consistent:
    \begin{align*}
      x_1    -2x_2   + x_3    = 0 \\
      2x_2    - 8x_3   = 8        \\
      5x_1           -5x_3    = 10
    \end{align*}
  }

  \sol{
    Having already found the solution for this system:
    \begin{align*}
      x_1 = 1 \\
      x_2 = 0 \\
      x_3 = -1
    \end{align*}
    We can determine that a solution exists, and due to the fact $x_2$ is uniquely determined by equation two, $x_3$ has
    only one possible value, and $x_1$ is also uniquely determined by equation one, we can also conclude this solution
    is unique.
  }
}

\ex{}{
  \qs{}{
    Determine if the following system is consistent:
    \begin{align*}
      x_2 - 4x_3 = 8         \\
      2x_1 - 3x_2 + 2x_3 = 1 \\
      4x_1 - 8x_2 + 12x_3 = 1
    \end{align*}
  }

  \sol{
    The augmented matrix is:
    \[
      \begin{bmatrix}  0 & 1 & -4 & 8 \\ 2 & -3 & 2 & 1 \\  4 & -8 & 12 & 1 \end{bmatrix}
    \]

    We interchange equations 1 and 2:
    \[
      \begin{bmatrix} 2 & -3 & 2 & 1 \\ 0 & 1 & -4 & 8 \\ 4 & -8 & 12 & 1 \end{bmatrix}
    \]

    \begin{align*}
      \begin{split}
        -4x_1 + 6x_2 - 4x_3 = -2 \\
        4x_1 - 8x_2 + 12x_3 = 1  \\
        \hline
        -2x_2 + 8x_3 = -1
      \end{split}
    \end{align*}

    \[
      \begin{bmatrix} 2 & -3 & 2 & 1 \\ 0 & 1 & -4 & 8 \\ 0 & -2 & 8 & -1 \end{bmatrix}
    \]

    \begin{align*}
      \begin{split}
        2x_2 - 8x_3 = 16  \\
        -2x_2 + 8x_3 = -1 \\
        \hline
        0 = 15
      \end{split}
    \end{align*}

    \[
      \begin{bmatrix} 2 & -3 & 2 & 1 \\ 0 & 1 & -4 & 8 \\ 0 & 0 & 0 & 15 \end{bmatrix}
    \]

    Now in its triangular form, we can determine the existence and uniqueness of the solutions:
    \begin{align*}
      2x_1 - 3x_2 + 2x_3 = 1 \\
      x_2 - 4x_3 = 8         \\
      0 = 15                 \\
    \end{align*}

    Since there are no coefficients for $x_1$, $x_2$, and $x_3$ in equation 3 equation 3 has no solution. This makes the
    solution set for this linear system $\{1, 8\}$. Because this set is the same as the solution set for the original
    linear system, $\{8, 1, 1\} $, the original system is inconsistent


  }
}



\section{Exercises}

\qs{}{
  Determine if the linear system represented by the augmented matrix below is consistent:
  \[
    \begin{bmatrix} 1 & 5 & 2 & -6 \\ 0 & 4 & - 7 & 2 \\ 0 & 0 & 5 & 0 \end{bmatrix}
  \]
}

\sol{
  \begin{align*}
    x_1 + 5x_2 + 2x_3 = -6                   \\
    4x_2 -7x_3 = 2                           \\
    5x_3 = 0                                 \\
    \\
    x_3 = 0                                  \\
    x_1 + 5x_2 = -6                          \\
    x_1 = -6 - 5x_2                          \\
    \\
    4x_2 = 2                                 \\
    x_2 = \frac{1}{2}                        \\
    \\
    x_1  = - 6- 5 \left( \frac{1}{2} \right) \\
    x_1 = -\frac{17}{2}
  \end{align*}

}

\qs{}{
  Solve the following systems:
  \begin{enumerate}
    \item
          \begin{align*}
            x_2 + 4x_3 = -5        \\
            x_1 + 3x_2 + 5x_3 = -2 \\
            3x_1 + 7x_2 + 7x_3 = 6
          \end{align*}
    \item
          \begin{align*}
            x_1 - 2x_4 = -3 \\
            2x_2 + 2x_3 = 0 \\
            x_3 + 3x_4 = 1  \\
            -2x_1 + 3x_2 + 2x_3 + x_4 = 5
          \end{align*}
  \end{enumerate}

}

\sol{
  \begin{enumerate}
    \item
          \[
            \begin{bmatrix} 0 & 1 & 4 & -5 \\ 1&3 & 5 & -2 \\ 3 & 7 & 7 & 6 \end{bmatrix}
          \]
          $R_1 \leftrightarrow R_3$
          \[
            \begin{bmatrix} 3 & 7 & 7 & 6 \\ 1&3 & 5 & -2 \\  0 & 1 & 4 & -5 \end{bmatrix}
          \]
          $\frac{1}{3}R_1 - R_2 \to R_2$
          \begin{align*}
            \begin{split}
              x_1 + \frac{7}{3}x_2 + \frac{7}{3}x_3 = 2 \\
              x_1 + 3x_2 + 5x_3 = -2                    \\
              \hline
              -\frac{2}{3}x_2 - \frac{8}{3}x_3 = 4
            \end{split}
          \end{align*}

          \[
            \begin{bmatrix} 3 & 7 & 7 & 6 \\ 0&-\frac{2}{3} & -\frac{8}{3} & 4 \\  0 & 1 & 4 & -5 \end{bmatrix}
          \]
          $3R_2$
          \[
            \begin{bmatrix} 3 & 7 & 7 & 6 \\ 0&-2 & -8 & 12 \\  0 & 1 & 4 & -5 \end{bmatrix}
          \]

          $-\frac{1}{2}R_2 - R_3 \to R_3$
          \begin{align*}
            \begin{split}
              x_2 + 4x_3 = -6 \\
              x_2 + 4x_3 = -5 \\
              \hline
              0 = -1          \\
            \end{split}
          \end{align*}

          \[
            \begin{bmatrix} 3 & 7 & 7 & 6 \\ 0&-2 & -8 & 12 \\  0 & 0 & 0 & -1 \end{bmatrix}
          \]

          Because the system has a contradiction in row 3, $0x_1 + 0x_2 + 0x_3 = -1$, the system has no solution and is
          therefore inconsistent.

    \item
          \[
            \begin{bmatrix} 1 & 0 & 0 & -2 & -3 \\ 0 & 2 & 2 & 0 & 0 \\ 0 & 0 & 1 & 3 & 1 \\ -2 & 3 & 2 & 1 & 5 \end{bmatrix}
          \]

          $R_1 \leftrightarrow R_4$
          \[
            \begin{bmatrix}
              -2 & 3 & 2 & 1  & 5  \\
              0  & 2 & 2 & 0  & 0  \\
              0  & 0 & 1 & 3  & 1  \\
              1  & 0 & 0 & -2 & -3 \\
            \end{bmatrix}
          \]

          $-\frac{1}{2} R_1 - R_4 \to R_4$
          \begin{align*}
            \begin{split}
              x_1 - \frac{3}{2}x_2 - x_3 - \frac{1}{2}x_4 = -\frac{5}{2} \\
              x_1 + 0 + 0 - 2x_4 = -3                                    \\
              \hline
              -\frac{3}{2}x_2 - x_3 + \frac{3}{2}x_4 = \frac{1}{2}
            \end{split}
          \end{align*}
          \[
            \begin{bmatrix}
              -2 & 3            & 2  & 1           & 5           \\
              0  & 2            & 2  & 0           & 0           \\
              0  & 0            & 1  & 3           & 1           \\
              0  & -\frac{3}{2} & -1 & \frac{3}{2} & \frac{1}{2} \\
            \end{bmatrix}
          \]

          $-\frac{3}{4}R_2 - R_4 \to R_4$
          \begin{align*}
            \begin{split}
              0  -\frac{3}{2}x_2 - \frac{3}{2}x_3 + 0 = 0             \\
              0 - \frac{3}{2}x_2 - x_3 + \frac{3}{2}x_4 = \frac{1}{2} \\
              \hline
              -\frac{1}{2}x_3 - \frac{3}{2}x_4 = -\frac{1}{2}
            \end{split}
          \end{align*}

          \[
            \begin{bmatrix}
              -2 & 3 & 2            & 1            & 5            \\
              0  & 2 & 2            & 0            & 0            \\
              0  & 0 & 1            & 3            & 1            \\
              0  & 0 & -\frac{1}{2} & -\frac{3}{2} & -\frac{1}{2} \\
            \end{bmatrix}
          \]

          $-\frac{1}{2}R_3 - R_4 \to R_4$

          \begin{align*}
            \begin{split}
              0 + 0 - \frac{1}{2}x_3 - \frac{3}{2}x_4 = -\frac{1}{2} \\
              0 + 0-\frac{1}{2}x_3 - \frac{3}{2}x_4 = -\frac{1}{2}   \\
              \hline
              0x_1 + 0x_2 + 0x_3 + 0x_4 = 0
            \end{split}
          \end{align*}

          \[
            \begin{bmatrix}
              -2 & 3 & 2 & 1 & 5 \\
              0  & 2 & 2 & 0 & 0 \\
              0  & 0 & 1 & 3 & 1 \\
              0  & 0 & 0 & 0 & 0 \\
            \end{bmatrix}
          \]

          $\frac{3}{2}R_2 - R_1 \to R_1$

          \begin{align*}
            \begin{split}
              0 + 3x_2 + 3x_3 + 0 = 0      \\
              -2x_1 + 3x_2+ 2x_3 + x_4 = 5 \\
              \hline
              2x_1 + 0 + x_3 - x_4 = -5
            \end{split}
          \end{align*}

          \[
            \begin{bmatrix}
              2 & 0 & 1 & -1 & -5 \\
              0 & 2 & 2 & 0  & 0  \\
              0 & 0 & 1 & 3  & 1  \\
              0 & 0 & 0 & 0  & 0  \\
            \end{bmatrix}
          \]
          $R_3 - R_1 \to R_1$
          \begin{align*}
            \begin{split}
              0 + 0 + x_3 + 3x_4 = 1    \\
              2x_1 + 0 + x_3 - x_4 = -5 \\
              \hline
              -2x_1 + 0 + 0 + 4x_4 = 6
            \end{split}
          \end{align*}

          \[
            \begin{bmatrix}
              -2 & 0 & 0 & 4 & 6 \\
              0  & 2 & 2 & 0 & 0 \\
              0  & 0 & 1 & 3 & 1 \\
              0  & 0 & 0 & 0 & 0 \\
            \end{bmatrix}
          \]
          $2R_3 - R_2 \to R_2$
          \begin{align*}
            \begin{split}
              0 + 0 + 2x_3 + 6x_4 = 2 \\
              0 + 2x_2 + 2x_3 + 0 = 0 \\
              \hline
              -2x_2 + 6x_4 = 2
            \end{split}
          \end{align*}

          \[
            \begin{bmatrix}
              -2 & 0  & 0 & 4 & 6 \\
              0  & -2 & 0 & 6 & 2 \\
              0  & 0  & 1 & 3 & 1 \\
              0  & 0  & 0 & 0 & 0 \\
            \end{bmatrix}
          \]
          $\frac{R_1}{-2}$
          \[
            \begin{bmatrix}
              1 & 0  & 0 & -2 & -3 \\
              0 & -2 & 0 & 6  & 2  \\
              0 & 0  & 1 & 3  & 1  \\
              0 & 0  & 0 & 0  & 0  \\
            \end{bmatrix}
          \]
          $\frac{R_2}{-2}$
          \[
            \begin{bmatrix}
              1 & 0 & 0 & -2 & -3 \\
              0 & 1 & 0 & -3 & -1 \\
              0 & 0 & 1 & 3  & 1  \\
              0 & 0 & 0 & 0  & 0  \\
            \end{bmatrix}
          \]

          \begin{align*}
            x_1 + -2x_4 = -3                   \\
            x_2 -3x_4 = -1                     \\
            x_3 + 3x_4 = 1                     \\
            0 = 0                              \\
            \\
            x_4 = \frac{1}{2}x_1 + \frac{3}{2} \\
            x_2 = -1 + 3x_4                    \\
            x_3 = 1 - 3x_4                     \\
            x_1 = -3 + 2x_4
          \end{align*}

  \end{enumerate}
}

\pagebreak
\qs{}{
  For the following matrices find the elementary row operation that transforms
  the first matrix into the second, and then find the reverse row operation that
  transforms the second matrix into the first
  \begin{enumerate}
    \item
          \[
            \begin{bmatrix}
              1 & 3   & -1 \\
              0 & -2  & 6  \\
              0 & - 5 & 9
            \end{bmatrix} ,
            \begin{bmatrix}
              1 & 3  & -4 \\
              0 & 1  & -3 \\
              0 & -5 & 9
            \end{bmatrix}
          \]
    \item
          \[
            \begin{bmatrix}
              1 & -2 & 1  & 0  \\
              0 & 5  & -2 & 8  \\
              4 & -1 & 3  & -6
            \end{bmatrix} ,
            \begin{bmatrix}
              1 & -2 & 1  & 0  \\
              0 & 5  & -2 & 8  \\
              0 & 7  & -1 & -6
            \end{bmatrix}
          \]
  \end{enumerate}
}

\sol{
  \begin{enumerate}
    \item
          Let the first matrix be $M_1$ and the second be $M_2$
          \begin{align*}
            M_1 \to M_2 & = -\frac{1}{2}R_2          \\
            M_2 \to M_1 & = \frac{R_2}{-\frac{1}{2}} \\
          \end{align*}
    \item
          Let the first matrix be $M_1$ and the second be $M_2$
          \begin{align*}
            M_1 \to M_2 & = -4 R_1 + R_3 \to R_3 \\
            M_2 \to M_1 & = R_3 - 4R_1           \\
          \end{align*}
  \end{enumerate}
}

\chapter{Row Reduction and Echelon Forms}
\dfn{Leading Entry}{
  The leftmost entry in a non-zero row
}

\dfn{Upper triangular matrix / Echelon Form}{
  A rectangular matrix is in \textit{echelon form / row echelon form}  if it has the following proprieties:
  \begin{itemize}
    \item All non-zero rows are above any rows of all zeros
    \item Each leading entry of a row is in a column to the right of the leading entry of the row above it
    \item All entries in a column below a leading entry are zeros
  \end{itemize}
}

\dfn{Reduced Row Echelon Form}{
  A matrix is in \textit{reduced row echelon form} if it meets all the conditions of a matrix in echelon form and:
  \begin{itemize}
    \item The leading entry in each non-zero row is 1
    \item Each leading 1 is the only non-zero entry in its column
  \end{itemize}
}

\dfn{Echelon Matrix}{
  A matrix that is in echelon form
}

\dfn{Reduced Echelon Matrix}{
  A matrix that is in reduced row echelon form
}


\thm{Uniqueness of a Row Reduced Echelon Form}{
  Each matrix is row equivalent to one and only one reduced row echelon form
}

Therefore if matrix $A$ is row equivalent to an echelon matrix $U$, $U$ is the \textbf{echelon form of} $A$, and if $A$ is row equivalent to a reduced echelon matrix $R$, $R$ is the \textbf{reduced echelon form of} $A$.

\section{Pivot Positions}

\dfn{Pivot Position}{
  A \textit{pivot position} in matrix $A$ is a location in $A$ that corresponds to a leading 1 in the reduced echelon
  form of $A$.
}
\dfn{Pivot Column}{
  A column of a matrix $A$ that contains a pivot position.
}

A pivot cannot be 0 and a pivot column cannot contain any other non-zero entries. Therefore when identifying pivot
positions we look for the first non-zero entry in each row, that is not in a column that already contains a pivot.

\ex{}{
  For the reduced echelon matrix:
  \[
    \begin{bmatrix} 1 & 4 & 5 & -9 & -7 \\ 0 & 2 & 4 & -6 & -6 \\ 0 & 0 & 0 & -5 & 0 \\  0 & 0 & 0 & 0 & 0 \end{bmatrix}
  \]

  The pivots positions are $\left( 1,1 \right), \left( 2,2 \right), \left( 3,4 \right)   $ and values are $\left( 1,2,-5 \right) $
}

\section{The Row Reduction Algorithm}

The row Reduction Algorithm reduces a matrix to it's echelon form in four steps with the fifth reducing the echelon
matrix to a reduced echelon matrix. These steps are:
\begin{enumerate}
  \item Start with the leftmost non-zero column, this is the first pivot column, with its pivot position at the top of
        the column
  \item Select a non-zero entry in the pivot column as a pivot, interchanging rows if necessary
  \item Use row replacement operations to create all zeros below the pivot.
  \item Cover the row containing the pivot and repeat the process for the submatrix that remains.
  \item Beginning with the rightmost pivot and working upward to the left, create zeros above each pivot. If the pivot
        is not 1 make it 1 via a scaling operation
\end{enumerate}

\ex{}{
  Given the matrix:
  \[
    \begin{bmatrix}
      0 & 3  & -6 & 6  & 4 & 5  \\
      3 & -7 & 8  & -5 & 8 & 9  \\
      3 & -9 & 12 & -9 & 6 & 15
    \end{bmatrix}
  \]

  We interchange the first and second rows to have the first item in the leftmost column be non-zero:
  \\
  $R_1 \leftrightarrow R_2$
  \[
    \begin{bmatrix}
      3 & -7 & 8  & -5 & 8 & 9  \\
      0 & 3  & -6 & 6  & 4 & 5  \\
      3 & -9 & 12 & -9 & 6 & 15
    \end{bmatrix}
  \]
  We then select the non-zero entry in the first column, 3, as the pivot for the first column, and use it to render zeros
  in the first column below the pivot
  $R_1 - R_3 \to R_3$
  \begin{align*}
    \begin{split}
      3x_1 - 7x_2 + 8x_3 - 5x_4 + 8x_5 = 9 \\
      3x_1 - 9x_2 + 12x_3 -9x_4 +6x_5 = 15 \\
      \hline
      2x_2 - 4x_3 + 4x_4 + 2x_5 = -6       \\
    \end{split}
  \end{align*}

  \[
    \begin{bmatrix}
      3 & -7 & 8  & -5 & 8 & 9  \\
      0 & 3  & -6 & 6  & 4 & 5  \\
      0 & 2  & -4 & 4  & 2 & -6
    \end{bmatrix}
  \]
  And repeat the process for the remaining columns: \\
  \[
    \frac{2}{3}R_2 - R_3 \to R_3
  \]
  \begin{align*}
    \begin{split}
      0 + 2x_2 - 4x_3 + 4x_4 + \frac{8}{3}x_5 = \frac{10}{3} \\
      0 + 2x_2 - 4x_3 + 4x_4 + 2x_5 = -6                     \\
      \hline
      0 + 0 + 0 + 0 + \frac{2}{3}x_5 = \frac{28}{3}
    \end{split}
  \end{align*}
  \[
    \begin{bmatrix}
      3 & -7 & 8  & -5 & 8           & 9            \\
      0 & 3  & -6 & 6  & 4           & 5            \\
      0 & 0  & 0  & 0  & \frac{2}{3} & \frac{28}{3}
    \end{bmatrix}
  \]

  \[
    6R_3 - R_2 \to R_2
  \]
  \begin{align*}
    \begin{split}
      0 + 0 + 0 + 0 + 4x_5 = 56         \\
      0 + 3x_2 - 6x_3 + 6x_4 + 4x_5 = 5 \\
      \hline
      -3x_2 + 6x_3 -6x_4 = 51
    \end{split}
  \end{align*}

  \[
    \begin{bmatrix}
      3 & -7 & 8 & -5 & 8           & 9            \\
      0 & -3 & 6 & -6 & 0           & 51           \\
      0 & 0  & 0 & 0  & \frac{2}{3} & \frac{28}{3}
    \end{bmatrix}
  \]

  \[
    12R_3 - R_1 \to R_1
  \]
  \begin{align*}
    \begin{split}
      0 + 0 + 0 + 0 + 8x_5 = 612          \\
      3x_1 -7x_2 + 8x_3 - 5x_4 + 8x_5 = 9 \\
      \hline
      -3x_1 + 7x_2 - 8x_3 + 5x_4  = 603
    \end{split}
  \end{align*}
  \[
    \begin{bmatrix}
      -3 & 7  & -8 & 5  & 0           & 603          \\
      0  & -3 & 6  & -6 & 0           & 51           \\
      0  & 0  & 0  & 0  & \frac{2}{3} & \frac{28}{3}
    \end{bmatrix}
  \]

  \[
    -\frac{7}{3}R_2 - R_1 \to R_1
  \]

  \begin{align*}
    \begin{split}
      0 + 7x_2 -14x_3  + 14x_4 + 0 = -119  \\
      -3x_1 + 7x_2 - 8x_3 + 5x_4 + 0 = 603 \\
      \hline
      3x_1 + 0 -6x_3 + 9x_4 = -722
    \end{split}
  \end{align*}
  \[
    \begin{bmatrix}
      3 & 0  & -6 & 9  & 0           & -722         \\
      0 & -3 & 6  & -6 & 0           & 51           \\
      0 & 0  & 0  & 0  & \frac{2}{3} & \frac{28}{3}
    \end{bmatrix}
  \]
  \[
    \frac{3}{2}R_3 \to R_3
  \]
  \[
    \begin{bmatrix}
      3 & 0  & -6 & 9  & 0 & -722 \\
      0 & -3 & 6  & -6 & 0 & 51   \\
      0 & 0  & 0  & 0  & 1 & 14   \\
    \end{bmatrix}
  \]

  \[
    -\frac{1}{3}R_2 \to R_2
  \]
  \[
    \begin{bmatrix}
      3 & 0 & -6 & 9 & 0 & -722 \\
      0 & 1 & -2 & 2 & 0 & -17  \\
      0 & 0 & 0  & 0 & 1 & 14   \\
    \end{bmatrix}
  \]

  \[
    \frac{1}{3}R_1 \to R_1
  \]
  \[
    \begin{bmatrix}
      1 & 0 & -2 & 3 & 0 & -\frac{722}{3} \\
      0 & 1 & -2 & 2 & 0 & -17            \\
      0 & 0 & 0  & 0 & 1 & 14             \\
    \end{bmatrix}
  \]
}

\section{Solutions of Linear Systems}

\dfn{Free Variable}{
  A variable that does not exist in a row of a matrix
}

\dfn{Parametric Equations}{
  Any equation that expresses the variables in a system of linear equations in terms of a free variable.
}

\dfn{Basic Variable}{
  A variable that exists in a row of a matrix
}

In deriving a reduced echelon matrix we can determine the consistency of a system of linear equations, and thereby
describe the solution set. For example in the case below we have the reduced echelon matrix:
\begin{align*}
  \begin{bmatrix}
    1 & 0 & -5 & 1 \\
    0 & 1 & 1  & 4 \\
    0 & 0 & 0  & 0
  \end{bmatrix}
  \\
  \\
  x_1 - 5x_3 = 1 \\
  x_2 + x_3 = 4  \\
  0 = 0
\end{align*}
There are three variables $x_1, x_2, \text{ and } x_3$, as there are four columns in this augmented matrix. The
variables $x_1$ and $x_2$ are referred to as basic variables as they have pivots in their columns, and $x_3$ is a free
variable as it does not have a pivot in its column, thus this system is consistent has has infinitely many solutions.
Describing the solution set of this equation we can say:
\begin{align*}
  \begin{cases}
    x_1 = 1 + 5x_3 \\
    x_2 = 4 - x_3  \\
    x_3 \text{ is free}
  \end{cases}
\end{align*}
The statement "$x_3$ is free" means that $x_3$ can take on any value, and by extension the values of $x_1$ and $x_2$
are determined by the value $x_3$ takes on.

\ex{}{
  \qs{}{
    Find the general solution of the linear system whose augmented matrix has been reduced to:
    \[
      \begin{bmatrix}
        1 & 6 & 2 & -5 & -2 & -4 \\
        0 & 0 & 2 & -8 & -1 & 3  \\
        0 & 0 & 0 & 0  & 1  & 7
      \end{bmatrix}
    \]
  }

  \sol{
    \[
      \begin{bmatrix}
        1 & 6 & 2 & -5 & -2 & -4 \\
        0 & 0 & 2 & -8 & -1 & 3  \\
        0 & 0 & 0 & 0  & 1  & 7
      \end{bmatrix}
    \]

    \[
      -1R_3 - R_2 \to R_2
    \]
    \begin{align*}
      \begin{split}
        0 + 0 + 0 + 0 - 1 = -7   \\
        0 + 0 + 2 - 8 - 1 = -3   \\
        \hline
        0 + 0 + -2 + 8 + 0 = -10 \\
      \end{split}
    \end{align*}
    \[
      \begin{bmatrix}
        1 & 6 & 2  & -5 & -2 & -4  \\
        0 & 0 & -2 & 8  & 0  & -10 \\
        0 & 0 & 0  & 0  & 1  & 7
      \end{bmatrix}
    \]

    \[
      -2R_3 - R_1 \to R_1
    \]
    \begin{align*}
      \begin{split}
        0 + 0 + 0 + 0 -2 = -14 \\
        1 + 6 + 2 - 5 - 2 = -4 \\
        \hline
        -1 - 6 - 2 + 5 + 0 = -10
      \end{split}
    \end{align*}
    \[
      \begin{bmatrix}
        -1 & -6 & -2 & 5 & 0 & -10 \\
        0  & 0  & -2 & 8 & 0 & -10 \\
        0  & 0  & 0  & 0 & 1 & 7
      \end{bmatrix}
    \]
    \[
      R_2 - R_1 \to R_1
    \]

    \begin{align*}
      \begin{split}
        0 + 0 -2 + 8 + 0 = -10 \\
        -1 -6 -2 + 5 + 0 = -10 \\
        \hline
        1 + 6 + 0 + 3 + 0 = 0
      \end{split}
    \end{align*}
    \[
      \begin{bmatrix}
        1 & 6 & 0  & 3 & 0 & 0   \\
        0 & 0 & -2 & 8 & 0 & -10 \\
        0 & 0 & 0  & 0 & 1 & 7
      \end{bmatrix}
    \]

    \begin{align*}
      -\frac{1}{2}R_2 \to R_2 \\
      \begin{bmatrix}
        1 & 6 & 0 & 3  & 0 & 0 \\
        0 & 0 & 1 & -4 & 0 & 5 \\
        0 & 0 & 0 & 0  & 1 & 7
      \end{bmatrix}
    \end{align*}

    \begin{align*}
      x_1 + 6x_2  +3x_4 = 0 \\
      x_3 - 4x_4 -  = 5     \\
      x_5 = 7               \\
      \\
      \\
      x_1 = -6x_2 - 3x_4
      x_3 =  5 + 4x_4       \\
      x_5 = 7
    \end{align*}
    $\therefore$
    \begin{align*}
      \begin{cases}
        x_1 = -6x_2 - 3x_4   \\
        x_2 \text{ is free } \\
        x_3 =  5 + 4x_4      \\
        x_4 \text{ is free}  \\
        x_5 = 7
      \end{cases}
    \end{align*}
  }
}

\subsection{Parametric Description of Solution Sets}
\dfn{Parametric Description}{
  A description of the solution set of a system of linear equations in terms of a free variable.
}

The descriptions given so far are all parametric, i.e., the free variables act as parameters that determine the values
of the basic variables.

\subsection{Uniqueness and Existence of Solutions}

\thm{Existence and Uniqueness Theorem}{
  A linear system is consistent if and only if the rightmost column of the augmented matrix is not a pivot column, i.e.,
  if and only if an echelon form of the augmented matrix has no row of the form:
  \[
    \begin{bmatrix} 0 & 0 & \cdots & 0 & b \end{bmatrix}
  \]
  where $b$ is a non-zero number.
  \\
  If a linear system is consistent then the solution set contains either:
  \begin{itemize}
    \item A unique solution, when there are no free variables.
    \item Infinitely many solutions in the presence of at least one free variable.
  \end{itemize}
}

\section{Exercises}


\qs{}{
  Find the general solutions of the systems whose augmented matrices are:
  \begin{enumerate}
    \item
          \[
            \begin{bmatrix}
              1 & 3 & 4 & 7 \\
              3 & 9 & 7 & 6
            \end{bmatrix}
          \]
    \item
          \[
            \begin{bmatrix}
              1  & -7 & 0  & 6  & 5  \\
              0  & 0  & 1  & -2 & -3 \\
              -1 & 7  & -4 & 2  & 7
            \end{bmatrix}
          \]
    \item
          \[
            \begin{bmatrix}
              1 & -3 & 0 & -1 & 0  & -2 \\
              0 & 1  & 0 & 0  & -4 & 1  \\
              0 & 0  & 0 & 1  & 9  & 4  \\
              0 & 0  & 0 & 0  & 0  & 0
            \end{bmatrix}
          \]
  \end{enumerate}
}

\sol{
  \begin{enumerate}
    \item
          \begin{align*}
            \begin{bmatrix}
              1 & 3 & 4 & 7 \\
              3 & 9 & 7 & 6
            \end{bmatrix}
            \\
            \\
            3R_1 - R_2 \to R_2           \\
            \begin{split}
              3 + 9 + 12 = 21 \\
              3 + 9 + 7 = 6   \\
              \hline
              0 + 0 + 5 = 15  \\
            \end{split}              \\
            \begin{bmatrix}
              1 & 3 & 4 & 7  \\
              0 & 0 & 5 & 15
            \end{bmatrix}               \\
            \\
            \frac{4}{5}R_2 - R_1 \to R_1 \\
            \begin{split}
              0 + 0 + 4  = 12 \\
              1 + 3 + 4  = 7  \\
              \hline
              -1 -3 + 0 = 5
            \end{split}              \\
            \begin{bmatrix}
              -1 & -3 & 0 & 5  \\
              0  & 0  & 5 & 15
            \end{bmatrix}             \\
            \\
            \frac{1}{5}R_2               \\
            \begin{bmatrix}
              -1 & -3 & 0 & 5 \\
              0  & 0  & 1 & 3
            \end{bmatrix}              \\
            \\
            -R_1                         \\
            \begin{bmatrix}
              1 & 3 & 0 & -5 \\
              0 & 0 & 1 & 3
            \end{bmatrix}               \\
            \\
            x_1 + 3x_2 = -5              \\
            x_3 = 3                      \\
            \\
            \begin{cases}
              x_1 = -5 -3x_2      \\
              x_2 \text{ is free} \\
              x_3 = 3             \\
            \end{cases}
          \end{align*}
    \item
          \begin{align*}
            \begin{bmatrix}
              1  & -7 & 0  & 6  & 5  \\
              0  & 0  & 1  & -2 & -3 \\
              -1 & 7  & -4 & 2  & 7
            \end{bmatrix}       \\
            \\
            R_2 \leftrightarrow R_3      \\
            \begin{bmatrix}
              1  & -7 & 0  & 6  & 5  \\
              -1 & 7  & -4 & 2  & 7  \\
              0  & 0  & 1  & -2 & -3 \\
            \end{bmatrix}       \\
            \\
            R_1 + R_2 \to R_2            \\
            \begin{split}
              1 - 7 + 0 + 6 = 5  \\
              -1 + 7 - 4 + 2 = 7 \\
              \hline
              0 + 0 - 4+ 8 = 12  \\
            \end{split}           \\
            \begin{bmatrix}
              1 & -7 & 0  & 6  & 5  \\
              0 & 0  & -4 & 8  & 12 \\
              0 & 0  & 1  & -2 & -3 \\
            \end{bmatrix}        \\
            \\
            \frac{1}{4}R_2 + R_3 \to R_3 \\
            \begin{split}
              0 + 0 -1 + 2 = 3   \\
              0 + 0 + 1 - 2 = -3 \\
              \hline
              0 + 0 + 0 + 0 = 0  \\
            \end{split}           \\
            \begin{bmatrix}
              1 & -7 & 0  & 6 & 5  \\
              0 & 0  & -4 & 8 & 12 \\
              0 & 0  & 0  & 0 & 0  \\
            \end{bmatrix}         \\
            \\
            -\frac{1}{4}R_2 \to R_2      \\
            \begin{bmatrix}
              1 & -7 & 0 & 6  & 5  \\
              0 & 0  & 1 & -2 & -3 \\
              0 & 0  & 0 & 0  & 0  \\
            \end{bmatrix}         \\
            \\
            x_1 - 7x_2 + 6x_4 = 5        \\
            x_3 - 2x_4 = -3              \\
            0 = 0                        \\
            \\
            x_1 = 5 - 7x_2 + 6x_4        \\
            x_3 = -3 + 2x_4              \\
            \\
            \begin{cases}
              x_1 = 5 - 7x_2 + 6x_4 \\
              x_2 \text{ is free }  \\
              x_3 = -3 + 2x_4       \\
              x_4 \text{ is free }
            \end{cases}
          \end{align*}
  \end{enumerate}
}

\qs{}{
  Find $h$ and $k$ such that the system has:
  \begin{enumerate}
    \item No solutions
    \item A unique solution
    \item Many solutions
  \end{enumerate}

  \begin{enumerate}
    \item
          \begin{align*}
            x_1 + hx_2 = 2 \\
            4x_1 + 8x_2 = k
          \end{align*}
    \item
          \begin{align*}
            x_1 + 3x_2 = 2 \\
            3x_1 + hx_2 = k
          \end{align*}
  \end{enumerate}
}

\sol{
  \begin{enumerate}
    \item
          \begin{align*}
            x_1 + hx_2 = 2 \\
            4x_1 + 8x_2 = k
          \end{align*}
          \begin{enumerate}
            \item
                  \begin{align*}
                    \begin{bmatrix}
                      1 & h & 2 \\
                      4 & 8 & k
                    \end{bmatrix}                                    \\
                    \\
                    -4R_1 + R_2 \to R_2                               \\
                    \begin{split}
                      -4 - 4h = 8 \\
                      4 + 8 = k   \\
                      0 + \left( -4h + 8 \right) = \left( -8 + k \right)
                    \end{split} \\
                    \begin{bmatrix}
                      1 & h                      & 2                    \\
                      0 & \left( -4h + 8 \right) & \left( -8 +k \right)
                    \end{bmatrix} \\
                    \\
                    -4h + 8 = 0                                       \\
                    h = 2                                             \\
                    \\
                    -8 + k \neq 0                                     \\
                    k \neq 8
                  \end{align*}
                  $\therefore$ $h = 2$ and $k \neq  8$
            \item
                  \begin{align*}
                    \begin{bmatrix}
                      1 & h & 2 \\
                      4 & 8 & k
                    \end{bmatrix}                                    \\
                    -4R_1 + R_2 \to R_2                               \\
                    \begin{split}
                      -4 - 4h = 8 \\
                      4 + 8 = k   \\
                      0 + \left( -4h + 8 \right) = \left( -8 + k \right)
                    \end{split} \\
                    \begin{bmatrix}
                      1 & h                      & 2                    \\
                      0 & \left( -4h + 8 \right) & \left( -8 +k \right)
                    \end{bmatrix} \\
                    \\
                    h \neq 2
                  \end{align*}
                  $\therefore$ $h \neq 2$
            \item
                  \begin{align*}
                    \begin{bmatrix}
                      1 & h & 2 \\
                      4 & 8 & k
                    \end{bmatrix}                                    \\
                    -4R_1 + R_2 \to R_2                               \\
                    \begin{split}
                      -4 - 4h = 8 \\
                      4 + 8 = k   \\
                      0 + \left( -4h + 8 \right) = \left( -8 + k \right)
                    \end{split} \\
                    \begin{bmatrix}
                      1 & h                      & 2                    \\
                      0 & \left( -4h + 8 \right) & \left( -8 +k \right)
                    \end{bmatrix} \\
                    -4h + 8 = 0                                       \\
                    h = 2                                             \\
                    \\
                    -8 + k = 0                                        \\
                    k = 8
                  \end{align*}
                  $\therefore$ $h = 2$ and $k = 8$

          \end{enumerate}
  \end{enumerate}
}




\end{document}
