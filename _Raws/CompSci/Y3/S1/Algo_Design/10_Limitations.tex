\documentclass[12pt letter]{report}
\input{./template/preamble}
\input{./template/macros}
\input{./template/letterfonts}

\title{\Huge{Limitations of Algorithm Power}}
\author{\huge{Madiba Hudson-Quansah}}
\date{}
\usepackage{parskip}

\setcounter{tocdepth}{4}
\setcounter{secnumdepth}{4}

\begin{document}
\maketitle
\newpage
\pdfbookmark[section]{\contentsname}{too}
\tableofcontents
\pagebreak

\chapter{Introduction}

\chapter{ $P$, $NP$ and $N P$-Complete Problems}

\dfn{Decision Problem}{
  A problem that requires a yes or no answer.
}

\section{ $P$ and $N P$ Problems}
\dfn{Polynomial Time}{
  A problem is said to be in $P$ if there exists an algorithm that can solve the problem in polynomial time in the worst
  case, i.e the
  time complexity of the algorithm is of the form $O \left( n^k \right) $ for some constant $k$. This class of problems
  is also called \textbf{tractable} problems.
}

\dfn{ $P$ Problems}{
  A class of problems that can be solved in polynomial time by deterministic (non-random) algorithms. Also called
  \textbf{Polynomial Time} problems.
}

\subsection{Non-deterministic Algorithms}
\dfn{ Non-deterministic Algorithm}{
  An algorithm that takes in as input an instance $I$ of a decision problem and does the following
  \begin{description}
    \item[Guess] Generates an arbitrary string $S$ that can be thought of as a candidate solution to the given
          instance $I$.
    \item[Verification] Using a deterministic algorithm that takes in the candidate solution $S$ and the problem
          instance $I$ as input and outputs a boolean value indicating whether $S$ actually represents a solution to
          the problem instance $I$.
  \end{description}
}
A Non-deterministic algorithm is said to solve a decision problem if and only if for every instance $I$ and candidate
solution $S$ it returns the correct answer, and never returns the wrong answer or a false positive, i.e. for a no
instance it returns a no answer and for a yes instance it returns a yes answer.

\subsection{ $NP$ Problems}
\dfn{ $NP$ Problems}{
  A class of decision problems that can be solved by a non-deterministic algorithm in polynomial time. The $NP$ stands
  for Non-deterministic Polynomial time.
}

Most decision problems are in $NP$:
\[
  P \subseteq NP
\]
This is true because if a problem is in $P$, we can use the deterministic polynomial time algorithm that solves it in
the verification step of the non-deterministic algorithm bypassing the guessing step and as this algorithm used in the
verification step is in polynomial time and is the only step in the non-deterministic algorithm, the entire
non-deterministic algorithm for the problem instance $I$ is in polynomial time. \\

$N P$ also contains problems that are not in $P$:
\begin{itemize}
  \item Hamiltonian Circuit Problem
  \item Knapsack Problem
  \item Travelling Salesman Problem
\end{itemize}

This leads to the question:
\[
  P \stackrel{?}{=} NP
\]
I.e. are all problems in $N P$ also in $P$ or is $P$ a proper subset of $NP$?

\section{ $N P$-Complete Problems}

\dfn{Polynomially Reducible}{
  A decision problem $D_1$ is said to be polynomially reducible to a decision problem $D_2$, if there exists a
  function $t$ that transforms instances of $D_1$ to instances of $D_2$ such that:
  \begin{enumerate}
    \item $t$ maps all yes instances of $D_1$ to yes instances of $D_2$, and all no instances of $D_1$ to no
          instances of $D_2$.
    \item $t$ is computable in polynomial time.
  \end{enumerate}
}

\dfn{ $NP$ Complete}{
  A decision problem $D$ is $N P$-Complete if:
  \begin{enumerate}
    \item It belongs to the class $NP$
    \item Every problem in $N P$ is polynomially reducible to $D$.
  \end{enumerate}
}




\end{document}
