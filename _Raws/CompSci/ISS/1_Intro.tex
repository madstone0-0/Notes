\documentclass[12pt letter]{report}
\input{./template/preamble}
\input{./template/macros}
\input{./template/letterfonts}

\title{\Huge{Introduction}}
\author{\huge{Madiba Hudson-Quansah}}
\date{}
\usepackage{parskip}

\setcounter{tocdepth}{4}
\setcounter{secnumdepth}{4}

\begin{document}
\maketitle
\newpage
\pdfbookmark[section]{\contentsname}{too}
\tableofcontents
\pagebreak

\chapter{Introduction}

\dfn{Security}{
  A state of being secure and free from danger or harm, i.e. the
  actions taken to make someone or something secure.
  Protection from adversaries (those who would do harm),
  intentionally or otherwise, is the goal of security.
}

\section{Key Objectives of Security}
\dfn{Repudiation}{
  The ability to deny the authenticity of one's signature on a document
  or a message that one originated.
}

\begin{description}
  \item[Confidentiality] - Covers data confidentiality and privacy.
    \begin{description}
      \item[Data Confidentiality]  - Information is not made
        available or disclosed to unauthorized individuals, entities,
        or processes.
      \item[Privacy] - Assures that individuals control or influence
        what information related to them may be collected and stored,
        and by whom and whom that information may be disclosed.
    \end{description}
  \item[Integrity] - Data Integrity and  System Integrity.
    \begin{description}
      \item[Data Integrity]  - The property that data has not been
        changed, destroyed, or lost in an unauthorized or accidental manner.
      \item[System Integrity] - Assures  that a system performs its
        intended function in an unimpaired manner, free from
        deliberate or inadvertent unauthorized manipulation of the system
    \end{description}
  \item[Availability] - Assures that systems work promptly and
    service is not denied to authorized users.
  \item[Authenticity] - The property of being genuine and being able
    to be verified and trusted. Ensures users are who they say they
    are and that each input arriving at the system came from a trusted source.
  \item[Accountability] - Assures that the actions of an entity can
    be traced uniquely to that entity. Supports non-repudiation,
    deterrence, fault isolation, intrusion detection, and
    after-the-fact forensic analysis.
\end{description}

The CIA triad (Confidentiality, Integrity, and Availability) is a
widely used model for understanding and implementing security measures.
Additionally Accountability and Authenticity are also important objectives
in security, making the complete set often referred to as the CIAAA
triad.

\chapter{Terminology}

\section{Asset}
\dfn{Asset / System Resource}{
  Anything that needs to be protected because it has value to an
  organization, both tangible and intangible.
}

There are general asset categories:
\begin{description}
  \item[Hardware]  - Physical devices that support computing and networking
  \item[Software] - Programs and system components that run on hardware
  \item[Data]- Information that organizations store, process and
    transmit. Generally the most important asset.
  \item[Communication facilities and Networks] - Infrastructure that
    enables data transfer and connectivity.
  \item[People] - Human actors who interact with or affect information systems.
\end{description}

\section{Vulnerability and Threats}
\dfn{Vulnerability}{
  A weakness in an asset that could be exploited or triggered by a
  threat source.
}

Examples of vulnerabilities include:
\begin{itemize}
  \item Unpatched applications or operating systems
  \item An unrestricted wireless access point.
  \item An open port on a firewall.
\end{itemize}

\dfn{Threat}{
  Any circumstance or event with the potential to adversely impact
  organizational operations, assets, or the nation through an
  information system via unauthorized access, destruction,
  disclosure, modification of information , and/or denial of service.
}

\dfn{Threat Agent}{
  An entity that can cause harm to an asset. Represents the source of
  a threat. Can be forces of nature, as well as human or non-human actors.
}

\dfn{Adversary}{
  An individual, group, organization or government that conducts or
  has the intent to conduct detrimental activates. Considered a human
  threat agent or attacker.
}

\dfn{Attack}{
  Any kind of malicious activity that attempts to collect, disrupt,
  deny, degrade or destroy information system resources or the
  information itself.
}

\section{Security Controls and Countermeasures}

\dfn{Security Control / Safeguard}{
  Any measure or mechanism put in place to protect the confidentiality,
  integrity, and availability of an information system. For example,
  policies, procedures, technical measures, and physical protections.
}

\dfn{Countermeasure}{
  A device or technique with the objective of the impairment of the
  operation effectiveness of adversarial activities. All
  countermeasures are security controls,
  but not all security controls are countermeasures. For example
  deploying an antivirus in response to a data breach.
}

Some security control categories are:
\begin{description}
  \item[Physical Controls]
    \begin{itemize}
      \item Locks
      \item Walls / Fences
      \item Guards
    \end{itemize}
  \item[Prodedural / Administrative Controls]
    \begin{itemize}
      \item Laws
      \item Regulations
      \item Policies
    \end{itemize}
  \item[Techinical]
    \begin{itemize}
      \item Passwords
      \item Access Controls
      \item Firewalls
    \end{itemize}
\end{description}

\section{Risk and Risk Management}

\dfn{Risk}{
  A measure of the extent to which an entity it threatened by a
  potential event. A function of
  \begin{itemize}
    \item The adverse impacts that would arise if the circumstance or
      event occurs
    \item The likelihood of occurrence.
  \end{itemize}
}

\end{document}
