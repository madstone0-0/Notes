\documentclass[12pt letter]{report}
\input{./template/preamble}
\input{./template/macros}
\input{./template/letterfonts}

\title{\Huge{Producing Data}}
\author{\huge{Madiba Hudson-Quansah}}
\date{}
\usepackage{parskip}

\begin{document}
\maketitle
\newpage
\pdfbookmark[section]{\contentsname}{too}
\tableofcontents
\pagebreak

\chapter{Introduction}

\dfn{Sample}{
	A subset of a larger population
}

In order to reliably draw conclusions about a population, we must ensure that our chosen sample is truly representative
of the population.

\dfn{Sampling Bias}{
	A sample that is not representative of a population due to systematic under or over estimation of the values of the
	variable of interest.
}

\dfn{A Good Sample}{
	A sample that is the right size, represents the population adequately, and is selected in a way that is free of bias
	i.e randomly.
}

\nt{Optimal sample size is greater than 30}

\chapter{Module 6: Sampling}

\dfn{Sampling Frame}{
	A list of individuals from which the sample is drawn
}

\section{Sampling Methods}

\subsection{Simple Random Sampling}

\dfn{Simple Random Sampling}{
	The simplest probability sampling method. Each member of the population has an equal chance of being selected.
}

\subsection{Systematic Sampling}
\dfn{Systematic Sampling}{
	Used when the population is naturally ordered. A starting point is randomly selected and then every $k^{\text{th}}$ member
	of the population is selected.
}

\subsection{Cluster Sampling}

\dfn{Cluster Sampling}{
	Used when the population is naturally divided into groups / clusters. A simple random / systematic sample of clusters is selected
	and we use all the individuals in the selected clusters as our sample.
}

\subsection{Stratified Sampling}

\dfn{Stratified Sampling}{
	Used when the population is naturally divide into sub populations / strata, e.g. gender. A simple random / systematic sample is
	selected from each stratum and the results are combined to form the sample.
}


\chapter{Module 7: Designing Studies}


\section{Introduction}

\dfn{Observational Study}{
	The variable(s) of interest are recorded as they naturally occur, with no interference from the researchers
	conducting the study
}

\dfn{Sample Survey}{
	A type of observational study where subjects report variables' values themselves, frequently giving their opinions.
}

\dfn{Experiment}{
	The researchers interfere with the variable(s) of interest, assigning the values of the independent variable to the
	subjects.
}

\section{Observational Study}

There are two types of observational studies:
\begin{itemize}
	\item Prospective - A study that follows subjects into the future
	\item Retrospective - A study that looks at data from the past
\end{itemize}

\section{Causation and Observational Studies}

As a result of the nature of observational studies they come with one glaring weakness: differing factors among subjects
could be lurking variables, affecting the results of the study. To contend with this observational studies often have to
control of the effects of lurking variables by separating each study group by similar possible lurking variables. But
even with this attempt at control it is impossible to account for a near unlimited amount of potential lurking
variables, so the claims of a causation based on an observational study can never be 100\% accurate.



\section{Causation and Experiments}

\dfn{Factor}{
	A variable whose levels are controlled by the experimenter / Independent variable
}

\dfn{Treatments / $ttt$}{
	Different imposed values of the factor
}

\dfn{Control Group}{
	A group of subjects that are not given any treatment
}

\dfn{Subjects}{
	Individuals on which the experiment is performed
}

\dfn{Treatment Group}{
	A group of subjects that are given a treatment
}

\subsection{Randomized Controlled Experiments}

\dfn{Randomized Controlled Experiment}{
	An experiment where subjects are randomly assigned to treatment groups
}

Under a randomized controlled experiment, the only difference between the treatment and control groups should be the
treatment itself. This allows us to draw conclusions about the effects of the treatment on the subjects.

\subsection{Blind and Double-Blind Experiments}

\dfn{Blind Experiment}{
	An experiment where the subjects do not know which treatment they are receiving
}

\dfn{Double-Blind Experiment}{
	An experiment where neither the subjects nor the researchers know which treatment the subjects are receiving
}

\subsection{Modifications to Randomization}

In some cases the design of an experiment may be enhanced by relaxing the requirement of total randomization and
blocking the subjects first.

\dfn{Block}{
	A group of subjects that are similar in ways that are expected to affect the response to the treatments
}

\dfn{Matched Pairs}{
	A type of blocking where each subject receives both treatments in a random order
}





\end{document}
