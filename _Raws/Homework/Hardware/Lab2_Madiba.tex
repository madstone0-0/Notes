\documentclass[12pt letter]{report}
\input{../template/preamble}
\input{../template/macros}
\input{../template/letterfonts}

\usepackage{parskip}
\title{\Huge{Lab 2}}
\author{\huge{Madiba Hudson-Quansah}}
\date{}


\begin{document}
\maketitle
\newpage

\section*{Part 1}
\qs{}{
  Draw the truth table for the above logical equations for a full adder.
}

\sol{
  \begin{table}[h!]
    \begin{center}
      \begin{tabular}{|c c c|c|c|}
        \hline
        $a$ & $b$ & cin & $ \left( a \oplus b \right) \oplus \text{cin} $ & $a\cdot b + b \cdot \text{cin} + a
        \cdot \text{cin}$                                                                                      \\ [0.5ex]
        \hline
        \hline
        0   & 0   & 0   & 0                                               & 0                                  \\
        0   & 0   & 1   & 1                                               & 0                                  \\
        0   & 1   & 1   & 0                                               & 1                                  \\
        1   & 0   & 1   & 0                                               & 1                                  \\
        1   & 1   & 1   & 1                                               & 1                                  \\
        1   & 1   & 0   & 0                                               & 1                                  \\
        1   & 0   & 0   & 1                                               & 0                                  \\
        0   & 1   & 0   & 1                                               & 1                                  \\
        \hline
      \end{tabular}
    \end{center}
  \end{table}
}

\section*{Part 2}

\qs{}{
  Compute the additions for the following as 2-bit additions. Indicate their Sum and Carry (overflow).
  \begin{enumerate}
    \item 00 + 11
    \item 10 + 01
    \item 11 + 11
    \item 01 + 11
  \end{enumerate}
}

\sol{
  \begin{enumerate}
    \item
          \begin{align*}
            \begin{array}{c c c}
                & 0 & 0 \\
              + & 1 & 1 \\
              \hline
                & 1 & 1
            \end{array} \\
            \text{Carry: 0}
          \end{align*}
    \item
          \begin{align*}
            \begin{array}{c c c}
                & 1 & 0 \\
              + & 0 & 1 \\
              \hline
                & 1 & 1
            \end{array} \\
            \text{Carry: 0}
          \end{align*}
    \item
          \begin{align*}
            \begin{array}{c c c}
                & 1 & 1 \\
              + & 1 & 1 \\
              \hline
                & 1 & 0
            \end{array} \\
            \text{Carry: 1}
          \end{align*}
    \item
          \begin{align*}
            \begin{array}{c c c}
                & 0 & 1 \\
              + & 1 & 1 \\
              \hline
                & 0 & 0
            \end{array} \\
            \text{Carry: 1}
          \end{align*}

  \end{enumerate}
}

\section*{Part 3}

\qs{}{
  Compute the arithmetic for the following as 2-bit subtractions using the 2’s complement approach. Indicate their Sum and Carry (overflow).
  \begin{enumerate}
    \item 00 - 11
    \item 10 - 01
    \item 11 - 11
    \item 01 - 11
  \end{enumerate}
}

\sol{
  \begin{enumerate}
    \item
          \begin{align*}
            \begin{array}{c c c}
                & 0 & 0                         \\
              - & 1 & 1                         \\
              \hline
                & 0 & 0                         \\
              + & 0 & 1 \tag*{Two's Complement} \\
              \hline
                & 0 & 1
            \end{array} \\
            \text{Carry: 0}
          \end{align*}
    \item
          \begin{align*}
            \begin{array}{c c c}
                & 1 & 0                         \\
              - & 0 & 1                         \\
              \hline
                & 1 & 0                         \\
              + & 1 & 1 \tag*{Two's Complement} \\
              \hline
                & 0 & 1
            \end{array} \\
            \text{Carry: 1}
          \end{align*}
    \item
          \begin{align*}
            \begin{array}{c c c}
                & 1 & 1                         \\
              - & 1 & 1                         \\
              \hline
                & 1 & 1                         \\
              + & 0 & 1 \tag*{Two's Complement} \\
              \hline
                & 0 & 0
            \end{array} \\
            \text{Carry: 1}
          \end{align*}
    \item
          \begin{align*}
            \begin{array}{c c c}
                & 0 & 1                         \\
              - & 1 & 1                         \\
              \hline
                & 0 & 1                         \\
              + & 0 & 1 \tag*{Two's Complement} \\
              \hline
                & 1 & 0
            \end{array} \\
            \text{Carry: 0}
          \end{align*}
  \end{enumerate}
}

\section*{Part 4}

\qs{}{
  Compute the additions for the following as 8-bit additions. Indicate their Sum and Carry.
  \begin{enumerate}
    \item 1010110 + 110100
    \item 11001011 + 1011010
    \item 110001 + 11100100
    \item 1010001 + 11001
  \end{enumerate}
}

\sol{
  \begin{enumerate}
    \item
          \begin{align*}
            \begin{array}{c c c c c c c c c c}
                & 0 & 1 & 0 & 1 & 0 & 1 & 1 & 0 \\
              + & 0 & 0 & 1 & 1 & 0 & 1 & 0 & 0 \\
              \hline
                & 1 & 0 & 0 & 0 & 1 & 0 & 1 & 0 \\
            \end{array} \\
            \text{Carry: 0}
          \end{align*}

    \item
          \begin{align*}
            \begin{array}{c c c c c c c c c c}
                & 1 & 1 & 0 & 0 & 1 & 0 & 1 & 1 \\
              + & 0 & 1 & 0 & 1 & 1 & 0 & 1 & 0 \\
              \hline
                & 0 & 0 & 1 & 0 & 0 & 1 & 0 & 1 \\
            \end{array} \\
            \text{Carry: 1}
          \end{align*}

    \item
          \begin{align*}
            \begin{array}{c c c c c c c c c c}
                & 0 & 0 & 1 & 1 & 0 & 0 & 0 & 1 \\
              + & 1 & 1 & 1 & 0 & 0 & 1 & 0 & 0 \\
              \hline
                & 0 & 0 & 0 & 1 & 0 & 1 & 0 & 1 \\
            \end{array} \\
            \text{Carry: 1}
          \end{align*}

    \item
          \begin{align*}
            \begin{array}{c c c c c c c c c c}
                & 0 & 1 & 0 & 1 & 0 & 0 & 0 & 1 \\
              + & 0 & 0 & 0 & 1 & 1 & 0 & 0 & 1 \\
              \hline
                & 0 & 1 & 1 & 0 & 1 & 0 & 1 & 0 \\
            \end{array} \\
            \text{Carry: 0}
          \end{align*}
  \end{enumerate}
}

\section*{Part 5}

\qs{}{
  Compute the arithmetic for the following as 8-bit subtractions using the 2’s complement approach. Indicate their Sum and Carry.
  \begin{enumerate}
    \item FF – E6
    \item A5 – 6F
    \item F2 – D7
    \item 110001 - 11100100
    \item 1010001 - 11001
  \end{enumerate}
}

\sol{
  \begin{enumerate}
    \item
          \begin{align*}
            \begin{array}{c c c c c c c c c}
                & F & F                                                 \\
              - & E & 6                                                 \\
              \hline
                & 1 & 1 & 1 & 1 & 1 & 1 & 1 & 1                         \\
              + & 0 & 0 & 0 & 1 & 1 & 0 & 1 & 0 \tag*{Two's Complement} \\
              \hline
                & 0 & 0 & 0 & 1 & 1 & 0 & 0 & 1                         \\
              \hline
                & 1 & 9                                                 \\
            \end{array} \\
            \text{Carry: 1}
          \end{align*}
    \item
          \begin{align*}
            \begin{array}{c c c c c c c c c}
                & A & 5 \\
              - & 6 & F \\
              \hline
                & 3 & 6 \\
            \end{array} \\
            \text{Carry: 1}
          \end{align*}
    \item
          \begin{align*}
            \begin{array}{c c c c c c c c c}
                & F & 2 \\
              - & D & 7 \\
              \hline
                & 1 & B \\
            \end{array} \\
            \text{Carry: 1}
          \end{align*}
    \item
          \begin{align*}
            \begin{array}{c c c c c c c c c}
                & 0 & 0 & 1 & 1 & 0 & 0 & 0 & 1                         \\
              - & 1 & 1 & 1 & 0 & 0 & 1 & 0 & 0                         \\
              \hline
                & 0 & 0 & 1 & 1 & 0 & 0 & 0 & 1                         \\
              + & 0 & 0 & 0 & 1 & 1 & 1 & 0 & 0 \tag*{Two's Complement} \\
              \hline
                & 0 & 1 & 0 & 0 & 1 & 1 & 0 & 1
            \end{array} \\
            \text{Carry: 0}
          \end{align*}
    \item
          \begin{align*}
            \begin{array}{c c c c c c c c c}
                & 0 & 1 & 0 & 1 & 0 & 0 & 0 & 1                         \\
              - & 0 & 0 & 0 & 1 & 1 & 0 & 0 & 1                         \\
              \hline
                & 0 & 1 & 0 & 1 & 0 & 0 & 0 & 1                         \\
              + & 1 & 1 & 1 & 0 & 0 & 1 & 1 & 1 \tag*{Two's Complement} \\
              \hline
                & 0 & 0 & 1 & 1 & 1 & 0 & 0 & 0
            \end{array} \\
            \text{Carry: 1}
          \end{align*}
  \end{enumerate}
}

\section*{Part 6}

\qs{}{
  How does the digital logic of a full adder circuit differ from that of a half adder?
}

\sol{
  A full adder includes a carry-in bit, but a half adder does not.
}

\qs{}{
  What role does the XOR gate play in the implementation of a full adder circuit?
}

\sol{
  The XOR gate is used to calculate the sum component of the full adder using the formula $ \left( a \oplus b \right)  \oplus \text{cin}$.
}

\qs{}{
  How does the carry-out bit in a full adder circuit influence subsequent addition operations?
}

\sol{
  The carry-out bit is used as the carry-in bit for the next full adder in a series of full adders. This allows for the
  addition of numbers with more than one bit, pushing the carry bit to the next full adder in the series.
}

\qs{}{
  Can a full adder circuit be used to perform subtraction? If so, how?
}

\sol{
  Yes. To do this the two's complement of the subtrahend is calculated and added to the minuend. This complement can be
  done by inverting the bits and adding 1 to the least significant bit, which can be done using a NOT gate or an XOR for
  conditional inversion.
}

\qs{}{
  What is the significance of the XOR and AND gates in a subtractor circuit?
}

\sol{
  In a purely subtractor circuit, the XOR gate is used to calculate the difference between the minuend and the
  subtrahend and the AND gate is used to calculate the borrow bit.
}

\end{document}
