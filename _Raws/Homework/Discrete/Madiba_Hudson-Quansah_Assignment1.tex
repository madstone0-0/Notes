\documentclass[12pt letter]{report}
\input{../template/preamble}
\input{../template/macros}
\input{../template/letterfonts}

\usepackage{parskip}
\usepackage{enumitem}
\title{\Huge{Assignment 1}}
\author{\huge{Madiba Hudson-Quansah}}
\date{January 2024}


\begin{document}
\maketitle
\newpage

\qs{}{
	\begin{align}
		p \to \neg p \\
		\left( p \leftrightarrow q  \right)  \oplus  \left( q \to \neg p  \right)
	\end{align}

}

\sol{
	\noindent\begin{enumerate}
		\item
		      \begin{table}[h!]
			      \begin{center}
				      \begin{tabular}{|c c|c|}
					      \hline
					      $p$ & $\neg p$ & $p \to \neg p$ \\ [0.5ex]
					      \hline
					      \hline
					      $T$ & $F$      & $F$            \\
					      $F$ & $T$      & $T$            \\
					      \hline
				      \end{tabular}
			      \end{center}
		      \end{table}

		\item
		      \begin{table}[h!]
			      \begin{center}
				      \begin{tabular}{|c c c|c|c|c|}
					      \hline
					      $q$ & $p$ & $\neg p$ & $p \leftrightarrow q$ & $q \to \neg p$ & $\left( p \leftrightarrow q  \right)  \oplus  \left( q \to \neg p  \right)$ \\ [0.5ex]
					      \hline
					      \hline
					      $T$ & $T$ & $F$      & $T$                   & $F$            & $T$                                                                         \\
					      $T$ & $F$ & $T$      & $F$                   & $T$            & $T$                                                                         \\
					      $F$ & $T$ & $F$      & $F$                   & $T$            & $T$                                                                         \\
					      $F$ & $F$ & $T$      & $T$                   & $T$            & $F$                                                                         \\
					      \hline
				      \end{tabular}
			      \end{center}
		      \end{table}

	\end{enumerate}
}

\qs{}{
	\begin{enumerate}
		\item
		      Willy gets caught whenever he cheats.
		\item A positive integer is a prime only if it has no divisors other than 1 and itself
		\item To be a citizen of this country, it is sufficient that you were born in the United States.
		\item Getting elected follows from knowing the right people.
		\item For you to get an A in this course, it is necessary that you learn how to solve discrete mathematics problems.
		\item Abena will go to Accra unless she misses the bus.
	\end{enumerate}
}

\sol{
	\begin{enumerate}
		\item If Willy cheats, then he will get caught.
		\item If a positive number is a prime, then it has no divisors other than 1 and itself.
		\item If you were born in the United States, then you can be a citizen of this country
		\item If you know the right people, then you can get elected.
		\item If you are to get an A in this course, then you must learn to solve discrete mathematics problems.
		\item If Abena does not miss the bus, then she will go to Accra.
	\end{enumerate}
}

\qs{}{
	\begin{enumerate}
		\item  If 1 +  1 =  3, then unicorns exist.
		\item If 1 +  1 =  2, then dogs can fly.
		\item 0 >  1 if and only if 2 >  1.
		\item 1 +  1 =  3 if and only if 2 < 1.
	\end{enumerate}
}

\sol{
	\begin{enumerate}
		\item True. This is because the premise is false rendering the whole statement true.
		\item False. This is because the premise is true but the conclusion is false rendering the whole statement
		      false.
		\item False. This is because this is a biconditional statement which requires both propositions to
		      have the same truth value, but here the first proposition is false but the second is true.
		\item True. This is because this is a biconditional statement which requires both propositions to have the same
		      truth value, and here both propositions have the same truth value of False.
	\end{enumerate}
}


\end{document}
