\documentclass[12pt letter]{report}
\input{../template/preamble}
\input{../template/macros}
\input{../template/letterfonts}

\usepackage{parskip}
\title{\Huge{Homework 2}}
\author{\huge{Madiba Hudson-Quansah}}
\date{}


\begin{document}
\maketitle
\newpage

\qs{}{
  What are the unsigned and signed decimal values of the following binary and hexadecimal numbers?
  \begin{enumerate}
    \item 10110110
    \item C1B3
  \end{enumerate}
}

\sol{
  \begin{enumerate}
    \item
          Unsigned
          \begin{align*}
            10110110 & = 1 \times 2^{7} + 1 \times 2^{5} + 1 \times 2^{4} + 1 \times 2^{2}  + 1 \times 2^{1} \\
                     & =  182                                                                                \\
          \end{align*}
          Signed
          \begin{align*}
            10110110 & = 01001001 + 00000001                             \\
                     & = 01001010                                        \\
                     & = 1\times 2^{6} + 1 \times 2^{3} + 1 \times 2^{1} \\
                     & = -74                                             \\
          \end{align*}
    \item Unsigned
          \begin{align*}
            C1B3 & = 12 \times 16^{3} + 1 \times 16^{2} + 11 \times 16^{1} + 3 \times 16^{0} \\
                 & = 49587                                                                   \\
          \end{align*}
          Signed
          \begin{align*}
            C1B3 & =  1100 0001 1011 0011    \\
                 & = 0011 1110 0100 1100 + 1 \\
                 & = 0011 1110 0100 1111     \\
                 & = 3D4F                    \\
          \end{align*}
  \end{enumerate}
}

\qs{}{
  Carry out the following additions. Indicate whether there is a carry or overflow.
  \begin{enumerate}
    \item 11010010 (binary) + 10111101 (binary)
    \item A1CF (hexadecimal) + B2D3 (hexadecimal)
  \end{enumerate}
}

\sol{
  \begin{enumerate}
    \item
          \begin{alignat*}{9}
             &  & 11010010                \\
             &  & \underline{+ 10111101 } \\
             &  & 110001111
          \end{alignat*}
          Overflow: yes as the result is 9 bits \\
          Carry: yes as there is a carry out of the MSB
          \begin{enumerate}
            \item
                  \begin{alignat*}{17}
                     &  & 1010 0001 1100 1111               \\
                     &  & \underline{+ 1011 0010 1101 0011} \\
                     &  & 101101010010100010
                  \end{alignat*}
          \end{enumerate}
          Overflow: yes as the result is 17 bits \\
          Carry: yes as there is a carry out of the MSB
  \end{enumerate}
}

\qs{}{
  Carry out the following subtractions. Indicate whether there is a borrow or overflow.
  \begin{enumerate}
    \item 11010010 (binary) - 10111101 (binary)
    \item 71CF (hexadecimal) - B2D3 (hexadecimal)
  \end{enumerate}
}

\sol{
  \begin{enumerate}
    \item
          \begin{alignat*}{8}
             &  & 11010010                \\
             &  & \underline{- 10111101 } \\
             &  & 11010010                \\
             &  & \underline{+ 01000011}  \\
             &  & 00010101
          \end{alignat*}
          Borrow: yes as there is a borrow out of the MSB \\
          Overflow: No as the result fits in 8 bits
    \item
          \begin{alignat*}{5}
             &  & 71CF                \\
             &  & \underline{- B2D3}  \\
             &  & 29135               \\
             &  & \underline{- 45795} \\
             &  & -16660              \\
             &  & -4104               \\
          \end{alignat*}
  \end{enumerate}
}
\qs{}{
  What is the decimal value of the following single-precision floating-point numbers?
  \begin{enumerate}
    \item 1 01011010 001 0100 0000 0000 0000 0000 (binary)
    \item  0100 0110 1100 1000 0000 0000 0000 0000 (binary)
  \end{enumerate}
}

\sol{
  \begin{enumerate}
    \item
          \begin{align*}
            S & = 1                                    \\
            F & = 1 + 1 \times 2^{-3} + 1\times 2^{-5} \\
            E & = 90 - 127                             \\
              & = -37                                  \\
            D & = (-1)^{S} \times F \times 2^{E}       \\
              & = -1 \times 1.125 \times 2^{-37}       \\
              & = -8.412825991399586e^{-12}            \\
          \end{align*}
    \item
          \begin{align*}
            S & = 0                                   \\
            F & = 1 + 1\times 2^{-1} + 1\times 2^{-4} \\
            E & = 141 - 127                           \\
              & = 14                                  \\
            D & = (-1)^{S} \times F \times 2^{E}      \\
              & = 1 \times 1.5625 \times 2^{14}       \\
              & = 25600.0                             \\
          \end{align*}
  \end{enumerate}

  \qs{}{
    Show the IEEE 754 binary representation for: -95.4 in:
    \begin{enumerate}
      \item Single Precision
      \item Double precision
    \end{enumerate}
  }

  \sol{
    \begin{align*}
      S                    & = 1                                        \\
      F                    & = 0.4 \times 2 = 0.8                       \\
                           & = 0.8 \times 2 = 1.6                       \\
                           & = 0.6 \times 2 = 1.2                       \\
                           & = 0.2 \times 2 = 0.4                       \\
                           & = 0.4 \times 2 = 0.8                       \\
                           & = 0.8 \times 2 = 1.6                       \\
                           & = 0.6 \times 2 = 1.2                       \\
                           & = 0.2 \times 2 = 0.4                       \\
                           & = 0.4 \times 2 = 0.8                       \\
                           & = 0.8 \times 2 = 1.6                       \\
                           & = 0.6 \times 2 = 1.2                       \\
                           & = 0.2 \times 2 = 0.4                       \\
                           & = 0.4 \times 2 = 0.8                       \\
                           & = 0.8 \times 2 = 1.6                       \\
                           & = 0.6 \times 2 = 1.2                       \\
                           & = 0.2 \times 2 = 0.4                       \\
                           & = 0.4 \times 2 = 0.8                       \\
                           & = 0.8 \times 2 = 1.6                       \\
                           & = 0.6 \times 2 = 1.2                       \\
                           & = 0.2 \times 2 = 0.4                       \\
                           & = 0.4 \times 2 = 0.8                       \\
                           & = 0.8 \times 2 = 1.6                       \\
                           & = 0.6 \times 2 = 1.2                       \\
      E                    & = 95 \div 2 = 47 \text{ rem } 1            \\
                           & = 47 \div 2 = 23 \text{ rem } 1            \\
                           & = 23 \div 2 = 11 \text{ rem } 1            \\
                           & = 11 \div 2 = 5 \text{ rem } 1             \\
                           & = 5 \div 2 = 2 \text{ rem } 1              \\
                           & = 2 \div 2 = 1 \text{ rem } 0              \\
                           & = 1 \div 2 = 0 \text{ rem } 1              \\
      \text{Un-normalized} & = 1011111.01100110011001100110011          \\
      \text{Normalized}    & = 1.01111101100110011001101 \tag*{Rounded} \\
    \end{align*}
    \begin{enumerate}
      \item
            \begin{align*}
              E & = 6 + 127 \\
                & = 133     \\
              \underbrace{1}_{\text{Sign}} \quad
              \underbrace{10000101}_{\text{Exponent}} \quad
              \underbrace{01111101100110011001101}_{\text{Fraction}}
            \end{align*}

      \item
            \begin{align*}
              E & = 6 + 1023 \\
                & = 1029     \\
              \underbrace{1}_{\text{Sign}} \quad
              \underbrace{10000000101}_{\text{Exponent}} \quad
              \underbrace{0111110110011001100110011001100110011001100110011001}_{\text{Fraction}}
            \end{align*}
    \end{enumerate}
  }
}

\qs{}{
  Given the following numbers:

  x = 1100 0110 1101 1000 0000 0000 0000 0000 (binary) and \\
  y = 0011 1110 1110 0000 0000 0000 0000 0000 (binary) are single-precision floating-point numbers. \\
  Perform the following operations showing all work:
  \begin{enumerate}
    \item x + y
    \item x * y
  \end{enumerate}
}

\sol{

  \begin{align*}
    S & = 1                                                    \\
    F & = 1 + 1\times 2^{-1} + 1\times 2^{-3} + 1\times 2^{-4} \\
    E & = 141 - 127                                            \\
      & = 14                                                   \\
    D & = (-1)^{S} \times F \times 2^{E}                       \\
      & = -1 \times 1.6875 \times 2^{14}                       \\
      & = -27648.0                                             \\
  \end{align*}
  \begin{align*}
    S & = 0                                   \\
    F & = 1 + 1\times 2^{-1} + 1\times 2^{-2} \\
    E & = 125 - 127                           \\
      & = -2                                  \\
    D & = (-1)^{S} \times F \times 2^{E}      \\
      & = 1 \times 1.75 \times 2^{-2}         \\
      & = 0.4375                              \\
  \end{align*}
  \begin{enumerate}
    \item
          \begin{alignat*}{33}
             &  & 1100 0110 1101 1000 0000 0000 0000 0000               \\
             &  & \underline{+ 0011 1110 1110 0000 0000 0000 0000 0000} \\
             &  & -27648.0                                              \\
             &  & \underline{+ 0.4375}                                  \\
             &  & -27647.5625                                           \\
          \end{alignat*}
          \begin{align*}
            S                    & = 1                                   \\
            F                    & = 0.5625 \times 2 = 1.125             \\
                                 & = 0.125 \times 2 = 0.25               \\
                                 & = 0.25 \times 2 = 0.5                 \\
                                 & = 0.5 \times 2 = 1.0                  \\
            E                    & = 27647 \div 2 = 13823 \text{ rem } 1 \\
                                 & = 13823 \div 2 = 6911 \text{ rem } 1  \\
                                 & = 6911 \div 2 = 3455 \text{ rem } 1   \\
                                 & = 3455 \div 2 = 1727 \text{ rem } 1   \\
                                 & = 1727 \div 2 = 863 \text{ rem } 1    \\
                                 & = 863 \div 2 = 431 \text{ rem } 1     \\
                                 & = 431 \div 2 = 215 \text{ rem } 1     \\
                                 & = 215 \div 2 = 107 \text{ rem } 1     \\
                                 & = 107 \div 2 = 53 \text{ rem } 1      \\
                                 & = 53 \div 2 = 26 \text{ rem } 1       \\
                                 & = 26 \div 2 = 13 \text{ rem } 0       \\
                                 & = 13 \div 2 = 6 \text{ rem } 1        \\
                                 & = 6 \div 2 = 3 \text{ rem } 0         \\
                                 & = 3 \div 2 = 1 \text{ rem } 1         \\
                                 & = 1 \div 2 = 0 \text{ rem } 1         \\
            \text{Un-normalized} & = 110101111111111.1001                \\
            \text{Normalized}    & = 1.101011111111111001                \\
            E                    & = 14 + 127                            \\
                                 & = 141                                 \\
            \\
            \underbrace{1}_{\text{Sign}} \quad
            \underbrace{10001101}_{\text{Exponent}} \quad
            \underbrace{101011111111111001}_{\text{Fraction}}
          \end{align*}
    \item
          \begin{alignat*}{33}
             &  & 1100 0110 1101 1000 0000 0000 0000 0000                    \\
             &  & \underline{\times 0011 1110 1110 0000 0000 0000 0000 0000} \\
             &  & -27648.0                                                   \\
             &  & \underline{\times 0.4375}                                  \\
             &  & -12 096                                                    \\
          \end{alignat*}
          \begin{align*}
            S                    & = 1                                  \\
            E                    & = 12096 \div 2 = 6048 \text{ rem } 0 \\
                                 & = 6048 \div 2 = 3024 \text{ rem } 0  \\
                                 & = 3024 \div 2 = 1512 \text{ rem } 0  \\
                                 & = 1512 \div 2 = 756 \text{ rem } 0   \\
                                 & = 756 \div 2 = 378 \text{ rem } 0    \\
                                 & = 378 \div 2 = 189 \text{ rem } 0    \\
                                 & = 189 \div 2 = 94 \text{ rem } 1     \\
                                 & = 94 \div 2 = 47 \text{ rem } 0      \\
                                 & = 47 \div 2 = 23 \text{ rem } 1      \\
                                 & = 23 \div 2 = 11 \text{ rem } 1      \\
                                 & = 11 \div 2 = 5 \text{ rem } 1       \\
                                 & = 5 \div 2 = 2 \text{ rem } 1        \\
                                 & = 2 \div 2 = 1 \text{ rem } 0        \\
                                 & = 1 \div 2 = 0 \text{ rem } 1        \\
            \text{Un-normalized} & = 10111101000000.0                   \\
            \text{Normalized}    & = 1.01111010000000                   \\
            E                    & = 13 + 127                           \\
                                 & = 140                                \\
            \\
            \underbrace{1}_{\text{Sign}} \quad
            \underbrace{10001100}_{\text{Exponent}} \quad
            \underbrace{01111010000000000000000}_{\text{Fraction}}      \\
          \end{align*}
  \end{enumerate}
}


%%%%%%%%%%%%%%%%%%%%%%%%%%%%%%%%%%%%%%%%%%%%%%%%%%%%%%%%%%%%%%%%%%%%%
\qs{}{
  IA-32 offers an 80-bit extended precision option with a 1-bit sign, 16-bit exponent, and 63-bit fraction (64-bit significand including the implied 1 before the binary point). Assume that extended precision is similar to single and double precision.
  \begin{enumerate}
    \item What is the bias in the exponent?
    \item What is the range (in absolute value) of normalized numbers that can be represented by the extended precision option?
  \end{enumerate}
}

\sol{
  \begin{enumerate}
    \item The bias for an exponent field of \(k\) bits is given by
          \[
            2^{k-1}-1.
          \]
          I.e.:
          \[
            2^{15}-1 = 32768-1 = 32767.
          \]
    \item For normalized numbers the exponent field \(e\) runs from 1 to \(2^{16}-2 = 65534\) (since 0 and all 1's are reserved). Therefore, the true exponent \(E = e - 32767\) varies from:
          \[
            E_{\min} = 1 - 32767 = -32766 \quad \text{to} \quad E_{\max} = 65534 - 32767 = 32767.
          \]
          Hence, the range of normalized numbers is from:
          \[
            1.0 \times 2^{-32766} \quad \text{up to} \quad (2-2^{-63}) \times 2^{32767}.
          \]
  \end{enumerate}
}

%%%%%%%%%%%%%%%%%%%%%%%%%%%%%%%%%%%%%%%%%%%%%%%%%%%%%%%%%%%%%%%%%%%%%
\qs{}{
  Using the refined division hardware, show the unsigned division of:
  \[
    \text{Dividend} = 11011001 \quad \text{by} \quad \text{Divisor} = 00001010
  \]
  The result of the division should be stored in the Remainder and Quotient registers.

  (Eight iterations are required. Show your steps.)
}

\sol{
  \begin{enumerate}
    \item \textbf{Initialize:}
          \[
            \text{Remainder } R = 0,\quad \text{Dividend bits: }1\,1\,0\,1\,1\,0\,0\,1.
          \]
    \item \textbf{Iteration 1:}
          \[
            R \leftarrow (0\ll1) \,|\, 1 = 1.
          \]
          \(1 < 10\), $\Rightarrow$ \(q_7 = 0\).
    \item \textbf{Iteration 2:}
          \[
            R \leftarrow (1\ll1) \,|\, 1 = 3.
          \]
          \(3 < 10 \Rightarrow q_6 = 0.\)
    \item \textbf{Iteration 3:}
          \[
            R \leftarrow (3\ll1) \,|\, 0 = 6.
          \]
          \(6 < 10 \Rightarrow q_5 = 0.\)
    \item \textbf{Iteration 4:}
          \[
            R \leftarrow (6\ll1) \,|\, 1 = 13.
          \]
          \(13 \ge 10\) $\Rightarrow$ \(13-10=3,\) $\Rightarrow$  \(q_4 = 1.\)
    \item \textbf{Iteration 5:}
          \[
            R \leftarrow (3\ll1) \,|\, 1 = 7.
          \]
          \(7 < 10 \Rightarrow q_3 = 0.\)
    \item \textbf{Iteration 6:}
          \[
            R \leftarrow (7\ll1) \,|\, 0 = 14.
          \]
          \(14 \ge 10 \Rightarrow 14-10=4,\) $\Rightarrow$ \(q_2 = 1.\)
    \item \textbf{Iteration 7:}
          \[
            R \leftarrow (4\ll1) \,|\, 0 = 8.
          \]
          \(8 < 10 \Rightarrow q_1 = 0.\)
    \item \textbf{Iteration 8:}
          \[
            R \leftarrow (8\ll1) \,|\, 1 = 17.
          \]
          \(17 \ge 10 \Rightarrow 17-10=7,\) $\Rightarrow$ \(q_0 = 1.\)
  \end{enumerate}

  \textbf{Final Registers:}
  \begin{itemize}
    \item Quotient bits (from \(q_7\) to \(q_0\)): \(0\,0\,0\,1\,0\,1\,0\,1 = 00010101_2\) (which is \(21_{10}\)).
    \item Remainder: \(7\) (or \(00000111_2\)).
  \end{itemize}

}

%%%%%%%%%%%%%%%%%%%%%%%%%%%%%%%%%%%%%%%%%%%%%%%%%%%%%%%%%%%%%%%%%%%%%
\qs{}{
  Using the refined signed multiplication algorithm, show the multiplication of:
  \[
    \text{Multiplicand} = 00101101 \quad \text{by} \quad \text{Multiplier} = 11010110 \quad (\text{signed})
  \]
  The multiplication result should be a 16-bit signed number stored in the HI and LO registers.

  (Eight iterations are required because there are 8 bits in the multiplier. Show your steps.)
}

\sol{
  Define registers:
  \[
    A \;(\text{Accumulator}, 8\text{ bits}),\quad Q \;(\text{Multiplier}, 8\text{ bits}),\quad Q_{-1} \;(\text{1 bit}),\quad \text{and } M \;(\text{Multiplicand}, 8\text{ bits}).
  \]
  Compute \(-M\):
  \[
    M = 00101101_2,\quad -M = 11010011_2.
  \]
  Initialize:
  \[
    A = 00000000,\quad Q = 11010110,\quad Q_{-1}=0.
  \]
  \begin{enumerate}
    \item \textbf{Iteration 1:} Look at \((Q_0, Q_{-1}) = (0,0)\). \(\rightarrow\) No addition/subtraction.
          \\
          Perform arithmetic right shift on \([A,Q,Q_{-1}]\):
          \[
            A\,Q\,Q_{-1}: \quad 00000000\,11010110\,0 \quad\rightarrow\quad 00000000\,01101011\,0.
          \]
    \item \textbf{Iteration 2:} Now, \((Q_0, Q_{-1}) = (1,0)\). \(\rightarrow\) Subtract \(M\):
          \[
            A \leftarrow A - M = 00000000 - 00101101 = 11010011.
          \]
          Then, arithmetic right shift:
          \[
            11010011\,01101011\,0 \quad\rightarrow\quad A = 11101001,\quad Q = 10110101,\quad Q_{-1} = 1.
          \]
    \item \textbf{Iteration 3:} Now, \((Q_0, Q_{-1}) = (1,1)\). \(\rightarrow\) No operation.
          \\
          Shift:
          \[
            11101001\,10110101\,1 \quad\rightarrow\quad A = 11110100,\quad Q = 11011010,\quad Q_{-1} = 1.
          \]
    \item \textbf{Iteration 4:} \((Q_0, Q_{-1}) = (0,1)\). \(\rightarrow\) Add \(M\):
          \[
            A \leftarrow A + M = 11110100 + 00101101 = 00100001 \quad (\text{with overflow discarded}).
          \]
          Shift:
          \[
            00100001\,11011010\,1 \quad\rightarrow\quad A = 00010000,\quad Q = 11101101,\quad Q_{-1} = 0.
          \]
    \item \textbf{Iteration 5:} \((Q_0, Q_{-1}) = (1,0)\). \(\rightarrow\) Subtract \(M\):
          \[
            A \leftarrow 00010000 - 00101101 = 11100011.
          \]
          Shift:
          \[
            11100011\,11101101\,0 \quad\rightarrow\quad A = 11110001,\quad Q = 11110110,\quad Q_{-1} = 1.
          \]
    \item \textbf{Iteration 6:} \((Q_0, Q_{-1}) = (0,1)\). \(\rightarrow\) Add \(M\):
          \[
            A \leftarrow 11110001 + 00101101 = 00011110.
          \]
          Shift:
          \[
            00011110\,11110110\,1 \quad\rightarrow\quad A = 00001111,\quad Q = 01111011,\quad Q_{-1} = 0.
          \]
    \item \textbf{Iteration 7:} \((Q_0, Q_{-1}) = (1,0)\). \(\rightarrow\) Subtract \(M\):
          \[
            A \leftarrow 00001111 - 00101101 = 11100010.
          \]
          Shift:
          \[
            11100010\,01111011\,0 \quad\rightarrow\quad A = 11110001,\quad Q = 00111101,\quad Q_{-1} = 1.
          \]
    \item \textbf{Iteration 8:} \((Q_0, Q_{-1}) = (1,1)\). \(\rightarrow\) No operation.
          \\
          Final shift:
          \[
            11110001\,00111101\,1 \quad\rightarrow\quad A = 11111000,\quad Q = 10011110,\quad Q_{-1} = 1.
          \]
  \end{enumerate}

  \textbf{Final Product:} The 16-bit product is the concatenation of \(A\) (HI) and \(Q\) (LO):
  \[
    \text{HI:LO} = 11111000\,10011110.
  \]
}

\end{document}
