\documentclass[12pt letter]{report}
\input{./template/preamble}
\input{./template/macros}
\input{./template/letterfonts}

\title{\Huge{Dynamic Programming}}
\author{\huge{Madiba Hudson-Quansah}}
\date{}
\usepackage{parskip}

\setcounter{tocdepth}{4}
\setcounter{secnumdepth}{4}

\begin{document}
\maketitle
\newpage
\pdfbookmark[section]{\contentsname}{too}
\tableofcontents
\pagebreak

\chapter{Introduction}

\dfn{Dynamic Programming}{
  A programming technique that involves breaking down a problem into smaller subproblems and solving each subproblem
  only once. The solutions to the subproblems are stored in a table, so that the solutions to the subproblems are not
  recomputed. This technique is used to solve optimization problems /A general algorithm design technique for solving
  problems defined by recurrences with overlapping subproblems, i.e.:
  \[
    T \left( n \right)  = T \left( n - 1 \right)  + T \left( n - 2 \right)
  \]
  Where the problem $T \left( n \right) $ is made up of the subproblems $T \left( n - 1 \right) $ and $T \left( n - 2
    \right) $ and $T \left( n - 1 \right) $ is made up of the subproblems $T \left( n - 2 \right) $ and $T \left( n - 3
    \right) $ and so on.
}

A Dynamic Programming problem is made up of two main components:
\begin{itemize}
  \item Overlapping Subproblems - Where the problem can be broken down into smaller subproblems that are solved
        multiple time during the process.
  \item Optimal Substructure - Where an optimal solution of the original problem can be constructed from the
        optimal solutions of its subproblems.
\end{itemize}

There are two main approaches to solving a Dynamic Programming problem:
\begin{itemize}
  \item Memoization - Top Down Approach
  \item Tabulation - Bottom Up Approach
\end{itemize}

\qs{}{
  Given a row of coins of the following values: $7, 5, 2, 10, 6, 3, 4, 8, 1$, pick up coins with a maximum value subject
  to no adjacent coins can be picked. Produce the optimal solutions in terms of $F$ array with a linear algorithm.
}

\sol{
  Given that there are two choices to make when picking coins, skipping the last coin and picking all the others or
  picking the last coin and skipping the second last coin. The max value of the problem becomes:
  \[
    F \left( n \right) = \text{max} \left(  C_1 + F \left( n-2 \right)\, , F \left( n-1 \right)  \right)\, , n > 1
  \]
  Where $C$ is the array of coins.
}

\[
F \left( i, j \right)  = 
\] 


\end{document}
