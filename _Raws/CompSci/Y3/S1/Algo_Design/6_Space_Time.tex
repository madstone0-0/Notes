\documentclass[12pt letter]{report}
\input{./template/preamble}
\input{./template/macros}
\input{./template/letterfonts}

\title{\Huge{Space and Time Trade-Offs}}
\author{\huge{Madiba Hudson-Quansah}}
\date{}
\usepackage{parskip}

\setcounter{tocdepth}{4}
\setcounter{secnumdepth}{4}

\begin{document}
\maketitle
\newpage
\pdfbookmark[section]{\contentsname}{too}
\tableofcontents
\pagebreak

\chapter{Introduction}

Space and time trade-offs are a common occurrence in computer science. The trade-off is a situation where one must
sacrifice one aspect of a system to gain another. In computer science, the trade-off is usually between space and time.
Space refers to the amount of memory used by a program, while time refers to the amount of time it takes to run the
program. In general, the more space a program uses, the faster it will run, and vice versa. The main idea in computing
is to pre-process the problem's input at least somewhat and store additional information obtained to speed up solving
the problem afterwards. This is called \textbf{input enhancement}. \\
Another technique that exploits space-for-time trade-offs uses extra space to facilitate faster and more flexible access
to the data. This is called \textbf{prestructuring }. \\

\chapter{Input Enhancement}

\section{Sorting By Counting}
\dfn{Comparison Counting Sort}{
  For each element $x$ in the input array $A$, count the number of elements less than $x$ and store this count in
  an auxiliary array $C$. Then, use the counts in $C$ to place each element in its correct position in the output
  array $B$.
}
\begin{algorithm}[H]
  \caption{ComparisonCountingSort $ \left( A \left[ 0 \ldots n - 1 \right]  \right) $}
  \Comment{}\\
  \Comment{Sorts an array by comparison counting }\\
  \Comment{Input: An array $A \left[ 0 \ldots n - 1 \right] $ of orderable elements} \\
  \Comment{Output: Array $S \left[ 0 \ldots n - 1 \right] $ of $A$'s elements sorted in nondecreasing order} \\
  \begin{algorithmic}[1]
    \For{ $i \gets 0$ to $n-1$}
    \State  $\text{Count}_i \gets 0 $
    \EndFor
    \For{ $i \gets 0$ to $n - 2$}
    \For{ $j \gets i + 1$ to $n - 1$}
    \If{ $A_i > A_j$}
    \State $\text{Count}_i \gets \text{Count}_i + 1$
    \Else
    \State $ \text{Count}_j \gets \text{Count}_j + 1$
    \EndIf
    \EndFor
    \EndFor
    \For{ $i \gets 0 $ to $n - 1$}
    \State $S_{\text{Count}_i} \gets A_i $
    \EndFor
    \State \Return $S$
  \end{algorithmic}
\end{algorithm}
The time complexity of this algorithm is $O \left( n^2 \right) $, where $n$ is the number of elements in the input array

\subsection{Distribution Counting Sort}
The idea of the comparison counting sort performs better when the elements to be sorted belong to a small set of
variables and the number of elements to be sorted is large. Given the array:

\begin{table}[H]

  \centering
  \begin{tabular}{|c|c|c|c|c|c|}
    \hline
    $13$ & $11$ & $12$ & $13$ & $12$ & $12$ \\
    \hline
  \end{tabular}
\end{table}
Whose elements are known to belong to the set $\{11, 12, 13\} $, and should not be overwritten in the process of
sorting. The frequency and distribution arrays are:
\begin{table}[H]
  \centering
  \begin{tabular}{c c c c}
    \hline
    Array Values        & $11$ & $12$ & $13$ \\
    \hline
    Frequencies         & $1$  & $3$  & $2$  \\
    Distribution values & $1$  & $4$  & $6$  \\
    \hline
  \end{tabular}
\end{table}
Now the distribution values indicate the final position of the last element of the corresponding value in the sorted
array.

\begin{algorithm}[H]
  \caption{DistributionCountingSort $ \left( A \left[ 0 \ldots n - 1 \right], l, u \right) $}
  \Comment{}\\
  \Comment{Sorts an array of integers from a limited range by distribution counting} \\
  \Comment{Input: An array $A \left[ 0 \ldots n - 1 \right] $ of integers between $l$ and $u$ ($l \leq u$)} \\
  \Comment{Output: Array $S \left[ 0 \ldots n - 1 \right] $ of $A$'s elements sorted in nondecreasing order} \\
  \begin{algorithmic}[1]
    \For{ $j \gets 0$ to $u - l$}
    \State $D_j \gets 0$ \Comment{Initialize frequencies}
    \EndFor
    \For{ $i \gets 0$ to $n - 1$}
    \State  $D_{A_i - l} \gets D_{A_i - l} + 1$ \Comment{Compute frequencies}
    \EndFor
    \For{ $j \gets 1$ to $u - l$}
    \State  $D_j \gets D_{j- 1} + D_j$
    \EndFor
    \For{ $i \gets n - 1$ down to $0$}
    \State $j \gets A_i - l$
    \State $S_{D_j - 1} \gets A_i$
    \State $D_j \gets D_j - 1$
    \EndFor
    \State \Return $S$
  \end{algorithmic}
\end{algorithm}

\end{document}
