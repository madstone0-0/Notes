\documentclass[12pt letter]{report}
\input{../template/preamble}
\input{../template/macros}
\input{../template/letterfonts}

\usepackage{parskip}
\usepackage{circuitikz}
\title{\Huge{Assignment 7}}
\author{\huge{Madiba Hudson-Quansah}}
\date{March 2024}


\begin{document}
\maketitle
\newpage

\qs{}{
  \[
    \overline{\left( \overline{x} + y \right) }\left( x + y + \overline{z} \right)
  \]
}

\sol{

  \begin{circuitikz} \draw
    (0, 10) node[not port] (not1) {}
    (3,9) node[or port] (or1) {}
    (4,9) node[not port] (not2) {}

    (1,4) node[or port] (or2) {}
    (4,5) node[or port] (or3) {}
    (1, 6) node[not port] (not3) {}


    (7,5.3) node[and port] (and1) {}

    (not1.in) node[left=1cm](x1) {$x$}
    (or1.in 2) node[left=3.4cm](y1) {$y$}

    (or3.in 1) node[left=4.4cm](z) {$z$}
    (or2.in 1) node[left=1.4cm](x2) {$x$}
    (or2.in 2) node[left=1.4cm](y2) {$y$}

    (not1.out) -| (or1.in 1)
    (or1.out) -| (not2.in)
    (not2.out) -| (and1.in 1)
    (or3.out) -> (and1.in 2)
    (or2.out) -| (or3.in 2)
    (not3.out) -| (or3.in 1)
    (x1) -| (not1.in)
    (y1) -| (or1.in 2)
    (z) -| (not3.in)
    (x2) -| (or2.in 1)
    (y2) -| (or2.in 2)

    ;
  \end{circuitikz}
}

\qs{}{
  \[
    1^3 + 2^3 + \ldots + n^3 = \left( \frac{n \left( n + 1 \right) }{2} \right)^2
  \]

  Whenever $n$ is a positive integer
}

\begin{myproof}

  Let $P \left( n \right) $ be $1^3 + 2^3 + \ldots + n^3 = \left( \frac{n \left( n + 1 \right) }{2} \right)^2$

  \noindent \underline{Basis Step} \\
  \\
  \begin{align*}
    P \left( 1 \right): 1^3 & = \left( \frac{1 \left( 1 + 1 \right) }{2} \right)^2 \\
    1                       & = \left( \frac{2}{2} \right)^2                       \\
    1                       & = 1                                                  \\
  \end{align*}

  \noindent \underline{Induction  Step} \\
  \\

  \noindent To complete this step I must prove $P \left( k \right) \to P \left( k + 1 \right)  $ for any positive integer $k$ \\
  Assume $P \left( k \right) $ is $T$  for some positive integer $k$, then \\
  \[
    P \left( k \right): 1^3 + 2^3 + \ldots + k^3 = \left( \frac{k^2 + k}{2} \right)^2
  \]

  Then $P \left( k + 1 \right) $ is
  \[
    P \left( k + 1 \right) : 1^3 + 2^3 + \ldots + \left( k + 1 \right) ^3 = \left( \frac{\left( k + 1 \right) \left( k + 2
        \right) }{2} \right)^2
  \]

  And can be expressed as $P \left( k + 1 \right): P \left( k \right) + \left( k + 1 \right)^3   $, therefore:
  \begin{align*}
    P \left( k + 1 \right): 1^3 + 2^3 + \ldots + k^3 +  \left( k + 1 \right)^3 & = \left( \frac{k^2 + k}{2} \right)^2 +
    \left( k + 1 \right)^3                                                                                                                                      \\
                                                                               & = \frac{\left( k^2 + k \right)^2 }{4} + \left( k + 1 \right)^3                 \\
                                                                               & = \frac{\left( k^2 + k \right)^2 + 4 \left( k + 1 \right)^3  }{4}              \\
                                                                               & = \frac{\left( k^2 + k \right)^2 + \left( 4k^3 + 12k^2 + 12k + 4 \right)  }{4} \\
                                                                               & = \frac{k^{4} + 2k^3 + k^2 +  4k^3 + 12k^2 + 12k + 4}{4}                       \\
                                                                               & = \frac{k^{4} + 6k^3 + 13k^2 + 12k + 4 }{4}                                    \\
                                                                               & = \frac{\left( k + 1 \right)^2 \left( k + 2 \right)^2}{4}                      \\
                                                                               & = \left( \frac{\left( k + 1 \right) \left( k +
    2\right)  }{2} \right)^2                                                                                                                                    \\
  \end{align*}
  $\therefore$ $P \left( k + 1 \right) $ is $T$ \\
  Hence we can conclude that $P \left( n \right) $ is true for all positive integers $n$

\end{myproof}

\qs{}{
  \[
    2 - 2 \times 7 + 2 \times 7^2 - \ldots + 2 \times \left( - 7 \right)^{n} = \frac{1 - \left( -7  \right)^{n + 1} }{4}
  \]

  Whenever $n$ is a non-negative integer
}

\begin{myproof}

  Let $P \left( n \right) $ be $2 - 2 \times 7 + 2 \times 7^2 - \ldots + 2 \times \left( - 7 \right)^{n} = \frac{1 - \left( -7  \right)^{n + 1} }{4}$

  \noindent \underline{Basis Step} \\
  \\
  \begin{align*}
    P \left( 0 \right): 2 & = \frac{1 - \left( -7  \right)^{0 + 1} }{4} \\
    2                     & = \frac{8}{4}                               \\
    2                     & = 2                                         \\
  \end{align*}

  \noindent \underline{Induction  Step} \\
  \\
  To complete this step I must prove $P \left( k \right) \to P \left( k + 1 \right)  $ for any non-negative integer $k$ \\
  Assume $P \left( k \right) $ is $T$  for some non-negative integer $k$, then \\
  \[
    P \left( k \right): 2 - 2 \times 7 + 2 \times 7^2 - \ldots + 2 \times \left( - 7 \right)^{k} = \frac{1 - \left( -7
      \right)^{k + 1} }{4}
  \]
  Then $P \left( k + 1 \right) $ is

  \[
    P \left( k + 1 \right): 2 - 2\times 7 + 2 \times 7^2 - \ldots + 2 \times \left( - 7 \right)^{\left( k + 1 \right) } =
    \frac{1 - \left( -7 \right)^{k + 2} }{4}
  \]
  And can be expressed as $P \left( k + 1 \right): P \left( k \right) + 2\times \left( -7 \right)^{\left( k + 1 \right)
    }   $, therefore:
  \begin{align*}
    P \left( k + 1 \right): 2 - 2\times 7 + 2 \times 7^2 - \ldots - 2 \times \left( -7 \right)^{k}  + 2 \times \left( -
    7 \right)^{\left( k + 1 \right) } & = \frac{1 - \left( - 7 \right)^{k + 1} }{4} + 2 \times \left( -7 \right)^{\left( k +
    1 \right) }                                                                                                                           \\
                                      & = \frac{1 - \left( - 7 \right)^{k + 1} + 8 \times \left( -7 \right)^{\left( k + 1 \right) }  }{4} \\
                                      & = \frac{1 + 8 \times  \left( - 7  \right)^{k + 1} - \left( -7 \right)^{k + 1}  }{4}               \\
    \text{Let } x = \left( -7 \right)^{k + 1}                                                                                             \\
                                      & = \frac{1 + 8x - x}{4}                                                                            \\
                                      & = \frac{1 + 7x}{4}                                                                                \\
                                      & = \frac{1 + 7^{1}\times \left( -7 \right)^{k + 1} }{4}                                            \\
                                      & = \frac{1 - \left( -7 \right)^{1} \times \left( - 7 \right)^{k + 1}   }{4}                        \\
                                      & = \frac{1- \left( -7 \right)^{k + 2} }{4}                                                         \\
  \end{align*}
  $\therefore$ $P \left( k + 1 \right) $ is $T$ \\
  Hence we can conclude that $P \left( n \right) $ is true for all non-negative integers $n$


\end{myproof}


\end{document}
