\documentclass[12pt letter]{report}
\input{./template/preamble}
\input{./template/macros}
\input{./template/letterfonts}

\title{\Huge{Matrix Algebra}}
\author{\huge{Madiba Hudson-Quansah}}
\date{}
\usepackage{parskip}

\setcounter{tocdepth}{4}
\setcounter{secnumdepth}{4}

\begin{document}
\maketitle
\newpage
\pdfbookmark[section]{\contentsname}{too}
\tableofcontents
\pagebreak

\chapter{Matrix Operations}


If $A$ is a $n \times m$ matrix then the scalar entry in the $i$th row and the $j$th column of $A$ is denoted by
$a_{ij}$, and is called the $ \left( i, j \right) $-entry. Each column of $A$ is a list of $m$ real numbers in the
$\mathbb{R}^{m}$ vector space. Therefore the columns of $A$ can be represented as vectors in $\mathbb{R}^{m}$:
\[
  A = \begin{bmatrix} \mbold{a}_1 & \mbold{a}_2 & \ldots \mbold{a}_n \end{bmatrix}
\]

\dfn{Diagonals}{
  The diagonal entries of a matrix $A$ of dimension $n \times m$, are the entries $a_{ij}$, where $i = j$. This is
  called the \textbf{main diagonal} of the matrix $A$. A \textbf{diagonal matrix} is a square matrix $n\times n$ whose
  non-diagonal entries are all zero.
}

\section{Sums and Scalar Multiples}

\dfn{Equality of Matrices}{
  Two matrices $A$ and $B$, are equal if:
  \begin{itemize}
    \item The are of the same size i.e, $m\times x$
    \item The corresponding entries are equal i.e, $A_{ij} = B_{ij}$
  \end{itemize}
}

\thm{Axioms of Matrix Addition}{
  Let $A, B$ and $C$ be matrices of the same size, and let $r$ and $s$ be scalars. Then the following axioms hold:
  \begin{description}
    \item[Communtativity] $A + B = B + A$
    \item[Associativity] $ \left( A + B \right) + C = A + \left( B + C\right)  $
    \item[Additive Identity] $A + 0 = A$
    \item[Distruibutivity 1] $r\left( A + B \right) = rA + rB$
    \item[Distruibutivity 2] $ \left( r + s \right) A = rA + sA $
    \item[Compatibility with Scalar Multiplication] $r\left( sA \right) = \left( rs \right)A$
  \end{description}
}

\section{Matrix Multiplication}

When a matrix $B$ multiples a vector $\mbold{x}$, it transforms $\mbold{x}$ into the vector $B\mbold{x}$. If this vector
is then multiples by another matrix $A$, the result is the vector $A \left( B \mbold{x}\right) $. Thus $A \left( B
  \mbold{x}\right) $ is produced by a composition of mappings / linear transformations. This can be also expressed as:
\[
  A \left( B \mbold{x}\right) = \left( AB \right) \mbold{x}
\]

Because, if $A$ is $m\times n$, $B$ is $n\times p$ and $\mbold{x}$ is in $\mathbb{R}^{p}$, can denote the columns of
$B$, by $\mbold{b}_1, \ldots, \mbold{b}_p$ and the entries of $\mbold{x}$ by, $x_1, \ldots, x_p$. Then
\[
  B \mbold{x} = x_1 \mbold{b}_1 + \ldots + x_p \mbold{b}_p
\]
By the linearity of matrix multiplication, we have:
\begin{align*}
  A \left( B \mbold{x} \right) & = A \left( x_1 \mbold{b}_1 \right) + \ldots + A \left( x_p \mbold{b}_p \right) \\
                               & = x_1 \left( A\mbold{b}_1 \right) + \ldots + x_p \left( A \mbold{b}_p \right)  \\
\end{align*}
The vector $A \left( B\mbold{x} \right) $ is then a linear combination of the vectors $A\mbold{b}_1, \ldots,
  A\mbold{b}_p$, using the entries of $\mbold{x}$ as weights. This can be expressed in matrix notation as:
\[
  A \left( B \mbold{x} \right)  = \begin{bmatrix} A \mbold{b}_1 & A \mbold{b}_2 & \ldots & A \mbold{b}_p \end{bmatrix}
  \mbold{x}
\]
\thm{}{
  If $A$ is an $m \times n$ matrix, and if $B$ is an $n \times p$ matrix with columns $\mbold{b}_1, \ldots, \mbold{b}_p$,
  then the product $AB$ is the $m \times p$ matrix whose columns are $A\mbold{b}_1, \ldots, A\mbold{b}_p$.That is:
  \[
    A \left( B \mbold{x} \right)  = \begin{bmatrix} A \mbold{b}_1 & A \mbold{b}_2 & \ldots & A \mbold{b}_p \end{bmatrix}
    \mbold{x}
  \]
}

\ex{}{
  \qs{}{
    Compute $AB$ where $A = \begin{bmatrix} 2 & 3 \\ 1 & -5 \end{bmatrix} $, and $B = \begin{bmatrix} 4  & 3 & 6 \\ 1 &
                -2 & 3\end{bmatrix} $
  }

  \sol{
    \begin{align*}
      A \mbold{b}_1 & = \begin{bmatrix} 2 & 3 \\ 1 & -5 \end{bmatrix} \begin{bmatrix} 4 \\ 1 \end{bmatrix}  \\
                    & = \begin{bmatrix}
                          8 + 3 \\
                          4 + -5
                        \end{bmatrix}                                                                      \\
                    & = \begin{bmatrix}
                          11 \\
                          -1
                        \end{bmatrix}                                                                      \\
      \\
      A \mbold{b}_2 & = \begin{bmatrix} 2 & 3 \\ 1 & -5 \end{bmatrix} \begin{bmatrix} 3 \\ -2 \end{bmatrix} \\
                    & = \begin{bmatrix}
                          6 - 6 \\
                          3 + 10
                        \end{bmatrix}                                                                      \\
                    & = \begin{bmatrix}
                          0 \\
                          13
                        \end{bmatrix}                                                                      \\
      \\
      A \mbold{b}_3 & = \begin{bmatrix} 2 & 3 \\ 1 & -5 \end{bmatrix} \begin{bmatrix} 6 \\ 3 \end{bmatrix}  \\
                    & = \begin{bmatrix}
                          21 \\
                          -9
                        \end{bmatrix}                                                                      \\
      \\
      AB            & =  \begin{bmatrix}
                           11 & 0  & 12 \\
                           -1 & 13 & -9
                         \end{bmatrix}                                                                     \\
    \end{align*}
  }
}

\thm{Row-Column Rule}{
If the product $AB$ is defined, them the entry in row $i$ and column $j$ of $AB$ is the sum of the products of
corresponding entries of the row $i$ of $A$ and column $j$ of $B$. If $ \left( A B \right)_{ij} $ denotes the $ \left(
  i, j \right) $-entry in $A B$, and if $A$ is an $m \times n$, then
\[
  \left( A B \right)_{ij} = a_{i1}b_{1j} + a_{i2}b_{2j} + \ldots + a_{in}b_{nj}
\]
}

\ex{}{
  Use the row–column rule to compute two of the entries in $A B$ for the
  matrices:
  \[
    A = \begin{bmatrix}
      2 & 3  \\
      1 & -5
    \end{bmatrix}, \,
    B = \begin{bmatrix}
      4 & 3  & 6 \\
      1 & -2 & 3
    \end{bmatrix}
  \]
  An inspection of the numbers involved will make it clear how
  the two methods for calculating $A B$ produce the same matrix. \\


  The dimensions of the resultant matrix is $2 \times 3$, therefore the entries of $A B$ are:
  \begin{align*}
    A B & = \begin{bmatrix}
              2 \left( 4 \right)+ 3 \left( 1 \right)  & 2 \left( 3 \right)  + 3 \left( -2 \right) & 2 \left( 6 \right) + 3
              \left( 3 \right)                                                                                             \\
              1 \left( 4 \right)  -5 \left( 1 \right) & 1 \left( 3 \right) - 5 \left( -2 \right)  & 1 \left( 6 \right) - 5
              \left( 3 \right)
            \end{bmatrix} \\
        & = \begin{bmatrix}
              11 & 0  & 21 \\
              -1 & 13 & 9  \\
            \end{bmatrix}                                                                                               \\
  \end{align*}
}

\ex{}{
  \qs{}{
    Find the entries in the second row of $AB$ where,
    \[
      A = \begin{bmatrix}
        2  & -5 & 0  \\
        -1 & 3  & -4 \\
        6  & -8 & -7 \\
        -3 & 0  & 9
      \end{bmatrix}, \,
      B = \begin{bmatrix}
        4 & -6 \\
        7 & 1  \\
        3 & 2
      \end{bmatrix}
    \]
  }

  \sol{
    \begin{align*}
      \begin{bmatrix} -1 & 3 & -4 \end{bmatrix} \begin{bmatrix} 4 & -6 \\ 7 & 1 \\ 3 & 2 \end{bmatrix} \\
      \begin{bmatrix}
        -4 + 21 - 12 & 6 + 3 - 8
      \end{bmatrix}                                                                         \\
      \begin{bmatrix} 5 & 1 \end{bmatrix}
    \end{align*}
  }
}

\thm{Axioms of Matrix Multiplication}{
  Let $A$ be an $m \times n$ matrix and let $B$ and $C$ have sizes for which  the indicated sums and products are
  defined:
  \begin{description}
    \item[Associativity]  $A \left( B C \right) = \left( A B \right)  C$
    \item[Left Distruibutivity] $A \left( B + C \right) = A B + A C $
    \item[Right Distruibutivity] $ \left( B + C \right) A = B A + C A $
    \item[Scalar Associativity] $ r \left( A B \right) = \left( r A \right) B = A \left( r B \right), \, \forall r, \, r
            \in \mathbb{F}$
    \item[Mutliplicative Identitiy] $I_{m}A = A = A I_{n} $
  \end{description}
}

\ex{}{
  \qs{}{
    Let $A = \begin{bmatrix} 5 & 1 \\ 3 & -2 \end{bmatrix} $ and $B = \begin{bmatrix} 2 & 0 \\ 4 & 3 \end{bmatrix} $. Show
    that these matrices do not commute, I.e, verify $A B \neq  B A$
  }

  \sol{
    \begin{align*}
      A B            & = \begin{bmatrix} 5 & 1 \\ 3 & -2 \end{bmatrix} \begin{bmatrix} 2 & 0 \\ 4 & 3 \end{bmatrix} \\
                     & = \begin{bmatrix}
                           14 & 3  \\
                           -2 & -6
                         \end{bmatrix}                                                                             \\
      \\
      B A            & = \begin{bmatrix} 2 & 0 \\ 4 & 3 \end{bmatrix} \begin{bmatrix} 5 & 1 \\ 3 & -2 \end{bmatrix} \\
                     & = \begin{bmatrix}
                           10 & 2  \\
                           29 & -2
                         \end{bmatrix}                                                                             \\
      \\
      \therefore A B & \neq BA
    \end{align*}
  }
}

\subsection{Powers of a Matrix}

\dfn{Powers of a Matrix}{
  If $A$ is an $n \times n$ matrix and if $k$ is a positive integer, then $A^k$ denotes the product of $k$ copies of
  $A$:
  \[
    A^{k} = A_1 \ldots A_k
  \]
  Where $A_1 = A_2 \wedge A_2 = A_3 \wedge \ldots \wedge A_{k-1} = A_k$
}

If $A$ is non-zero and if $\mbold{x}$ is in $\mathbb{R}^{n}$, then $A^{k} \mbold{x}$ is the result of left-multiplying
$\mbold{x}$ by $A$ repeatedly $k$ times.

If $k = 0$, then $A^{0} \mbold{x}$ is $\mbold{x}$. Thus $A^{0}$ is interpreted as the Identity matrix.

\section{The Transpose of a Matrix}

\dfn{The Transpose of a Matrix}{
Given a matrix $A$, its \textit{transpose}, denoted by $A^T$, is defined by transforming  the rows of $A$ into columns.
For example:
\[
  \begin{bmatrix} 1 & 2 & 3 \\ 4 & 5 & 6  \end{bmatrix}^{T} = \begin{bmatrix} 1 & 4 \\ 2 & 5 \\ 3 & 6 \end{bmatrix}
\]
Therefore formally, the transpose of a matrix $A_{m,n}$ is defined as:
\[
  A^T_{m,n}  = A_{n,m}
\]
}

Therefore, let $A$ and $B$ denote matrices whose sizes are appropriate for the following sums and products:
\begin{enumerate}
  \item $ \left( A^{T} \right)^{T} = A$
  \item $ \left( A + B \right)^{T} = A^{T} + B^{T} $
  \item $\forall r \in \mathbb{F}, \, \left( r A \right)^{T} = r A^{T}  $
  \item $ \left( A B \right)^{T} = B ^{T} A^{T} $ \label{thm:asctrans}
\end{enumerate}

Usually $ \left( A B \right)^{T} $ is not equal $A ^{T} B^{T}$, even when $A$ and $B$ have dimensions such that $A ^{T}
  B^{T}$ is defined. The generalization of axiom \ref{thm:asctrans} to products more than two factors is as follows:

\thm{}{
  The transpose of a product of matrices equals the product of their transpose in the reverse order.
}

\chapter{The Inverse Of A Matrix}

\section{Invertibility}

\end{document}
