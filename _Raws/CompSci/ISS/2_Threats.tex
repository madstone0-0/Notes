\documentclass[12pt letter]{report}
\input{./template/preamble}
\input{./template/macros}
\input{./template/letterfonts}

\title{\Huge{Threats and Attacks}}
\author{\huge{Madiba Hudson-Quansah}}
\date{}
\usepackage{parskip}

\setcounter{tocdepth}{4}
\setcounter{secnumdepth}{4}

\begin{document}
\maketitle
\newpage
\pdfbookmark[section]{\contentsname}{too}
\tableofcontents
\pagebreak

\chapter{Threat Categories}

There are 12 categories of threats:
\begin{itemize}
  \item Compromises to Intellectual Property
  \item Deviations in quality of service
  \item Espionage / Trespass
  \item Forces of Nature
  \item Human error / Failure
  \item Information Extortion
  \item Sabotage / Vandalism
  \item Software Attacks
  \item Technical Hardware failures / Errors
  \item Technical Software failures / Errors
  \item Technological Obsolescence
  \item Theft
\end{itemize}

\section{Compromises to Intellectual Property}

\dfn{Intellectual Property}{
  Creation, ownership, and control of original ideas, information,
  and creative works.
}

IP breaches compromise the \textbf{Confidentiality} of information,
some examples are piracy, and copyright infringement.

\subsection{Technical Controls}

\begin{itemize}
  \item Watermarking
  \item Embedded Code
  \item Registration Keys
  \item Intentional Corruption
  \item Obfuscation
\end{itemize}

\subsection{Administrative/Legal Controls}

\begin{itemize}
  \item  End User License Agreements (EULA)
  \item Copyrights
  \item Licenses
\end{itemize}

\section{Deviations in Quality of Service}

Deviations in quality of service occur when products or services are
not delivered as expected, this compromises the \textbf{Availability}
of a system. This is commonly cause by:
\begin{itemize}
  \item Failure of interdependent support systems
  \item Supply Chain failures
  \item Failure of critical infrastructure (e.g power grid)
\end{itemize}

\subsection{Technical Controls}
\begin{itemize}
  \item Redundancy
  \item Failover Systems / Redundant Systems (e.g. Redundant Array of
    Independent Disks (RAID))
  \item Backup Internet Service Providers (ISPs)
  \item Load Balancing
  \item Cloud Geographic Distribution
\end{itemize}

\subsection{Administrative Controls}
\begin{itemize}
  \item Service Level Agreements (SLAs) - Contract between service
    provider and customer that specifies the level of service
    expected during its term.
  \item Policies and Procedures to deal with QoS issues
\end{itemize}

\section{Espionage / Trespass}

\dfn{Espionage / Spying}{
  Gaining unauthorized access to the information of an organization.
  The practice of obtaining information about an organization without
  the permission of the holder of the information.
}

\dfn{Trespass}{
  The act of entering someone's land or property without permission.
}

Espionage and Trespass primarily compromise the
\textbf{Confidentiality}. Some examples include:
\begin{itemize}
  \item Shoulder Surfing - Observing a person's private information
    over their shoulder.
  \item Industrial Espionage
  \item Hacking
  \item Governmental Espionage
  \item Social Engineering
\end{itemize}

\subsection{Technical Controls}
\begin{itemize}
  \item Intrusion Detection Systems (IDS) / Intrusion Prevention Systems (IPS)
  \item Firewalls
  \item Encryption
\end{itemize}

\subsection{Administrative Controls}
\begin{itemize}
  \item Access Control Policies
  \item Background checks
  \item Non-Disclosure Agreements (NDAs)
\end{itemize}

\subsection{Physical Controls}
\begin{itemize}
  \item Security Guards
  \item CCTV Surveillance
  \item Privacy Screens
\end{itemize}

\section{Forces of Nature}

Disrupts not only individual loves but also the storage, transmission
and use of information. Natural disasters primarily compromises
\textbf{Availability} and \textbf{Integrity}. Some examples include:
\begin{itemize}
  \item Fires
  \item Floods
  \item Earthquakes
\end{itemize}

\subsection{Technical Controls}
\begin{itemize}
  \item Offsite Backups
  \item System Redundancy and Failover
  \item Uninterruptible Power Supplies (UPS)
\end{itemize}

\subsection{Administrative Controls}
\begin{itemize}
  \item Disaster Recovery Plan (DRP)
  \item Business Continuity Plan (BCP)
  \item Risk Assessments
\end{itemize}

\subsection{Physical Controls}
\begin{itemize}
  \item Secure Facility design
  \item Geographic Diversity
\end{itemize}

\section{Human Error / Failure}
Occurs when users or administrators make mistakes that compromise
information systems. This primarily compromises \textbf{Confidentiality},
\textbf{Integrity}, and \textbf{Availability}. Some examples include:
\begin{description}
  \item[Social Engineering]  - The use of social skills for the purpose
    of gaining unauthorized access to information or systems.
  \item[Other]
    \begin{itemize}
      \item Phishing
      \item Spearpshising
      \item Whaling
      \item Watering Hole Attacks
      \item Pretexting
      \item Vishing
    \end{itemize}
\end{description}

\subsection{Technical Controls}
\begin{itemize}
  \item Email Filtering
  \item Anti-Phishing software
  \item Multifactor Authentication
\end{itemize}

\subsection{Administrative Controls}
\begin{itemize}
  \item Regular Security Awareness Training
  \item Phishing Simulations
  \item Clear Reporting Procedures
\end{itemize}

\subsection{Operational Controls}
\begin{itemize}
  \item Incident Response Plans
  \item Monitoring Unusaal Account Activity
  \item Regular Software Updates and Patch Management
\end{itemize}

\section{Information Extortion / Cyberextortion}

An attacker steals information from a computer system and demands
compensation for it's return or non-disclosure. Cyberextortion
primarily compromises \textbf{Availability}, then \textbf{Integrity}
and \textbf{Confidentiality} if data is exfiltrated. There are two
types of ransomware attacks:
\begin{itemize}
  \item Lock screen - Prevents access to the device
  \item Encryption - Encryptions sensitive files
\end{itemize}

\subsection{Technical Controls}
\begin{itemize}
  \item Anti-Malware Software
  \item Email and Web Filtering
  \item Data Backups
\end{itemize}

\subsection{Administrative Controls}

\begin{itemize}
  \item User Training
  \item Incident Response Plan
  \item Regular Software Updates and Patch Management
\end{itemize}

\section{Sabotage / Vandalism}

\dfn{Sabotage}{
  The obstruction of a system's operations or intended functions.
  Sabotage is often directed with an explicit reason / goal.
}

\dfn{Vandalism}{
  The intentional destruction or defacement of property. The main
  intention of vandalism is destruction.
}

Involves the deliberate sabotage of a computer system or business, or
acts of vandalism to destroy and asset or damage the image of an
organization. This primarily compromises \textbf{Integrity} and
\textbf{Availability}. Some examples include:

\begin{itemize}
  \item Website defacing - Erodes consumer confidence
  \item Hacktivism / Cyberactivism
  \item Cyberterrorism / Cyberwarfare.
\end{itemize}

\subsection{Technological Controls}

\begin{itemize}
  \item Web Application Firewalls (WAF)
  \item Distributed Denial of Service (DDoS) protection
  \item Website defacement monitoring
  \item Backups
\end{itemize}

\subsection{Administrative Controls}
\begin{itemize}
  \item Policies and Procedures
  \item Legal Enforcement
  \item Awareness Training
\end{itemize}

\subsection{Operational Controls}
\begin{itemize}
  \item Physical security of equipment.
  \item Host Redundancy
  \item Incident Response
  \item Monitoring
\end{itemize}

\section{Software Attacks}

\dfn{Malware}{
  Malicious software designed to infiltrate or damage a computer
  system without the owner's informed consent.
}

Involve designing and deploying malware to compromise a system.
Software attacks may overwhelm the processing capabilities of online
systems or allow access to protected systems by hidden means.
Software attacks primarily compromise \textbf{Confidentiality},
\textbf{Integrity}, and \textbf{Availability} to varying severities.
Some examples include:
\begin{description}
  \item[Traditinoal Malware] Viruses, Worms, Trojans, Spyware,
    Adware, Ransomware, Rootkits, Keyloggers
  \item[Access/Control Malware] Backdoors, Botnets, Remote Access Trojans (RATs)
  \item[Disruption Attacks] Denial of Service (DoS), Distributed
    Denial of Service (DDoS)
  \item[Interception Attacks] Spoofing, Man-In-the-Middle (MitM),
    Session Hijacking, Packet sniffing, pharming
\end{description}

\dfn{Viruses}{
  A type of malware that attaches itself to a legitimate program or
  file and spreads to other programs and files when executed.
}

\dfn{Worm}{
  A type of malware that can replicate itself and spread
  independently without needing to attach to a host program.
}

\dfn{Trojan}{
  A type of malware that disguises itself as a legitimate program or
  file to trick users into installing it, often creating a backdoor
  for unauthorized access.
}

\dfn{Logic Bomb}{
  A type of malware that is triggered by a specific event or
  condition, such as a date or the deletion of a file.
}

\dfn{Zero-Day Attack}{
  An attack that exploits a previously unknown vulnerability in a
  computer application, meaning there is no existing patch or fix for
  the vulnerability.
}

\dfn{Pharming}{
  A cyberattack that redirects a website's traffic to a fraudulent
  website, often through DNS cache poisoning or modifying the hosts
  file on a victim's computer.
}

\dfn{Phising}{
  A cyberattack that uses fraudulent emails or websites to trick
  individuals into revealing sensitive information, such as passwords
  or credit card numbers.
}

\subsection{Technical Controls}

\begin{itemize}
  \item Anti-Virus / Anti-Malware Software
  \item Firewalls
  \item Intrusion Detection Systems (IDS) / Intrusion Prevention Systems (IPS)
  \item Patch Management
  \item Secure Software Development Practices
\end{itemize}

\subsection{Administrative Controls}

\begin{itemize}
  \item Security Policies on patching
  \item User Training
  \item Software Inventory Management
\end{itemize}

\subsection{Operational Controls}

\begin{itemize}
  \item Incident Response Planning and Policies
  \item  Malware Analysis Teams
  \item Threat Intelligence Sharing
  \item Backups
\end{itemize}

\section{Technical Hardware Failures / Errors}

Occurs when a manufacturer distributes equipment containing a known
or unknown flaw. This primarily compromises \textbf{Availability} and
\textbf{Integrity}. Some examples include:
\begin{itemize}
  \item Pentium II Floating Point Division Bug
  \item Spectre and Meltdown CPU Vulnerabilities
  \item Hardware crashes
  \item RAM failures
\end{itemize}

\subsection{Technical Controls}

\begin{itemize}
  \item Redundancy (e.g RAID)
  \item Error Checking RAM
  \item Uninterruptible Power Supplies (UPS)
\end{itemize}

\subsection{Administrative Controls}

\begin{itemize}
  \item Vendor Risk Assessment
  \item Procurement Policies
  \item Warranty and Maintenance Agreements
\end{itemize}

\subsection{Operational Controls}

\begin{itemize}
  \item Preventative Maintenance Schedules
  \item Hardware Monitoring Tools
  \item Asset Lifecycle Management
\end{itemize}

\section{Technical Software Failures / Errors}

Occurs when software contains undetected bugs or flaws that lead to
vulnerabilities. Some of these common vulnerabilities include:
\begin{description}
  \item[Input and Validation Issues] - Can compromise \textbf{Confidentiality},
    \textbf{Integrity}.
    \begin{itemize}
      \item Buffer overrun
      \item Command injection
      \item Cross-site scripting (XSS)
      \item SQL injection
    \end{itemize}
  \item[Cryptographic and Authentication Failures] - Can compromise
    \textbf{Confidentiality}
    \begin{itemize}
      \item Failure to use cryptographically strong random numbers
      \item Improper use of SSL
      \item Unauthenticated key exchange
    \end{itemize}
  \item[Logic and Control Issues] - Can compromise \textbf{Integrity} and
    \textbf{Availability}
    \begin{itemize}
      \item Catching exceptions
      \item Failure to handle errors
      \item Race Conditions
    \end{itemize}
  \item[Data Protection Failures] - Compromises \textbf{Confidentiality}
    \begin{itemize}
      \item Failure to protect network traffic
      \item Failure to store and protect data securely
      \item  Information leakage
    \end{itemize}
\end{description}

\subsection{Technical Controls}
\begin{itemize}
  \item Static Application Security Testing (SAST)
  \item Dynamic Application Security Testing (DAST)
  \item Runtime Application Self-Protection (RASP)
  \item Web Application Firewalls (WAF)
\end{itemize}

\subsection{Administrative Controls}
\begin{itemize}
  \item Secure Coding Practices
\end{itemize}

\subsection{Operational Controls}
\begin{itemize}
  \item Code reviews
  \item Peer Audits
  \item Penetration Testing
\end{itemize}

\section{Technological Obsolescence}

Old and outdated technology that is no longer supported or maintained
primarily compromises \textbf{Availability} and \textbf{Integrity}.

\subsection{Technical Controls}
\begin{itemize}
  \item Regular Patching
  \item System Upgrades
  \item Migration to Supported Platforms
  \item  Virtualisation
\end{itemize}

\subsection{Administrative Controls}
\begin{itemize}
  \item IT governance frameworks
  \item Lifecycle Management Policies
  \item Vendor Risk Management
\end{itemize}

\subsection{Operational Controls}
\begin{itemize}
  \item Asset Inventory and Monitoring
  \item End of Life (EOL) Planning
  \item Scheduled decommissioning of obsolete systems
\end{itemize}

\section{Theft}

Illegally taking physical, electronic, or intellectual property,
compromises \textbf{Confidentially}.

\subsection{Technical Controls}
\begin{itemize}
  \item Encryption of sensitive data
  \item Access Control
  \item Data Loss Prevention (DLP) solutions
\end{itemize}

\subsection{Administrative Controls}
\begin{itemize}
  \item Security Policies
  \item Employee Background Checks
  \item IP protection agreements
\end{itemize}

\subsection{Operational Controls}
\begin{itemize}
  \item CCTV Surveillance
  \item Security Guards
  \item Asset Tracking
\end{itemize}

\chapter{Threat Modelling}

\section{Introduction}

\dfn{Threat Modelling}{
  A systematic listing of different ways a threat may be realized. In
  this process potential threats are identified, categorized and
  analysed, to identify the potential harm, the probability of
  occurrence (risk), the priority of concern, and security controls,
  or countermeasures to mitigate the threats.
}

Threat modelling is useful as it helps to:
\begin{itemize}
  \item Identify ways to improve software development processes
  \item Improves the system's configuration management and security controls
  \item Identify vulnerabilities in software
  \item Increase user awareness, incident detection and response.
\end{itemize}

\section{Threat Modelling Methodologies}

Threat modelling commonly involves the following tasks:
\begin{description}
  \item[Determining Vulnerabilities]  - Identify potential
    vulnerabilities in the systems that could lead to compromise of
    the CIA triad.
  \item[Determine Adversaries] - Identify potential adversaries,
    their motivations, and capabilities
  \item[Determine Potential Attack Vectors] - Identify whether a
    given threat source has the means to exploit the vulnerability.
\end{description}

\dfn{Risk Assessment}{
  The overall process of identifying, evaluating, and estimating the
  levels of risk involved in a situation, to determine an acceptable
  level of risk and the appropriate ways to eliminate or control the
  risk.
}

Threat modelling can be performed either as a \textbf{proactive}
measure during design and development or a \textbf{reactive} measure
once a product has been deployed.

\subsection{Defensive / Proactive Approach}

Proactive threat modelling takes place during the early stages of
systems development, and involves predicting threats and designing
defences before deployment. This is more cost-effective and
successful as security solutions are integrated in the development process.
Some examples of proactive threat modelling methodologies include:
\begin{description}
  \item[Security Development Lifecycle (SDL)] - Aims to reduce the
    number of security-related design and coding defects, and to
    reduce the severity of any remaining defects.
  \item[STRIDE (Spoofing, Tampering, Repudiation, Information
    Disclosure, Denial of Service, Elevation of Privilege)] - A model
    for identifying computer security threats.
  \item[PASTA (Process for Attack Simulation and Threat Analysis)] - A
    seven-step, risk-centric methodology that aims to align business
    objectives and technical requirements.
\end{description}

\subsubsection{STRIDE}

STRIDE is a mnemonic for a set of threats
\begin{description}
  \item[Spoofing] - An attack with the goal of gaining access to a
    target system using a falsified identity. Compromises
    \textbf{Authentication}
  \item[Tampering] - Any action resulting in unauthorized changes or
    manipulation of data, whether in transit or in storage.
    Compromises \textbf{Integrity}
  \item[Repudiation] - The ability of a user or attacker to deny
    having performed an action or activity by maintaining plausible
    deniability. Compromises \textbf{Non-repudiation}
  \item[Information Disclosure] - The exposure of private,
    confidential, or controlled information to external or
    unauthorized entities. Compromises \textbf{Confidentiality}
  \item[Denial of Service] - An attack that attempts to prevent
    authorized use of a resource. Compromises \textbf{Availability}
  \item[Elevation of Privilege] - An attack where a limited user
    account is transformed into a full privileged user account with
    greater powers and access. Compromises \textbf{Authorization}
\end{description}

\subsubsection{PASTA}

Involves simulating attacks to IT applications analysing the threats
their origin, the risks they pose to an organization, and how to
mitigate them. Organizations can determine the most appropriate
countermeasures that must deployed to mitigate risk. It is a
risk-centric approach that aims at selecting / developing
countermeasures in relation to the value of the assets being protected.

\subsubsection{STRIDE vs PASTA}

\begin{itemize}
  \item PASTA is risk-centric, STRIDE is threat-centric.
  \item PASTA is a seven-step process, STRIDE is a categorization
    model.
  \item PASTA focuses on simulating attacks, STRIDE focuses on identifying
    threats.
  \item PASTA is an end-to-end methodology, STRIDE is a component of threat
    modelling.
\end{itemize}

\subsection{Adversarial / Reactive Approach}

Reactive threat modelling takes place after a product has been
created / deployed. It is the core concept behind ethical hacking
penetration testing, source code review, and fuzz testing.

\dfn{Fuzzing}{
  An automated software testing method that injects invalid,
  malformed or unexpected inputs into a system to reveal software
  defects and vulnerabilities.
}

\section{Threat Intelligence}

\dfn{Threat Intelligence}{
  Knowledge that allows the prevention or mitigation of cyber-attacks
  by studying the threat data and providing information on adversaries.
}

Threat intelligence is useful as it helps to:
\begin{itemize}
  \item Identify, prepare and prevent attacks by providing
    information on attackers, their motives and capabilities.
  \item Prepares organizations to be proactive with predictive capabilities
  \item Identify, classify and prioritize threats to ensure effective
    documentation and reporting.
\end{itemize}

\section{Reduction Analysis / Decomposition}

\dfn{Reduction Analysis / Decomposition}{
  A method of breaking down a complex system into smaller, more
  manageable components to gain a greater understanding of the logic
  of the product, its internal components and its interactions with
  external elements.
}

Reduction Analysis focuses on
\begin{description}
  \item[Software] - Subroutines, models, or objects
  \item[System] - Computers and Operating Systems
  \item[Network] - Protocols
  \item[Business infrastructure] - Departments, tasks and networks.
\end{description}

Reduction must identify five key concepts:
\begin{description}
  \item[Trust Boundaries] - Points in a system where data or control
    flows from one component to another, and where the level of trust
    changes.
  \item[Data Flows] - The movement of data between components in a system.
  \item[Inputs] - The data or control signals that enter a system or component.
  \item[Privileged Functions] - Operations or actions that require elevated
    permissions or access rights to perform.
  \item[Security Approach]
\end{description}

\section{Prioritization and Response}

Once threats have been identified, they must be prioritized based on
a specified technique.
Some techniques include:
\begin{itemize}
  \item Probability / Damage matrix
  \item High / Medium / Low (H/M/L) rating
  \item DREAD (Damage, Reproducibility, Exploitability, Affected Users,
    Discoverability)
\end{itemize}

\subsection{Probability / Damage Matrix}

Produces a risk severity number on a scale from 1 to 100 by multiplying
the probability of occurrence (1-10) by the potential damage (1-10).

\subsection{H/M/L Rating}

Classifies threats into three categories:
\begin{description}
  \item[High] - Threats that are likely to occur and could cause
    significant damage.
  \item[Medium] - Threats that have a moderate likelihood of occurring
    and could cause moderate damage.
  \item[Low] - Threats that are unlikely to occur and would cause
    minimal damage.
\end{description}

\subsection{DREAD}

DREAD stands for:
\begin{description}
  \item[Damage Potential]  - How severe is the damage likely yo be if
    the threat is realized?
  \item[Reproducibility] - How complicated is it for attackers to
    reproduce the attack?
  \item[Exploitability] - How hard is it to launch the attack?
  \item[Affected Users] - How many users are likely to be affected by
    the attack?
  \item[Discoverability] - How easy is it for an attacker to discover the
    vulnerability?
\end{description}

\chapter{Risk Management}

\dfn{Risk}{
  The probability of an unwanted occurrence such as an adverse event or loss.
}

\dfn{Risk Management}{
  The process of identifying, assessing, and controlling risks facing
  an organization.
}

\section{Components of Risk Management}
Risk management involves the following components:
\begin{description}
  \item[Risk Identification]  - The process of examining an
    organization's current IT security situation to recognize
    enumerate and document the risks to its IT assets.
  \item[Risk Assessment] - Determination of the extent to which an
    organization's IT assets are exposed to risk.
  \item[Risk Control] - Application of controls that reduce the risks
    to an organization's ID assets to an acceptable level.
\end{description}

\dfn{Risk Appetite / Tolerance}{
  The quantity and nature of risk that an organization is willing to
  accept as they evaluate the trade-offs between risk and cost. The
  general rule is to never spend more to protect an asset than the
  asset is worth.
}

\dfn{Residual Risk}{
  Risk that is left over after the risk management process has
  concluded. The reaming risk that has not been removed, shifted or
  planned for when vulnerabilities have been controlled as much as possible.
}

Residual risk is a combined function of:
\begin{itemize}
  \item A threat minus the effect of threat-reducing controls
  \item A vulnerability minus the effect of vulnerability-reducing controls
  \item An asset minus the effect of asset value-protecting controls
\end{itemize}

The goal of risk management is to reduce the residual risk to a level
within an organization's risk appetite, I.e less than or equal to the
risk tolerance.

\subsection{Risk Identification}

Risk identification involves the following steps:
\begin{description}
  \item[Plan and Organize] -
  \item[Identify, inventory and categorize assets]
  \item[Classify, value and prioritize assets]
  \item[Identify and prioritize threats]
  \item[Specify asset vulnerabilities]
\end{description}

\end{document}
