\documentclass[12pt letter]{report}
%%%%%%%%%%%%%%%%%%%%%%%%%%%%%%%%%
% PACKAGE IMPORTS
%%%%%%%%%%%%%%%%%%%%%%%%%%%%%%%%%


\usepackage[tmargin=2cm,rmargin=1in,lmargin=1in,margin=0.85in,bmargin=2cm,footskip=.2in]{geometry}
\usepackage{amsmath,amsfonts,amsthm,amssymb,mathtools}
\usepackage[varbb]{newpxmath}
\usepackage{xfrac}
\usepackage[makeroom]{cancel}
\usepackage{mathtools}
\usepackage{bookmark}
\usepackage{enumitem}
\usepackage{hyperref,theoremref}
\hypersetup{
  pdftitle={Assignment},
  colorlinks=true, linkcolor=doc!90,
  bookmarksnumbered=true,
  bookmarksopen=true
}
\usepackage[most,many,breakable]{tcolorbox}
\usepackage{xcolor}
\usepackage{varwidth}
\usepackage{varwidth}
\usepackage{etoolbox}
%\usepackage{authblk}
\usepackage{nameref}
\usepackage{multicol,array}
\usepackage{tikz-cd}
\usepackage[ruled,vlined,linesnumbered]{algorithm2e}
\usepackage{comment} % enables the use of multi-line comments (\ifx \fi) 
\usepackage{import}
\usepackage{xifthen}
\usepackage{pdfpages}
\usepackage{transparent}
\usepackage{xcolor,colortbl,array,amssymb}
\usepackage{venndiagram}
\usepackage{listings}
\usepackage{fontspec}

\setmainfont{LibertinusSerif}[
  Extension = .otf,
  Path = /usr/share/fonts/libertinus/,
  UprightFont = *-Regular,
  ItalicFont = *-Italic,
  BoldFont = *-Bold,
  BoldItalicFont = *-BoldItalic,
]

\setmonofont{CaskaydiaCoveNerdFontMono}[
  Extension = .ttf,
  Path = /usr/share/fonts/TTF/,
  UprightFont = *-Regular,
  ItalicFont = *-Italic,
  BoldFont = *-Bold,
  BoldItalicFont = *-BoldItalic,
]

\definecolor{listing-background}{HTML}{F7F7F7}
\definecolor{listing-rule}{HTML}{B3B2B3}
\definecolor{listing-numbers}{HTML}{B3B2B3}
\definecolor{listing-text-color}{HTML}{000000}
\definecolor{listing-keyword}{HTML}{435489}
\definecolor{listing-keyword-2}{HTML}{1284CA} % additional keywords
\definecolor{listing-keyword-3}{HTML}{9137CB} % additional keywords
\definecolor{listing-identifier}{HTML}{435489}
\definecolor{listing-string}{HTML}{00999A}
\definecolor{listing-comment}{HTML}{8E8E8E}

\lstdefinestyle{eisvogel_listing_style}{
language         = java,
xleftmargin      = 0.6em,
framexleftmargin = 0.4em,
backgroundcolor  = \color{listing-background},
basicstyle       = \color{listing-text-color}\linespread{1.0}%
\lst@ifdisplaystyle%
\fi\ttfamily{},
breaklines       = true,
frame            = single,
framesep         = 0.19em,
rulecolor        = \color{listing-rule},
frameround       = ffff,
tabsize          = 4,
numberstyle      = \color{listing-numbers},
aboveskip        = 1.0em,
belowskip        = 0.1em,
abovecaptionskip = 0em,
belowcaptionskip = 1.0em,
keywordstyle     = {\color{listing-keyword}\bfseries},
keywordstyle     = {[2]\color{listing-keyword-2}\bfseries},
keywordstyle     = {[3]\color{listing-keyword-3}\bfseries\itshape},
sensitive        = true,
identifierstyle  = \color{listing-identifier},
commentstyle     = \color{listing-comment},
stringstyle      = \color{listing-string},
showstringspaces = false,
escapeinside     = {/*@}{@*/}, % Allow LaTeX inside these special comments
literate         =
  {á}{{\'a}}1 {é}{{\'e}}1 {í}{{\'i}}1 {ó}{{\'o}}1 {ú}{{\'u}}1
{Á}{{\'A}}1 {É}{{\'E}}1 {Í}{{\'I}}1 {Ó}{{\'O}}1 {Ú}{{\'U}}1
{à}{{\`a}}1 {è}{{\`e}}1 {ì}{{\`i}}1 {ò}{{\`o}}1 {ù}{{\`u}}1
{À}{{\`A}}1 {È}{{\`E}}1 {Ì}{{\`I}}1 {Ò}{{\`O}}1 {Ù}{{\`U}}1
{ä}{{\"a}}1 {ë}{{\"e}}1 {ï}{{\"i}}1 {ö}{{\"o}}1 {ü}{{\"u}}1
{Ä}{{\"A}}1 {Ë}{{\"E}}1 {Ï}{{\"I}}1 {Ö}{{\"O}}1 {Ü}{{\"U}}1
{â}{{\^a}}1 {ê}{{\^e}}1 {î}{{\^i}}1 {ô}{{\^o}}1 {û}{{\^u}}1
{Â}{{\^A}}1 {Ê}{{\^E}}1 {Î}{{\^I}}1 {Ô}{{\^O}}1 {Û}{{\^U}}1
{œ}{{\oe}}1 {Œ}{{\OE}}1 {æ}{{\ae}}1 {Æ}{{\AE}}1 {ß}{{\ss}}1
{ç}{{\c c}}1 {Ç}{{\c C}}1 {ø}{{\o}}1 {å}{{\r a}}1 {Å}{{\r A}}1
{€}{{\EUR}}1 {£}{{\pounds}}1 {«}{{\guillemotleft}}1
{»}{{\guillemotright}}1 {ñ}{{\~n}}1 {Ñ}{{\~N}}1 {¿}{{?`}}1
{…}{{\ldots}}1 {≥}{{>=}}1 {≤}{{<=}}1 {„}{{\glqq}}1 {“}{{\grqq}}1
{”}{{''}}1
}
\lstset{style=eisvogel_listing_style,
  numbers=left}

%
% Java (Java SE 12, 2019-06-22)
%
\lstdefinelanguage{Java}{
  morekeywords={
      % normal keywords (without data types)
      abstract,assert,break,case,catch,class,continue,default,
      do,else,enum,exports,extends,final,finally,for,if,implements,
      import,instanceof,interface,module,native,new,package,private,
      protected,public,requires,return,static,strictfp,super,switch,
      synchronized,this,throw,throws,transient,try,volatile,while,
      % var is an identifier
      var
    },
  morekeywords={[2] % data types
      % primitive data types
      boolean,byte,char,double,float,int,long,short,
      % String
      String,
      % primitive wrapper types
      Boolean,Byte,Character,Double,Float,Integer,Long,Short
      % number types
      Number,AtomicInteger,AtomicLong,BigDecimal,BigInteger,DoubleAccumulator,DoubleAdder,LongAccumulator,LongAdder,Short,
      % other
      Object,Void,void
    },
  morekeywords={[3] % literals
      % reserved words for literal values
      null,true,false,
    },
  sensitive,
  morecomment  = [l]//,
  morecomment  = [s]{/*}{*/},
  morecomment  = [s]{/**}{*/},
  morestring   = [b]",
  morestring   = [b]',
}

\lstdefinelanguage{XML}{
  morestring      = [b]",
  moredelim       = [s][\bfseries\color{listing-keyword}]{<}{\ },
  moredelim       = [s][\bfseries\color{listing-keyword}]{</}{>},
  moredelim       = [l][\bfseries\color{listing-keyword}]{/>},
  moredelim       = [l][\bfseries\color{listing-keyword}]{>},
  morecomment     = [s]{<?}{?>},
  morecomment     = [s]{<!--}{-->},
  commentstyle    = \color{listing-comment},
  stringstyle     = \color{listing-string},
  identifierstyle = \color{listing-identifier}
}


\newcommand\mycommfont[1]{\footnotesize\ttfamily\textcolor{blue}{#1}}
\SetCommentSty{mycommfont}
\newcommand{\incfig}[1]{%
  \def\svgwidth{\columnwidth}
  \import{./figures/}{#1.pdf_tex}
}

\usepackage{tikzsymbols}
\renewcommand\qedsymbol{$\Laughey$}


%\usepackage{import}
%\usepackage{xifthen}
%\usepackage{pdfpages}
%\usepackage{transparent}


%%%%%%%%%%%%%%%%%%%%%%%%%%%%%%
% SELF MADE COLORS
%%%%%%%%%%%%%%%%%%%%%%%%%%%%%%



\definecolor{myg}{RGB}{56, 140, 70}
\definecolor{myb}{RGB}{45, 111, 177}
\definecolor{myr}{RGB}{199, 68, 64}
\definecolor{mytheorembg}{HTML}{F2F2F9}
\definecolor{mytheoremfr}{HTML}{00007B}
\definecolor{mylenmabg}{HTML}{FFFAF8}
\definecolor{mylenmafr}{HTML}{983b0f}
\definecolor{mypropbg}{HTML}{f2fbfc}
\definecolor{mypropfr}{HTML}{191971}
\definecolor{myexamplebg}{HTML}{F2FBF8}
\definecolor{myexamplefr}{HTML}{88D6D1}
\definecolor{myexampleti}{HTML}{2A7F7F}
\definecolor{mydefinitbg}{HTML}{E5E5FF}
\definecolor{mydefinitfr}{HTML}{3F3FA3}
\definecolor{notesgreen}{RGB}{0,162,0}
\definecolor{myp}{RGB}{197, 92, 212}
\definecolor{mygr}{HTML}{2C3338}
\definecolor{myred}{RGB}{127,0,0}
\definecolor{myyellow}{RGB}{169,121,69}
\definecolor{myexercisebg}{HTML}{F2FBF8}
\definecolor{myexercisefg}{HTML}{88D6D1}


%%%%%%%%%%%%%%%%%%%%%%%%%%%%
% TCOLORBOX SETUPS
%%%%%%%%%%%%%%%%%%%%%%%%%%%%

\setlength{\parindent}{1cm}
%================================
% THEOREM BOX
%================================

\tcbuselibrary{theorems,skins,hooks}
\newtcbtheorem[number within=section]{Theorem}{Theorem}
{%
  enhanced,
  breakable,
  colback = mytheorembg,
  frame hidden,
  boxrule = 0sp,
  borderline west = {2pt}{0pt}{mytheoremfr},
  sharp corners,
  detach title,
  before upper = \tcbtitle\par\smallskip,
  coltitle = mytheoremfr,
  fonttitle = \bfseries\sffamily,
  description font = \mdseries,
  separator sign none,
  segmentation style={solid, mytheoremfr},
}
{th}

\tcbuselibrary{theorems,skins,hooks}
\newtcbtheorem[number within=chapter]{theorem}{Theorem}
{%
  enhanced,
  breakable,
  colback = mytheorembg,
  frame hidden,
  boxrule = 0sp,
  borderline west = {2pt}{0pt}{mytheoremfr},
  sharp corners,
  detach title,
  before upper = \tcbtitle\par\smallskip,
  coltitle = mytheoremfr,
  fonttitle = \bfseries\sffamily,
  description font = \mdseries,
  separator sign none,
  segmentation style={solid, mytheoremfr},
}
{th}


\tcbuselibrary{theorems,skins,hooks}
\newtcolorbox{Theoremcon}
{%
  enhanced
  ,breakable
  ,colback = mytheorembg
  ,frame hidden
  ,boxrule = 0sp
  ,borderline west = {2pt}{0pt}{mytheoremfr}
  ,sharp corners
  ,description font = \mdseries
  ,separator sign none
}

%================================
% Corollery
%================================
\tcbuselibrary{theorems,skins,hooks}
\newtcbtheorem[number within=section]{Corollary}{Corollary}
{%
  enhanced
  ,breakable
  ,colback = myp!10
  ,frame hidden
  ,boxrule = 0sp
  ,borderline west = {2pt}{0pt}{myp!85!black}
  ,sharp corners
  ,detach title
  ,before upper = \tcbtitle\par\smallskip
  ,coltitle = myp!85!black
  ,fonttitle = \bfseries\sffamily
  ,description font = \mdseries
  ,separator sign none
  ,segmentation style={solid, myp!85!black}
}
{th}
\tcbuselibrary{theorems,skins,hooks}
\newtcbtheorem[number within=chapter]{corollary}{Corollary}
{%
  enhanced
  ,breakable
  ,colback = myp!10
  ,frame hidden
  ,boxrule = 0sp
  ,borderline west = {2pt}{0pt}{myp!85!black}
  ,sharp corners
  ,detach title
  ,before upper = \tcbtitle\par\smallskip
  ,coltitle = myp!85!black
  ,fonttitle = \bfseries\sffamily
  ,description font = \mdseries
  ,separator sign none
  ,segmentation style={solid, myp!85!black}
}
{th}


%================================
% LENMA
%================================

\tcbuselibrary{theorems,skins,hooks}
\newtcbtheorem[number within=section]{Lenma}{Lenma}
{%
  enhanced,
  breakable,
  colback = mylenmabg,
  frame hidden,
  boxrule = 0sp,
  borderline west = {2pt}{0pt}{mylenmafr},
  sharp corners,
  detach title,
  before upper = \tcbtitle\par\smallskip,
  coltitle = mylenmafr,
  fonttitle = \bfseries\sffamily,
  description font = \mdseries,
  separator sign none,
  segmentation style={solid, mylenmafr},
}
{th}

\tcbuselibrary{theorems,skins,hooks}
\newtcbtheorem[number within=chapter]{lenma}{Lenma}
{%
  enhanced,
  breakable,
  colback = mylenmabg,
  frame hidden,
  boxrule = 0sp,
  borderline west = {2pt}{0pt}{mylenmafr},
  sharp corners,
  detach title,
  before upper = \tcbtitle\par\smallskip,
  coltitle = mylenmafr,
  fonttitle = \bfseries\sffamily,
  description font = \mdseries,
  separator sign none,
  segmentation style={solid, mylenmafr},
}
{th}


%================================
% PROPOSITION
%================================

\tcbuselibrary{theorems,skins,hooks}
\newtcbtheorem[number within=section]{Prop}{Proposition}
{%
  enhanced,
  breakable,
  colback = mypropbg,
  frame hidden,
  boxrule = 0sp,
  borderline west = {2pt}{0pt}{mypropfr},
  sharp corners,
  detach title,
  before upper = \tcbtitle\par\smallskip,
  coltitle = mypropfr,
  fonttitle = \bfseries\sffamily,
  description font = \mdseries,
  separator sign none,
  segmentation style={solid, mypropfr},
}
{th}

\tcbuselibrary{theorems,skins,hooks}
\newtcbtheorem[number within=chapter]{prop}{Proposition}
{%
  enhanced,
  breakable,
  colback = mypropbg,
  frame hidden,
  boxrule = 0sp,
  borderline west = {2pt}{0pt}{mypropfr},
  sharp corners,
  detach title,
  before upper = \tcbtitle\par\smallskip,
  coltitle = mypropfr,
  fonttitle = \bfseries\sffamily,
  description font = \mdseries,
  separator sign none,
  segmentation style={solid, mypropfr},
}
{th}


%================================
% CLAIM
%================================

\tcbuselibrary{theorems,skins,hooks}
\newtcbtheorem[number within=section]{claim}{Claim}
{%
  enhanced
  ,breakable
  ,colback = myg!10
  ,frame hidden
  ,boxrule = 0sp
  ,borderline west = {2pt}{0pt}{myg}
  ,sharp corners
  ,detach title
  ,before upper = \tcbtitle\par\smallskip
  ,coltitle = myg!85!black
  ,fonttitle = \bfseries\sffamily
  ,description font = \mdseries
  ,separator sign none
  ,segmentation style={solid, myg!85!black}
}
{th}



%================================
% Exercise
%================================

\tcbuselibrary{theorems,skins,hooks}
\newtcbtheorem[number within=section]{Exercise}{Exercise}
{%
  enhanced,
  breakable,
  colback = myexercisebg,
  frame hidden,
  boxrule = 0sp,
  borderline west = {2pt}{0pt}{myexercisefg},
  sharp corners,
  detach title,
  before upper = \tcbtitle\par\smallskip,
  coltitle = myexercisefg,
  fonttitle = \bfseries\sffamily,
  description font = \mdseries,
  separator sign none,
  segmentation style={solid, myexercisefg},
}
{th}

\tcbuselibrary{theorems,skins,hooks}
\newtcbtheorem[number within=chapter]{exercise}{Exercise}
{%
  enhanced,
  breakable,
  colback = myexercisebg,
  frame hidden,
  boxrule = 0sp,
  borderline west = {2pt}{0pt}{myexercisefg},
  sharp corners,
  detach title,
  before upper = \tcbtitle\par\smallskip,
  coltitle = myexercisefg,
  fonttitle = \bfseries\sffamily,
  description font = \mdseries,
  separator sign none,
  segmentation style={solid, myexercisefg},
}
{th}

%================================
% EXAMPLE BOX
%================================

\newtcbtheorem[number within=section]{Example}{Example}
{%
  colback = myexamplebg
  ,breakable
  ,colframe = myexamplefr
  ,coltitle = myexampleti
  ,boxrule = 1pt
  ,sharp corners
  ,detach title
  ,before upper=\tcbtitle\par\smallskip
  ,fonttitle = \bfseries
  ,description font = \mdseries
  ,separator sign none
  ,description delimiters parenthesis
}
{ex}

\newtcbtheorem[number within=chapter]{example}{Example}
{%
  colback = myexamplebg
  ,breakable
  ,colframe = myexamplefr
  ,coltitle = myexampleti
  ,boxrule = 1pt
  ,sharp corners
  ,detach title
  ,before upper=\tcbtitle\par\smallskip
  ,fonttitle = \bfseries
  ,description font = \mdseries
  ,separator sign none
  ,description delimiters parenthesis
}
{ex}

%================================
% DEFINITION BOX
%================================

\newtcbtheorem[number within=section]{Definition}{Definition}{enhanced,
  before skip=2mm,after skip=2mm, colback=red!5,colframe=red!80!black,boxrule=0.5mm,
  attach boxed title to top left={xshift=1cm,yshift*=1mm-\tcboxedtitleheight}, varwidth boxed title*=-3cm,
  boxed title style={frame code={
          \path[fill=tcbcolback]
          ([yshift=-1mm,xshift=-1mm]frame.north west)
          arc[start angle=0,end angle=180,radius=1mm]
          ([yshift=-1mm,xshift=1mm]frame.north east)
          arc[start angle=180,end angle=0,radius=1mm];
          \path[left color=tcbcolback!60!black,right color=tcbcolback!60!black,
            middle color=tcbcolback!80!black]
          ([xshift=-2mm]frame.north west) -- ([xshift=2mm]frame.north east)
          [rounded corners=1mm]-- ([xshift=1mm,yshift=-1mm]frame.north east)
          -- (frame.south east) -- (frame.south west)
          -- ([xshift=-1mm,yshift=-1mm]frame.north west)
          [sharp corners]-- cycle;
        },interior engine=empty,
    },
  fonttitle=\bfseries,
  title={#2},#1}{def}
\newtcbtheorem[number within=chapter]{definition}{Definition}{enhanced,
  before skip=2mm,after skip=2mm, colback=red!5,colframe=red!80!black,boxrule=0.5mm,
  attach boxed title to top left={xshift=1cm,yshift*=1mm-\tcboxedtitleheight}, varwidth boxed title*=-3cm,
  boxed title style={frame code={
          \path[fill=tcbcolback]
          ([yshift=-1mm,xshift=-1mm]frame.north west)
          arc[start angle=0,end angle=180,radius=1mm]
          ([yshift=-1mm,xshift=1mm]frame.north east)
          arc[start angle=180,end angle=0,radius=1mm];
          \path[left color=tcbcolback!60!black,right color=tcbcolback!60!black,
            middle color=tcbcolback!80!black]
          ([xshift=-2mm]frame.north west) -- ([xshift=2mm]frame.north east)
          [rounded corners=1mm]-- ([xshift=1mm,yshift=-1mm]frame.north east)
          -- (frame.south east) -- (frame.south west)
          -- ([xshift=-1mm,yshift=-1mm]frame.north west)
          [sharp corners]-- cycle;
        },interior engine=empty,
    },
  fonttitle=\bfseries,
  title={#2},#1}{def}



%================================
% Solution BOX
%================================

\makeatletter
\newtcbtheorem{question}{Question}{enhanced,
  breakable,
  colback=white,
  colframe=myb!80!black,
  attach boxed title to top left={yshift*=-\tcboxedtitleheight},
  fonttitle=\bfseries,
  title={#2},
  boxed title size=title,
  boxed title style={%
      sharp corners,
      rounded corners=northwest,
      colback=tcbcolframe,
      boxrule=0pt,
    },
  underlay boxed title={%
      \path[fill=tcbcolframe] (title.south west)--(title.south east)
      to[out=0, in=180] ([xshift=5mm]title.east)--
      (title.center-|frame.east)
      [rounded corners=\kvtcb@arc] |-
      (frame.north) -| cycle;
    },
  #1
}{def}
\makeatother

%================================
% SOLUTION BOX
%================================

\makeatletter
\newtcolorbox{solution}{enhanced,
  breakable,
  colback=white,
  colframe=myg!80!black,
  attach boxed title to top left={yshift*=-\tcboxedtitleheight},
  title=Solution,
  boxed title size=title,
  boxed title style={%
      sharp corners,
      rounded corners=northwest,
      colback=tcbcolframe,
      boxrule=0pt,
    },
  underlay boxed title={%
      \path[fill=tcbcolframe] (title.south west)--(title.south east)
      to[out=0, in=180] ([xshift=5mm]title.east)--
      (title.center-|frame.east)
      [rounded corners=\kvtcb@arc] |-
      (frame.north) -| cycle;
    },
}
\makeatother

%================================
% Question BOX
%================================

\makeatletter
\newtcbtheorem{qstion}{Question}{enhanced,
  breakable,
  colback=white,
  colframe=mygr,
  attach boxed title to top left={yshift*=-\tcboxedtitleheight},
  fonttitle=\bfseries,
  title={#2},
  boxed title size=title,
  boxed title style={%
      sharp corners,
      rounded corners=northwest,
      colback=tcbcolframe,
      boxrule=0pt,
    },
  underlay boxed title={%
      \path[fill=tcbcolframe] (title.south west)--(title.south east)
      to[out=0, in=180] ([xshift=5mm]title.east)--
      (title.center-|frame.east)
      [rounded corners=\kvtcb@arc] |-
      (frame.north) -| cycle;
    },
  #1
}{def}
\makeatother

\newtcbtheorem[number within=chapter]{wconc}{Wrong Concept}{
  breakable,
  enhanced,
  colback=white,
  colframe=myr,
  arc=0pt,
  outer arc=0pt,
  fonttitle=\bfseries\sffamily\large,
  colbacktitle=myr,
  attach boxed title to top left={},
  boxed title style={
      enhanced,
      skin=enhancedfirst jigsaw,
      arc=3pt,
      bottom=0pt,
      interior style={fill=myr}
    },
  #1
}{def}



%================================
% NOTE BOX
%================================

\usetikzlibrary{arrows,calc,shadows.blur}
\tcbuselibrary{skins}
\newtcolorbox{note}[1][]{%
  enhanced jigsaw,
  colback=gray!20!white,%
  colframe=gray!80!black,
  size=small,
  boxrule=1pt,
  title=\textbf{Note:-},
  halign title=flush center,
  coltitle=black,
  breakable,
  drop shadow=black!50!white,
  attach boxed title to top left={xshift=1cm,yshift=-\tcboxedtitleheight/2,yshifttext=-\tcboxedtitleheight/2},
  minipage boxed title=1.5cm,
  boxed title style={%
      colback=white,
      size=fbox,
      boxrule=1pt,
      boxsep=2pt,
      underlay={%
          \coordinate (dotA) at ($(interior.west) + (-0.5pt,0)$);
          \coordinate (dotB) at ($(interior.east) + (0.5pt,0)$);
          \begin{scope}
            \clip (interior.north west) rectangle ([xshift=3ex]interior.east);
            \filldraw [white, blur shadow={shadow opacity=60, shadow yshift=-.75ex}, rounded corners=2pt] (interior.north west) rectangle (interior.south east);
          \end{scope}
          \begin{scope}[gray!80!black]
            \fill (dotA) circle (2pt);
            \fill (dotB) circle (2pt);
          \end{scope}
        },
    },
  #1,
}

%%%%%%%%%%%%%%%%%%%%%%%%%%%%%%
% SELF MADE COMMANDS
%%%%%%%%%%%%%%%%%%%%%%%%%%%%%%


\newcommand{\thm}[2]{\begin{Theorem}{#1}{}#2\end{Theorem}}
\newcommand{\cor}[2]{\begin{Corollary}{#1}{}#2\end{Corollary}}
\newcommand{\mlenma}[2]{\begin{Lenma}{#1}{}#2\end{Lenma}}
\newcommand{\mprop}[2]{\begin{Prop}{#1}{}#2\end{Prop}}
\newcommand{\clm}[3]{\begin{claim}{#1}{#2}#3\end{claim}}
\newcommand{\wc}[2]{\begin{wconc}{#1}{}\setlength{\parindent}{1cm}#2\end{wconc}}
\newcommand{\thmcon}[1]{\begin{Theoremcon}{#1}\end{Theoremcon}}
\newcommand{\ex}[2]{\begin{Example}{#1}{}#2\end{Example}}
\newcommand{\dfn}[2]{\begin{Definition}[colbacktitle=red!75!black]{#1}{}#2\end{Definition}}
\newcommand{\dfnc}[2]{\begin{definition}[colbacktitle=red!75!black]{#1}{}#2\end{definition}}
\newcommand{\qs}[2]{\begin{question}{#1}{}#2\end{question}}
\newcommand{\pf}[2]{\begin{myproof}[#1]#2\end{myproof}}
\newcommand{\nt}[1]{\begin{note}#1\end{note}}

\newcommand*\circled[1]{\tikz[baseline=(char.base)]{
    \node[shape=circle,draw,inner sep=1pt] (char) {#1};}}
\newcommand\getcurrentref[1]{%
  \ifnumequal{\value{#1}}{0}
  {??}
  {\the\value{#1}}%
}
\newcommand{\getCurrentSectionNumber}{\getcurrentref{section}}
\newenvironment{myproof}[1][\proofname]{%
  \proof[\bfseries #1: ]%
}{\endproof}

\newcommand{\mclm}[2]{\begin{myclaim}[#1]#2\end{myclaim}}
\newenvironment{myclaim}[1][\claimname]{\proof[\bfseries #1: ]}{}

\newcounter{mylabelcounter}

\makeatletter
\newcommand{\setword}[2]{%
  \phantomsection
  #1\def\@currentlabel{\unexpanded{#1}}\label{#2}%
}
\makeatother




\tikzset{
  symbol/.style={
      draw=none,
      every to/.append style={
          edge node={node [sloped, allow upside down, auto=false]{$#1$}}}
    }
}


% deliminators
\DeclarePairedDelimiter{\abs}{\lvert}{\rvert}
\DeclarePairedDelimiter{\norm}{\lVert}{\rVert}

\DeclarePairedDelimiter{\ceil}{\lceil}{\rceil}
\DeclarePairedDelimiter{\floor}{\lfloor}{\rfloor}
\DeclarePairedDelimiter{\round}{\lfloor}{\rceil}

\newsavebox\diffdbox
\newcommand{\slantedromand}{{\mathpalette\makesl{d}}}
\newcommand{\makesl}[2]{%
  \begingroup
  \sbox{\diffdbox}{$\mathsurround=0pt#1\mathrm{#2}$}%
  \pdfsave
  \pdfsetmatrix{1 0 0.2 1}%
  \rlap{\usebox{\diffdbox}}%
  \pdfrestore
  \hskip\wd\diffdbox
  \endgroup
}
\newcommand{\dd}[1][]{\ensuremath{\mathop{}\!\ifstrempty{#1}{%
      \slantedromand\@ifnextchar^{\hspace{0.2ex}}{\hspace{0.1ex}}}%
    {\slantedromand\hspace{0.2ex}^{#1}}}}
\ProvideDocumentCommand\dv{o m g}{%
  \ensuremath{%
    \IfValueTF{#3}{%
      \IfNoValueTF{#1}{%
        \frac{\dd #2}{\dd #3}%
      }{%
        \frac{\dd^{#1} #2}{\dd #3^{#1}}%
      }%
    }{%
      \IfNoValueTF{#1}{%
        \frac{\dd}{\dd #2}%
      }{%
        \frac{\dd^{#1}}{\dd #2^{#1}}%
      }%
    }%
  }%
}
\providecommand*{\pdv}[3][]{\frac{\partial^{#1}#2}{\partial#3^{#1}}}
%  - others
\DeclareMathOperator{\Lap}{\mathcal{L}}
\DeclareMathOperator{\Var}{Var} % varience
\DeclareMathOperator{\Cov}{Cov} % covarience
\DeclareMathOperator{\E}{E} % expected

% Since the amsthm package isn't loaded

% I prefer the slanted \leq
\let\oldleq\leq % save them in case they're every wanted
\let\oldgeq\geq
\renewcommand{\leq}{\leqslant}
\renewcommand{\geq}{\geqslant}

% % redefine matrix env to allow for alignment, use r as default
% \renewcommand*\env@matrix[1][r]{\hskip -\arraycolsep
%     \let\@ifnextchar\new@ifnextchar
%     \array{*\c@MaxMatrixCols #1}}


%\usepackage{framed}
%\usepackage{titletoc}
%\usepackage{etoolbox}
%\usepackage{lmodern}


%\patchcmd{\tableofcontents}{\contentsname}{\sffamily\contentsname}{}{}

%\renewenvironment{leftbar}
%{\def\FrameCommand{\hspace{6em}%
%		{\color{myyellow}\vrule width 2pt depth 6pt}\hspace{1em}}%
%	\MakeFramed{\parshape 1 0cm \dimexpr\textwidth-6em\relax\FrameRestore}\vskip2pt%
%}
%{\endMakeFramed}

%\titlecontents{chapter}
%[0em]{\vspace*{2\baselineskip}}
%{\parbox{4.5em}{%
%		\hfill\Huge\sffamily\bfseries\color{myred}\thecontentspage}%
%	\vspace*{-2.3\baselineskip}\leftbar\textsc{\small\chaptername~\thecontentslabel}\\\sffamily}
%{}{\endleftbar}
%\titlecontents{section}
%[8.4em]
%{\sffamily\contentslabel{3em}}{}{}
%{\hspace{0.5em}\nobreak\itshape\color{myred}\contentspage}
%\titlecontents{subsection}
%[8.4em]
%{\sffamily\contentslabel{3em}}{}{}  
%{\hspace{0.5em}\nobreak\itshape\color{myred}\contentspage}



%%%%%%%%%%%%%%%%%%%%%%%%%%%%%%%%%%%%%%%%%%%
% TABLE OF CONTENTS
%%%%%%%%%%%%%%%%%%%%%%%%%%%%%%%%%%%%%%%%%%%

\usepackage{tikz}
\definecolor{doc}{RGB}{0,60,110}
\usepackage{titletoc}
\contentsmargin{0cm}
\titlecontents{chapter}[3.7pc]
{\addvspace{30pt}%
  \begin{tikzpicture}[remember picture, overlay]%
    \draw[fill=doc!60,draw=doc!60] (-7,-.1) rectangle (-0.9,.5);%
    \pgftext[left,x=-3.5cm,y=0.2cm]{\color{white}\Large\sc\bfseries Chapter\ \thecontentslabel};%
  \end{tikzpicture}\color{doc!60}\large\sc\bfseries}%
{}
{}
{\;\titlerule\;\large\sc\bfseries Page \thecontentspage
  \begin{tikzpicture}[remember picture, overlay]
    \draw[fill=doc!60,draw=doc!60] (2pt,0) rectangle (4,0.1pt);
  \end{tikzpicture}}%
\titlecontents{section}[3.7pc]
{\addvspace{2pt}}
{\contentslabel[\thecontentslabel]{2pc}}
{}
{\hfill\small \thecontentspage}
[]
\titlecontents*{subsection}[3.7pc]
{\addvspace{-1pt}\small}
{}
{}
{\ --- \small\thecontentspage}
[ \textbullet\ ][]

\makeatletter
\renewcommand{\tableofcontents}{%
  \chapter*{%
    \vspace*{-20\p@}%
    \begin{tikzpicture}[remember picture, overlay]%
      \pgftext[right,x=15cm,y=0.2cm]{\color{doc!60}\Huge\sc\bfseries \contentsname};%
      \draw[fill=doc!60,draw=doc!60] (13,-.75) rectangle (20,1);%
      \clip (13,-.75) rectangle (20,1);
      \pgftext[right,x=15cm,y=0.2cm]{\color{white}\Huge\sc\bfseries \contentsname};%
    \end{tikzpicture}}%
  \@starttoc{toc}}
\makeatother

%From M275 "Topology" at SJSU
\newcommand{\id}{\mathrm{id}}
\newcommand{\taking}[1]{\xrightarrow{#1}}
\newcommand{\inv}{^{-1}}

%From M170 "Introduction to Graph Theory" at SJSU
\DeclareMathOperator{\diam}{diam}
\DeclareMathOperator{\ord}{ord}
\newcommand{\defeq}{\overset{\mathrm{def}}{=}}

%From the USAMO .tex files
\newcommand{\ts}{\textsuperscript}
\newcommand{\dg}{^\circ}
\newcommand{\ii}{\item}

% % From Math 55 and Math 145 at Harvard
% \newenvironment{subproof}[1][Proof]{%
% \begin{proof}[#1] \renewcommand{\qedsymbol}{$\blacksquare$}}%
% {\end{proof}}

\newcommand{\liff}{\leftrightarrow}
\newcommand{\lthen}{\rightarrow}
\newcommand{\opname}{\operatorname}
\newcommand{\surjto}{\twoheadrightarrow}
\newcommand{\injto}{\hookrightarrow}
\newcommand{\On}{\mathrm{On}} % ordinals
\DeclareMathOperator{\img}{im} % Image
\DeclareMathOperator{\Img}{Im} % Image
\DeclareMathOperator{\coker}{coker} % Cokernel
\DeclareMathOperator{\Coker}{Coker} % Cokernel
\DeclareMathOperator{\Ker}{Ker} % Kernel
\DeclareMathOperator{\rank}{rank}
\DeclareMathOperator{\Spec}{Spec} % spectrum
\DeclareMathOperator{\Tr}{Tr} % trace
\DeclareMathOperator{\pr}{pr} % projection
\DeclareMathOperator{\ext}{ext} % extension
\DeclareMathOperator{\pred}{pred} % predecessor
\DeclareMathOperator{\dom}{dom} % domain
\DeclareMathOperator{\ran}{ran} % range
\DeclareMathOperator{\Hom}{Hom} % homomorphism
\DeclareMathOperator{\Mor}{Mor} % morphisms
\DeclareMathOperator{\End}{End} % endomorphism

\newcommand{\eps}{\epsilon}
\newcommand{\veps}{\varepsilon}
\newcommand{\ol}{\overline}
\newcommand{\ul}{\underline}
\newcommand{\wt}{\widetilde}
\newcommand{\wh}{\widehat}
\newcommand{\vocab}[1]{\textbf{\color{blue} #1}}
\providecommand{\half}{\frac{1}{2}}
\newcommand{\dang}{\measuredangle} %% Directed angle
\newcommand{\ray}[1]{\overrightarrow{#1}}
\newcommand{\seg}[1]{\overline{#1}}
\newcommand{\arc}[1]{\wideparen{#1}}
\DeclareMathOperator{\cis}{cis}
\DeclareMathOperator*{\lcm}{lcm}
\DeclareMathOperator*{\argmin}{arg min}
\DeclareMathOperator*{\argmax}{arg max}
\newcommand{\cycsum}{\sum_{\mathrm{cyc}}}
\newcommand{\symsum}{\sum_{\mathrm{sym}}}
\newcommand{\cycprod}{\prod_{\mathrm{cyc}}}
\newcommand{\symprod}{\prod_{\mathrm{sym}}}
\newcommand{\Qed}{\begin{flushright}\qed\end{flushright}}
\newcommand{\parinn}{\setlength{\parindent}{1cm}}
\newcommand{\parinf}{\setlength{\parindent}{0cm}}
% \newcommand{\norm}{\|\cdot\|}
\newcommand{\inorm}{\norm_{\infty}}
\newcommand{\opensets}{\{V_{\alpha}\}_{\alpha\in I}}
\newcommand{\oset}{V_{\alpha}}
\newcommand{\opset}[1]{V_{\alpha_{#1}}}
\newcommand{\lub}{\text{lub}}
\newcommand{\del}[2]{\frac{\partial #1}{\partial #2}}
\newcommand{\Del}[3]{\frac{\partial^{#1} #2}{\partial^{#1} #3}}
\newcommand{\deld}[2]{\dfrac{\partial #1}{\partial #2}}
\newcommand{\Deld}[3]{\dfrac{\partial^{#1} #2}{\partial^{#1} #3}}
\newcommand{\lm}{\lambda}
\newcommand{\uin}{\mathbin{\rotatebox[origin=c]{90}{$\in$}}}
\newcommand{\usubset}{\mathbin{\rotatebox[origin=c]{90}{$\subset$}}}
\newcommand{\lt}{\left}
\newcommand{\rt}{\right}
\newcommand{\bs}[1]{\boldsymbol{#1}}
\newcommand{\exs}{\exists}
\newcommand{\st}{\strut}
\newcommand{\dps}[1]{\displaystyle{#1}}

\newcommand{\sol}{\setlength{\parindent}{0cm}\textbf{\textit{Solution:}}\setlength{\parindent}{1cm} }
\newcommand{\solve}[1]{\setlength{\parindent}{0cm}\textbf{\textit{Solution: }}\setlength{\parindent}{1cm}#1 \Qed}

\preto\tabular{\setcounter{magicrownumbers}{0}}
\newcounter{magicrownumbers}
\newcommand\rownumber{\stepcounter{magicrownumbers}\arabic{magicrownumbers}}
\def\rownumber{}

\newenvironment{deduction}
{\begin{tabular}{@{}>{$}c<{$}@{\enspace}>{$}l<{$}@{}}\arrayrulecolor{blue!50}}
		{\end{tabular}}
\newcommand{\premise}[1]{&#1\\}
\newcommand{\conclusion}[1]{\cline{2-2}\therefore&#1}


% Things Lie
\newcommand{\kb}{\mathfrak b}
\newcommand{\kg}{\mathfrak g}
\newcommand{\kh}{\mathfrak h}
\newcommand{\kn}{\mathfrak n}
\newcommand{\ku}{\mathfrak u}
\newcommand{\kz}{\mathfrak z}
\DeclareMathOperator{\Ext}{Ext} % Ext functor
\DeclareMathOperator{\Tor}{Tor} % Tor functor
\newcommand{\gl}{\opname{\mathfrak{gl}}} % frak gl group
\renewcommand{\sl}{\opname{\mathfrak{sl}}} % frak sl group chktex 6

% More script letters etc.
\newcommand{\SA}{\mathcal A}
\newcommand{\SB}{\mathcal B}
\newcommand{\SC}{\mathcal C}
\newcommand{\SF}{\mathcal F}
\newcommand{\SG}{\mathcal G}
\newcommand{\SH}{\mathcal H}
\newcommand{\OO}{\mathcal O}

\newcommand{\SCA}{\mathscr A}
\newcommand{\SCB}{\mathscr B}
\newcommand{\SCC}{\mathscr C}
\newcommand{\SCD}{\mathscr D}
\newcommand{\SCE}{\mathscr E}
\newcommand{\SCF}{\mathscr F}
\newcommand{\SCG}{\mathscr G}
\newcommand{\SCH}{\mathscr H}

% Mathfrak primes
\newcommand{\km}{\mathfrak m}
\newcommand{\kp}{\mathfrak p}
\newcommand{\kq}{\mathfrak q}

% number sets
\newcommand{\RR}[1][]{\ensuremath{\ifstrempty{#1}{\mathbb{R}}{\mathbb{R}^{#1}}}}
\newcommand{\NN}[1][]{\ensuremath{\ifstrempty{#1}{\mathbb{N}}{\mathbb{N}^{#1}}}}
\newcommand{\ZZ}[1][]{\ensuremath{\ifstrempty{#1}{\mathbb{Z}}{\mathbb{Z}^{#1}}}}
\newcommand{\QQ}[1][]{\ensuremath{\ifstrempty{#1}{\mathbb{Q}}{\mathbb{Q}^{#1}}}}
\newcommand{\CC}[1][]{\ensuremath{\ifstrempty{#1}{\mathbb{C}}{\mathbb{C}^{#1}}}}
\newcommand{\PP}[1][]{\ensuremath{\ifstrempty{#1}{\mathbb{P}}{\mathbb{P}^{#1}}}}
\newcommand{\HH}[1][]{\ensuremath{\ifstrempty{#1}{\mathbb{H}}{\mathbb{H}^{#1}}}}
\newcommand{\FF}[1][]{\ensuremath{\ifstrempty{#1}{\mathbb{F}}{\mathbb{F}^{#1}}}}
% expected value
\newcommand{\EE}{\ensuremath{\mathbb{E}}}
\newcommand{\charin}{\text{ char }}
\DeclareMathOperator{\sign}{sign}
\DeclareMathOperator{\Aut}{Aut}
\DeclareMathOperator{\Inn}{Inn}
\DeclareMathOperator{\Syl}{Syl}
\DeclareMathOperator{\Gal}{Gal}
\DeclareMathOperator{\GL}{GL} % General linear group
\DeclareMathOperator{\SL}{SL} % Special linear group

%---------------------------------------
% BlackBoard Math Fonts :-
%---------------------------------------

%Captital Letters
\newcommand{\bbA}{\mathbb{A}}	\newcommand{\bbB}{\mathbb{B}}
\newcommand{\bbC}{\mathbb{C}}	\newcommand{\bbD}{\mathbb{D}}
\newcommand{\bbE}{\mathbb{E}}	\newcommand{\bbF}{\mathbb{F}}
\newcommand{\bbG}{\mathbb{G}}	\newcommand{\bbH}{\mathbb{H}}
\newcommand{\bbI}{\mathbb{I}}	\newcommand{\bbJ}{\mathbb{J}}
\newcommand{\bbK}{\mathbb{K}}	\newcommand{\bbL}{\mathbb{L}}
\newcommand{\bbM}{\mathbb{M}}	\newcommand{\bbN}{\mathbb{N}}
\newcommand{\bbO}{\mathbb{O}}	\newcommand{\bbP}{\mathbb{P}}
\newcommand{\bbQ}{\mathbb{Q}}	\newcommand{\bbR}{\mathbb{R}}
\newcommand{\bbS}{\mathbb{S}}	\newcommand{\bbT}{\mathbb{T}}
\newcommand{\bbU}{\mathbb{U}}	\newcommand{\bbV}{\mathbb{V}}
\newcommand{\bbW}{\mathbb{W}}	\newcommand{\bbX}{\mathbb{X}}
\newcommand{\bbY}{\mathbb{Y}}	\newcommand{\bbZ}{\mathbb{Z}}

%---------------------------------------
% MathCal Fonts :-
%---------------------------------------

%Captital Letters
\newcommand{\mcA}{\mathcal{A}}	\newcommand{\mcB}{\mathcal{B}}
\newcommand{\mcC}{\mathcal{C}}	\newcommand{\mcD}{\mathcal{D}}
\newcommand{\mcE}{\mathcal{E}}	\newcommand{\mcF}{\mathcal{F}}
\newcommand{\mcG}{\mathcal{G}}	\newcommand{\mcH}{\mathcal{H}}
\newcommand{\mcI}{\mathcal{I}}	\newcommand{\mcJ}{\mathcal{J}}
\newcommand{\mcK}{\mathcal{K}}	\newcommand{\mcL}{\mathcal{L}}
\newcommand{\mcM}{\mathcal{M}}	\newcommand{\mcN}{\mathcal{N}}
\newcommand{\mcO}{\mathcal{O}}	\newcommand{\mcP}{\mathcal{P}}
\newcommand{\mcQ}{\mathcal{Q}}	\newcommand{\mcR}{\mathcal{R}}
\newcommand{\mcS}{\mathcal{S}}	\newcommand{\mcT}{\mathcal{T}}
\newcommand{\mcU}{\mathcal{U}}	\newcommand{\mcV}{\mathcal{V}}
\newcommand{\mcW}{\mathcal{W}}	\newcommand{\mcX}{\mathcal{X}}
\newcommand{\mcY}{\mathcal{Y}}	\newcommand{\mcZ}{\mathcal{Z}}


%---------------------------------------
% Bold Math Fonts :-
%---------------------------------------

%Captital Letters
\newcommand{\bmA}{\boldsymbol{A}}	\newcommand{\bmB}{\boldsymbol{B}}
\newcommand{\bmC}{\boldsymbol{C}}	\newcommand{\bmD}{\boldsymbol{D}}
\newcommand{\bmE}{\boldsymbol{E}}	\newcommand{\bmF}{\boldsymbol{F}}
\newcommand{\bmG}{\boldsymbol{G}}	\newcommand{\bmH}{\boldsymbol{H}}
\newcommand{\bmI}{\boldsymbol{I}}	\newcommand{\bmJ}{\boldsymbol{J}}
\newcommand{\bmK}{\boldsymbol{K}}	\newcommand{\bmL}{\boldsymbol{L}}
\newcommand{\bmM}{\boldsymbol{M}}	\newcommand{\bmN}{\boldsymbol{N}}
\newcommand{\bmO}{\boldsymbol{O}}	\newcommand{\bmP}{\boldsymbol{P}}
\newcommand{\bmQ}{\boldsymbol{Q}}	\newcommand{\bmR}{\boldsymbol{R}}
\newcommand{\bmS}{\boldsymbol{S}}	\newcommand{\bmT}{\boldsymbol{T}}
\newcommand{\bmU}{\boldsymbol{U}}	\newcommand{\bmV}{\boldsymbol{V}}
\newcommand{\bmW}{\boldsymbol{W}}	\newcommand{\bmX}{\boldsymbol{X}}
\newcommand{\bmY}{\boldsymbol{Y}}	\newcommand{\bmZ}{\boldsymbol{Z}}
%Small Letters
\newcommand{\bma}{\boldsymbol{a}}	\newcommand{\bmb}{\boldsymbol{b}}
\newcommand{\bmc}{\boldsymbol{c}}	\newcommand{\bmd}{\boldsymbol{d}}
\newcommand{\bme}{\boldsymbol{e}}	\newcommand{\bmf}{\boldsymbol{f}}
\newcommand{\bmg}{\boldsymbol{g}}	\newcommand{\bmh}{\boldsymbol{h}}
\newcommand{\bmi}{\boldsymbol{i}}	\newcommand{\bmj}{\boldsymbol{j}}
\newcommand{\bmk}{\boldsymbol{k}}	\newcommand{\bml}{\boldsymbol{l}}
\newcommand{\bmm}{\boldsymbol{m}}	\newcommand{\bmn}{\boldsymbol{n}}
\newcommand{\bmo}{\boldsymbol{o}}	\newcommand{\bmp}{\boldsymbol{p}}
\newcommand{\bmq}{\boldsymbol{q}}	\newcommand{\bmr}{\boldsymbol{r}}
\newcommand{\bms}{\boldsymbol{s}}	\newcommand{\bmt}{\boldsymbol{t}}
\newcommand{\bmu}{\boldsymbol{u}}	\newcommand{\bmv}{\boldsymbol{v}}
\newcommand{\bmw}{\boldsymbol{w}}	\newcommand{\bmx}{\boldsymbol{x}}
\newcommand{\bmy}{\boldsymbol{y}}	\newcommand{\bmz}{\boldsymbol{z}}

%---------------------------------------
% Scr Math Fonts :-
%---------------------------------------

\newcommand{\sA}{{\mathscr{A}}}   \newcommand{\sB}{{\mathscr{B}}}
\newcommand{\sC}{{\mathscr{C}}}   \newcommand{\sD}{{\mathscr{D}}}
\newcommand{\sE}{{\mathscr{E}}}   \newcommand{\sF}{{\mathscr{F}}}
\newcommand{\sG}{{\mathscr{G}}}   \newcommand{\sH}{{\mathscr{H}}}
\newcommand{\sI}{{\mathscr{I}}}   \newcommand{\sJ}{{\mathscr{J}}}
\newcommand{\sK}{{\mathscr{K}}}   \newcommand{\sL}{{\mathscr{L}}}
\newcommand{\sM}{{\mathscr{M}}}   \newcommand{\sN}{{\mathscr{N}}}
\newcommand{\sO}{{\mathscr{O}}}   \newcommand{\sP}{{\mathscr{P}}}
\newcommand{\sQ}{{\mathscr{Q}}}   \newcommand{\sR}{{\mathscr{R}}}
\newcommand{\sS}{{\mathscr{S}}}   \newcommand{\sT}{{\mathscr{T}}}
\newcommand{\sU}{{\mathscr{U}}}   \newcommand{\sV}{{\mathscr{V}}}
\newcommand{\sW}{{\mathscr{W}}}   \newcommand{\sX}{{\mathscr{X}}}
\newcommand{\sY}{{\mathscr{Y}}}   \newcommand{\sZ}{{\mathscr{Z}}}


%---------------------------------------
% Math Fraktur Font
%---------------------------------------

%Captital Letters
\newcommand{\mfA}{\mathfrak{A}}	\newcommand{\mfB}{\mathfrak{B}}
\newcommand{\mfC}{\mathfrak{C}}	\newcommand{\mfD}{\mathfrak{D}}
\newcommand{\mfE}{\mathfrak{E}}	\newcommand{\mfF}{\mathfrak{F}}
\newcommand{\mfG}{\mathfrak{G}}	\newcommand{\mfH}{\mathfrak{H}}
\newcommand{\mfI}{\mathfrak{I}}	\newcommand{\mfJ}{\mathfrak{J}}
\newcommand{\mfK}{\mathfrak{K}}	\newcommand{\mfL}{\mathfrak{L}}
\newcommand{\mfM}{\mathfrak{M}}	\newcommand{\mfN}{\mathfrak{N}}
\newcommand{\mfO}{\mathfrak{O}}	\newcommand{\mfP}{\mathfrak{P}}
\newcommand{\mfQ}{\mathfrak{Q}}	\newcommand{\mfR}{\mathfrak{R}}
\newcommand{\mfS}{\mathfrak{S}}	\newcommand{\mfT}{\mathfrak{T}}
\newcommand{\mfU}{\mathfrak{U}}	\newcommand{\mfV}{\mathfrak{V}}
\newcommand{\mfW}{\mathfrak{W}}	\newcommand{\mfX}{\mathfrak{X}}
\newcommand{\mfY}{\mathfrak{Y}}	\newcommand{\mfZ}{\mathfrak{Z}}
%Small Letters
\newcommand{\mfa}{\mathfrak{a}}	\newcommand{\mfb}{\mathfrak{b}}
\newcommand{\mfc}{\mathfrak{c}}	\newcommand{\mfd}{\mathfrak{d}}
\newcommand{\mfe}{\mathfrak{e}}	\newcommand{\mff}{\mathfrak{f}}
\newcommand{\mfg}{\mathfrak{g}}	\newcommand{\mfh}{\mathfrak{h}}
\newcommand{\mfi}{\mathfrak{i}}	\newcommand{\mfj}{\mathfrak{j}}
\newcommand{\mfk}{\mathfrak{k}}	\newcommand{\mfl}{\mathfrak{l}}
\newcommand{\mfm}{\mathfrak{m}}	\newcommand{\mfn}{\mathfrak{n}}
\newcommand{\mfo}{\mathfrak{o}}	\newcommand{\mfp}{\mathfrak{p}}
\newcommand{\mfq}{\mathfrak{q}}	\newcommand{\mfr}{\mathfrak{r}}
\newcommand{\mfs}{\mathfrak{s}}	\newcommand{\mft}{\mathfrak{t}}
\newcommand{\mfu}{\mathfrak{u}}	\newcommand{\mfv}{\mathfrak{v}}
\newcommand{\mfw}{\mathfrak{w}}	\newcommand{\mfx}{\mathfrak{x}}
\newcommand{\mfy}{\mathfrak{y}}	\newcommand{\mfz}{\mathfrak{z}}


\title{\Huge{Orthogonality and Least Squares}}
\author{\huge{Madiba Hudson-Quansah}}
\date{}
\usepackage{parskip}

\setcounter{tocdepth}{4}
\setcounter{secnumdepth}{4}

\begin{document}
\maketitle
\newpage
\pdfbookmark[section]{\contentsname}{too}
\tableofcontents
\pagebreak

\chapter{Inner Product, Length and Orthogonality}

\section{Inner Product}

\dfn{Inner / Dot Product}{
  If $\mbold{u}$ and $\mbold{v}$ are vectors in $\mathbb{R}^{n}$, then we regard $\mbold{u}$ and $\mbold{v}$ as
  $n\times 1$ matrices. The transpose of $\mbold{u}^{T}$ is a $1 \times n$ matrix, and the matrix product
  $\mbold{u}^{T}\mbold{v}$ is a $1\times 1$ matrix, a scalar. This scalar is called the \textit{inner / dot product} of
  $\mbold{u}$ and $\mbold{v}$
  which can also be referred to as:
  \[
    \mbold{u} \cdot \mbold{v}
  \]
  Which breaks down into:
  \[
    \mbold{u}^{T} \times \mbold{v}
  \]
  When $\mbold{u} = \begin{bmatrix} u_1 \\ u_2 \\ \vdots \\ u_n \end{bmatrix} $ and $\mbold{v} = \begin{bmatrix} v_1 \\
      v_2 \\ \vdots \\ v_n\end{bmatrix} $, is then defined as:
  \begin{align*}
    \begin{bmatrix} u_1 & u_2 & \ldots u_n \end{bmatrix} \begin{bmatrix} v_1 \\ v_2 \\ \vdots \\ v_n \end{bmatrix} =
    u_1v_1 + u_2v_2 + \ldots + u_n v_n
  \end{align*}
}

\ex{}{
  \qs{}{
    Compute $\mbold{u} \cdot \mbold{v}$ and $\mbold{v} \cdot \mbold{u}$ for $\mbold{u}=\begin{bmatrix} 2 \\ -5 \\ -1
      \end{bmatrix} $ and $\mbold{v} = \begin{bmatrix} 3 \\ 2 \\ -3 \end{bmatrix} $
  }

  \sol{
    \begin{align*}
      \mbold{u} \cdot \mbold{v} = \mbold{u}^{T}\times \mbold{v} & = \begin{bmatrix} 2 & -5 & -1 \end{bmatrix}  \begin{bmatrix} 3 \\ 2 \\ -3 \end{bmatrix}         \\
                                                                & = 3 \left( 2 \right) + \left( -5 \right) \left( 2 \right) + \left( -1 \right) \left( -3 \right) \\
                                                                & = -1                                                                                            \\
    \end{align*}

    \begin{align*}
      \mbold{v} \cdot \mbold{u} = \mbold{v}^{T} \times \mbold{u} & = \begin{bmatrix} 3 & 2 & -3 \end{bmatrix} \begin{bmatrix} 2 \\ -5 \\ -1
                                                                                                              \end{bmatrix} \\
                                                                 & = 2\left(
      3\right) +
      2 \left(
      -5 \right)
      + \left(
      -3 \right)
      \left( -1 \right)                                                                                                                    \\
                                                                 & = -1                                                                    \\
    \end{align*}
  }
}

\thm{Axioms of Inner / Dot products}{
  Let $\mbold{u}$ and $\mbold{v}$, and $\mbold{w}$ be vectors in $\mathbb{R}^{n}$, and let $c$ be a scalar. Then
  \begin{enumerate}
    \item $\mbold{u} \cdot \mbold{v} = \mbold{v} \cdot  \mbold{u}$
    \item $ \left( \mbold{u} + \mbold{v} \right) \cdot \mbold{w} = \mbold{u} \cdot \mbold{w} + \mbold{v} \cdot \mbold{w} $
    \item $ \left( c \mbold{u} \right) \cdot \mbold{v} = \mbold{u} \cdot \left( c \mbold{v} \right)  $
    \item $\mbold{u} \cdot \mbold{u} \geq 0$, and $\mbold{u}\cdot \mbold{u} = 0$ if and only if $\mbold{u} = \mbold{0}$
  \end{enumerate}
}

\section{Length of a Vector}

\dfn{Length of a Vector}{
  If $\mbold{v}$ is in $\mathbb{R}^{n}$, with entries $v_1, \ldots, v_n$, then the square root of $\mbold{v}\cdot
    \mbold{v}$ is defined because $\mbold{v} \cdot \mbold{v}$ is non-negative. Therefore the \textit{length / norm}  of
  $\mbold{v}$ is the non-negative scalar $ \norm{ \mbold{v}  } $, defined:
  \[
    \norm{ \mbold{v} }  = \sqrt{\mbold{v} \cdot \mbold{v}} = \sqrt{v^2_1 + v^2_2 + \ldots + v^2_n} \text{
      and }  \norm{ \mbold{v}  } ^2 = \mbold{v}\cdot \mbold{v}
  \]

}

And similarly for any scalar $c$, the length of $c\mbold{v}$ is $ \abs{c} $ times the length of $\mbold{v}$, i.e:
\[
  \norm{ c\mbold{v}} =  \abs{c } \times \|\mbold{v}\|
\]

\dfn{Unit Vector}{
  A vector whose length is 1. If we divide a non zero vector by it's length, i.e. multiply by $\frac{1}{\|\mbold{v}\|}$, we obtain a unit vector $\mbold{u}$. This process of creating a unit vector $\mbold{u}$ from
  $\mbold{v}$ can be called \textit{normalizing} $\mbold{v}$, and the resulting $\mbold{u}$ is in the same direction
  as $\mbold{v}$
}

\section{Distance in $\mathbb{R}^{n}$}

\dfn{Distance between two vectors}{
  For $\mbold{u}$ and $\mbold{v}$ in $\mathbb{R}^{n}$, the \textit{distance between} $\mbold{u}$ and $\mbold{v}$,
  expressed as $\text{dist} \left( \mbold{u}, \mbold{v} \right) $, is the length of the vector $\mbold{u}-\mbold{v}$:
  \[
    \text{dist} \left( \mbold{u}, \mbold{v} \right)  =  \norm{ \mbold{u} - \mbold{v}  }
  \]
  Then defined:
  \begin{align*}
    \text{dist} \left( \mbold{u}, \mbold{v} \right) = \|\mbold{u} - \mbold{v}\| & = \sqrt{ \left( \mbold{u} - \mbold{v}
    \right) \cdot  \left( \mbold{u} - \mbold{v} \right)  }                                                                                                     \\
                                                                                & = \sqrt{ \left( u_1 - v_1 \right)^2 + \ldots + \left( u_n - v_n \right)^2  } \\
  \end{align*}

}
In $\mathbb{R}^{2}$ and $\mathbb{R}^{3}$, this is basically the same as the Euclidean distance between two points.

\ex{}{
  \qs{}{
    Compute the distance between the vectors $\mbold{u}= \left( 7, 1 \right) $ and $\mbold{v} = \left( 3, 2 \right) $
  }

  \sol{
    \begin{align*}
      \text{dist} \left( \mbold{u}, \mbold{v} \right) & =  \|\mbold{u} - \mbold{v}\|                                                                \\
                                                      & = \sqrt{ \left( \mbold{u} - \mbold{v} \right) \cdot \left( \mbold{u} - \mbold{v} \right)  } \\
                                                      & = \sqrt{\begin{bmatrix} 4 \\ -1 \end{bmatrix}\cdot
      \begin{bmatrix} 4 \\ -1 \end{bmatrix}  }                                                                                                      \\
                                                      & = \sqrt{4^2 + \left( -1 \right)^2 }                                                         \\
                                                      & = \sqrt{17}                                                                                 \\
    \end{align*}
  }
}

\section{Orthogonal Vectors}

Consider $\mathbb{R}^{2}$ and $\mathbb{R}^{3}$ and two lines through the origin determined by vectors $\mbold{u}$ and
$\mbold{v}$. These lines are geometrically perpendicular if and only if the distance from $\mbold{u}$ to $\mbold{v}$ is
the same as the distance from $\mbold{u}$ to $-\mbold{v}$. This is equivalent to saying the squares of the distances are
the same. Therefore:
\begin{align*}
  \left[ \text{dist} \left( \mbold{u}, -\mbold{v} \right)  \right] ^2 & = \|\mbold{u} - \left( -\mbold{v} \right) \|^2 =
  \|\mbold{u} + \mbold{v}\|^2                                                                                                                                                         \\
                                                                      & = \left( \mbold{u} + \mbold{v} \right) \cdot \left( \mbold{u} + \mbold{v} \right)                             \\
                                                                      & = \mbold{u} \cdot \left( \mbold{u} + \mbold{v} \right) + \mbold{v} \cdot \left( \mbold{u} + \mbold{v} \right) \\
                                                                      & = \mbold{u}\cdot \mbold{u} + \mbold{u} \cdot \mbold{v} + \mbold{v}\cdot \mbold{u} + \mbold{v}\cdot \mbold{v}  \\
                                                                      & = \|\mbold{u}^2\| + \|\mbold{v}\|^2 +
  2\mbold{u}\cdot \mbold{v}                                                                                                                                                           \\
\end{align*}
And then $\text{dist} \left( \mbold{u}, \mbold{v} \right) $:
\begin{align*}
  \left[ \text{dist} \left( \mbold{u}, \mbold{v} \right)  \right] & = \|\mbold{u}\|^2 + \|\mbold{v}\|^2 - 2\mbold{u}\cdot \mbold{v} \\
\end{align*}
This shows that the two squared distances are only equal if and only if $2\mbold{u}\cdot \mbold{v} = -2\mbold{u}\cdot
  \mbold{v}$, which happens if and only if $\mbold{u} \cdot \mbold{v} = 0$

\dfn{Orthogonality}{
  Two vectors $\mbold{u}$ and $\mbold{v}$ in $\mathbb{R}^{n}$ are orthogonal, to each other, if $\mbold{u} \cdot
    \mbold{v} = 0$
}

This then confirms that the zero vector $\mbold{0}$ is orthogonal to every vector in $\mathbb{R}^{n}$, since
$\mbold{0}^{T} \mbold{v} = 0$ for every $\mbold{v}$.

\thm{The Pythagorean Theorem}{
  If $\mbold{u}$ and $\mbold{v}$ are orthogonal vectors in $\mathbb{R}^{n}$, then:
  \[
    \|\mbold{u} + \mbold{v}\|^2 = \|\mbold{u}\|^2 + \|\mbold{v}\|^2
  \]
}

\section{Exercises}

\qs{}{
  Let $\mbold{a} = \begin{bmatrix} -2 \\ 1 \end{bmatrix} $ and $\mbold{b} = \begin{bmatrix} -3 \\ 1 \end{bmatrix} $.
  Compute $\frac{\displaystyle \mbold{a} \cdot \mbold{b}}{\displaystyle \mbold{a}\cdot \mbold{a}}$ and $ \left( \displaystyle \frac{\mbold{a} \cdot
      \mbold{b}}{\mbold{a}\cdot \mbold{a}} \right) \mbold{a} $
}

\sol{
  \begin{align*}
    \mbold{a} \cdot \mbold{b}                                                                                       & = \left( -2 \right) \left( -3 \right) + 1           \\
                                                                                                                    & = 7                                                 \\
    \mbold{a}\cdot \mbold{a}                                                                                        & = \left( -2 \right)^2 + 1                           \\
                                                                                                                    & = 5                                                 \\
    \frac{\displaystyle \mbold{a} \cdot \mbold{b}}{\displaystyle \mbold{a}\cdot \mbold{a}}                          & =
    \frac{7}{5}                                                                                                                                                           \\
    \\
    \left( \frac{\displaystyle \mbold{a} \cdot \mbold{b}}{\displaystyle \mbold{a}\cdot \mbold{a}} \right) \mbold{a} & =
    \frac{7}{5} \begin{bmatrix} -2 \\ 1 \end{bmatrix}                                                                                                                     \\
                                                                                                                    & = \begin{bmatrix} -2.8 \\ \frac{7}{5} \end{bmatrix} \\
  \end{align*}
}

\qs{}{
  Let $\mbold{c} = \begin{bmatrix} \frac{4}{3} \\ -1 \\ \frac{2}{3} \end{bmatrix}$ and $\mbold{d} = \begin{bmatrix} 5 \\
      6 \\ -1\end{bmatrix} $.
  \begin{enumerate}
    \item Find a unit vector $\mbold{u}$ in the direction of $\mbold{c}$
    \item Show that $\mbold{d}$ is orthogonal to $\mbold{c}$.
    \item Use the results of parts (1) and (2) to explain why $d$ must be orthogonal to the unit vector $\mbold{u}$
  \end{enumerate}
}

\sol{
  \begin{enumerate}
    \item
          \begin{align*}
            \|\mbold{c}\| & = \sqrt{\mbold{c} \cdot \mbold{c}}                                                                                                                                                           \\
                          & = \sqrt{ \left( \frac{4}{3} \right)^2 + \left( -1 \right)^2 + \left( \frac{2}{3} \right)^2   }                                                                                               \\
                          & = \frac{\sqrt{ 29}}{3}                                                                                                                                                                       \\
            \mbold{u}     & = \frac{1}{\frac{\sqrt{ 29}}{3}} \begin{bmatrix} \frac{4}{3} \\ -1 \\ \frac{2}{3} \end{bmatrix}                                                                                              \\
                          & = \frac{3\sqrt{29} }{29} \begin{bmatrix} \frac{4}{3} \\ -1 \\ \frac{2}{3} \end{bmatrix}                                                                                                      \\
                          & =  \begin{bmatrix} \displaystyle \frac{4\sqrt{29} }{29} \\ \displaystyle \frac{3\sqrt{29} }{29} \\ \displaystyle \frac{2\sqrt{29} }{29} \end{bmatrix} \\
            \\
            \|\mbold{u}\| & = \sqrt{\mbold{u}\cdot \mbold{u}}                                                                                                                                                            \\
                          & = \sqrt{ \left( \frac{4\sqrt{29} }{29} \right) ^2 + \left( \frac{3\sqrt{29} }{29} \right) ^2
            + \left( \frac{2\sqrt{29} }{29} \right) ^2}                                                                                                                                                                  \\
                          & = 1                                                                                                                                                                                          \\
          \end{align*}
    \item
          If $\mbold{d}$ is orthogonal to $\mbold{c}$ then $\mbold{d} \cdot \mbold{c} = 0$
          \begin{align*}
            \mbold{d} \cdot \mbold{c} & = \mbold{d}^{T} \times \mbold{c}                                                                          \\
                                      & = \begin{bmatrix} 5 & 6 & -1 \end{bmatrix} \begin{bmatrix} \frac{4}{3} \\ -1 \\ \frac{2}{3} \end{bmatrix} \\
                                      & = 5 \left( \frac{4}{3} \right) + 6 \left( -1 \right)  -1 \left( \frac{2}{3} \right)                       \\
                                      & = \frac{20}{3} - 6 - \frac{2}{3}                                                                          \\
                                      & = 0                                                                                                       \\
          \end{align*}
          $\therefore$ $\mbold{c}$ and $\mbold{d}$ are orthogonal to each other.
    \item $\mbold{d}$ is orthogonal to the unit vector $\mbold{u}$ because $\mbold{d}$ is orthogonal to $\mbold{c}$
          of which $\mbold{u}$ is a scalar multiple of. I.e $\mbold{u}$ is in the form $k \mbold{c}$ for some $k$ and:
          \[
            \mbold{d}\cdot \mbold{u} = \mbold{d} \cdot \left( k \mbold{c} \right) =  k \left( \mbold{d}\cdot c \right) = k
            \left( 0 \right) = 0
          \]
  \end{enumerate}
}

\chapter{Orthogonal Sets}

\dfn{Orthogonal Set}{
  If $S = \{\mbold{u}_1,\ldots,\mbold{u}_p\} $ is an orthogonal set of non-zero vectors in $\mathbb{R}^{n}$, then $S$
  is linearly independent and hence is a basis for the subspace spanned by $S$.
}

\dfn{Orthogonal Basis}{
  An orthogonal basis for a subspace $W$ of $\mathbb{R}^{n}$ is a basis for $W$ that is also an orthogonal set.
}

\thm{}{
  Let $\{\mbold{u}_1,\ldots,\mbold{u}_p\} $ be an orthogonal basis for a subspace $W$ of $\mathbb{R}^{n}$. For each
  $\mbold{y}$ in $W$, the weights in the linear combination
  \[
    \mbold{y} = c_1 \mbold{u}_1 + \ldots + c_p \mbold{u}_p
  \]
  are given by
  \[
    c_j = \frac{\mbold{y}\cdot \mbold{u}_j}{\mbold{u}_j\cdot \mbold{u}_j } \quad \left( j = 1, \ldots, p \right)
  \]
}

\ex{}{
  \qs{}{
    The set $S = \{\mbold{u}_1, \mbold{u}_2, \mbold{u}_3\} $, where
    \[
      \mbold{u}_1 = \begin{bmatrix} 3 \\ 1\\ 1 \end{bmatrix} , \mbold{u}_2 = \begin{bmatrix} -1 \\ 2 \\ 1 \end{bmatrix} ,
      \mbold{u}_3 = \begin{bmatrix} -\frac{1}{2} \\ -2 \\ \frac{7}{2} \end{bmatrix}
    \]
    is an orthogonal basis for $\mathbb{R}^{3}$. Express the vector $\mbold{y} = \begin{bmatrix} 6 \\ 1\\ -8 \end{bmatrix}
    $ as a linear combination of the vectors in $S$
  }

  \sol{
    \begin{align*}
      \mbold{y} & = \frac{\mbold{y}\cdot \mbold{u}_1}{\mbold{u}_1 \cdot \mbold{u}_1} \mbold{u}_1
      + \frac{\mbold{y}\cdot \mbold{u}_2}{\mbold{u}_2 \cdot \mbold{u}_2} \mbold{u}_2
      + \frac{\mbold{y}\cdot \mbold{u}_3}{\mbold{u}_3 \cdot \mbold{u}_3} \mbold{u}_3
      \\
                & = \frac{11}{11}\mbold{u}_1 + -\frac{12}{6}\mbold{u}_2 + -\frac{33}{\frac{33}{2}}\mbold{u}_3 \\
                & = \mbold{u}_1 - 2\mbold{u}_2 - 2\mbold{u}_3                                                 \\
    \end{align*}
  }
}

\chapter{Orthogonal Projections}

\dfn{Orthogonal Projection}{
  Let $W$ be a subspace of $\mathbb{R}^{n}$, and let $\mbold{y}$ be in $\mathbb{R}^{n}$. The \textit{orthogonal
    projection} of $\mbold{y}$ onto $W$, denoted $\text{proj}_W \mbold{y}$, is the closest point in $W$ to $\mbold{y}$.
  This point is obtained by adding the orthogonal projection of $\mbold{y}$ onto the orthogonal complement of $W$ to the
  orthogonal projection of $\mbold{y}$ onto $W$.

}

\ex{}{
  \qs{}{
    Let $\{\mbold{u}_1, \ldots, \mbold{u}_5\} $ be an orthogonal basis for a subspace $W$ of $\mathbb{R}^{5}$, and let
    \[
      \mbold{y} = c_1 \mbold{u}_1 + \ldots + c_5 \mbold{u}_5
    \]
    Consider the subspace $W = \text{Span}\{\mbold{u}_1,\mbold{u}_2\} $, and write $\mbold{y}$ as the sum of a vector
    $\mbold{z}_1$ in $W$ and a vector $\mbold{z}_2$ in $W^{\perp}$
  }

  \sol{
    \begin{align*}
      \mbold{z}_1                           & = \mbold{y} - \mbold{z}_2                                                              \\
      \mbold{y}                             & = \mbold{z}_2 + \mbold{z}_1                                                            \\
      \\
      \mbold{z}_1                           & = c_1\mbold{u}_1 + c_2\mbold{u}_2                                                      \\
      \mbold{z}_2                           & = c_3\mbold{u}_3 + c_4 \mbold{u}_4 + c_5 \mbold{u}_5                                   \\
      \\
      \mbold{y} = \mbold{z}_1 + \mbold{z}_2 & = c_1\mbold{u}_1 + c_2\mbold{u}_2 + c_3\mbold{u}_3 + c_4 \mbold{u}_4 + c_5 \mbold{u}_5
    \end{align*}
  }
}

\thm{The Orthogonal Decomposition Theorem}{
  \label{orthog_decomp}
  Let $W$ be a subspace of $\mathbb{R}^{n}$. Then each $\mbold{y}$ in $\mathbb{R}^{n}$ can be written uniquely in the
  form:
  \[
    \mbold{y} = \hat{\mbold{y}} + \mbold{z}
  \]
  where $\hat{\mbold{y}}$ is in $W$ and $\mbold{z}$ is in $W^{\perp}$. In fact, if $\{\mbold{u}_1, \ldots, \mbold{u}_p\}
  $, is any orthogonal basis of $W$, then:
  \[
    \hat{\mbold{y}} = \frac{\mbold{y}\cdot \mbold{u}_1}{\mbold{u}_1 \cdot \mbold{u}_1} \mbold{u}_1 + \ldots + \frac{\mbold{y}\cdot \mbold{u}_p}{\mbold{u}_p \cdot \mbold{u}_p} \mbold{u}_p
  \]
  and $\mbold{z} = \mbold{y} - \hat{\mbold{y}}$
}

\ex{}{
  \qs{}{
    Let $\mbold{u}_1 = \begin{bmatrix} 2 \\ 5 \\ -1 \end{bmatrix} $, $\mbold{u}_2 = \begin{bmatrix} -2 \\ 1\\ 1
      \end{bmatrix} $, and $\mbold{y} = \begin{bmatrix} 1 \\ 2\\ 3 \end{bmatrix} $. Observe that $ \{\mbold{u}_1,
      \mbold{u}_2\} $ is an orthogonal basis for $W = \text{Span} \{\mbold{u}_1, \mbold{u}_2\} $. Write $\mbold{y}$ as
    the sum of a vector in $W$ and a vector orthogonal to $W$
  }

  \sol{
    The orthogonal projection of $\mbold{y}$ onto $W$ is:
    \begin{align*}
      \hat{\mbold{y}} & = \frac{\mbold{y} \cdot \mbold{u}_1 }{\mbold{u}_1 \cdot \mbold{u}_1} \mbold{u}_1 + \frac{\mbold{y} \cdot
      \mbold{u}_2}{\mbold{u}_2 \cdot \mbold{u}_2} \mbold{u}_2                                                                    \\
                      & = \frac{9}{30}\begin{bmatrix} 2 \\ 5 \\ -1 \end{bmatrix} + \frac{3}{6} \begin{bmatrix} -2 \\ 1\\ 1
                                                                                               \end{bmatrix}        \\
                      & = \begin{bmatrix}
                            \frac{3}{5} \\ \frac{3}{2} \\ -\frac{3}{10}  \end{bmatrix} + \begin{bmatrix}
                                                                                         -1 \\ \frac{3}{6} \\ \frac{3}{6}
                                                                                       \end{bmatrix}           \\
                      & = \begin{bmatrix}
                            -\frac{2}{5} \\
                            2            \\
                            \frac{1}{5}
                          \end{bmatrix}                                                                                         \\
    \end{align*}
    So:
    \begin{align*}
      \mbold{z} & =   \mbold{y}- \hat{\mbold{y}}                              \\
                & = \begin{bmatrix} 1 \\ 2\\ 3 \end{bmatrix} - \begin{bmatrix}
                                                                 -\frac{2}{5} \\
                                                                 2            \\
                                                                 \frac{1}{5}
                                                               \end{bmatrix} \\
                & = \begin{bmatrix}
                      \frac{7}{5} \\
                      0           \\
                      \frac{14}{5}
                    \end{bmatrix}                                            \\
    \end{align*}
    $\mbold{z}$ is orthogonal to $W$ due to \ref{orthog_decomp}, so $\mbold{y}$ can be expressed as:
    \[
      \mbold{y} = \begin{bmatrix}
        \displaystyle -\frac{2}{5} \\
        \displaystyle 2            \\
        \displaystyle \frac{1}{5}
      \end{bmatrix}  +  \begin{bmatrix}
        \displaystyle \frac{7}{5} \\
        \displaystyle 0           \\
        \displaystyle \frac{14}{5}
      \end{bmatrix}
    \]
  }
}

\section{Properties of Orthogonal Projections}

If $\{\mbold{u}_1, \ldots, \mbold{u}_p\} $ is an orthogonal basis for $W$ and if $\mbold{y}$ happens to be in $W$, then
the formula for $\text{proj}_W \mbold{y}$ is exactly the same as the representation of $\mbold{y}$ in terms of the
basis. In this case, $\text{proj}_W \mbold{y} = \mbold{y}$.

\thm{}{
  If $\mbold{y}$ is in $W = \text{Span}\{\mbold{u}_1, \ldots, \mbold{u}_p\} $, then $\text{proj}_W \mbold{y} = \mbold{y}$
}

This leads to the next theorem:
\thm{The Best Approximation Theorem}{
  Let $W$ be a subspace of $\mathbb{R}^{n}$, let $\mbold{y}$ be any vector in $\mathbb{R}^{n}$, and let $\hat{\mbold{y}}$
  be an orthogonal projection of $\mbold{y}$  onto $W$. Then $\hat{\mbold{y}}$ is the closest point in $W$ to $\mbold{y}$,
  in the sense that
  \[
    \|\mbold{y}-\hat{\mbold{y}}\| < \|\mbold{y}-\mbold{v}\|
  \]
  For all $\mbold{v}$ in $W$ distinct from $\hat{\mbold{y}}$
}

\dfn{Orthonormality}{
  A set of vectors is orthonormal if each of them are orthogonal to each other and have a length of 1.
}

\thm{Orthonormal Basis}{
  If $\{\mbold{u}_1, \ldots, \mbold{u}_p\} $ is an Orthonormal basis for a subspace $W$ of $\mathbb{R}^{n}$, then
  \[
    \text{proj}_W \mbold{y} = \left( \mbold{y} \cdot \mbold{u}_1 \right) \mbold{u}_1 + \left( \mbold{y}\cdot \mbold{u}_2
    \right) \mbold{u}_2 + \ldots + \left( \mbold{y}\cdot \mbold{u}_p \right) \mbold{u}_p
  \]
  If  $U = \begin{bmatrix} \mbold{u}_1 & \mbold{u}_2 & \ldots & \mbold{u}_p \end{bmatrix} $, then
  \[
    \text{proj}_W \mbold{y} = U U ^{T}\mbold{y} \forall \mbold{y} \in \mathbb{R}^{n}
  \]
}


\subsection{Exercises}

\qs{}{
  The distance from a point $\mbold{y}$ in $\mathbb{R}^{n}$ to a subspace $W$ is defined as the distance from
  $\mbold{y}$ to the nearest point in $W$. Find the distance from $\mbold{y}$ to $W = \text{Span}\{\mbold{u}_1,
    \mbold{u}_2\} $, where
  \[
    \mbold{y} = \begin{bmatrix} -1 \\ -5 \\ 10 \end{bmatrix} , \mbold{u}_1 = \begin{bmatrix} 5 \\ -2 \\ 1 \end{bmatrix},
    \mbold{u}_2 = \begin{bmatrix} 1 \\ 2\\ -1 \end{bmatrix}
  \]
}

\sol{
  \begin{align*}
    \hat{\mbold{y}} & = \frac{\mbold{y} \cdot \mbold{u}_1}{\mbold{u}_1 \cdot \mbold{u}_1} \mbold{u}_1 + \frac{\mbold{y} \cdot
    \mbold{u}_2}{\mbold{u}_2 \cdot \mbold{u}_2}                                                                                               \\
                    & = \frac{15}{30} \begin{bmatrix} 5 \\ -2 \\ 1 \end{bmatrix}   + -\frac{21}{6} \begin{bmatrix} 1 \\ 2 \\ -1 \end{bmatrix} \\
                    & = \begin{bmatrix} \frac{5}{2} \\ -1 \\ \frac{15}{30} \end{bmatrix} + \begin{bmatrix} -\frac{21}{6}
                                                                                             \\ -7 \\  \frac{21}{6}\end{bmatrix}                \\
                    & = \begin{bmatrix}
                          -1 \\
                          -8 \\
                          4
                        \end{bmatrix}                                                                                                        \\
    \mbold{z}       & = \mbold{y} - \hat{\mbold{y}}                                                                                           \\
                    & = \begin{bmatrix} -1 \\ -5 \\ 10 \end{bmatrix} - \begin{bmatrix}
                                                                         -1 \\
                                                                         -8 \\
                                                                         4
                                                                       \end{bmatrix}                                                         \\
                    & = \begin{bmatrix}
                          0 \\
                          3 \\
                          6
                        \end{bmatrix}                                                                                                        \\
    \|\mbold{z}\|   & = \sqrt{3^2 + 6^2}                                                                                                      \\
                    & = \sqrt{45}                                                                                                             \\
  \end{align*}
}

\chapter{The Gram-Schmidt Process}

The Gram-Schmidt process is a simple algorithm for producing an orthogonal or orthonormal basis for any non-zero
subspace of $\mathbb{R}^{n}$.

\ex{}{
  \qs{}{
    Let  $\mbold{x}_1 = \begin{bmatrix} 1 \\ 1\\ 1\\ 1\\  \end{bmatrix} $ and $\mbold{x}_2= \begin{bmatrix} 0 \\ 1\\ 1\\
        1\end{bmatrix} $, and $\mbold{x}_3 = \begin{bmatrix} 0 \\ 0\\ 1\\ 1\\  \end{bmatrix} $. Then $\{\mbold{x}_1,
      \mbold{x}_2, \mbold{x}_3\} $ is clearly linearly independent and thus a basis for a subspace $W$ of
    $\mathbb{R}^{4}$. Construct and orthogonal basis for $W$.
  }

  \sol{
    \begin{description}
      \item[Step 1] Let $\mbold{v}_1 = \mbold{x}_1$ and $W_1 = \text{Span } \{\mbold{x}_1\} = \text{Span } \{\mbold{v}_1\} $
      \item[Step 2] Let $\mbold{v}_2$ be the vector produced by subtracting from $\mbold{x}_2$ is projection onto the subspace
            $W_1$. That is, let
            \begin{align*}
              \mbold{v}_2 & = \mbold{x}_2 - \text{proj}_{W_1} \mbold{x}_2                                                                                                                        \\
                          & = \mbold{x}_2 - \frac{\mbold{x}\cdot \mbold{v}_1}{\mbold{v}_1\cdot \mbold{v}_1} \mbold{v}_1                                                                          \\
              \text{Since } \mbold{v}_1 = \mbold{x}_1                                                                                                                                            \\
                          & = \begin{bmatrix} 0 \\ 1\\ 1\\ 1 \end{bmatrix} - \frac{3}{4} \begin{bmatrix} 1 \\ 1 \\ 1\\ 1
                                                                                         \end{bmatrix} = \begin{bmatrix} -\frac{3}{4} \\ \frac{1}{4} \\ \frac{1}{4} \\ \frac{1}{4} \end{bmatrix} \\
            \end{align*}
            $\mbold{v}_2$ is the component of $\mbold{x}_2$ orthogonal to $\mbold{x}_1$, and $\{\mbold{v}_1,
              \mbold{v}_2\} $ is an orthogonal basis for the subspace $W_2$ spanned by $\mbold{x}_1$ and $\mbold{x}_2$
      \item[Step 2' (Optional)] If possible scale $\mbold{v}_2$ to simplify future calculations. Since $\mbold{v}_2$ has
            fractional entries, it is convenient to scale it by a factor of 4 and replace $\{\mbold{v}_1,
              \mbold{v}_2\} $ by the orthogonal basis
            \[
              \mbold{v}_1 = \begin{bmatrix} 1 \\ 1\\ 1\\ 1 \end{bmatrix},\, \mbold{v}^{\prime}_2 = \begin{bmatrix}
                -3 \\ 1\\ 1\\ 1 \end{bmatrix}
            \]
      \item [Step 3] Let $\mbold{v}_3$ be the vector produced by subtracting from $\mbold{x}_3$  its projection
            onto the subspace $W_2$, using the orthogonal basis $\{\mbold{v}_1, \mbold{v}^{\prime}_2\} $ to compute this
            projection to $W_2$:
            \begin{align*}
              \text{proj}_{W_2} \mbold{x}_3                    & = \frac{\mbold{x}_3 \cdot \mbold{v}_1}{\mbold{v}_1\cdot
                \mbold{v}_1}\mbold{v}_1 +
              \frac{\mbold{x}_3 \cdot \mbold{v}^{\prime}_2}{\mbold{v}^{\prime}_2 \cdot
              \mbold{v}^{\prime}_2}\mbold{v}^{\prime}_2                                                                            \\
              \mbold{x}_3 \cdot \mbold{v}_1                    & = 0 + 0 + 1 + 1                                                   \\
                                                               & = 2                                                               \\
              \mbold{v}_1 \cdot \mbold{v}_1                    & = 1 + 1+ 1+1                                                      \\
                                                               & = 4                                                               \\
              \mbold{x}_3 \cdot \mbold{v}^{\prime}_2           & = 0 + 0 + 1 + 1                                                   \\
                                                               & = 2                                                               \\
              \mbold{v}^{\prime}_2\cdot   \mbold{v}^{\prime}_2 & =9 + 1 + 1 + 1                                                    \\
                                                               & = 12                                                              \\
                                                               & = \frac{2}{4} \begin{bmatrix} 1 \\ 1\\ 1\\ 1
                                                                               \end{bmatrix} + \frac{2}{12} \begin{bmatrix} -3 \\ 1\\
                                                                                                              1  \\ 1\end{bmatrix} \\
                                                               & = \begin{bmatrix}
                                                                     \frac{2}{4} \\
                                                                     \frac{2}{4} \\
                                                                     \frac{2}{4} \\
                                                                     \frac{2}{4} \\
                                                                   \end{bmatrix} +
              \begin{bmatrix}
                -\frac{1}{2} \\
                \frac{1}{6}  \\
                \frac{1}{6}  \\
                \frac{1}{6}  \\
              \end{bmatrix}
              \\
                                                               & = \begin{bmatrix}
                                                                     0           \\
                                                                     \frac{2}{3} \\
                                                                     \frac{2}{3} \\
                                                                     \frac{2}{3} \\
                                                                   \end{bmatrix}                                                  \\
              \\
              \mbold{v}_3                                      & = \mbold{x}_3 - \text{proj}_{W_2}\mbold{x}_3                      \\
                                                               & = \begin{bmatrix} 0 \\ 0 \\ 1\\ 1 \end{bmatrix} - \begin{bmatrix}
                                                                                                                     0           \\
                                                                                                                     \frac{2}{3} \\
                                                                                                                     \frac{2}{3} \\
                                                                                                                     \frac{2}{3} \\
                                                                                                                   \end{bmatrix}  \\
                                                               & = \begin{bmatrix}
                                                                     0            \\
                                                                     -\frac{2}{3} \\
                                                                     \frac{1}{3}  \\
                                                                     \frac{1}{3}
                                                                   \end{bmatrix}                                                  \\
            \end{align*}
            Therefore the orthogonal basis for $W$ is $\{\mbold{v}_1, \mbold{v}^{\prime}_2,\mbold{v}_3 \} $, where
            \[
              \mbold{v}_1 = \begin{bmatrix} 1 \\ 1\\ 1\\ 1 \end{bmatrix},\, \mbold{v}^{\prime}_2 = \begin{bmatrix}
                -3 \\ 1\\ 1\\ 1 \end{bmatrix}
              \mbold{v}_3 = \begin{bmatrix} 0 \\ -\frac{2}{3} \\ \frac{1}{3}  \\ \frac{1}{3} \end{bmatrix}
            \]
    \end{description}
  }
}

\pagebreak
\thm{The Gram-Schmidt Process}{
  Given a basis $\{\mbold{x}_1, \ldots, \mbold{x}_p\} $ for a non-zero subspace $W$ of $\mathbb{R}^{n}$, define
  \begin{align*}
    \mbold{v}_1 & = \mbold{x}_1                                                                                   \\
    \mbold{v}2  & = \mbold{x}_2 - \frac{\mbold{x}_2\cdot \mbold{v}_1}{\mbold{v}_1\cdot \mbold{v}_1} \mbold{v}_1   \\
    \mbold{v}_3 & =  \mbold{x}_3 - \frac{\mbold{x}_3\cdot \mbold{v}_1}{\mbold{v}_1\cdot \mbold{v}_1} \mbold{v}_1
    - \frac{\mbold{x}_3\cdot \mbold{v}_2}{\mbold{v}_2\cdot \mbold{v}_2} \mbold{v}_2                               \\
    \vdots                                                                                                        \\
    \mbold{v}_p & = \mbold{x}_p - \frac{\mbold{x}_p\cdot \mbold{v}_1}{\mbold{v}_1\cdot \mbold{v}_1} \mbold{v}_1 -
    \frac{\mbold{x}_p \cdot  \mbold{v}_2}{\mbold{v}_2\cdot \mbold{v}_2} \mbold{v}_2 - \ldots - \frac{\mbold{x}_p \cdot
      \mbold{v}_{p-1} }{\mbold{v}_{p-1} \cdot \mbold{v}_{p-1}} \mbold{v}_{p-1}
  \end{align*}
  Then $\{\mbold{v}_1,\ldots, \mbold{v}_p\} $ is an orthogonal basis for $W$. In addition
  \[
    \text{Span }\{\mbold{v}_1, \ldots, \mbold{v}_k\} = \text{Span }\{\mbold{x}_1, \ldots, \mbold{x}_k\} \quad \text{ for }
    1 \leq k \leq p
  \]
}

\section{Orthonormal Bases}

An orthonormal basis is constructed easily from an orthogonal basis $\{\mbold{v}_1, \ldots, \mbold{v}_p\} $, by
normalizing all the $\mbold{v}_k$.

\ex{}{
  Given the constructed basis
  \[
    \mbold{v}_1 = \begin{bmatrix} 3 \\ 6 \\ 0 \end{bmatrix} , \mbold{v}_2 = \begin{bmatrix} 0 \\ 0 \\ 2 \end{bmatrix}
  \]
  An orthonormal basis is
  \begin{align*}
    \mbold{u}_1 & = \frac{1}{\|\mbold{v}_1\|} = \frac{1}{\sqrt{45}} \begin{bmatrix} 3 \\ 6\\ 0 \end{bmatrix} =
    \begin{bmatrix} \frac{1}{\sqrt{5}} \\ \frac{2}{\sqrt{5}} \\ 0 \end{bmatrix}  \\
    \mbold{u}_2 & = \frac{1}{\|\mbold{v}_2\|}\mbold{v}_2 = \begin{bmatrix} 0 \\ 0 \\ 1 \end{bmatrix}           \\
  \end{align*}
}

\subsection{Exercises}

\qs{}{
  Let $A = \begin{bmatrix}
      1 & -1 & 4  \\
      1 & 4  & -2 \\
      1 & 4  & 2  \\
      1 & -1 & 0
    \end{bmatrix} $. Find an orthonormal basis for the column space of $A$.
}

\sol{
  \begin{align*}
    \begin{bmatrix}
      1 & -1 & 4  \\
      1 & 4  & -2 \\
      1 & 4  & 2  \\
      1 & -1 & 0
    \end{bmatrix}             \\
    R_1 - R_2  \rightarrow R_2 \\
    \begin{bmatrix}
      1 & -1 & 4 & 0 \\
      0 & -5 & 6 & 0 \\
      1 & 4  & 2 & 0 \\
      1 & -1 & 0 & 0 \\
    \end{bmatrix}
    \\
    R_1 - R_3  \rightarrow R_3 \\
    \begin{bmatrix}
      1 & -1 & 4 & 0 \\
      0 & -5 & 6 & 0 \\
      0 & -5 & 2 & 0 \\
      1 & -1 & 0 & 0 \\
    \end{bmatrix}
    \\
    R_1 - R_4  \rightarrow R_4 \\
    \begin{bmatrix}
      1 & -1 & 4 & 0 \\
      0 & -5 & 6 & 0 \\
      0 & -5 & 2 & 0 \\
      0 & 0  & 4 & 0 \\
    \end{bmatrix}
    \\
    R_2 - R_3  \rightarrow R_3 \\
    \begin{bmatrix}
      1 & -1 & 4 & 0 \\
      0 & -5 & 6 & 0 \\
      0 & 0  & 4 & 0 \\
      0 & 0  & 4 & 0 \\
    \end{bmatrix}
  \end{align*}
  \begin{align*}
    R_3 - R_4  \rightarrow R_4              \\
    \begin{bmatrix}
      1 & -1 & 4 & 0 \\
      0 & -5 & 6 & 0 \\
      0 & 0  & 4 & 0 \\
      0 & 0  & 0 & 0 \\
    \end{bmatrix}
    \\
    \frac{1}{5}R_2 - R_1  \rightarrow R_1   \\
    \begin{bmatrix}
      -1 & 0  & \frac{-14}{5} & 0 \\
      0  & -5 & 6             & 0 \\
      0  & 0  & 4             & 0 \\
      0  & 0  & 0             & 0 \\
    \end{bmatrix}
    \\
    \frac{3}{2}R_3 - R_2  \rightarrow R_2   \\
    \begin{bmatrix}
      -1 & 0 & \frac{-14}{5} & 0 \\
      0  & 5 & 0             & 0 \\
      0  & 0 & 4             & 0 \\
      0  & 0 & 0             & 0 \\
    \end{bmatrix}
    \\
    \frac{-7}{10}R_3 - R_1  \rightarrow R_1 \\
    \begin{bmatrix}
      1 & 0 & 0 & 0 \\
      0 & 5 & 0 & 0 \\
      0 & 0 & 4 & 0 \\
      0 & 0 & 0 & 0 \\
    \end{bmatrix}
    \\
    \frac{1}{5}R_2 \to R_2                  \\
    \begin{bmatrix}
      1 & 0 & 0 & 0 \\
      0 & 1 & 0 & 0 \\
      0 & 0 & 4 & 0 \\
      0 & 0 & 0 & 0 \\
    \end{bmatrix}
    \\
    \frac{1}{4}R_3 \to R_3                  \\
    \begin{bmatrix}
      1 & 0 & 0 & 0 \\
      0 & 1 & 0 & 0 \\
      0 & 0 & 1 & 0 \\
      0 & 0 & 0 & 0 \\
    \end{bmatrix}
    \\
  \end{align*}
  Therefore $\text{Col }A = \left\{ \begin{bmatrix} 1 \\ 1\\ 1\\ 1 \end{bmatrix}, \begin{bmatrix} -1 \\ 4\\ 4\\ -1
    \end{bmatrix}, \begin{bmatrix} 4 \\ -2\\ 2\\ 0 \end{bmatrix}    \right\} $ \\

  \noindent Let $\mbold{v}_1 = \mbold{x}_1$ and $W_1 = \text{Span } \{\mbold{v}1\} $
  \begin{align*}
    \mbold{v}_2          & = \mbold{x}_2 - \text{proj}_{W_1}\mbold{x}_2                                                                        \\
                         & = \mbold{x}_2 - \frac{\mbold{x}_2 \cdot \mbold{v}_1}{\mbold{v}_1 \cdot \mbold{v}_1}\mbold{v}_1                      \\
                         & = \begin{bmatrix} -1 \\ 4 \\ 4 \\ -1 \end{bmatrix} - \frac{6}{4} \begin{bmatrix} 1 \\ 1\\ 1\\ 1 \end{bmatrix}       \\
                         & = \begin{bmatrix}
                               -\frac{5}{2} \\
                               \frac{5}{2}  \\
                               \frac{5}{2}  \\
                               -\frac{5}{2} \\
                             \end{bmatrix}                                                                                                    \\
    \mbold{v}^{\prime}_2 & = \begin{bmatrix} -5 \\ 5\\ 5\\ -5 \end{bmatrix}                                                                    \\
    \\
    \text{Now let } W_2  & = \text{Span } \{\mbold{v}_1, \mbold{v}^{\prime}_2\}                                                                \\
    \mbold{v}_3          & = \mbold{x}_3 - \text{proj}_{W_2} \mbold{x}_3                                                                       \\
                         & = \mbold{x}_3
    - \frac{\mbold{x}_3 \cdot \mbold{v}_1}{\mbold{v}_1 \cdot \mbold{v}_1} \mbold{v}_1
    - \frac{\mbold{x}_3 \cdot \mbold{v}^{\prime}_2}{\mbold{v}^{\prime}_2 \cdot \mbold{v}^{\prime}_2} \mbold{v}^{\prime}_2
    \\
                         & = \begin{bmatrix} 4 \\ -2\\ 2\\ 0 \end{bmatrix} = \frac{4}{4} \begin{bmatrix} 1 \\ 1\\ 1\\ 1 \end{bmatrix} - \left(
    -\frac{20}{100}\right) \begin{bmatrix} -5 \\ 5\\ 5\\ -5 \end{bmatrix}                                                                      \\
                         & = \begin{bmatrix} 2 \\ -2\\ 2\\ -2 \end{bmatrix}
  \end{align*}
  Therefore the orthogonal basis for the column space of $a$ is $\left\{ \begin{bmatrix} 1 \\ 1\\ 1\\ 1 \end{bmatrix},
    \begin{bmatrix} -5 \\ 5\\ 5\\ -5 \end{bmatrix}, \begin{bmatrix} 2 \\ -2\\ 2\\ -2 \end{bmatrix} \right\} $. The orthonormal basis is
  \begin{align*}
    \mbold{u}_1 & = \frac{1}{\|\mbold{v}_1\|} = \frac{1}{\sqrt{4}} \begin{bmatrix} 1 \\ 1\\ 1\\ 1 \end{bmatrix} =
    \begin{bmatrix} \frac{1}{2} \\ \frac{1}{2} \\ \frac{1}{2} \\ \frac{1}{2} \end{bmatrix}                            \\
    \mbold{u}_2 & = \frac{1}{\|\mbold{v}_2\|} = \frac{1}{\sqrt{100}} \begin{bmatrix} -5 \\ 5\\ 5\\ -5 \end{bmatrix} =
    \begin{bmatrix} -\frac{1}{2} \\ \frac{1}{2} \\ \frac{1}{2} \\ -\frac{1}{2} \end{bmatrix}                          \\
    \mbold{u}_3 & = \frac{1}{\|\mbold{v}_3\|} = \frac{1}{\sqrt{16}} \begin{bmatrix} 2 \\ -2\\ 2\\ -2 \end{bmatrix} =
    \begin{bmatrix} \frac{1}{2} \\ -\frac{1}{2} \\ \frac{1}{2} \\ -\frac{1}{2} \end{bmatrix}                          \\
  \end{align*}
  Therefore the orthonormal basis for the column space of $A$ is $\left\{ \begin{bmatrix} \frac{1}{2} \\ \frac{1}{2}\\
      \frac{1}{2} \\ \frac{1}{2}\end{bmatrix}, \begin{bmatrix} -\frac{1}{2} \\ \frac{1}{2}\\ \frac{1}{2}\\ -\frac{1}{2}
    \end{bmatrix}, \begin{bmatrix} \frac{1}{2} \\ -\frac{1}{2}\\ \frac{1}{2}\\ -\frac{1}{2} \end{bmatrix} \right\} $
}

\chapter{Exercises}

\qs{}{
  Show that $\{\mbold{u}_1, \mbold{u}_2, \mbold{u}_3\} $ is an orthogonal basis for $\mathbb{R}^{3}$. Then express
  $\mbold{x}$ as a linear combination of the $\mbold{u}s$
  \[
    \mbold{u}_1 = \begin{bmatrix} 3 \\ -3 \\ 0 \end{bmatrix} , \mbold{u} = \begin{bmatrix} 2 \\ 2 \\ -1 \end{bmatrix} ,
    \mbold{u}_3 = \begin{bmatrix} 1 \\ 1 \\ 4 \end{bmatrix}, \text{ and } \mbold{x} = \begin{bmatrix} 5 \\ -3 \\ 1 \end{bmatrix}
  \]
}

\sol{
  For the basis be orthogonal $\mbold{u}_1 \cdot \mbold{u}_2 = 0$, $\mbold{u}_2 \cdot \mbold{u}_3 = 0$, and $\mbold{u}_1
    \cdot \mbold{u}_3 = 0$.
  \begin{align*}
    \mbold{u}_1 \cdot \mbold{u}_2 & = \begin{bmatrix} 3 \\ -3 \\ 0 \end{bmatrix} \cdot \begin{bmatrix} 2 \\ 2 \\ -1 \end{bmatrix} \\
                                  & = 3(2) + (-3)(2) + 0(-1)                                                                      \\
                                  & = 6 - 6                                                                                       \\
                                  & = 0                                                                                           \\
    \\
    \mbold{u}_2 \cdot \mbold{u}_3 & = \begin{bmatrix} 2 \\ 2 \\ -1 \end{bmatrix} \cdot \begin{bmatrix} 1 \\ 1 \\ 4 \end{bmatrix}  \\
                                  & = 2(1) + 2(1) + (-1)(4)                                                                       \\
                                  & = 2 + 2 - 4                                                                                   \\
                                  & = 0                                                                                           \\
    \mbold{u}_3 \cdot \mbold{u}_1 & =  \begin{bmatrix} 1 \\ 1 \\ 4 \end{bmatrix} \cdot \begin{bmatrix} 3 \\ -3 \\ 0 \end{bmatrix} \\
                                  & = 1(3) + 1(-3) + 4(0)                                                                         \\
                                  & = 3 - 3                                                                                       \\
                                  & = 0                                                                                           \\
  \end{align*}
  Therefore $\{\mbold{u}_1, \mbold{u}_2, \mbold{u}_3\} $ is an orthogonal basis for $\mathbb{R}^{3}$. To express
  $\mbold{x}$ as a linear combination of the $\mbold{u}s$:
  \begin{align*}
    \mbold{x} = c_1 \mbold{u}_1 + c_2 \mbold{u}_2 + c_3 \mbold{u}_3 \\
    \begin{bmatrix}
      3  & 2  & 1 & 5  \\
      -3 & 2  & 1 & -3 \\
      0  & -1 & 4 & 1
    \end{bmatrix}                                                \\
    -1R_1 - R_2  \rightarrow R_2                                    \\
    \begin{bmatrix}
      3 & 2  & 1  & 5  \\
      0 & -4 & -2 & -2 \\
      0 & -1 & 4  & 1  \\
    \end{bmatrix}
    \\
    \frac{1}{4}R_2 - R_3  \rightarrow R_3                           \\
    \begin{bmatrix}
      3 & 2  & 1            & 5            \\
      0 & -4 & -2           & -2           \\
      0 & 0  & \frac{-9}{2} & \frac{-3}{2} \\
    \end{bmatrix}
    \\
    \frac{-1}{2}R_2 - R_1  \rightarrow R_1                          \\
    \begin{bmatrix}
      -3 & 0  & 0            & -4           \\
      0  & -4 & -2           & -2           \\
      0  & 0  & \frac{-9}{2} & \frac{-3}{2} \\
    \end{bmatrix}
    \\
    \frac{4}{9}R_3 - R_2  \rightarrow R_2                           \\
    \begin{bmatrix}
      -3 & 0 & 0            & -4           \\
      0  & 4 & 0            & \frac{4}{3}  \\
      0  & 0 & \frac{-9}{2} & \frac{-3}{2} \\
    \end{bmatrix}
    \\
    \frac{-1}{3}R_1 \to R_1                                         \\
    \begin{bmatrix}
      1 & 0 & 0            & \frac{4}{3}  \\
      0 & 4 & 0            & \frac{4}{3}  \\
      0 & 0 & \frac{-9}{2} & \frac{-3}{2} \\
    \end{bmatrix}
    \\
    \frac{1}{4}R_2 \to R_2                                          \\
    \begin{bmatrix}
      1 & 0 & 0            & \frac{4}{3}  \\
      0 & 1 & 0            & \frac{1}{3}  \\
      0 & 0 & \frac{-9}{2} & \frac{-3}{2} \\
    \end{bmatrix}
    \\
    \frac{-2}{9}R_3 \to R_3                                         \\
    \begin{bmatrix}
      1 & 0 & 0 & \frac{4}{3} \\
      0 & 1 & 0 & \frac{1}{3} \\
      0 & 0 & 1 & \frac{1}{3} \\
    \end{bmatrix}                                         \\
    c_1 = \frac{4}{3}                                               \\
    c_2 = \frac{1}{3}                                               \\
    c_3 = \frac{1}{3}                                               \\
    \\
    \mbold{x} = \frac{4}{3} \begin{bmatrix} 3 \\ -3 \\ 0 \end{bmatrix} + \frac{1}{3} \begin{bmatrix} 2 \\ 2\\ -1
                                                                                     \end{bmatrix}  + \frac{1}{3} \begin{bmatrix} 1 \\ 1\\ 4 \end{bmatrix}
  \end{align*}
}

\qs{}{
  Let $W$ be the subspace spanned by the $\mbold{u}s$, and write $\mbold{y}$ as the sum of a vector in $W$ and a vector
  orthogonal to $W$.
  \[
    \mbold{y} = \begin{bmatrix} 1 \\ 3 \\ 5 \end{bmatrix}, \mbold{u}_1 = \begin{bmatrix} 1 \\ 3\\ -2 \end{bmatrix} ,
    \mbold{u}_2 = \begin{bmatrix} 5 \\ 1 \\ 4 \end{bmatrix}
  \]
}

\sol{
  We can express $\mbold{y}$ as the sum of a vector in $W$ and a vector orthogonal to $W$ using the orthogonal
  decomposition theorem, finding  the orthogonal projection of $\mbold{y}$ onto $W$, and subtracting that from
  $\mbold{y}$ to find the vector orthogonal to $W$.
  \begin{align*}
    \hat{\mbold{y}} & = \frac{0}{14} \begin{bmatrix} 1 \\ 3\\ -2 \end{bmatrix} + \frac{28}{42} \begin{bmatrix} 5 \\ 1 \\
                                                                                                 4\end{bmatrix} \\
                    & = \begin{bmatrix}
                          \displaystyle \frac{10}{3} \\
                          \displaystyle \frac{2}{3}  \\
                          \displaystyle \frac{8}{3}
                        \end{bmatrix}                                                                    \\
    \mbold{z}       & = \mbold{y} - \hat{\mbold{y}}                                                                   \\
                    & = \begin{bmatrix} 1 \\ 3 \\ 5 \end{bmatrix} - \begin{bmatrix}
                                                                      \frac{10}{3} \\
                                                                      \frac{2}{3}  \\
                                                                      \frac{8}{3}
                                                                    \end{bmatrix}                                    \\
    \mbold{z}       & = \begin{bmatrix} -\frac{7}{3} \\
                          \frac{7}{3}  \\ \frac{7}{3}\end{bmatrix}                                                      \\
    \mbold{y}       & = \mbold{z} + \hat{\mbold{y}}                                                                   \\
                    & = \begin{bmatrix} -\frac{7}{3} \\
                          \frac{7}{3}  \\ \frac{7}{3}\end{bmatrix}+   \begin{bmatrix}
                                                                      \frac{10}{3} \\
                                                                      \frac{2}{3}  \\
                                                                      \frac{8}{3}
                                                                    \end{bmatrix}                                    \\
  \end{align*}
}

\end{document}
