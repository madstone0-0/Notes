\documentclass[12pt letter]{report}
\input{../template/preamble}
\input{../template/macros}
\input{../template/letterfonts}

\usepackage{parskip}
\title{\Huge{Assignment 4}}
\author{\huge{Madiba Hudson-Quansah}}
\date{}


\begin{document}
\maketitle
\newpage
\gdef\rownumber{\stepcounter{magicrownumbers}\arabic{magicrownumbers}}

\qs{}{
	\begin{math}
		\begin{deduction}
			\premise{ \left( p \wedge t \right) \to  \left( r \vee s \right)   }
			\premise{ q \to \left( u \wedge t \right)  }
			\premise{ u \to p }
			\premise{ \neg s }
			\premise{ q }
			\conclusion{ r }
		\end{deduction}
	\end{math}
}

\sol{
	\begin{table}[h!]
		\begin{center}
			\begin{tabular}{ | @{\makebox[3em][r]{\rownumber\space}} | c | c | }
				\hline

				\multicolumn{1}{ | @{\makebox[3em][r]} |c| }{Steps}         & \multicolumn{1}{|c|}{Reasons}          \\
				\hline
				\hline
				$ \left( p \wedge t \right) \to \left( r \vee s \right)   $ & Premise 1                              \\
				$q \to \left( u \wedge t \right) $                          & Premise 2                              \\
				$u \to p$                                                   & Premise 3                              \\
				$\neg s$                                                    & Premise 4                              \\
				$q$                                                         & Premise 5                              \\
				$u \wedge  t$                                               & By Modus Ponens from 2 and 5           \\
				$u$                                                         & By Simplification from 6               \\
				$p$                                                         & By Modus Ponens from 3 and 7           \\
				$t$                                                         & By Simplification from 6               \\
				$p \wedge t$                                                & By Conjunction from 8 and 9            \\
				$r \vee s$                                                  & By Modus Ponens from 1 and 10          \\
				$r$                                                         & By Disjunctive Syllogism from 11 and 4 \\
				\hline
			\end{tabular}
		\end{center}
	\end{table}
}

\qs{}{
	"Jane is a student in this class. Jane grew up in a family of entrepreneurs. Everyone who grew up
	in a family of entrepreneurs can build a thriving business. Therefore, someone in this class can
	build a thriving business."
}

\sol{

	\begin{align*}
		\mathbb{U}_{x} \text{: All people}                                      \\
		\\
		S \left( x \right)  \text{: $x$ is a student in this class.}            \\
		E \left( x \right)  \text{: $x$ grew up in a family of entrepreneurs. } \\
		B \left( x \right)  \text{: $x$ can build a thriving business}          \\
		\begin{deduction}
			\premise{ S \left( \text{Jane} \right)  }
			\premise{ E \left( \text{Jane} \right)  }
			\premise{ \forall x \left( E \left( x \right) \to B \left( x \right)   \right)  }
			\conclusion{ \exists x \left( S \left( x \right) \wedge B \left( x \right)   \right)  }
		\end{deduction}
	\end{align*}

	\begin{table}[h!]
		\begin{center}
			\begin{tabular}{ | @{\makebox[3em][r]{\rownumber\space}} | c | c | }
				\hline
				\multicolumn{1}{ | @{\makebox[3em][r]{~}} |c| }{Steps}                      & \multicolumn{1}{|c|}{Reasons} \\
				\hline
				\hline
				$S \left( \text{Jane} \right)  $                                            & Premise 1                     \\
				$E \left( \text{Jane} \right)  $                                            & Premise 2                     \\
				$\forall x \left( E \left( x \right) \to B \left( x \right)   \right) $     & Premise 3                     \\
				$E \left( \text{Jane} \right) \to B \left( \text{Jane} \right)  $           & Universal Instantiation from
				3.                                                                                                          \\
				$B \left( \text{Jane} \right) $                                             & By Modus Ponens from 2 and 4. \\
				$S \left( \text{Jane} \right) \wedge B \left( \text{Jane} \right)  $        & By Conjunction from 1 and 5.
				\\
				$\exists x \left( S \left( x \right) \wedge  B \left( x \right)   \right) $ & Existential Generalization
				from 6.                                                                                                     \\

				\hline
			\end{tabular}
		\end{center}
	\end{table}
}

\qs{}{
	The product of two odd numbers is odd
}

\begin{myproof}
	Using a direct proof I will show that if $x$ and $y$ are odd, then $x \times  y$ is odd.\\
	Assume that $x$ and $y$ are odd.\\
	Then $\exists k \in \mathbb{Z} \, x = 2k + 1 $ and $\exists t \in \mathbb{Z} \, y = 2t + 1$\\
	For $x \times y$ to be odd
	\[
		x \times y = 2z + 1
	\]
	Where $z$ is an integer\\
	\begin{align*}
		x \times y & = \left( 2k + 1 \right) \times \left( 2t + 1 \right) \\
		           & = 4kt + 2k + 2t + 1                                  \\
		           & = 2(2kt + k + t) + 1                                 \\
		\text{Let } z = 2kt + k + t                                       \\
		           & = 2z + 1                                             \\
	\end{align*}
	Since $z$ is the sum of integers it is an integer.\\
	Hence If $x$ and $y$ are odd, then $x \times y$ is odd.
\end{myproof}


\end{document}
