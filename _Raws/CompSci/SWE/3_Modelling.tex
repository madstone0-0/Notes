\documentclass[12pt letter]{report}
\input{./template/preamble}
\input{./template/macros}
\input{./template/letterfonts}

\title{\Huge{Modelling}}
\author{\huge{Madiba Hudson-Quansah}}
\date{}
% \usepackage{plantuml}
\usepackage{parskip}

\setcounter{tocdepth}{4}
\setcounter{secnumdepth}{4}

\begin{document}
\maketitle
\newpage
\pdfbookmark[section]{\contentsname}{too}
\tableofcontents
\pagebreak

\chapter{Principles that guide practice}

\chapter{Unified Modelling Language (UML)}

\section{Introduction}

\dfn{UML}{
  A standard graphical language for modelling object-oriented software systems.
}

UML can be used to model the following:
\begin{description}
  \item[Class Diagrams] - Describe classes and their relationships.
  \item[Interaction Diagrams] - Show the behaviour of systems in terms of how they interact with each other.
  \item[State Diarams and Activity Diagrams] - Show how systems behave internally
  \item[Component and Deployment Diagrams] - Show how the various components of systems are arranged logically and physically.
\end{description}

\section{Class Diagrams}
The main symbols seen in class diagrams are:
\begin{description}
  \item[Class]  - Represents the type of data, represented by a rectangle with three compartments.
  \item[Association] - Represents linkage between instances of classes, represented by a line connecting the classes.
  \item[Attribute] - Simple data found in classes and their instances, represented by a name and a type.
  \item[Operation] - Represent the functions performed by classes, and their instances, represented by a name and a type.
  \item[Generalization] - Groups classes into inheritance hierarchies, represented by a line with a triangle pointing to the superclass.
\end{description}

\subsection{Class}

\begin{description}
  \item[Class Name] - The name of the class.
  \item[Attributes] - The data that the class holds, with format \lstinline{visibility attributeName: type}. Where the available visibilities are:
        \begin{description}
          \item[+] - Public
          \item[-] - Private
          \item[\#] - Protected
          \item[\~] - Package
        \end{description}
  \item[Operations] - The functions that the class can performed, with signature \lstinline{operationName(parameters): returnType}.
\end{description}

% \begin{plantuml}
%   @startuml
%   class Class {
%       + attribute1: type
%       - attribute2: type
%       # attribute3: type
%       ~ attribute4: type
%
%       + operation1(parameters): returnType
%       - operation2(parameters): returnType
%       # operation3(parameters): returnType
%       ~ operation4(parameters): returnType
%     }
%   @enduml
% \end{plantuml}

\subsection{Associations and Multiplicity}
Show that two classes are related to each other, with symbols indicating the multiplicity of the relationship at the end of the line. With the following symbols:
\begin{description}
  \item[0..1] - Zero or one
  \item[0..* / *] - Zero or more
  \item[$n$] - Exactly $n$
  \item[$m..n$] - Between $m$ and $n$
  \item[0, $m..n$] - Zero or between $m$ and $n$
\end{description}

There are three classes of multiplicities:
\begin{description}
  \item[Many-to-Many] - Both classes can have many instances of the other, for example a student can take many courses and a course can be taken by many students.
  \item[Many-to-One / One-to-Many] - One class can have many instances of the other, for example a branch can have many workers.
  \item[One-to-One] - One class can have only one instance of the other, for example a person can have only one passport.
\end{description}

It is possible for an association to connect a class to itself, this is called a \textbf{self-association / reflexive association}. For example, a person can be married to another person. \\


Associations are bi-directional by default, but can be made one way by adding an arrowhead to the line. \\

\subsection{Generalization}
Generalization is used to show that one class is a superclass of another class. The superclass is the parent class, and the subclass is the child class. The subclass inherits the attributes and operations of the superclass. The discriminator is a label that describes the criteria used in the specialization \\



\end{document}
