\documentclass[12pt letter]{report}
\input{./template/preamble}
\input{./template/macros}
\input{./template/letterfonts}

\title{\Huge{Trigonometry and Derivatives}}
\author{\huge{Madiba Hudson-Quansah}}
\date{March 2023}
\usepackage{parskip}

\begin{document}
\maketitle
\newpage
\pdfbookmark[section]{\contentsname}{toc}
\tableofcontents
\pagebreak

\chapter{}
\section{Sine}

\begin{align*}
	f(x) = \sin(x)                \\
	-1 \leq        \sin(x) \leq 1 \\
	\sin(0) = 0                   \\
\end{align*}

For all integer multiples $\pi $, $\sin()$ attains 0

\[
	\sin(k\pi) = 0,  \text{ Where k is an integer}
\]

The graph of sine is periodic with a period of $2\pi$, meaning it repeats itself every interval of $2\pi$

\subsection{Derivative of $\sin(x)$}

\dfn{$\sin(x)$}{
	if $y=\sin(x)$

	\[
		y^{\prime} = \cos(x)
	\]
}

\begin{myproof}
	\begin{align*}
		y^{\prime} = \lim_{h \to 0} \frac{f(x+h) -f(x)}{h}                                              \\
		\frac{\sin(x+h) -\sin(x)}{h}                                                                    \\
		\therefore \frac{\sin(x+h) -\sin(x)}{h}                                                         \\
		\frac{\sin(x)\cos\left(h\right) + \cos\left(x\right)\sin\left(h\right) - \sin\left(x\right)}{h} \\
		\frac{\sin(\cos(h)-1) + \cos(x)\sin(h)}{h}                                                      \\
		y^{\prime} = \lim_{h \to 0} \frac{\sin(\cos(h)-1) + \cos(x)\sin(h)}{h}                          \\
		y^{\prime} = \lim_{h \to 0} \frac{\sin(\cos(h) -1 )}{h} + \frac{\cos(x)\sin(h)}{h}              \\
		= \lim_{h \to 0} \frac{\sin(\cos(h) -1 )}{h} + \lim_{h \to 0} \frac{\cos(x)\sin(h)}{h}          \\[20pt]
		\because \lim_{x \to a} f(x) + \lim_{x \to a} g(x) = \lim_{x \to a}(f(x) + g(x))                \\[20pt]
		= \sin(x)\lim_{h \to 0}  \frac{\cos(h) -1 }{h} + \cos(x)\lim_{h \to 0} \frac{\sin(h)}{h}        \\[20pt]
		\because \lim_{x \to a} (kf(x)) = k \lim_{x \to a} f(x)                                         \\[20pt]
		\lim_{h \to 0^{-}}\frac{\cos(h) -1 }{h} = 0                                                     \\
		\lim_{h \to 0^{+}} \frac{\cos(h) -1 }{h} = 0                                                    \\
		\therefore  \lim_{h \to 0} \frac{\cos(h) -1 }{h} =  0                                           \\
		= \sin(x) \times 0 +\cos(x)\lim_{h \to 0} \frac{\sin(h)}{h}                                     \\
		= \cos(x)\lim_{h \to 0} \frac{\sin(h)}{h}                                                       \\
		\lim_{h \to 0^{-}} \frac{\sin(h)}{h} = 1                                                        \\
		\lim_{h \to 0^{+}} \frac{\sin(h)}{h} = 1                                                        \\[20pt]
		\therefore \lim_{h \to 0} \frac{\sin(h)}{h} = 1                                                 \\
		\therefore \cos(x)\lim_{h \to 0} \frac{\sin(h)}{h} = \cos(x) \times 1                           \\[20pt]
		 & = \cos(x)                                                                                    \\
	\end{align*}
\end{myproof}

\section{Cosine}

\begin{align*}
	f(x) = \cos(x)                \\
	-1 \leq        \cos(x) \leq 1 \\
	\cos(0) = 1                   \\
\end{align*}

\section{Tangent}

\begin{align*}
	f(x) = \tan(x)                \\
	-1 \leq        \tan(x) \leq 1 \\
	\tan(0) = 0                   \\
\end{align*}

\subsection{Vertical Asymptote}

\begin{note}
	A vertical line $(x, 0)$ where the values of a function rise or fall infinitely
\end{note}

The line $x=a$ is a vertical asymptote of $f(x)$ if
\[
	\lim_{x \to a} f(x) = \pm \infty
\]

The zero points of $\cos(x)$ create a vertical asymptote in relation to $\tan(x)$

\subsection{Derivative of $\tan(x)$}

\dfn{$\tan(x)$}{
	if $y=\tan(x)$
	\[
		y^{\prime} = \sec^2(x)
	\]
}

\begin{myproof}
	\begin{align*}
		y = \tan(x) = \frac{\sin(x)}{\cos(x)}                                       \\[20pt]
		y^{\prime} = \frac{\cos(x)(\cos(x))- \sin(x)(-\sin(x))}{(\cos(x))^2}        \\[20pt]
		\because y^{\prime} = \frac{v \times u^{\prime} - u \times v^{\prime}}{v^2} \\
		y^{\prime} = \frac{\cos^2(x) + \sin^2(x)}{\cos^2(x)}                        \\
		y^{\prime} = \frac{\cos^2(x)}{\cos^2(x)} + \frac{\sin^2(x)}{\cos^2(x)}      \\
		y^{\prime} = 1 + (\frac{\sin(x)}{\cos(x)})^2                                \\
		y^{\prime} = 1 + \tan^2(x)                                                  \\
		 & = \sec^2(x)
	\end{align*}
\end{myproof}

\section{Secant}

\begin{align*}
	f(x) = \sec(x)               \\
	1 \leq        \sec(x) \leq 1 \\
	\sec(0) = 1                  \\
\end{align*}

\subsection{Derivative of $\sec(x)$}
\dfn{$\sec(x)$}{
	if $y=\sec(x)$
	\[
		y^{\prime} = \sec(x)\tan(x)
	\]
}

\chapter{}

\section{Inverse Trigonometric Functions}
In the case where
\[
	f[g(x)] = x
\] and
\[
	g[f(x)] = x
\]
When can say the function $f$ is the inverse of function $g$, due to the function $g$ being able to extract the origninal input of the function $f$.

Therefore the derivative of the inverse function $y = \sin^{-1}(x)$ is as follows:
\begin{align*}
	\sin(y) = x                                          \\
	\frac{d}{dx}\sin(y) = \frac{d}{dx} x                 \\
	\cos(y)\frac{dy}{dx} = 1                             \\
	\frac{dy}{dx} = \frac{1}{\cos(y)}                    \\[20pt]
	\text{Using the identity } \cos^2(y) + \sin^2(y) = 1 \\
	\cos(x) = \sqrt{1 - \sin^2(x)}                       \\
	\frac{dy}{dx} = \frac{1}{\sqrt{1 - \sin^2(y)}}       \\
	\text{And since } x = \sin(y)                        \\
	\sin^2(y) = x^2                                      \\
	 & \frac{dy}{dx} = \frac{1}{\sqrt{1 - x^2}}          \\
\end{align*}

\pagebreak

\ex{Find the derivative of $\cos^{-1}(x)$}{
	\begin{align*}
		y = \cos^{-1}(x)                                     \\
		\cos(y) = x                                          \\
		-\sin(y)\frac{dy}{dx} = 1                            \\
		\frac{dy}{dx} = -\frac{1}{\sin(y)}                   \\
		\text{Using the identity } \cos^2(y) + \sin^2(y) = 1 \\
		\sin(y) = \sqrt{1 - \cos^2(y)}                       \\
		\frac{dy}{dx} = -\frac{1}{\sqrt{1 - \cos^2(y)}}      \\
		\text{And since } x = \cos(y)                        \\
		\cos^2(y) = x^2                                      \\
		 & =  \frac{dy}{dx} = -\frac{1}{\sqrt{1 - x^2}}      \\
	\end{align*}
}

\ex{Find the derivative of $\tan^{-1}(x)$}{
	\begin{align*}
		\tan(y) = x                                         \\
		\sec^2(y)\frac{dy}{dx} = 1                          \\
		\frac{dy}{dx} = \frac{1}{\sec^2(y)}                 \\
		\text{From the identity } 1 + \tan^2(y) = \sec^2(y) \\
		\frac{dy}{dx} = \frac{1}{1 + \tan^2(y)}             \\
		\text{And since } x = \tan(y)                       \\
		\tan^2(y) = x^2                                     \\
		 & = \frac{dy}{dx} = \frac{1}{1 + x^2}              \\
	\end{align*}
}

\chapter{}
\section{Relationships}

\begin{align*}
	\cos(x-\frac{\pi}{2}) = \sin(x) \\
	\sin(x + \frac{\pi}{2}) = \cos(x)
\end{align*}

\section{Identities}
\subsection{Reciprocal Identities}

\begin{align*}
	\sin(\theta) = \frac{1}{\csc(\theta)} \;\; \text{or} \;\; \csc(\theta) = \frac{1}{\sin(\theta)} \\[20pt]
	\cos(\theta) = \frac{1}{\sec(\theta)}\;\; \text{or} \;\;\sec(\theta) = \frac{1}{\cos(\theta)}   \\[20pt]
	\tan(\theta) = \frac{1}{\cot(\theta)}\;\; \text{or} \;\; \cot(\theta) = \frac{1}{\tan(\theta)}
\end{align*}

\subsection{Pythagorean Identities}

\begin{align*}
	\sin^2(\theta) + \cos^2(\theta) = 1 \\
	1 + \tan^2(\theta) = \sec^2(\theta) \\
	\csc^2(\theta) = 1 + \cot^2(\theta)
\end{align*}

\subsection{Ratio Identities}
\begin{align*}
	\tan(\theta) = \frac{\sin(\theta)}{\cos(\theta)} \\
	\cot(\theta) = \frac{\cos(\theta)}{\sin(\theta)}
\end{align*}

\subsection{Sum and Difference of Angles}
\begin{align*}
	\sin(\alpha + \beta) = \sin(\alpha) \times \cos(\beta) + \cos(\alpha) \times \cos(\beta)       \\
	\sin(\alpha-\beta) = \sin(\alpha)  \times \cos(\beta) - \cos(\alpha)  \times \sin(\beta)       \\
	\cos(\alpha+\beta) = \cos(\alpha)  \times \cos(\beta) - \sin(\alpha)  \times  \sin(\beta)      \\
	\cos(\alpha - \beta) = \cos(\alpha)  \times \cos(\beta) + \sin(\alpha)  \times  \sin(\beta)    \\
	\tan(\alpha + \beta) = \frac{\tan(\alpha) + \tan(\beta)}{1- \tan(\alpha)  \times  \tan(\beta)} \\
	\tan(\alpha - \beta) = \frac{\tan(\alpha) - \tan(\beta)}{1+ \tan(\alpha)  \times  \tan(\beta)}
\end{align*}

\subsection{Double Angles}
\begin{align*}
	\sin(2\theta) = 2\sin(\theta) \cos(\theta)      \\
	\cos(2\theta) = \cos^2(\theta) - \sin^2(\theta) \\
	= 2\cos^2(\theta) - 1                           \\
	= 1- 2 \sin^2(\theta)                           \\
	\tan(2\theta)= \frac{2\tan(\theta)}{1 - \tan^2(\theta)}
\end{align*}

\end{document}
