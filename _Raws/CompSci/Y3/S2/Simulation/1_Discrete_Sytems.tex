\documentclass[12pt letter]{report}
\input{./template/preamble}
\input{./template/macros}
\input{./template/letterfonts}

\title{\Huge{Discrete Systems}}
\author{\huge{Madiba Hudson-Quansah}}
\date{}
\usepackage{parskip}

\setcounter{tocdepth}{4}
\setcounter{secnumdepth}{4}

\begin{document}
\maketitle
\newpage
\pdfbookmark[section]{\contentsname}{too}
\tableofcontents
\pagebreak

\chapter{Introduction to Modelling}

\section{The Modelling Framework}

\dfn{System}{
  Something in the real world we are interested in modelling
}

\dfn{Abstraction}{
  Deciding which elements of the real world to include and which parts to leave out
}

\dfn{Model}{
  The resulting abstracted system, which is a description of the system that includes only essential features. Models
  can be represented in diagrams, equations which can be used for mathematical analysis and computer programs which can
  run as \textit{simulations}
}

\dfn{Simulation}{
  The process of designing a model of a real system, and conducting experiments with this model
}

The result of analysis and simulation could be a prediction about what the system will do, an explanation the system's
behaviour, or a design intended to achieve a purpose.\\

Predictions can be validated and testing by taking measurements from the real world and comparing the data obtained with
the results from analysis and simulation. \\

Some reasons to simulate a system include:
\begin{itemize}
  \item Estimate some quantity or measure of effectiveness (MOE)
  \item Gain an understanding of the behaviour of the system
  \item Evaluate various alternative strategies
\end{itemize}

\dfn{Iterative Modelling}{
  Starting with a simple model, and adding on features gradually starting with the most essential ones.
}

\dfn{Internal Validation}{
  Comparing the results of successive models from iterative modelling.
}


\dfn{External Validation}{
  Comparing the results from the real world to generated models.
}

\subsection{Types of Models}
\begin{description}
  \item[Physical models]  - Resemble the system being studied.
  \item[Scaled models] - Resemble the system under study but at a different size, i.e. a scaled up model of an atom.
  \item[Analog models] - A model where the properties of the real object is represented by a substituted property that
        often behaves in a similar manner, e.g. voltage through an electronic analogue computer network may represent flow of
        goods through a system.
  \item[Schematic model] - A pictorial representation of a system, e.g. a blueprint or a graph.
  \item[Games / Man-Machine models] - Management games, war games, planning competition.
  \item[Simulation models] - Have no human interaction, an abstract model, such as a discrete-event system simulation
        model.
  \item[Mathematical models] - A model where symbols represent entities.
  \item[Heuristic model] - A collection of descriptors or decision rules, usually computer based which is not
        limited by physical, diagrammatic or mathematical bounds.
\end{description}

\subsection{Characteristics of Models}

\begin{description}
  \item[Static Model] - A model that does not change / move with time. Examples include Monte Carlo Sampling.
  \item[Dynamic Model] - A model that changes with time. Examples include Time series regression.
  \item[Deterministic Model] - A model that has a single solution for a given input state or configuration, like a pure
        function in functional programming.
  \item[Stochastic / Non-Deterministic Model] - A model whose output mimics probabilistic/random phenomena.
  \item[Discrete Model] - A model that represents the system in a discrete manner where variables change at distinct and
        countable points in time.
  \item[Continuous Model] - A model that represents the system in a continuous manner where variables change continuously
        over time.
\end{description}

\section{Testing the Falling Penny Myth}

\dfn{The Falling Penny Myth}{
  A penny dropped from the top of the Empire State building would be going so fast when it hit the pavement that it
  would be embedded in the concrete, or if it hit a person it would break their skull
}

\nt{
  The Standard Equations of Motion (SUVAT):
  \begin{align*}
    v   & = u + at                           \\
    s   & = ut + \frac{1}{2} at^2            \\
    s   & = \left( \frac{u + v}{2} \right) t \\
    v^2 & = u^2 + 2as                        \\
    s   & = vt + \frac{1}{2} at^2            \\
  \end{align*}
  Where:
  \begin{description}
    \item[ $s$] - Distance
    \item[ $u$] - Initial Velocity
    \item[ $v$]  - Velocity
    \item[ $a$] - Acceleration
    \item [ $t$] - Time
  \end{description}
}

We can test this by considering two models. The first disregards the effect of air resistance, in which case the
primary force acting on the penny is gravity, which causes the penny to accelerate downward.

If the initial velocity is 0 $u = 0$ and the acceleration $a$ is constant the velocity after $t$ seconds can be found
using the one of the standard equations of motion:
\[
  v = u + at = at
\]
and the distance the penny is dropped is:
\[
  s = ut + \frac{1}{2} a t^2 = \frac{1}{2} at^2
\]

To find the time the penny hits the sidewalk we can make $t$ the subject of this equation:
\begin{align*}
  s            & = \frac{1}{2} at^2    \\
  2s           & = at^2                \\
  \frac{2s}{a} & = t^2                 \\
  t            & = \sqrt{\frac{2s}{a}}
\end{align*}

When $a = 9.8 \text{m/s}^2$ due to gravity, and the height of the Empire State Building $s = 381 \text{m}$, $t = 8.8
  \text{s}$. Computing the velocity we get $86.24 \text{m/s}$. However we must take into account that this model is
based on simplifications, such as gravity is constant. Gravity is however different on different parts of the globe
and it gets weaker as you move away from the surface of the earth. This however results in small negligible differences
so ignoring them can be a good choice for this problem.

Disregarding air resistance however is not a good choice as it has substantial effects on the velocity of the penny, as
when the penny reaches about $29 \text{m/s}$ the upward force of air resistance equals the downward force of gravity and
the penny stops accelerating, this point is known as the penny's terminal velocity in air.

This leads into the second model where the penny accelerates until it reaches terminal velocity of $29 \text{m/s}$ and
hits the sidewalk at this velocity, which is not fast enough to damage it.




\end{document}
