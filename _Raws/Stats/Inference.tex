\documentclass[12pt letter]{report}
\input{./template/preamble}
\input{./template/macros}
\input{./template/letterfonts}

\title{\Huge{Inference}}
\author{\huge{Madiba Hudson-Quansah}}
\date{}
\usepackage{parskip}

\setcounter{tocdepth}{4}
\setcounter{secnumdepth}{4}

\begin{document}
\maketitle
\newpage
\pdfbookmark[section]{\contentsname}{too}
\tableofcontents
\pagebreak

\chapter{Module 13: Inference}

\dfn{Statistical Inference}{
  Inferring something about a population from a sample.
}

\dfn{Point Estimation}{
  Estimating an unknown parameter using a single number, calculated from the sample data.
}

\dfn{Interval Estimation}{
  Estimating an unknown parameter using an Interval of values that is likely to contain the true value of the
  parameter, and state how confident we are that the interval contains the true value.
}

\dfn{Hypothesis Testing}{
  Making decisions about the population parameter based on the sample data.
}

\section{Inference for One Variable}

Depending on the type of variable we are interested in the population parameter we infer about changes:
\begin{itemize}
  \item Categorical : Population Proportion $p$
  \item Quantitative : Population Mean $\mu$
\end{itemize}

\chapter{Module 14: Estimation}

\section{Point Estimation}

\dfn{Point Estimator}{
  A statistic that provides an estimate of a population parameter.
}

The point estimator also changes based on the type of variable examined:
\begin{itemize}
  \item Categorical: Sample Proportion / $\hat{p}$
  \item Quantitative: Sample Mean /  $\bar{x}$
\end{itemize}

\nt{
  The larger the sample size, the more accurate the point estimate.
}

\section{Interval Estimation}

\dfn{Confidence Interval}{
  An interval of values that is likely to contain the true value of the population parameter.
}

Interval estimation is based on the point estimate and the margin of error.

\subsection{Confidence Intervals for the Population Mean}

For a quantitative variable with a normally distributed sample mean distribution due to the Central Limit Theorem,
construing a 95\% confidence interval consists of the following steps:
\begin{itemize}
  \item Identify mean $\overline{X}$, which for a sample mean distribution is approximately equal to $\mu $
  \item Find the standard deviation $S$ of the sample mean distribution, $\frac{\sigma}{\sqrt{n}}$
  \item Find $\overline{X} \pm 2 S $, which are your upper and lower bounds of the confidence interval
\end{itemize}

Therefore generally the confidence interval is:
\[
  \overline{x} \pm 2 \times \frac{\sigma}{\sqrt{n}}
\]

\subsubsection{Other Levels of Confidence}

Constructing a 99\% confidence interval for $\mu $ can be done using:
\[
  \overline{x} \pm 2.576 \times \frac{\sigma}{\sqrt{n}}
\]

And a 90\% confidence interval for $\mu $ can be found using:
\[
  \overline{x} \pm 1.645 \times \frac{\sigma}{\sqrt{n}}
\]

\subsection{General Structure of Confidence Intervals}

A confidence interval has the following form:
\[
  \overline{x} \pm z^{*} \times \frac{\sigma }{\sqrt{n}}
\]

Where $z^*$ is general notation for the multiplier that depends on the level of confidence. \\
The confidence interval can then also be expressed in the form:
\[
  \overline{x} \pm m
\]

Where $m = z^* \times \frac{\sigma}{\sqrt{n}}$ and $\overline{x}$ is the point estimator for the unknown population mean $\mu $\\
$m$ is called the margin of error, since it represents the maximum estimation error for a given level of confidence.

\nt{
  A larger sample size makes for a smaller margin of error.
}

\subsection{Sample Size Calculations}

The sample size required to estimate the population mean $\mu $ with a margin of error $m$ at a level of confidence $C$ can be found using:
\[
  n = \left( \frac{z^*  \sigma }{m} \right)^2
\]
Which is rounded up to the nearest whole number.

\subsection{When $\sigma $ is unknown}

When the population standard deviation $\sigma $ is unknown, the sample standard deviation $s$ is used instead, but as a
result we need to use a different set of confidence multipliers $t^*$, associated with the $t$ distribution. The
interval is therefore:
\[
  \overline{x} \pm t^* \times \frac{s}{\sqrt{n}}
\]

These multipliers depend not only on the level of confidence, but also on the sample size $n$. \\
For large values of $n$, the $t$ distribution approaches the standard normal distribution, and the $t^*$ multipliers,
therefore the $z^*$ multipliers can be used, i.e. $t^* \approx z^*$ and the confidence interval becomes:
\[
  \overline{x} \pm z^* \times \frac{s}{\sqrt{n}}
\]


\section{Confidence Intervals for the Population Proportion}

For a categorical variable, the population proportion $p$ can be estimated using the sample proportion $\hat{p}$, and the
margin of error $m$, i.e. the confidence interval is:
\[
  \hat{p} \pm m
\]

Where $m$ is:

\[
  m = z^* \times \sqrt{\frac{\hat{p}(1-\hat{p})}{n}}
\]

Therefore:
\[
  \hat{p} \pm z^* \times \sqrt{\frac{\hat{p}\left( 1 - \hat{p} \right) }{n}}
\]

\subsection{Sample Size Calculations}

The sample size required to estimate the population proportion $p$ with a margin of error $m$ at a level of confidence
$C$ can be found using:

\[
  n = \frac{4\hat{p} \left( 1 - \hat{p} \right) }{m^2}
\]

\end{document}
