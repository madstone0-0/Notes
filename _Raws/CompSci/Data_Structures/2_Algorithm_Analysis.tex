\documentclass[12pt letter]{report}
\input{./template/preamble}
\input{./template/macros}
\input{./template/letterfonts}

\title{\Huge{Algorithm Analysis}}
\author{\huge{Madiba Hudson-Quansah}}
\date{}
\usepackage{parskip}

\setcounter{tocdepth}{4}
\setcounter{secnumdepth}{4}

\begin{document}
\maketitle
\newpage
\pdfbookmark[section]{\contentsname}{too}
\tableofcontents
\pagebreak

\chapter{Analysing Algorithms}

Analysing an algorithm involves prediction the resources that the algorithm will require. The resources that are of interest are:
\begin{itemize}
  \item Memory
  \item Computation Time
  \item Bandwidth
  \item Energy Consumption
\end{itemize}

In analysing an algorithm represented in pseudocode we can determine how long the algorithm takes to run by examining
how many times each line of the code is executed  and how long each line takes to execute. To do this we first derive a
precise formula for the running time, then simplify the formula using a convenient notation that allows us to compare
running times of different algorithms. For this example we will use the insertion-sort algorithm with the pseudocode
defined below:

\begin{algorithm}[H]
  \caption{Insertion-Sort $\left( A,n \right) $}
  \begin{algorithmic}[1]
    \For{index $= 2$ to $n$}
    \State prevIndex := index $- 1$
    \State nextElement := $A$[index]
    \Comment{Insert $A$[index] into the sorted array $A[1:\text{index}-1]$}
    \While{$A$[prevIndex] $>$ nextElement and prevIndex $\geq 0$}
    \State $A$[prevIndex $+ 1$] := $A$[prevIndex]
    \EndWhile
    \State     $A$[prevIndex $ + 1$] = nextElement
    \EndFor
  \end{algorithmic}
\end{algorithm}


We first must acknowledge that the running time of an algorithm depends on the input which we will call $A$. Also we
must acknowledge that the insertion-sort algorithm can take a different amount of time sorting two arrays of the same
size depending on how already sorted the arrays are. We then describe the running time of an algorithm as a function of
the size of the input $n$. To do so we must clearly define the terms "running time", "size of input", and be clear about
what scenario we are considering, whether it be best-case, worst-case, or average-case.

The best description of the \textit{size of input} depends on the problem being solved. In this case for a sorting algorithm the
best notion of the size of input is the number of elements in the array to be sorted. The \textit{running time} of an
algorithm is the number of instructions and data accesses made by the algorithm, how this is accounted for differs from
computer to computer but using the \textbf{RAM model} in which each simple operation takes a constant amount of time, we
can generalize the running time of an algorithm as the number of comparisons and data movements made by the algorithm.
Therefore in this framework each execution of the $k$th line of pseudocode takes $c_k$ time units where $c_k$ is a
constant.

First for each $i = 2,3,\ldots,n$, let $t_i$ denote the number of times the \lstinline{WHILE} loop test in line 5 is executed for
that value of $i$. Below is a table of lines, the number of times each line is executed and cost of each line.

\begin{table}
  \begin{center}
    \begin{tabular}[c]{|l|l|l|}
      \hline
      \multicolumn{1}{|c|}{\textbf{Line}} &
      \multicolumn{1}{c|}{\textbf{Cost}}  &
      \multicolumn{1}{c|}{\textbf{Times}}                                                    \\
      \hline
      1                                   & $c_1$ & $n$                                      \\

      2                                   & $c_2$ & $n-1$                                    \\

      3                                   & $c_3$ & $n-1$                                    \\

      4                                   & $c_4$ & $\sum_{i=2}^{n} t_i$                     \\

      5                                   & $c_5$ & $\sum_{i=1}^{n} \left( t_i - 1 \right) $ \\

      6                                   & $c_6$ & $\sum_{i=1}^{n} \left( t_i - 1 \right) $ \\

      7                                   & $c_7$ & $\sum_{i=1}^{n} \left( t_i - 1 \right) $ \\

      8                                   & $c_8$ & $n-1$                                    \\


      \hline
    \end{tabular}
  \end{center}
\end{table}

\chapter{Exercises}

\qs{}{
  Describe an algorithm that takes as input a list of $n$ integers and finds the number of negative integers in the list
}

\sol{
  \begin{algorithm}[H]
    \caption{Count-Negative $\left( A,n \right) $}
    \begin{algorithmic}[1]
      \State count := 0
      \For{i := 1 to $n$}
      \If{$A$[i] < $0$}
      \State count = count + 1
      \EndIf
      \EndFor
      \State \Return count
    \end{algorithmic}
  \end{algorithm}
}

\qs{}{
  Describe an algorithm that takes as in put a list of $n$ integers and produces as outputs the largest difference
  obtained by subtracting an integer in the list from the one following it.
}

\sol{
  \begin{algorithm}[H]
    \caption{Largest-Difference $\left( A, n \right) $}
    \begin{algorithmic}[1]
      \State largest := $A_{2} - A_{1}$
      \For{i := 2 to $n - 1$}
      \State diff := $A_{i+1} - A_{i}$
      \If{ diff $>$ largest}
      \State largest = diff
      \EndIf
      \EndFor
      \State \Return largest
    \end{algorithmic}
  \end{algorithm}
}

\qs{}{

}

\sol{
  \begin{algorithm}[H]
    \caption{Repeated-Ints $\left( A, n \right) $}
    \begin{algorithmic}[1]
      \State j := 0
      \State i := 2
      \While{$ i \leq n$}
      \If{$A_i = A_{i-1}$}
      \State j := $j + 1$
      \State $B_j$ := $A_i$
      \Comment{$B$ is the list of repeated values}
      \While{$i \leq n$ and $A_i = C_i$}
      \State i := $i + 1$
      \EndWhile
      \EndIf
      \State i := $i + 1$
      \EndWhile
      \State \Return $B$
    \end{algorithmic}
  \end{algorithm}
}



\end{document}
