\documentclass[12pt letter]{report}
\input{../template/preamble}
\input{../template/macros}
\input{../template/letterfonts}

\title{\Huge{Limits and Continuity}}
\author{\huge{Madiba Hudson-Quansah}}
\date{}
\usepackage{parskip}

\begin{document}
\maketitle
\newpage
\pdfbookmark[section]{\contentsname}{too}
\tableofcontents
\pagebreak

\chapter{}

\begin{itemize}
	\item A function's limit can be one of it's own outputs. When this happens the function called a continuous function.

	\item A Limit of a function is not necessarily one of the function's output values.

	\item A Limit of a function can be one of the function values.

\end{itemize}

\[
	\lim_{x\to a} f(x) = f(a) \\
\]

If a limit can be found at $x = a$ then we can say this function is continuous as $x = a$

This means for a function to be continuous at $x = a$

\begin{itemize}
	\item $f(a)$ must exist
	\item $\lim_{x \to a} f(x)$ must exist
	\item $\lim_{x\to a} f(x) = f(a)$
\end{itemize}

% \subsection{Algebraic Test for Limits}
\subsection{Checking continuity graphically}

A graph with no breaks is continuous. Graphically a continuous function should have none of the following

\begin{itemize}
	\item Holes - Holes in a graph are called removable discontinuity, because they can be determined by studying neighbouring values using Limits.
	\item Jumps, due to a sudden change in values. \begin{align*}
		      f(x) = \begin{cases}
			             x^2, \;\; \text{if } x< 1 \\
			             x + 4, \;\; \text{if } x \geq 1
		             \end{cases}
	      \end{align*}
\end{itemize}

\end{document}
