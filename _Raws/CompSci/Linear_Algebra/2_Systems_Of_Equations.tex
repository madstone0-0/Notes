\documentclass[12pt letter]{report}
\input{./template/preamble}
\input{./template/macros}
\input{./template/letterfonts}

\title{\Huge{Systems of Linear Equations}}
\author{\huge{Madiba Hudson-Quansah}}
\date{}
\usepackage{parskip}

\setcounter{tocdepth}{4}
\setcounter{secnumdepth}{4}

\begin{document}
\maketitle
\newpage
\pdfbookmark[section]{\contentsname}{too}
\tableofcontents
\pagebreak

\chapter{Introduction}

\dfn{Linear Equation}{
  An equation in the form
  \[
    a_1x_1 + a_2x_2 + \ldots + a_nx_n = b
  \]
  where the constant $b$ and coefficients $a_1, a_2, \ldots, a_n$  are real or complex numbers.
}

\dfn{System of Linear Equations}{
  A collection of one or more linear equations involving the same set of variables. When a system of linear equations is written in the form
  \begin{align*}
    a_1 x_1 + a_2 x_2 + a_3 x_3 = b_1 \\
    a_4 x_1 + a_6 x_3 = b_2           \\
  \end{align*}
  The set of variables takes on the longest subscript in the system. In this case, the variables are $x_1, x_2, x_3$.
}

\dfn{Solution of a System of Linear Equations}{
  The \textit{solution} of a system of linear equations is a list of values, $\left( s_1, s_2, \ldots, s_n \right) $
  that makes each equation in the system a true statement when the values are substituted for the variables, i.e. $x_1,
    x_2, \ldots, x_n$ and $s_1, s_2, \ldots, s_n$, where $s_n$ is substituted for $x_n$
}

\dfn{Solution Set}{
  The set of all possible solutions of a system of linear equations.
}

\dfn{Equivalence}{
  Two linear systems are said to be \textit{equivalent} if they have the same solution set.
}

\dfn{Consistency}{
  A system of linear equations is said to be \textit{consistent} if it has at least one solution, and
  \textit{inconsistent} if it has no solution.
}

A system of linear equations can either have:
\begin{itemize}
  \item No solution - Equations do not intersect
  \item Exactly one / Unique solution - Equations intersect at a single point
  \item Infinitely many solutions - Equations are the same
\end{itemize}

\section{Matrix Notation}

A system of linear equations can be represented in matrix form two ways:
\begin{itemize}
  \item Coefficient Matrix
  \item Augmented Matrix
\end{itemize}

\subsection{Coefficient Matrix}

\dfn{Coefficient Matrix}{
  Denoted by $A$, the coefficient matrix is a matrix that contains the coefficients of the variables in the system of
  linear equations with the coefficients of each equation making up each row.
}

\ex{}{
  For the system of linear equations:
  \begin{align*}
    a_1 x_2 + a_2 x_2 + a_3 x_3 = b_1 \\
    a_4 x_1 + a_5 x_2 + a_6 x_3 = b_2 \\
    a_7 x_1 + a_8 x_2 + a_9 x_3 = b_3 \\
  \end{align*}
  The coefficient matrix is:
  \[
    \begin{bmatrix} a_1 & a_2 & a_3 \\ a_4 & a_5 & a_6 \\  a_7 & a_8 & a_9 \end{bmatrix}
  \]

}

\subsection{Augmented Matrix}
\dfn{Augmented Matrix}{
  Denoted by $[A|B]$, the augmented matrix is a matrix that contains the coefficients of the variables in the system of linear equations with the constant terms of each equation making up the last column.
}

\ex{}{
  For the system of linear equations:
  \begin{align*}
    a_1 x_2 + a_2 x_2 + a_3 x_3 = b_1 \\
    a_4 x_1 + a_5 x_2 + a_6 x_3 = b_2 \\
    a_7 x_1 + a_8 x_2 + a_9 x_3 = b_3 \\
  \end{align*}
  The augmented matrix is:
  \[
    \begin{bmatrix} a_1 & a_2 & a_3 & b_1\\ a_4 & a_5 & a_6 & b_2 \\ a_7 & a_8 & a_9  & b_3 \end{bmatrix}
  \]
}

\dfn{Size of a Matrix}{
  The size of a matrix, denoted by $m \times n$, is the number of rows and columns in the matrix respectively.
  If $n = m$ then the matrix is said to be square, if not, it is said to be rectangular.
}

\section{Solving Linear Systems}

\dfn{Pivot}{
  Diagonal non-zero elements  of a linear system
}

\dfn{Forward Elimination Process}{
  The process used to change a system into an upper triangular matrix
}
\dfn{Backward Substitution Method}{
  The process of deriving a solution from an upper triangular matrix
}

\dfn{Identity Matrix}{
  A matrix containing all zeros with pivots of 1
  \[
    \begin{bmatrix} 1 & 0 & 0 \\ 0 & 1 & 0 \\ 0 & 0 & 1 \end{bmatrix}
  \]
}

\dfn{Upper triangular matrix / Echelon Form}{

}


One procedure used to solve linear systems is that of \textit{simplification}. This involves replacing one linear system
with a simpler equivalent system. This is done by applying the following operations to the system:
\begin{description}
  \item[Replacement] Replace one equation by the sum of itself and a multiple of another equation.
  \item[Interchange] Interchange two equations.
  \item[Scaling] Multiply all the terms in an equation by a non-zero constant.
\end{description}

\ex{}{
  \qs{}{
    Solve the system
    \begin{align*}
      x_1    -2x_2   + x_3    = 0 \\
      2x_2    - 8x_3   = 8        \\
      5x_1           -5x_3    = 10
    \end{align*}
  }

  \sol{
    Using the augmented matrix representation, we have:
    \[
      \begin{bmatrix} 1 & -2 & 1 & 0 \\ 0 & 2 & -8 & 8 \\ 5 & 0 & -5 &10   \end{bmatrix}
    \]
    Then we times the first equation through by $-5$ and add it to the third equation to replace the third equation:
    \begin{align*}
      \begin{split}
        -5x_1 + 10x_2 - 5x_3 = 0 \\
        5x_1 - 5x_3 = 10         \\
        \hline
        10x_2 - 10x_3 = 10       \\
      \end{split}
    \end{align*}
    Giving us:
    \[
      \begin{bmatrix} 1 & -2 &  1 &  0  \\ 0 & 2 & -8 & 8 \\ 0 & 10 & -10 & 10  \end{bmatrix}
    \]
    We then eliminate $x_2$ by multiplying equation 2 by $-5$ and add it again to the third equation again replacing it:
    \begin{align*}
      \begin{split}
        -10x_2 + 40x_3 = -40 \\
        10x_2 - 10x_3 = 10   \\
        \hline
        30x_3 = -30          \\
      \end{split}
    \end{align*}
    Giving us:
    \[
      \begin{bmatrix} 1 & -2 & 1 & 0 \\ 0 & 2 & -8 & 8 \\ 0 & 0 & 30 & -30 \end{bmatrix}
    \]
    This new system has a triangular form, i.e.
    \begin{align*}
      x_1 - 2x_2 + x_3 = 0 \\
      2x_2 - 8x_3 = 8      \\
      30x_3 = -30          \\
    \end{align*}

    \noindent We then continue eliminating variables until one remains in each equation:
    \begin{align*}
      \begin{split}
        -x_3 = 1             \\
        x_1 - 2x_2 + x_3 = 0 \\
        \hline
        x_1 - 2x_2 = 1
      \end{split}
    \end{align*}
    \[
      \begin{bmatrix} 1 & -2 & 0 & 1 \\ 0 & 2 & -8 & 8 \\ 0 & 0 & 30  &-30 \end{bmatrix}
    \]


    \begin{align*}
      \begin{split}
        8x_3 = -8       \\
        2x_2 - 8x_3 = 8 \\
        \hline
        2x_2 = 0        \\
      \end{split}
    \end{align*}

    \[
      \begin{bmatrix} 1 & -2 & 0 & 1 \\ 0 & 2& 0 & 0 \\ 0 & 0 & 30 & -30  \end{bmatrix}
    \]

    \begin{align*}
      \begin{split}
        2x_2 = 0       \\
        x_1 - 2x_2 = 1 \\
        \hline
        x_1 = 1        \\
      \end{split}
    \end{align*}

    \[
      \begin{bmatrix} 1 & 0 & 0 & 1 \\ 0 & 2 & 0 & 0 \\ 0 & 0 & 30 & -30  \end{bmatrix}
    \]

    Giving us the system:
    \begin{align*}
      x_1 = 1     \\
      2x_2 = 0    \\
      30x_3 = -30 \\
    \end{align*}
    Which simplifies into:
    \begin{align*}
      x_1 = 1 \\
      x_2 = 0 \\
      x_3 = -1
    \end{align*}

  }
}

\dfn{Row Equivalence}{
  Two matrices are row equivalent if there is a sequence of elementary row operations that transforms one matrix into
  the other
}

\thm{}{
  If the augmented matrices of two linear systems are row equivalent, then the two equations have the same solution set.
}

\section{Identifying Existence and Uniqueness}

To determine the nature of a linear system we must answer two fundamental questions:
\begin{itemize}
  \item Is the system consistent? / Does a solution exist?
  \item If a solution exists, is it the only one? / Is the solution unique
\end{itemize}

\ex{}{
  \qs{}{
    Determine if the following system is consistent:
    \begin{align*}
      x_1    -2x_2   + x_3    = 0 \\
      2x_2    - 8x_3   = 8        \\
      5x_1           -5x_3    = 10
    \end{align*}
  }

  \sol{
    Having already found the solution for this system:
    \begin{align*}
      x_1 = 1 \\
      x_2 = 0 \\
      x_3 = -1
    \end{align*}
    We can determine that a solution exists, and due to the fact $x_2$ is uniquely determined by equation two, $x_3$ has
    only one possible value, and $x_1$ is also uniquely determined by equation one, we can also conclude this solution
    is unique.
  }
}

\ex{}{
  \qs{}{
    Determine if the following system is consistent:
    \begin{align*}
      x_2 - 4x_3 = 8         \\
      2x_1 - 3x_2 + 2x_3 = 1 \\
      4x_1 - 8x_2 + 12x_3 = 1
    \end{align*}
  }

  \sol{
    The augmented matrix is:
    \[
      \begin{bmatrix}  0 & 1 & -4 & 8 \\ 2 & -3 & 2 & 1 \\  4 & -8 & 12 & 1 \end{bmatrix}
    \]

    We interchange equations 1 and 2:
    \[
      \begin{bmatrix} 2 & -3 & 2 & 1 \\ 0 & 1 & -4 & 8 \\ 4 & -8 & 12 & 1 \end{bmatrix}
    \]

    \begin{align*}
      \begin{split}
        -4x_1 + 6x_2 - 4x_3 = -2 \\
        4x_1 - 8x_2 + 12x_3 = 1  \\
        \hline
        -2x_2 + 8x_3 = -1
      \end{split}
    \end{align*}

    \[
      \begin{bmatrix} 2 & -3 & 2 & 1 \\ 0 & 1 & -4 & 8 \\ 0 & -2 & 8 & -1 \end{bmatrix}
    \]

    \begin{align*}
      \begin{split}
        2x_2 - 8x_3 = 16  \\
        -2x_2 + 8x_3 = -1 \\
        \hline
        0 = 15
      \end{split}
    \end{align*}

    \[
      \begin{bmatrix} 2 & -3 & 2 & 1 \\ 0 & 1 & -4 & 8 \\ 0 & 0 & 0 & 15 \end{bmatrix}
    \]

    Now in its triangular form, we can determine the existence and uniqueness of the solutions:
    \begin{align*}
      2x_1 - 3x_2 + 2x_3 = 1 \\
      x_2 - 4x_3 = 8         \\
      0 = 15                 \\
    \end{align*}

    Since there are no coefficients for $x_1$, $x_2$, and $x_3$ in equation 3 equation 3 has no solution. This makes the
    solution set for this linear system $\{1, 8\}$. Because this set is the same as the solution set for the original
    linear system, $\{8, 1, 1\} $, the original system is inconsistent


  }
}

\section{Exercises}

\qs{}{
  Determine if the linear system represented by the augmented matrix below is consistent:
  \[
    \begin{bmatrix} 1 & 5 & 2 & -6 \\ 0 & 4 & - 7 & 2 \\ 0 & 0 & 5 & 0 \end{bmatrix}
  \]
}

\sol{
  \begin{align*}
    x_1 + 5x_2 + 2x_3 = -6                   \\
    4x_2 -7x_3 = 2                           \\
    5x_3 = 0                                 \\
    \\
    x_3 = 0                                  \\
    x_1 + 5x_2 = -6                          \\
    x_1 = -6 - 5x_2                          \\
    \\
    4x_2 = 2                                 \\
    x_2 = \frac{1}{2}                        \\
    \\
    x_1  = - 6- 5 \left( \frac{1}{2} \right) \\
    x_1 = -\frac{17}{2}
  \end{align*}

}

\qs{}{
  Solve the following systems:
  \begin{enumerate}
    \item
          \begin{align*}
            x_2 + 4x_3 = -5        \\
            x_1 + 3x_2 + 5x_3 = -2 \\
            3x_1 + 7x_2 + 7x_3 = 6
          \end{align*}
    \item
          \begin{align*}
            x_1 - 2x_4 = -3 \\
            2x_2 + 2x_3 = 0 \\
            x_3 + 3x_4 = 1  \\
            -2x_1 + 3x_2 + 2x_3 + x_4 = 5
          \end{align*}
  \end{enumerate}

}

\sol{
  \begin{enumerate}
    \item
          \[
            \begin{bmatrix} 0 & 1 & 4 & -5 \\ 1&3 & 5 & -2 \\ 3 & 7 & 7 & 6 \end{bmatrix}
          \]
          $R_1 \leftrightarrow R_3$
          \[
            \begin{bmatrix} 3 & 7 & 7 & 6 \\ 1&3 & 5 & -2 \\  0 & 1 & 4 & -5 \end{bmatrix}
          \]
          $\frac{1}{3}R_1 - R_2 \to R_2$
          \begin{align*}
            \begin{split}
              x_1 + \frac{7}{3}x_2 + \frac{7}{3}x_3 = 2 \\
              x_1 + 3x_2 + 5x_3 = -2                    \\
              \hline
              -\frac{2}{3}x_2 - \frac{8}{3}x_3 = 4
            \end{split}
          \end{align*}

          \[
            \begin{bmatrix} 3 & 7 & 7 & 6 \\ 0&-\frac{2}{3} & -\frac{8}{3} & 4 \\  0 & 1 & 4 & -5 \end{bmatrix}
          \]
          $3R_2$
          \[
            \begin{bmatrix} 3 & 7 & 7 & 6 \\ 0&-2 & -8 & 12 \\  0 & 1 & 4 & -5 \end{bmatrix}
          \]

          $-\frac{1}{2}R_2 - R_3 \to R_3$
          \begin{align*}
            \begin{split}
              x_2 + 4x_3 = -6 \\
              x_2 + 4x_3 = -5 \\
              \hline
              0 = -1          \\
            \end{split}
          \end{align*}

          \[
            \begin{bmatrix} 3 & 7 & 7 & 6 \\ 0&-2 & -8 & 12 \\  0 & 0 & 0 & -1 \end{bmatrix}
          \]

          Because the system has a contradiction in row 3, $0x_1 + 0x_2 + 0x_3 = -1$, the system has no solution and is
          therefore inconsistent.

    \item
          \[
            \begin{bmatrix} 1 & 0 & 0 & -2 & -3 \\ 0 & 2 & 2 & 0 & 0 \\ 0 & 0 & 1 & 3 & 1 \\ -2 & 3 & 2 & 1 & 5 \end{bmatrix}
          \]

          $R_1 \leftrightarrow R_4$
          \[
            \begin{bmatrix}
              -2 & 3 & 2 & 1  & 5  \\
              0  & 2 & 2 & 0  & 0  \\
              0  & 0 & 1 & 3  & 1  \\
              1  & 0 & 0 & -2 & -3 \\
            \end{bmatrix}
          \]

          $-\frac{1}{2} R_1 - R_4 \to R_4$
          \begin{align*}
            \begin{split}
              x_1 - \frac{3}{2}x_2 - x_3 - \frac{1}{2}x_4 = -\frac{5}{2} \\
              x_1 + 0 + 0 - 2x_4 = -3                                    \\
              \hline
              -\frac{3}{2}x_2 - x_3 + \frac{3}{2}x_4 = \frac{1}{2}
            \end{split}
          \end{align*}
          \[
            \begin{bmatrix}
              -2 & 3            & 2  & 1           & 5           \\
              0  & 2            & 2  & 0           & 0           \\
              0  & 0            & 1  & 3           & 1           \\
              0  & -\frac{3}{2} & -1 & \frac{3}{2} & \frac{1}{2} \\
            \end{bmatrix}
          \]

          $-\frac{3}{4}R_2 - R_4 \to R_4$
          \begin{align*}
            \begin{split}
              0  -\frac{3}{2}x_2 - \frac{3}{2}x_3 + 0 = 0             \\
              0 - \frac{3}{2}x_2 - x_3 + \frac{3}{2}x_4 = \frac{1}{2} \\
              \hline
              -\frac{1}{2}x_3 - \frac{3}{2}x_4 = -\frac{1}{2}
            \end{split}
          \end{align*}

          \[
            \begin{bmatrix}
              -2 & 3 & 2            & 1            & 5            \\
              0  & 2 & 2            & 0            & 0            \\
              0  & 0 & 1            & 3            & 1            \\
              0  & 0 & -\frac{1}{2} & -\frac{3}{2} & -\frac{1}{2} \\
            \end{bmatrix}
          \]

          $-\frac{1}{2}R_3 - R_4 \to R_4$

          \begin{align*}
            \begin{split}
              0 + 0 - \frac{1}{2}x_3 - \frac{3}{2}x_4 = -\frac{1}{2} \\
              0 + 0-\frac{1}{2}x_3 - \frac{3}{2}x_4 = -\frac{1}{2}   \\
              \hline
              0x_1 + 0x_2 + 0x_3 + 0x_4 = 0
            \end{split}
          \end{align*}

          \[
            \begin{bmatrix}
              -2 & 3 & 2 & 1 & 5 \\
              0  & 2 & 2 & 0 & 0 \\
              0  & 0 & 1 & 3 & 1 \\
              0  & 0 & 0 & 0 & 0 \\
            \end{bmatrix}
          \]

          $\frac{3}{2}R_2 - R_1 \to R_1$

          \begin{align*}
            \begin{split}
              0 + 3x_2 + 3x_3 + 0 = 0      \\
              -2x_1 + 3x_2+ 2x_3 + x_4 = 5 \\
              \hline
              2x_1 + 0 + x_3 - x_4 = -5
            \end{split}
          \end{align*}

          \[
            \begin{bmatrix}
              2 & 0 & 1 & -1 & -5 \\
              0 & 2 & 2 & 0  & 0  \\
              0 & 0 & 1 & 3  & 1  \\
              0 & 0 & 0 & 0  & 0  \\
            \end{bmatrix}
          \]
          $R_3 - R_1 \to R_1$
          \begin{align*}
            \begin{split}
              0 + 0 + x_3 + 3x_4 = 1    \\
              2x_1 + 0 + x_3 - x_4 = -5 \\
              \hline
              -2x_1 + 0 + 0 + 4x_4 = 6
            \end{split}
          \end{align*}

          \[
            \begin{bmatrix}
              -2 & 0 & 0 & 4 & 6 \\
              0  & 2 & 2 & 0 & 0 \\
              0  & 0 & 1 & 3 & 1 \\
              0  & 0 & 0 & 0 & 0 \\
            \end{bmatrix}
          \]
          $2R_3 - R_2 \to R_2$
          \begin{align*}
            \begin{split}
              0 + 0 + 2x_3 + 6x_4 = 2 \\
              0 + 2x_2 + 2x_3 + 0 = 0 \\
              \hline
              -2x_2 + 6x_4 = 2
            \end{split}
          \end{align*}

          \[
            \begin{bmatrix}
              -2 & 0  & 0 & 4 & 6 \\
              0  & -2 & 0 & 6 & 2 \\
              0  & 0  & 1 & 3 & 1 \\
              0  & 0  & 0 & 0 & 0 \\
            \end{bmatrix}
          \]
          $\frac{R_1}{-2}$
          \[
            \begin{bmatrix}
              1 & 0  & 0 & -2 & -3 \\
              0 & -2 & 0 & 6  & 2  \\
              0 & 0  & 1 & 3  & 1  \\
              0 & 0  & 0 & 0  & 0  \\
            \end{bmatrix}
          \]
          $\frac{R_2}{-2}$
          \[
            \begin{bmatrix}
              1 & 0 & 0 & -2 & -3 \\
              0 & 1 & 0 & -3 & -1 \\
              0 & 0 & 1 & 3  & 1  \\
              0 & 0 & 0 & 0  & 0  \\
            \end{bmatrix}
          \]

          \begin{align*}
            x_1 + -2x_4 = -3                   \\
            x_2 -3x_4 = -1                     \\
            x_3 + 3x_4 = 1                     \\
            0 = 0                              \\
            \\
            x_4 = \frac{1}{2}x_1 + \frac{3}{2} \\
            x_2 = -1 + 3x_4                    \\
            x_3 = 1 - 3x_4                     \\
            x_1 = -3 + 2x_4
          \end{align*}

  \end{enumerate}
}

\pagebreak
\qs{}{
  For the following matrices find the elementary row operation that transforms
  the first matrix into the second, and then find the reverse row operation that
  transforms the second matrix into the first
  \begin{enumerate}
    \item
          \[
            \begin{bmatrix}
              1 & 3   & -1 \\
              0 & -2  & 6  \\
              0 & - 5 & 9
            \end{bmatrix} ,
            \begin{bmatrix}
              1 & 3  & -4 \\
              0 & 1  & -3 \\
              0 & -5 & 9
            \end{bmatrix}
          \]
    \item
          \[
            \begin{bmatrix}
              1 & -2 & 1  & 0  \\
              0 & 5  & -2 & 8  \\
              4 & -1 & 3  & -6
            \end{bmatrix} ,
            \begin{bmatrix}
              1 & -2 & 1  & 0  \\
              0 & 5  & -2 & 8  \\
              0 & 7  & -1 & -6
            \end{bmatrix}
          \]
  \end{enumerate}
}

\sol{
  \begin{enumerate}
    \item
          Let the first matrix be $M_1$ and the second be $M_2$
          \begin{align*}
            M_1 \to M_2 & = -\frac{1}{2}R_2          \\
            M_2 \to M_1 & = \frac{R_2}{-\frac{1}{2}} \\
          \end{align*}
    \item
          Let the first matrix be $M_1$ and the second be $M_2$
          \begin{align*}
            M_1 \to M_2 & = -4 R_1 + R_3 \to R_3 \\
            M_2 \to M_1 & = R_3 - 4R_1           \\
          \end{align*}
  \end{enumerate}
}


\end{document}
