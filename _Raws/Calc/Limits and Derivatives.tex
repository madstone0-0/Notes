\documentclass[12pt letter]{report}
\input{./template/preamble}
\input{./template/macros}
\input{./template/letterfonts}

\title{\Huge{Limits and Derivatives}}
\author{\huge{Madiba Hudson-Quansah}}
\date{March 2023}
\usepackage{parskip}

\begin{document}
\maketitle
\newpage
\pdfbookmark[section]{\contentsname}{toc}
\tableofcontents
\pagebreak
\numberwithin{equation}{section}
\chapter{Limits}
\section{Cases}

\subsection{Case 1}

Changing input values, output values remain the same $\therefore$ The change in $y$ with respect to changes $x$ values is $0$.

This function is called a constant function

\[
	f(x) = k, \text{ for all input } x
\]

\subsection{Case 2}

Changing input values result in changing output values/ different output values.

\[
	f(x) = \text{ not a constant function}
\]

\subsection{Case 3}

Changing input values result in a constant change in output values.

\[
	\Delta f(x) = \text{ constant}
\]

This function is called a linear function.

\subsection{Case 4}


Changing input values result in a non constant change in output values
\[
	\Delta f(x) = \text{ not constant}
\]

\pagebreak
\subsection{Questions}
\qs{}{
	Find the derivative of $f(x) = x^2$ at $x=1$
}
\sol{
	\begin{align*}
		{\frac{d}{dx}f(x)}_{x\to 1^-}  = 2 \\
		{\frac{d}{dx}f(x)}_{x\to 1^+}  = 2 \\
		 & \frac{d}{dx} x^2 = 2            \\
	\end{align*}
}


\qs{}{
	Investigate the existence or otherwise the derivatives if the following functions:
	\begin{align}
		\label{q:1}f(x) =\begin{cases}
			                 x^2, & \text{if } \leq 2 \\
			                 1+2x & \text{if } > 2
		                 \end{cases}
		\text{at }x=2   \\
		\label{q:2}g(x) = |x| = \begin{cases}
			                        -x, & \text{if } x<0      \\
			                        x,  & \text{if } x \geq 0
		                        \end{cases}
		\text{at } x= 0 \\
		\nonumber\end{align}
}
\sol{\ref{q:1}}{
	\begin{align*}
		\frac{d}{dx}f(x)_{x\to 2^{-}}  = 4 \\
		\frac{d}{dx}f(x)_{x\to 2^{+}}  = \text{is not constant}
	\end{align*}
}

\sol{\ref{q:2}}{
	\begin{align*}
		\frac{d}{dx}g(x)_{x\to 0^-} = 1  \\
		\frac{d}{dx}g(x)_{x \to 0^+} = 1 \\
		 & = \frac{d}{dx}g(x) = 1
	\end{align*}
}

\chapter{Derivatives}
\section{First Principle}
\begin{itemize}
	\item Slope or Gradient of a non-linear function
	\item Derivative = $\frac{dy}{dx}$ = Instantaneous rate of change of a function
	\item Slope of a line = $\frac{\Delta y}{\Delta x}$ = Average rate of change
	\item Leverage average rate of change to obtain the instantaneous rate of change = \textbf{First Principle}
	      \begin{equation*}
		      \frac{dy}{dx} = \lim_{h \to 0} \frac{f(x+h)- f(x)}{h}
	      \end{equation*}
	\item The derivative is a function at a point and a number at another point
\end{itemize}

\subsubsection{
	\dfn{Secant}{A line going though two points on a curve}
}

\subsection{Questions}

\qs{}{Find the derivative of $f(x) = x^2$ using first principle}
\sol{
	\begin{align*}
		\frac{df}{dx} = \lim_{h \to 0} \frac{f(x+h) -f(x)}{h}         \\
		\frac{(x+h)^2-x^2}{h}                                         \\
		\frac{2hx+h^2}{h}                                             \\
		2x + (0)                                                      \\
		 & \frac{df}{dx} = \lim_{h \to 0} \frac{f(x+h) -f(x)}{h} = 2x
	\end{align*}
}

\qs{}{Find the derivative of $f(x) = \frac{1}{x}$ using first principle}

\sol{
	\begin{align*}
		\frac{df}{dx} = \lim_{h \to 0} \frac{f(x+h) -f(x)}{h}                       \\
		\frac{df}{dx} = \lim_{h \to 0} (\frac{1}{(x+h)} - \frac{1}{x}) \div h       \\
		\frac{-h}{x^2+hx} \div h                                                    \\
		\frac{-h}{x^2+hx} \times \frac{1}{h}                                        \\
		-\frac{1}{x^2+hx}                                                           \\
		-\frac{1}{x^2+0x}                                                           \\
		 & \frac{df}{dx} = \lim_{h \to 0} \frac{f(x+h) -f(x)}{h} =   -\frac{1}{x^2} \\
	\end{align*}
}

\qs{}{Find the derivative of $f(x)=\sqrt{x}$}

\sol{
	\begin{align*}
		\frac{df}{dx} = \lim_{h \to 0} \frac{f(x+h) -f(x)}{h}                                                                   \\
		\frac{df}{dx} = \lim_{h \to 0}  = \frac{\sqrt{x+h} -\sqrt{x}}{h} \times \frac{\sqrt{x+h}+\sqrt{x}}{\sqrt{x+h}+\sqrt{x}} \\
		\frac{df}{dx} = \lim_{h \to 0} = \frac{(\sqrt{x+h}- \sqrt{x}) - (\sqrt{x+h}+ \sqrt{x})}{h(\sqrt{x+h}+ \sqrt{x})}        \\
		\frac{df}{dx} = \lim_{h \to 0} = \frac{x+h-x}{h(\sqrt{x+h}+ \sqrt{x})}                                                  \\
		\frac{df}{dx} = \lim_{h \to 0} = \frac{h}{h(\sqrt{x+h}+ \sqrt{x})}                                                      \\
		\frac{df}{dx} = \lim_{h \to 0} = \frac{1}{\sqrt{x+h} +\sqrt{x}}                                                         \\
		\frac{1}{\sqrt{x+0} +\sqrt{x}}                                                                                          \\
		\frac{1}{2\sqrt{x}}                                                                                                     \\
		 & \frac{df}{dx} = \lim_{h \to 0} \frac{f(x+h) -f(x)}{h} = \frac{1}{2\sqrt{x}}                                          \\
	\end{align*}
}

\section{Techniques of Differentiation}

\subsection{Power Rule}

\[
	y = x^{n}
\]
Where $n$ is a real number

\[
	\frac{dy}{dx} = n x^{n-1}
\]

\subsubsection{Examples}

\ex{$y = x^2, \text{ n = 2}$}{
	\begin{align*}
		\frac{dy}{dx} =2x^{2-1} \\
		 & = 2x^1               \\
		 & = 2x                 \\
	\end{align*}
}

\ex{$y = \frac{1}{x^2}$}{
	\begin{align*}
		y = x^{-2}, \text{ n = -2} \\
		 & y^{\prime} = -2x
	\end{align*}
}

\ex{$y = \sqrt{x}$}{
	\begin{align*}
		y = x^{\frac{1}{2}}                     \\
		y^{\prime} = \frac{x^{-\frac{1}{2}}}{2} \\
		y^{\prime} = \frac{1}{2\sqrt{x}}
	\end{align*}
}

\ex{$y = \frac{1}{\sqrt{x}} + x^3 -1$}{
	\begin{align*}
		y^{\prime} = x^{-\frac{1}{2}} +x^3 -1            \\
		y^{\prime} = -\frac{1}{2}x^{-\frac{3}{2}} + 3x^2 \\
	\end{align*}
}

\ex{$y = -x^{-8} + 3x^2$}{
\begin{align*}
	y^{\prime} = -8x^{-9} + 6x
\end{align*}
}

\section{Chain Rule}
\dfn{Chain Function}{A function that is composed of two or more functions, i.e, $f[g]$}

In the case where
\[
	y = (2x+1)^2
\]
The derivative is:
\[
	y^{\prime} = 4(2x+1)
\]

\subsubsection{Examples}
\ex{$y = (3x^2+2x)^5$}{
	\begin{align*}
		y = (5)(6x+2)(3x^2+2x)^4 \\
		y= (30x+10)(3x^2+2x)
	\end{align*}
}

\begin{align}
	\label{qs:1}y = 15x^9 - 3x^{12} +5x - 46                            \\
	\label{qs:2}y = 2t^6+7t^{-6}                                        \\
	\label{qs:3}y = 8x^3-\frac{1}{3x^5} + x- 23                         \\
	\label{qs:4}y = \sqrt{x} + 9\sqrt[3]{x^4} - \frac{2}{\sqrt[5]{x^2}} \\
	\label{qs:5}y = \sqrt[3]{x^2}(2x-x^2)                               \\
	\label{qs:6}y = \frac{2t^5+t^2-5}{t^2}                              \\
	\label{qs:7}y = 2x^3 + \frac{300}{x^3} + 4
\end{align}


\ex{\ref{qs:1}}{
	\[
		y^{\prime} = 185x^{8} - 36x^{11} +5
	\]
}

\ex{\ref{qs:2}}{
	\[
		y^{\prime} = 12t^{5} = 42t^{-7}
	\]
}

\ex{\ref{qs:3}}{
	\begin{align*}
		y = 8x^3 - \frac{1}{3x^{5}} + x -23  \\
		y = 8x^3 - \frac{1}{3}x^{-5} + x -23 \\
		y^{\prime} = 24x^2 + \frac{5}{3}x^{-6} + 1
	\end{align*}
}

\ex{\ref{qs:4}}{
	\begin{align*}
		y = \sqrt{x} + 9\sqrt[3]{7} - \frac{2}{\sqrt[5]{x^2}}                                     \\
		y = x^{\frac{1}{2}} + 9(x^{7})^{\frac{1}{3}} - 2(x^2)^{-\frac{1}{5}}                      \\
		y = x^{\frac{1}{2}} + 9x^{\frac{7}{3}} - 2x^{-\frac{2}{5}}                                \\
		y^{\prime} = \frac{1}{2}x^{-\frac{1}{2}} +21x^{\frac{4}{3}} + \frac{4}{5}x^{-\frac{7}{5}} \\
		y^{\prime} = \frac{1}{2\sqrt{x}} + 21x^{\frac{4}{3}} + \frac{4}{5}x^{-\frac{7}{5}}
	\end{align*}
}

\ex{\ref{qs:5}}{
	\begin{align*}
		y = (x^2)^{\frac{1}{3}}(2x-x^2)        \\
		y = x^{\frac{2}{3}}(2x-x^2)            \\
		y = 2x^{\frac{5}{3}} - x^{\frac{2}{3}} \\
		y^{\prime} = \frac{10}{3}x^{\large\frac{2}{3}} - \frac{8}{3}x^{\large\frac{5}{3}}
	\end{align*}
}

\ex{\ref{qs:6}}{
	\begin{align*}
		y = 2t^3 + 1 -\frac{5}{t^2} \\
		y = 2t^3 + 1 - 5t^{-2}      \\
		y^{\prime} = 6t^2 + 10t^{-3}
	\end{align*}
}

\ex{\ref{qs:7}}{
	\begin{align*}
		y = 2x^3 + 300x^{-3} + 4 \\
		y^{\prime} = 6x^2 - 900x^{-4}
	\end{align*}
}
\pagebreak

\section{Product Rule}
Given $y=u\times v$, then the derivative is given by:

\[
	\frac{dy}{dx} = u^{\prime} \times v + v^{\prime} \times u
\]

\subsubsection{Examples}

\ex{$y = (x^2+1)(x^3-x)$}{
	\begin{align*}
		y^{\prime} = (2x)(x^3-x) + (x^2+1)(3x^2-1) \\
		y^{\prime} = 2x^4 - 2x^2 + 3x^4 +2x^2-1    \\
		 & y^{\prime} = 5x^4 -1
	\end{align*}
}

\ex{$y=(6x^3-x)(10-20x)$}{
	\begin{align*}
		y^{\prime} = (18x^2-1)(10-20x) + (-20)(6x^3-x)      \\
		y^{\prime} = 180x^2 -10 -360x^3 + 20x  - 120x^3+20x \\
		y^{\prime} & = -480x^3+ 180x^2+ 40x - 10
	\end{align*}
}

\subsubsection{Questions}

\qs{}{$y = (4t^2-t)(t^3-8t^2+12)$}
\sol {
	\begin{align*}
		y^{\prime} = (8t- 1)(t^3-8t^2+12) + (3t^2-16t)(4t^2-t)                              \\
		y^{\prime} = 8t^{4} - 64t^3 + 96t - t^3 + 8t^2 -12 + 12t^{4} - 3t^3 - 64t^3 + 16t^2 \\
		y^{\prime} = 20t^{4} -132t^3 + 24t^2 +96t -12                                       \\
	\end{align*}
}

\qs{}{$y = (1+\sqrt{x^3})(x^{-3}-2\sqrt[3]{x})$}
\sol{
	\begin{align*}
		y = (1+(x^3)^{\frac{1}{2}})(x^{-3}-2\sqrt[3]{x})                                                                                        \\
		y = (1+x^{\frac{1}{2}})(x^{-3}-2(x^{\frac{1}{3}}))                                                                                      \\
		y^{\prime} = (\frac{3}{2}x^{\frac{1}{2}})(x^{-3}-2x^{\frac{1}{3}}) + (-3x^{4} - \frac{2}{3}x^{-\frac{2}{3}}-\frac{2}{3}x^{\frac{5}{6}}) \\y^{\prime} = \frac{3}{2}x^{-\frac{5}{2}}-3x^{\frac{5}{6}} - 3x^{-4}-3x^{-\frac{5}{2}} - \frac{2}{3}x^{-\frac{2}{3}} - \frac{2}{3}x^{\frac{5}{6}} \\
		y^{\prime} = -\frac{11}{3}x^{\frac{5}{6}} - \frac{3}{2}x^{-\frac{5}{2}} - \frac{2}{3}x^{-\frac{2}{3}} - 3x^{-4}
	\end{align*}
}

\qs{}{$y = (4-t^2)(1+5t^2)$}
\sol{
	\begin{align*}
		y^{\prime} = (-2t)(1+5t^2) + (10t)(4-t^2) \\
		y^{\prime} = -2t - 10t^3 + 40t - 10t^3    \\
		y^{\prime} = 20t^3 + 38t
	\end{align*}
}

\qs{}{$y = (x-\frac{2}{x^2})(7-2x^3)$}
\sol{
	\begin{align*}
		y = (x-2x^{-2})(7-2x^3)                                  \\
		y^{\prime} = (1+4x^{-3})(7-2x^3) + (-6x^2)(x-2x^{-2})    \\
		y^{\prime} = 7 - 2x^3 + 28x^{-3} - 8x^{0} - 6x^3 + 12x^0 \\
		y^{\prime} = 7 - 2x^3 + 28x^{-3} - 8 - 6x^3 + 12         \\
		 & y^{\prime} = -8x^3+\frac{28}{x^3}+11
	\end{align*}
}

\qs{}{$y = (3-x)(1-2x+x^2)$}
\sol{
	\begin{align*}
		y^{\prime} = (-1)(1-2x+x^2) + (2x-2)(3-x)   \\
		y^{\prime} = -1 + 2x -x^2 + 6x-2x^2 - 6 +2x \\
		y^{\prime} & = -7 + 10x - 3x^2
	\end{align*}
}
\pagebreak

\section{Quotient Rule}

Where $\large{y=\frac{f(x)}{g(x)}}$, the derivative is given by:

\begin{align*}
	\frac{dy}{dx} = \frac{g(x) \times f^{^{\prime}}(x) - f(x) \times g^{^{\prime}}(x)}{(g(x))^2}
\end{align*}

Or where \large{$\frac{u}{v}$}

\begin{align*}
	y^{\prime} = \frac{v \times u^{\prime} - u \times v^{\prime}}{v^2}
\end{align*}

\end{document}
