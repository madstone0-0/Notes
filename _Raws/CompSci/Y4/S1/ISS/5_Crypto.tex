\documentclass[12pt letter]{report}
\input{./template/preamble}
\input{./template/macros}
\input{./template/letterfonts}

\title{\Huge{Cryptography}}
\author{\huge{Madiba Hudson-Quansah}}
\date{}
\usepackage{parskip}

\setcounter{tocdepth}{4}
\setcounter{secnumdepth}{4}

\begin{document}
\maketitle
\newpage
\pdfbookmark[section]{\contentsname}{too}
\tableofcontents
\pagebreak

\chapter{Public Key Infrastructure}

\dfn{Public Key Infrastructure (PKI) }{
  The systems the brings cryptography into real world use. Uses
  certificates, trusted authorities and public key cryptography to
  securely identify entities on a network.
}

PKI is the integrated system of software, encryption methodologies,
protocols, legal agreements, and third-party services enabling users
to communicate securely over insecure networks like the internet
using public key cryptography.

PKI solves the problem of \textbf{trust} in public key cryptography by
using a \textbf{trusted third party} called a \textbf{Certificate
Authority (CA)} to verify the identity of entities and issue digital
certificates that bind public keys to those identities.

PKI protections information assets in all forms:
\begin{description}
  \item[Authentication]  Validate
  \item[Integrity]
  \item[Confidentiality]
  \item[Authorization]
  \item[Non-Repudation]
\end{description}

\section{Components of PKI}
\begin{description}
  \item[Certificate Authority (CA)]:
    \begin{itemize}
      \item Trusted third party that issues digital certificates
    \end{itemize}
  \item[Registration Authority (RA)]
    \begin{itemize}
      \item Handles certification functions in collaboration with a
        CA, verifying registration information, generating end-user
        keys, and validating and revoking user certificates.
    \end{itemize}
  \item[Digital Certificate (X.509)]
    \begin{itemize}
      \item Owner's public key
      \item Owner's identify information
      \item Digital signature of the CA
      \item Validity period
      \item Certificate serial number
    \end{itemize}
  \item[Certificate Store/Repository ] - Central location where
    certificates and CRLS (Certificate Revocation Lists) are stored and managed.
  \item[Certificate Revocation List (CRL)] - A list of certificates
    that have been revoked before their expiration date.
  \item[Management Protocols]
    \begin{itemize}
      \item Manages the Lifecycle of keys and certificates
      \item Organizes and manages communicators among CAs, RAs, and
        end users, while ensuring secure transmission of certificates
        and related information.
    \end{itemize}
  \item[Policies and Procedures]
    \begin{itemize}
      \item Assists an organization in the application and management
        of certificates in the formalization of legal liabilities and
        limitations and in business use.
    \end{itemize}
\end{description}

\end{document}
