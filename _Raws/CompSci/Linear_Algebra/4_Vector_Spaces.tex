\documentclass[12pt letter]{report}
\input{./template/preamble}
\input{./template/macros}
\input{./template/letterfonts}

\title{\Huge{Vector Spaces}}
\author{\huge{Madiba Hudson-Quansah}}
\date{}
\usepackage{parskip}

\setcounter{tocdepth}{4}
\setcounter{secnumdepth}{4}

\begin{document}
\maketitle
\newpage
\pdfbookmark[section]{\contentsname}{too}
\tableofcontents
\pagebreak

\chapter{Vector Spaces and Subspaces}

\section{Introduction}

\dfn{Vector Space}{
  A \textit{vector space} is a non empty set $V$ of objects, called vectors, on which are defined two operations,
  addition and multiplication by scalars, e.g. real numbers, subject to the following axioms which must hold for all
  vectors $\mbold{u}, \mbold{v}$ and $\mbold{w}$ in $V$ and for all scalars $c$ and $d$.
  \begin{enumerate}
    \item The sum of $\mbold{u}$ and $\mbold{v}$, denoted by $\mbold{u} + \mbold{v}$, is in $V$. \label{ax:1}
    \item $\mbold{u} + \mbold{v} = \mbold{v} + \mbold{u}$ \label{ax:2}
    \item $ \left( \mbold{u} + \mbold{v} \right) + \mbold{w} = \mbold{u} + \left( \mbold{v} + \mbold{w} \right)  $
          \label{ax:3}
    \item There is a zero vector $\mbold{0}$ in $V$ such that $\mbold{u} + \mbold{0} = \mbold{u}$ \label{ax:4}
    \item For each $\mbold{u}$ in $V$, there is a vector $-\mbold{u}$ in $V$ such that $\mbold{u} + \left( -
            \mbold{u} \right) = \mbold{0} $ \label{ax:5}
    \item The scalar multiple of $\mbold{u}$ by $c$, denoted by $c\mbold{u}$, is in $V$. \label{ax:6}
    \item $c\left( \mbold{u} + \mbold{v} \right) = c\mbold{u} + c\mbold{v}$ \label{ax:7}
    \item $\left( c + d \right)\mbold{u} = c\mbold{u} + d\mbold{u}$ \label{ax:8}
    \item $c\left( d\mbold{u} \right) = \left( cd \right)\mbold{u}$ \label{ax:9}
    \item $1\mbold{u} = \mbold{u}$ \label{ax:10}
  \end{enumerate}
}

Using these axioms one can show that the zero vector in axiom \ref{ax:4} is unique, and the vector $-\mbold{u}$, called
the \textbf{negative} of $\mbold{u}$ in axiom \ref{ax:5} is unique for each $\mbold{u}$ in $V$, outlined in:

\thm{}{
  \begin{align}
    0\mbold{u}  & = \mbold{0}                    \\
    c \mbold{0} & = \mbold{0}                    \\
    -\mbold{u}  & = \left( -1 \right)  \mbold{u}
  \end{align}
}

\section{Subspaces}

In many problems, a vector space consists of an appropriate set of vectors from a larger vector space. In this case
only, three of the ten axioms need to be checked to determine if the subset is a vector space, the rest are satisfied
automatically.

\dfn{Subspace}{
  A subset $H$ of the vector space $V$, where:
  \begin{enumerate}
    \item The zero vector of $V$ is in $H$.
    \item $H$ is closed under vector addition. That is for each $\mbold{u}$ and $\mbold{v}$ in $H$, the sum of
          $\mbold{u} + \mbold{v}$ is in $H$.
    \item $H$ is closed under scalar multiplication. That is for each $\mbold{u}$ in $H$ and each scalar $c$, the
          scalar multiple $c\mbold{u}$ is in $H$.
  \end{enumerate}
}

These properties guarantee that a subspace $H$ of $V$ is also a vector space, under the defined vector space operations.
This means that every subspace is a vector space and conversely every vector space is a subspace (of itself and possibly
of a larger vector space).

\ex{}{
  \qs{}{
    The vector space $\mathbb{R}^{2}$ is not a vector space of $\mathbb{R}^{3}$ because $\mathbb{R}^{2}$ is not even a
    subset of $\mathbb{R}^{3}$. The set
    \[
      H = \left\{\begin{bmatrix} s \\ t \\ 0 \end{bmatrix} : s \text{ and $t$ are real} \right\}
    \]
    is a subset of $\mathbb{R}^{3}$ that "looks" and "acts" like $\mathbb{R}^{2}$ even though it is logically distinct
    from $\mathbb{R}^{2}$. Show that $H$ is subset of $\mathbb{R}^{3}$
  }

  \sol{
    \begin{itemize}
      \item The zero vector is in $H$
      \item $H$ is closed under vector addition and scalar multiplication as these operations on vectors in $H$ always
            produce vectors whose third entry is zero and thus belong to $H$.
    \end{itemize}
    Thus $H$ is as subspace of $\mathbb{R}^3$
  }
}

\subsection{Subspace Spanned by a Set}

One way of describing a subspace is as a linear combination of vectors that span the subspace.

\ex{}{
  \qs{}{
    Given $\mbold{v}_1$ and $\mbold{v}_2$ in a vector space $V$, let $H = \text{Span}\{\mbold{v}_1, \mbold{v}_2\} $. Show
    that $H$ is a subspace of $V$
  }

  \sol{
    \begin{itemize}
      \item The zero vector is in $H$ as:
            \[
              \mbold{0} = 0\mbold{v}_1 + 0\mbold{v}_2
            \]
      \item To show that $H$ is closed under vector addition and scalar multiplication, take two arbitrary vectors in
            $H$, say
            \[
              \mbold{u} = s_1 \mbold{v}_1 + s_2 \mbold{v}_2 \text{ and } \mbold{w} = t_1 \mbold{v}_1 + t_2 \mbold{v}_2
            \]
            By axioms \ref{ax:2}, \ref{ax:3}, and \ref{ax:8} for the vector space $V$:
            \begin{align*}
              \mbold{u} + \mbold{w} & = \left( s_1 \mbold{v}_1 + s_2 \mbold{v}_2 \right) + \left( t_1 \mbold{v}_1 + t_2 \mbold{v}_2 \right) \\
                                    & = \left( s_1 + t_1 \right)\mbold{v}_1 + \left( s_2 + t_2 \right) \mbold{v}_2                          \\
            \end{align*}
            The result is still in $H$ as it can still be spanned from $\text{Span}\{\mbold{v}_1, \mbold{v}_2\} $, with
            weights $ \left( s_1 + t_1 \right) $ and $ \left( s_2 + t_2 \right) $

            Furthermore:
            \begin{align*}
              c \mbold{u} & = c \left( s_1 \mbold{v}_1 + s_2 \mbold{v}_2  \right)                 \\
                          & = \left( c s_1 \right) \mbold{v}_1 + \left( c s_2 \right) \mbold{v}_2 \\
            \end{align*}
            therefore $H$ is also closed under scalar multiplication.

    \end{itemize}
  }
}

\thm{}{ \label{thm:1}
  If $\mbold{v}_1, \ldots, \mbold{v}_p$ are in vector space $V$, then $\text{Span}\{\mbold{v}_1, \ldots, \mbold{v}_p\}
  $ is a subspace of $V$.
}

We can call $\text{Span}\{\mbold{v}_1, \ldots, \mbold{v}_p\} $ \textbf{the subspace spanned} by $\{\mbold{v}_1, \ldots,
  \mbold{v}_p\} $. Therefore given any subspace $H$ of $V$, a \textbf{spanning set} for $H$ is a set $\{\mbold{v}_1,
  \ldots, \mbold{v}_p\} $ in $H$ such that $H = \text{Span}\{\mbold{v}_1, \ldots, \mbold{v}_p\} $.

\ex{}{
  \qs{}{
    Let $H$ be the set of all vectors of the form $ \left( a - 3b, b - a, a, b \right) $, where $a$ and $b$ are
    arbitrary scalars. That is let $H = \{ \left( a - 3b, b-a, a, b \right) : a \text{ and $b$ in } \mathbb{R} \} $.
    Show that $H$ is a subspace of $\mathbb{R}^{4}$
  }

  \sol{
    \begin{align*}
      H & = \begin{bmatrix}
              a - 3b \\
              b - a  \\
              a      \\
              b
            \end{bmatrix}                    \\
        & = a\begin{bmatrix}
               1  \\
               -1 \\
               1  \\
               0
             \end{bmatrix} + b \begin{bmatrix}
                                 -3 \\
                                 1  \\
                                 0  \\
                                 1
                               \end{bmatrix} \\
        & = a \mbold{v}_1 + b \mbold{v}_2     \\
    \end{align*}
    Thus $H$ is a subspace of $\mathbb{R}^4$ by theorem \ref{thm:1}
  }
}

\ex{}{
  \qs{}{
    For what value(s) of $h$ will $\mbold{y}$ be in the subspace of $\mathbb{R}^3$ spanned by $\mbold{v}_1, \mbold{v}_2,
      \mbold{v}_3$, if
    \[
      \mbold{v}_1 = \begin{bmatrix} 1 \\ -1\\ 2 \end{bmatrix},\, \mbold{v}_2 = \begin{bmatrix} 5 \\ -4 \\ -7
      \end{bmatrix},\, \mbold{v}_3 = \begin{bmatrix} -3 \\ 1 \\ 0 \end{bmatrix},\,  \text{ and } \mbold{y} = \begin{bmatrix} -4\\ 3 \\ h \end{bmatrix}
    \]
  }

  \sol{
    The subspace of $\mathbb{R}^3$ Span$\{\mbold{v}_1, \mbold{v}_2, \mbold{v}_3\} $. $\mbold{y}$ will be in the subspace
    if the span of $\mbold{v}_1, \mbold{v}_2, \mbold{v}_3$ contains $\mbold{y}$, that is if $\mbold{y}$ can be written
    \[
      x_1 \mbold{v}_1 + x_2 \mbold{v}_2 + x_3 \mbold{v}_3 = \mbold{y}
    \]
    And thus the matrix equation:
    \[
      A \mbold{x} = \mbold{y}
    \]
    Where $A = \begin{bmatrix}
        1  & 5  & -3 \\
        -1 & -4 & 1  \\
        -2 & -7 & 0
      \end{bmatrix} $ and $\mbold{x} = \begin{bmatrix} x_1 \\ x_2 \\ x_3 \end{bmatrix} $

    \begin{align*}
      \begin{bmatrix}
        1  & 5  & -3 & -4 \\
        -1 & -4 & 1  & 3  \\
        -2 & -7 & 0  & h
      \end{bmatrix}    \\
      -R_1 - R_2 \to R_2   \\
      \begin{bmatrix}
        1  & 5  & -3 & -4 \\
        0  & 1  & -2 & -1 \\
        -2 & -7 & 0  & h
      \end{bmatrix}    \\
      -2 R_2 + R_3 \to R_3 \\
      \begin{bmatrix}
        1 & 5  & -3 & -4    \\
        0 & 1  & -2 & -1    \\
        0 & -3 & 6  & 8 - h \\
      \end{bmatrix}  \\
      -3R_2 - R_3 \to R_3  \\
      \begin{bmatrix}
        1 & 5 & -3 & -4     \\
        0 & 1 & -2 & -1     \\
        0 & 0 & 0  & -5 + h
      \end{bmatrix}
    \end{align*}
    $\therefore$ The system $A \mbold{x} = \mbold{y}$ is only consistent if $h = 5$, and thus $\mbold{y}$ is in the
    subspace spanned by Span$\{\mbold{v}_1, \mbold{v}_2, \mbold{v}_3\} $ if and only if $h = 5$
  }
}

\section{Exercises}

\qs{}{
  Show that the set $H$ of all points in $\mathbb{R}^2$ of the form $ \left( 3s, 2 + 5s \right) $ is not a vector space,
  by showing that it is not closed under scalar multiplication. (Find a specific vector $\mbold{u}$ in $H$ and a scalar
  $c$ such that $c \mbold{u}$ is not in $H$)
}

\sol{
  Let $\mbold{u} = \begin{bmatrix} 3 \\ 7 \end{bmatrix} $ and $c = 2$. Then:
  \begin{align*}
    2 \mbold{u} & = 2 \left( \begin{bmatrix} 3 \\ 7 \end{bmatrix}  \right) \\
                & = \begin{bmatrix} 6 \\ 14 \end{bmatrix}                  \\
  \end{align*}
  This implies there is some $s$ such that $\begin{bmatrix} 3s \\ 2 + 5s \end{bmatrix} = \begin{bmatrix} 6 \\ 14
    \end{bmatrix}  $, but for this to be true $s$ would need to be equal to $2$ and $2.4$ which is impossible. Therefore $H$
  is not closed under scalar multiplication and thus is not a vector space.
}

\qs{}{
  Let $W = \text{Span} \{\mbold{v}_1,\ldots, \mbold{v}_p\} $, where $\mbold{v}_1,\ldots, \mbold{v}_p$ are in a vector
  space $V$. Show that $\mbold{v}_k$ is in $W$ for $1 \leq k \leq p$.
}

\sol{
  If $W = \text{Span} \{\mbold{v}_1,\ldots, \mbold{v}_p\} $, then the contents of $W$ for example $\mbold{v}_1$ can be
  written as linear combination of the spanned vectors, that is:
  \begin{align*}
    \mbold{v}_1 = 1 \mbold{v}_1 + 0 \mbold{v}_2 + \ldots + 0 \mbold{v}_p
  \end{align*}
  Therefore if $1 \leq k \leq p$, then $\mbold{v}_k$ is in $W$ because:
  \[
    \mbold{v}_k = 0 \mbold{v}_1 + \ldots + 0\mbold{v}_{k-1} + 1\mbold{v}_k + 0\mbold{v}_{k+1} + \ldots + 0 \mbold{v}_p
  \]
}

\qs{}{
  An $n\times n$ matrix $A$ is said to be \textit{symmetric} if $A = A^{T}$. Let $S$ be the set of all $3\times 3$
  symmetric matrices. Show that $S$ is a subspace of $M_{3\times 3}$, the vector space of all $3\times 3$ matrices.
}

\sol{
  To prove that $S$ is a subspace of $M_{3\times 3}$, I must show:
  \begin{description}
    \item[The Zero vector Is in $S$] Since the zero vector is symmetric $S$ contains the zero vector as:
          \[
            \mbold{0} = \mbold{0}^{T}
          \]
    \item[$S$ is closed under vector addition] Let $A$ and $B$ be in $S$, hence $A = A^{T}$ and $B = B^{T}$
          \begin{align*}
            \left( A + B \right)^{T} & = A^{T} + B^{T} \\
                                     & = A + B         \\
          \end{align*}
          Thus $A + B$ is symmetric and is in $S$
    \item[$S$ is closed under scalar multiplication] Let $A$ be in $S$ and $c$ be a scalar
          \begin{align*}
            \left( cA \right)^{T} & = c \left( A \right)^{T} \\
                                  & = cA                     \\
          \end{align*}
          Thus $cA$ is symmetric and is in $S$
  \end{description}
  $\therefore$ $S$ is a subspace of $M_{3\times 3}$
}

\qs{}{
  Let $V$ be the first quadrant in the $xy$-plane; that is, let
  \[
    V = \left\{ \begin{bmatrix} x \\ y \end{bmatrix} : x \geq 0 \text{ and } y \geq 0 \right\}
  \]
  \begin{enumerate}
    \item If $\mbold{u}$ and $\mbold{v}$ are in $V$, is $\mbold{u} + \mbold{v}$ in $V$? Why?
    \item Find a specific vector $\mbold{u}$ in $V$ and specific scalar $c$ such that $c\mbold{u}$ is not in $V$.
  \end{enumerate}
}

\sol{
  \begin{enumerate}
    \item If $\mbold{u}$ and $\mbold{v}$ are $V$, then indeed $\mbold{u} + \mbold{v}$ are in $V$, because the sum of
          these two vectors will always have positive $x$ and $y$ components and will therefore always be in the first
          quadrant of the $xy$-plane.
    \item For $\mbold{u} = \begin{bmatrix} 3 \\ 4 \end{bmatrix} $ and $c = -2$
          \begin{align*}
            -2 \mbold{u} & = -2 \left( \begin{bmatrix} 3\\ 4  \end{bmatrix}  \right) \\
                         & = \begin{bmatrix} -6 \\ -8 \end{bmatrix}                  \\
          \end{align*}
  \end{enumerate}
}

\qs{}{
  Determine if the given sets are subspaces of $\mathbb{P}_n$ for an appropriate value of $n$. Justify your answers.
  \begin{enumerate}
    \item All polynomials in the form $\mbold{p} \left( t \right) = at^2 $, where $a \in \mathbb{R}$.
    \item All polynomials in the form $\mbold{p} \left( t \right) = a + t^2 $, where $a \in \mathbb{R}$
    \item All polynomials of degree at most $3$, with integers as coefficients.
    \item All polynomials in $\mathbb{P}_n$ such that $\mbold{p} \left( 0 \right) = 0 $
  \end{enumerate}
}

\sol{
  \begin{enumerate}
    \item Yes this is a subspace of $\mathbb{P}_n$ as:
          \begin{description}
            \item[Contains the zero vector] When $a = 0$, $\mbold{p} \left( t \right) = 0t^2 = 0$.
            \item[Closed under vector additon] Let $\mbold{w}$ and $\mbold{q}$ be polynomials in the appropriate form
                  \begin{align*}
                    \mbold{w} + \mbold{q} & = \left( wt^2 \right) + \left( qt^2 \right) \\
                                          & = \left( w + q \right)t^2                   \\
                    \text{Let } w + q     & = a \text{, then}                           \\
                                          & = a t^2                                     \\
                  \end{align*}
            \item[Closed under scalar multiplication] Let $\mbold{w}$ be a polynomial in the appropriate form and
                  $c$ be a scalar.
                  \begin{align*}
                    c \mbold{w}     & =  c \left( w t^2 \right) \\
                                    & = \left( c w \right) t^2  \\
                    \text{Let } c w & = a \text{, then}         \\
                                    & = a t^2                   \\
                  \end{align*}

          \end{description}
  \end{enumerate}
}

\chapter{Null Space, Column Space, and Linear Transformations}

\section{The Null Space of a Matrix}

\dfn{Null Space}{
  The \textit{null space} of an $m \times n$ matrix $A$, denoted by $\text{Nul } A$, is the set of all solutions of
  the homogeneous equation $A \mbold{x} = \mbold{0}$. In set notation:
  \[
    \text{Nul }A = \{\mbold{x} : \mbold{x} \text{ is in $\mathbb{R}^{n}$ and } A\mbold{x} = \mbold{0}\}
  \]
}

\ex{}{
  \qs{}{
    Let $A$ be the matrix $\begin{bmatrix}
        1  & -3 & -2 \\
        -5 & 9  & 1
      \end{bmatrix} $, and let $\mbold{u} = \begin{bmatrix} 5 \\ 3 \\ -2 \end{bmatrix} $. Determine if $\mbold{u}$ belongs
    to the null space of $A$.
  }

  \sol{
    This is basically asking us to verify if $\mbold{u}$ satisfies the equation $A \mbold{u} = \mbold{0}$

    \begin{align*}
      \begin{bmatrix}
        1  & -3 & -2 \\
        -5 & 9  & 1
      \end{bmatrix}       \begin{bmatrix} 5 \\ 3\\ -2 \end{bmatrix} & = \begin{bmatrix} 0 \\  0  \end{bmatrix} \\
      \begin{bmatrix}
        5  -9 + 4    \\
        -25 + 27  -2 \\
      \end{bmatrix}                                               & = \begin{bmatrix} 0 \\ 0 \end{bmatrix}     \\
    \end{align*}
    $\therefore$ $\mbold{u}$ is in the null space of $A$.
  }
}

\thm{}{
  The null space of an $m\times n$ matrix $A$ is a subspace of $\mathbb{R}^{n}$, equivalently, the set of all
  solutions to a system $A\mbold{x} = \mbold{0} $ of $m$ homogeneous linear equations in $n$ unknowns is a subspace
  of $\mathbb{R}^{n}$.
}

\subsection{An Explicit Description of the Null Space of a Matrix}

There is no obvious relation between the vectors in $\text{Nul } A$ and the entries $A$. We say that $\text{Nul } A$ is
defined implicitly, as it is defined by a condition that must be checked. However solving the equation $A \mbold{x} =
  \mbold{0}$ amounts to producing an explicit description of $\text{Nul } A$.

\ex{}{
  \qs{}{
    Find a spanning set for the null space of the matrix
    \[
      A = \begin{bmatrix} -3 & 6 & -1 & 1 & -7 \\ 1 & -2 & 2 & 3 & -1 \\ 2 & -4 & 5 & 8 & -4 \end{bmatrix}
    \]
  }

  \sol{
    The first step is to find the general solution of $A \mbold{x} = \mbold{0}$ in terms of free variables. Therefore:
    \begin{align*}

      A = \begin{bmatrix} -3 & 6 & -1 & 1 & -7 \\ 1 & -2 & 2 & 3 & -1 \\ 2 & -4 & 5 & 8 & -4 \end{bmatrix} \\
    \end{align*}
  }
}


\end{document}
