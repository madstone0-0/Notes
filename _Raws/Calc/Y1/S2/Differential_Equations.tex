\documentclass[12pt letter]{report}
\input{./template/preamble}
\input{./template/macros}
\input{./template/letterfonts}

\title{\Huge{Differential Equations}}
\author{\huge{Madiba Hudson-Quansah}}
\date{\today}
\usepackage{parskip}

\begin{document}
\maketitle
\newpage
\pdfbookmark[section]{\contentsname}{too}
\tableofcontents
\pagebreak

\chapter{Introduction}

\dfn{Differential Equation}{
	A differential equation is an equation in the form
	\[
		y = f(x)
	\]

	Where $f(x)$ is an unknown function and including one or more of it's derivatives, i.e. $f^{\prime}(x)$.

	A solution to such an equation is a function $f(x)$ that satisfies the differential equation when $f$ and it's
	derivatives are substituted into the equation.

}

\qs{}{
	Verify that the function $y = e ^{-3x} + 2x + 3$ is a solution to the differential equation $y^{\prime} + 3y = 6x +
		11$
}

\sol{
	\begin{align*}
		y = e ^{-3x} + 2x + 3                                                      \\
		y^{\prime} = -3e ^{-3x} + 2                                                \\
		\\
		\text{Let } y^{\prime} = -3e ^{-3x} + 2 \text{ and } y = e ^{-3x} + 2x + 3 \\
		(-3e^{-3x} + 2) + 3(e ^{-3x} + 2x + 3) = 6x + 11                           \\
		-3e ^{-3x} + 2 + 3e ^{-3x} + 6x + 9 = 6x + 11                              \\
		6x + 11 = 6x + 11                                                          \\
		\text{Lhs } = \text{Rhs  } \therefore \text{ y is a solution}
	\end{align*}
}

\pagebreak
\qs{}{
	Verify that $y = 2e ^{3x} - 2x -2$ is a solution to the differential equation $y^{\prime} - 3y = 6x+ 4$
}

\sol{
	\begin{align*}
		y = 2e ^{3x} - 2x -2                                                  \\
		y^{\prime} = 6e ^{3x}-2                                               \\
		\\
		\text{Let } y^{\prime} = 6e ^{3x}-2 \text{ and } y = 2e ^{3x} - 2x -2 \\
		(6e ^{3x}-2) - 3(2e ^{3x} - 2x -2) = 6x+ 4                            \\
		6e ^{3x} - 2 -6e ^{3x} + 6x + 6 = 6x + 4                              \\
		7x + 4 = 6x + 4                                                       \\
		\text{Lhs } = \text{Rhs  } \therefore \text{ y is a solution}
	\end{align*}
}

\section{Order of a differential equation}

\dfn{Order of a differential equation}{
The order of a differential equation is the highest order of any derivate of the unknown function that appears in
the equation. I.e. the order of the differential equation:
\[
	x^2y^{\prime\prime\prime} - 3xy^{\prime\prime} + xy^{\prime} - 3y = \sin(x)
\]
is 3 because the highest order of any derivate of $y$ is $y^{\prime\prime\prime}$
}

\chapter{General and Particular Solutions}

\dfn{General Solution}{
	The general solution of a differential equation is a solution that contains an arbitrary constant. I.e. the general
	solution of the differential equation
	\[
		y^{\prime} = 2x
	\]
	is
	\[
		y = x^2 + c
	\]
	where $c$ is an arbitrary constant.
}

\dfn{Particular Solution}{
	A particular solution of a differential equation is a solution that does not contain an arbitrary constant. I.e. the
	particular solution of the differential equation
	\[
		y^{\prime} = 2x
	\]
	is
	\[
		y = x^2 + 1
	\]
}

\qs{}{
	Find the particular and general solution to the differential equation $y^{\prime} = 2x$ passing through the point
	$\left( 2,7 \right) $
}

\sol{
	\begin{align*}
		y^{\prime} = 2x                           \\
		\frac{dy}{dx} = 2x                        \\
		\mathrm{d}y = 2x \enspace \mathrm{d}x     \\
		\int 1  \mathrm{d}y = \int 2x \mathrm{d}x \\
		y = x^2 + c                               \\
		\text{General equation } y = x^2 + c      \\
		\\
		\left( 2, 7 \right)                       \\
		7 = (2)^2 + c                             \\
		c = 3                                     \\
		\text{Particular equation } y = x^2 + 3
	\end{align*}
}

\section{Initial Value Problems}

\dfn{Initial Value Problem}{
	An initial value problem is a differential equation with an initial condition. I.e. the initial value problem
	\[
		y^{\prime} = 2x \text{ with } y(2) = 7
	\]
}

\qs{}{
	Verify that the function $y = 2e^{-2t} + e^{t}$ is a solution to the initial-value problem
	\[
		y^{\prime} + 2y = 3e^{t} \text{ with } y(0) = 3
	\]
}

\sol{
	\begin{align*}
		y = 2e^{-2t} + e^{t}                                                         \\
		y^{\prime} = -4e^{-2t} + e^{t}                                               \\
		\\
		\text{Let } y^{\prime} = -4e^{-2t} + e^{t} \text{ and } y = 2e^{-2t} + e^{t} \\
		(-4e^{-2t} + e^{t}) + 2(2e^{-2t} + e^{t}) = 3e^{t}                           \\
		-4e^{-2t} + e^{t} + 4e^{-2t} + 2e^{t} = 3e^{t}                               \\
		3e^{t} = 3e^{t}                                                              \\
		\text{Lhs } = \text{Rhs  } \therefore \text{ y is a solution}                \\
		\\
		y(0) = 2e^{0} + e^{0}                                                        \\
		y(0) = 3                                                                     \\
		\therefore \text{ y is a solution to the initial value problem}
	\end{align*}
}

\pagebreak
\qs{}{
Verify that $y=3e^{2t} + 4\sin (t)$ is a solution to the initial-value problem
\[
	y^{\prime} - 2y = 4\cos (t) - 8\sin (t) \text{ with } y(0) = 3
\]
}

\sol{
	\begin{align*}
		y = 3e^{2t} + 4 \sin(t)                                                \\
		y^{\prime} = 6e^{2t} + 4\cos(t)                                        \\
		\\
		(6e^{2t} + 4 \cos(t)) - 2(3e^{2t} + 4 \sin(t)) = 4 \cos(t) - 8 \sin(t) \\
		6e^{2t} + 4 \cos(t) -6e^{2t}- 8 \sin(t) = 4 \cos(t) - 8 \sin(t)        \\
		4 \cos(t) - 8 \sin(t) = 4 \cos(t) - 8 \sin(t)                          \\
		\text{Lhs } = \text{Rhs  } \therefore \text{ y is a solution}          \\
		\\
		y(0) = 3e^{0} + 4 \sin(0)                                              \\
		y(0) = 3 + 0                                                           \\
		y(0) = 3                                                               \\
		\therefore \text{ y is a solution to the initial value problem}
	\end{align*}
}

\qs{}{
	Solve the following initial-value problem
	\[
		y^{\prime} = x^2 - 4x + 3 - 6e^{x} \text{ with } y(0) = 8
	\]
}

\sol{
	\begin{align*}
		y^{\prime} = x^2 - 4x + 3 - 6e^{x}                                              \\
		\frac{dy}{dx} = x^2 - 4x + 3 - 6e^{x}                                           \\
		1 \enspace \mathrm{d}y = (x^2 - 4x + 3 - 6e^{x}) \enspace \mathrm{d}x           \\
		\int 1 \enspace \mathrm{d}y = \int (x^2 - 4x + 3 - 6e^{x}) \enspace \mathrm{d}x \\
		y = \frac{1}{3}x^3 - 2x^2 + 3x - 6e^{x} + c                                     \\
		y(0) = \frac{1}{3}(0)^3 - 2(0)^2 + 3(0) - 6e^{0} + c                            \\
		8 = -6 + c                                                                      \\
		c = 14                                                                          \\
		 & y = \frac{1}{3}x^3 - 2x^2 + 3x - 6e^{x} + 14
	\end{align*}
}

\chapter{Application}

\section{Physics}

In Physics we use the knowledge that the forces acting on an object may result in motion and Newton's second law of
motion $F = ma$, where $F$ represents force, $m$ represents mass, and $a$ represents acceleration, to derive an equation
that can be solved to find the velocity of an object at a given time.

For instance if we have an object with mass $m$ falling or rising from/to a height, the acceleration due to gravity will be approx.
$g= 9.8 m/ s^2$. Then representing the velocity of the object as $v(t)$, we can represent the object falling as $v(t) <
	0$ and rising as $v(t) > 0$.

We can then setup an initial-value problem to find the velocity $v(t)$ at any time $t$.Therefore using Newton's second
law of motion $F = ma$ we can represent acceleration $a$ as the derivate of the object's velocity at a given time
$v^{\prime}(t)$ giving us
\[
	F = mv^{\prime}(t)
\]
However this force $F$ is the force of gravity acting on the object, therefore again using Newton's second law, we can
represent this force as
\[
	F_g = -mg
\]
$-$ negative since the force of gravity always works downwards. Therefore we obtain the equation

\[
	F = F_g
\]
Which then becomes


\begin{align*}
	mv^{\prime}(t) = -mg  \\
	 & v^{\prime}(t) = -g \\
\end{align*}

With the initial-value being the initial velocity i.e. when $t = 0$ giving us the complete initial-value problem.

\[
	v^{\prime}(t) = -g \text{ with } v(0) = v_0
\]


\pagebreak
\qs{}{
	Suppose a rock falls from rest from a height of 100 meters and the only force acting on it is gravity. Find a
	equation for the velocity $v(t)$ as a function of time, measured in meters per second.
}

\sol{
	\begin{align*}
		v^{\prime}(t) = -g \text{ with } v(0) = v_0 \\
		\int v^{\prime}(t) \enspace  \mathrm{d}t = \int -9.8 \enspace  \mathrm{d}t
		v(t) = -9.8t + c                            \\
		\\
		v(0) = -9.8(0) + c                          \\
		v(0) = c                                    \\
		c = 0                                       \\
		 & v(t) = -9.8t
	\end{align*}
}

Another question to ask is how high the object will be above the earth's surface at a given point in time. Let $s(t)$
represent the height above the Earth's  surface of the object. Then using the knowledge that $s(t) = \int v(t)$ /
$s^{\prime}(t) = v(t)$ we can generate an initial-value problem

\[
	s^{\prime}(t) = v(t) \text{ with } s(0) = s_0
\]

\qs{}{
	A baseball is thrown upward from a height of 3 meters above the Earth's surface with an initial velocity of 10 $m/s$, and the only force acting on it is gravity. The ball has a mass of 0.15 $kg$.
	\begin{enumerate}
		\item Find the position $s(t)$ of the baseball at time $t$.
		\item What is its height after 2 seconds?
	\end{enumerate}
}

\sol {
	\begin{enumerate}
		\item
		      \begin{align*}
			      s^{\prime}(t) = v(t)                                \\
			      \text{Let } v(t) = -9.8t + 10 \text{ and } s(0) = 3 \\
			      \int s^{\prime}() = \int (-9.8t + 10)               \\
			      s(t) = -4.9t^2 + 10t + c                            \\
			      \\
			      s(0) = -4.9(0)^2 + 10(0) + c                        \\
			      3 = c
			      s(t) = -4.9t^2 + 10t + 3
		      \end{align*}
		\item
		      \begin{align*}
			      t = 2
			      s(2) = -4.9(2)^2 + 10(2) + 3 \\
			      s(2) = -19.6 + 20 + 3        \\
			      s(2) = 3.4 m
		      \end{align*}
	\end{enumerate}
}




\end{document}
