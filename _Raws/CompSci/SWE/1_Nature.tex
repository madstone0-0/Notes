\documentclass[12pt letter]{report}
\input{./template/preamble}
\input{./template/macros}
\input{./template/letterfonts}

\title{\Huge{Software and the Nature of Software}}
\author{\huge{Madiba Hudson-Quansah}}
\date{}
\usepackage{parskip}

\setcounter{tocdepth}{4}
\setcounter{secnumdepth}{4}

\begin{document}
\maketitle
\newpage
\pdfbookmark[section]{\contentsname}{too}
\tableofcontents
\pagebreak

\chapter{The Nature of Software}

\section{Defining Software}

\dfn{Software}{
  Software is instructions, that when executed provide desired features, function and performance; data structures
  that enable the programs to adequately manipulate information in both hard copy and virtual forms that describes the
  operation and use of the programs.
}

The characteristics of software:
\begin{description}
  \item[Software is developed or engineered; it is not manufactured in the classical sense] - There may exist
        similarities between the development and manufacture of software and hardware respectively, like quality through
        design, but the manufacturing phase for hardware can introduce quality problems that are non-existent in software
        contexts.
  \item[Software doesn't "wear out"] - Usually hardware exhibits relatively high failure rates early in it's life, and
        corrected and the failure rate drops to a steady-state level for some period of time. Software does not behave
        the same, with decreasing failure rate with spikes when when defects are discovered as changes are made.
  \item[Although the industry is moving towards component-based construction, most software continues to be custom built] -
        As the field matures more and more best practices are codified and a set of standard design
        components are established. This allows the engineer to focus on the truly innovative elements of a design.
\end{description}

\section{Software Application Domains}
\begin{description}
  \item[System Software]  - A collection of programs written to service other programs. Characterized by heavy
        interaction with computer hardware, heavy usage by multiple users, concurrent operations that require scheduling,
        complex data structures, and multiple external interfaces.
  \item[Application Software] - Stand-alone programs that solve a specific business need. Process business or technical
        data in a way that facilities business operations or management/technical decision making.
  \item[Engineering / Scientific Software] - Characterized by number crunching algorithms, which application areas
        ranging from astronomy to volcanology, from automotive stress analysis to space shuttle orbital dynamics, and from
        molecular biology to automated manufacturing.
  \item[Embedded software] - Resides within a product or system and is used to implement and control features and
        functions for the end user and the system itself.
  \item[Product-line software] - Designed to provide a specific capability for use by many different customers.
  \item[Web Applications] - A set of linked hypertext files that represent information using text and graphics.
  \item[Artificial Intelligence software] - Makes use of non-numerical algorithms to solve complex problems that re not
        amenable to computation or straightforward analysis.
\end{description}

\chapter{Software Engineering}

\dfn{Software Engineering}{
  The application of a systematic, disciplined, quantifiable approach to the development, operation and maintenance of
  software; that is the application of engineering to software.
}

Software Engineering, like an onion, has layers and with any engineering approach has it's foundation in an
organizational commitment to quality.
\begin{description}
  \item[Process Layer]  - The foundation of software engineering, enables rational and timely development of computer
        software by defining a framework that must be established for effective delivery of software engineering
        technology.
  \item[Method Layer] - Provide technical descriptions and guides for building software, encompassing a broad array of
        tasks including communication, requirement analysis, design modelling, program construction, testing and support.
  \item[Tools Layer] - Provide automated or semi-automated support for the process and the methods.
\end{description}

\section{The Software Process}
\dfn{Process}{
  A collection of activities, actions, and tasks that are performed when some work product is to be created
}


\end{document}
