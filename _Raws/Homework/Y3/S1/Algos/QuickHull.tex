\documentclass[12pt]{article}
\title{Qhull}
\author{Madiba Hudson-Quansah}
\date{November 2024}

\usepackage{graphicx} % For including images
\usepackage{setspace} % For line spacing
\usepackage{titling} % For customizing the title
\usepackage[utf8]{inputenc}
\usepackage{algorithm} 
\usepackage{algpseudocode} % For typesetting algorithms
% \usepackage{biblatex} % For references
% \addbibresource{./Algos.bib}
\usepackage[margin=1in]{geometry}
\usepackage{amsmath} % For math

\setlength{\droptitle}{-10em} % Adjusts the space above the title
\doublespacing % Set double spacing for the document

\begin{document}

\begin{titlepage}
  \begin{center}

    % University Logo
    % \includegraphics[width=0.8\textwidth]{Ashesi LOGO.jpg} 

    % Title of the document
    \textbf{Quick Hull} \\



    Madiba Hudson-Quansah \\
    Kelvin K. Ahiakpor \\
    Ronelle Cudjoe \\
    Tanitoluwa O. Adebayo \\

    \textbf{Algorithm Design and Analysis} \\
    \textbf{Mr.} Samspon Asare \\
    \textbf{Prof.} Olaf Hall-Holt\\
    November 12, 2024 \\


  \end{center}
\end{titlepage}

\section{What is a Convex Hull?}
\section{Algorithms for finding the convex hull?}
\section{Quick Hull}

\singlespacing
\begin{algorithm}[H]
  \caption{QuickHull $ \left( S \right) $}
  \Comment{}\\
  \Comment{Find convex hull from the set $S$ of $n$ points, where $n \geq 3$} \\
  \Comment{Input: A set $S$ of $n$ points} \\
  \Comment{Output: The set $H$ of the points that make up the convex hull} \\
  \begin{algorithmic}[1]
    \State $H := \left\{  \right\} $ \Comment{The convex hull}
    \State Find left and right most points, $A$ and $B$ to convex hull
    \State The segment $AB$ divides the remaining $n - 2$ points into two groups $S_1$ and $S_2$ where $S_1$ are points
    in $S$ that are on the right size of the segment $AB$
    \State \Call{FindHull}{$S_1, A, B$}
    \State \Call{FindHull}{$S_2, B, A$}

    \\
    \\

    \Function{FindHull}{$S_k, P, Q$} \\
    \Comment{Find points in convex hull from the set $S_k$ of points} \\
    \Comment{that are on the right side of the segment $PQ$}\\
    \If{ $S_k$ has no points}
    \State \Return
    \EndIf
    \State From the set of points $S_k$ find the farthest point $C$ from the segment $PQ$.
    \State Add point $C$ to $H$ at the location between $P$ and $Q$
    \State The three points $P$, $Q$ and $C$ partition the remaining points of $S_k$ into 3 subsets: $S_0$, $S_1$ and
    $S_2$, where:
    \State $S_0$ are points inside the triangle $PCQ$, $S_1$ are points on the right side of the segment $PC$, and
    $S_2$ are points on the right side of the segment $CQ$
    \State \Call{FindHull}{$S_1, P, C$}
    \State \Call{FindHull}{$S_2, C, Q$}
    \EndFunction
  \end{algorithmic}
\end{algorithm}
\doublespacing

\end{document}
