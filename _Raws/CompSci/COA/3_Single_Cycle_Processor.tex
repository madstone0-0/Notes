\documentclass[12pt letter]{report}
%%%%%%%%%%%%%%%%%%%%%%%%%%%%%%%%%
% PACKAGE IMPORTS
%%%%%%%%%%%%%%%%%%%%%%%%%%%%%%%%%


\usepackage[tmargin=2cm,rmargin=1in,lmargin=1in,margin=0.85in,bmargin=2cm,footskip=.2in]{geometry}
\usepackage{amsmath,amsfonts,amsthm,amssymb,mathtools}
\usepackage[varbb]{newpxmath}
\usepackage{xfrac}
\usepackage[makeroom]{cancel}
\usepackage{mathtools}
\usepackage{bookmark}
\usepackage{enumitem}
\usepackage{hyperref,theoremref}
\hypersetup{
  pdftitle={Assignment},
  colorlinks=true, linkcolor=doc!90,
  bookmarksnumbered=true,
  bookmarksopen=true
}
\usepackage[most,many,breakable]{tcolorbox}
\usepackage{xcolor}
\usepackage{varwidth}
\usepackage{varwidth}
\usepackage{etoolbox}
%\usepackage{authblk}
\usepackage{nameref}
\usepackage{multicol,array}
\usepackage{tikz-cd}
\usepackage[ruled,vlined,linesnumbered]{algorithm2e}
\usepackage{comment} % enables the use of multi-line comments (\ifx \fi) 
\usepackage{import}
\usepackage{xifthen}
\usepackage{pdfpages}
\usepackage{transparent}
\usepackage{xcolor,colortbl,array,amssymb}
\usepackage{venndiagram}
\usepackage{listings}
\usepackage{fontspec}

\setmainfont{LibertinusSerif}[
  Extension = .otf,
  Path = /usr/share/fonts/libertinus/,
  UprightFont = *-Regular,
  ItalicFont = *-Italic,
  BoldFont = *-Bold,
  BoldItalicFont = *-BoldItalic,
]

\setmonofont{CaskaydiaCoveNerdFontMono}[
  Extension = .ttf,
  Path = /usr/share/fonts/TTF/,
  UprightFont = *-Regular,
  ItalicFont = *-Italic,
  BoldFont = *-Bold,
  BoldItalicFont = *-BoldItalic,
]

\definecolor{listing-background}{HTML}{F7F7F7}
\definecolor{listing-rule}{HTML}{B3B2B3}
\definecolor{listing-numbers}{HTML}{B3B2B3}
\definecolor{listing-text-color}{HTML}{000000}
\definecolor{listing-keyword}{HTML}{435489}
\definecolor{listing-keyword-2}{HTML}{1284CA} % additional keywords
\definecolor{listing-keyword-3}{HTML}{9137CB} % additional keywords
\definecolor{listing-identifier}{HTML}{435489}
\definecolor{listing-string}{HTML}{00999A}
\definecolor{listing-comment}{HTML}{8E8E8E}

\lstdefinestyle{eisvogel_listing_style}{
language         = java,
xleftmargin      = 0.6em,
framexleftmargin = 0.4em,
backgroundcolor  = \color{listing-background},
basicstyle       = \color{listing-text-color}\linespread{1.0}%
\lst@ifdisplaystyle%
\fi\ttfamily{},
breaklines       = true,
frame            = single,
framesep         = 0.19em,
rulecolor        = \color{listing-rule},
frameround       = ffff,
tabsize          = 4,
numberstyle      = \color{listing-numbers},
aboveskip        = 1.0em,
belowskip        = 0.1em,
abovecaptionskip = 0em,
belowcaptionskip = 1.0em,
keywordstyle     = {\color{listing-keyword}\bfseries},
keywordstyle     = {[2]\color{listing-keyword-2}\bfseries},
keywordstyle     = {[3]\color{listing-keyword-3}\bfseries\itshape},
sensitive        = true,
identifierstyle  = \color{listing-identifier},
commentstyle     = \color{listing-comment},
stringstyle      = \color{listing-string},
showstringspaces = false,
escapeinside     = {/*@}{@*/}, % Allow LaTeX inside these special comments
literate         =
  {á}{{\'a}}1 {é}{{\'e}}1 {í}{{\'i}}1 {ó}{{\'o}}1 {ú}{{\'u}}1
{Á}{{\'A}}1 {É}{{\'E}}1 {Í}{{\'I}}1 {Ó}{{\'O}}1 {Ú}{{\'U}}1
{à}{{\`a}}1 {è}{{\`e}}1 {ì}{{\`i}}1 {ò}{{\`o}}1 {ù}{{\`u}}1
{À}{{\`A}}1 {È}{{\`E}}1 {Ì}{{\`I}}1 {Ò}{{\`O}}1 {Ù}{{\`U}}1
{ä}{{\"a}}1 {ë}{{\"e}}1 {ï}{{\"i}}1 {ö}{{\"o}}1 {ü}{{\"u}}1
{Ä}{{\"A}}1 {Ë}{{\"E}}1 {Ï}{{\"I}}1 {Ö}{{\"O}}1 {Ü}{{\"U}}1
{â}{{\^a}}1 {ê}{{\^e}}1 {î}{{\^i}}1 {ô}{{\^o}}1 {û}{{\^u}}1
{Â}{{\^A}}1 {Ê}{{\^E}}1 {Î}{{\^I}}1 {Ô}{{\^O}}1 {Û}{{\^U}}1
{œ}{{\oe}}1 {Œ}{{\OE}}1 {æ}{{\ae}}1 {Æ}{{\AE}}1 {ß}{{\ss}}1
{ç}{{\c c}}1 {Ç}{{\c C}}1 {ø}{{\o}}1 {å}{{\r a}}1 {Å}{{\r A}}1
{€}{{\EUR}}1 {£}{{\pounds}}1 {«}{{\guillemotleft}}1
{»}{{\guillemotright}}1 {ñ}{{\~n}}1 {Ñ}{{\~N}}1 {¿}{{?`}}1
{…}{{\ldots}}1 {≥}{{>=}}1 {≤}{{<=}}1 {„}{{\glqq}}1 {“}{{\grqq}}1
{”}{{''}}1
}
\lstset{style=eisvogel_listing_style,
  numbers=left}

%
% Java (Java SE 12, 2019-06-22)
%
\lstdefinelanguage{Java}{
  morekeywords={
      % normal keywords (without data types)
      abstract,assert,break,case,catch,class,continue,default,
      do,else,enum,exports,extends,final,finally,for,if,implements,
      import,instanceof,interface,module,native,new,package,private,
      protected,public,requires,return,static,strictfp,super,switch,
      synchronized,this,throw,throws,transient,try,volatile,while,
      % var is an identifier
      var
    },
  morekeywords={[2] % data types
      % primitive data types
      boolean,byte,char,double,float,int,long,short,
      % String
      String,
      % primitive wrapper types
      Boolean,Byte,Character,Double,Float,Integer,Long,Short
      % number types
      Number,AtomicInteger,AtomicLong,BigDecimal,BigInteger,DoubleAccumulator,DoubleAdder,LongAccumulator,LongAdder,Short,
      % other
      Object,Void,void
    },
  morekeywords={[3] % literals
      % reserved words for literal values
      null,true,false,
    },
  sensitive,
  morecomment  = [l]//,
  morecomment  = [s]{/*}{*/},
  morecomment  = [s]{/**}{*/},
  morestring   = [b]",
  morestring   = [b]',
}

\lstdefinelanguage{XML}{
  morestring      = [b]",
  moredelim       = [s][\bfseries\color{listing-keyword}]{<}{\ },
  moredelim       = [s][\bfseries\color{listing-keyword}]{</}{>},
  moredelim       = [l][\bfseries\color{listing-keyword}]{/>},
  moredelim       = [l][\bfseries\color{listing-keyword}]{>},
  morecomment     = [s]{<?}{?>},
  morecomment     = [s]{<!--}{-->},
  commentstyle    = \color{listing-comment},
  stringstyle     = \color{listing-string},
  identifierstyle = \color{listing-identifier}
}


\newcommand\mycommfont[1]{\footnotesize\ttfamily\textcolor{blue}{#1}}
\SetCommentSty{mycommfont}
\newcommand{\incfig}[1]{%
  \def\svgwidth{\columnwidth}
  \import{./figures/}{#1.pdf_tex}
}

\usepackage{tikzsymbols}
\renewcommand\qedsymbol{$\Laughey$}


%\usepackage{import}
%\usepackage{xifthen}
%\usepackage{pdfpages}
%\usepackage{transparent}


%%%%%%%%%%%%%%%%%%%%%%%%%%%%%%
% SELF MADE COLORS
%%%%%%%%%%%%%%%%%%%%%%%%%%%%%%



\definecolor{myg}{RGB}{56, 140, 70}
\definecolor{myb}{RGB}{45, 111, 177}
\definecolor{myr}{RGB}{199, 68, 64}
\definecolor{mytheorembg}{HTML}{F2F2F9}
\definecolor{mytheoremfr}{HTML}{00007B}
\definecolor{mylenmabg}{HTML}{FFFAF8}
\definecolor{mylenmafr}{HTML}{983b0f}
\definecolor{mypropbg}{HTML}{f2fbfc}
\definecolor{mypropfr}{HTML}{191971}
\definecolor{myexamplebg}{HTML}{F2FBF8}
\definecolor{myexamplefr}{HTML}{88D6D1}
\definecolor{myexampleti}{HTML}{2A7F7F}
\definecolor{mydefinitbg}{HTML}{E5E5FF}
\definecolor{mydefinitfr}{HTML}{3F3FA3}
\definecolor{notesgreen}{RGB}{0,162,0}
\definecolor{myp}{RGB}{197, 92, 212}
\definecolor{mygr}{HTML}{2C3338}
\definecolor{myred}{RGB}{127,0,0}
\definecolor{myyellow}{RGB}{169,121,69}
\definecolor{myexercisebg}{HTML}{F2FBF8}
\definecolor{myexercisefg}{HTML}{88D6D1}


%%%%%%%%%%%%%%%%%%%%%%%%%%%%
% TCOLORBOX SETUPS
%%%%%%%%%%%%%%%%%%%%%%%%%%%%

\setlength{\parindent}{1cm}
%================================
% THEOREM BOX
%================================

\tcbuselibrary{theorems,skins,hooks}
\newtcbtheorem[number within=section]{Theorem}{Theorem}
{%
  enhanced,
  breakable,
  colback = mytheorembg,
  frame hidden,
  boxrule = 0sp,
  borderline west = {2pt}{0pt}{mytheoremfr},
  sharp corners,
  detach title,
  before upper = \tcbtitle\par\smallskip,
  coltitle = mytheoremfr,
  fonttitle = \bfseries\sffamily,
  description font = \mdseries,
  separator sign none,
  segmentation style={solid, mytheoremfr},
}
{th}

\tcbuselibrary{theorems,skins,hooks}
\newtcbtheorem[number within=chapter]{theorem}{Theorem}
{%
  enhanced,
  breakable,
  colback = mytheorembg,
  frame hidden,
  boxrule = 0sp,
  borderline west = {2pt}{0pt}{mytheoremfr},
  sharp corners,
  detach title,
  before upper = \tcbtitle\par\smallskip,
  coltitle = mytheoremfr,
  fonttitle = \bfseries\sffamily,
  description font = \mdseries,
  separator sign none,
  segmentation style={solid, mytheoremfr},
}
{th}


\tcbuselibrary{theorems,skins,hooks}
\newtcolorbox{Theoremcon}
{%
  enhanced
  ,breakable
  ,colback = mytheorembg
  ,frame hidden
  ,boxrule = 0sp
  ,borderline west = {2pt}{0pt}{mytheoremfr}
  ,sharp corners
  ,description font = \mdseries
  ,separator sign none
}

%================================
% Corollery
%================================
\tcbuselibrary{theorems,skins,hooks}
\newtcbtheorem[number within=section]{Corollary}{Corollary}
{%
  enhanced
  ,breakable
  ,colback = myp!10
  ,frame hidden
  ,boxrule = 0sp
  ,borderline west = {2pt}{0pt}{myp!85!black}
  ,sharp corners
  ,detach title
  ,before upper = \tcbtitle\par\smallskip
  ,coltitle = myp!85!black
  ,fonttitle = \bfseries\sffamily
  ,description font = \mdseries
  ,separator sign none
  ,segmentation style={solid, myp!85!black}
}
{th}
\tcbuselibrary{theorems,skins,hooks}
\newtcbtheorem[number within=chapter]{corollary}{Corollary}
{%
  enhanced
  ,breakable
  ,colback = myp!10
  ,frame hidden
  ,boxrule = 0sp
  ,borderline west = {2pt}{0pt}{myp!85!black}
  ,sharp corners
  ,detach title
  ,before upper = \tcbtitle\par\smallskip
  ,coltitle = myp!85!black
  ,fonttitle = \bfseries\sffamily
  ,description font = \mdseries
  ,separator sign none
  ,segmentation style={solid, myp!85!black}
}
{th}


%================================
% LENMA
%================================

\tcbuselibrary{theorems,skins,hooks}
\newtcbtheorem[number within=section]{Lenma}{Lenma}
{%
  enhanced,
  breakable,
  colback = mylenmabg,
  frame hidden,
  boxrule = 0sp,
  borderline west = {2pt}{0pt}{mylenmafr},
  sharp corners,
  detach title,
  before upper = \tcbtitle\par\smallskip,
  coltitle = mylenmafr,
  fonttitle = \bfseries\sffamily,
  description font = \mdseries,
  separator sign none,
  segmentation style={solid, mylenmafr},
}
{th}

\tcbuselibrary{theorems,skins,hooks}
\newtcbtheorem[number within=chapter]{lenma}{Lenma}
{%
  enhanced,
  breakable,
  colback = mylenmabg,
  frame hidden,
  boxrule = 0sp,
  borderline west = {2pt}{0pt}{mylenmafr},
  sharp corners,
  detach title,
  before upper = \tcbtitle\par\smallskip,
  coltitle = mylenmafr,
  fonttitle = \bfseries\sffamily,
  description font = \mdseries,
  separator sign none,
  segmentation style={solid, mylenmafr},
}
{th}


%================================
% PROPOSITION
%================================

\tcbuselibrary{theorems,skins,hooks}
\newtcbtheorem[number within=section]{Prop}{Proposition}
{%
  enhanced,
  breakable,
  colback = mypropbg,
  frame hidden,
  boxrule = 0sp,
  borderline west = {2pt}{0pt}{mypropfr},
  sharp corners,
  detach title,
  before upper = \tcbtitle\par\smallskip,
  coltitle = mypropfr,
  fonttitle = \bfseries\sffamily,
  description font = \mdseries,
  separator sign none,
  segmentation style={solid, mypropfr},
}
{th}

\tcbuselibrary{theorems,skins,hooks}
\newtcbtheorem[number within=chapter]{prop}{Proposition}
{%
  enhanced,
  breakable,
  colback = mypropbg,
  frame hidden,
  boxrule = 0sp,
  borderline west = {2pt}{0pt}{mypropfr},
  sharp corners,
  detach title,
  before upper = \tcbtitle\par\smallskip,
  coltitle = mypropfr,
  fonttitle = \bfseries\sffamily,
  description font = \mdseries,
  separator sign none,
  segmentation style={solid, mypropfr},
}
{th}


%================================
% CLAIM
%================================

\tcbuselibrary{theorems,skins,hooks}
\newtcbtheorem[number within=section]{claim}{Claim}
{%
  enhanced
  ,breakable
  ,colback = myg!10
  ,frame hidden
  ,boxrule = 0sp
  ,borderline west = {2pt}{0pt}{myg}
  ,sharp corners
  ,detach title
  ,before upper = \tcbtitle\par\smallskip
  ,coltitle = myg!85!black
  ,fonttitle = \bfseries\sffamily
  ,description font = \mdseries
  ,separator sign none
  ,segmentation style={solid, myg!85!black}
}
{th}



%================================
% Exercise
%================================

\tcbuselibrary{theorems,skins,hooks}
\newtcbtheorem[number within=section]{Exercise}{Exercise}
{%
  enhanced,
  breakable,
  colback = myexercisebg,
  frame hidden,
  boxrule = 0sp,
  borderline west = {2pt}{0pt}{myexercisefg},
  sharp corners,
  detach title,
  before upper = \tcbtitle\par\smallskip,
  coltitle = myexercisefg,
  fonttitle = \bfseries\sffamily,
  description font = \mdseries,
  separator sign none,
  segmentation style={solid, myexercisefg},
}
{th}

\tcbuselibrary{theorems,skins,hooks}
\newtcbtheorem[number within=chapter]{exercise}{Exercise}
{%
  enhanced,
  breakable,
  colback = myexercisebg,
  frame hidden,
  boxrule = 0sp,
  borderline west = {2pt}{0pt}{myexercisefg},
  sharp corners,
  detach title,
  before upper = \tcbtitle\par\smallskip,
  coltitle = myexercisefg,
  fonttitle = \bfseries\sffamily,
  description font = \mdseries,
  separator sign none,
  segmentation style={solid, myexercisefg},
}
{th}

%================================
% EXAMPLE BOX
%================================

\newtcbtheorem[number within=section]{Example}{Example}
{%
  colback = myexamplebg
  ,breakable
  ,colframe = myexamplefr
  ,coltitle = myexampleti
  ,boxrule = 1pt
  ,sharp corners
  ,detach title
  ,before upper=\tcbtitle\par\smallskip
  ,fonttitle = \bfseries
  ,description font = \mdseries
  ,separator sign none
  ,description delimiters parenthesis
}
{ex}

\newtcbtheorem[number within=chapter]{example}{Example}
{%
  colback = myexamplebg
  ,breakable
  ,colframe = myexamplefr
  ,coltitle = myexampleti
  ,boxrule = 1pt
  ,sharp corners
  ,detach title
  ,before upper=\tcbtitle\par\smallskip
  ,fonttitle = \bfseries
  ,description font = \mdseries
  ,separator sign none
  ,description delimiters parenthesis
}
{ex}

%================================
% DEFINITION BOX
%================================

\newtcbtheorem[number within=section]{Definition}{Definition}{enhanced,
  before skip=2mm,after skip=2mm, colback=red!5,colframe=red!80!black,boxrule=0.5mm,
  attach boxed title to top left={xshift=1cm,yshift*=1mm-\tcboxedtitleheight}, varwidth boxed title*=-3cm,
  boxed title style={frame code={
          \path[fill=tcbcolback]
          ([yshift=-1mm,xshift=-1mm]frame.north west)
          arc[start angle=0,end angle=180,radius=1mm]
          ([yshift=-1mm,xshift=1mm]frame.north east)
          arc[start angle=180,end angle=0,radius=1mm];
          \path[left color=tcbcolback!60!black,right color=tcbcolback!60!black,
            middle color=tcbcolback!80!black]
          ([xshift=-2mm]frame.north west) -- ([xshift=2mm]frame.north east)
          [rounded corners=1mm]-- ([xshift=1mm,yshift=-1mm]frame.north east)
          -- (frame.south east) -- (frame.south west)
          -- ([xshift=-1mm,yshift=-1mm]frame.north west)
          [sharp corners]-- cycle;
        },interior engine=empty,
    },
  fonttitle=\bfseries,
  title={#2},#1}{def}
\newtcbtheorem[number within=chapter]{definition}{Definition}{enhanced,
  before skip=2mm,after skip=2mm, colback=red!5,colframe=red!80!black,boxrule=0.5mm,
  attach boxed title to top left={xshift=1cm,yshift*=1mm-\tcboxedtitleheight}, varwidth boxed title*=-3cm,
  boxed title style={frame code={
          \path[fill=tcbcolback]
          ([yshift=-1mm,xshift=-1mm]frame.north west)
          arc[start angle=0,end angle=180,radius=1mm]
          ([yshift=-1mm,xshift=1mm]frame.north east)
          arc[start angle=180,end angle=0,radius=1mm];
          \path[left color=tcbcolback!60!black,right color=tcbcolback!60!black,
            middle color=tcbcolback!80!black]
          ([xshift=-2mm]frame.north west) -- ([xshift=2mm]frame.north east)
          [rounded corners=1mm]-- ([xshift=1mm,yshift=-1mm]frame.north east)
          -- (frame.south east) -- (frame.south west)
          -- ([xshift=-1mm,yshift=-1mm]frame.north west)
          [sharp corners]-- cycle;
        },interior engine=empty,
    },
  fonttitle=\bfseries,
  title={#2},#1}{def}



%================================
% Solution BOX
%================================

\makeatletter
\newtcbtheorem{question}{Question}{enhanced,
  breakable,
  colback=white,
  colframe=myb!80!black,
  attach boxed title to top left={yshift*=-\tcboxedtitleheight},
  fonttitle=\bfseries,
  title={#2},
  boxed title size=title,
  boxed title style={%
      sharp corners,
      rounded corners=northwest,
      colback=tcbcolframe,
      boxrule=0pt,
    },
  underlay boxed title={%
      \path[fill=tcbcolframe] (title.south west)--(title.south east)
      to[out=0, in=180] ([xshift=5mm]title.east)--
      (title.center-|frame.east)
      [rounded corners=\kvtcb@arc] |-
      (frame.north) -| cycle;
    },
  #1
}{def}
\makeatother

%================================
% SOLUTION BOX
%================================

\makeatletter
\newtcolorbox{solution}{enhanced,
  breakable,
  colback=white,
  colframe=myg!80!black,
  attach boxed title to top left={yshift*=-\tcboxedtitleheight},
  title=Solution,
  boxed title size=title,
  boxed title style={%
      sharp corners,
      rounded corners=northwest,
      colback=tcbcolframe,
      boxrule=0pt,
    },
  underlay boxed title={%
      \path[fill=tcbcolframe] (title.south west)--(title.south east)
      to[out=0, in=180] ([xshift=5mm]title.east)--
      (title.center-|frame.east)
      [rounded corners=\kvtcb@arc] |-
      (frame.north) -| cycle;
    },
}
\makeatother

%================================
% Question BOX
%================================

\makeatletter
\newtcbtheorem{qstion}{Question}{enhanced,
  breakable,
  colback=white,
  colframe=mygr,
  attach boxed title to top left={yshift*=-\tcboxedtitleheight},
  fonttitle=\bfseries,
  title={#2},
  boxed title size=title,
  boxed title style={%
      sharp corners,
      rounded corners=northwest,
      colback=tcbcolframe,
      boxrule=0pt,
    },
  underlay boxed title={%
      \path[fill=tcbcolframe] (title.south west)--(title.south east)
      to[out=0, in=180] ([xshift=5mm]title.east)--
      (title.center-|frame.east)
      [rounded corners=\kvtcb@arc] |-
      (frame.north) -| cycle;
    },
  #1
}{def}
\makeatother

\newtcbtheorem[number within=chapter]{wconc}{Wrong Concept}{
  breakable,
  enhanced,
  colback=white,
  colframe=myr,
  arc=0pt,
  outer arc=0pt,
  fonttitle=\bfseries\sffamily\large,
  colbacktitle=myr,
  attach boxed title to top left={},
  boxed title style={
      enhanced,
      skin=enhancedfirst jigsaw,
      arc=3pt,
      bottom=0pt,
      interior style={fill=myr}
    },
  #1
}{def}



%================================
% NOTE BOX
%================================

\usetikzlibrary{arrows,calc,shadows.blur}
\tcbuselibrary{skins}
\newtcolorbox{note}[1][]{%
  enhanced jigsaw,
  colback=gray!20!white,%
  colframe=gray!80!black,
  size=small,
  boxrule=1pt,
  title=\textbf{Note:-},
  halign title=flush center,
  coltitle=black,
  breakable,
  drop shadow=black!50!white,
  attach boxed title to top left={xshift=1cm,yshift=-\tcboxedtitleheight/2,yshifttext=-\tcboxedtitleheight/2},
  minipage boxed title=1.5cm,
  boxed title style={%
      colback=white,
      size=fbox,
      boxrule=1pt,
      boxsep=2pt,
      underlay={%
          \coordinate (dotA) at ($(interior.west) + (-0.5pt,0)$);
          \coordinate (dotB) at ($(interior.east) + (0.5pt,0)$);
          \begin{scope}
            \clip (interior.north west) rectangle ([xshift=3ex]interior.east);
            \filldraw [white, blur shadow={shadow opacity=60, shadow yshift=-.75ex}, rounded corners=2pt] (interior.north west) rectangle (interior.south east);
          \end{scope}
          \begin{scope}[gray!80!black]
            \fill (dotA) circle (2pt);
            \fill (dotB) circle (2pt);
          \end{scope}
        },
    },
  #1,
}

%%%%%%%%%%%%%%%%%%%%%%%%%%%%%%
% SELF MADE COMMANDS
%%%%%%%%%%%%%%%%%%%%%%%%%%%%%%


\newcommand{\thm}[2]{\begin{Theorem}{#1}{}#2\end{Theorem}}
\newcommand{\cor}[2]{\begin{Corollary}{#1}{}#2\end{Corollary}}
\newcommand{\mlenma}[2]{\begin{Lenma}{#1}{}#2\end{Lenma}}
\newcommand{\mprop}[2]{\begin{Prop}{#1}{}#2\end{Prop}}
\newcommand{\clm}[3]{\begin{claim}{#1}{#2}#3\end{claim}}
\newcommand{\wc}[2]{\begin{wconc}{#1}{}\setlength{\parindent}{1cm}#2\end{wconc}}
\newcommand{\thmcon}[1]{\begin{Theoremcon}{#1}\end{Theoremcon}}
\newcommand{\ex}[2]{\begin{Example}{#1}{}#2\end{Example}}
\newcommand{\dfn}[2]{\begin{Definition}[colbacktitle=red!75!black]{#1}{}#2\end{Definition}}
\newcommand{\dfnc}[2]{\begin{definition}[colbacktitle=red!75!black]{#1}{}#2\end{definition}}
\newcommand{\qs}[2]{\begin{question}{#1}{}#2\end{question}}
\newcommand{\pf}[2]{\begin{myproof}[#1]#2\end{myproof}}
\newcommand{\nt}[1]{\begin{note}#1\end{note}}

\newcommand*\circled[1]{\tikz[baseline=(char.base)]{
    \node[shape=circle,draw,inner sep=1pt] (char) {#1};}}
\newcommand\getcurrentref[1]{%
  \ifnumequal{\value{#1}}{0}
  {??}
  {\the\value{#1}}%
}
\newcommand{\getCurrentSectionNumber}{\getcurrentref{section}}
\newenvironment{myproof}[1][\proofname]{%
  \proof[\bfseries #1: ]%
}{\endproof}

\newcommand{\mclm}[2]{\begin{myclaim}[#1]#2\end{myclaim}}
\newenvironment{myclaim}[1][\claimname]{\proof[\bfseries #1: ]}{}

\newcounter{mylabelcounter}

\makeatletter
\newcommand{\setword}[2]{%
  \phantomsection
  #1\def\@currentlabel{\unexpanded{#1}}\label{#2}%
}
\makeatother




\tikzset{
  symbol/.style={
      draw=none,
      every to/.append style={
          edge node={node [sloped, allow upside down, auto=false]{$#1$}}}
    }
}


% deliminators
\DeclarePairedDelimiter{\abs}{\lvert}{\rvert}
\DeclarePairedDelimiter{\norm}{\lVert}{\rVert}

\DeclarePairedDelimiter{\ceil}{\lceil}{\rceil}
\DeclarePairedDelimiter{\floor}{\lfloor}{\rfloor}
\DeclarePairedDelimiter{\round}{\lfloor}{\rceil}

\newsavebox\diffdbox
\newcommand{\slantedromand}{{\mathpalette\makesl{d}}}
\newcommand{\makesl}[2]{%
  \begingroup
  \sbox{\diffdbox}{$\mathsurround=0pt#1\mathrm{#2}$}%
  \pdfsave
  \pdfsetmatrix{1 0 0.2 1}%
  \rlap{\usebox{\diffdbox}}%
  \pdfrestore
  \hskip\wd\diffdbox
  \endgroup
}
\newcommand{\dd}[1][]{\ensuremath{\mathop{}\!\ifstrempty{#1}{%
      \slantedromand\@ifnextchar^{\hspace{0.2ex}}{\hspace{0.1ex}}}%
    {\slantedromand\hspace{0.2ex}^{#1}}}}
\ProvideDocumentCommand\dv{o m g}{%
  \ensuremath{%
    \IfValueTF{#3}{%
      \IfNoValueTF{#1}{%
        \frac{\dd #2}{\dd #3}%
      }{%
        \frac{\dd^{#1} #2}{\dd #3^{#1}}%
      }%
    }{%
      \IfNoValueTF{#1}{%
        \frac{\dd}{\dd #2}%
      }{%
        \frac{\dd^{#1}}{\dd #2^{#1}}%
      }%
    }%
  }%
}
\providecommand*{\pdv}[3][]{\frac{\partial^{#1}#2}{\partial#3^{#1}}}
%  - others
\DeclareMathOperator{\Lap}{\mathcal{L}}
\DeclareMathOperator{\Var}{Var} % varience
\DeclareMathOperator{\Cov}{Cov} % covarience
\DeclareMathOperator{\E}{E} % expected

% Since the amsthm package isn't loaded

% I prefer the slanted \leq
\let\oldleq\leq % save them in case they're every wanted
\let\oldgeq\geq
\renewcommand{\leq}{\leqslant}
\renewcommand{\geq}{\geqslant}

% % redefine matrix env to allow for alignment, use r as default
% \renewcommand*\env@matrix[1][r]{\hskip -\arraycolsep
%     \let\@ifnextchar\new@ifnextchar
%     \array{*\c@MaxMatrixCols #1}}


%\usepackage{framed}
%\usepackage{titletoc}
%\usepackage{etoolbox}
%\usepackage{lmodern}


%\patchcmd{\tableofcontents}{\contentsname}{\sffamily\contentsname}{}{}

%\renewenvironment{leftbar}
%{\def\FrameCommand{\hspace{6em}%
%		{\color{myyellow}\vrule width 2pt depth 6pt}\hspace{1em}}%
%	\MakeFramed{\parshape 1 0cm \dimexpr\textwidth-6em\relax\FrameRestore}\vskip2pt%
%}
%{\endMakeFramed}

%\titlecontents{chapter}
%[0em]{\vspace*{2\baselineskip}}
%{\parbox{4.5em}{%
%		\hfill\Huge\sffamily\bfseries\color{myred}\thecontentspage}%
%	\vspace*{-2.3\baselineskip}\leftbar\textsc{\small\chaptername~\thecontentslabel}\\\sffamily}
%{}{\endleftbar}
%\titlecontents{section}
%[8.4em]
%{\sffamily\contentslabel{3em}}{}{}
%{\hspace{0.5em}\nobreak\itshape\color{myred}\contentspage}
%\titlecontents{subsection}
%[8.4em]
%{\sffamily\contentslabel{3em}}{}{}  
%{\hspace{0.5em}\nobreak\itshape\color{myred}\contentspage}



%%%%%%%%%%%%%%%%%%%%%%%%%%%%%%%%%%%%%%%%%%%
% TABLE OF CONTENTS
%%%%%%%%%%%%%%%%%%%%%%%%%%%%%%%%%%%%%%%%%%%

\usepackage{tikz}
\definecolor{doc}{RGB}{0,60,110}
\usepackage{titletoc}
\contentsmargin{0cm}
\titlecontents{chapter}[3.7pc]
{\addvspace{30pt}%
  \begin{tikzpicture}[remember picture, overlay]%
    \draw[fill=doc!60,draw=doc!60] (-7,-.1) rectangle (-0.9,.5);%
    \pgftext[left,x=-3.5cm,y=0.2cm]{\color{white}\Large\sc\bfseries Chapter\ \thecontentslabel};%
  \end{tikzpicture}\color{doc!60}\large\sc\bfseries}%
{}
{}
{\;\titlerule\;\large\sc\bfseries Page \thecontentspage
  \begin{tikzpicture}[remember picture, overlay]
    \draw[fill=doc!60,draw=doc!60] (2pt,0) rectangle (4,0.1pt);
  \end{tikzpicture}}%
\titlecontents{section}[3.7pc]
{\addvspace{2pt}}
{\contentslabel[\thecontentslabel]{2pc}}
{}
{\hfill\small \thecontentspage}
[]
\titlecontents*{subsection}[3.7pc]
{\addvspace{-1pt}\small}
{}
{}
{\ --- \small\thecontentspage}
[ \textbullet\ ][]

\makeatletter
\renewcommand{\tableofcontents}{%
  \chapter*{%
    \vspace*{-20\p@}%
    \begin{tikzpicture}[remember picture, overlay]%
      \pgftext[right,x=15cm,y=0.2cm]{\color{doc!60}\Huge\sc\bfseries \contentsname};%
      \draw[fill=doc!60,draw=doc!60] (13,-.75) rectangle (20,1);%
      \clip (13,-.75) rectangle (20,1);
      \pgftext[right,x=15cm,y=0.2cm]{\color{white}\Huge\sc\bfseries \contentsname};%
    \end{tikzpicture}}%
  \@starttoc{toc}}
\makeatother

%From M275 "Topology" at SJSU
\newcommand{\id}{\mathrm{id}}
\newcommand{\taking}[1]{\xrightarrow{#1}}
\newcommand{\inv}{^{-1}}

%From M170 "Introduction to Graph Theory" at SJSU
\DeclareMathOperator{\diam}{diam}
\DeclareMathOperator{\ord}{ord}
\newcommand{\defeq}{\overset{\mathrm{def}}{=}}

%From the USAMO .tex files
\newcommand{\ts}{\textsuperscript}
\newcommand{\dg}{^\circ}
\newcommand{\ii}{\item}

% % From Math 55 and Math 145 at Harvard
% \newenvironment{subproof}[1][Proof]{%
% \begin{proof}[#1] \renewcommand{\qedsymbol}{$\blacksquare$}}%
% {\end{proof}}

\newcommand{\liff}{\leftrightarrow}
\newcommand{\lthen}{\rightarrow}
\newcommand{\opname}{\operatorname}
\newcommand{\surjto}{\twoheadrightarrow}
\newcommand{\injto}{\hookrightarrow}
\newcommand{\On}{\mathrm{On}} % ordinals
\DeclareMathOperator{\img}{im} % Image
\DeclareMathOperator{\Img}{Im} % Image
\DeclareMathOperator{\coker}{coker} % Cokernel
\DeclareMathOperator{\Coker}{Coker} % Cokernel
\DeclareMathOperator{\Ker}{Ker} % Kernel
\DeclareMathOperator{\rank}{rank}
\DeclareMathOperator{\Spec}{Spec} % spectrum
\DeclareMathOperator{\Tr}{Tr} % trace
\DeclareMathOperator{\pr}{pr} % projection
\DeclareMathOperator{\ext}{ext} % extension
\DeclareMathOperator{\pred}{pred} % predecessor
\DeclareMathOperator{\dom}{dom} % domain
\DeclareMathOperator{\ran}{ran} % range
\DeclareMathOperator{\Hom}{Hom} % homomorphism
\DeclareMathOperator{\Mor}{Mor} % morphisms
\DeclareMathOperator{\End}{End} % endomorphism

\newcommand{\eps}{\epsilon}
\newcommand{\veps}{\varepsilon}
\newcommand{\ol}{\overline}
\newcommand{\ul}{\underline}
\newcommand{\wt}{\widetilde}
\newcommand{\wh}{\widehat}
\newcommand{\vocab}[1]{\textbf{\color{blue} #1}}
\providecommand{\half}{\frac{1}{2}}
\newcommand{\dang}{\measuredangle} %% Directed angle
\newcommand{\ray}[1]{\overrightarrow{#1}}
\newcommand{\seg}[1]{\overline{#1}}
\newcommand{\arc}[1]{\wideparen{#1}}
\DeclareMathOperator{\cis}{cis}
\DeclareMathOperator*{\lcm}{lcm}
\DeclareMathOperator*{\argmin}{arg min}
\DeclareMathOperator*{\argmax}{arg max}
\newcommand{\cycsum}{\sum_{\mathrm{cyc}}}
\newcommand{\symsum}{\sum_{\mathrm{sym}}}
\newcommand{\cycprod}{\prod_{\mathrm{cyc}}}
\newcommand{\symprod}{\prod_{\mathrm{sym}}}
\newcommand{\Qed}{\begin{flushright}\qed\end{flushright}}
\newcommand{\parinn}{\setlength{\parindent}{1cm}}
\newcommand{\parinf}{\setlength{\parindent}{0cm}}
% \newcommand{\norm}{\|\cdot\|}
\newcommand{\inorm}{\norm_{\infty}}
\newcommand{\opensets}{\{V_{\alpha}\}_{\alpha\in I}}
\newcommand{\oset}{V_{\alpha}}
\newcommand{\opset}[1]{V_{\alpha_{#1}}}
\newcommand{\lub}{\text{lub}}
\newcommand{\del}[2]{\frac{\partial #1}{\partial #2}}
\newcommand{\Del}[3]{\frac{\partial^{#1} #2}{\partial^{#1} #3}}
\newcommand{\deld}[2]{\dfrac{\partial #1}{\partial #2}}
\newcommand{\Deld}[3]{\dfrac{\partial^{#1} #2}{\partial^{#1} #3}}
\newcommand{\lm}{\lambda}
\newcommand{\uin}{\mathbin{\rotatebox[origin=c]{90}{$\in$}}}
\newcommand{\usubset}{\mathbin{\rotatebox[origin=c]{90}{$\subset$}}}
\newcommand{\lt}{\left}
\newcommand{\rt}{\right}
\newcommand{\bs}[1]{\boldsymbol{#1}}
\newcommand{\exs}{\exists}
\newcommand{\st}{\strut}
\newcommand{\dps}[1]{\displaystyle{#1}}

\newcommand{\sol}{\setlength{\parindent}{0cm}\textbf{\textit{Solution:}}\setlength{\parindent}{1cm} }
\newcommand{\solve}[1]{\setlength{\parindent}{0cm}\textbf{\textit{Solution: }}\setlength{\parindent}{1cm}#1 \Qed}

\preto\tabular{\setcounter{magicrownumbers}{0}}
\newcounter{magicrownumbers}
\newcommand\rownumber{\stepcounter{magicrownumbers}\arabic{magicrownumbers}}
\def\rownumber{}

\newenvironment{deduction}
{\begin{tabular}{@{}>{$}c<{$}@{\enspace}>{$}l<{$}@{}}\arrayrulecolor{blue!50}}
		{\end{tabular}}
\newcommand{\premise}[1]{&#1\\}
\newcommand{\conclusion}[1]{\cline{2-2}\therefore&#1}


% Things Lie
\newcommand{\kb}{\mathfrak b}
\newcommand{\kg}{\mathfrak g}
\newcommand{\kh}{\mathfrak h}
\newcommand{\kn}{\mathfrak n}
\newcommand{\ku}{\mathfrak u}
\newcommand{\kz}{\mathfrak z}
\DeclareMathOperator{\Ext}{Ext} % Ext functor
\DeclareMathOperator{\Tor}{Tor} % Tor functor
\newcommand{\gl}{\opname{\mathfrak{gl}}} % frak gl group
\renewcommand{\sl}{\opname{\mathfrak{sl}}} % frak sl group chktex 6

% More script letters etc.
\newcommand{\SA}{\mathcal A}
\newcommand{\SB}{\mathcal B}
\newcommand{\SC}{\mathcal C}
\newcommand{\SF}{\mathcal F}
\newcommand{\SG}{\mathcal G}
\newcommand{\SH}{\mathcal H}
\newcommand{\OO}{\mathcal O}

\newcommand{\SCA}{\mathscr A}
\newcommand{\SCB}{\mathscr B}
\newcommand{\SCC}{\mathscr C}
\newcommand{\SCD}{\mathscr D}
\newcommand{\SCE}{\mathscr E}
\newcommand{\SCF}{\mathscr F}
\newcommand{\SCG}{\mathscr G}
\newcommand{\SCH}{\mathscr H}

% Mathfrak primes
\newcommand{\km}{\mathfrak m}
\newcommand{\kp}{\mathfrak p}
\newcommand{\kq}{\mathfrak q}

% number sets
\newcommand{\RR}[1][]{\ensuremath{\ifstrempty{#1}{\mathbb{R}}{\mathbb{R}^{#1}}}}
\newcommand{\NN}[1][]{\ensuremath{\ifstrempty{#1}{\mathbb{N}}{\mathbb{N}^{#1}}}}
\newcommand{\ZZ}[1][]{\ensuremath{\ifstrempty{#1}{\mathbb{Z}}{\mathbb{Z}^{#1}}}}
\newcommand{\QQ}[1][]{\ensuremath{\ifstrempty{#1}{\mathbb{Q}}{\mathbb{Q}^{#1}}}}
\newcommand{\CC}[1][]{\ensuremath{\ifstrempty{#1}{\mathbb{C}}{\mathbb{C}^{#1}}}}
\newcommand{\PP}[1][]{\ensuremath{\ifstrempty{#1}{\mathbb{P}}{\mathbb{P}^{#1}}}}
\newcommand{\HH}[1][]{\ensuremath{\ifstrempty{#1}{\mathbb{H}}{\mathbb{H}^{#1}}}}
\newcommand{\FF}[1][]{\ensuremath{\ifstrempty{#1}{\mathbb{F}}{\mathbb{F}^{#1}}}}
% expected value
\newcommand{\EE}{\ensuremath{\mathbb{E}}}
\newcommand{\charin}{\text{ char }}
\DeclareMathOperator{\sign}{sign}
\DeclareMathOperator{\Aut}{Aut}
\DeclareMathOperator{\Inn}{Inn}
\DeclareMathOperator{\Syl}{Syl}
\DeclareMathOperator{\Gal}{Gal}
\DeclareMathOperator{\GL}{GL} % General linear group
\DeclareMathOperator{\SL}{SL} % Special linear group

%---------------------------------------
% BlackBoard Math Fonts :-
%---------------------------------------

%Captital Letters
\newcommand{\bbA}{\mathbb{A}}	\newcommand{\bbB}{\mathbb{B}}
\newcommand{\bbC}{\mathbb{C}}	\newcommand{\bbD}{\mathbb{D}}
\newcommand{\bbE}{\mathbb{E}}	\newcommand{\bbF}{\mathbb{F}}
\newcommand{\bbG}{\mathbb{G}}	\newcommand{\bbH}{\mathbb{H}}
\newcommand{\bbI}{\mathbb{I}}	\newcommand{\bbJ}{\mathbb{J}}
\newcommand{\bbK}{\mathbb{K}}	\newcommand{\bbL}{\mathbb{L}}
\newcommand{\bbM}{\mathbb{M}}	\newcommand{\bbN}{\mathbb{N}}
\newcommand{\bbO}{\mathbb{O}}	\newcommand{\bbP}{\mathbb{P}}
\newcommand{\bbQ}{\mathbb{Q}}	\newcommand{\bbR}{\mathbb{R}}
\newcommand{\bbS}{\mathbb{S}}	\newcommand{\bbT}{\mathbb{T}}
\newcommand{\bbU}{\mathbb{U}}	\newcommand{\bbV}{\mathbb{V}}
\newcommand{\bbW}{\mathbb{W}}	\newcommand{\bbX}{\mathbb{X}}
\newcommand{\bbY}{\mathbb{Y}}	\newcommand{\bbZ}{\mathbb{Z}}

%---------------------------------------
% MathCal Fonts :-
%---------------------------------------

%Captital Letters
\newcommand{\mcA}{\mathcal{A}}	\newcommand{\mcB}{\mathcal{B}}
\newcommand{\mcC}{\mathcal{C}}	\newcommand{\mcD}{\mathcal{D}}
\newcommand{\mcE}{\mathcal{E}}	\newcommand{\mcF}{\mathcal{F}}
\newcommand{\mcG}{\mathcal{G}}	\newcommand{\mcH}{\mathcal{H}}
\newcommand{\mcI}{\mathcal{I}}	\newcommand{\mcJ}{\mathcal{J}}
\newcommand{\mcK}{\mathcal{K}}	\newcommand{\mcL}{\mathcal{L}}
\newcommand{\mcM}{\mathcal{M}}	\newcommand{\mcN}{\mathcal{N}}
\newcommand{\mcO}{\mathcal{O}}	\newcommand{\mcP}{\mathcal{P}}
\newcommand{\mcQ}{\mathcal{Q}}	\newcommand{\mcR}{\mathcal{R}}
\newcommand{\mcS}{\mathcal{S}}	\newcommand{\mcT}{\mathcal{T}}
\newcommand{\mcU}{\mathcal{U}}	\newcommand{\mcV}{\mathcal{V}}
\newcommand{\mcW}{\mathcal{W}}	\newcommand{\mcX}{\mathcal{X}}
\newcommand{\mcY}{\mathcal{Y}}	\newcommand{\mcZ}{\mathcal{Z}}


%---------------------------------------
% Bold Math Fonts :-
%---------------------------------------

%Captital Letters
\newcommand{\bmA}{\boldsymbol{A}}	\newcommand{\bmB}{\boldsymbol{B}}
\newcommand{\bmC}{\boldsymbol{C}}	\newcommand{\bmD}{\boldsymbol{D}}
\newcommand{\bmE}{\boldsymbol{E}}	\newcommand{\bmF}{\boldsymbol{F}}
\newcommand{\bmG}{\boldsymbol{G}}	\newcommand{\bmH}{\boldsymbol{H}}
\newcommand{\bmI}{\boldsymbol{I}}	\newcommand{\bmJ}{\boldsymbol{J}}
\newcommand{\bmK}{\boldsymbol{K}}	\newcommand{\bmL}{\boldsymbol{L}}
\newcommand{\bmM}{\boldsymbol{M}}	\newcommand{\bmN}{\boldsymbol{N}}
\newcommand{\bmO}{\boldsymbol{O}}	\newcommand{\bmP}{\boldsymbol{P}}
\newcommand{\bmQ}{\boldsymbol{Q}}	\newcommand{\bmR}{\boldsymbol{R}}
\newcommand{\bmS}{\boldsymbol{S}}	\newcommand{\bmT}{\boldsymbol{T}}
\newcommand{\bmU}{\boldsymbol{U}}	\newcommand{\bmV}{\boldsymbol{V}}
\newcommand{\bmW}{\boldsymbol{W}}	\newcommand{\bmX}{\boldsymbol{X}}
\newcommand{\bmY}{\boldsymbol{Y}}	\newcommand{\bmZ}{\boldsymbol{Z}}
%Small Letters
\newcommand{\bma}{\boldsymbol{a}}	\newcommand{\bmb}{\boldsymbol{b}}
\newcommand{\bmc}{\boldsymbol{c}}	\newcommand{\bmd}{\boldsymbol{d}}
\newcommand{\bme}{\boldsymbol{e}}	\newcommand{\bmf}{\boldsymbol{f}}
\newcommand{\bmg}{\boldsymbol{g}}	\newcommand{\bmh}{\boldsymbol{h}}
\newcommand{\bmi}{\boldsymbol{i}}	\newcommand{\bmj}{\boldsymbol{j}}
\newcommand{\bmk}{\boldsymbol{k}}	\newcommand{\bml}{\boldsymbol{l}}
\newcommand{\bmm}{\boldsymbol{m}}	\newcommand{\bmn}{\boldsymbol{n}}
\newcommand{\bmo}{\boldsymbol{o}}	\newcommand{\bmp}{\boldsymbol{p}}
\newcommand{\bmq}{\boldsymbol{q}}	\newcommand{\bmr}{\boldsymbol{r}}
\newcommand{\bms}{\boldsymbol{s}}	\newcommand{\bmt}{\boldsymbol{t}}
\newcommand{\bmu}{\boldsymbol{u}}	\newcommand{\bmv}{\boldsymbol{v}}
\newcommand{\bmw}{\boldsymbol{w}}	\newcommand{\bmx}{\boldsymbol{x}}
\newcommand{\bmy}{\boldsymbol{y}}	\newcommand{\bmz}{\boldsymbol{z}}

%---------------------------------------
% Scr Math Fonts :-
%---------------------------------------

\newcommand{\sA}{{\mathscr{A}}}   \newcommand{\sB}{{\mathscr{B}}}
\newcommand{\sC}{{\mathscr{C}}}   \newcommand{\sD}{{\mathscr{D}}}
\newcommand{\sE}{{\mathscr{E}}}   \newcommand{\sF}{{\mathscr{F}}}
\newcommand{\sG}{{\mathscr{G}}}   \newcommand{\sH}{{\mathscr{H}}}
\newcommand{\sI}{{\mathscr{I}}}   \newcommand{\sJ}{{\mathscr{J}}}
\newcommand{\sK}{{\mathscr{K}}}   \newcommand{\sL}{{\mathscr{L}}}
\newcommand{\sM}{{\mathscr{M}}}   \newcommand{\sN}{{\mathscr{N}}}
\newcommand{\sO}{{\mathscr{O}}}   \newcommand{\sP}{{\mathscr{P}}}
\newcommand{\sQ}{{\mathscr{Q}}}   \newcommand{\sR}{{\mathscr{R}}}
\newcommand{\sS}{{\mathscr{S}}}   \newcommand{\sT}{{\mathscr{T}}}
\newcommand{\sU}{{\mathscr{U}}}   \newcommand{\sV}{{\mathscr{V}}}
\newcommand{\sW}{{\mathscr{W}}}   \newcommand{\sX}{{\mathscr{X}}}
\newcommand{\sY}{{\mathscr{Y}}}   \newcommand{\sZ}{{\mathscr{Z}}}


%---------------------------------------
% Math Fraktur Font
%---------------------------------------

%Captital Letters
\newcommand{\mfA}{\mathfrak{A}}	\newcommand{\mfB}{\mathfrak{B}}
\newcommand{\mfC}{\mathfrak{C}}	\newcommand{\mfD}{\mathfrak{D}}
\newcommand{\mfE}{\mathfrak{E}}	\newcommand{\mfF}{\mathfrak{F}}
\newcommand{\mfG}{\mathfrak{G}}	\newcommand{\mfH}{\mathfrak{H}}
\newcommand{\mfI}{\mathfrak{I}}	\newcommand{\mfJ}{\mathfrak{J}}
\newcommand{\mfK}{\mathfrak{K}}	\newcommand{\mfL}{\mathfrak{L}}
\newcommand{\mfM}{\mathfrak{M}}	\newcommand{\mfN}{\mathfrak{N}}
\newcommand{\mfO}{\mathfrak{O}}	\newcommand{\mfP}{\mathfrak{P}}
\newcommand{\mfQ}{\mathfrak{Q}}	\newcommand{\mfR}{\mathfrak{R}}
\newcommand{\mfS}{\mathfrak{S}}	\newcommand{\mfT}{\mathfrak{T}}
\newcommand{\mfU}{\mathfrak{U}}	\newcommand{\mfV}{\mathfrak{V}}
\newcommand{\mfW}{\mathfrak{W}}	\newcommand{\mfX}{\mathfrak{X}}
\newcommand{\mfY}{\mathfrak{Y}}	\newcommand{\mfZ}{\mathfrak{Z}}
%Small Letters
\newcommand{\mfa}{\mathfrak{a}}	\newcommand{\mfb}{\mathfrak{b}}
\newcommand{\mfc}{\mathfrak{c}}	\newcommand{\mfd}{\mathfrak{d}}
\newcommand{\mfe}{\mathfrak{e}}	\newcommand{\mff}{\mathfrak{f}}
\newcommand{\mfg}{\mathfrak{g}}	\newcommand{\mfh}{\mathfrak{h}}
\newcommand{\mfi}{\mathfrak{i}}	\newcommand{\mfj}{\mathfrak{j}}
\newcommand{\mfk}{\mathfrak{k}}	\newcommand{\mfl}{\mathfrak{l}}
\newcommand{\mfm}{\mathfrak{m}}	\newcommand{\mfn}{\mathfrak{n}}
\newcommand{\mfo}{\mathfrak{o}}	\newcommand{\mfp}{\mathfrak{p}}
\newcommand{\mfq}{\mathfrak{q}}	\newcommand{\mfr}{\mathfrak{r}}
\newcommand{\mfs}{\mathfrak{s}}	\newcommand{\mft}{\mathfrak{t}}
\newcommand{\mfu}{\mathfrak{u}}	\newcommand{\mfv}{\mathfrak{v}}
\newcommand{\mfw}{\mathfrak{w}}	\newcommand{\mfx}{\mathfrak{x}}
\newcommand{\mfy}{\mathfrak{y}}	\newcommand{\mfz}{\mathfrak{z}}


\title{\Huge{Single Cycle Processor}}
\author{\huge{Madiba Hudson-Quansah}}
\date{}
\usepackage{parskip}
\usepackage{tabularx}

\setcounter{tocdepth}{4}
\setcounter{secnumdepth}{4}

\begin{document}
\maketitle
\newpage
\pdfbookmark[section]{\contentsname}{too}
\tableofcontents
\pagebreak

\chapter{Instructions}

Each instruction in MIPS is the same size (32 bits). There are three types of instructions in MIPS:
\begin{itemize}
  \item R-Type: Register, example \texttt{add \$t0, \$t1, \$t2}
  \item I-Type: Immediate, example \texttt{addi \$t0, \$t1, 100}
  \item J-Type: Jump, example \texttt{j 1000}
\end{itemize}

MIPS supports various addressing modes, including:
\begin{description}
  \item[Immediate] - Operand is a constant value embedded in the instruction itself. Example: \texttt{addi \$t0, \$t1, 100}
  \item[Register] - Operand is stored in a register. Example: \texttt{add \$t0, \$t1, \$t2}
  \item[Base] - Operand is stored in memory, and the address is calculated using a base register and an offset. Example: \texttt{lw \$t0, 100(\$t1)}
  \item[PC-Relative] - Operand is stored in memory, and the address is calculated using the program counter (PC) and an offset. Example: \texttt{beq \$t0, \$t1, 100}
  \item[Pseudo-Direct] - Operand is stored in memory, and the address is calculated using a pseudo-direct addressing mode. Example: \texttt{j 1000}
\end{description}

\section{R-Type Instructions}

Register operations. Arithmetic and logical operations directly on registers. MIPS defines 32 general purpose register each 32 bits wide. The registers are numbered from 0 to 31, with register 0 always being 0.

\subsection{Structure of R-Type Instructions}

\begin{table}[h!]
  \begin{center}
    \begin{tabular}{|c|c|c|c|c|c|c|}
      \hline
      \texttt{opcode} & \texttt{rs} & \texttt{rt} & \texttt{rd} & \texttt{shamt} & \texttt{funct} \\
      \hline
      6 bits          & 5 bits      & 5 bits      & 5 bits      & 5 bits         & 6 bits         \\
      \hline
    \end{tabular}
  \end{center}
\end{table}

For R-type instructions, the \texttt{opcode} field is always 0, i.e. $000000$ as the \texttt{funct} determines the operation type. This is only for R-type instructions.
Where:
\begin{description}
  \item[\texttt{opcode}] The operation code. Used to specify the operation.
  \item[\texttt{rs}] The source register. Used to store the first operand.
  \item[\texttt{rt}] The target register. Used to store the second operand.
  \item[\texttt{rd}] The destination register. Used to store the result of the operation.
  \item[\texttt{shamt}] The shift amount. Used to specify the number of bits to shift.
  \item[\texttt{funct}] The function code. Used to specify the operation, example \texttt{add}, \texttt{sub}, \texttt{and}, \texttt{or}, etc.
\end{description}

\section{I-Type Instructions}
Memory access and immediate values.

\begin{table}[h!]
  \begin{center}
    \begin{tabular}{|c|c|c|c|c|c|c|}
      \hline
      \texttt{opcode} & \texttt{rs} & \texttt{rt} & \texttt{immediate} \\
      \hline
      6 bits          & 5 bits      & 5 bits      & 16 bits            \\
      \hline
    \end{tabular}
  \end{center}
\end{table}

Where:
\begin{description}
  \item[\texttt{opcode}] The operation code. Used to specify the operation.
  \item[\texttt{rs}] The source register. Used to store the register operand
  \item[\texttt{rt}] The target register. The desination register.
  \item [\texttt{immediate}] The immediate value. Used to store the immediate value operand
\end{description}

\section{J-Type Instructions}

Branching and jumping.

\begin{table}[h!]
  \begin{center}
    \begin{tabular}{|c|c|c|c|c|c|c|}
      \hline
      \texttt{opcode} & \texttt{address} \\
      \hline
      6 bits          & 26 bits          \\
      \hline
    \end{tabular}
  \end{center}
\end{table}

Where:
\begin{description}
  \item[\texttt{opcode}] The operation code. Used to specify the operation.
  \item[\texttt{address}] The address. Target address of the jump.
\end{description}

\chapter{Processor Design}

In designing a processor for an ISA one goes through the following steps
\begin{description}
  \item[Analyse the Instruction Set ] - To determine data path requirements and control signals
  \item[Select Data path components and clocking methodology] - Based on the instruction set
  \item[Assemble the datapath meeting the requirements]
  \item[Analyse the implementation of each instruction] - To determine the needed control signals
  \item[Assemble the control logic] - To generate the control signals
\end{description}

Taking into account only a subset of the MIPS instructions:
\begin{description}
  \item[ALU Instructions (R-Type)] - \texttt{add, sub, and, or, slt}
  \item[Immediate Instructions (I-Type)] - \texttt{addi, lw, sw, beq}
  \item[Load and Store (I-Type)] - \texttt{lw, sw}
  \item[Branch (I-Type)] - \texttt{beq, bne}
  \item[Jump (J-Type)] - \texttt{j}
\end{description}

\section{Register Transfer Level (RTL)}

\dfn{Register Transfer Level (RTL)}{
  A description of data flow between registers in a processor. It is a way of describing the implementation of a processors instructions at the register level.
}

Instructions are first fetched from memory using the \textbf{Program Counter (PC)} to get the address. The instruction is then decoded to determine the operation type and then executed using accordingly.
\begin{description}
  \item[ADD]  - $\text{Reg (rd)} \gets \text{Reg (rs)} + \text{Reg (rt)}; PC \gets PC + 4$
\end{description}

In terms of processor design a hybrid Von-Nuemann Harvard architecture is used, where instruction memory is separate from data memory. In terms of MIPS there are
31 general purpose registers each 32-bit wide with $R0$ always being 0. The program counter is stored in the PC registers and an adder is used to increment the PC.

\section{Data Path Components and Clocking Methodology}

\subsection{Data Path Components}

\dfn{Data Path}{
  The part of the processor that is responsible for executing instructions by performing arithmetic operations, data movement and logical operations.
}

The data path is made up of the following components:
\begin{description}
  \item[Registers] - Small fast memory within the processor. Used to store data temporarily during execution. Registers can include
        \begin{description}
          \item[General Purpose Registers]  - Used to hold data and intermediate results during execution
          \item[Special-Purpose Registers] - Used for specific purposes such as keeping track of the program counter (PC), instruction register (IR), stack pointer (SP), status flags, etc.
        \end{description}
  \item[Arithmetic Logic Unit (ALU)] - Performs arithmetic and logical operations on data. Arithmetic operations include addition, subtraction, multiplication
        and logical operations include AND, OR, NOT, XOR, etc.
  \item[Data Paths] - The connections between registers, the ALU and other functional units of the processor.
  \item[Memory Interface] - Enables communication between the processor and system memory (RAM). Includes components such as the
        memory address register (MAR) for storing memory addresses, and memory data register (MDR) for storing data read from or written to memory.
  \item[Multiplexers and Demultiplexers] - For selecting between multiple data paths between different components such as registers and ALUs based on control signals generated by the control unit
  \item[Buses] - Communication pathways that transfer data and control signals between different components of the CPU and system memory. Buses
        include the data bus, address bus, and control bus.
  \item[Instruction Fetch and Decode Logic] - The instruction fetch unit retrieves instructions from memory based on the program counter (PC) and
        send them to the instruction decode unit which decodes the instruction to determine appropriate operation to execute.
  \item[Pipeline Stages] - The data path can also be divided in to pipeline stages to improve throughput and performance. Each pipeline stage
        performs a specific task such as instruction fetch, decode, execute, and write-back.
  \item[Cache Interface] - For storing frequently accessed data and instruction closer to the CPU cores for faster access.
  \item[Control Signals] - Generated by the control unit to coordinate the operation of different components within the data path.
        Control signals determine which operations are performed by the ALU, which data paths are selected and when data is transferred between registers and memory.
\end{description}

The design of the Data path is critical to processor design as it determines how fast the processor can execute instructions and how efficiently
it can perform computations. It must be optimized to ensure that instructions are executed correctly and efficiently while minimizing latency
and maximizing throughput. Modern processors employ complex data paths with some of the following features:
\begin{itemize}
  \item Pipelining
  \item Out-of-order execution
  \item Speculative execution
\end{itemize}

\subsubsection{Pipelining}\label{sec:pipelining}

\dfn{Pipelining}{
  The execution of an instruction is broken down into multiple stages, i.e. fetch, decode, etc, then stages of instruction execution are executed in an overlapping manner allowing multiple instructions to executed simultaneously.
}

\subsubsection{Out-of-order Execution}\label{sec:out-of-order-execution}

\dfn{Out-of-order Execution}{
  Instructions are executed in an order different from the order they appear in the program, based on data dependencies and resource availability.
  This allows the processor to execute instructions more efficiently by avoiding stalls and keeping the execution units busy.
}

\subsubsection{Speculative Execution}\label{sec:speculative-execution}

\dfn{Speculative Execution}{
  The processor executes instructions under an execution branch before the result of the branch is known, i.e. if the branch is taken or not.
  This potentially allows the processor to be more performant by executing instructions that are likely to be executed.
}

\subsection{Clocking Methodology}

\dfn{Clocking Methodology}{
  The method used to synchronize the operations of the different components of the processor. This defines when data can be written to and read from.
}

We assume edge-triggered clocking where all state changes happen on the same clock edge. Data must be valid and stable before the clock edge and the output must be valid after the clock edge. Edge-trigged clocking allows data to be read and written during the same clock cycle, for example data could be read during the
rising edge and written during the falling edge. \\

With edge-triggered clocking the clock cycle must be long enough to allow all operations to complete, therefore the clock cycle speed is determined by
the time taken by slowest instruction. This shown by the equation:
\[
  T_{\text{cycle}} \geq T_{\text{clk} - \text{q}} + T_{\text{max\_comb}} + T_s
\]
Where:
\begin{description}
  \item[$T_{\text{cycle}}$] - The clock cycle time
  \item[$T_{\text{clk}}$] - The clock to output delay through register, i.e time taken for data to be read from a register
  \item[$T_{\text{max\_comb}}$] - The longest delay through combinatorial logic, i.e. the longest time taken for data to be processed (instruction execution)
  \item [$T_s$] - Setup time, the time data must be stable before the clock edge
  \item [$T_h$] - Hold time, the time data must be stable after the clock edge, normally satisfied since $T_{\text{clk} - \text{q}} > T_h$
\end{description}

\subsubsection{Clock Skew}

\dfn{Clock Skew}{
  The absolute difference in time between when two storage elements see a clock edge.
}

Clock skew arises from the fact that the clock signal uses different paths each with differing delays to reach state elements, like flip-flops. This
has to be accounted for to ensure that the cycle time can accommodate all read and write operation irrespective of the time the registers they are clocked.
This results in the new equation:
\[
  T_{\text{cycle}} \geq T_{\text{clk} - \text{q}} + T_{\text{max\_comb}} + T_s + T_{\text{skew}}
\]
This skew amount can however be minimized by balancing the clock delays to all storage elements.

\chapter{Instruction Type Data Paths}

\section{Control Signals}
The MIPS ISA defines the following control signals:
\begin{description}
  \item[IsJAL] - Used when the instruction is a Jump and Link instruction, which saves the return address in the link register.
  \item[IsShamt] - Determines if the instruction is a shift operation which requires the shamt field to be used.
  \item[ALUSrc] - Chooses whether the second operand for the ALU comes from a register ($rt$) or an immediate value.
  \item[IsSyscall] - Determines whether the instruction is a system call.
  \item[ExtOp] - Determines whether the immediate value should be zero-extended or sign-extended.
  \item[MemRead] - Controls whether memory should be read to.
  \item[Jump] - Determines whether the instruction is a jump instruction.
  \item[Branch] - Determines whether the instruction is a branch instruction.
  \item[RegDst] - Determines whether the destination register comes from $rt$ (I-type) or $rd$ (R-type).
  \item[IsJR] - Determines whether the instruction is a jump register instruction, which jumps to the address in a register.
  \item[ReadRS/ReadRT] - Determines whether the instruction should read from the source register ($rs$) or target register ($rt$).
  \item[ALUOp] - Determines the operation to be performed by the ALU, such as addition, subtraction, logical AND, etc.
  \item[RegWR] - Determines whether the instruction should write to a register.
\end{description}

\section{R-Type Instructions}

For R-type instructions the datapath is as follows:
\begin{description}
  \item[IF]  - The instruction is fetched from instruction memory (ROM) using the program counter and stored in the instruction register (IR).
  \item[ID] - The instruction is decoded by the control unit to determine the operation type and the destination ($rd$) and operand registers ($rs$ and $rt$). At this stage these are the control signals activated:
        \begin{description}
          \item[ALUOp] - As opcode is 0, the ALUOp is determined by the funct field of the instruction.
          \item[ALUSrc] - The second operand for the ALU comes from the register file ($rt$)
          \item[RegDst] - The destination register comes from the $rd$ field of the instruction.
          \item[RegWR] - Enables writing of the ALU result.
        \end{description}
  \item[EX] - The ALU performs the operation specified by the ALUOp.
  \item[WB] - The result of the ALU operation is written back to the destination register ($rd$) in the register file.
\end{description}

\section{I-Type Instructions}
For I-type instructions the datapath is as follows:
\begin{description}
  \item[IF]  - The instruction is fetched from instruction memory (ROM) using the program counter and stored in the instruction register (IR).
  \item[ID] - The instruction is decoded by the control unit to determine the operation type $op$, and the destination ($rt$), operand register ($rs$) and immediate value. At this stage these are the control signals activated:
        \begin{description}
          \item[ALUOp] - The ALUOp is determined by the opcode field of the instruction.
          \item[ALUSrc] - The second operand for the ALU comes from the immediate value.
          \item[RegDst] - The destination register comes from the $rt$ field of the instruction.
          \item[RegWR] - Enables writing of the ALU result.
          \item[ExtOp] - Determines whether the immediate value should be zero-extended or sign-extended as the 16-bit immediate value needs to be extended to 32 bits for ALU operations.
        \end{description}
  \item[EX] - The ALU performs the operation specified by the ALUOp using the immediate value and the value in the $rs$ register.
  \item[MEM] - Activated depending on the type of instruction, i.e. load/store instructions
        \begin{description}
          \item[Load] - The memory address is computed by adding the immediate value to the value in the $rs$ register. The data is then read from memory and stored in the $rt$ register.
          \item[Store] - The memory address is computed by adding the immediate value to the value in the $rs$ register. The data from the $rt$ register is then written to memory.
          \item[Branch] - The branch target address is computed by adding the immediate value to the program counter (PC). The branch is taken if the condition is met.
        \end{description}
  \item[WB] - The result of the ALU operation is written back to the destination register ($rt$) in the register file.
\end{description}

\section{J-Type Instructions}
For J-type instructions the datapath is as follows:
\begin{description}
  \item[IF]  - The instruction is fetched from instruction memory (ROM) using the program counter and stored in the instruction register (IR).
  \item[ID] - The instruction is decoded by the control unit to determine the operation type $op$, and the target address. At this stage these are the control signals activated:
        \begin{description}
          \item[Jump] - The jump target address is computed by shifting the 26-bit address left by 2 bits and then concatenating with the upper 4 bits of $PC + 4$
          \item[IsJAL] - Determines whether the instruction is a jump and link instruction, which saves the return address in the link register.
        \end{description}
  \item[EX] - The new program counter (PC) is computed by adding the jump target address to the current program counter (PC), depending on the type of jump instruction.
\end{description}

\chapter{Limitations of Single Cycle Processors}

Each instruction execution can be categorized into the following categories based on their common stages:
\begin{description}
  \item [ALU]
        \begin{table}[H]
          \begin{center}
            \begin{tabular}{|c|c|c|c|}
              \hline
              Instruction Fetch & Decode / Register Read & ALU & Register Write \\
              \hline
            \end{tabular}
          \end{center}
        \end{table}
  \item[  Load ]
        \begin{table}[H]
          \begin{center}
            \begin{tabular}{|c|c|c|c|c|}
              \hline
              Instruction Fetch & Decode / Register Read & Compute Address & Memory Read & Register Write \\
              \hline
            \end{tabular}
          \end{center}
        \end{table}
  \item[  Store ]
        \begin{table}[H]
          \begin{center}
            \begin{tabular}{|c|c|c|c|}
              \hline
              Instruction Fetch & Decode / Register Read & Compute Address & Memory Write \\
              \hline
            \end{tabular}
          \end{center}
        \end{table}
  \item[  Branch]
        \begin{table}[H]
          \begin{center}
            \begin{tabular}{|c|c|c|}
              \hline
              Instruction Fetch & Register read / Branch Target & Compare and Update PC \\
              \hline
            \end{tabular}
          \end{center}
        \end{table}
  \item[  Jump ]
        \begin{table}[H]
          \begin{center}
            \begin{tabular}{|c|c|}
              \hline
              Instruction Fetch & Decode / Update PC \\
              \hline
            \end{tabular}
          \end{center}
        \end{table}
\end{description}

With the Load instruction category being the slowest, meaning the cycle time must be long enough to accommodate the load instruction. This has several implications:
\begin{description}
  \item[Clock Speed Constraint] - The clock cycle time must be long enough to accommodate the longest instruction, i.e. the load instruction. This means that
        shorter instructions waste time waiting for the longer ones to complete and in the case where a shorter one completes before the cycle is over, the processor
        is idle for the rest of the cycle.
  \item[Resource Utilization] - All resources such as ALUs, memory and registers are only used once per cycle. This means that the processor is not
        fully utilising its resources most of the time.
  \item[Complexity in Control] - The control unit must manage all the operations in a single cycle increasing the complexity of design.
  \item[Power Consumption] - Due to the need to complete all operations in a single cycle, the processor will need to consume more power drive all the
        functional units simultaneously.
  \item[Scalabity] - As more instructions are added to the ISA, the cycle time will need to be adjusted to accommodate the longest instruction, placing
        a limit on the ability to scale the processor for better performance.
  \item[Peformance] - Since the clock cycle time is constrained by the slowest instruction, the processor is not able to achieve the best performance
        by executing multiple instructions simultaneously.
\end{description}

\section{Overcoming Limitations}

To overcome the limitations of the single cycle processor, the following techniques can be employed:
\begin{description}
  \item[Pipelining] \ref{sec:pipelining}
  \item[Multi-Cycle Design] - Breaks down an instruction execution into multiple stages, with each stage taking one cycle to execute. This allows the clock cycle
        time to be reduced as it is no longer constrained by the slowest instruction rather the slowest stage.
  \item[Superscalar Architecture] - Involves the use of multiple execution units, i.e. multiple ALUs, allowing multiple instructions to be executed
        simultaneously. By fetching and dispatching multiple instructions per cycle, superscalar processors increase instruction throughput.
  \item[Out-of-order Execution] \ref{sec:out-of-order-execution}
  \item[Branch Prediction and Speculative Execution] - Predicts the outcome of a branch instruction before it is executed, allowing the processor to speculatively execute \ref{sec:speculative-execution}
  \item[Instruction Level Parallelism] - Techniques such as loop unrolling, software pipelining, vectorization and instruction reordering can allow for more
        parallelism in instruction execution.
  \item[Caching] - Improves performance by storing frequently accessed data and instructions closer to the processor cores reducing the latency
        of memory access.
  \item[Register Renaming] - Dynamically allocates physical registers to logical registers, i.e. dynamically mapping concurrently used logical registers to avoid what is the hardware equivalent of a race condition , to avoid hazards and dependencies between instructions.

\end{description}

\section{Multi-Cycle vs Pipelining}

\begin{table}[h!]
  \begin{center}
    \begin{tabularx}{\textwidth}{|X|X|X|}
      \hline
      \textbf{Feature}     & \textbf{Multi-Cycle Processor Design}                                                                   & \textbf{Pipelined Processor Design}                                                    \\ [0.5ex]
      \hline
      \hline
      Execution Phases     & Instructions are executed over several cycles in the form of stages                                     & Instructions are divided into stages that overlap execution                            \\
      \hline
      Hardware Utilization & Resources are reused across different cycles for different stages                                       & More hardware resources are needed to execute multiple stages simultaneously           \\
      \hline
      Control Complexity   & Complex control unit normally implemented as a finite state machine (FSM)                               & Complex control logic needed to manage simultaneous execution of multiple stages       \\
      \hline
      Cycle Time           & Clock cycle time is determined by the slowest stage                                                     & Clock cycle time is determined by the longest stage                                    \\
      \hline
      Performance          & Instructions are completed in several cycles, but each cycle is faster than in a single-cycle processor & Instruction throughput can approach one instruction per cycle due to stage overlapping \\
      \hline
    \end{tabularx}
  \end{center}
\end{table}

\section{Multi-Cycle Implementation}

First break each instruction into five stages:
\begin{itemize}
  \item Instruction Fetch (IF)
  \item Instruction Decode / Register Read or Target Address for jump/branch. (ID)
  \item Execution / Memory address computation or branch outcome. (EX)
  \item Memory Access / ALU instruction completion (MEM)
  \item Load instruction completion (WB)
\end{itemize}

Each stage takes one clock cycle to complete, i.e. the categories of instructions break down into the following cycle counts
\begin{description}
  \item[ALU] - 4
  \item[Load] - 5
  \item[Store] -4
  \item[Branch] - 3
  \item[Jump] - 2
\end{description}

\ex{}{
  \qs{}{
    Assuming the following operation times for components:
    \begin{itemize}
      \item Access time for instruction and data memories is 200 ps
      \item Delay in ALU and adders is 180 ps
      \item Delay in decode and register file access (read or write) is 150 ps
      \item Ignore all other delays in PC, mux, extender, and wires
    \end{itemize}
    Which of the following designs would be faster and by how much?
    \begin{enumerate}
      \item Single cycle processor
      \item Multi-cycle processor
    \end{enumerate}
    Assume the following instruction mix: 40\% ALU, 20\% Load, 10\% Store, 20\% Branch, 10\% Jump
  }

  \sol{
    \begin{description}
      \item[ALU]  - $200 + 150 + 180 + 150 = 680 \text{ps}$
      \item[Load] - $200 + 150 + 180 + 200 + 150 = 880 \text{ps}$
      \item [Store] - $200 + 150 + 180 + 150 = 680 \text{ps}$
      \item[Branch] - $200 + 150 + 180 = 530 \text{ps}$
      \item[Jump] - $200 + 180 = 280 \text{ps}$
    \end{description}
    \begin{enumerate}
      \item Each single cycle processor cycle has to be as long as the longest instruction, i.e. $880 \text{ps}$
      \item Each multi cycle processor cycle has to be as long as the longest stage i.e. $200 \text{ps}$.
            \begin{align*}
              \text{CPI} & = 0.4\times 4 + 0.2 \times 5 + 0.1 \times 4 + 0.2 \times 3 + 0.1 \times 2 \\
                         & = 3.8                                                                     \\
            \end{align*}
            $ \therefore \text{Clock Period} = 3.8 \times 200 = 760$
    \end{enumerate}
    \[
      \frac{880}{760} = 1.16
    \]
    $\therefore$ the multi-cycle design is faster by about 16 \%
  }
}


\end{document}
